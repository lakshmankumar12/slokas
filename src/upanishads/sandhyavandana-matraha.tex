\subsection{\eng{Sandyavandana Mantraha}}
\subsubsection{सर्वा देवता आपः (4.27)}
आपो॒ वा, इ॒दग्ं सर्वं॒-विँश्वा॑ \\
भू॒तान्यापः॑ प्रा॒णा वा आपः॑ प॒शव॒ \\
आपो-ऽन्न॒मापो -ऽमृ॑त॒मापः॑ स॒म्राडापो॑ वि॒राडापः॑ स्व॒राडापः॒\\
छन्दा॒ग्॒स्यापो॒ ज्योती॒ग्॒ष्यापो॒ यजू॒ग्॒ष्यापः॑ स॒त्यमाप॒-स्सर्वा॑ \\
दे॒वता॒ आपो॒ भूर्भुव॒स्सुव॒राप॒ ओम् ॥ \\
29.1 (तै. अर. 6.29.1)

\subsubsection{सन्ध्यावन्दन मन्त्राः (4.28)}
आपः॑ पुनन्तु पृथि॒वी-म्पृ॑थि॒वी पू॒ता पु॑नातु॒ माम् ।\\
पु॒नन्तु॒ ब्रह्म॑ण॒स्पति॒-र्ब्रह्म॑ पू॒ता पु॑नातु॒ माम् ।\\
यदुच्छि॑ष्ट॒-मभो᳚ज्यं॒-यँद्वा॑ दु॒श्चरि॑त॒-म्मम॑ ।\\
सर्व॑-म्पुनन्तु॒ मामापो॑-ऽस॒ताञ्च॑ प्रति॒ग्रह॒ग्ग्॒ स्वाहा᳚ ॥\\
30.2 (तै. अर. 6.30.1)\\
\\
अग्निश्च मा मन्युश्च मन्युपतयश्च मन्यु॑कृते॒भ्यः ।\\
पापेभ्यो॑ रक्ष॒न्ताम् । यदह्ना पाप॑मका॒र्॒षम् ।\\
मनसा वाचा॑ हस्ता॒भ्याम् । पद्भ्या-मुदरे॑ण शि॒श्ना ।\\
अह॒स्तद॑वलु॒पन्तु । यत्किञ्च॑ दुरि॒त-म्मयि॑ ।\\
इदमह-माममृ॑त यो॒नौ । सत्ये ज्योतिषि जुहो॑मि स्वा॒हा ॥ \\
31.1 (तै. अर. 6.31.1)\\
\\
सूर्यश्च मा मन्युश्च मन्युपतयश्च मन्यु॑कृते॒भ्यः ।\\
पापेभ्यो॑ रक्ष॒न्ताम् । यद्रात्रिया पाप॑मका॒र्॒षम् ।\\
मनसा वाचा॑ हस्ता॒भ्याम् । पद्भ्या-मुदरे॑ण शि॒श्ना ।\\
रात्रि॒-स्तद॑वलु॒पन्तु । यत्किञ्च॑ दुरि॒त-म्मयि॑ ।\\
इदमह-माममृ॑त यो॒नौ । सूर्ये ज्योतिषि जुहो॑मि स्वा॒हा ॥ \\
32.1 (तै. अर. 6.32.1)\\

\subsubsection{प्रणवस्य ऋष्यादि विवरणं (4.29)}
ओमित्येकाक्ष॑र-म्ब्र॒ह्म । अग्निर्देवता ब्रह्म॑ इत्या॒र्​षम् ।\\
गायत्र-ञ्छन्द-म्परमात्मं॑ सरू॒पम् । सायुज्यं-विँ॑नियो॒गम् ॥ \\
33.1 (तै. अर. 6.33.1)\\

\subsubsection{गायत्र्यावाहन मन्त्राः (4.30)}
आया॑तु॒ वर॑दा दे॒वी॒ अ॒क्षर॑-म्ब्रह्म॒ सम्मि॑तम् ।\\
गा॒य॒त्री᳚-ञ्छन्द॑सा-म्मा॒तेद-म्ब्र॑ह्म जु॒षस्व॑ मे ।\\
यदह्ना᳚-त्कुरु॑ते पा॒प॒-न्तदह्ना᳚-त्प्रति॒मुच्य॑ते ।\\
य-द्रात्रिया᳚-त्कुरु॑ते पा॒प॒-न्त-द्रात्रिया᳚-त्प्रति॒मुच्य॑ते ।\\
सर्व॑ व॒र्णे म॑हादे॒वि॒ स॒न्ध्या वि॑द्ये स॒रस्व॑ति ॥ \\
34.2 (तै. अर. 6.34.1)\\
\\
ओजो॑-ऽसि॒ सहो॑-ऽसि॒ बल॑मसि॒ भ्राजो॑-ऽसि \\
दे॒वाना॒-न्धाम॒नामा॑॑-ऽसि॒ विश्व॑मसि वि॒श्वायु॒-स्सर्व॑मसि \\
स॒र्वायु-रभिभूरों-गायत्री-मावा॑हया॒मि॒ सावित्री-मावा॑हया॒मि॒ \\
सरस्वती-मावा॑हया॒मि॒ छन्दर्​षी-नावा॑हया॒मि॒ श्रिय-मावा॑हया॒मि॒ \\
गायत्रिया गायत्री छन्दो विश्वामित्र ऋषि-स्सविता \\
देवता-ऽग्निर्मुख-म्ब्रह्मा शिरो विष्णुर्​हृदयग्ं रुद्र-श्शिखा \\
पृथिवीयोनिः प्राणापान-व्यानोदान-समाना सप्राणा श्वेतवर्णा \\
साङ्ख्यायन-सगोत्रा गायत्री चतुर्विग्ंशत्यक्षरा त्रिपदा॑ \\
षट्कु॒क्षिः॒ पञ्च शीर्​षोपनयने वि॑नियो॒गः॒-\\
ओ-म्भूः । ओ-म्भुवः । ओग्ं सुवः । ओ-म्महः । \\
ओ-ञ्जनः । ओ-न्तपः । ओग्ं स॒त्यम् । \\
ओ-न्त-थ्स॑वि॒तुर्वरे᳚ण्य॒-म्भर्गो॑ दे॒वस्य॑ धीमहि । \\
धियो॒ यो नः॑ प्रचो॒दया᳚त् । \\
ओमापो॒ ज्योती॒ रसो॒-ऽमृत॒-म्ब्रह्म॒ भूर्भुव॒स्सुव॒रोम् ॥ \\
35.2 (तै. अर. 6.35.1)\\

\subsubsection{गायत्री उपस्थान मन्त्राः (4.31)}
उ॒त्तमे॑ शिख॑रे जा॒ते॒ भू॒म्या-म्प॑र्वत॒ मूर्ध॑नि ।\\
ब्रा॒ह्मणे᳚भ्यो-ऽभ्य॑नुज्ञा॒ता॒ ग॒च्छ दे॑वि य॒थासु॑खम् ।\\
स्तुतो मया वरदा वे॑दमा॒ता॒ प्रचोदयन्ती पवने᳚ द्विजा॒ता ।\\
आयुः पृथिव्यां-द्रविण-म्ब्र॑ह्मव॒र्च॒स॒-म्मह्य-न्दत्वा प्रजातु-म्ब्र॑ह्मलो॒कम् ॥ \\
36.2 (तै. अर. 6.36.1)\\

\subsubsection{कामो-ऽकार्​षीत् - मन्युरकार्​षीत् मन्त्रः (4.41)}}
कामो-ऽकार्​षी᳚-न्नमो॒ नमः ।\\
कामो-ऽकार्​षी-त्कामः करोति \\
नाह-ङ्करोमि कामः कर्ता नाह-ङ्कर्ता कामः॑ \\
कार॒यिता नाह॑-ङ्कार॒यिता एष ते काम कामा॑य स्वा॒हा ॥\\
61.1 (तै. अर. 6.61.1)\\
\\
मन्युरकार्​षी᳚-न्नमो॒ नमः ।\\
मन्युरकार्​षी-न्मन्युः करोति \\
नाह-ङ्करोमि मन्युः कर्ता नाह-ङ्कर्ता मन्युः॑ \\
कार॒यिता नाह॑-ङ्कार॒यिता एष ते मन्यो मन्य॑वे स्वा॒हा ॥\\
62.1 (तै. अर. 6.62.1)\\

\subsubsection{द॒धि॒क्राव्.ण्णो॑ अकारिषं}
द॒धि॒क्राव्.ण्णो॑ अकारिषं । जि॒ष्णोरश्व॑स्य वा॒जिनः॑ ।\\
सु॒र॒भि नो॒ मुखा॑ कर॒त् प्र ण॒ आयूꣳ॑षि तारिषत् ।\\
आपो॒ हि ष्ठा म॑यो॒भुव॒स्ता न॑ ऊ॒र्जे द॑धातन । म॒हे रणा॑य॒ चक्ष॑से ।\\
यो व॑ श्शि॒वत॑मो॒ रस॒स्तस्य॑ भाजयते॒ह नः॑ ।\\
उ॒श॒तीरि॑व मा॒तरः॑ । तस्मा॒ अर॑ङ्गमाम वो॒ यस्य॒ क्षया॑य॒ जिन्व॑थ ।\\
आपो॑ ज॒नय॑था च नः ॥\\