\section{\eng{Narayanana Upanishad}}
ॐ स॒ह ना॑ववतु । स॒ह नौ॑ भुनक्तु । \\
स॒ह वी॒र्यं॑ करवावहै ।\\
ते॒ज॒स्विना॒वधी॑तमस्तु॒ मा वि॑द्विषा॒वहै᳚ ॥\\
ॐ शान्तिः॒ शान्तिः॒ शान्तिः॑ ॥\\
\\
ॐ अथ पुरुषो ह वै नारायणोऽकामयत प्रजाः सृ॑जेये॒ति ।\\
ना॒रा॒य॒णात्प्रा॑णो जा॒यते । मनः सर्वेन्द्रि॑याणि॒ च ।\\
खं-वाँयुर् ज्योति रापः पृथिवी विश्व॑स्य धा॒रिणी ।\\
ना॒रा॒य॒णाद्ब्र॑ह्मा जा॒यते ।\\
ना॒रा॒य॒णाद्रु॑द्रो जा॒यते ।\\
ना॒रा॒य॒णादि॑न्द्रो जा॒यते ।\\
ना॒रा॒य॒णात्प्रजापतयः प्र॑जाय॒न्ते ।\\
ना॒रा॒य॒णाद् द्वाद शादित्या रुद्रा \\
वसवस् सर्वाणिच छ॑न्दाग्ं॒सि ।\\
ना॒रा॒य॒णादेव समु॑त् पद् य॒न्ते ।\\
ना॒रा॒य॒णे प्र॑वर्त॒न्ते ।\\
ना॒रा॒य॒णे प्र॑लीय॒न्ते ॥\\
\\
ओम् । अथ नित्यो ना॑राय॒णः । ब्र॒ह्मा ना॑राय॒णः ।\\
शि॒वश्च॑ नाराय॒णः । श॒क्रश्च॑ नाराय॒णः ।\\
द्या॒वा॒ पृ॒थि॒व्यौ च॑ नाराय॒णः । का॒लश्च॑ नाराय॒णः ।\\
दि॒शश्च॑ नाराय॒णः । ऊ॒र्ध्वश्च॑ नाराय॒णः ।\\
अ॒धश्च॑ नाराय॒णः । अ॒न्त॒र्ब॒हिश्च॑ नाराय॒णः ।\\
नारायण एवे॑दग्ं स॒र्वम् ।\\
यद्भू॒तं-यँच्च॒ भव्यम्᳚ ।\\
निष्कलो निरञ्जनो निर्विकल्पो\\
निराख्यातः शुद्धो देव एको॑ नाराय॒णः । \\
न द्वि॒तीयो᳚स्ति॒ कश्चि॑त् । य ए॑वं-वेँ॒द ।\\
स विष्णुरेव भवति स विष्णुरे॑व भ॒वति ॥\\
\\
ओमित्य॑ग्रे व्या॒हरेत् । नम इ॑ति प॒श्चात् ।\\
ना॒रा॒य॒णा येत् यु॑परि॒ष्टात् ।\\
ओमि॑त्ये का॒क्षरम् । नम इति॑ द्वे अ॒क्षरे ।\\
ना॒रा॒य॒णा येति पञ्चा᳚क्षरा॒णि ।\\
एतद्वै नारायणस् याष्टाक्ष॑रं प॒दम् ।\\
यो ह वै नारायणस् याष्टाक्षरं पद॑मध्ये॒ति ।\\
अनपब्रवस् सर्वमा॑युरे॒ति ।\\
विन्दते प्रा॑जा प॒त्यग्ं रायस् पोषं॑ गौ प॒त्यम् ।\\
ततोऽमृतत्वमश्नुते ततोऽमृतत्वमश्नु॑त इ॒ति ।\\
य ए॑वं-वेँ॒द ॥\\
\\
प्रत्यगानन्दं ब्रह्म पुरुषं प्रण व॑स्व रू॒पम् ।\\
अकार उकार मका॑र इ॒ति ।\\
तानेकधा समभरत्तदे त॑दो मि॒ति ।\\
यमुक्त्वा॑ मुच्य॑ते यो॒गी॒ ज॒न्म॒ संसा॑र ब॒न्धनात् ।\\
ॐ नमो नारायणा येति म॑न्त्रोपा॒सकः ।\\
वैकुण्ठभुवनलोकं॑ गमि॒ष्यति ।\\
तदिदं परं पुण्डरीकं-विँ॑ज्ञा न॒घनम् ।\\
तस्मात् तदिदा॑ वन्मा॒त्रम् ।\\
ब्रह्मण्यो देव॑की पु॒त्रो॒ ब्रह्मण्यो म॑धुसू॒दनोम् ।\\
सर्वभूतस्थमेकं॑ नारा॒यणम् ।\\
कारण रूप मकार प॑र ब्र॒ह्मोम् ।\\
एतद थर्व शिरो॑योऽधी॒ते प्रा॒तर॑धी या॒नो॒ रात्रिकृतं पापं॑ नाश॒यति ।\\
सा॒य म॑धीया॒नो॒ दिवसकृतं पापं॑ नाश॒यति ।\\
माध् यन् दिन मादित्या भिमुखो॑ऽधीया॒नः॒ \\
पञ्च पातको पपातका᳚त् प्रमु॒च्यते ।\\
सर्व वेद पारायण पु॑ण्यं-लँ॒भते ।\\
नारायण सायुज्य म॑वाप्नो॒ति॒ नारायण सायुज्यम॑वाप्नो॒ति ।\\
य ए॑वं-वेँ॒द । इत्यु॑प॒निष॑त् ॥\\
\\
ॐ स॒ह ना॑ववतु । स॒ह नौ॑ भुनक्तु । स॒ह वी॒र्यं॑ करवावहै ।\\
ते॒ज॒स्विना॒वधी॑तमस्तु॒ मा वि॑द्विषा॒वहै᳚ ॥\\
ॐ शान्तिः॒ शान्तिः॒ शान्तिः॑ ॥\\
