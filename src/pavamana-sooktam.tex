\section{\eng{Punniyahavachanam}}
\subsection{\eng{Pavamana Sooktam}}
ॐ ॥ हिर॑ण्यवर्णा॒: शुच॑यः पाव॒का \\
यासु॑ जा॒तः क॒श्यपो॒ यास्विन्द्र॑: ।\\
अ॒ग्निं या गर्भं॑ दधि॒रे विरू॑पा॒स्ता न॒ \\
आप॒श्शग्ग् स्यो॒ना भ॑वन्तु ॥\\
\\
यासा॒ग्ं॒ राजा॒ वरु॑णो॒ याति॒ मध्ये॑ सत्यानृ॒ते, \\
अ॑व॒ पश्यं॒ जना॑नाम् ।\\
म॒धु॒श् चुत॒श् शुच॑ यो॒ याः \\
पा॑व॒कास्ता न॒ आप॒श्शग्ग् स्यो॒ना भ॑वन्तु ॥\\
\\
यासां᳚ दे॒वा दि॒वि कृ॒ण्वन्ति॑ भ॒क्षं या,\\
अ॒न्तरि॑क्षे बहु॒धा भव॑न्ति ।\\
याः पृ॑थि॒वीं पय॑सो॒न् दन्ति शु॒क्रास्ता न॒ \\
आप॒श्शग्ग् स्यो॒ना भ॑वन्तु ॥\\
\\
शि॒वेन॑ मा॒ चक्षु॑षा पश्यता पश्शि॒वया॑\\
त॒नुवो प॑स् पृश त॒त् वचं॑ मे ।\\
सर्वाग्ं॑ अ॒ग्नीग्ं र॑प्सु॒षदो॑ हुवे वो॒ मयि॒\\
वर्चो॒ बल॒मो जो॒ निध॑त्त ॥\\
\\
पव॑मान॒स्सुव॒र्जन॑: । प॒वित्रे॑ण॒ विच॑र्षणिः ।\\
यः पोता॒ स पु॑नातु मा । पु॒नन्तु॑ मा देव ज॒नाः ।\\
पु॒नन्तु॒ मन॑वो धि॒या । पु॒नन्तु॒ विश्व॑ आ॒यव॑: ।\\
जात॑वेदः प॒वित्र॑वत् । प॒वित्रे॑ण पुनाहि मा ।\\
शु॒क्रेण॑ देव॒ दीद्य॑त् । अग्ने॒ क्रत्वा॒ क्रतू॒ग्ं॒ रनु॑ ।\\
यत्ते॑ प॒वित्र॑म॒र्चिषि॑ । अग्ने॒ वित॑त मन्त॒रा ।\\
ब्रह्म॒ तेन॑ पुनीमहे । उ॒भाभ्यां᳚ देवसवितः ।\\
प॒वित्रे॑ण स॒वेन॑ च । इ॒दं ब्रह्म॑ पुनीमहे ।\\
वै॒श्व॒ दे॒वी पु॑न॒ती दे॒व्या गा᳚त् ।\\
\\
यस्यै॑ ब॒ह्वीस् त॒नुवो॑ वी॒त पृ॑ष्ठाः ।\\
तया॒ मद॑न्तः सध॒ माद्ये॑षु ।\\
व॒यग्ग् स्या॑म॒ पत॑यो रयी॒णाम् ।\\
वै॒श्वा॒न॒रो र॒श्मिभि॑र्मा पुनातु ।\\
वातः॑ प्रा॒णेने॑ षि॒रो म॑यो॒ भूः ।\\
द्यावा॑ पृथि॒वी पय॑सा॒ पयो॑भिः ।\\
ऋ॒ताव॑री य॒ज्ञिये॑ मा पुनीताम् ॥\\
\\
बृ॒हद् भिः॑ सवि त॒स् तृभिः॑ । वर्षि॑ष् ठैर् देव॒ मन्म॑भिः । \\
अग्ने॒ दक्षैः᳚ पुनाहि मा । येन॑ दे॒वा, अपु॑नत । \\
येनापो॑ दि॒व्यङ्कशः॑ । तेन॑ दि॒व्येन॒ ब्रह्म॑णा । \\
इ॒दं ब्रह्म॑ पुनीमहे । यः पा॑वमा॒नी र॒द्ध्येति॑ । \\
ऋषि॑भि॒स् सम्भृ॑त॒ग्ं॒ रसम्᳚ । सर्व॒ग्ं॒ स पू॒त म॑श्नाति । \\
स्व॒दि॒तं मा॑त॒रिश्व॑ना । पा॒व॒मा॒नीर्यो अ॒ध्येति॑ । \\
ऋषि॑भि॒स्सम्भृ॑त॒ग्ं॒ रसम्᳚ । तस्मै॒ सर॑स्वती दुहे । \\
क्षी॒रग्ं स॒र्पिर् मधू॑ द॒कम् ॥\\
\\
पा॒व॒मा॒नीस् स्व॒स्त् यय॑नीः । सु॒दुघा॒हि पय॑स्वतीः । \\
ऋषि॑भि॒स्सम्भृ॑तो॒ रसः॑ । ब्रा॒ह्म॒णेष्व॒मृतग्ं॑ हि॒तम् । \\
पा॒व॒मा॒नीर्दि॑शन्तु नः । इ॒मं लो॒कमथो॑ अ॒मुम् । \\
कामा॒न्थ्सम॑र् धयन्तु नः । दे॒वी‍र्दे॒वैः स॒माभृ॑ताः । \\
पा॒व॒मा॒नीस् स्व॒स्त् यय॑नीः । सु॒दु घा॒हि घृ॑त॒श् चुतः॑ । \\
ऋषि॑भिः॒ सम्भृ॑तो॒ रसः॑ । ब्रा॒ह्म॒णेष्व॒मृतग्ं॑ हि॒तम् । \\
\\
येन॑ दे॒वाः प॒वित्रे॑ण । आ॒त्मानं॑ पु॒नते॒ सदा᳚ । \\
तेन॑ स॒हस्र॑धारेण । पा॒व॒मा॒न्यः पु॑नन्तु मा । \\
प्रा॒जा॒प॒त्यं प॒वित्रम्᳚ । श॒तोद्या॑मग्ं हिर॒ण्मयम्᳚ । \\
तेन॑ ब्रह्म॒ विदो॑ व॒यम् । पू॒तं ब्रह्म॑ पुनीमहे । \\
इन्द्र॑स्सुनी॒ती स॒हमा॑ पुनातु । सोम॑स् स्व॒स्त्या व॑रुणस् स॒मीच्या᳚ । \\
य॒मो राजा᳚ प्रमृ॒ णाभिः॑ पुनातु मा । \\
जा॒तवे॑दा मो॒र्जय॑न्त्या पुनातु । भूर्भुव॒स्सुवः॑ ॥\\
\\
तच्छं॒ योरावृ॑णीमहे । गा॒तुं य॒ज्ञाय॑ ।\\
गा॒तुं य॒ज्ञप॑तये । दैवी᳚स्स्व॒स्तिर॑स्तु नः । \\
स्व॒स्तिर्मानु॑षेभ्यः । ऊ॒र्ध्वं जि॑गातु भेष॒जम् ।\\
शं नो॑ अस्तु द्वि॒पदे᳚ । शं चतु॑ष्पदे ॥\\
\\
ॐ शान्तिः॒ शान्तिः॒ शान्तिः॑ ॥\\
\\
ॐ नमो॒ ब्रह्म॑णे॒ नमो॑ अस्त् व॒ग्नये॒ नमः॑ पृथि॒व्यै नम॒ ओष॑धीभ्यः ।\\
नमो॑ वा॒चे नमो॑ वा॒चस्पत॑ये॒ नमो॒ विष्ण॑वे बृह॒ते क॑रोमि ॥\\
\subsection{\eng{Vasthoshpatha Mantram }}
वास्तो᳚ष्पते॒ प्रति॑जानी ह्॒यस्मान्  स्वा॑वे॒शो अ॑नमी॒वो भवानः। \\
यत्वे महे॒॑ प्रति॒ तन्नो॑ जुषस्व॒ शन्न॑ एघि द्वि॒पदे॒ शं चतु॑ष्पदे। \\
वास्तो᳚ष्पते श॒ग्मया सꣳ॒सदा॑ते सक्षी॒महि॑र॒ण्वया॑ गातु॒मत्या᳚ ॥\\
\\
आवः॒ , क्षेम॑ उ॒तयो गे॒वर॑न्नो यू॒यं पा॑त स्व॒स्तिभि॒ स्सदा॑नः \\
वास्तो᳚ष्पते प्र॒तर॑णो न एधि॒ गोभि॒रश्वे॑ भिरिन्दो ।\\
अ॒जरा॑ सस्ते सख्ये स्या॑म पि॒तवे॑ पु॒त्रान् प्रति॑नो जुषस्व ।\\
अ॒मी॒व॒हा वास्तो᳚ष्पते॒ विश्वा॑ रू॒पाण्या॑ वि॒शन्न् । सखा॑ सुशेव॑ एधिनः \\
\\
शि॒व॒ꣳ शि॒वं भूर्भुव॒स्सुवो॒ भूर्भुव॒स्सुवो॒ भूर्भुव॒स्सुवः॑ ॥\\
\\
आपो॒हिष्ठा म॑यो॒ भुवः ता॑न ऊ॒र्जे द॑धातन म॒हेरणा॑य चक्ष॑से ॥\\
योवः॑ शि॒वतमो॒ रसः तस्य॑ भाजयते॒हनः उश॒तीरिव मा॒तरः ॥\\
तस्मा॒ अरं॑गमाम वः यस्य॒क्षया॑य॒ जिन्व॑थ आपो॑ ज॒नय॑था च नः ॥\\
\subsection{\eng{Prokshana Mantram}}
दे॒वस्य॑ त्वा सवि॒तुः प्र॑स॒वे ।अ॒श्विनो᳚-र्बा॒हुभ्या᳚म् । पू॒ष्णो हस्ता᳚भ्याम् ॥\\
अ॒श्विनो॒ र्भैषज्येन  तेज॑से ब्रह्मवर्च॒सा या॒भिषिञ्चामि ॥ \\
दे॒वस्य॑ त्वा सवितुः प्रस॒वे । अ॒श्विनो᳚ र्बा॒हुभ्या᳚म् । पूष्णो हस्ता᳚भ्याम् ।\\
सर॑स्वत्यै॒ भैष॑ज्येन । वी॒र्याया॒-न्नाद्या॑या॒-भिषि॑ञ्चामि ॥ \\
दे॒वस्य॑ त्वा सवि॒तुः प्र॑स॒वे । अ॒श्विनो᳚-र्बा॒हुभ्या᳚म् । पूष्णो हस्ता᳚भ्याम् ।\\
इन्द्र॑स्येन्द्रि॒येण । श्रि॒यै यशसे॒ बलाया॒-भिषिञ्चामि ॥\\
\\
च॒वाजिनम्᳚ । सोम॒ꣳ॒ राजा॑नं॒ वरु॑ण मा॒ग्नि म॒न्वार॑ भामहे ।\\
आ॒दि॒त्यान् विष्णु॒ꣳ॒ सूर्यं॑ ब्र॒ह्माणं॑ च बृहस्पति᳚म् ।\\
दे॒वस्य॑ त्वा सवि॒तुः प्र॑स॒वे᳚ऽश्विनो᳚-र्बा॒हुभ्यां᳚ \\
पू॒ष्णो हस्ता᳚भ्या॒ꣳ॒ सर॑स्वत्यै वा॒चो य॒न्तु र्य॒न्त्रे णा॒ग् नेस्त्वा॒ \\
साम् रा᳚ज्ये ना॒भि षि॑ञ्चा॒मीन् द्र॑स् यत्वा \\
साम्रा᳚ज्ये ना॒भि षि॑ञ्चामी॒\\
बृह॒स्पते᳚स्त्वा॒ साम्रा᳚ज्ये ना॒भि षि॑ञ्चामी॥\\
\\
आयु॒राशा᳚स्ते । सु॒प्र॒जा॒-स्तव माशा᳚स्ते ।\\
स॒जा॒त॒ व॒न॒-स्या माशा᳚स्ते। उत्त॑रां देव य॒ज्या माशा᳚स्ते। \\
भूयो॑ हवि॒ष्कर॑ण माशा᳚स्ते । दि॒व्यं घामाशा᳚स्ते ।\\
विश्वं॑ प्रि॒य माशा᳚स्ते । यद॒नेन ह॒विषाऽऽशा᳚स्ते ।\\
तद॑श्या॒ त्तदृ॑ ध्यात् । तद॑स्मै दे॒वारा॑ सन्ताम् ।\\
तद॒ग्नि र्दे॒वो दे॒वेभ्यो॒ वन॑ते व॒य म॒ग्ने-र्मानु॑षाः ।\\
इ॒ष्टं च॑ वी॒तं च॑ उ॒मे च॑ नो॒ द्यावा॑ पृथि॒वि अꣳ ह॑सः स्पाताम् \\
इ॒ह गति॑-र्वा॒मस्ये॒दं च॑ । नमो॑ दे॒वेभ्यः ॥\\
\\
द्रु॒प॒दा दि॒वे-न्मु॑ मुचानः। स्वि॒न्नः स्ना॒त्वी मला॑दिव। \\
पू॒तं प॒वित्रे॑णे॒ वाज्यम्᳚ । आप॑ श्शुन्धन्तु॒ मैन॑सः। \\
भूर्भुव॒सुवो॒ भूर्भुव॒सुवो॒ भूर्भुव॒सुवः ॥\\
\\
आपो॒ वा इ॒दꣳ सर्वं॒ विश्वा॑ भू॒तान्यापः॑ प्रा॒णा वा आपः॑ \\
प॒शव॒ आपोऽन्न॒-मापोऽमृ॑त॒-मापः॑ स॒म्राडापो॑ वि॒राडापः॑ \\
स्व॒राडाप॒-श्छन्दा॒ꣳ॒ स्यापो॒ ज्योति॒ꣳ॒ ष्यापो॒ यजू॒ꣳ॒ ष्याप \\
स्स॒त्य माप॒-स्सर्वा दे॒वता॒ आपो॒ भूर्भुव॒सुव॒-राप॒ ओं ॥\\
\subsection{\eng{Prashana Mantram}}
प्राशनं मंत्र जपं कुर्यात् \\
\\
आप इद्वा उपेषजीः आपो अमी वचातनीः \\
आप सर्वस्य भेषजीः तास्ते कृण्वन्तु भेषजम \\
अकाल मृत्यु हरणम् सर्व व्याधि निवारणम् \\
सर्व पाप क्षयकरम् वरुण पदोदकम् शुभम् \\
\\
आ॒म॒या॒वी चि॑न्वीत। आपो वै भे॑षजम्। \\
भेष॒जमे॒ वास्मै॑ करोति। सर्व॒ मायु॑रेति। \\
\\
॥उदकशन्ति मन्त्र पाठः समाप्तः ॥\\
