\section{\eng{Panchayudha Stotram - }पञ्चायुध स्तोत्रम्}
स्फुरत्सहस्रारशिखातितीव्रं\\
सुदर्शनं भास्करकोटितुल्यम् ।\\
सुरद्विषां प्राणविनाशि विष्णोः\\
चक्रं सदाऽहं शरणं प्रपद्ये ॥ 1 ॥\\
\\
विष्णोर्मुखोत्थानिलपूरितस्य\\
यस्य ध्वनिर्दानवदर्पहन्ता ।\\
तं पाञ्चजन्यं शशिकोटिशुभ्रं\\
शङ्खं सदाऽहं शरणं प्रपद्ये ॥ 2 ॥\\
\\
हिरण्मयीं मेरुसमानसारां\\
कौमोदकीं दैत्यकुलैकहन्त्रीम् ।\\
वैकुण्ठवामाग्रकराभिमृष्टां\\
गदां सदाऽहं शरणं प्रपद्ये ॥ 3 ॥\\
\\
रक्षोऽसुराणां कठिनोग्रकण्ठ-\\
-च्छेदक्षरच्छोणितदिग्धधारम् ।\\
तं नन्दकं नाम हरेः प्रदीप्तं\\
खड्गं सदाऽहं शरणं प्रपद्ये ॥ 4 ॥\\
\\
यज्ज्यानिनादश्रवणात् सुराणां\\
चेतांसि निर्मुक्तभयानि सद्यः ।\\
भवन्ति दैत्याशनिबाणवर्षि\\
शार्ङ्गं सदाऽहं शरणं प्रपद्ये ॥ 5 ॥\\
\\
इमं हरेः पञ्चमहायुधानां\\
स्तवं पठेद्योऽनुदिनं प्रभाते ।\\
समस्तदुःखानि भयानि सद्यः\\
पापानि नश्यन्ति सुखानि सन्ति ॥ 6 ॥\\
\\
वने रणे शत्रु जलाग्निमध्ये\\
यदृच्छयापत्सु महाभयेषु ।\\
इदं पठन् स्तोत्रमनाकुलात्मा\\
सुखी भवेत् तत्कृत सर्वरक्षः ॥ 7 ॥\\
\\
[* अधिक श्लोकाः –\\
यच्चक्रशङ्खं गदखड्गशार्ङ्गिणं\\
पीताम्बरं कौस्तुभवत्सलाञ्छितम् ।\\
श्रियासमेतोज्ज्वलशोभिताङ्गं\\
विष्णुं सदाऽहं शरणं प्रपद्ये ॥\\
\\
जले रक्षतु वाराहः स्थले रक्षतु वामनः ।\\
अटव्यां नारसिंहश्क्ष्च सर्वतः पातु केशवः ॥ *]\\
