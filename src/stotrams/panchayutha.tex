\section{\eng{Panchayudha Stotram - }पञ्चायुध स्तोत्रम्}
स्फुरत् सहस्रार शिखाति तीव्रं\\
सुदर्शनं भास्कर कोटि तुल्यम् ।\\
सुरद्विषां प्राण विनाशि विष्णोः\\
चक्रं सदाऽहं शरणं प्रपद्ये ॥ 1 ॥\\
\\
विष्णोर् मुखोत् थानिल पूरितस्य\\
यस्य ध्वनिर् दान वदर्प हन्ता ।\\
तं पाञ्चजन्यं शशि कोटि शुभ्रं\\
शङ्खं सदाऽहं शरणं प्रपद्ये ॥ 2 ॥\\
\\
हिरण्मयीं मेरु समान सारां\\
कौमोदकीं दैत्य कुलैक हन्त्रीम् ।\\
वैकुण्ठ वामाग्र कराभि   मृष्टां\\
गदां सदाऽहं शरणं प्रपद्ये ॥ 3 ॥\\
\\
रक्षोऽ सुराणां कठिनोग्र कण्ठच्-\\
छेदक्षरच् छोणित दिग्ध धारम् ।\\
तं नन्दकं नाम हरेः प्रदीप्तं\\
खड्गं सदाऽहं शरणं प्रपद्ये ॥ 4 ॥\\
\\
यज् ज्यानि नादश् श्रवणात् सुराणां\\
चेतांसि निर्मुक्त भयानि सद्यः ।\\
भवन्ति दैत्या शनि बाण वर्षि\\
शार्ङ्गं सदाऽहं शरणं प्रपद्ये ॥ 5 ॥\\
\\
इमं हरेः पञ्च महा युधानां\\
स्तवं पठेद्योऽ नुदिनं प्रभाते ।\\
समस्त दुःखानि भयानि सद्यः\\
पापानि नश्यन्ति सुखानि सन्ति ॥ 6 ॥\\
\\
वने रणे शत्रु जलाग्नि मध्ये\\
यदृच्छया पत्सु महा भयेषु ।\\
इदं पठन् स्तोत्र मना कुलात्मा\\
सुखी भवेत् तत्कृत सर्व रक्षः ॥ 7 ॥\\
\\
[* अधिक श्लोकाः –\\
यच्चक्रशङ्खं गदखड्गशार्ङ्गिणं\\
पीताम्बरं कौस्तुभवत्सलाञ्छितम् ।\\
श्रियासमेतोज्ज्वलशोभिताङ्गं\\
विष्णुं सदाऽहं शरणं प्रपद्ये ॥\\
\\
जले रक्षतु वाराहः स्थले रक्षतु वामनः ।\\
अटव्यां नारसिंहश्क्ष्च सर्वतः पातु केशवः ॥ *]\\
