\section{\eng{Lalitha Sahasranamam}}
\subsection{\eng{Nyasaha}}
प्रणो देवी सरस्वती वाजेभिर्वजिनीवती धीनामवित्रयवतु\\
मानो दि गो बृहद: पर्वतादा सरस्वथि यजथा गन्तु यज्ञं \\
पवन् देवी जुजु शना गृताची शग् मान् भोपा च मुशती श्रुणोतु \\
वाग्देव्यै नमः \\
\\
ॐ अस्य श्री ललिता सहस्र नामस् स्तोत्र माला मन्त्रस्य ।\\
वशिन्या दि वाग् देवता ऋषयः ।\\
अनुष्टुप् छन्दः ।\\
श्रीललिता परमेश्वरी देवता ।\\
श्रीमद् वाग् भव कूटेति बीजम् ।\\
मध्य कूटेति शक्तिः ।\\
शक्ति कूटेति कीलकम् ।\\
श्रीललिता महा त्रिपुर सुन्दरी प्रसाद सिद्धिद् द्वारा \\
चिन् तित फला वाप्त्-यर्थे (जपे) पारायणे विनियोगः ।\\
\subsection{\eng{Dhyanam}}
सिन्दूरारुण विग्रहां त्रिनयनां माणिक्य मौलिस्फुरत्\\
तारा नायक शेखरां स्मित मुखी मापीन वक्षो रुहाम् ।\\
पाणिभ् यामलि पूर्ण रत्न चषकं रक्तोत् पलं विभ् रतीं\\
सौम्यां रत्न घटस्थ रक्त चरणां ध्यायेत् परा मम्बिकाम् ॥\\
\subsection{\eng{Shloka 1-20}}
ॐ श्रीमाता श्रीमहा राज्ञी श्रीमत् सिंहास नेश्वरी ।\\
चिदग्नि कुण्ड सम्भूता देव कार्य समुद्यता ॥1॥\\
\\
उद्यद् भानु सहस् राभा चतुर् बाहु समन्विता ।\\
रागस् वरूप पाशाढ्या क्रोधा काराङ्कु शोज् ज्वला ॥2॥\\
\\
मनो रूपेक्षु कोदण्डा पञ्च तन्मात्र सायका ।\\
निजा रुणप् प्रभा पूर मज्जद् ब्रह्माण्ड मण्डला ॥3॥\\
\\
चम्प काशोक पुन्नाग सौगन् धिक लसत् कचा ।\\
कुरु विन्द मणिश् श्रेणी कनत् कोटीर मण्डिता ॥4॥\\
\\
अष्टमी चन्द्र विभ्राज दलि कस्थल शोभिता ।\\
मुख चन्द्रक लङ्काभ मृग नाभि विशे षका ॥5॥\\
\\
वद नस् मर माङ्गल्य गृह तोरण चिल्लिका ।\\
वक्त्र लक्ष्मी परीवाह चलन् मीनाभ लोचना ॥6॥\\
\\
नव चम्पक पुष् पाभ नासा दण्ड विराजिता ।\\
तारा कान्ति तिरस् कारि नासा भरण भासुरा ॥7॥\\
\\
कदम्ब मञ्जरी कॢप्त कर्ण पूर मनोहरा ।\\
ताटङ्क युगली भूत तप नो डुप मण्डला ॥8॥\\
\\
पद्म राग शिला दर्श परि भावि कपोल भूः ।\\
नव विद्रुम बिम्ब श्री न्यक् कारि रद नच्छदा ॥9॥\\
{\small\eng{or} दशनच्छदा}\\
\\
शुद्ध विद् याङ्कु रा कारद् द्विज पङ्क् तिद् द्वयोज् ज्वला ।\\
कर्पूर वीटि कामोद समा कर्षद् (कर्षि) दिगन् तरा ॥10॥\\
\\
निज सल्लाप मा धुर्य विनिर् भर्त्-सित कच्छपी । \\
{\small\eng{or} निज संलाप}\\
मन् दस् मितप् प्रभा पूर मज्जत् कामेश मानसा ॥11॥\\
\\
अना कलित सादृश्य चुबु कश्री विरा जिता । (चिबुकश्री)\\
कामेश बद्ध माङ्गल्य सूत्र शो भित कन्धरा ॥12॥\\
\\
कन काङ् गद केयूर कम नीय भुजान् विता ।\\
रत्नग् ग्रै वेय चिन् ताक लोल मुक्ता फलान् विता ॥13॥\\
\\
कामेश्वरप् प्रेम रत्न मणिप् प्रति पणस् स्तनी ।\\
नाभ्या लवाल रो मालि लता फल कुचद् वयी ॥14॥\\
\\
लक्ष्य रोम लता धार तास मुन्नेय मध्यमा ।\\
स्तन भार दलन् मध्य पट्ट बन्ध वलित् त्रया ॥15॥\\
\\
अरुणा रुण कौ सुम्भ वस्त्र भास् वत् कटी तटी ।	\\
रत्न किङ्किणि कारम्य रश नादा मभू षिता ॥16॥\\
\\
कामेशग् ज्ञात सौभाग्य मार्द वोरुद् द्वयान् विता ।\\
माणिक्य मकु टाकार जानुद् वय विरा जिता ॥17॥\\
\\
इन्द्र गोप परि क्षिप्तस् स्मर तू णाभ जङ् घिका ।\\
गूढ गुल्फा कूर्म पृष्ठ जयिष् णुप् प्रप दान् विता ॥18॥\\
\\
नख दी धिति सं छन्न नमज् जन तमो गुणा ।\\
पदद् वयप् प्रभा जाल परा कृत सरो रुहा ॥19॥\\
\\
सिञ्जान मणि मञ्जीर मण्डि तश् श्री पदाम् बुजा ।\\
{\small\eng{or} शिञ्जान}\\
मराली मन्द गमना महा लावण्य शेवधिः ॥20॥\\
\subsection{\eng{Shloka 21-40}}
सर्वा रुणाऽ नवद् याङ्गी सर्वा भरण भूषिता ।\\
शिव कामेश् वराङ् कस्था शिवा स्वाधीन वल्लभा ॥21॥\\
\\
सुमेरु मध्य श‍ृङ् गस्था श्रीमन् नगर नायिका ।\\
चिन्ता मणि गृहान् तस्था पञ्चब् ब्रह्मा सनस् थिता ॥22॥\\
\\
महा पद्मा टवी संस्था कदम्ब वन वासिनी ।\\
सुधा सागर मध् यस्था कामाक्षी काम दायिनी ॥23॥\\
\\
देवर् षिगण संघातस् स्तूय मानात् मवै भवा ।\\
भण्डा सुर वधोद् युक्त शक्ति सेना समन्विता ॥24॥\\
\\
सम्पत् करी समा रूढ सिन्धुरव् व्रज सेविता ।\\
अश्वा रूढा धिष् ठिताश्व कोटि कोटि भिरा वृता ॥25॥\\
\\
चक्र राज रथा रूढ सर्वा युध परिष् कृता ।\\
गेय चक्र रथा रूढ मन्त्रिणी परि सेविता ॥26॥ \\
\\
किरि चक्र रथा रूढ दण्ड नाथा पुरस् कृता ।\\
ज्वाला मालिनि काक् क्षिप्त वह्निप् प्राकार मध्यगा ॥27॥\\
\\
भण्ड सैन्य वधोद् युक्त शक्ति विक्रम हर्षिता ।\\
नित्या पराक् रमा टोप निरीक् क्षण समुत् सुका ॥28॥\\
\\
भण्ड पुत्र वधोद् युक्त बाला विक्रम नन्दिता ।\\
मन् त्रिण् यं बावि रचित विषङ्ग वध तोषिता ॥29॥\\
\\
विशुक्रप् प्राण हरण वाराही वीर्य नन्दिता ।\\
कामेश्वर मुखा लोक कल्पितश् श्री गणेश्वरा ॥30॥\\
\\
महा गणेश निर् भिन्न विघ्न यन्त्रप् प्रहर् षिता ।\\
भण्डा सुरेन्द्र निर् मुक्त शस्त्रप् प्रत्-यस्त्र वर्षिणी ॥31॥\\
\\
कराङ् गुलिन खोत् पन्न नारायण दशा कृतिः ।\\
महा पाशु पतास् त्राग्नि निर् दग् धा सुर सैनिका ॥32॥\\
\\
कामेश् वरास् स्त्र निर् दग्ध सभण्डा सुर शून्यका ।\\
ब्रह्मो पेन्द्र महेन् द्रादि देव संस्तुत वैभवा ॥33॥\\
\\
हर नेत्राग्नि सन् दग्ध काम सञ्जीव नौ षधिः ।\\
श्रीमद् वाग् भव कूटैकस् स्वरूप मुख पङ्कजा ॥34॥\\
\\
कण्ठा धः कटि पर्यन्त मध्य कूटस् स्वरूपिणी ।\\
शक्ति कूटैक तापन्न कट्य धो भाग धारिणी ॥35॥\\
\\
मूल मन्त्रात् मिका मूल कूटत् त्रय कले वरा ।\\
कुला मृतैक रसिका कुल संकेत पालिनी ॥36॥\\
\\
कुलाङ्गना कुलान् तस्था कौलिनी कुल योगिनी ।\\
अकुला समयान् तस्था सम याचार तत्परा ॥37॥\\
\\
मूला धारैक निलया ब्रह्मग् ग्रन्थि विभेदिनी ।\\
मणि पूरान् त रुदिता विष्णुग् ग्रन्थि विभेदिनी ॥38॥\\
\\
आज्ञा चक्रान् तरा लस्था रुद्रग् ग्रन्थि विभेदिनी ।\\
सहस् राराम् बुजा रूढा सुधा साराभि वर्षिणी ॥39॥\\
\\
तडिल् लता सम रुचिः षट् चक्रो परि संस्थिता ।\\
महा सक्तिः कुण्ड लिनी बिस तन्तु तनी यसी ॥40॥\\
\subsection{\eng{Shloka 41-60}}
भवानी भाव नागम्या भवा रण्य कुठा रिका ।\\
भद्रप् प्रिया भद्र मूर्तिर् भक्त सौभाग्य दायिनी ॥41॥\\
\\
भक्तिप् प्रिया भक्ति गम्या भक्ति वश्या भयापहा ।\\
शाम्भवी शारदा राध्या शर्वाणी शर्म दायिनी ॥42॥\\
\\
शाङ्करी श्रीकरी साध्वी शरच् चन्द्र निभा नना ।\\
शातो दरी शान्ति मती निरा धारा निरञ्जना ॥43॥\\
\\
निर् लेपा निर्मला नित्या निरा कारा निरा कुला ।\\
निर्गुणा निष्कला शान्ता निष्कामा निरु पप् लवा ॥44॥\\
\\
नित्य मुक्ता निर्वि कारा निष् प्रपञ्चा निराश् श्रया ।\\
नित्य शुद्धा नित्य बुद्धा निर वद्या निरन्तरा ॥45॥\\
\\
निष् कारणा निष् कलङ्का निरु पाधिर् निरीश् वरा ।\\
नीरागा राग मथनी निर् मदा मद नाशिनी ॥46॥\\
\\
निश् चिन्ता निर हंकारा निर् मोहा मोह नाशिनी ।\\
निर्ममा ममता हन्त्री निष् पापा पाप नाशिनी ॥47॥\\
\\
निष् क्रोधा  क्रोध शमनी निर् लोभा लोभ नाशिनी ।\\
निस् संशया संश यघ्नी निर्भवा भव नाशिनी ॥48॥\\
\\
निर्वि कल्पा निरा बाधा निर् भेदा भेद नाशिनी ।\\
निर् नाशा मृत्यु मथनी निष् क्रिया निष् परिग् ग्रहा ॥49॥\\
\\
निस् तुला नील चि कुरा निर पाया निरत् यया ।\\
दुर्लभा दुर्गमा दुर्गा दुःख हन्त्री सुखप् प्रदा ॥50॥\\
\\
दुष्ट दूरा दुरा चार शमनी दोष वर्जिता ।\\
सर्वज्ञा सान्द्र करुणा समा नाधिक वर्जिता ॥51॥\\
\\
सर्व शक्ति मयी सर्व मङ्गला सद्गतिप् प्रदा ।\\
सर्वेश्वरी सर्व मयी सर्व मन्त्रस् स्वरूपिणी ॥52॥\\
\\
सर्व यन्त्रात् मिका सर्व तन्त्र रूपा मनोन्मनी ।\\
माहेश्वरी महा देवी महा लक्ष्मीर् मृडप् प्रिया ॥53॥\\
\\
महा रूपा महा पूज्या महा पातक नाशिनी ।\\
महा माया महा सत् त्वा महा शक्तिर् महा रतिः ॥54॥\\
\\
महा भोगा महैश् वर्या महा वीर्या महा बला ।\\
महा बुद्धिर् महा सिद्धिर् महा योगेश् वरेश् वरी ॥55॥\\
\\
महा तन्त्रा महा मन्त्रा महा यन्त्रा महा सना ।\\
महा यागक् क्रमा राध्या महा भैरव पूजिता ॥56॥\\
\\
महेश्वर महा कल्प महा ताण्डव साक्षिणी ।\\
महा कामेश महिषी महा, त्रिपुर सुन्दरी ॥57॥\\
\\
चतुष् षष् ट्यु पचा राढ्या चतुष् षष् टिक लामयी ।\\
महा चतुष् षष्टि कोटि योगिनी गण सेविता ॥58॥\\
\\
मनु विद्या चन्द्र विद्या चन्द्र मण्डल मध्यगा ।\\
चारु रूपा चारु हासा चारु चन्द्र कलाधरा ॥59॥\\
\\
चरा चर जगन् नाथा चक्र राज निके तना ।\\
पार्वती पद्म नयना पद्म राग समप् प्रभा ॥60॥\\
\subsection{\eng{Shloka 61-80}}
पञ्चप् प्रेता सना सीना पञ्चब् ब्रह्मस् स्व रूपिणी ।\\
चिन्मयी परमा नन्दा विज्ञान घन रूपिणी ॥61॥\\
\\
ध्या नध् ध्या तृध् ध्येय रूपा धर्मा धर्म विवर् जिता ।\\
विश्व रूपा जाग रिणी स्व पन्ती तैज सात् मिका ॥62॥\\
\\
सुप् ताप् प्राज्ञात् मिका तुर्या सर्वा वस्था विवर् जिता ।\\
सृष्टि कर्त्री ब्रह्म रूपा गोप्त्री गोविन्द रूपिणी ॥63॥\\
\\
संहा रिणी रुद्र रूपा तिरोधा नक रीश्वरी ।\\
सदा शिवाऽ नुग्र हदा पञ्च कृत्य परायणा ॥64॥\\
\\
भानु मण्डल मध् यस्था भैरवी भग मालिनी ।\\
पद्मा सना भगवती पद्म नाभ सहोदरी ॥65॥\\
\\
उन् मेष निमि षोत् पन्न विपन्न भुव नावलीः ।\\
सहस्र शीर् ष वदना सहस्राक्षी सहस्रपात् ॥66॥\\
\\
आ ब्रह्म कीट जननी वर्णाश् श्रम विधा यिनी ।\\
निजा ज्ञा रूप निगमा पुण्या पुण्य फल प्रदा ॥67॥\\
\\
श्रुति सीमन्त सिन्दूरी कृत पादाब् ज धूलिका ।\\
सकला गम सन् दोह शुक्ति सम् पुट मौक् तिका ॥68॥\\
\\
पुरु षार्थप् प्रदा पूर्णा भोगिनी भुव नेश्वरी ।\\
अम्बिकाऽ नादि निधना हरिब् ब्रह्मेन्द्र सेविता ॥69॥\\
\\
नारा यणी नाद रूपा नाम रूप विवर्जिता ।\\
ह्रीं कारी ह्री मती हृद्या हेयो पादेय वर्जिता ॥70॥\\
\\
राज राजार् चिता राज्ञी रम्या राजीव लोचना ।\\
रञ्जनी रमणी रस्या रणत् किङ्किणि मेखला ॥71॥\\
\\
रमा राकेन्दु वदना रति रूपा रतिप् प्रिया ।\\
रक्षा करी राक्ष सघ्नी रामा रमण लम्पटा ॥72॥\\
\\
काम्या काम कला रूपा कदम्ब कुसुमप् प्रिया ।\\
कल्याणी जगती कन्दा करुणा रस सागरा ॥73॥\\
\\
कला वती कला लापा कान्ता कादम्बरीप् प्रिया ।\\
वरदा वाम नयना वारुणी मद विह् वला ॥74॥\\
\\
विश्वा धिका वेद वेद्या विन्ध् याचल निवा सिनी ।\\
विधा त्री वेद जननी विष्णु माया विला सिनी ॥75॥\\
\\
क्षेत्रस् स्वरूपा क्षेत्रे शी क्षेत्रक् क्षेत्रज्ञ पालिनी ।\\
क्षय वृद्धि विनिर् मुक्ता क्षेत्र पाल समर् चिता ॥76॥\\
\\
विजया विमला वन्द्या वन्दारु जन वत्सला ।\\
वाग्वा दिनी वाम केशी वह्नि मण्डल वासिनी ॥77॥\\
\\
भक्ति मत् कल्प लतिका पशु पाश विमो चिनी ।\\
संहृता शेष पा षण्डा सदाचार प्रवर् तिका ॥78॥ \\
{\small\eng{or} पाखण्डा}\\
\\
तापत् त्रयाग्नि सन् तप्त समाह् ह्लादन चन्द्रिका ।\\
तरुणी ताप साराध्या तनु मध्या तमोऽ पहा ॥79॥\\
\\
चितिस् तत् पद लक्ष् यार्था चिदेक रस रूपिणी ।\\
स्वात् मानन्द लवी भूतब् ब्रह्माद् द्यानन्द सन्ततिः ॥80॥\\
\subsection{\eng{Shloka 81-100}}
परा प्रत् यक् चिती रूपा पश् यन्ती पर देवता ।\\
मध्यमा वैखरी रूपा भक्त मानस हंसिका ॥81॥\\
\\
कामेश्वरप् प्राण नाडी कृतज्ञा काम पूजिता ।\\
श‍ृङ्गार रस सम्पूर्णा जया जालन् धरस् स्थिता ॥82॥\\
\\
ओड्या णपीठ निलया बिन्दु मण्डल वासिनी ।\\
रहो यागक् क्रमा राध्या रहस् तर्पण तर्पिता ॥83॥\\
\\
सद्यः प्रसा दिनी विश्व साक्षिणी साक्षि वर्जिता ।\\
षडङ्ग देव ता युक्ता षाड् गुण्य परि पूरिता ॥84॥\\
\\
नित्यक् क्लिन्ना निरुपमा निर्वाण सुख दायिनी ।\\
नित्या षोड शिका रूपा श्री कण् ठार्ध शरी रिणी ॥85॥\\
\\
प्रभा वती प्रभा रूपा प्रसिद्धा परमेश्वरी ।\\
मूलप् प्रकृति रव् यक्ता व्यक्ता व्यक्तस् स्व रूपिणी ॥86॥\\
\\
व्या पिनी विवि धा कारा विद्या विद्यास् स्व रूपिणी ।\\
महा कामेश नयन कुमुदाह् ह्लाद कौमुदी ॥87॥\\
\\
भक्त हार्द तमो भेद भानु मद् भानु सन्ततिः ।\\
शिव दूती शिवा राध्या शिव मूर्तिः शिवङ्करी ॥88॥\\
\\
शिवप् प्रिया शिव परा शिष् टेष्टा शिष्ट पूजिता ।\\
अप्रमेया स्वप् प्रकाशा मनो वाचा मगो चरा ॥89॥\\
\\
चिच् छक् तिश् चेतना रूपा जड शक्तिर् जडात् मिका ।\\
गायत्री व्या हृतिः सन्ध्या द्विज वृन्द निषे विता ॥90॥\\
\\
तत् त्वा सना तत् त्वमयी पञ्च कोशान् तरस् स्थिता ।\\
निः सीम महिमा नित्य यौ वना मद शालिनी ॥91॥\\
\\
मद घूर्णि तरक् ताक्षी मद पाटल गण्डभूः ।\\
चन्दनद् द्रव दिग् धाङ्गी चाम् पेय कुसुमप् प्रिया ॥92॥\\
\\
कुशला कोम लाकारा कुरु कुल्ला कुलेश्वरी ।\\
कुल कुण्डा लया कौल मार्ग तत् पर सेविता ॥93॥\\
\\
कुमार गण ना थाम्बा तुष्टिः पुष्टिर् मतिर् धृतिः ।\\
शान्तिः स्वस्ति मती कान्तिर् नन्दिनी विघ्न नाशिनी ॥94॥\\
\\
तेजो वती त्रिन यना लोलाक्षी काम रूपिणी ।\\
मालिनी हंसिनी माता मल याचल वासिनी ॥95॥\\
\\
सुमुखी नलिनी सुभ् रूश् शो भना सुर नायिका ।\\
काल कण्ठी कान्ति मती क्षोभिणी सूक्ष्म रूपिणी ॥96॥\\
\\
वज्रेश् वरी वाम देवी वयोऽ वस्था विवर्जिता ।\\
सिद् धेश्वरी सिद्ध विद्या सिद्ध माता यशस्विनी ॥97॥\\
\\
विशुद्ध चक्र निलयाऽऽ रक्त वर्णात् त्रिलो चना ।\\
खट् वाङ् गादिप् प्रह रणा वद नैक समन्विता ॥98॥\\
\\
पाय सान्नप् प्रिया त्वक् स्था पशु लोक भयङ्करी ।\\
अमृ तादि महा शक्ति संवृता डाकि नीश्वरी ॥99॥\\
\\
अना हताब्ज निलयाश् श्यामा भावद नद् वया ।\\
दं ष्ट्रोज् ज्वलाऽ क्षमा लादि धरा रुधिर संस्थिता ॥100॥\\
\subsection{\eng{Shloka 101-120}}
काल रा त्र्यादि शक्त् यौघ वृतास् स्निग् धौ दन प्रिया ।\\
महा वीरेन्द्र वरदा रा किण् यम्बास् स्व रूपिणी ॥101॥\\
\\
मणि पूराब्ज निलया वद नत्रय सय्युता ।\\
वज्रा दिकायु धोपेता डामर्या दिभि रावृता ॥102॥\\
\\
रक्त वर्णा मांस निष्ठा गुडान्नप् प्रीत मानसा ।\\
समस्त भक्त सुखदा लाकिन् यम्बास् स्व रूपिणी ॥103॥\\
\\
स्वा धिष्ठा नाम्बुज गता चतुर् वक्त्र मनोहरा ।\\
शूलाद् द्यायुध सम्पन्ना पीत वर्णाऽ तिगर् विता ॥104॥\\
\\
मेदो निष्ठा मधुप् प्रीता बन्धिन् यादि समन्विता ।\\
दध् यन्ना सक्त हृदया काकिनी रूप धारिणी ॥105॥\\
\\
मूला धाराम् बुजा रूढा पञ्च वक्त्रास्ऽ स्थिसं स्थिता ।\\
अङ्कु शादिप् प्रह रणा वर दादि निषे विता ॥106॥\\
\\
मुद् गौ दना सक्त चित्ता साकिन् यम्बास् स्व रूपिणी ।\\
आज्ञा चक् राब्ज निलया शुक्ल वर्णा षडा नना ॥107॥\\
\\
मज्जा संस्था हंस वती मुख्य शक्ति समन्विता ।\\
हरि द्रान् नैक रसिका हाकिनी रूप धारिणी ॥108॥\\
\\
सहस्र दल पद् मस्था सर्व वर्णोप शोभिता ।\\
सर्वा युध धरा शुक्ल संस्थिता सर्व तोमुखी ॥109॥\\
\\
सर्वौ दनप् प्रीत चित्ता याकिन् यम्बास् स्वरूपिणी ।\\
स्वाहा स्वधाऽ मतिर् मेधा श्रुतिः स्मृति रनुत्तमा ॥110॥\\
\\
पुण्य कीर्तिः पुण्य लभ्या {\small\eng{labya}} पुण्यश् श्रवण कीर्तना ।\\
पुलो मजार् चिता बन्ध मोचनी बन्धु रालका ॥111॥\\
{\small\eng{or} बर्बरालका}\\
\\
विमर्श रूपिणी विद्या विय दादि जगत् प्रसूः ।\\
सर्वव् व्याधिप् प्रश मनी सर्व मृत्यु निवा रिणी ॥112॥\\
\\
अग्र गण्याऽ चिन्त्य रूपा कलि कल्मष नाशिनी ।\\
कात् यायनी काल हन्त्री कमलाक्ष निषेविता ॥113॥\\
\\
ताम्बूल पूरित मुखी दाडिमी कुसुमप् प्रभा ।\\
मृगाक्षी मोहिनी मुख्या मृडानी मित्र रूपिणी ॥114॥\\
\\
नित्य तृप्ता भक्त निधिर् नियन्त्री निखि लेश्वरी ।\\
मैत् त्र्यादि वास नालभ्या महाप् प्रलय साक्षिणी ॥115॥\\
\\
परा शक्तिः परा निष्ठा प्रज्ञा नघन रूपिणी ।\\
माध्वी पाना लसा मत्ता मातृका वर्ण रूपिणी ॥116॥\\
\\
महा कैलास निलया मृणाल मृदु दोर् लता ।\\
मह नीया दया मूर्तिर् महा साम्राज्य शालिनी ॥117॥\\
\\
आत्म विद्या महा विद्या श्रीविद्या काम सेविता ।\\
श्री षोड शाक् षरी विद्या त्रिकूटा काम कोटिका ॥118॥\\
\\
कटाक्ष किङ्करी भूत कमला कोटि सेविता ।\\
शिरः स्थिता चन्द्र निभा भालस् थेन्द्र धनुः प्रभा ॥119॥\\
\\
हृद यस्था रवि प्रख्या त्रिको णान्तर दीपिका ।\\
दाक्षा यणी दैत्य हन्त्री दक्ष यज्ञ विना शिनी ॥120॥\\
\subsection{\eng{Shloka 121-140}}
दरान् दोलित दीर् घाक्षी दर हासोज् ज्वलन् मुखी ।\\
गुरु मूर्तिर् गुण निधिर् गोमाता गुह जन्मभूः ॥121॥\\
\\
देवेशी दण्ड नी तिस्था दह रा काश रूपिणी ।\\
प्रति पन् मुख्य रा कान्त तिथि मण्डल पूजिता ॥122॥\\
\\
कलात् मिका कला नाथा काव्या लाप विनोदिनी ।\\
{\small\eng{or} विमोदिनी}\\
सचा मर रमा वाणी सव्य दक्षिण सेविता ॥123॥\\
\\
आदि शक्तिर् अमे याऽऽत्मा परमा पाव नाकृतिः ।\\
अनेक कोटि ब्रह्माण्ड जननी दिव्य विग्रहा ॥124॥\\
\\
क्लीं कारी केवला गुह्या कैवल्य पद दायिनी ।\\
त्रिपुरा त्रिज गद् वन्द्या त्रिमूर् तिस् त्रिद शेश्वरी ॥125॥\\
\\
त्र्यक्ष री दिव्य गन्धाढ् ढ्या सिन्दूर तिल काञ्चिता ।\\
उमा शैलेन्द्र तनया गौरी गन्धर्व सेविता ॥126॥\\
\\
विश्व गर्भा स्वर्ण गर्भाऽ वरदा वाग धीश्वरी ।\\
ध्यान गम्याऽ परिच् छेद्या ज्ञानदा ज्ञान विग्रहा ॥127॥\\
\\
सर्व वेदान्त संवेद्या सत्या नन्दस् स्वरूपिणी ।\\
लोपा मुद्रार् चिता लीला कॢप्त ब्रह्माण्ड मण्डला ॥128॥\\
\\
अदृश्या दृश्य रहिता विज्ञात्री वेद्य वर्जिता ।\\
योगिनी योगदा योग्या योगा नन्दा युगन् धरा ॥129॥\\
\\
इच्छा शक्तिग् ज्ञान शक्तिक् क्रिया शक्तिस् स्वरूपिणी ।\\
सर्वा धारा सुप्र तिष्ठा सद सद् रूप धारिणी ॥130॥\\
\\
अष्ट मूर्ति रजा जैत्री लोक यात्रा विधा यिनी । \\
{\small\eng{or} अजाजेत्री}\\
एका किनी भूम रूपा निर्द् वैताद् द्वैत वर्जिता ॥131॥\\
\\
अन्नदा वसुदा वृद् धा ब्रह्मात् मैक् यस् स्वरूपिणी ।\\
बृहती ब्राह्मणी ब्राह्मी ब्रह्मा नन्दा बलिप् प्रिया ॥132॥\\
\\
भाषा रूपा बृहत् सेना भावा भाव विवर्जिता ।\\
सुखा राध्या शुभ करी शोभना सुलभा गतिः ॥133॥\\
\\
राज राजेश्वरी राज्य दायिनी राज्य वल्लभा ।\\
राजत् कृपा राज पीठ निवे शित निजाश् श्रिता ॥134॥\\
\\
राज्य लक्ष्मीः कोश नाथा चतु रङ्ग बलेश्वरी ।\\
साम्राज्य दायिनी सत्य सन्धा सागर मेखला ॥135॥\\
\\
दीक्षिता दैत्य शमनी सर्व लोक वशङ्करी ।\\
सर्वार्थ दात्री सावित्री सच्चि दानन्द रूपिणी ॥136॥\\
\\
देश काला परिच् छिन्ना सर्वगा सर्व मोहिनी ।\\
सरस्वती शास्त्र मयी गुहाम्बा गुह्य रूपिणी ॥137॥\\
\\
सर्वो पाधि विनिर् मुक्ता सदा शिव पतिव् व्रता ।\\
सम्प्रदा येश्वरी साध्वी गुरु मण्डल रूपिणी ॥138॥\\
\\
कुलोत् तीर्णा भगा राध्या माया मधु मती मही ।\\
गणाम्बा गुह्य काराध्या कोम लाङ्गी गुरुप् प्रिया ॥139॥\\
\\
स्व तन्त्रा सर्व तन्त्रेशी दक्षिणा मूर्ति रूपिणी ।\\
सन कादि समा राध्या शिवग् ज्ञानप् प्रदायिनी ॥140॥\\
\subsection{\eng{Shloka 141-160}}
चित् कलाऽऽ नन्द कलिका प्रेम रूपा प्रियङ्करी ।\\
नाम पारायणप् प्रीता नन्दि विद्या नटेश्वरी ॥141॥\\
\\
मिथ्या जगद धिष् ठाना मुक्ति दा मुक्ति रूपिणी ।\\
लास्यप् प्रिया लयकरी लज्जा रम्भादि वन्दिता ॥142॥\\
\\
भव दाव सुधा वृष्टिः पापा रण्य दवा नला ।\\
दौर्भाग्य तूल वातूला जराध् वान्त रविप् प्रभा ॥143॥\\
\\
भाग् याब्धि चन्द्रिका भक्त चित्त केकि घना घना ।\\
रोग पर्वत दम् भोलिर् मृत्यु दारु कुठा रिका ॥144॥\\
\\
महेश्वरी महा काली महा ग्रासा महा शना ।\\
अपर्णा चण्डिका चण्ड मुण्डा सुर निषू दिनी ॥145॥\\
\\
क्षरा क्षरात् मिका सर्व लोकेशी विश्व धारिणी ।\\
त्रिवर्ग दात्री सुभगा त्र्यम्बका त्रिगु णात्मिका ॥146॥\\
\\
स्वर्गा पवर् गदा शुद्धा जपा पुष्प निभा कृतिः ।\\
ओजो वतीद् द्युति धरा यज्ञ रूपा प्रियव् व्रता ॥147॥\\
\\
दुरा राध्या दुरा धर्षा पाटली कुसुमप् प्रिया ।\\
महती मेरु निलया मन्दार कुसुमप् प्रिया ॥148॥\\
\\
वीरा राध्या विराड् ड्रूपा विरजा विश्वतो मुखी ।\\
प्रत्यग् ग्रूपा परा काशा प्राणदा प्राण रूपिणी ॥149॥\\
\\
मार्ताण्ड भैरवा राध्या मन्त्रिणीन् यस्त राज्यधूः । \\
{\small\eng{or} मार्तण्ड}\\
त्रिपु रेशी जयत् सेना निस्त्रै गुण्या परा परा ॥150॥\\
\\
सत्यग् ज्ञाना नन्द रूपा साम रस्य परायणा ।\\
कपर्दिनी कला माला काम धुक् काम रूपिणी ॥151॥\\
\\
कला निधिः काव्य कला रसज्ञा रस शेवधिः ।\\
पुष्टा पुरा तना पूज्या पुष्करा पुष्क रेक्षणा ॥152॥\\
\\
परं ज्योतिः परं धाम पर माणुः परात् परा ।\\
पाश हस्ता पाश हन्त्री पर मन्त्र विभे दिनी ॥153॥\\
\\
मूर्ताऽ मूर्ताऽ नित्य तृप्ता मुनि मानस हंसिका ।\\
सत्यव् व्रता सत्य रूपा सर्वान् तर्या मिनी सती ॥154॥\\
\\
ब्रह्माणी ब्रह्म जननी बहु रूपा बुधार् चिता ।\\
प्रस वित्री प्रचण् डाऽऽज्ञा प्रतिष्ठा प्रकटा कृतिः ॥155॥\\
\\
प्राणेश्वरी प्राण दात्री पञ्चा शत् पीठ रूपिणी ।\\
विश‍ृङ्खला विविक् तस्था वीर माता वियत् प्रसूः ॥156॥\\
\\
मुकुन्दा मुक्ति निलया मूल विग्रह रूपिणी ।\\
भावज्ञा भवरो गघ्नी भव चक्र प्रवर् तिनी ॥157॥\\
\\
छन्दः सारा शास्त्र सारा मन्त्र सारा तलो दरी ।\\
उदार कीर्ति रुद् दाम वैभवा वर्ण रूपिणी ॥158॥\\
\\
जन्म मृत्यु जरा तप्त जन विश्रान्ति दायिनी ।\\
सर्वो पनिष दुद् घुष्टा शान्त् यतीत कलात् मिका ॥159॥\\
\\
गम्भीरा गगनान् तस्था गर्विता गान लोलुपा ।\\
कल्पना रहिता काष्ठाऽ कान्ता कान्तार्ध विग्रहा ॥160॥\\
\subsection{\eng{Shloka 160-180}}
कार्य कारण निर् मुक्ता काम केलि तरङ्गिता ।\\
कनत् कन कता टङ्का लीला विग्रह धारिणी ॥161॥\\
\\
अजाक् क्षयवि निर् मुक्ता मुग्धाक् क्षिप्रप् प्रसा दिनी ।\\
अन्तर् मुख समा राध्या बहिर् मुख सुदुर् लभा ॥162॥\\
\\
त्रयी त्रिवर्ग निलया त्रिस्था त्रिपुर मालिनी ।\\
निरा मया निरा लम्बा स्वात् मारामा सुधासृतिः ॥163॥ \\
{\small\eng{or} सुधास्रुतिः}\\
\\
संसार पङ्क निर् मग्न समुद् धरण पण्डिता ।\\
यज्ञप् प्रिया यज्ञ कर्त्री यज मान स्वरूपिणी ॥164॥\\
\\
धर्मा धारा धनाध् यक्षा धन धान्य विवर्धिनी ।\\
विप्रप् प्रिया विप्र रूपा विश्वभ् भ्रमण कारिणी ॥165॥\\
\\
विश्वग् ग्रासा विद्रु माभा {\small\eng{maba}} वैष्णवी विष्णु रूपिणी ।\\
अयो निर् योनि निलया कूटस्था कुल रूपिणी ॥166॥\\
\\
\\
वीर गोष्ठी प्रिया वीरा नैष् कर्म्या नाद रूपिणी ।\\
विज्ञान कलना कल्या विदग्धा बैन्द वासना ॥167॥\\
\\
तत् त्वा धिका तत् त्व मयी तत् त्व मर्थस् स्वरूपिणी ।\\
साम गानप् प्रिया सौम्या सदा शिव कुटुम् बिनी ॥168॥ \\
{\small\eng{or} सोम्या}\\
\\
सव्या पसव्य मार् गस्था सर्वा पद्वि निवारिणी ।\\
स्वस्थास् स्व-भाव मधुरा धीरा धीर समर्चिता ॥169॥\\
\\
चैतन् यार्घ्य समा राध्या चैतन्य कुसुमप् प्रिया ।\\
सदो दिता सदा तुष्टा तरुणा दित्य पाटला ॥170॥\\
\\
दक्षिणा दक्षिणा राध्या दरस् मेर मुखां बुजा ।\\
कौलिनी केवलाऽ नर्घ्य कैवल्य पद दायिनी ॥171॥\\
\\
स्तोत्रप् प्रिया स्तुति मती श्रुति संस्तुत वैभवा ।\\
मनस् विनी मान वती महेशी मङ्गला कृतिः ॥172॥\\
\\
विश्व माता जगद् धात्री विशा लाक्षी विरा गिणी ।\\
प्रगल्भा परमो दारा परा मोदा मनो मयी ॥173॥\\
\\
व्योम केशी विमा नस्था वज्रिणी वाम केश्वरी ।\\
पञ्च यज्ञप् प्रिया पञ्चप् प्रेत मञ्चा धिशा यिनी ॥174॥\\
\\
पञ्चमी पञ्च भूतेशी पञ्च संख्यो पचा रिणी ।\\
शाश्वती शाश्व तैश्वर्या शर्मदा शम्भु मोहिनी ॥175॥\\
\\
धरा धर सुता धन्या धर्मिणी धर्म वर्धिनी ।\\
लोका तीता गुणा तीता सर्वा तीता शमात् मिका ॥176॥\\
\\
बन् धूक कुसुमप् प्रख्या बाला लीला विनोदिनी ।\\
सुमङ्गली सुख करी सुवे षाढ्या सुवा सिनी ॥177॥\\
\\
सुवा सिन्यर् चनप् प्रीताऽऽ शोभना शुद्ध मानसा ।\\
बिन्दु तर्पण सन्तुष्टा पूर्वजा त्रिपुराम् बिका ॥178॥\\
\\
दश मुद्रा समा राध्या त्रिपु राश्री वशङ्करी ।\\
ज्ञान मुद्राग् ज्ञान गम्याग् ज्ञान ज्ञेयस् स्वरूपिणी ॥179॥\\
\\
योनि मुद्रा त्रि खण्डेशी त्रिगु णाम्बा त्रिको णगा ।\\
अन घाऽद् भुत चारित्रा वाञ्छि तार्थप् प्रदा यिनी ॥180॥\\
\subsection{\eng{Shloka 181-End}}
अभ्यासा तिशयग् ज्ञाता षडध् वातीत रूपिणी ।\\
अव्याज करुणा मूर्तिर् अज्ञानध् ध्वान्त दीपिका ॥181॥\\
\\
आबाल गोप विदिता सर्वा नुल् लङ्घ्य शासना ।\\
श्री चक्र राज निलया श्री मत् त्रिपुर सुन्दरी ॥182॥\\
\\
श्री शिवा शिव शक्त् यैक्य रूपिणी ललितां बिका ।\\
एवं श्री ललिता देव्या नां नां साहस् रकं जगुः ॥\\
\\
॥ इति श्रीब्रह्माण्डपुराणे उत्तरखण्डे श्रीहयग्रीवागस्त्यसंवादे\\
श्रीललिता सहस्रनाम स्तोत्र कथनं सम्पूर्णम् ॥\\
