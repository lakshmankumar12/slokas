\section{श्री विष्णु सहस्र नाम स्तोत्रम्}
\\
ॐ शुक्लाम्बरधरं विष्णुं शशिवर्णं चतुर्भुजम् ।\\
प्रसन्नवदनं ध्यायेत् सर्वविघ्नोपशान्तये ॥ 1 ॥\\
\\
नारायणं नमस्कृत्य नरं चैव नरोत्तमम्। \\
देवीं सरस्वतीं व्यासं ततो जयमुदीरयेत्।\\
\subsection{पूर्व पीठिका}
\\
व्यासं वसिष्ठ नप्तारं शक्तेः पौत्रमकल्मषं ।\\
पराशरात्मजं वन्दे शुकतातं तपोनिधिं ॥ 3 ॥\\
\\
व्यासाय विष्णु रूपाय व्यासरूपाय विष्णवे ।\\
नमो वै ब्रह्मनिधये वासिष्ठाय नमो नमः ॥ 4 ॥\\
\\
अविकाराय शुद्धाय नित्याय परमात्मने ।\\
सदैक रूप रूपाय विष्णवे सर्वजिष्णवे ॥ 5 ॥\\
\\
यस्य स्मरणमात्रेण जन्मसंसारबन्धनात् ।\\
विमुच्यते नमस्तस्मै विष्णवे प्रभविष्णवे ॥ 6 ॥\\
\\
ॐ नमो विष्णवे प्रभविष्णवे ।\\
\\
श्री वैशम्पायन उवाच\\
श्रुत्वा धर्मा नशेषेण पावनानि च \\
युधिष्ठिरः शान्तनवं पुनरेवाभ्य भाषत ॥ 7 ॥\\
\\
श्री युधिष्ठिर उवाच\\
किमेकं दैवतं लोके किं वाऽप्येकं परायणं\\
स्तुवन्तः कं कमर्चन्तः प्राप्नुयुर्मानवाः शुभम् ॥ 8 ॥\\
\\
को धर्मः सर्वधर्माणां भवतः परमो मतः ।\\
किं जपन्मुच्यते जन्तुर्जन्मसंसार बन्धनात् ॥ 9 ॥\\
\\
श्री भीष्म उवाच\\
जगत्प्रभुं देवदेव मनन्तं पुरुषोत्तमं ।\\
स्तुवन्नाम सहस्रेण पुरुषः सततोत्थितः ॥ 10 ॥\\
\\
तमेव चार्चयन्नित्यं भक्त्या पुरुषमव्ययं ।\\
ध्यायन् स्तुवन्नमस्यंश्च यजमानस्तमेव च ॥ 11 ॥\\
\\
अनादि निधनं विष्णुं सर्वलोक महेश्वरं ।\\
लोकाध्यक्षं स्तुवन्नित्यं सर्व दुःखातिगो \\
भवेत् ॥ 12 ॥\\
\\
ब्रह्मण्यं सर्व धर्मज्ञं लोकानां कीर्ति वर्धनं ।\\
लोकनाथं महद्भूतं सर्वभूत भवोद्भवम्॥ 13 ॥\\
\\
एष मे सर्व धर्माणां धर्मोऽधिक तमोमतः ।\\
यद्भक्त्या पुण्डरीकाक्षं स्तवैरर्चेन्नरः सदा ॥ 14 ॥\\
\\
परमं यो महत्तेजः परमं यो महत्तपः ।\\
परमं यो महद्ब्रह्म परमं यः परायणम् । 15 ॥\\
\\
पवित्राणां पवित्रं यो मङ्गलानां च मङ्गलं ।\\
दैवतं देवतानां च भूतानां योऽव्ययः पिता ॥ 16 ॥\\
\\
यतः सर्वाणि भूतानि भवन्त्यादि युगागमे ।\\
यस्मिंश्च प्रलयं यान्ति पुनरेव युगक्षये ॥ 17 ॥\\
\\
तस्य लोक प्रधानस्य जगन्नाथस्य भूपते ।\\
विष्णोर्नाम सहस्रं मे श्रुणु पाप भयापहम् ॥ 18 ॥\\
\\
यानि नामानि गौणानि विख्यातानि महात्मनः ।\\
ऋषिभिः परिगीतानि तानि वक्ष्यामि भूतये ॥ 19 ॥\\
\\
ऋषिर्नाम्नां सहस्रस्य वेदव्यासो महामुनिः ॥\\
छन्दोऽनुष्टुप् तथा देवो भगवान् देवकीसुतः ॥ 20 ॥\\
\\
अमृतां शूद्भवो बीजं शक्तिर्देवकिनन्दनः ।\\
त्रिसामा हृदयं तस्य शान्त्यर्थे विनियुज्यते ॥ 21 ॥\\
\\
विष्णुं जिष्णुं महाविष्णुं प्रभविष्णुं महेश्वरं ॥\\
अनेकरूप दैत्यान्तं नमामि पुरुषोत्तमम् ॥ 22 ॥\\
\subsection{पूर्वन्यासः}
अस्य श्री विष्णोर्दिव्य सहस्रनाम स्तोत्र महामन्त्रस्य ॥\\
श्री वेदव्यासो भगवान् ऋषिः ।\\
अनुष्टुप् छन्दः ।\\
श्रीमहाविष्णुः परमात्मा श्रीमन्नारायणो देवता ।\\
अमृतांशूद्भवो भानुरिति बीजं ।\\
देवकीनन्दनः स्रष्टेति शक्तिः ।\\
उद्भवः, क्षोभणो देव इति परमोमन्त्रः ।\\
शङ्खभृन्नन्दकी चक्रीति कीलकम् ।\\
शार्ङ्गधन्वा गदाधर इत्यस्त्रम् ।\\
रथाङ्गपाणि रक्षोभ्य इति नेत्रं ।\\
त्रिसामासामगः सामेति कवचम् ।\\
आनन्दं परब्रह्मेति योनिः ।\\
ऋतुस्सुदर्शनः काल इति दिग्बन्धः ॥\\
श्रीविश्वरूप इति ध्यानं ।\\
श्री महाविष्णु प्रीत्यर्थे सहस्रनाम जपे \\
         पारायणे विनियोगः ।\\
\subsection{ध्यानम्}
क्षीरो धन्वत् प्रदेशे शुचिमणि \\
विलसत्सैकते मौक्तिकानां\\
माला क्लुप्ता सनस्थः स्फटिक\\
मणि निभैर् मौक्तिकैर् मण्डिताङ्गः ।\\
शुभ्रै रभ्रै रदभ्रै रुपरि विरचितैर् मुक्त पीयूष वर्षैः\\
आनन्दी नः पुनीया दरि नलिन गदा \\
शङ्ख पाणिर् मुकुन्दः ॥ 1 ॥\\
\\
भूः पादौ यस्य नाभिर् विय दसुर निलश्चन्द्र \\
सूर्यौ च नेत्रे\\
कर्णा वाशाः शिरो द्यौर् मुखमपि दहनो \\
यस्य वास्तेय मब्धिः ।\\
अन्तःस्थं यस्य विश्वं सुर नर खगगो भोगि गन्धर्व दैत्यैः\\
चित्रं रं रम्यते तं त्रिभु वन वपुशं \\
विष्णु मीशं नमामि ॥ 2 ॥\\
\\
ॐ नमो भगवते वासुदेवाय !\\
\\
शान्ताकारं भुजगशयनं पद्मनाभं सुरेशं\\
विश्वाधारं गगनसदृशं मेघवर्णं शुभाङ्गम् ।\\
लक्ष्मीकान्तं कमलनयनं योगिहृर्ध्यानगम्यम्\\
वन्दे विष्णुं भवभयहरं सर्वलोकैकनाथम् ॥ 3 ॥\\
\\
मेघश्यामं पीतकौशेयवासं\\
श्रीवत्साकं कौस्तुभोद्भासिताङ्गम् ।\\
पुण्योपेतं पुण्डरीकायताक्षं\\
विष्णुं वन्दे सर्वलोकैकनाथम् ॥ 4 ॥\\
\\
नमः समस्त भूतानां आदि भूताय भूभृते ।\\
अनेकरूप रूपाय विष्णवे प्रभविष्णवे ॥ 5॥\\
\\
सशङ्खचक्रं सकिरीटकुण्डलं\\
सपीतवस्त्रं सरसीरुहेक्षणं ।\\
सहार वक्षःस्थल शोभि कौस्तुभं\\
नमामि विष्णुं शिरसा चतुर्भुजम् । 6॥\\
\\
छायायां पारिजातस्य हेमसिंहासनोपरि\\
आसीनमम्बुदश्याममायताक्षमलङ्कृतम् ॥ 7 ॥\\
\\
चन्द्राननं चतुर्बाहुं श्रीवत्साङ्कित वक्षसम्\\
रुक्मिणी सत्यभामाभ्यां सहितं कृष्णमाश्रये ॥ 8 ॥\\
\subsection{स्तोत्रम्}
\subsubsection{\eng{Sloka 1-20}}
विश्वं विष्णुर्वषट्कारो भूतभव्यभवत्प्रभुः ।\\
भूतकृद्भूतभृद्भावो भूतात्मा भूतभावनः ॥ 1 ॥\\
\\
पूतात्मा परमात्मा च मुक्तानां परमागतिः ।\\
अव्ययः पुरुषः साक्षी क्षेत्रज्ञोऽक्षर एव च ॥ 2 ॥\\
\\
योगो योगविदां नेता प्रधान पुरुषेश्वरः ।\\
नारसिंहवपुः श्रीमान् केशवः पुरुषोत्तमः ॥ 3 ॥\\
\\
सर्वः शर्वः शिवः स्थाणुर्भूतादिर्निधिरव्ययः ।\\
सम्भवो भावनो भर्ता प्रभवः प्रभुरीश्वरः ॥ 4 ॥\\
\\
स्वयम्भूः शम्भुरादित्यः पुष्कराक्षो महास्वनः ।\\
अनादि निधनो धाता विधाता धातुरुत्तमः ॥ 5 ॥\\
\\
अप्रमेयो हृषीकेशः पद्मनाभोऽमरप्रभुः ।\\
विश्वकर्मा मनुस्त्वष्टा स्थविष्ठः स्थविरो ध्रुवः ॥ 6 ॥\\
\\
अग्राह्यः शाश्वतः कृष्णो लोहिताक्षः प्रतर्दनः ।\\
प्रभूतस्त्रिक कुब्धाम पवित्रं मङ्गलं परम् ॥ 7 ॥\\
\\
ईशानः प्राणदः प्राणो ज्येष्ठः श्रेष्ठः प्रजापतिः ।\\
हिरण्यगर्भो भूगर्भो माधवो मधुसूदनः ॥ 8 ॥\\
\\
ईश्वरो विक्रमीधन्वी मेधावी विक्रमः क्रमः ।\\
अनुत्तमो दुराधर्षः कृतज्ञः कृतिरात्मवान्॥ 9 ॥\\
\\
सुरेशः शरणं शर्म विश्वरेताः प्रजाभवः ।\\
अहस्संवत्सरो व्यालः प्रत्ययः सर्वदर्शनः ॥ 10 ॥\\
\\
अजस्सर्वेश्वरः सिद्धः सिद्धिः सर्वा दिरच्युतः ।\\
वृषाक पिर मेयात्मा सर्व योग विनिस्सृतः ॥ 11 ॥\\
\\
वसुर्वसुमनाः सत्यः समात्मा सम्मितस्समः ।\\
अमोघः पुण्डरीकाक्षो वृषकर्मा वृषाकृतिः ॥ 12 ॥\\
\\
रुद्रो बहुशिरा बभ्रुर्विश्वयोनिः शुचिश्रवाः ।\\
अमृतः शाश्वतस्थाणुर्वरारोहो महातपाः ॥ 13 ॥\\
\\
सर्वगः सर्व विद्भानुर्विष्वक्सेनो जनार्दनः ।\\
वेदो वेदविदव्यङ्गो वेदाङ्गो वेदवित्कविः ॥ 14 ॥\\
\\
लोकाध्यक्षः सुराध्यक्षो धर्माध्यक्षः कृताकृतः ।\\
चतुरात्मा चतुर्व्यूहश्चतुर्दं ष्ट्रश्चतुर्भुजः ॥ 15 ॥\\
		\\
भ्राजिष्णुर्भोजनं भोक्ता सहिष्नुर्जगदादिजः ।\\
अनघो विजयो जेता विश्वयोनिः पुनर्वसुः ॥ 16 ॥\\
\\
उपेन्द्रो वामनः प्रांशुरमोघः शुचिरूर्जितः ।\\
अतीन्द्रः सङ्ग्रहः सर्गो धृतात्मा नियमो यमः ॥ 17 ॥\\
\\
वेद्यो वैद्यः सदायोगी वीरहा माधवो मधुः ।\\
अतीन्द्रियो महामायो महोत्साहो महाबलः ॥ 18 ॥\\
\\
महाबुद्धिर्महावीर्यो महाशक्तिर्महाद्युतिः ।\\
अनिर्देश्यवपुः श्रीमानमेयात्मा महाद्रिधृक् ॥ 19 ॥\\
\\
महेश्वासो महीभर्ता श्रीनिवासः सताङ्गतिः ।\\
अनिरुद्धः सुरानन्दो गोविन्दो गोविदां पतिः ॥ 20 ॥\\
\subsubsection{\eng{Sloka 21-40}}
\\
मरीचिर्दमनो हंसः सुपर्णो भुजगोत्तमः ।\\
हिरण्यनाभः सुतपाः पद्मनाभः प्रजापतिः ॥ 21 ॥\\
\\
अमृत्युः सर्वदृक् सिंहः सन्धाता सन्धिमान् स्थिरः ।\\
अजो दुर्मर्षणः शास्ता विश्रुतात्मा सुरारिहा ॥ 22 ॥\\
\\
गुरुर्गुरुतमो धाम सत्यः सत्यपराक्रमः ।\\
निमिषोऽनिमिषः स्रग्वी वाचस्पति रुदारधीः ॥ 23 ॥\\
\\
अग्रणीर् ग्रामणीः श्रीमान् न्यायो नेता समीरणः\\
सहस्रमूर्धा विश्वात्मा सहस्राक्षः सहस्रपात् ॥ 24 ॥\\
\\
आवर्तनो निवृत्तात्मा संवृतः सम्प्रमर्दनः ।\\
अहः संवर्तको वह्नि रनिलो धरणीधरः ॥ 25 ॥\\
\\
सुप्रसादः प्रसन्नात्मा विश्वधृग्विश्वभुग्विभुः ।\\
सत्कर्ता सत्कृतः साधुर्जह्नुर्नारायणो नरः ॥ 26 ॥\\
\\
असङ्ख्येयोऽप्रमेयात्मा विशिष्टः शिष्टकृच्छुचिः ।\\
सिद्धार्थः सिद्धसङ्कल्पः सिद्धिदः सिद्धि साधनः ॥ 27 ॥\\
\\
वृषाही वृषभो विष्णुर्वृषपर्वा वृषोदरः ।\\
वर्धनो वर्धमानश्च विविक्तः श्रुतिसागरः ॥ 28 ॥\\
\\
सुभुजो दुर्धरो वाग्मी महेन्द्रो वसुदो वसुः ।\\
नैकरूपो बृहद्रूपः शिपिविष्टः प्रकाशनः ॥ 29 ॥\\
\\
ओजस्तेजोद्युतिधरः प्रकाशात्मा प्रतापनः ।\\
ऋद्दः स्पष्टाक्षरो मन्त्रश्चन्द्रांशुर्भास्करद्युतिः ॥ 30 ॥\\
\\
अमृतां शूद् भवो भानुः शशबिन्दुः सुरेश्वरः ।\\
औषधं जगतः सेतुः सत्य धर्म पराक्रमः ॥ 31 ॥\\
\\
भूत भव्य भवन्नाथः पवनः पावनोऽनलः ।\\
कामहा काम कृत्कान्तः कामः काम प्रदः प्रभुः ॥32॥\\
\\
युगादि कृद् युगा वर्तो नैक मायो महाशनः ।\\
अदृश् यो व्यक्त रूपश्च सहस्र जिद नन् तजित्॥33॥\\
\\
इष्टोऽविशिष्टः शिष् टेष्टः शिखण्डी नहुषो वृषः ।\\
क्रोधहा क्रोध कृत् कर्ता विश्व बाहुर् महीधरः॥34 ॥\\
\\
अच्युतः प्रथि तः प्राणः प्राणदो वास वानुजः ।\\
अपां निधिर धिष्ठान मप्र मत्तः प्रतिष्ठितः ॥ 35 ॥\\
\\
स्कन्दः स्कन्द धरो धुर्यो वरदो वायु वाहनः ।\\
वासु देवो बृहद् भानु रादिदेवः पुरन्धरः ॥ 36 ॥\\
\\
अशो कस् तार णस्तारः शूरः शौरिर् जनेश्वरः ।\\
अनुकूलः शता वर्तः पद्मी पद्म निभेक्षणः ॥ 37 ॥\\
\\
पद्म नाभोऽ रविन्दाक्षः पद्म गर्भः शरी रभृत् ।\\
महर्धिर् ऋद् धो वृद् धात्मा महाक्षो गरुडध्वजः॥38॥\\
\\
अतुलः शरभो भीमः सम यज्ञो हविर् हरिः ।\\
सर्व लक्षण लक्षण्यो लक्ष्मी वान् समि तिञ्जयः॥39॥\\
\\
विक्षरो रोहि तो मार्गो हेतुर् दामोदरः सहः ।\\
मही धरो महा भागो वेग वा नमि ताशनः ॥ 40 ॥\\
\subsubsection{\eng{Sloka 41-60}}
उद्भवः, क्षोभणो देवः श्री गर्भः परमेश्वरः ।\\
करणं कारणं कर्ता विकर्ता गहनो गुहः ॥ 41 ॥\\
\\
व्यव सायो व्यवस् थानः संस्थानः स्थान दो ध्रुवः ।\\
परर्द् धिः पर मस्पष्टस् तुष्टः पुष्टः शुभे क्षणः॥42 ॥\\
\\
रामो विरामो विरजो मार्गो नेयो नयोऽनयः ।\\
वीरः शक्ति मतां श्रेष्ठो धर्मो धर्म विदुत्तमः ॥ 43 ॥\\
\\
वैकुण्ठः पुरुषः प्राणः प्राणदः प्रणवः पृथुः ।\\
हिरण्य गर्भः शत्रुघ्नो व्याप्तो वायु रधो क्षजः॥ 44 ॥\\
\\
ऋतुः सुदर्शनः कालः पर मेष्ठी परि ग्रहः ।\\
उग्रः संवत्सरो दक्षो विश्रामो विश्व दक्षिणः ॥ 45 ॥\\
\\
विस्तारः स्थाव रस्थाणुः प्रमाणं बीज मव्ययं ।\\
अर्थोऽनर्थो महा कोशो महा भोगो महा धनः ॥ 46 ॥\\
\\
अनिर् विण्णः स्थविष्ठो भूर् धर्म यूपो महामखः ।\\
नक्षत्रनेमिर् नक्षत्री क्षमः, क्षामः समी हनः ॥ 47 ॥\\
\\
यज्ञ इज्यो महेज् यश्च क्रतुः सत्रं सताङ्गतिः ।\\
सर्व दर्शी विमुक् तात्मा सर्वज्ञो ज्ञान मुत्तमं ॥ 48 ॥\\
\\
सुव्रतः सुमुखः सूक्ष्मः सुघोषः सुखदः सुहृत् ।\\
मनो हरो जित क्रोधो वीर बाहुर् विदारणः ॥ 49 ॥\\
\\
स्वा पनः स्वव शो व्यापी नै कात्मा नैक कर्मकृत्। ।\\
वत्सरो वत्स लो वत्सी रत्न गर्भो धनेश्वरः ॥ 50 ॥\\
\\
धर्म गुब् धर्म कृध् दर्मी सद सत्क्षर मक्षरम्॥\\
अवि ज्ञाता सहस्त्रां शुर् विधाता कृत लक्षणः॥51॥\\
\\
गभस्ति नेमिः सत् त्वस्थः सिंहो भूत महेश्वरः ।\\
आदि देवो महा देवो देवेशो देवभृद् गुरुः ॥ 52 ॥\\
\\
उत्तरो गोप तिर् गोप्ता ज्ञान गम्यः पुरा तनः ।\\
शरीर भूत भृद् भोक्ता कपीन्द्रो भूरि दक्षिणः ॥ 53 ॥\\
\\
सोमपो ऽमृत पःसोमः पुरु जित् पुरु सत्तमः ।\\
विनयो जयः सत्य सन्धो दाशार्हः सात्वतां पतिः॥54 ॥\\
\\
जीवो विनयि ता-साक्षी मुकुन्दो ऽमित विक्रमः ।\\
अम्भो निधिर नन्तात्मा महो दधि शयोन् तकः॥55 ॥\\
\\
अजो महार् हः स्वा भाव्यो जिता मित्रः प्रमो दनः ।\\
आ नन्दो ऽनन्द नोनन्दः सत्य धर्मा त्रिविक्रमः॥ 56 ॥\\
\\
महर्षिः कपि लाचार्यः कृतज्ञो मेदि नीपतिः ।\\
त्रिप दस् त्रिद शाध् यक्षो महा शृङ्गः कृतान् तकृत्॥57॥\\
\\
महावराहो गोविन्दः सुषेणः कनकाङ्गदी ।\\
गुह्यो गभीरो गहनो गुप्तश्चक्र गदाधरः ॥ 58 ॥\\
\\
वेधाः स्वाङ्गोऽजितः कृष्णो दृढः सङ्कर्षणोऽच्युतः।\\
वरुणो वारुणो वृक्षः पुष्कराक्षो महामनाः ॥ 59 ॥\\
\\
भगवान् भगहाऽऽनन्दी वनमाली हलायुधः ।\\
आदित्यो ज्योतिरादित्यः सहिष्णुर्गतिसत्तमः ॥ 60 ॥\\
\subsubsection{\eng{Sloka 61-80}}
सुधन्वा खण्ड परशुर् दारुणो द्रविण प्रदः ।\\
दिवः स्पृक् सर्व दृग् व्यासो वाचस्पति रयो निजः॥61॥\\
\\
त्रिसामा सामगः साम निर्वाणं भेषजं भिषक् ।\\
सन्यास कृच्छ मः शान्तो निष्ठा शान्तिः परायणम्॥62॥\\
\\
शुभाङ्गः शान्ति दः स्रष्टा कुमुदः कुव लेशयः ।\\
गोहितो गोपतिर् गोप्ता वृष भाक्षो वृष प्रियः ॥ 63 ॥\\
\\
अनि वर्ती निवृत्तात्मा सङ् क्षेप्ता क्षेम कृच्छिवः ।\\
श्रीवत् सवक्षाः श्रीवासः श्रीपतिः श्रीमतां वरः॥64 ॥\\
\\
श्रीदः श्रीशः श्रीनिवासः श्रीनिधिः श्री विभा वनः ।\\
श्रीधरः श्रीकरः श्रेयः श्रीमांल् लोक त्रयाश्रयः॥ 65 ॥\\
\\
स्वक्षः स्वङ्गः शता नन्दो नन्दिर् ज्योतिर् गणेश्वरः ।\\
विजितात्मा ऽविधेयात्मा सत् कीर् तिच् छिन्न संशयः\\
\\
उदीर्णः सर्व तश् चक्षु रनीशः शाश्व तस्थिरः ।\\
भूशयो भूषणो भूतिर् विशोकः शोक नाशनः ॥ 67 ॥\\
\\
अर् चिष् मानर् चितः कुम्भो विशुद् धात्मा विशोधनः।\\
अनिरुद्धो ऽप्रति रथः प्रद्युम्नो ऽमितविक्रमः ॥ 68 ॥\\
\\
काल नेमि निहा वीरः शौरिः शूर जनेश्वरः ।\\
त्रिलोकात्मा त्रिलो केशः केशवः केशिहा हरिः॥69॥\\
\\
कामदेवः कामपालः कामी कान्तः कृतागमः ।\\
अनिर् देश्यव पुर्विष्णुर् वीरो ऽनन्तो धनञ्जयः॥70॥\\
\\
ब्रह्मण्यो ब्रह्म कृद् ब्रह्मा ब्रह्म ब्रह्म वि वर्धनः ।\\
ब्रह्मविद् ब्राह्मणो ब्रह्मी ब्रह्मज्ञो ब्राह्मण प्रियः ॥71 ॥\\
\\
महा क्रमो महा कर्मा महा तेजा महो रगः ।\\
महा क्रतुर् महा यज्वा महा यज्ञो महा हविः ॥ 72 ॥\\
\\
स्तव्यः स्तव प्रियः स्तोत्रं स्तुतिः स्तो ता रण प्रियः ।\\
पूर्णः पूर यिता पुण्यः पुण्य कीर्ति रना मयः ॥ 73 ॥\\
\\
मनो जवस् तीर् थकरो वसु रेता वसु प्रदः ।\\
वसु प्रदो वासु देवो वसुर् वसुमना हविः ॥ 74 ॥\\
\\
सद् गतिः सत् कृतिः सत्ता सद् भूतिः सत् परायणः ।\\
शूर सेनो यदु श्रेष्ठः सन्नि वासः सुया मुनः ॥ 75 ॥\\
\\
भूता वासो वासु देवः सर्वा सुनि लयोऽनलः ।\\
दर्पहा दर्पदो दृप्तो दुर्धरोऽ थापराजितः ॥ 76 ॥\\
\\
विश्व मूर्तिर् महा मूर्तिर् दीप्त मूर्ति रमूर्तिमान् ।\\
अनेक मूर्ति रव्यक्तः शतमूर्तिः शता ननः ॥ 77 ॥\\
\\
एको नैकः सवः कः किं यत्तत् पदमनुत्तमं ।\\
लोक बन्धुर् लोकनाथो माधवो भक्त वत्सलः॥78 ॥\\
\\
सुवर्ण वर्णो हे माङ्गो वराङ्गश् चन्द नाङ्गदी ।\\
वीरहा विषमः शून्यो घृता शीर चलश्चलः ॥ 79 ॥\\
\\
अमा नीमान दोमान्यो लोक स्वामी त्रिलो कधृक् ।\\
सुमेधा मेध जोधन्यः सत्य मेधा धरा धरः ॥ 80 ॥\\
\\
\subsubsection{\eng{Sloka 81-100}}
तेजोऽ वृषो द्युति धरः सर्व-शस्त्र-भृतांवरः ।\\
प्रग्रहो निग्रहो व्यग्रो नैकशृङ्गो गदा ग्रजः ॥ 81 ॥\\
\\
चतुर् मूर्तिश् चतुर्बाहुश् चतुर् व्यूहश् चतुर्गतिः ।\\
चतुरात्मा चतुर् भावश् चतुर् वेद विदेकपात् ॥ 82 ॥\\
\\
समा वर्तोऽ निवृत्तात्मा दुर्जयो दुर तिक्रमः ।\\
दुर्लभो दुर्गमो दुर्गो दुरा वासो दुरा रिहा ॥ 83 ॥\\
\\
शुभाङ्गो लोकसारङ्गः सुतन्तु स्तन्तु वर्धनः ।\\
इन्द्रकर्मा महाकर्मा कृतकर्मा कृतागमः ॥ 84 ॥\\
\\
उद्भवः सुन्दरः सुन्दो रत्ननाभः सुलोचनः ।\\
अर्को वाज सनः शृङ्गी जयन्तः\\
 सर्वविज्जयी ॥ 85 ॥\\
\\
सुवर्ण बिन्दु रक्षोभ्यः सर्व वागीश् वरेश्वरः ।\\
महाहृदो महागर्तो महाभूतो महानिधिः ॥ 86 ॥\\
\\
कुमुदः कुन्दरः कुन्दः पर्जन्यः पावनोऽनिलः ।\\
अमृतां शोऽ मृत वपुः सर्वज्ञः सर्व तोमुखः ॥ 87 ॥\\
\\
सुलभः सुव्रतः सिद्धः शत्रु जिच् छत्रु तापनः ।\\
न्यग्रोधो ऽदुम्बरोश् वत्थश् चाणूरान्ध्र निषूदनः॥88॥\\
\\
सहस्रार्चिः सप्तजिह्वः सप्तैधाः सप्तवाहनः ।\\
अमूर्ति रनघोऽ चिन्त्यो भयकृद् भयनाशनः ॥ 89 ॥\\
\\
अणुर् बृहत् कृशः स्थूलो गुण भृन्निर् गुणो महान् ।\\
अधृतः स्वधृ तःस् वास्यः प्राग्वं शोवं शवर्धनः ॥90॥\\
\\
भार भृत् कथि तो-योगी योगी शः सर्व कामदः ।\\
आश्रमः श्रमणः, क्षामः सुपर्णो वायुवाहनः ॥ 91 ॥\\
\\
धनुर्धरो धनुर् वेदो दण्डो दम-यिता दमः ।\\
अपरा जितः सर्वसहो नियन्ताऽ नियमोऽ यमः॥92 ॥\\
\\
सत्-त्व वान् सात्-त्विकः सत्यः सत्य धर्म परायणः ।\\
अभि प्रायः प्रियार् होऽर्हः प्रिय कृत्  प्रीति वर्धनः ॥93॥\\
\\
\\
विहाय सग तिर् ज्योतिः सुरु चिर् हुत भुग्विभुः ।\\
रविर् विरो चनः सूर्यः सविता रवि लोचनः ॥ 94 ॥\\
\\
अनन्तो हुत भुग् भोक्ता सुखदो नैकजोऽ ग्रजः ।\\
अनिर् विण्णः सदामर्षी लोका धिष्ठा\\
 नमद् भुतः॥95॥\\
सनात् सना तन तमः कपिलः कपि रव्ययः ।\\
स्वस्तिदः स्वस्ति कृत् स्वस्तिः \\
स्वस्ति भुक् स्वस्ति दक्षिणः ॥ 96 ॥\\
\\
अरौद्रः कुण्डली चक्री विक्रम् यूर्जि तशा सनः ।\\
शब्दा तिगः शब्द सहः शिशिरः शर्व रीकरः ॥ 97 ॥\\
\\
अक्रूरः पेशलो दक्षो दक्षिणः, क्षमिणां वरः ।\\
विद् वत्तमो वी तभयः पुण्य-श्रवण कीर्तनः ॥ 98 ॥\\
\\
उत्ता-रणो दुश् कृतिहा पुण्यो दुः स्वप्न नाशनः ।\\
वीरहा रक्षणः सन्तो जीवनः पर्य वस्थितः ॥ 99 ॥\\
\\
अनन्त रूपोऽ नन्तश्रीर् जित मन्युर् भयापहः ।\\
चतुरश्रो गभी-रात्मा विदिशो \\
व्या-दिशो दिशः ॥100॥\\
\subsubsection{\eng{Sloka 101-108}}
अनादिर् भूर्भुवो लक्ष्मीः सुवीरो रुचि राङ्गदः ।\\
जननो जन जन्मादिर् भीमो भीम पराक्रमः ॥ 101 ॥\\
\\
आधार निलयोऽ धाता पुष्प हासः प्रजागरः ।\\
ऊर्ध्वगः सत्प था चारः प्राणदः प्रणवः पणः ॥102 ॥\\
\\
प्रमाणं प्राण निलयः प्राण भृत् प्राण जीवनः ।\\
तत्त्वं तत्त्व विदे कात्मा जन्म मृत्यु जरा तिगः॥103 ॥\\
\\
भूर्भुवः स्वस् तरुस् तारः सविता प्रपिता महः ।\\
यज्ञो यज्ञ पतिर् यज्वा यज्ञाङ्गो यज्ञ वाहनः॥104॥\\
\\
यज्ञ भृद् यज्ञ कृद् यज्ञी यज्ञ भुक् यज्ञ साधनः ।\\
यज्ञान्त कृद् यज्ञ गुह्य मन्न मन्नाद एव च ॥ 105 ॥\\
\\
आत्म योनिः स्वयञ् जातो वैखानः साम गायनः ।\\
देवकी नन्दनः स्रष्टा क्षिती शः पापना शनः॥106 ॥\\
\\
शङ्ख भृन् नन्दकी चक्री शार्ङ्ग धन्वा गदाधरः ।\\
रथाङ्ग पाणि रक्षोभ्यः सर्व प्रहरणा युधः ॥ 107 ॥\\
\\
श्री सर्व प्रहरणा युध ॐ नम इति ।\\
\\
वन माली गदी शार्ङ्गी शङ्खी चक्री च नन्दकी ।\\
श्रीमान्नारायणो विष्णुर्वासुदेवोऽभिरक्षतु ॥ 108 ॥\\
\\
(श्री वासुदेवोऽभिरक्षतु ॐ नम इति ।)\\
\subsection{फलश्रुतिः}
इतीदं कीर्त नीयस्य केशवस्य महात्मनः ।\\
नाम्नां सहस्रं दिव्याना मशे षेण प्रकीर्तितम्। ॥ 1 ॥\\
\\
य इदं शृणु यान्नित्यं यश्चापि परि कीर्तयेत्॥\\
नाशुभं प्राप्नुयात् किञ्चित् सोऽमु त्रेह च मानवः ॥ 2 ॥\\
\\
वेदान्तगो ब्राह्म णःस्यात् क्षत्रियो विजयी भवेत् ।\\
वैश्यो धन समृद्धः स्यात् शूद्रः सुखम वाप्नुयात् ॥ 3 ॥\\
\\
धर्मार्थी प्राप्नु याद्धर्म मर्थार्थी चार्थ माप्नुयात् ।\\
कामान वाप्नुयात् कामी प्रजार्थी चाप्नु यात्प्रजाम्। ॥ 4 ॥)\\
\\
भक्तिमान् यः सदोत्थाय शुचिस्तद्ग तमानसः ।\\
सहस्रं वासुदेवस्य नाम्नामेतत् प्रकीर्तयेत् ॥ 5 ॥\\
\\
यशः प्राप्नोति विपुलं याति प्राधान्य मेव च ।\\
अचलां श्रिय माप्नोति श्रेयः प्राप्नोत्य नुत्तमम्। ॥ 6 ॥\\
\\
न भयं क्वचि दाप्नोति वीर्यं तेजश्च विन्दति ।\\
भवत्यरोगो द्युतिमान् बलरूप गुणान्वितः ॥ 7 ॥\\
\\
रोगार्तो मुच्य ते रोगाद् बद्धो मुच्येत बन्धनात् ।\\
भयान् मुच्येत भीतस्तु मुच्ये तापन्न आपदः ॥ 8 ॥\\
\\
दुर्गाण् यतित रत्याशु पुरुषः पुरुषोत्तमम् ।\\
स्तुवन्नाम सहस्रेण नित्यं भक्ति समन्वितः ॥ 9 ॥\\
\\
वासु देवाश्रयो मर्त्यो वासुदेव परायणः ।\\
सर्व पाप विशुद्धात्मा याति ब्रह्म सना तनम्। ॥ 10 ॥\\
\\
न वासुदेव भक्ताना मशुभं विद्यते क्वचित् ।\\
जन्ममृत्यु जराव्याधि भयं नैवोप जायते ॥ 11 ॥\\
\\
इमं स्तव मधीयानः श्रद्धा भक्ति समन्वितः ।\\
युज्ये तात्म सुख क्षान्ति श्रीधृति स्मृति कीर्तिभिः ॥ 12 ॥\\
\\
न क्रोधो न च मात्सर्यं न लोभो नाशुभामतिः ।\\
भवन्ति कृतपुण्यानां भक्तानां पुरुषोत्तमे ॥ 13 ॥\\
\\
द्यौःस चन्द्रार्क नक्षत्रा खं दिशो भूर्म होदधिः ।\\
वासुदेवस्य वीर्येण विधृतानि महात्मनः ॥ 14 ॥\\
\\
ससुरा सुर गन्धर्वं सयक्षो रग राक्षसं ।\\
जगद्वशे वर्ततेदं कृष्णस्य सच राचरम्। ॥ 15 ॥\\
\\
इन्द्रियाणि मनो बुद्धिः सत्त्वं तेजो बलं धृतिः ।\\
वासुदेवात् मकान् याहुः, क्षेत्रं क्षेत्रज्ञ एव च ॥ 16 ॥\\
\\
सर्वा गमाना माचारः प्रथमं परिकल्पते ।\\
आचार प्रभवो धर्मो धर्मस्य प्रभु रच्युतः ॥ 17 ॥\\
\\
ऋषयः पितरो देवा महा भूतानि धातवः ।\\
जङ्गमा जङ्गमं चेदं जगन् नारायणोद्भवं ॥ 18 ॥\\
\\
योगोज्ञानं तथा साङ्ख्यं विद्याः शिल्पादिकर्म च ।\\
वेदाः शास्त्राणि विज्ञान मेतत्सर्वं जनार्दनात् ॥ 19 ॥\\
\\
एको विष्णुर्महद्भूतं पृथग् भूतान् यनेकशः ।\\
त्रींलो कान् व्याप्य भूतात्मा भुङ्क्ते विश्व भुगव् ययः ॥ 20 ॥\\
\\
इमं स्तवं भगवतो विष्णोर्व्यासेन कीर्तितं ।\\
पठेद्य इच्चेत्पुरुषः श्रेयः प्राप्तुं सुखानि च ॥ 21 ॥\\
\\
विश्वेश्वर मजं देवं जगतः प्रभु रव्ययम्।\\
भजन्ति ये पुष्क राक्षं न ते यान्ति पराभवं ॥ 22 ॥\\
\\
न ते यान्ति पराभवं ॐ नम इति ।\\
\\
अर्जुन उवाच\\
पद्मपत्र विशालाक्ष पद्मनाभ सुरोत्तम ।\\
भक्ताना मनुरक्तानां त्राता भव जनार्दन ॥ 23 ॥\\
\\
श्रीभगवानुवाच\\
यो मां नाम सहस्रेण स्तोतु मिच्छति पाण्डव ।\\
सोऽहमेकेन श्लोकेन स्तुत एव न संशयः ॥ 24 ॥\\
\\
स्तुत एव न संशय ॐ नम इति ।\\
\\
व्यास उवाच\\
वास नाद् वासु देवस्य वासितं भुवनत्रयम् ।\\
सर्व भूतनि वासोऽसि वासुदेव नमोऽस्तु ते ॥ 25 ॥\\
\\
श्रीवासुदेव नमोस्तुत ॐ नम इति ।\\
\\
पार्वत्युवाच\\
केनोपायेन लघुना विष्णोर्नामसहस्रकं ।\\
पठ्यते पण्डितैर्नित्यं श्रोतुमिच्छाम्यहं प्रभो ॥ 26 ॥\\
\\
ईश्वर उवाच\\
श्रीराम राम रामेति रमे रामे मनोरमे ।\\
सहस्रनाम तत्तुल्यं रामनाम वरानने ॥ 27 ॥\\
\\
श्रीराम नाम वरानन ॐ नम इति ।\\
\\
ब्रह्मोवाच \\
नमोऽस्त्वनन्ताय सहस्र मूर्तये सहस्रपादा क्षिशिरो रुबाहवे ।\\
सहस्रनाम्ने पुरुषाय शाश्वते सहस्रकोटी युगधारिणे नमः ॥ 28 ॥\\
\\
श्री सहस्रकोटी युगधारिणे नम ॐ नम इति ।\\
\\
सञ्जय उवाच\\
यत्र योगेश्वरः कृष्णो यत्र पार्थो धनुर्धरः ।\\
तत्र श्रीर्विजयो भूति र्ध्रुवा नीतिर्मतिर्मम ॥ 29 ॥\\
\\
श्री भगवान् उवाच\\
अनन्याश्चिन्त यन्तो मां  ये जनाः पर्युपासते ।\\
तेषां नित्या भियुक्तानां योग क्षेमं वहाम्यहम्। ॥ 30 ॥\\
\\
परित्राणाय साधूनां विनाशाय च दुष्कृताम्। ।\\
धर्मसंस्थाप नार्थाय सम्भवामि युगे युगे ॥ 31 ॥\\
\\
आर्ताः विषण्णाः शिथिलाश्च भीताः घोरेषु च व्याधिषु वर्तमानाः ।\\
सङ्कीर्त्य नारायणशब्दमात्रं विमुक्तदुःखाः सुखिनो भवन्ति ॥ 32 ॥\\
\\
कायेन वाचा मनसेन्द्रियैर्वा बुद्ध्यात्मना वा प्रकृतेः स्वभावात् ।\\
करोमि यद्यत्सकलं परस्मै नारायणायेति समर्पयामि ॥ 33 ॥\\
\\
