\subsection{\eng{Tvamagne Rudro Anuvakaha}}
त्वम॑ऽग्ने रु॒द्रो, असु॑रो म॒हो दि॒वस् त्वग्ं शर्धो॒ मारु॑तं पृ॒क्ष ई॑शिषे।\\
त्वं वातै॑ ररु॒णैर् या॑सि शङ्ख॒ यस्त्वं पू॒षा वि॑ध॒तः पा॑सि॒ नुत्मना᳚॥ \\
\\
आ वो॒ राजा॑ नमध् व॒रस्य॑ रु॒द्रग्ं होता॑रग्ं सत्य॒ यज॒ग्ं॒ रोद॑स् योः।\\
अ॒ग्निं पु॒रा त॑न यि॒त्नो र॒चित् ता॒द् धिर॑ण्य रूप॒ मव॑से कृणुध्वम्॥\\
\\
अग्निर् होता निष॑सा दा॒ यजी॑य् यानु॒ पस्थे॑ मा॒तुः सु॑र॒भावु॑ लो॒के। \\
युवा॑ क॒विः पु॑रुनि॒ष्ठ ऋ॒तावा॑ ध॒र्ता  कृ॑ष्टी॒ नामु॒त मध्य॑ इ॒द्धः।\\
\\
सा॒ध्वी म॑कर् दे॒ववी॑ तिन्नो, अ॒द्य य॒ज्ञस्य॑ जि॒ह्वाम॑ विदाम॒ गुह्या᳚म्\\
स आयु॒राऽगा᳚त् सुर॒भिर्वसा॑नो भ॒द्राम॑कर् दे॒वहू॑ तिन्नो, अ॒द्य॥\\
\\
अक्र॑न्द द॒ग्निः स्त॒न य॑न्नि व॒द्यौः क्षामा॒ रेरि॑हद् वी॒रुधः॑ सम॒ञन्न्।\\
स॒द्यो ज॑ज्ञा॒नो विही मि॒द्धो, अख्य॒ दारो द॑सी भा॒नुना॑ भात्य॒न्तः॥\\
\\
त्वे वसू॑नि पुर्वणी कहोतर् दो॒षा वस्तो॒ रेरि॑रे य॒ज्ञिया॑सः।\\
क्षामे॑ व॒विश्वा॒ भुव॑नानि॒ यस् मि॒न्थ् सꣳ सौ भ॑गानि दधि॒रे पा॑व॒के॥\\
\\
तुभ्यं ता, अ॑ङ्गिरस्तम॒ विश्वाः᳚ सुक् क्षि॒तयः॒ पृथ॑क्।\\
अग्ने॒ कामा॑य येमिरे॥\\
\\
अ॒श्याम॒ तंकाम॑ मऽग्ने॒ तवो॒त् य॑श्याम॑ र॒यिꣳ र॑यिवः सु॒वीरम्᳚\\
अ॒श्याम॒ वाज॑म॒भि वा॒जय॑न् तो॒ऽश्याम॑द्  द्यु॒म्न म॑ज रा॒जर॑न्ते॥\\
\\
श्रेष्ठं॑य विष्ठ भार॒ ताऽग्ने᳚ द्युमन्त॒ माभ॑र।\\
वसो॑ पुरु॒स् पृहꣳ॑ र॒यिम्॥\\
 \\
सश्वि॑ ता॒नस् त॑न्य॒तू रो॑च न॒स्था, अ॒जरे॑भि॒र् नान॑दद् भि॒र् यवि॑ष्ठः।\\
यः पा॑व॒कः पु॑रु॒तमः॑ पुरू॒णि॑ पृ॒थून् य॒ग्नि र॑नु॒याति॒ भर्वन्न्॥\\
\\
आयु॑ष्टे वि॒श्वतो॑ दध द॒य म॒ग्निर् वरे᳚ण्यः।\\
पुन॑स्ते प्रा॒ण आऽय॑ति॒ परा॒ यक्ष्मꣳ॑ सुवामि ते॥\\
\\
आ॒यु॒र्दा, अ॑ग्ने ह॒विषो॑ जुषा॒णो घृ॒त प्र॑तीको घृ॒तयो॑ निरेधि।\\
घृ॒तं पी॒त्वा मधु॒ चारु॒ गव्यं॑ पि॒तेव॑ पु॒त्रम॒भि र॑क्ष तादि॒मम्।॥\\
\\
तस्मै॑ ते प्रति॒हर्य॑ते॒ जात॑वेदो॒ विच॑र्षणे।\\
अग्ने॒ जना॑मि सुष्टु॒ तिम्॥\\
\\
दि॒वस्परि॑ प्रथमं ज॑ज्ञे, अ॒ग्नि र॒स्मद् द्वि॒तीयं॒ परि॑ जा॒तवे॑दाः।\\
तृ॒तीय॑ म॒प्सु नृ॒मणा॒, अज॑स्र॒ मिन्धा॑न एनं जरते स्वा॒धीः॥\\
\\
शुचिः॑ पावक॒ वन्द्योऽग्ने॑ बृहद् विरो॑चसे।\\
त्वं घृ॒ते भि॒राहु॑तः॥\\
\\
दृ॒शा॒नो रु॒क्म उ॒र्व्याव् य॑द् यौद् दु॒र्मर् ष॒ मायुः॑ श्रि॒ये रु॑चा॒नः।\\
अ॒ग्नि र॒मृतो॑, अभव॒द् वयो॑भिः यदे॑ नं॒द्यौर ज॑नयत् सु॒रेताः᳚॥\\
\\
आ यदि॒षे नृ॒पतिं॒ तेज॒ आन॒ट् शुचि॒ रेतो॒ निषिक्तं॒ द्यौर॒भीके᳚।\\
अ॒ग्निः शर्ध मनव॒द्यं युवा॑नग्ग् स्वा॒ धियं जनयत् सू॒दय॑च्च ॥\\
\\
सते जीयसा॒ मन॑सा॒ त्वोत॑ उ॒त शि॑क्षस्-वप्-प्र॒त्-यस्य॑ शि॒क्षोः।\\
अग्ने॑ रा॒यो नृत॑ मस्य॒ प्रभू॑तौ भू॒याम॑ ते सुष्टुत य॑श्च॒ वस्वः॑।\\
\\
अग्ने॒ सह॑न्त॒ माभ॑र द्यु॒म्-नस्य॑  प्रा॒सहा॑ र॒यिम्।\\
विश्वा॒ यश् च॑र्ष॒णी॒ रभ्या॑सा वाजे॑षु सा॒सह॑त्।\\
\\
तम॑ग्ने पृतना॒सहꣳ॑ र॒यिꣳ स॑हस्व॒ आ भ॑र।\\
त्वꣳ हि स॒त्यो, अद्भु॑तो दा॒ता वाज॑स्य गोम॑तः॥\\
\\
उ॒क्षान्ना॑य व॒शान्ना॑य॒ सोम॑ पृष्ठाय वे॒धसे᳚।\\
स्तोमै᳚र् विधे मा॒ऽग् नये᳚॥\\
\\
व॒द्मा हि सू॑नो॒, अस्य॑द् म॒सद् वा॑ च॒क्रे, अ॒ग्निर् ज॒नु षाऽज् मान्नम्᳚।\\
सत्वन्न॑ ऊर् जसन॒ ऊर्जं॑ धा॒ राजे॑ वजे रवृ॒के क्षे᳚ष् य॒न्तः॥\\
\\
अग्न॒ आयूꣳ॑ षि पवस॒ आसु॒ वोर् ज॒मिषं॑ च नः।\\
आ॒रे बा॑धस्व दु॒च्छु ना᳚म्॥\\
\\
अग्ने॒ पव॑स् व॒स् वपा॑, अ॒स्मे वर्चः॑॑ सु॒वीर्यम्᳚\\
दध॒त्पोषꣳ॑ र॒यिं मयि॑॥\\
\\
अग्ने॑ पावक रो॒चिषा॑ म॒न्द्रया॑ देव जिह्वया᳚।\\
आ दे॒वान् व॑क्षि॒ यक्षि॑ च॥\\
\\
स नः॑ पावक दीदि॒ वोऽग्ने॑ दे॒वाꣳ इ॒हाऽऽव॑ह।\\
उप॑ य॒ज्ञꣳ ह॒विश् च॑नः॥\\
\\
अ॒ग्निः शुचि॑व् व्रत तमः॒ शुचि॒र् विप्रः॒ शुचिः॑ क॒विः।\\
शुची॑ रोचत॒ आहु॑तः॥\\
\\
उद॑ग्ने॒ शुच॑ य॒स्तव॑ शु॒क्रा, भ्राज॑न्त ईरते।\\
तव॒ ज्योतीग्ग्॑ष् य॒र्चयः॑॥\\
\\
त्वम॑ग्ने रु॒द्रो असु॑रो म॒हो दि॒वः। त्वꣳ शर्धो॒ मारु॑तं पृ॒क्ष ई॑शिषे।\\
त्वं वातै॑ररु॒णैर्या॑सि शङ्ग॒यः। त्वं पू॒षा वि॑ध॒तः पा॑सि॒ नु त्मना᳚।\\
\\
देवा॑ दे॒वेषु॑ श्रयद्ध्वम्। प्रथ॑मा द्वि॒तीये॑षु श्रयद्ध्वम्। \\
द्विती॑यास् तृ॒तीये॑षु श्रयद्ध्वम्। तृती॑याश् चतु॒र्थेषु॑ श्रयद्ध्वम्। \\
च॒तु॒र्थाः प॑श्च॒मेषु॑ श्रयद्ध्वम्। प॒ञ्च॒माः ष॒ष् ठेषु॑ श्रयद्ध्वम्।\\
\\
ष॒ष्ठाः स॑प्त॒मेषु॑ श्रयद्ध्वम्। स॒प्त॒मा, अ॑ष्ट॒मेषु॑ श्रयद्ध्वम्। \\
अ॒ष्ट॒मा-न॑व॒मेषु॑ श्रयद्ध्वम्। न॒व॒मा-द॑श॒मेषु॑ श्रयद्ध्वम्। \\
द॒श॒मा, ए॑का द॒शेषु॑ श्रयद्ध्वम्। ए॒का॒द॒शा द्वा॑ द॒शेषु॑ श्रयद्ध्वम्। \\
द्वा॒ द॒शास् त्र॑यो द॒शेषु॒॑ श्रयद्ध्वम्। त्र॒यो॒ द॒शाश् च॑तर् द॒शेषु॑ श्रयद्ध्वम्। \\
च॒त॒र् द॒शाः प॑ञ्च द॒शेषु॑ श्रयद्ध्वम्। प॒ञ्च॒ द॒शा: षो॑ड॒शेषु॑ श्रयद्ध्वम्। \\
षो॒ड॒शाः स॑प्त द॒शेषु॑ श्रयद्ध्वम्। स॒प्त॒ द॒शा, अ॑ष्टा द॒शेषु॑ श्रयद्ध्वम्।\\
\\
अ॒ष्टा॒द॒शा, ए॑कान् नवि॒ꣳ॒ शेषु॑ श्रयद्ध्वम्। ए॒का॒न् न॒विꣳ॒ शा वि॒ꣳ॒शेषु॑ श्रयद्ध्वम्। \\
वि॒ꣳ॒शा, ए॑क वि॒ꣳ॒ शेषु॑ श्रयद्ध्वम्। ए॒क॒वि॒ꣳ॒ शा द्वा॑ वि॒ꣳ॒शेषु॑ श्रयद्ध्वम्। \\
द्वा॒ वि॒ꣳ॒शास् त्र॑योवि॒ꣳ॒ शेषु॑ श्रयद्ध्वम्। त्र॒यो॒वि॒ꣳ॒ शाश् च॑तर् वि॒ꣳ॒ शेषु॑ श्रयद्ध्वम्। \\
च॒त॒र् वि॒ꣳ॒शाः प॑ञ्चवि॒ꣳ॒ शेषु॑ श्रयद्ध्वम्। प॒ञ्च॒वि॒ꣳ॒ शाः ष॑ड् वि॒ꣳ॒शेषु॑ श्रयद्ध्वम्। \\
ष॒ड् विꣳ॒शाः स॑प्त वि॒ꣳ॒ शेषु॑ श्रयद्ध्वम्। स॒प्त॒ विꣳ॒शा, अ॑ष्टा वि॒ꣳ॒ शेषु॑ श्रयद्ध्वम्। \\
\\
अ॒ष्टा॒ वि॒ꣳ॒ शा, ए॑कान् नत्रि॒ꣳ॒ शेषु॑ श्रयद्ध्वम्। ए॒का॒न् न॒त्रि॒ꣳ शास् त्रि॒ꣳ॒ शेषु॑ श्रयद्ध्वम्। \\
त्रि॒ꣳ॒ शा, ए॑कत्रि॒ꣳ॒ शेषु॑ श्रयद्ध्वम्। ए॒क॒त्रि॒ꣳ॒ शा द्वा᳚ त्रि॒ꣳ॒ शेषु॑ श्रयद्ध्वम्।\\
द्वा॒त्रि॒ꣳ॒ शास् त्र॑यस् त्रि॒ꣳ॒ शेषु॑ श्रयद्ध्वम् । \\
\\
देवा᳚स् त्रिरेका दशा॒स् त्रिस् त्र॑यस् त्रिꣳशाः। \\
उत्त॑रे भवत । उत्त॑र वर्त् मान॒ उत्त॑र सत्वानः। यत्का॑म इ॒दं जु॒होमि॑।\\
तन्मे॒ समृ॑द् यताम्। व॒यग्ग् स्या॑म॒ पत॑यो रयी॒ णाम्। भूर्भुवः॒ स्वः॑ स्वाहा᳚॥\\
\\
अस्त्रायफट्\\
