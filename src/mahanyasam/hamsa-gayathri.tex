\subsection{\eng{Hamsa gayathri stotram}}
अस्य श्री हंस गायत्री स्तोत्र महामन्त्रस्य। अव्यक्त पर ब्रह्म ऋषिः ।\\
{\small अनुष्टुप् छन्दः।} अव्यक्त गायत्रि छन्दः। \\
परम हंसो देवता । हंसां बीजं। हंसीं शक्तिः।\\
हंसों कीलकं। परम हंस प्रसाद सिध्द्यर्थे जपे विनियोगः॥\\
\\
हंसां आङ्गुष्ठाभ्यां नमः ।\\
हंसीं तर्जनीभ्यां नमः ।\\
हंसूं मध्यमाभ्यां नमः ।\\
हंसैं अनामिकाभ्यां नमः ।\\
हंसौं कनिष्ठिकाभ्यां नमः ।\\
हंसः करतलकर पृष्ठाभ्यां नमः ॥\\
\\
हंसां हृदयाय नमः ।\\
हंसीं शिरसे स्वाहा।\\
हंसूं शिखायै वषट्।\\
हंसैं कवचायहम्।\\
हंसौं नेत्रत्रयाय वषट्।\\
हंसः अस्त्राय फट्।\\
\\
भूर्भुव॒ स्सुव॒रोमिति दिग्बन्धः ॥\\
\\
\subsubsection{\eng{Dhyanam}}
ध्यानं।\\
गमा गमस्थं गमनादि शून्यं चिद्रूपदीपं तिमिरापहारम्।\\
पश्यामि ते सर्वजनान्तरस्थं नमामि हंसं परमात्मरूपम्॥\\
देहो देवालयः प्रोक्तो जीवो देवस्सनातनः।\\
त्यजे दज्ञान निर्माल्यं सोऽहं भावेन पूजयेत्॥\\
\\
हंसो हंसः परम हंसः  \\
हंसस् सोऽहं सोऽहं हंसः॥\\
\\
हं॒स॒ हंसा॒य॑ वि॒द्महे॑ परमहंसा॒य॑ धीमही।\\
तन्नो॑ हंसः प्रचो॒दया᳚त्॥\\
\\
हंस हंसेति योब्रूयाध् दं॑सो ना॑म स॒दाशि॑वः।\\
एवं न्यास विधिं॒ कृत्वा ततस् संपुट मारभेत्॥\\
