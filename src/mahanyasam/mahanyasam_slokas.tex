\section{\eng{Mahanyasam}}
\subsection{\eng{Panchanga Rudra Nyasaha}}
ओं नमो भगवते॑ रुद्रा॒य ॥\\
\\
अथातः पंचाग रुद्राणां\\
न्यास पूर्वकं जपहोमा	र्च नाभिषेक विधिं वया᳚ख्या॒स्यामः ॥\\
\\
ओं  ओंकार मन्त्र संयुक्तं नित्यं ध्यायन्ति योगिनः।\\
कामदं मोक्षदं तस्मै ओं नकाराय नमोनमः॥\\
\\
ओं भूर्भुवस्सुवः॑॥ \\
\\
ओं नं - नम॑स्ते रुद्र म॒न्यव॑ उ॒तोत॒ इष॑वे॒ नम॑: ।\\
नम॑स्ते अस्तु॒ धन्व॑ने बा॒हुभ्या॑मु॒त ते॒ नम॑: ।\\
\\
ओं कं खं गं घं डं । ओं नमो भगवते॑ रुद्रा॒य ॥\\
\\
नं ओं - पूर्वाङ्ग रुद्राय नमः॥ 1॥\\
\\
महादेवं महात्मानं महा पातक नाशनम्।\\
महा पाप हरं वन्दे मकाराय नमोनमः॥\\
\\
ओं भूर्भुवस्सुवः॑॥ ओं मं \\
निध॑नपतये॒ नमः । निध॑नपतान्तिकाय॒ नमः ।\\
ऊर्ध्वाय॒ नमः । ऊर्ध्वलिङ्गाय॒ नमः ।\\
हिरण्याय॒ नमः । हिरण्यलिङ्गाय॒ नमः ।\\
सुवर्णाय॒ नमः । सुवर्णलिङ्गाय॒ नमः ।\\
दिव्याय॒ नमः । दिव्यलिङ्गाय॒ नमः ।\\
भवाय॒ नमः । भवलिङ्गाय॒ नमः ।\\
शर्वाय॒ नमः । शर्वलिङ्गाय॒ नमः ।\\
शिवाय॒ नमः । शिवलिङ्गाय॒ नमः ।\\
ज्वलाय॒ नमः । ज्वललिङ्गाय॒ नमः ।\\
आत्माय॒ नमः । आत्मलिङ्गाय॒ नमः ।\\
परमाय॒ नमः । परमलिङ्गाय॒ नमः ।\\
एतत्  सोमस्य॑ सूर् यस्य॒ सर्व\\
लिङ्ग॑ꣳस्था प॒य॒ति॒ पाणि मन्त्रं॑ पवि॒त्रम् ॥\\
\\
ओं छं जं झं ञं नं । ओं नमो भगवते॑ रुद्रा॒य ॥\\
मं ओं - दक्षिणाङ्ग रुद्राय नमः॥ 2 ॥\\
\\
शिवं शान्तं जगन्नाधं लोकानुग्रह कारणम्।\\
शिवमेकं परं वन्दे शिकाराय नमोनमः॥\\
\\
ओं भूर्भुवस्सुवः॑॥ ओं शिं \\
अपै॑तु मृ॒त्यु र॒मृतं॑ न॒ आग॑न् वैवस् व॒तो नो॒ अभ॒यं कृणोतु।\\
प॒र्णं वन॒स्पते॑ रिवा॒भिन॑श् शीयताग्ं र॒यिस् सच॑तां न॒श्श ची पतिः॑॥\\
\\
ओं  टं ठं डं ढं णं । ओं नमो भगवते॑ रुद्रा॒य ॥\\
शिं ओं - पश्चिमाङ्ग रुद्राय नमः॥ 3 ॥ \\
\\
वाहनं वृषभो यस्य वासुकिः कण्ठ भूषणम् ।\\
वामे शक्तिधरं वन्दे वकाराय नमोनमः॥\\
\\
ओं भूर्भुवस्सुवः॑॥ ओं वां \\
प्राणानां ग्रन्थिरसि रुद्रो मा॑ विशा॒न्तकः ।\\
तेनान्नेना᳚प्याय॒स्व ।\\
\\
ओं  तं थं दं घं नं । ओं नमो भगवते॑ रुद्रा॒य ॥\\
वां औं - उत्तराङ्ग रुद्राय नमः॥ 4॥\\
\\
यत्र कुत्र स्थितं देवं सर्व व्यापिन मीश्वरम्।\\
यल्लिङ्गं पूज येन्नित्यं यकाराय नमोनमः॥।\\
\\
ओं भूर्भुवस्सुवः॑॥ ओं यं\\
यो रु॒द्रो अ॒ग्नौ यो अ॒प्सु य ओष॑धीषु॒ यो रु॒द्रो\\
विश्वा॒ भुव॑ना वि॒वेश॒ तस्मै॑ रु॒द्राय॒ नमो॑ अस्तु ।\\
\\
ओं पं फं बं भं मं । ओं नमो भगवते॑ रुद्रा॒य ॥\\
यं ओं - ऊर्ध्वाङ्ग रुद्राय नमः ॥ 5 ॥\\
\subsection{\eng{Panchamuka Nyasam}}
तत्पुरु॑षाय वि॒द्महे॑ महादे॒वाय॑ धीमहि । \\
तन्नो॑ रुद्रः प्रचो॒दया᳚त् ॥\\
\\
संवर्ताग्नि तटित्प्रदीप्त कनक प्रस्पर्दि तेजोमयं\\
गम्भीरध्वनि  मिश्रितोग्र दहन प्रोद् भा सिता म्राधरम्\\
अर्धेन् दुद् युति लोल पिङ्गळ जटाभार प्रबद्धोरगं\\
वन्दे सिद्ध सुरासुरेन्द्र नमितं पूर्वं मुखंशूलिनः \\
\\
ओं अं कं खं गं घं डं । आं ओं ओं नमो भगवते॑ रुद्रा॒य ॥\\
ओं नं - पूर्व मुखाय नमः॥\\
अ॒घोरे᳚भ्योऽथ॒ घोरे᳚भ्यो॒ घोर॒घोर॑तरेभ्यः । \\
स॒र्वे᳚तः॑ सर्व॒ शर्वे᳚भ्यो॒ नम॑स्ते अस्तु रु॒द्र रू॑पेभ्यः ॥\\
\\
कालाभ्र भ्रमराञ्ज नद्युतिनिभं व्यावृत्त पिङ्गेक्षणं\\
कर्णोद्भासित भोगि मस्तक मणिं प्रोद्भिन्न दंष्ट्राङ्कुरम् ।\\
सर्प प्रोतक पाल शुक्ति शकल व्याकीर्ण सञ्चारगं\\
वन्दे दक्षिण मीश्व रस्य कुटिल भ्रूभङ्गरौद्रं मुखम् \\
\\
ओं इं छं जं झं ञं नं । ईं ओं ओं नमो भगवते॑ रुद्रा॒य ॥\\
ओं मं  - दक्षिण मुखाय नमः॥\\
\\
स॒द्योजा॒तं प्र॑पद्या॒मि॒ \\
स॒द्योजा॒ताय॒ वै नमो॒ नमः॑ ।\\
भ॒वे भ॑वे॒ नाति॑भवे भवस्व॒ माम् । \\
भ॒वोद्भ॑वाय॒ नमः ॥\\
\\
प्रालेयाचल चन्द्र कुन्द धवलं गोक्षीर फेन प्रभं\\
भस्माभ् यक्त मनङ्ग देह दहन ज्वाला वली लोचनम् ।\\
ब्रह्मेन्द्रादि मरुद्गणैः स्तुतिपरै रभ्यर्चितं योगि भिः\\
वन्देऽहं सकलं कलङ्क रहितं स्थाणोर्मुखं पश्चिमम् ॥\\
\\
ओं उं टं ठं डं ढं णं । ऊं ओं ओं नमो भगवते॑ रुद्रा॒य ॥\\
ओं  शिं - पश्चिम मुखाय नमः॥\\
\\
वा॒म॒दे॒वाय॒ नमो᳚ ज्ये॒ष्ठाय॒ नमः॑ श्रे॒ष्ठाय॒ नमो॑ रु॒द्राय॒ नमः॒ \\
काला॑य नमः॒ कल॑ विकरणाय॒ नमो॒ \\
बल॑ विकरणाय॒ नमो॒\\
बला॑य॒ नमो॒ बल॑प्रमथनाय॒ नमः॒ \\
सर्व॑ भूत दमनाय॒ नमो॑ म॒नोन्म॑नाय॒ नमः॒ ॥\\
\\
गौरं कुङ्कुम पङ्किलं सुतिलकं व्यापाण्डु गण्डस्थलं\\
भ्रूविक्षेप कटाक्ष वीक्षण लसत् संसक्त कर्णोत्पलम् ।\\
स्निग्धं बिम्ब फलाधर प्रहसितं नीलाल कालङ्कृतं\\
वन्दे पूर्ण शशाङ्क मण्डल निभं वक्त्रं हरस्योत्तरम् ॥\\
 \\
\\
ओं एं तं थं दं घं नं । ऐं ओं ओं नमो भगवते॑ रुद्रा॒य ॥\\
ओं वां - उत्तर मुखाय नमः॥\\
\\
ईशानः सर्व॑ विद्या॒ना॒ मीश्वरः सर्व॑भूता॒नां॒\\
ब्रह्माधि॑पति॒र्ब्रह्म॒णोऽधि॑पति॒ \\
र्ब्रह्मा॑ शि॒वो मे॑ अस्तु सदाशि॒वोम् ॥\\
\\
व्यक्ता व्यक्त गुणेतरं सुविमलं षट्त्रिं शतत् त्वात्मकं\\
तस्मा दुत्तर तत्त्व मक्षरमिति ध्येयं सदा योगिभिः ।\\
वन्दे तामस वर्जितं त्रिणयनं सूक्ष्मा तिसूक्ष्मात्परं\\
शान्तं पञ्चम मीश्वरस्य वदनं खव्यापि तेजोमयम् ॥\\
\\
\\
ओं ओं पं फं बं भं मं । औं ओं ओं नमो भगवते॑ रुद्रा॒य ॥\\
ओं यं - ऊर्ध्व मुखाय नमः॥\\
\\
पूर्वे पशुपतिः पातु दक्षिणे पातु शंकरः।\\
पश्चिमे पातु विश्वेशो नीलकण्ठस्तथोत्तरे॥\\
\\
ऐशान्यां पातुमां शर्वो ह्याग् नेय्यां पार्वती पतिः।\\
नैर्ऋर्त्यां पातुमां रुद्रो वायव्यां नीललोहितः॥\\
ऊर्ध्वे त्रिलोचनः पात् अधरायां महेश्वरः।\\
एताभ्यो दश दिग् भ्यस्तु सर्वतः पातु शंकरः॥\\
\\
\subsection{\eng{Keshadhi Padhanta nyasaha}}
ओं या ते॑ रुद्र शि॒वा त॒नूरघो॒राऽपा॑पकाशिनी ।\\
तया॑ नस्त॒नुवा॒ शन्त॑मया॒ गिरि॑शन्ता॒भिचा॑कशीहि ॥ शिखायै नमः॥ (1)\\
{\small\eng{Tuft}}\\
\\
अ॒स्मिन्म॑ह॒त्य॑र्ण॒वे᳚ऽन्तरि॑क्षे भ॒वा अधि॑ ।\\
तेषाग्ं॑ सहस्रयोज॒नेऽव॒धन्वा॑नि तन्मसि ॥ शिरसे नमः॥ (2)\\
{\small\eng{Top o\eng{f} Head}}\\
\\
स॒हस्रा॑णि सहस्र॒शो ये रु॒द्रा अधि॒ भूम्या᳚म् ।\\
तेषाग्ं॑ सहस्रयोज॒नेऽव॒धन्वा॑नि तन्मसि ॥ ललाटाय नमः॥ (3)\\
{\small\eng{Forehead}}\\
\\
ह॒ग्ं॒ सश् शुचि॒ षद् वसुरन् तरिक्ष॒सद् धोता वेदि॒ष दति थिर् दरो ण॒सत्।\\
नृ॒षद् वर॒सद्रु तसब् यो मसदब् जा गोजा ऋतजा अद्रिजा ऋतं बृहत्॥ \\
भ्रुवोर्मध्याय नमः॥ (4)\\
{\small\eng{Middle o\eng{f} Eyebrows}}\\
\\
त्र्य॑म्बकं यजामहे सुग॒न्धिं पु॑ष्टि॒वर्ध॑नम् ।\\
उ॒र्वा॒रु॒कमि॑व॒ बन्ध॑नान्मृ॒त्योर्मु॑क्षीय॒ माऽमृता᳚त् ॥  नेत्राभ्यां नमः॥ (5)\\
{\small\eng{Eyes}}\\
\\
नम॒: स्रुत्या॑य च॒ पथ्या॑य च॒    नम॑: का॒ट्या॑य च नी॒प्या॑य च॒ ॥ \\
कर्णाभ्यां नमः॥ (6)\\
{\small\eng{Ears}}\\
\\
मा न॑स्तो॒के तन॑ये॒ मा न॒ आयु॑षि॒ मा नो॒ गोषु॒ मा नो॒ अश्वे॑षु रीरिषः ।\\
वी॒रान्मा नो॑ रुद्र भामि॒तोऽव॑धीर्ह॒विष्म॑न्तो॒  नम॑सा विधेम ते ॥  \\
नासिकायै नमः॥ (7)\\
{\small\eng{Nose}}\\
\\
अ॒व॒तत्य॒ धनु॒स्तवग्ं सह॑स्राक्ष॒ शते॑षुधे ।\\
नि॒शीर्य॑ श॒ल्यानां॒ मुखा॑ शि॒वो न॑: सु॒मना॑ भव ॥ मुखाय नमः॥ (8)\\
{\small\eng{Face}}\\
\\
नील॑ग्रीवाः शिति॒कण्ठा᳚: श॒र्वा अ॒धः क्ष॑माच॒राः ।\\
तेषाग्ं॑ सहस्रयोज॒नेऽव॒धन्वा॑नि तन्मसि ॥ कण्ठाय नमः॥ (9)\\
{\small\eng{Throat}}\\
\\
नील॑ग्रीवाः शिति॒कण्ठा॒ दिवग्ं॑ रु॒द्रा उप॑श्रिताः ।\\
तेषाग्ं॑ सहस्रयोज॒नेऽव॒धन्वा॑नि तन्मसि ॥ उपकण्ठाय नमः। (10)\\
{\small\eng{Lower Neck}}\\
\\
नम॑स्ते अ॒स्त्वायु॑धा॒याना॑तताय धृ॒ष्णवे᳚ ।\\
उ॒भाभ्या॑मु॒त ते॒ नमो॑ बा॒हुभ्यां॒ तव॒ धन्व॑ने ॥ बाहुभ्यां नमः। (11)\\
{\small\eng{Shoulders}}\\
\\
या ते॑ हे॒तिर्मी॑ढुष्टम॒ हस्ते॑ ब॒भूव॑ ते॒ धनु॑: ।\\
तया॒ऽस्मान् वि॒श्वत॒स्त्वम॑य॒क्ष्मया॒ परि॑ब्भुज ॥ उपबाहुभ्यां नमः॥ (12)\\
{\small\eng{Elbow to Wrist}}\\
\\
{\small\eng{Not in challakere rendition}}\\
परि॑णो रु॒द्रस्य॑ हे॒तिर्वृ॑णक्तु॒ परि॑ त्वे॒षस्य॑ दुर्म॒तिर॑घा॒योः।\\
अव॑ स्थि॒रा म॒घव॑द्भ्यस्तनुष्व॒ मीढ्व॑स्तो॒काय॒ तन॑याय मृडय॥\\
मणिबन्धाभ्यां नमः॥ (13)\\
{\small\eng{Wrists}}\\
\\
ये ती॒र्थानि॑ प्र॒चर॑न्ति सृ॒काव॑न्तो निष॒ङ्गिण॑: ।\\
तेषाग्ं॑ सहस्रयोज॒नेऽव॒धन्वा॑नि तन्मसि ॥ हस्ताभ्यां नमः॥ (14)\\
{\small\eng{Hands}}\\
\\
स॒द्योजा॒तं प्र॑पद्या॒मि॒ स॒द्योजा॒ताय॒ वै नमो॒ नमः॑ ।\\
भ॒वे भ॑वे॒ नाति॑भवे भवस्व॒ माम् । भ॒वोद्भ॑वाय॒ नमः ॥ अङ्गुष्ठाभ्यां नमः॥ (15)\\
{\small\eng{Roll Ring Fingers on Thumb o\eng{f} each hand}}\\
\\
वा॒म॒दे॒वाय॒ नमो᳚ ज्ये॒ष्ठाय॒ नमः॑ श्रे॒ष्ठाय॒ नमो॑ रु॒द्राय॒ नमः॒ \\
काला॑य नमः॒ कल॑ विकरणाय॒ नमो॒ बल॑ विकरणाय॒ नमो॒\\
बला॑य॒ नमो॒ बल॑प्रमथनाय॒ नमः॒ सर्व॑ भूत दमनाय॒ \\
नमो॑ म॒नोन्म॑नाय॒ नमः॒ ॥ तर्जनीभ्यां नमः॥ (16)\\
{\small\eng{Roll Thumb on Ring Fingers o\eng{f} both Hands}}\\
\\
अ॒घोरे᳚भ्योऽथ॒ घोरे᳚भ्यो॒ घोर॒घोर॑तरेभ्यः । \\
स॒र्वे᳚तः॑ सर्व॒ शर्वे᳚भ्यो॒ नम॑स्ते अस्तु रु॒द्र रू॑पेभ्यः ॥ मध्यमाभ्यां नमः॥ (17)\\
{\small\eng{Roll Thumb on Middle Fingers o\eng{f} both Hands}}\\
\\
तत्पुरु॑षाय वि॒द्महे॑ महादे॒वाय॑ धीमहि । \\
तन्नो॑ रुद्रः प्रचो॒दया᳚त् ॥ अनामिकाभ्यां नमः॥ (18)\\
{\small\eng{Roll Thumb on Ring Fingers o\eng{f} both Hands}}\\
\\
ईशानः सर्व॑ विद्या॒ना॒ मीश्वरः सर्व॑भूता॒नां॒\\
ब्रह्माधि॑पति॒र्ब्रह्म॒णोऽधि॑पति॒ र्ब्रह्मा॑ शि॒वो मे॑ अस्तु सदाशि॒वोम् ॥\\
कनिष्ठिकाभ्यां नमः॥ (19)\\
{\small\eng{Roll Thumb on Little Fingers o\eng{f} both Hands}}\\
\\
{\small\eng{Not in challakere rendition}}\\
नमो हिरण्यबाहवे हिरण्यवर्णाय हिरण्यरूपाय \\
हिरण्यपतयेऽम्बिकापतय उमापतये\\
पशुपतये॑ नमो॒ नमः॥ करतल करपृष्ठाभ्यां नमः॥ (20)\\
{\small\eng{Rub each palm over other, front and back}}\\
\\
नमो॑ वः किरि॒केभ्यो॑ दे॒वाना॒ग्ं॒ हृद॑येभ्यः । हृदयाय नमः॥ (21)\\
{\small\eng{Touch Heart}}\\
\\
नमो॑ ग॒णेभ्यो॑ ग॒णप॑तिभ्यश्च वो॒ नमः॑ ॥ पृष्ठाय नमः॥ (22)\\
{\small\eng{Touch Back}}\\
\\
नम॒स्तक्ष॑भ्यो रथका॒रेभ्य॑श्च वो॒  नमः॑ ॥ कक्षाभ्यां नमः॥ (21)\\
{\small\eng{Armpit to Waist}}\\
\\
नमो॒ हिर॑ण्यबाहवे सेना॒न्ये॑ दि॒शां च॒ पत॑ये॒  नमः॑ ॥ पार्श्वाभ्यां नमः॥ (22)\\
{\small\eng{Trunk}}\\
\\
विज्यं॒ धनु॑: कप॒र्दिनो॒ विश॑ल्यो॒ बाण॑वाग्ं उ॒त ।\\
अने॑शन्न॒स्येष॑व आ॒भुर॑स्य निष॒ङ्गथि॑: ॥ जठराय नमः॥ (23)\\
{\small\eng{Stomach}}\\
\\
हि॒र॒ण्य॒ग॒र्भः सम॑वर्त॒ताग्रे॑ भू॒तस्य॑ जा॒तः पति॒रेक॑ आसीत्।\\
स दा॑धार पृथि॒वीं द्यामु॒तेमां कस्मै॑ दे॒वाय॑ ह॒विषा॑ विधेम॥ \\
नाभ्यै नमः। (24)	\\
{\small\eng{Navel}}\\
\\
मीढु॑ष्टम॒ शिव॑तम शि॒वो न॑: सु॒मना॑ भव । प॒र॒मे वृ॒क्ष \\
आयु॑धन्नि॒धाय॒ कृत्तिं॒ वसा॑न॒ आच॑र॒ पिना॑कं॒ बिभ्र॒दाग॑हि ॥ \\
कट्यै नमः॥ (25)\\
{\small\eng{Waist}}\\
    \\
ये भू॒ताना॒मधि॑पतयो विशि॒खास॑: कप॒र्दिन॑: ॥\\
तेषाग्ं॑ सहस्रयोज॒नेऽव॒धन्वा॑नि तन्मसि ॥ गुह्याय नमः॥ (26)\\
{\small\eng{Upper Reproductive Organs}}\\
\\
ये अन्ने॑षु वि॒विध्य॑न्ति॒ पात्रे॑षु॒ पिब॑तो॒ जनान्॑ ।\\
तेषाग्ं॑ सहस्रयोज॒नेऽव॒धन्वा॑नि तन्मसि ॥ अण्डाभ्यां नमः॥ (27)\\
{\small\eng{Lower Reproductive Organs}}\\
\\
स शि॑रा जा॒तवे॑दाः। अ॒क्षरं॑ पर॒मं प॒दं। वे॒दाना॒ꣳ॒ शिर॑ उत्त॒मम्।\\
जातवे॑दसे॒ शिर॑सि मा॒ता ब्रह्म॒ भूर्भुव॒स्सुवरोम्‌॥ अपानाय नमः॥ (28)\\
{\small\eng{Anus}}\\
\\
मा नो॑ म॒हान्त॑मु॒त मा नो॑ अर्भ॒कं मा न॒ उक्ष॑न्तमु॒त मा न॑ उक्षि॒तम् ।\\
मा नो॑ऽवधीः पि॒तरं॒ मोत मा॒तरं॑ प्रि॒या मा न॑स्त॒नुवो॑ रुद्र रीरिषः ॥\\
ऊरुभ्यां नमः॥ (29)\\
{\small\eng{Thighs}}\\
\\
एष ते रुद्र भागस्तं जुषस्व तेनाऽवसेन परो \\
मूर्जवतोऽ तीह्य वतत धन्वा पिनाक हस्तः कृत्तिवासाः॥ \\
जानुभ्यां नमः॥ (30)\\
{\small\eng{Knees}}\\
\\
स॒ꣳ॒सृ॒ष्ट॒ जित् सो॑ म॒पा बा॑हु श॒र्ध्यू᳚र्ध्व धन्वा॒ प्रति॑हिता भि॒रस्ता᳚ ।\\
बृह॑स्पते॒ परि॑दीया॒ रथे॑न रक्षो॒हाऽमित्राꣳ॑ अप॒बा ध॑मानः॥ \\
जङ्घाभ्यां नमः॥ (31)\\
{\small\eng{Knees to Ankles}}\\
\\
विश्वं॑ भू॒तं भुव॑नं चि॒त्रं ब॑हु॒धा जा॒तं जाय॑मानं च॒ यत्।\\
सर्वो॒ ह्ये॑ष रु॒द्रस्तस्मै॑ रु॒द्राय॒ नमो॑ अस्तु ॥ गुल्फाभ्यां नमः॥ (32)\\
{\small\eng{Ankles}}\\
\\
ये प॒थां प॑थि॒रक्ष॑य ऐलबृ॒दा य॒व्युधः॑।\\
तेषाꣳ॑ सहस्रऽयोज॒नेऽव॒धन्वा॑नि तन्मसि ॥ पादाभ्यां नमः॥ (33)\\
{\small\eng{Feet}}\\
\\
अध्य॑वोचदधिव॒क्ता प्र॑थ॒मो दैव्यो॑ भि॒षक्।\\
अहीग्॑श्च॒ सर्वा᳚ञ्ज॒म्भय॒न्थ्सर्वा᳚श्च यातुधा॒न्यः ॥ कवचाय नमः॥ (34)\\
{\small\eng{Cross hands across chest touching shoulder}}\\
\\
नमो॑ बि॒ल्मिने॑ च कव॒चिने॑ च॒नमः॑ श्रु॒ताय॑ च श्रुतसे॒नाय॑ च ॥ \\
उपकवचाय नमः ॥ (34)\\
{\small\eng{kavacha at elbow level}}\\
\\
नमो॑ अस्तु॒ नील॑ग्रीवाय सहस्रा॒क्षाय॑ मी॒ढुषे᳚ ।\\
अथो ये अस्य सत्वांनोऽहं तेभ्योऽकरन्नमः॥ तृतीय नेत्राय नमः॥ (35)\\
{\small\eng{Index/Middle/Ring at eyes/middle o\eng{f} eyebrows}}\\
\\
प्रमु॑ञ्च॒ धन्व॑न॒स्त्वमु॒भयो॒रार्त्रि॑यो॒र्ज्याम्। \\
याश्च॑ ते॒ हस्त॒ इष॑वः॒ परा॒ ता भ॑गवो वप ॥ अस्त्राय नमः ॥ (36)\\
{\small\eng{Slap index/middle o\eng{f} right on left palm}}\\
\\
य ए॒ताव॑न्तश्च॒ भूयाꣳ॑ सश्च॒ दिशो॑ रु॒द्रा वि॑तस्थि॒रे।\\
तेषाꣳ॑ सहस्रऽयोज॒नेऽव॒धन्वा॑नि तन्मसि ॥ दिग्बन्धाय नमः॥ (37)\\
{\small\eng{Snap middle/thumb with click sound across self}}\\
\\
ओं नमो भगवते॑ रुद्रा॒य ॥\\
\subsection{\eng{Dashanga Nyasaha}}
\\
आं मूर्ध्ने नमः ॥ नं नासिकायै नमः॒ ॥ मों ललाटाय नमः ॥\\
भं मुखाय नमः ॥ गं कण्ठाय नमः ॥  वं हृदयाय नमः॥\\
तें दक्षिण हस्ताय नमः ॥  रुं वाम हस्ताय नमः ॥\\
द्रां नाभ्यै नमः ॥ यं पादाभ्यां नमः॒॥\\
\\
\subsection{\eng{Panchanga nyasaha}}
\\
स॒द्योजा॒तं प्र॑पद्या॒मि॒ स॒द्योजा॒ताय॒ वै नमो॒ नमः॑ ।\\
भ॒वे भ॑वे॒ नाति॑भवे भवस्व॒ माम् । भ॒वोद्भ॑वाय॒ नमः । पादाभ्यां नमः ॥\\
\\
वा॒म॒दे॒वाय॒ नमो᳚ ज्ये॒ष्ठाय॒ नमः॑  श्रे॒ष्ठाय॒ नमो॑ रु॒द्राय॒ नमः॒ \\
काला॑य नमः॒ कल॑ विकरणाय॒ नमो॒  बल॑ विकरणाय॒ नमो॒\\
बला॑य॒ नमो॒ बल॑प्रमथनाय॒ नमः॒ \\
सर्व॑ भूत दमनाय॒ नमो॑ म॒नोन्म॑नाय॒ नमः॒ ॥ ऊरुमध्यमाभ्यां नमः ॥\\
\\
अ॒घोरे᳚भ्योऽथ॒ घोरे᳚भ्यो॒ घोर॒घोर॑तरेभ्यः । \\
स॒र्वे᳚तः॑ सर्व॒ शर्वे᳚भ्यो॒ नम॑स्ते अस्तु रु॒द्र रू॑पेभ्यः\\
हृदयाय नमः ॥\\
\\
तत्पुरु॑षाय वि॒द्महे॑ महादे॒वाय॑ धीमहि । \\
तन्नो॑ रुद्रः प्रचो॒दया᳚त्  ॥ मुखाय नमः ॥\\
\\
ईशानः सर्व॑ विद्या॒ना॒ मीश्वरः सर्व॑भूता॒नां॒\\
ब्रह्माधि॑पति॒र्ब्रह्म॒णोऽधि॑पति॒ \\
र्ब्रह्मा॑ शि॒वो मे॑ अस्तु सदाशि॒वोम् ॥ मूर्ध्ने नमः ॥\\
\\
\subsection{\eng{Hamsa gayathri stotram}}
\\
अस्य श्री हंस गायत्री स्तोत्र महामन्त्रस्य। अव्यक्त पर ब्रह्म ऋषिः ।\\
अव्यक्त गायत्रि छन्दः। परम हंसो देवता । हंसां बीजं। हंसीं शक्तिः।\\
 हंसों कीलकं। परम हंस प्रसाद सिध्द्यर्थे जपे विनियोगः॥\\
\\
हंसां आङ्गुष्ठाभ्यां नमः ।\\
हंसीं तर्जनीभ्यां नमः ।\\
हंसूं मध्यमाभ्यां नमः ।\\
हंसैं अनामिकाभ्यां नमः ।\\
हंसौं कनिष्ठिकाभ्यां नमः ।\\
हंसः करतलकर पृष्ठाभ्यां नमः ॥\\
\\
हंसां हृदयाय नमः ।\\
हंसीं शिरसे स्वाहा।\\
हंसूं शिखायै वषट्।\\
हंसैं कवचायहम्।\\
हंसौं नेत्रत्रयाय वषट्।\\
हंसः अस्त्राय फट्।\\
\\
भूर्भुव॒ स्सुव॒रोमिति दिग्बन्धः ॥\\
\\
\subsection{\eng{Dhyanam}}
\\
ध्यानं।\\
गमा गमस्थं गमनादि शून्यं चिद्रूपदीपं तिमिरापहारम्।\\
पश्यामि ते सर्वजनान्तरस्थं नमामि हंसं परमात्मरूपम्॥\\
देहो देवालयः प्रोक्तो जीवो देवस्सनातनः।\\
त्यजे दज्ञान निर्माल्यं सोऽहं भावेन पूजयेत्॥\\
\\
हंसो हंसः परम हंसः  \\
हंसस् सोऽहं सोऽहं हंसः॥\\
\\
हं॒स॒ हंसा॒य॑ वि॒द्महे॑ परमहंसा॒य॑ धीमही।\\
तन्नो॑ हंसः प्रचो॒दया᳚त्॥\\
\\
हंस हंसेति योब्रूयाध् दं॑सो ना॑म स॒दाशि॑वः।\\
एवं न्यास विधिं॒ कृत्वा ततस् संपुट मारभेत्॥\\
\subsection{\eng{Dik Samputa Nyasaha}}
ॐ भूर्भुव॒स्सुव॒रों ।\\
ॐ लं । त्रातारमिंन्द्र॑ मवि॒तार॒मिन्द्रꣳ हवे॑ हवे सु॒हव॒ꣳ॒ शूर॒मिन्द्रम्᳚ । \\
हु॒वे नु श॒क्रं पु॑रहू॒त मिन्द्रग्ग्॑ स्व॒स्ति नो॑ म॒घवा॑ धा॒त्विन्द्रः॑ ॥\\
\\
लं भर्व धिग्बागे, इन्द्राय वज्रर्हस्ताय देवाधिपतये, ऐरावत वाहनाय - \\
सांगाय सायुधाय सशक्तिस परिवाराय -  उमामहेश्वर पार्षदाय नमः।\\
लं इन्द्राय नमः ।  पूर्व दिग्भागे, इन्द्रः सुप्रीतो  वरदो भवतु ॥  (1)\\
\\
\\
ॐ भूर्भुव॒स्सुव॒रों ।\\
\\
रं । त्वन्नो॑, अग्ने॒ वरु॑णस्य वि॒द्वान् दे॒वस्य॒ हेडोऽव॑ यासि सीष्ठाः ।\\
यजि॑ष्ठो॒  वह्नि॑ तम॒श् शोशु॑ चानो॒ विश्वा॒, द्वेषाꣳ॑सि॒प् प्रमु॑ मुग् घ्य॒स् मत् ॥ \\
\\
रं आग् नेय धिग्बागे, अग्नये शक्ति हस्ताय तेजोऽधि पतयेऽ, अज वाहनाय\\
सांगाय सायुधाय सशक्तिस परिवाराय -  उमामहेश्वर पार्षदाय नमः।\\
रं अग्नये नमः । आग्नेय दिग्भागे अग्निः सुप्रीतो  वरदो भवतु ॥ (2)\\
\\
ॐ भूर्भुव॒स्सुव॒रों ।\\
हं । सु॒गन्नः॒ पन्था॒ मभ॑यं कृणोतु । यस्मि॒न् नक्ष॑त्रे य॒म एति॒ राजा᳚ ।\\
यस्मि॑न् नेन म॒भ्य षिं॑ चन्त दे॒वाः । तद॑स्य चि॒त्रꣳ ह॒विषा॑ यजाम ॥\\
\\
हं दक्षिण धिग्बागे यमाय दण्ड हस्ताय धर्माधि पतये महिष वाहनाय\\
सांगाय सायुधाय सशक्तिस परिवाराय -  उमामहेश्वर पार्षदाय नमः।\\
हं यमाय नमः । दक्षिण दिग्भागे यमः सुप्रीतो  वरदो भवतु ॥ (3)\\
\\
ॐ भूर्भुव॒स्सुव॒रों ।\\
षं । असु॑न्वन्त मय॑जमान मिच् छस् ते॒ नस् ये॒त् यान् तस्क॑रस् यान् वे॑षि।\\
अ॒न्य म॒स्म दि॑च्छ॒ सात॑ इ॒त्या नमो॑ देवि-निर्ऋते॒ तुभ्य॑ मस्तु ॥\\
\\
षं निर्ऋति दिग्भागे निर्ऋतये खड्ग हस्ताय रक्षो धिपतये नर वाहनाय\\
सांगाय सायुधाय सशक्तिस परिवाराय -  उमामहेश्वर पार्षदाय नमः।\\
षं निर्ऋतये नमः । निर्ऋति दिग्भागे निर्ऋतिस्सुप्रीतो वरदो भवतु ॥ (4)\\
\\
ॐ भूर्भुव॒स्सुव॒रों ।\\
वं । तत्वा॑ यामि॒ ब्रह्म॑णा॒ वन्द॑ मा॒नस् तदा शा᳚स्ते॒ यज॑मानो ह॒विर्भिः॑ ।\\
अहे॑डमानो वरुणे॒ हबो॒ध् युरु॑शꣳ स॒मान॒ आयुः॒ प्रमो॑षीः ॥\\
\\
वं पश्चिम दिग्भागे वरुणाय पाश हस्ताय जलाधि पतये मकर वाहनाय\\
सांगाय सायुधाय सशक्तिस परिवाराय -  उमामहेश्वर पार्षदाय नमः।\\
वं वरुणाय नमः । पश्चिम दिग्भागे वरुणः सुप्रीतो वरदो भवतु ॥ (5)\\
\\
ॐ भूर्भुव॒स्सुव॒रों ।\\
यं । आ नो॑ नि॒युद् भि॑श् श॒तिनी॑ भिरध् व॒रम् । \\
स॒ह॒स् रिणी॑ भि॒रुप॑ याहि य॒ज्ञम् ।\\
वायो॑, अ॒स्मिन् ह॒विषि॑ मादयस्व । यू॒यं पा॑तस् स्व॒स्ति भि॒स् सदा॑ नः॥\\
\\
यं वायव्य दिग्भागे वायवे सांकु शध् वज हस्ताय प्राणाधि पतये मृग वाहनाय\\
सांगाय सायुधाय सशक्तिस परिवाराय -  उमामहेश्वर पार्षदाय नमः।\\
यं वायवे नमः । वायव्य दिग्भागे  वायुः सुप्रीतो  वरदो भवतु ॥ (6)\\
\\
ॐ भूर्भुव॒स्सुव॒रों ।\\
सं । व॒यꣳ सो॑ मव् व्र॒ते तव॑ । मन॑स्त॒ नू षु॒बिभ् र॑तः । प्र॒जा व॑न्तो, अशीमहि ।\\
\\
सं उत्तर दिग्भागे सोमाय अमृत कलश हस्ताय नक्षत्राधिपतये, अश्व वाहनाय\\
सांगाय सायुधाय सशक्तिस परिवाराय -  उमामहेश्वर पार्षदाय नमः।\\
सं सोमाय नमः । उत्तर दिग्भागे  सोमः सुप्रीतो वरदो भवतु ॥ (7)\\
\\
ॐ भूर्भुव॒स्सुव॒रों ।\\
शं । तमी शा᳚न॒ञ् जग॑तस् त॒स्थु ष॒स्पतिम्᳚ धि॒यं॒ जि॒न्व मव॑से हूमहे व॒यम् ।\\
पू॒षा नो॒ यथा॒ वेद॑सा॒ मस॑द् वृ॒धे र॒क्षि॒ता पायु॒र द॑ब् शस् स्व॒स् तये᳚ ॥\\
\\
शं ईशान दिग्भागे, ईशानाय त्रिशूल हस्ताय भूताधि पतये वृषभ वाहनाय\\
सांगाय सायुधाय सशक्तिस परिवाराय -  उमामहेश्वर पार्षदाय नमः।\\
शं ईशानाय नमः  ईशान्य दिग्भागे, ईशानः सुप्रीतो  वरदो भवतु ॥ (8)\\
\\
ॐ भूर्भुव॒स्सुव॒रों ।\\
खं । अ॒स्मे रु॒द्रा मे॒हना॒ पर्व॑ तासो वृत्र॒ हत्ये॒ भर॑ हूतौ स॒जोषाः᳚ ॥\\
यश् शंस॑ते स्तुव॒ते धायि॑ प॒ज्र इन्द्र॑ज् ज्येष्ठा, अ॒स्माम् अ॑वन्तु दे॒वाः ॥\\
\\
खं ऊर्ध्व दिग्भागे ब्रह्मणे पद्महस्ताय प्रजाधिपतये हंस वाहनाय\\
सांगाय सायुधाय सशक्तिस परिवाराय -  उमामहेश्वर पार्षदाय नमः।\\
खं ब्रह्मणे नमः । ऊर्ध्व दिग्भागे ब्रह्मा सुप्रीतो  वरदो भवतु ॥  (9)\\
\\
ॐ भूर्भुव॒स्सुव॒रों ।\\
ह्रीं । स्यो॒ना पृ॑थि विभवा॑ऽ नृक्ष॒रा नि॒वे श॑नि । यच्छा॑ न॒श् शर्म॑ स॒प्रथाः᳚ ।\\
\\
ह्रीं अधो दिग्भागे विष्णवे चक्र हस्ताय लोखाधिपतये गरुड वाहनाय\\
सांगाय सायुधाय सशक्तिस परिवाराय -  उमामहेश्वर पार्षदाय नमः।\\
ह्रीं विष्णवे नमः । अधो दिग्भागे वष्णुस्सुप्रीतो वरदो भवतु ॥  (10)\\
\\
\subsection{\eng{Shodashanga Roudri Karanam}}
\\
वि॒भू॑रसिप् प्र॒वाह॑ णो॒ रौद्रे॒णानी॑केन पा॒हि माऽ᳚ग्ने पिपृ॒हि मा॒ मा मा॑ हिꣳसीः ॥\\
\\
वह्नि॑रसि हव्य॒ वाह॑नो॒ रौद्रे॒णानी॑केन पा॒हि माऽ᳚ग्ने पिपृ॒हि मा॒ मा मा॑ हिꣳसीः ॥\\
\\
श्वा॒ त्रो॑सि॒ प्रचे॑ता॒ रौद्रे॒णानी॑केन पा॒हि माऽ᳚ग्ने पिपृ॒हि मा॒ मा मा॑ हिꣳसीः ॥\\
\\
तु॒थो॑सि वि॒श्व वे॑दा॒ रौद्रे॒णानी॑केन पा॒हि माऽ᳚ग्ने पिपृ॒हि मा॒ मा मा॑ हिꣳसीः ॥\\
\\
उ॒शिग॑ सिक॒वी रौद्रे॒णानी॑केन पा॒हि माऽ᳚ग्ने पिपृ॒हि मा॒ मा मा॑ हिꣳसीः ॥  (5)\\
\\
अंघा॑रि रसि॒ बंभा॑री॒ रौद्रे॒णानी॑केन पा॒हि माऽ᳚ग्ने पिपृ॒हि मा॒ मा मा॑ हिꣳसीः ॥\\
\\
अ॒व॒स् युर॑सि॒ दुव॑स् वा॒न् रौद्रे॒णानी॑केन पा॒हि माऽ᳚ग्ने पिपृ॒हि मा॒ मा मा॑ हिꣳसीः ॥\\
\\
शु॒न्ध्यू र॑सि मार् जा॒लीयो॒ रौद्रे॒णानी॑केन पा॒हि माऽ᳚ग्ने पिपृ॒हि मा॒ मा मा॑ हिꣳसीः ॥\\
\\
सं॒म्राड॑सि कृ॒शा नू॒ रौद्रे॒णानी॑केन पा॒हि माऽ᳚ग्ने पिपृ॒हि मा॒ मा मा॑ हिꣳसीः ॥\\
\\
प॒रि॒ षद्यो॑सि॒ पव॑ मानो॒ रौद्रे॒णानी॑केन पा॒हि माऽ᳚ग्ने पिपृ॒हि मा॒ मा मा॑ हिꣳसीः ॥ (10)\\
\\
प्र॒तक् वा॑सि॒ नभ॑स् वा॒न् रौद्रे॒णानी॑केन पा॒हि माऽ᳚ग्ने पिपृ॒हि मा॒ मा मा॑ हिꣳसीः ॥\\
\\
असं॑ मृष् टोसि हव्य॒ सूदो॒ रौद्रे॒णानी॑केन पा॒हि माऽ᳚ग्ने पिपृ॒हि मा॒ मा मा॑ हिꣳसीः ॥\\
\\
ऋ॒त धा॑मासि॒ सुव॑र् ज्योती॒ रौद्रे॒णानी॑केन पा॒हि माऽ᳚ग्ने पिपृ॒हि मा॒ मा मा॑ हिꣳसीः ॥\\
\\
ब्रह्म॑ ज्योति रसि॒ सुव॑र् धामा॒ रौद्रे॒णानी॑केन पा॒हि माऽ᳚ग्ने पिपृ॒हि मा॒ मा मा॑ हिꣳसीः ॥\\
\\
अ॒जो᳚स् येकपा॒द् रौद्रे॒णानी॑केन पा॒हि माऽ᳚ग्ने पिपृ॒हि मा॒ मा मा॑ हिꣳसीः ॥ (15)\\
\\
अहि॑रसि बु॒ध् नियो॒ रौद्रे॒णानी॑केन पा॒हि माऽ᳚ग्ने पिपृ॒हि मा॒ मा मा॑ हिꣳसीः ॥\\
\\
त्व गस् थिग तैः सर्व पापैः प्रमुच्यते । सर्व भूतेष्व परा जितो भवति ।\\
ततो भूतप् प्रेत पिशाचब् ब्रह्मराक्ष सयक्ष यमदूत शाकिनी डाकिनी \\
सर्पश् श्वापद तस्करद् ज्वरा द्युपद् रवो पघाताः । \\
सर्वे ज्वलन्तं पश्यन्तु । मां रक्षन्तु ।\\
यजमानं  सक कुटुम्बं सर्वे बक्त महाजनानाम्  च  रक्षन्तु ॥\\
\\
\subsection{\eng{Guhyadhi Mastakaantam Shadanganyasam}}
\\
मनो॒ ज्योति॑र् जुष ता॒माज्यं॒ विच् छि॑न्नं य॒ज्ञꣳ समि॒मंद॑ धातु। \\
बृह॒स्पति॑ स्तनुता मि॒मंनो॒ विश्वे॑ दे॒वा, इ॒ह मा॑द यन्ताम्॥ गुह्याय नमः॥\\
\\
अबो᳚ध् य॒ग् निस् स॒मिधा जना॑नां॒ प्र॑ति धे॒नु मि॑वा य॒ती मु॒षासम्᳚। \\
य॒ह्वा, इ॑व॒प् प्र॒वया मु॒ज्जि हा॑नाः॒ प्रभा॒ नव॑स् सिस् रते॒ नाक॒ मच्छा॑॥ नाभ्यै नमः॥\\
\\
अ॒ग्निर् मू॒र्धा दि॒वः क॒कुत् पतिः॑ पृथि॒व्या, अ॒यम्। \\
अ॒पाꣳ रेताꣳ॑ सि जिन्वति । हृदयाय नमः ॥\\
\\
मू॒र्धानं॑ दि॒वो, अ॑र॒तिं पृ॑थि॒व्या वै᳚श्वान॒र मृ॒ताय॑ जा॒त म॒ग्निम्‌। \\
क॒विꣳ स॒म्राज॒ मति॑थिं॒ जना॑ना मा॒सन्ना पात्रं॑ जन यन्त दे॒वाः॥ कण्ठाय नमः॥ \\
\\
मर्मा॑णि ते॒ वर्म॑भिश्छादयामि॒ सोम॑स्त्वा॒ राजा॒ऽमृते॑ना॒भिऽव॑स्ताम्।\\
उ॒रोर्वरी॑यो॒ वरि॑वस्ते अस्तु॒ जय॑न्तं त्वामनु॑ मदन्तु दे॒वाः॥ मुखाय नमः।\\
\\
जा॒तवे॑दा॒ यदि॑ वा पाव॒कोऽसि॑। वै॒श्वा॒न॒रो यदि॑ वा वैद् युतोसि॑। \\
शं प्र॒जाभ्यो॒ यज॑मानाय लो॒कम्। ऊर्जं॒ पु॒ष् तिं दद॑ द॒भ्याव॑ वृथ् स्व॥ शिरसे नमः॥\\
\\
\subsection{\eng{Atma Rakshaha}}
\\
ब्रह्मा᳚त् म॒न् वद॑ सृजत। तद॑ कामयत। समा॒त् मना॑ पद् ये॒येति॑। \\
\\
आत् म॒न्नात् म॒न् नित्याम॑न् त्रयत। तस्मै॑ दश॒मꣳ हू॒तः\eng{f} प्रत्य॑श्रुणोत्।\\
स दश॑ हूतोऽभवत्। दश॑ हूतो ह॒ वै ना मै॒षः। \\
तं वा, ए॒तन् दश॑हू त॒ꣳ सन्तम्᳚। दश॑हो॒तेत् याच॑क्षते प॒रोक्षे॑ण। प॒रोक्ष॑ प्रिया, इव॒ हि दे॒वाः॥\\
\\
आत् म॒न्नात् म॒न् नित्याम॑न् त्रयत। तस्मै॑ सप्त॒मꣳ हूतः\eng{f} प्रत्य॑श्रुणोत्। \\
स स॒प्त हू॑तोऽभवत्। स॒प्त हू॑तो ह॒ वै ना मै॒षः। \\
तं वा, ए॒तꣳ स॒प्तहू॑ त॒ꣳ सन्तम्᳚। स॒प्तहो॒तेत् याच॑क्षते प॒रोक्षे॑ण। प॒रोक्ष॑ प्रिया, इव॒ हि दे॒वाः॥\\
\\
आत् म॒न्नात् म॒न् नित्याम॑न् त्रयत। तस्मै॑ ष॒ष्ठꣳ हू॒तः\eng{f} प्रत्य॑श्रुणोत्। \\
स षड् ढू॑तोऽ भवत्। षड् ढू॑तो ह वै नामै॒षः। \\
तं वा, ए॒तꣳ षड्ढू॒॑त॒ꣳ॒ सन्तम्᳚। षड्ढूो॒तेत् याच॑क्षते प॒रोक्षे॑ण। प॒रोक्ष प्रिया, इव॒ हि दे॒वाः॥\\
\\
आत् म॒न्नात् म॒न् नित्याम॑न् त्रयत। तस्मै॑ पञ्च॒मꣳ हू॒तः\eng{f} प्रत्य॑श्रुणोत्। \\
स पञ्च॑ हूतोऽभवत्। पञ्च॑ हूतो ह॒ वै नामै॒षः। \\
तं वा, ए॒तं पञ्च॑हूत॒ꣳ॒ सन्तम्᳚। पञ्च॑हो॒तेत् याच॑क्षते प॒रोक्षे॑ण। प॒रोक्ष॑ प्रिया, इव॒ हि दे॒वाः॥\\
\\
आत् म॒न्नात् म॒न् नित्याम॑न् त्रयत। तस्मै॑ चतु॒र्थꣳ हू॒तः\eng{f} प्रत्य॑श्रुणोत्। \\
स चतु॑र् हूतोऽभवत् । चतु॑र् हूतो ह॒ वै नामै॒षः। \\
तं वा,  ए॒तं चतु॑र् हू॒त॒ꣳ॒ सन्तम्᳚। चतु॑र् होतेत् याच॑क्षते प॒रोक्षे॑ण। प॒रोक्ष॑ प्रिया, इव॒ हि दे॒वाः॥\\
\\
तम॑ब् ब्रवीत्। त्वं वै मे॒ नेदि॑ष्ठꣳ हूतः प्रत्य॑श् श्रौषीः। \\
त्वयै॑ नानाख् या॒तार॒ इति॑। तस्मा॒न्नु है॑ना॒हु॒श् चतु॑र् होतार॒ इत्या च॑क्षते। \\
तस्मा᳚च् छुश् रू॒षुः पु॒त्रा णा॒ꣳ॒ हृद् य॑तमः। ने दि॑ष्ठो॒ हृद् य॑तमः॒।\\
नेदि॑ष्ठो॒ ब्रह्म॑णो भवति। य ए॒वं वेद॑॥ आत्मने नमः ।\\
\\
\subsection{\eng{Shiva Sankalpam}}
\\
येने॒दं भू॒तं भुव॑नं भवि॒श्यत् परि॑-गृही तम॒ मृते॑ न॒सर्वम्᳚॥ \\
ये॑न य॒ज्ञस् त्रा॑यते॑ स॒प्त हो॑ता॒ तन्मे॒ मनः॑ शि॒वसं॑क॒ल्पम॑स्तु॥ (1)\\
\\
येन॒ कर्मा॑णिप् प्र॒चर॑न्ति॒ धीरा॒ यतो॑ वा॒चा मन॑सा चारु॒ यन्ति॑। \\
यत्सं मि॑तं॒ मनः॑ संचर॑न्ति॒\\
{\small यत्सं॒ मित॒ मनु॑सं॒यन्ति॑प् प्रा॒णि न॒स्} तन्मे॒ मनः॑ शि॒वसं॑क॒ल्पम॑स्तु॥ (2)\\
\\
येन॒ कर्मा᳚ण्य॒ पसो॑ मनी॒षिणो॑ य॒ज्ञे कु॑ण्वन्ति वि॒दथे॑षु॒ धीराः᳚। \\
यद॑ पू॒र्वय् य॒क्ष मन्तं॑ प्र॒जानां॒ तन्मे॒ मनः॑ शि॒वसं॑क॒ल्पम॑स्तु॥ (3)\\
\\
यत् प्र॒ज्ञान॑ मु॒त चेतो॒ धृति॑श्च॒ यज् ज्योति॑-र॒न्त र॒मृतं॑ प्र॒जासु॑।\\
यस्मा॒न्न ऋ॒ते कि॑ञ्-च॒न कर्म॑क् क्रि॒यते॒ तन्मे॒ मनः॑ शि॒वसं॑क॒ल्पम॑स्तु॥ (4)\\
\\
सु॒षा॒ र॒थि रश्वा॑ निव॒यं म॑नु॒ष्या᳚न् मे॒नि॒युते॑ प॒शुभि॑र् वा॒जिनी॑ वान्। \\
हृ॒त् प्र॒वि॒ष् ठय् यद च॑र॒य् यविष्ठं॒ तन्मे॒ मनः॑ शि॒वसं॑क॒ल्पम॑स्तु॥ (5)\\
\\
यस्मि॒न् नृच॒स् साम॒ यजूꣳ॑ षि॒ यस्मि॑न् प्रति॒ष्ठा र॑श॒ना भावि॒ भाराः᳚। \\
यस् मिग्ग्श् चि॒त्तꣳ सर्व॒ मोतं॑ प्र॒जानां॒ तन्मे॒ मनः॑ शि॒वसं॑क॒ल्पम॑स्तु॥ (6)\\
 \\
यदत्र॑ ष॒ष्ठं त्रि॒शतꣳ॑ सु॒वीर्यं॑ य॒ज्ञस्य॑ गुह्यं॒ नव॑ना व॒माय्यं᳚। \\
दश॑ पञ्चत् त्रि॒ꣳ॒ शत॒य् यत् परं॒ तन्मे॒ मनः॑ शि॒वसं॑क॒ल्पम॑स्तु॥ (7)\\
\\
यज् जाग्र॑तो दू॒र मु॒दैतु॒ सर्वं॒ तत्-सु॒प्-तस्य॒-तथै॒ वेति॑।\\
दू॒रं॒ग॒ मं ज्योति॑ षां॒ ज्यो॒ति रेकं॒ तन्मे॒ मनः॑ शि॒वसं॑क॒ल्पम॑स्तु॥ (8)\\
\\
येने॒दं विश्वं॒ जग॑तो ब॒भूव॒ ये दे॒वापि॑ मह॒तो जा॒तवे॑दाः।\\
तदे॒वाग् निस् तद् वा॒युस् तत् सूर्य॒स् तदु॑-च॒न्द्र मा॒स्तन्मे॒ मनः॑\\
 शि॒वसं॑क॒ल्पम॑स्तु॥ (9)\\
\\
येन॒द् द्यौः पृ॑थि॒वी चा॒न् तरि॑क्षं च॒ ये पर्व॑ताः प्र॒दिशो॒ दिश॑श्च। \\
येने॒दं जग॒द् ध्याप्तं॑ प्र॒जानां॒ तन्मे॒ मनः॑ शि॒वसं॑क॒ल्पम॑स्तु॥ (10)\\
\\
ये मनो॒ हृद॑ यय्ये च॑ दे॒वा ये दि॒व्या, आपो॒ ये सूर्य॑ र॒श् मिः। \\
ते श्रोत्रे॒ चक्षु॑षी सं॒चर॑न् तन्॒ तन्मे॒ मनः॑ शि॒वसं॑क॒ल्पम॑स्तु॥ (11)\\
\\
अचि॑न् त्यञ्॒चा प्र॑मे यंचव् व्य॒क्ता॒, व्यक्त॑ परं॒ चय॑त्। \\
सूक्ष्मा᳚त् सूक्ष्म त॑रन्, ज्ञे॒यं तन्मे॒ मनः॑ शि॒वसं॑क॒ल्पम॑स्तु॥ (12)\\
\\
एका॑ च द॒श च॑ श॒तं च॑ स॒हस्र॑ञ् चा॒ यु॑तञ्च। \\
नि॒यु तं॑च प्र॒यु त॒ञ्चार् बु॑दञ् च॒न् य॑र् बुदंच \\
तन्मे॒ मनः॑ शि॒वसं॑क॒ल्पम॑स्तु॥ (13)\\
\\
ये प॑ञ्च पञ्चा॒द॒श श॒तꣳ स॒हस्र॑ म॒युतं॒ न्य॑र् बुदं च। \\
ए अ॑ग्नि चि॒त्तेष् ट॑का॒स्ताꣳ शरी॑रं॒ तन्मे॒ मनः॑ शि॒वसं॑क॒ल्पम॑स्तु ॥ (14)\\
\\
वेदा॒हमे॒तं पुरु॑षं म॒हान्त॑ मादि॒त्यव॑र्णं॒ तम॑सः॒ पर॑स्तात्। \\
यस्य॒ योनिं॒ परि॒ पश्य॑न्ति॒ धीरा॒स्तन्मे॒ मनः॑ शि॒वसं॑क॒ल्पम॑स्तु॥ (15)\\
\\
यस्यैतं धीराः᳚ पु॒नन्ति॑ क॒वयो᳚ ब्र॒ह्माण॑ मे॒तं त्वा॑ वृणुत॒ मिन्दुं᳚। \\
स्था॒व॒रं जङ्ग॑मं॒ द्यौरा॑का॒शं॒ तन्मे॒ मनः॑ शि॒वसं॑क॒ल्पम॑स्तु॥ (16)\\
\\
परा᳚त् प॒रत॑रं ब्र॒ह्म॒ त॒त् परा᳚त् पर॒तो ह॑रिः। \\
य॒त् परा᳚त् पर॑तोऽ धी॒शं॒ तन्मे॒ मनः॑ शि॒वसं॑क॒ल्पम॑स्तु॥ (17)\\
\\
परा᳚त् प॒रत॑रं चैव त॒त् परा᳚च् चैव॒ यत् प॑रम्। \\
य॒त् परा᳚त् पर॑तो ज्ञे॒यं॒ तन्मे॒ मनः॑ शि॒वसं॑क॒ल्पम॑स्तु॥ (18)\\
\\
या वेदा दिषु॑-गाय॒त्री स॒र्वव् व्या॑पी म॒हेश्व॑री। \\
ऋग् य॑जु॒स् सामा॑ थर् वै॒श्च॒ तन्मे॒ मनः॑ शि॒वसं॑क॒ल्पम॑स्तु॥ (19)\\
\\
यो वै॑ दे॒वं म॑हादे॒वं॒ प्र॒य॒तः प्र॑णत॒श् शु॑चिः। \\
यस्सर्वे॑ सर्व॑ वेदै॒श्च तन्मे॒ मनः॑ शि॒वसं॑क॒ल्पम॑स्तु॥ (20)\\
 \\
प्र॒य॒तः॒ प्रण॑वोंका॒रं प्र॒णवं॑ पुरु॒षोत्त॑मम्। \\
ओंका॑रं॒ प्रण॑वात्मा॒नं तन्मे॒ मनः॑ शि॒वसं॑क॒ल्पम॑स्तु॥ (21)\\
 \\
योऽसौ॑ स॒र्वेषु॑ वेदे॒षु प॒ठ्यते᳚ ह्यय॒ मीश्व॑र:। \\
अकायो॑ निर्गु॑णो ह्या॒त्मा तन्मे॒ मनः॑ शि॒वसं॑क॒ल्पम॑स्तु॥ (22)\\
\\
गोभि॒र्जुष्टं॒ धने॑न॒ह् ह्यायु॑षा च॒ बले॑ नच। \\
प्र॒जया॑ प॒शुभिः॑ पुष्करा॒क्षं तन्मे॒ मनः॑ शि॒वसं॑क॒ल्पम॑स्तु॥ (23)\\
  \\
त्र्यं॑बकं यजामहे सुग॒न्धिं पु॑ष्ति॒वर्ध॑नम्। उ॒र्वा॒रु॒कमि॑व॒ \\
बन्ध॑नान्मृ॒त्योर्मु॑क्षीय॒ माऽमृता॒ तन्मे॒ मनः॑ शि॒वसं॑क॒ल्पम॑स्तु॥ (24)\\
\\
कैला॑स॒ शिख॑रे र॒म्ये॒ शङ्कर॑स्य शि॒वाल॑ये। \\
दे॒वता᳚स् तत्र॑ मोदन्ति॒ तन्मे॒ मनः॑ शि॒वसं॑क॒ल्पम॑स्तु॥ (25)\\
\\
कैला॑स॒ शिखरा वा॒सं हि॒मव॑द् गिरि॒ संस्थिथं । \\
नी॒ल॒क॒ण्ठं त्रि॑नेत्रं च तन्मे॒ मनः॑ शि॒वसं॑क॒ल्पम॑स्तु॥ (26)\\
\\
वि॒श्व त॑श् चक्षुरु॒त वि॒श्व तो॑मुखो वि॒श्व तो॑हस्त उ॒त वि॒श्व त॑स्पात्।\\
सं बा॒हुभ्यां॒ नम॑ति॒-सं-पत॑त् त्रै॒र् द्यावा॑ पृथि॒वी ज॒नय॑न् दे॒व\\
 एक॒स्तन्मे॒ मनः॑ शि॒वसं॑क॒ल्पम॑स्तु॥ (27)\\
\\
चतुरो॑ वे॒दा न॑धीयी॒त स॒र्व शा᳚स् त्रम॒यं वि॑दुः।  \\
इ॒ति॒हा॒स पु॑राणा॒नां॒ तन्मे॒ मनः॑ शि॒वसं॑क॒ल्पम॑स्तु॥ (28)\\
\\
मा नो॑ म॒हान्त॑मु॒त मा नो॑, अर्भ॒कं मा न॒ उक्ष॑न्तमु॒त मा न॑ उक्षि॒तम्। \\
मा नो॑ऽवधीः पि॒तरं॒ मोत मा॒तरं॑ प्रि॒या मा न॑स्त॒नुवो॑ \\
रुद्र रीरिष॒स्तन्मे॒ मनः॑ शि॒वसं॑क॒ल्पम॑स्तु॥  (29)\\
\\
मान॑स्तो॒के तन॑ये॒ मा न॒ आयु॑षि॒ मा नो॒ गोषु॒ मा नो॒, अश्वे॑षु रीरिषः। \\
वी॒रान्मा नो॑ रुद्र भामि॒तो ऽव॑धीर् ह॒विष्म॑न्तो॒ नम॑सा \\
विधेम ते॒ तन्मे॒ मनः॑ शि॒वसं॑क॒ल्पम॑स्तु॥ (30)\\
\\
ऋ॒तꣳ स॒त्यं प॑रं ब्र॒ह्म॒ पु॒रुषं॑ कृष्ण॒पिङ्ग॑लम्। \\
ऊ॒र्ध्वरे॑तंवि॑रूपा॒क्षं॒ वि॒श्वरू॑पाय॒ वै नमो॒ नम॒स्तन्मे॒ मनः॑ शि॒वसं॑क॒ल्पम॑स्तु॥ (31)\\
\\
कद्रु॒द्राय॒ प्रचे॑तसे मी॒ढुष्ट॑माय॒ तव्य॑से। \\
वो॒चेम॒ शंत॑मꣳ हृ॒दे। सर्वो॒ ह्ये॑ष रु॒द्रस्तस्मै॑ रु॒द्राय॒ नमो॑ अस्तु॒ \\
तन्मे॒ मनः॑ शि॒वसं॑क॒ल्पम॑स्तु॥ (32)\\
\\
ब्रह्म॑ जज्ञा॒नं प्र॑थ॒मं पु॒रस्ता॒द् विसी॑ म॒तस् सु॒रुचो॑ वे॒न आ॑वः। \\
स बु॒ध्निया॑, उप॒मा, अ॑स्य वि॒ष्ठास् स॒तश्च॒ योनि॒मस॑तश्च॒ \\
विव॒स् तन्मे॒ मनः॑ शि॒वसं॑क॒ल्पम॑स्तु॥ (33)\\
\\
यः प्रा॑ण॒तो नि॑मिष॒तो म॑हि॒त्वै॒क इद्राजा॒ जग॑तो ब॒भूव॑।\\
य ईशे॑, अ॒स्यद् द्वि॒पद॒श्चतु॑ष्पदः॒ कस्मै॑ दे॒वाय॑ ह॒विषा॑ विधेम॒ \\
तन्मे॒ मनः॑ शि॒वसं॑क॒ल्पम॑स्तु॥ (34)\\
\\
य आ᳚त् म॒दा ब॑ल॒दा यस्य॒ विश्व॑ उ॒पास॑ते प्र॒शिषं॒ यस्य॑ दे॒वाः। \\
यस्य॑ छायाऽमृतं यस्य॑ मृ॒त्युः कस्मै॑ दे॒वाय॑ ह॒विषा॑ विधेम॒ \\
तन्मे॒ मनः॑ शि॒वसं॑क॒ल्पम॑स्तु॥ (35)\\
\\
यो रु॒द्रो, अ॒ग्नौ यो, अ॒प्सु य ओष॑धीषु॒ यो रु॒द्रो विश्वा॒ \\
भुव॑नाऽऽवि॒वेश॒ तस्मै॑ रु॒द्राय॒ नमो॑, अस्तु॒ तन्मे॒ \\
मनः॑ शि॒वसं॑क॒ल्पम॑स्तु॥ (36)\\
\\
ग॒न्ध॒द्वा॒रां दु॑राध॒र्षां॒ नि॒त्यपु॑ष्टां करी॒षिणी᳚म्। \\
ई॒श्वरीꣳ॑ सर्व॑भूता॒नां॒ त्वामि॒होप॑ह्रये॒ श्रियं॒  \\
तन्मे॒ मनः॑ शि॒वसं॑क॒ल्पम॑स्तु॥ (37)\\
\\
नमकं॑ चम॑कं चै॒व पु॒रुषसू᳚क्तं च॒ यद् विदुः। \\
महादेवं च तत्तुल्यं॒ तन्मे॒ मनः॑ शि॒वसं॑क॒ल्पम॑स्तु॥ (38)\\
\\
य इ॒दꣳ शिव॑संक॒ल्प॒ꣳ॒ स॒दा ध्या॑यन्ति॒ब् ब्राह्म॑णाः। \\
ते परं॒ मोक्षं॑ गमिष्यन्ति॒ तन्मे॒ मनः॑ शि॒वसं॑क॒ल्पम॑स्तु ॥ (39)\\
हृदयाय नमः।\\
\\
\subsection{\eng{Purusha Suktam}}
\\
स॒हस्र॑शीर्षा॒ पुरु॑षः । स॒ह॒स्रा॒क्षः स॒हस्र॑पात् ।\\
स भूमिं॑ वि॒श्वतो॑ वृ॒त्वा । अत्य॑तिष्ठद्दशाङ्गु॒लम् ।\\
पुरु॑ष ए॒वेदग्ं सर्वम्᳚ । यद्भू॒तं यच्च॒ भव्यम्᳚ ।\\
उ॒तामृ॑त॒त्वस्येशा॑नः । य॒दन्ने॑नाति॒रोह॑ति ।\\
ए॒तावा॑नस्य महि॒मा ।\\
अतो॒ ज्यायाग्॑श्च॒ पूरु॑षः ॥ १ ॥\\
\\
पादो᳚ऽस्य॒ विश्वा॑ भू॒तानि॑ । त्रि॒पाद॑स्या॒मृतं॑ दि॒वि ।\\
त्रि॒पादू॒र्ध्व उदै॒त्पुरु॑षः ।\\
पादो᳚ऽस्ये॒हाऽऽभ॑वा॒त्पुन॑: ।\\
ततो॒ विष्व॒ङ्व्य॑क्रामत् ।\\
सा॒श॒ना॒न॒श॒ने अ॒भि । तस्मा᳚द्वि॒राड॑जायत ।\\
वि॒राजो॒ अधि॒ पूरु॑षः । स जा॒तो अत्य॑रिच्यत ।\\
प॒श्चाद्भूमि॒मथो॑ पु॒रः ॥ २ ॥\\
\\
यत्पुरु॑षेण ह॒विषा᳚ । दे॒वा य॒ज्ञमत॑न्वत ।\\
व॒स॒न्तो अ॑स्यासी॒दाज्यम्᳚ । ग्री॒ष्म इ॒ध्मश्श॒रद्ध॒विः ।\\
स॒प्तास्या॑सन्परि॒धय॑: । त्रिः स॒प्त स॒मिध॑: कृ॒ताः ।\\
दे॒वा यद्य॒ज्ञं त॑न्वा॒नाः ।\\
अब॑ध्न॒न्पुरु॑षं प॒शुम् ।\\
तं य॒ज्ञं ब॒र्हिषि॒ प्रौक्षन्॑ ।\\
पुरु॑षं जा॒तम॑ग्र॒तः ॥ ३ ॥\\
\\
तेन॑ दे॒वा अय॑जन्त । सा॒ध्या ऋष॑यश्च॒ ये ।\\
तस्मा᳚द्य॒ज्ञात्स॑र्व॒हुत॑: । सम्भृ॑तं पृषदा॒ज्यम् ।\\
प॒शूग्‍स्ताग्‍श्च॑क्रे वाय॒व्यान्॑ । आ॒र॒ण्यान्ग्रा॒म्याश्च॒ ये ।\\
तस्मा᳚द्य॒ज्ञात्स॑र्व॒हुत॑: । ऋच॒: सामा॑नि जज्ञिरे ।\\
छन्दाग्ं॑सि जज्ञिरे॒ तस्मा᳚त् । यजु॒स्तस्मा॑दजायत ॥ ४ ॥\\
\\
तस्मा॒दश्वा॑ अजायन्त । ये के चो॑भ॒याद॑तः ।\\
गावो॑ ह जज्ञिरे॒ तस्मा᳚त् । तस्मा᳚ज्जा॒ता अ॑जा॒वय॑: ।\\
यत्पुरु॑षं॒ व्य॑दधुः । क॒ति॒धा व्य॑कल्पयन् ।\\
मुखं॒ किम॑स्य॒ कौ बा॒हू । कावू॒रू पादा॑वुच्येते ।\\
ब्रा॒ह्म॒णो᳚ऽस्य॒ मुख॑मासीत् । बा॒हू रा॑ज॒न्य॑: कृ॒तः ॥ ५ ॥\\
\\
ऊ॒रू तद॑स्य॒ यद्वैश्य॑: । प॒द्भ्याग्ं शू॒द्रो अ॑जायत ।\\
च॒न्द्रमा॒ मन॑सो जा॒तः । चक्षो॒: सूर्यो॑ अजायत ।\\
मुखा॒दिन्द्र॑श्चा॒ग्निश्च॑ । प्रा॒णाद्वा॒युर॑जायत ।\\
नाभ्या॑ आसीद॒न्तरि॑क्षम् । शी॒र्ष्णो द्यौः सम॑वर्तत ।\\
प॒द्भ्यां भूमि॒र्दिश॒: श्रोत्रा᳚त् ।\\
तथा॑ लो॒काग्ं अ॑कल्पयन् ॥ ६ ॥\\
\\
वेदा॒हमे॒तं पुरु॑षं म॒हान्तम्᳚ ।\\
आ॒दि॒त्यव॑र्णं॒ तम॑स॒स्तु पा॒रे ।\\
सर्वा॑णि रू॒पाणि॑ वि॒चित्य॒ धीर॑: ।\\
नामा॑नि कृ॒त्वाऽभि॒वद॒न्॒ यदास्ते᳚ ।\\
धा॒ता पु॒रस्ता॒द्यमु॑दाज॒हार॑ ।\\
श॒क्रः प्रवि॒द्वान्प्र॒दिश॒श्चत॑स्रः ।\\
तमे॒वं वि॒द्वान॒मृत॑ इ॒ह भ॑वति ।\\
नान्यः पन्था॒ अय॑नाय विद्यते ।\\
य॒ज्ञेन॑ य॒ज्ञम॑यजन्त दे॒वाः ।\\
तानि॒ धर्मा॑णि प्रथ॒मान्या॑सन् ।\\
ते ह॒ नाकं॑ महि॒मान॑: सचन्ते ।\\
यत्र॒ पूर्वे॑ सा॒ध्याः सन्ति॑ दे॒वाः ॥ ७ ॥\\
शिरसे स्वाहा ॥\\
\subsubsection{\eng{Uttara Narayanam}}
अ॒द्भ्यः सम्भू॑तः पृथि॒व्यै रसा᳚च्च ।\\
वि॒श्वक॑र्मण॒: सम॑वर्त॒ताधि॑ ।\\
तस्य॒ त्वष्टा॑ वि॒दध॑द्रू॒पमे॑ति ।\\
तत्पुरु॑षस्य॒ विश्व॒माजा॑न॒मग्रे᳚ ।\\
वेदा॒हमे॒तं पुरु॑षं म॒हान्तम्᳚ ।\\
आ॒दि॒त्यव॑र्णं॒ तम॑स॒: पर॑स्तात् ।\\
तमे॒वं वि॒द्वान॒मृत॑ इ॒ह भ॑वति ।\\
नान्यः पन्था॑ विद्य॒तेय॑ऽनाय ।\\
प्र॒जाप॑तिश्चरति॒ गर्भे॑ अ॒न्तः ।\\
अ॒जाय॑मानो बहु॒धा विजा॑यते ॥ ८ ॥\\
\\
तस्य॒ धीरा॒: परि॑जानन्ति॒ योनिम्᳚ ।\\
मरी॑चीनां प॒दमि॑च्छन्ति वे॒धस॑: ।\\
यो दे॒वेभ्य॒ आत॑पति ।\\
यो दे॒वानां᳚ पु॒रोहि॑तः ।\\
पूर्वो॒ यो दे॒वेभ्यो॑ जा॒तः ।\\
नमो॑ रु॒चाय॒ ब्राह्म॑ये ।\\
रुचं॑ ब्रा॒ह्मं ज॒नय॑न्तः ।\\
दे॒वा अग्रे॒ तद॑ब्रुवन् ।\\
यस्त्वै॒वं ब्रा᳚ह्म॒णो वि॒द्यात् ।\\
तस्य॑ दे॒वा अस॒न् वशे᳚ ॥ ९ ॥\\
\\
ह्रीश्च॑ ते ल॒क्ष्मीश्च॒ पत्॒न्यौ᳚ ।\\
अ॒हो॒रा॒त्रे पा॒र्श्वे । नक्ष॑त्राणि रू॒पम् ।\\
अ॒श्विनौ॒ व्यात्तम्᳚ । इ॒ष्टं म॑निषाण ।\\
अ॒मुं म॑निषाण । सर्वं॑ मनिषाण ॥ १० ॥\\
शिकायै वषट् ॥\\
\subsection{\eng{Aprathiratham}}
आ॒शुः शिशा॑नो वृष॒भो न यु॒ध्मो घ॒नाघ॒नः क्षोभ॑णश्चर्षणी॒नाम्।\\
सं॒क्त्रन्द॑ नोऽ निमि॒ष ए॑कवी॒रः श॒तꣳ सेना॑, अजयथ् सा॒कमिन्द्रः॑॥ (1)\\
\\
सं॒क्रन्द॑ नेना निमि॒षेण॑ जि॒ष्णुना॑ युत्का॒रेण॑ दुश् च्य॒वनेन॑ धृ॒ष्णुना᳚।\\
तदिन्द्रे॑ण जयत॒-तथ् स॑हध्वं॒ युधो॑ नर॒ इषु॑हस्तेन॒ वृष्णा᳚॥ (2)\\
\\
स इषु॑ हस् तैः॒सनि॑ ष॒ङ्गि भि॑र् व॒शी सग्ग् स्र॑ष्टा॒ सयुध॒ इन्द्रो॑ ग॒णेन॑।\\
स॒ꣳ॒ सृ॒ष्ट॒ जिथ्सो॑ म॒पा बा॑हु श॒ध्यू᳚र्ध्व ध॑न्वा॒, प्रति॑हिता भि॒रस्ता᳚॥ (3)\\
\\
बृह॑स्पते॒ परि॑ दीया॒ रथे॑न रक्षो॒हाऽ मित्राꣳ॑ अप॒बा ध॑मानः।\\
प्र॒भञ् जन्थ् सेनाः᳚ प्र॒मृणो यु॒धा जय॑न् न॒स्माक॑ मेध्य वि॒ता रथा॑नाम्॥ (4)\\
\\
गो॒त्र॒भिदं॑ गो॒विदं॒ वज्र॑बाहुं॒ जय॑न्त॒ मज्म॑ प्रमृ॒णन्त॒ मोज॑सा।\\
इ॒मꣳ स॑जाता॒, अनु॑ वीरयध्व॒ मिन्द्रꣳ॑ सखा॒योऽ नु॒सꣳ र॑भध्वम्॥ (5)\\
\\
ब॒ल॒वि॒ज्ञा॒यः स्थविरः॒ प्रवीरः॒ सह॑स्वान् , वा॒जी सह॑मान उ॒ग्रः।\\
अ॒भिवी॑रो, अ॒भिस॑त्वा सहो॒जा जैत्र॑मिन्द्र॒ रथ॒मा ति॑ष्ठ गो॒वित्॥ (6)\\
\\
अ॒भिगो॒त्राणि॒ सह॑सा॒ गाह॑मानोऽ दा॒यो वी॒रः श॒तम॑न्यु॒ रिन्द्रः॑|\\
दु॒श् च्य॒व॒नः पृ॑तना॒ षाड॑ यु॒ध्द्यो᳚ऽ स्माक॒ꣳ॒ सेना॑, अवतु॒ प्र यु॒थ्सु॥ (7)\\
\\
ड्न्द्र॑ आसां ने॒ता बृह॒स्पति॒र् दक्षि॑णा य॒ज्ञः पु॒र एतु॒ सोमः॑।\\
दे॒व॒से॒नाना॑ मभिभञ्ज ती॒नां जय॑न्तीनां म॒रुतो॑ य॒न् त्वग्रे᳚ ॥ (8)\\
\\
इन्द्र॑स्य॒ वृष्णो॒ वरु॑णस्य॒ राज्ञ॑ आदि॒त्याना᳚ म॒रुता॒ꣳ॒ शर्ध्ध॑ उ॒ग्रम्।\\
म॒हाम॑नसां भुवनच्य॒ वानां॒ घोषो॑ दे॒वानां॒ जय॑ता॒ मुद॑स्थात्। (9)\\
\\
अ॒स्माक॒ मिन्द्रः॒ समृ॑ते षुध् व॒जेष् व॒स्मा कंया, इष॑व॒स्ता ज॑यन्तु।\\
अ॒स्माकं॑ वी॒रा, उत्त॑रे भवन्त् व॒स्मानु॑ देवा, अवता॒ हवे॑षु॥ (10)\\
\\
उद् ध॑र्षय मघव॒न् नायु॑धा॒न् युथ्सत्व॑नां माम॒कानां॒ महाꣳ॑ सि।\\
उद् वृ॑त् रहन् वा॒जिनां॒ वाजि॑ना॒न् युद्रथा॑नां॒ जय॑तामेतु॒ घोषः॑॥ (11)\\
\\
उप॒ प्रेत॒ जय॑ता नरः स्थि॒रा वः॑ सन्तु बा॒हवः॑।\\
इन्द्रो॑ वः॒ शर्म॑ यच् छत्वना धृ॒ष्याय थाऽस॑थ। (12)\\
\\
अव॑सृष्टा॒ परा॑ पत॒ शर॑व्ये॒, ब्रह्म॑सꣳ शिता।\\
गच्छा॒ मित्रा॒न् प्रवि॑श॒ मैषां॒ कं च॒नोच् छि॑षः॥ (13)\\
\\
मर्मा॑णि ते॒ वर्म॑भिश् छादयामि॒ सोम॑स्त्वा॒ राजा॒मृते॑ ना॒भिऽ व॑स्तां।\\
उ॒रोर् वरी॑यो॒ वरि॑वस्ते, अस्तु॒ जय॑न् तं॒त्वा-मनु॑ मदन्तु दे॒वाः॥ (14)\\
\\
यत्र॑ बा॒णाः स॒म्पत॑न्ति कुमा॒रा वि॑शि॒खा, इ॑व।\\
इन्द्रो॑ न॒स्तत्र॑ वृत्र॒हा वि॑श्वा॒हा शर्म॑ यच्छतु। (15)\\
\\
असु॑रा नजय॒न् तदप् प्र॑तिरथस्या, \\
प्रतिर थ॒त्वं यदप् प्र॑तिरथन् द्वि॒ती यो॒हो ता॒न्वाहा᳚ \\
प्र॒त्ये॑ वते न॒यज॑ मानो॒ भ्रातृ॑व्यां जय॒त्यथो॒,\\
अन॑ भिजितमे॒ वाभिज॑यति दश॒र्चं भ॑वति॒ दशा᳚क्षरा \\
वि॒राड् विराजे॒ मौ लो॒कौ विधृ॑ता व॒नयो᳚र् लो॒कयो॒र् विधृ॑त्या॒,\\
अथो॒ दशा᳚क्षरा वि॒राडन्नं॑ वि॒राड् वि॒राज् ये॒वान् नाद्ये॒ प्रति॑ तिष्ठ॒त्य स॑दिव॒वा,\\
अ॒न्तरि॑क्ष म॒न्तरि॑क्ष मि॒वाग्नी᳚ ध्र॒माग्नी᳚ध्ने\\
कवचाय हुं \\
\\
\subsection{\eng{Prathipurusha mithyunuvakaha}}
प्र॒ति॒ पू॒रु॒ष मेक॑ कपाला॒न् निर्व॑ प॒त्येक॒ मति॑रिक् \\
तं॒याव॑न्तो गृ॒ह्याः᳚ स्मस्तेभ्यः॒ कम॑करं\\
पशू॒नाꣳ शर्मा॑ऽसि॒ शर्म॒ यज॑मानस्य॒ शर्म॑ मे \\
य॒च्छैक॑ ए॒व रु॒द्रो नद् वि॒तीया॑य तस्थ \\
आ॒खुस्ते॑ रुद्र प॒शुस्तं जु॑षस्वै॒ष ते॑ रुद्र \\
भा॒गः स॒हस् वस्राऽम् बि॑कया॒ तं जु॑षस्व\\
भेष॒जं गवेऽश्वा॑य॒ पुरु॑षाय भे ष॒जमथो॑, \\
अ॒स्मभ्यं॑ भे ष॒जꣳ सुभे॑ष जं॒यथाऽ स॑ति।\\
\\
सु॒गं मे॒षाय॑ मे॒ष्या॑, अवा᳚म्ब रु॒द्रम॑दि म॒ह्यव॑ दे॒वं त्र्यं॑बकम्।\\
यथा॑ नः॒ श्रेय॑ सःकर॒द् यथा॑ नो॒\\
वस्य॑ सःकर॒द्यथा॑ नः पशु॒मतः॒ करद्यथा॑ नोव् व्यव सा॒यया᳚त्॥\\
\\
त्र्यं॑बकं यजामहे सुग॒न्धिंपु॑ष्टि॒वर्ध॑नम्। \\
उ॒र्वा॒रु॒कमि॑व॒ बन्ध॑नान्मृ॒त्योर्मु॑क्षीय॒माऽमृता᳚त्।\\
\\
ए॒षते॑ रुद्रभा॒गस्तं जु॑षस्व॒ तेना॑ऽव॒सेन॑\\
प॒रो मूज॑व॒तो ती॒ह्यव॑त त धन्वा॒ पिना॑कहस्तः॒ कृत्ति॑वासाः \\
प्र॒ति॒ पू॒रु॒ष मेक॑ कपा ला॒न् निर्व॑पति।\\
जा॒ता, ए॒व प्र॒जारु॒द्रान् नि॒रव॑दयते। एक॒मति॑रिक्तम्।\\
ज॒नि॒ष्य मा॑णा ए॒व प्र॒जारु॒द्रान् नि॒रव॑दयते। \\
एक॑ कपाला भवन्ति। ए॒क॒ धैवरु॒द्रन् नि॒रव॑दयते।\\
नाभि घा॑रयति। यद॑भि घा॒रये᳚त्‌। \\
अ॒न्त॒र॒व॒ चा॒रिणꣳ॑ रु॒द्रं कु॑र्यात्। \\
ए॒को॒ल् मु॒केन॑यन्ति। (1)\\
\\
तद्धि रु॒द्रस्य॑ भाग॒ धेयम्᳚। \\
इ॒मां दिश॑य्यन्ति। ए॒षावै रु॒द्रस्य॒दिक्।\\
स्वाया॑ मे॒वदि॒शि रु॒द्रन् नि॒रव॑दयते। \\
रु॒द्रोवा, अ॑प॒शुका॑या॒, आहु॑त्यै नाति॑ष्ठत।\\
अ॒सौते॑ प॒शुरिति॒ निर्दि॑ शे॒द्यं द्वि॒ष्यात्। \\
य॒मेवद् वेष्टि॑। तम॑स्मै प॒शुं निर्दि॑ शति। \\
यदि॒ नद् वि॒ष्यात्। आ॒खुस्ते॑ प॒शुरिति॑ ब्रूयात्। (2)\\
\\
नग्रा॒म्यान् प॒शून्, हि॒नस्ति॑। \\
नार॒ण्यान्। च॒तु॒ष्प॒थे जु॑होति।\\
ए॒षवा, अ॑ग्नी॒नां पड्वी॑ शो॒नाम॑। अ॒ग्नि॒वत् ये॒वजु॑होति। \\
म॒ध्य॒मेन॑ प॒र्णेन॑ जुहोति, स्रुग् घ्ये॑षा। \\
अथो॒खलु॑। अ॒न्त॒मे नै॒वहो॑ त॒व्यम्‌᳚। \\
अ॒न्त॒त ए॒वरु॒द्रन् नि॒र व॑दयते। (3)\\
\\
ए॒षते॑रुद्रभा॒गः स॒हस् वस्राऽम्बि॑क॒ येत्या॑ह। \\
श॒रद्वा, अ॒स्याम् बि॑का॒स् वसा᳚। \\
तया॒वा, ए॒षहि॑नस्ति।\\
यग्ं हि॒नस्ति॑। तयै॒ वैनग्ं स॒॑हश॑मयति। \\
भे॒ष॒जंगव॒ इत्या॑ह। याव॑न्त ए॒व ग्रा॒म्याःप॒शवः॑।\\
तेभ्यो॑ भेष॒जंक॑रोति। अवा᳚म्ब रु॒द्रम॑दि म॒हीत्या॑ह। \\
आ॒शिष॑ मे॒वै तामा शा᳚स्ते। (4)\\
\\
त्र्यं॑बकंयजामह॒ इत्या॑ह। \\
मृ॒त्योर्मु॑क्षीय॒माऽमृ॒ता दिति॒ वावै तदा॑ह। उत्कि॑रन्ति।\\
भग॑स्यलीफ् सन्ते। मूते॑ कृ॒त्वाऽऽ स॑जन्ति। \\
यथा॒ जन॑य्य॒ते॑ऽ वसंक॒रोति॑। ता॒दृ गे॒वतत्।\\
ए॒षते॑ रुद्र भा॒ग इत्या॑ह नि॒रव॑त्त्यै। अप्र॑ती क्ष॒माय॑न्ति। \\
अ॒पःपरि॑षिञ्चति। \\
रु॒द्रस् या॒न् तर् हि॑त्यै । प्रवा, ए॒ते᳚ऽस्माल्लो॒काच् च्य॑वन्ते। \\
येत्र्य॑म्ब कै॒श्चर॑न्ति। आ॒दि॒त्यं च॒रुं पुन॒रेत् य॒निर्व॑पति।\\
इ॒यंवा, अदि॑तिः। अ॒स्यामे॒व प्रति॑ तिष्ठन्ति॥ (5)\\
\\
वि॒भ्राड् बृहत् पि॒बतु सो॒म्यं \\
मध्वायु॒र् दधद् य॒॑ज्ञप॑ता॒ ववि॑हृतम्।\\
वात॑जू तो॒यो अ॑भि॒ रक्ष॑ ति॒त्मना᳚ \\
प्र॒जाः पु॑पोष पुरु॒धा विरा᳚जति॥\\
\\
नेत्रत्रयाय वौषट्॥\\
\\
\subsection{\eng{Tvamagne Rudro Anuvakaha}}
\\
त्वम॑ऽग्ने रु॒द्रो, असु॑रो म॒हो दि॒वस् त्वग्ं शर्धो॒ मारु॑तं पृ॒क्ष ई॑शिषे।\\
त्वं वातै॑ ररु॒णैर् या॑सि शङ्ख॒ यस्त्वं पू॒षा वि॑ध॒तः पा॑सि॒ नुत्मना᳚॥ \\
\\
आ वो॒ राजा॑ नमध् व॒रस्य॑ रु॒द्रग्ं होता॑रग्ं सत्य॒ यज॒ग्ं॒ रोद॑स् योः।\\
अ॒ग्निं पु॒रा त॑न यि॒त्नो र॒चित् ता॒द् धिर॑ण्य रूप॒ मव॑से कृणुध्वम्॥\\
\\
अग्निर् होता निष॑सा दा॒ यजी॑य् यानु॒ पस्थे॑ मा॒तुः सु॑र॒भावु॑ लो॒के। \\
युवा॑ क॒विः पु॑रुनि॒ष्ठ ऋ॒तावा॑ ध॒र्ता  कृ॑ष्टी॒ नामु॒त मध्य॑ इ॒द्धः।\\
\\
सा॒ध्वी म॑कर् दे॒ववी॑ तिन्नो, अ॒द्य य॒ज्ञस्य॑ जि॒ह्वाम॑ विदाम॒ गुह्या᳚म्\\
स आयु॒राऽगा᳚त् सुर॒भिर्वसा॑नो भ॒द्राम॑कर् दे॒वहू॑ तिन्नो, अ॒द्य॥\\
\\
अक्र॑न्द द॒ग्निः स्त॒न य॑न्नि व॒द्यौः क्षामा॒ रेरि॑हद् वी॒रुधः॑ सम॒ञन्न्।\\
स॒द्यो ज॑ज्ञा॒नो विही मि॒द्धो, अख्य॒ दारो द॑सी भा॒नुना॑ भात्य॒न्तः॥\\
\\
त्वे वसू॑नि पुर्वणी कहोतर् दो॒षा वस्तो॒ रेरि॑रे य॒ज्ञिया॑सः।\\
क्षामे॑ व॒विश्वा॒ भुव॑नानि॒ यस् मि॒न्थ् सꣳ सौ भ॑गानि दधि॒रे पा॑व॒के॥\\
\\
तुभ्यं ता, अ॑ङ्गिरस्तम॒ विश्वाः᳚ सुक् क्षि॒तयः॒ पृथ॑क्।\\
अग्ने॒ कामा॑य येमिरे॥\\
\\
अ॒श्याम॒ तंकाम॑ मऽग्ने॒ तवो॒त् य॑श्याम॑ र॒यिꣳ र॑यिवः सु॒वीरम्᳚\\
अ॒श्याम॒ वाज॑म॒भि वा॒जय॑न् तो॒ऽश्याम॑द्  द्यु॒म्न म॑ज रा॒जर॑न्ते॥\\
\\
श्रेष्ठं॑य विष्ठ भार॒ ताऽग्ने᳚ द्युमन्त॒ माभ॑र।\\
वसो॑ पुरु॒स् पृहꣳ॑ र॒यिम्॥\\
 \\
सश्वि॑ ता॒नस् त॑न्य॒तू रो॑च न॒स्था, अ॒जरे॑भि॒र् नान॑दद् भि॒र् यवि॑ष्ठः।\\
यः पा॑व॒कः पु॑रु॒तमः॑ पुरू॒णि॑ पृ॒थून् य॒ग्नि र॑नु॒याति॒ भर्वन्न्॥\\
\\
आयु॑ष्टे वि॒श्वतो॑ दध द॒य म॒ग्निर् वरे᳚ण्यः।\\
पुन॑स्ते प्रा॒ण आऽय॑ति॒ परा॒ यक्ष्मꣳ॑ सुवामि ते॥\\
\\
आ॒यु॒र्दा, अ॑ग्ने ह॒विषो॑ जुषा॒णो घृ॒त प्र॑तीको घृ॒तयो॑ निरेधि।\\
घृ॒तं पी॒त्वा मधु॒ चारु॒ गव्यं॑ पि॒तेव॑ पु॒त्रम॒भि र॑क्ष तादि॒मम्।॥\\
\\
तस्मै॑ ते प्रति॒हर्य॑ते॒ जात॑वेदो॒ विच॑र्षणे।\\
अग्ने॒ जना॑मि सुष्टु॒ तिम्॥\\
\\
दि॒वस्परि॑ प्रथमं ज॑ज्ञे, अ॒ग्नि र॒स्मद् द्वि॒तीयं॒ परि॑ जा॒तवे॑दाः।\\
तृ॒तीय॑ म॒प्सु नृ॒मणा॒, अज॑स्र॒ मिन्धा॑न एनं जरते स्वा॒धीः॥\\
\\
शुचिः॑ पावक॒ वन्द्योऽग्ने॑ बृहद् विरो॑चसे।\\
त्वं घृ॒ते भि॒राहु॑तः॥\\
\\
दृ॒शा॒नो रु॒क्म उ॒र्व्याव् य॑द् यौद् दु॒र्मर् ष॒ मायुः॑ श्रि॒ये रु॑चा॒नः।\\
अ॒ग्नि र॒मृतो॑, अभव॒द् वयो॑भिः यदे॑ नं॒द्यौर ज॑नयत् सु॒रेताः᳚॥\\
\\
आ यदि॒षे नृ॒पतिं॒ तेज॒ आन॒ट् शुचि॒ रेतो॒ निषिक्तं॒ द्यौर॒भीके᳚।\\
अ॒ग्निः शर्ध मनव॒द्यं युवा॑नग्ग् स्वा॒ धियं जनयत् सू॒दय॑च्च ॥\\
\\
सते जीयसा॒ मन॑सा॒ त्वोत॑ उ॒त शि॑क्षस्-वप्-प्र॒त्-यस्य॑ शि॒क्षोः।\\
अग्ने॑ रा॒यो नृत॑ मस्य॒ प्रभू॑तौ भू॒याम॑ ते सुष्टुत य॑श्च॒ वस्वः॑।\\
\\
अग्ने॒ सह॑न्त॒ माभ॑र द्यु॒म्-नस्य॑  प्रा॒सहा॑ र॒यिम्।\\
विश्वा॒ यश् च॑र्ष॒णी॒ रभ्या॑सा वाजे॑षु सा॒सह॑त्।\\
\\
तम॑ग्ने पृतना॒सहꣳ॑ र॒यिꣳ स॑हस्व॒ आ भ॑र।\\
त्वꣳ हि स॒त्यो, अद्भु॑तो दा॒ता वाज॑स्य गोम॑तः॥\\
\\
उ॒क्षान्ना॑य व॒शान्ना॑य॒ सोम॑ पृष्ठाय वे॒धसे᳚।\\
स्तोमै᳚र् विधे मा॒ऽग् नये᳚॥\\
\\
व॒द्मा हि सू॑नो॒, अस्य॑द् म॒सद् वा॑ च॒क्रे, अ॒ग्निर् ज॒नु षाऽज् मान्नम्᳚।\\
सत्वन्न॑ ऊर् जसन॒ ऊर्जं॑ धा॒ राजे॑ वजे रवृ॒के क्षे᳚ष् य॒न्तः॥\\
\\
अग्न॒ आयूꣳ॑ षि पवस॒ आसु॒ वोर् ज॒मिषं॑ च नः।\\
आ॒रे बा॑धस्व दु॒च्छु ना᳚म्॥\\
\\
अग्ने॒ पव॑स् व॒स् वपा॑, अ॒स्मे वर्चः॑॑ सु॒वीर्यम्᳚\\
दध॒त्पोषꣳ॑ र॒यिं मयि॑॥\\
\\
अग्ने॑ पावक रो॒चिषा॑ म॒न्द्रया॑ देव जिह्वया᳚।\\
आ दे॒वान् व॑क्षि॒ यक्षि॑ च॥\\
\\
स नः॑ पावक दीदि॒ वोऽग्ने॑ दे॒वाꣳ इ॒हाऽऽव॑ह।\\
उप॑ य॒ज्ञꣳ ह॒विश् च॑नः॥\\
\\
अ॒ग्निः शुचि॑व् व्रत तमः॒ शुचि॒र् विप्रः॒ शुचिः॑ क॒विः।\\
शुची॑ रोचत॒ आहु॑तः॥\\
\\
उद॑ग्ने॒ शुच॑ य॒स्तव॑ शु॒क्रा, भ्राज॑न्त ईरते।\\
तव॒ ज्योतीग्ग्॑ष् य॒र्चयः॑॥\\
\\
त्वम॑ग्ने रु॒द्रो असु॑रो म॒हो दि॒वः। त्वꣳ शर्धो॒ मारु॑तं पृ॒क्ष ई॑शिषे।\\
त्वं वातै॑ररु॒णैर्या॑सि शङ्ग॒यः। त्वं पू॒षा वि॑ध॒तः पा॑सि॒ नु त्मना᳚।\\
\\
देवा॑ दे॒वेषु॑ श्रयद्ध्वम्। प्रथ॑मा द्वि॒तीये॑षु श्रयद्ध्वम्। \\
द्विती॑यास् तृ॒तीये॑षु श्रयद्ध्वम्। तृती॑याश् चतु॒र्थेषु॑ श्रयद्ध्वम्। \\
च॒तु॒र्थाः प॑श्च॒मेषु॑ श्रयद्ध्वम्। प॒ञ्च॒माः ष॒ष् ठेषु॑ श्रयद्ध्वम्।\\
\\
ष॒ष्ठाः स॑प्त॒मेषु॑ श्रयद्ध्वम्। स॒प्त॒मा, अ॑ष्ट॒मेषु॑ श्रयद्ध्वम्। \\
अ॒ष्ट॒मा-न॑व॒मेषु॑ श्रयद्ध्वम्। न॒व॒मा-द॑श॒मेषु॑ श्रयद्ध्वम्। \\
द॒श॒मा, ए॑का द॒शेषु॑ श्रयद्ध्वम्। ए॒का॒द॒शा द्वा॑ द॒शेषु॑ श्रयद्ध्वम्। \\
द्वा॒ द॒शास् त्र॑यो द॒शेषु॒॑ श्रयद्ध्वम्। त्र॒यो॒ द॒शाश् च॑तर् द॒शेषु॑ श्रयद्ध्वम्। \\
च॒त॒र् द॒शाः प॑ञ्च द॒शेषु॑ श्रयद्ध्वम्। प॒ञ्च॒ द॒शा: षो॑ड॒शेषु॑ श्रयद्ध्वम्। \\
षो॒ड॒शाः स॑प्त द॒शेषु॑ श्रयद्ध्वम्। स॒प्त॒ द॒शा, अ॑ष्टा द॒शेषु॑ श्रयद्ध्वम्।\\
\\
अ॒ष्टा॒द॒शा, ए॑कान् नवि॒ꣳ॒ शेषु॑ श्रयद्ध्वम्। ए॒का॒न् न॒विꣳ॒ शा वि॒ꣳ॒शेषु॑ श्रयद्ध्वम्। \\
वि॒ꣳ॒शा, ए॑क वि॒ꣳ॒ शेषु॑ श्रयद्ध्वम्। ए॒क॒वि॒ꣳ॒ शा द्वा॑ वि॒ꣳ॒शेषु॑ श्रयद्ध्वम्। \\
द्वा॒ वि॒ꣳ॒शास् त्र॑योवि॒ꣳ॒ शेषु॑ श्रयद्ध्वम्। त्र॒यो॒वि॒ꣳ॒ शाश् च॑तर् वि॒ꣳ॒ शेषु॑ श्रयद्ध्वम्। \\
च॒त॒र् वि॒ꣳ॒शाः प॑ञ्चवि॒ꣳ॒ शेषु॑ श्रयद्ध्वम्। प॒ञ्च॒वि॒ꣳ॒ शाः ष॑ड् वि॒ꣳ॒शेषु॑ श्रयद्ध्वम्। \\
ष॒ड् विꣳ॒शाः स॑प्त वि॒ꣳ॒ शेषु॑ श्रयद्ध्वम्। स॒प्त॒ विꣳ॒शा, अ॑ष्टा वि॒ꣳ॒ शेषु॑ श्रयद्ध्वम्। \\
\\
अ॒ष्टा॒ वि॒ꣳ॒ शा, ए॑कान् नत्रि॒ꣳ॒ शेषु॑ श्रयद्ध्वम्। ए॒का॒न् न॒त्रि॒ꣳ शास् त्रि॒ꣳ॒ शेषु॑ श्रयद्ध्वम्। \\
त्रि॒ꣳ॒ शा, ए॑कत्रि॒ꣳ॒ शेषु॑ श्रयद्ध्वम्। ए॒क॒त्रि॒ꣳ॒ शा द्वा᳚ त्रि॒ꣳ॒ शेषु॑ श्रयद्ध्वम्।\\
द्वा॒त्रि॒ꣳ॒ शास् त्र॑यस् त्रि॒ꣳ॒ शेषु॑ श्रयद्ध्वम् । \\
\\
देवा᳚स् त्रिरेका दशा॒स् त्रिस् त्र॑यस् त्रिꣳशाः। \\
उत्त॑रे भवत । उत्त॑र वर्त् मान॒ उत्त॑र सत्वानः। यत्का॑म इ॒दं जु॒होमि॑।\\
तन्मे॒ समृ॑द् यताम्। व॒यग्ग् स्या॑म॒ पत॑यो रयी॒ णाम्। भूर्भुवः॒ स्वः॑ स्वाहा᳚॥\\
\\
अस्त्रायफट्\\
\subsection{\eng{Panchanga sagrucchajapeth}}
स॒द्योजा॒तं प्र॑पद्या॒मि॒ स॒द्योजा॒ताय॒ वै नमो॒ नमः॑। \\
भ॒वे भ॑वे॒ नाति॑भवे भवस्व॒ माम्। भ॒वोद्भ॑वाय॒ नमः॑॥\\
\\
वा॒म॒दे॒वाय॒ नमो᳚ ज्ये॒ष्ठाय॒ नम॑श्श्रे॒ष्ठाय॒ नमो॑ रु॒द्राय॒ नमः॒ काला॑य॒ नमः॒ कल॑विकरणाय॒\\
नमो॒ बल॑विकरणाय॒ नमो॒ बला॑य॒ नमो॒ बल॑प्रमथनाय॒ नम॒स्सर्व॑भूतदमनाय॒ नमो॑\\
म॒नोन्म॑नाय॒ नमः॑॥\\
\\
अ॒घोरे᳚भ्योऽथ॒ घोरे᳚भ्यो घोर॒घोर॑तरेभ्यः। \\
स॒र्वे᳚तः॑ सर्व॒ शर्वे᳚भ्यो॒ नम॑स्ते अस्तु रु॒द्ररू॑पेभ्यः॥\\
\\
तत्पुरु॑षाय वि॒द्महे॑ महादे॒वाय॑ धीमहि। तन्नो॑ रुद्रः प्रचो॒दया᳚त्॥\\
\\
ईशानस्सर्व॑विद्या॒नां ईश्वरस्सर्व॑भूता॒नां॒ ब्रह्माधि॑पति॒-\\
रब्रह्म॒णोऽधि॑पति॒-रब्रह्मा॑ शि॒वोमे॑ अस्तु सदाशि॒वोम्॥\\
\subsubsection{\eng{Alternate}}
ह॒ग्ं॒ सश् शुचि॒ षद् वसुरन् तरि क्ष॒सद् धोता वेदि॒ष दति थिर् दरो ण॒सत्।\\
नृ॒षद् वर॒सद् रुत सब् यो मसदब् जा गोजा, ऋतजा, अद्रिजा, ऋतं बृहत्॥ \\
\\
प्रतद् विष्णु॑स्स्तवते वी॒र्या॑य । मृ॒गो न भी॒मः कु॑च॒रो गि॑रि॒ष्ठाः । \\
यस्यो॒रुषु॑ त्रि॒षु वि॒क्रम॑णेषु । अधि॑क्षि॒यन्ति॒ भुव॑नानि॒ विश्वा᳚  । \\
\\
त्र्य॑म्बकं यजामहे सुग॒न्धिं पु॑ष्टि॒वर्ध॑नम् ।\\
उ॒र्वा॒रु॒कमि॑व॒ बन्ध॑नान्मृ॒त्योर्मु॑क्षीय॒ माऽमृता᳚त् ।\\
\\
तत्स॑ वि॒तुर् वृ॑णीमहे । व॒यं दे॒वस्य॒ भोज॑नम् । \\
श्रेष्ठꣳ॑  सर्व॒ धात॑मं । तुरं॒ भग॑स्य धीमहि ॥\\
\\
विष्णु॒र् योनिं॑ कल्पयतु । त्वष्टा॑ रु॒पाणि॑ पिꣳशतु । \\
आसि॑ चतु प्र॒जाप॑तिः । धा॒ता गर्भं॑ दधातु मे ॥\\
\subsection{\eng{Ashtanga Pranamaha}}
हि॒र॒ण्य॒ग॒र्भः सम॑वर्त॒ताग्रे॑ भू॒तस्य॑ जा॒तः पति॒रेक॑ आसीत्।\\
स दाधा॑र पृथि॒वीं द्यामु॒तेमां कस्मै॑ दे॒वाय ह॒विषा॑ विधेम॥\\
ॐ उमा महेश्वराभ्यां नमः     (1)\\
\\
यः प्रा॑ण॒तो नि॑मिष॒तो म॑हि॒त् वै॒क इद्राजा॒ जग॑तो ब॒भूव॑।\\
य ईशे॑, अ॒स्यद् द्वि॒पद॒श् चतु॑ष्पदः॒ कस्मै॑ दे॒वाय॑ ह॒विषा॑ विधेम॥\\
ॐ उमा महेश्वराभ्यां नमः     (2)\\
\\
ब्रह्म॑ जज्ञा॒नं प्र॑थ॒मं पु॒रस्ता॒द् विसी॑ म॒तस् सु॒रुचो॑ वे॒न आ॑वः।\\
स बु॒ध् निया॑, उप॒मा, अ॑स्य वि॒ष्टास् स॒तश्च॒ यो॒नि मस॑ तश्च॒ विवः॑॥\\
ॐ उमा महेश्वराभ्यां नमः     (3)\\
\\
म॒ही द्यौः पृ॑थि॒वी च॑ न इ॒मं यज्ञं मि॑मिक्षताम्।\\
पि॒पृ॒तां नो॒-भरी॑ मभिः॥\\
ॐ उमा महेश्वराभ्यां नमः     (4)\\
\\
उप॑श् वासय पृथि॒वी मु॒तद् यां पु॑रु॒त्रा ते॑ मनुतां॒ विष्टि॑तं॒ जग॑त्।\\
स दुन्दु॑बे स॒जू रिन्द्रे॑ण दे॒वैर् दू॒राद् दवी॑यो॒, अप॑से ध॒शत् रून्॥\\
ॐ उमा महेश्वराभ्यां नमः     (5)\\
\\
अग्ने॒ नय॑ सु॒पता॑ रा॒ये, अ॒स्मान्. विश्वा॑नि देव व॒युना॑नि वि॒द्वान्।\\
यु॒यो॒ध् य॑स्मज् जु॑हु-रा॒ण-मेनो॒ भूयि॑ष् ठान्ते॒ नम॑ उक्तिं विधेम॥\\
ॐ उमा महेश्वराभ्यां नमः     (6)\\
 \\
या ते॑, अग्ने॒ रुद्रि॑या त॒नूस् तया॑नः पाहि॒ तस्या᳚स्ते॒ स्वाहा॒ याते॑,\\
अग्नेऽ याश॒या र॑जाश॒या ह॑राश॒या त॒नूर् वर्-षि॑ष्ठा-गह्वरे॒ष्-ठोग्रं वचो॒,\\
अपा वधीं त्वे॒षव् वचो॒, अपा॑ वधी॒ग् स्वाहा᳚॥   \\
ॐ उमा महेश्वराभ्यां नमः     (7)\\
\\
इ॒मं य॑मप् प्रस् त॒र माहि सीदाङ्-गि॑रोभिः\eng{f} पि॒तृभिः॑स् संविदा॒नः।\\
आत्वा॒ मन्त्राः᳚ कवि श॒स्ता व॑हन्-त्वे॒ना रा॑जन्. ह॒विषा॑ माद यस्व॥\\
ॐ उमा महेश्वराभ्यां नमः     (8)\\
\\
उरसा शिरसा दृष्ट्या मनसा वचसा तथा।\\
पद्भ्यां  कराभ्यां कर्णाभ्यां प्रणा मोष्टाङ्ग उच्यते॥\\
\\
