\section{\eng{Kalasheshu Dhyanam}}
ध्यायेन्निरा मयं वस्तु सर्गस् थितिल यादिकं ।\\
निर्गुणन् निष्कळन् नित्यं मनो वाचा मगोचरं ॥ 1\\
\\
गंगाधरं शशिधरं जटामकुट शोभितं ।\\
श्वेत भूति त्रिपुण्ड्रेण विरा जित ललाटकं ॥ 2\\
\\
लोचन त्रय संपन्नं स्वर्ण कुण्डल शोभितं\\
स्मेरा ननञ् चतुर्बाहुं मुक्ता हारोप शोभितं ॥ 3\\
\\
अक्ष मालां सुधा कुंभञ् चिन्मयीं मुद्रिका मपि\\
पुस्तकं च भुजैर् दिव्यैर् दधानं पार्वती पतिं ॥ 4\\
\\
श्वेतां बरधरं श्वेतं रत्न सिंहास नस्थितं\\
सर्वा भीष्ट प्रदा तारं वट मूल निवासिनं ॥ 5\\
\\
वामांगे संस्थितां गौरीं बालार्कायुत सन्निभां\\
जपा कुसुम साहस्र समा नश्रिय-मीश्वरीं ॥ 6\\
\\
सुवर्ण रत्न खचित मकुटेन विराजितां\\
ललाटप॒ संराजत् सल्लग्नति लकाञ्चितां ॥ 7\\
\\
राजीवायत नेत्रान्तां नीलोत्पल दळेक्षणां\\
सन्तप्त हेम रचित ताटङ्का भरणान्वितां ॥ 8\\
\\
तांबूल चर्व णरत रक्त जिह्वा विराजितां\\
पताका भरणो पेतां मुक्ता हारोप शोभितां ॥ 9\\
\\
स्वर्ण कङ्कण सय्युक्तैश् चतुर् भिर् बाहु भिर्युतां ।\\
सुवर्ण रत्न खचित काञ्ची दाम विराजितां ॥ 10\\
\\
कद लील लितस्तंभ सन्नि भोरुयु गान्वितां\\
श्रिया विराजित पदां भक्त त्राण परायणां ॥ 11\\
\\
अन्योन्याश् लिष्ट हृद् बाहु गौरी शङ्कर संज्ञकं\\
सनातनं परं ब्रह्म परमात्मान मव्ययं ॥ 12\\
\\
आवाह यामि जगता मीश्वरं परमेश्वरं ।\\
मंगला यतनं देवं युवान मतिसुन्दरं ।\\
ध्यायेत् कल्पत रोर्मूले सुखासीनं सहोमया ॥ 13\\
\\
आगच्छाऽऽ गच्छ भगवन् देवेश परमेश्वर ।\\
सच्चिदानन्द भूतेश पार्वती च नमोऽस्तुते\\
\\
आत्वा॑ वहन्तु॒ हर॑य॒स्सचे॑तसः श्वे॒तैरश्वै᳚ स्स॒ह के॑तु॒मद्भिः॑ ।\\
वाता॑जितै॒ र्बल॑वद्भि॒ र्मनो॑जवै॒ राया॑हि शी॒घ्रं मम॑ ह॒व्याय॑ श॒र्वों ।\\
\\
त्र्यं॑बकंँयजामहे सुग॒न्धिं पु॑ष्टि॒वर्ध॑नं ।\\
उ॒र्वा॒रु॒कमि॑व॒ बन्ध॑नान् मृ॒त्योर्मु॑क्षीय॒ माऽमृता᳚त् ।\\
\\
गौ॒रीमि॑माय सलि॒लानि॒ तक्ष॒त्येक॑पदी द्वि॒पदि॒ सा चतु॑ष्पदी ।\\
अ॒ष्टाप॑दी॒ नव॑पदी बभू॒वुषी॑ स॒हस्रा᳚क्षरा पर॒मे व्यो॑मन् ।\\