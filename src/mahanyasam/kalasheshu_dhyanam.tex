\section{\eng{Kalasheshu Dhyanam}}
ध्यायेन्निरामयं वस्तु सर्गस्थिति लयादिकं ।\\
निर्गुणं निष्कळं नित्यं मनो वाचामगोचरं ॥ 1\\
\\
गंगाधरं शशिधरं जटामकुट शोभितं ।\\
श्वेतभूति त्रिपुण्ड्रेण विराजित ललाटकं ॥ 2\\
\\
लोचनत्रय संपन्नं स्वर्णकुण्डल शोभितं\\
स्मेराननं चतुर्बाहुं मुक्ताहारोपशोभितं ॥ 3\\
\\
अक्षमालां सुधाकुंभं चिन्मयीं मुद्रिकामपि\\
पुस्तकं च भुजै र्दिव्यै र्दधानं पार्वतीपतिं ॥ 4\\
\\
श्वेतांबरधरं श्वेतं रत्नसिंहासन स्थितं\\
सर्वाभीष्ट प्रदातारं वटमूल निवासिनं ॥ 5\\
\\
वामांगे संस्थितां गौरीं बालार्कायुत सन्निभां\\
जपाकुसुमसाहस्र समानश्रिय-मीश्वरीं ॥ 6\\
\\
सुवर्णरत्नखचित मकुटेन विराजितां\\
ललाटप॒ संराजत् संलग्नतिलकाञ्चितां ॥ 7\\
\\
राजीवायतनेत्रान्तां नीलोत्पल दळेक्षणां\\
संतप्त हेमरचित ताटङ्काभरणान्वितां ॥ 8\\
\\
तांबूल चर्वणरत रक्त जिह्वा विराजितां\\
पताका भरणोपेतां मुक्ता हारोप शोभितां ॥ 9\\
\\
स्वर्ण कंकण संयुक्तै श्चतुर्भि र्बाहुभिर्युतां ।\\
सुवर्ण रत्नखचित काञ्चीदाम विराजितां ॥ 10\\
\\
कदलीललितस्तंभ संन्निभोरुयुगान्वितां\\
श्रिया विराजितपदां भक्तत्राण परायणां ॥ 11\\
\\
अन्योन्या-श्लिष्टहृद् बाहु गौरीशङ्कर-संज्ञकं\\
सनातनं परंब्रह्म परमात्मान-मव्ययं ॥ 12\\
\\
(मंगलायतनं देवं युवान-मतिसुन्दरं । ध्यायेत् कल्पतरोर्मूले सुखासीनं\\
सहोमया ॥ आवाहयामि जगता-मीश्वरं परमेश्वरं ।) 13\\
\\
(आगच्छाऽऽगच्छ भगवन् देवेश परमेश्वर ।\\
सच्चिदानन्द भूतेश पार्वती च नमोऽस्तुते)\\
\\
आत्वा॑ वहन्तु॒ हर॑य॒स्सचे॑तसः श्वे॒तैरश्वै᳚ स्स॒ह के॑तु॒मद्भिः॑ ।\\
वाता॑जितै॒ र्बल॑वद्भि॒ र्मनो॑जवै॒ राया॑हि शी॒घ्रं मम॑ ह॒व्याय॑ श॒र्वों ।\\
\\
त्र्यं॑बकंँयजामहे सुग॒न्धिं पु॑ष्टि॒वर्ध॑नं ।\\
उ॒र्वा॒रु॒कमि॑व॒ बन्ध॑नान् मृ॒त्योर्मु॑क्षीय॒ माऽमृता᳚त् ।\\
गौ॒रीमि॑माय सलि॒लानि॒ तक्ष॒त्येक॑पदी द्वि॒पदि॒ सा चतु॑ष्पदी ।\\
अ॒ष्टाप॑दी॒ नव॑पदी बभू॒वुषी॑ स॒हस्रा᳚क्षरा पर॒मे व्यो॑मन् ।\\
