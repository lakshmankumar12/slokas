\subsection{\eng{Aprathiratham}}
आ॒शुः शिशा॑नो वृष॒भो न यु॒ध्मो घ॒नाघ॒नः क्षोभ॑णश्चर्षणी॒नाम्।\\
सं॒क्त्रन्द॑ नोऽ निमि॒ष ए॑कवी॒रः श॒तꣳ सेना॑, अजयथ् सा॒कमिन्द्रः॑॥ (1)\\
\\
सं॒क्रन्द॑ नेना निमि॒षेण॑ जि॒ष्णुना॑ युत्का॒रेण॑ दुश् च्य॒वनेन॑ धृ॒ष्णुना᳚।\\
तदिन्द्रे॑ण जयत॒-तथ् स॑हध्वं॒ युधो॑ नर॒ इषु॑हस्तेन॒ वृष्णा᳚॥ (2)\\
\\
स इषु॑ हस् तैः॒सनि॑ ष॒ङ्गि भि॑र् व॒शी सग्ग् स्र॑ष्टा॒ सयुध॒ इन्द्रो॑ ग॒णेन॑।\\
स॒ꣳ॒ सृ॒ष्ट॒ जिथ्सो॑ म॒पा बा॑हु श॒ध्यू᳚र्ध्व ध॑न्वा॒, प्रति॑हिता भि॒रस्ता᳚॥ (3)\\
\\
बृह॑स्पते॒ परि॑ दीया॒ रथे॑न रक्षो॒हाऽ मित्राꣳ॑ अप॒बा ध॑मानः।\\
प्र॒भञ् जन्थ् सेनाः᳚ प्र॒मृणो यु॒धा जय॑न् न॒स्माक॑ मेध्य वि॒ता रथा॑नाम्॥ (4)\\
\\
गो॒त्र॒भिदं॑ गो॒विदं॒ वज्र॑बाहुं॒ जय॑न्त॒ मज्म॑ प्रमृ॒णन्त॒ मोज॑सा।\\
इ॒मꣳ स॑जाता॒, अनु॑ वीरयध्व॒ मिन्द्रꣳ॑ सखा॒योऽ नु॒सꣳ र॑भध्वम्॥ (5)\\
\\
ब॒ल॒वि॒ज्ञा॒यः स्थविरः॒ प्रवीरः॒ सह॑स्वान् , वा॒जी सह॑मान उ॒ग्रः।\\
अ॒भिवी॑रो, अ॒भिस॑त्वा सहो॒जा जैत्र॑मिन्द्र॒ रथ॒मा ति॑ष्ठ गो॒वित्॥ (6)\\
\\
अ॒भिगो॒त्राणि॒ सह॑सा॒ गाह॑मानोऽ दा॒यो वी॒रः श॒तम॑न्यु॒ रिन्द्रः॑|\\
दु॒श् च्य॒व॒नः पृ॑तना॒ षाड॑ यु॒ध्द्यो᳚ऽ स्माक॒ꣳ॒ सेना॑, अवतु॒ प्र यु॒थ्सु॥ (7)\\
\\
ड्न्द्र॑ आसां ने॒ता बृह॒स्पति॒र् दक्षि॑णा य॒ज्ञः पु॒र एतु॒ सोमः॑।\\
दे॒व॒से॒नाना॑ मभिभञ्ज ती॒नां जय॑न्तीनां म॒रुतो॑ य॒न् त्वग्रे᳚ ॥ (8)\\
\\
इन्द्र॑स्य॒ वृष्णो॒ वरु॑णस्य॒ राज्ञ॑ आदि॒त्याना᳚ म॒रुता॒ꣳ॒ शर्ध्ध॑ उ॒ग्रम्।\\
म॒हाम॑नसां भुवनच्य॒ वानां॒ घोषो॑ दे॒वानां॒ जय॑ता॒ मुद॑स्थात्। (9)\\
\\
अ॒स्माक॒ मिन्द्रः॒ समृ॑ते षुध् व॒जेष् व॒स्मा कंया, इष॑व॒स्ता ज॑यन्तु।\\
अ॒स्माकं॑ वी॒रा, उत्त॑रे भवन्त् व॒स्मानु॑ देवा, अवता॒ हवे॑षु॥ (10)\\
\\
उद् ध॑र्षय मघव॒न् नायु॑धा॒न् युथ्सत्व॑नां माम॒कानां॒ महाꣳ॑ सि।\\
उद् वृ॑त् रहन् वा॒जिनां॒ वाजि॑ना॒न् युद्रथा॑नां॒ जय॑तामेतु॒ घोषः॑॥ (11)\\
\\
उप॒ प्रेत॒ जय॑ता नरः स्थि॒रा वः॑ सन्तु बा॒हवः॑।\\
इन्द्रो॑ वः॒ शर्म॑ यच् छत्वना धृ॒ष्याय थाऽस॑थ। (12)\\
\\
अव॑सृष्टा॒ परा॑ पत॒ शर॑व्ये॒, ब्रह्म॑सꣳ शिता।\\
गच्छा॒ मित्रा॒न् प्रवि॑श॒ मैषां॒ कं च॒नोच् छि॑षः॥ (13)\\
\\
मर्मा॑णि ते॒ वर्म॑भिश् छादयामि॒ सोम॑स्त्वा॒ राजा॒मृते॑ ना॒भिऽ व॑स्तां।\\
उ॒रोर् वरी॑यो॒ वरि॑वस्ते, अस्तु॒ जय॑न् तं॒त्वा-मनु॑ मदन्तु दे॒वाः॥ (14)\\
\\
यत्र॑ बा॒णाः स॒म्पत॑न्ति कुमा॒रा वि॑शि॒खा, इ॑व।\\
इन्द्रो॑ न॒स्तत्र॑ वृत्र॒हा वि॑श्वा॒हा शर्म॑ यच्छतु। (15)\\
\\
असु॑रा नजय॒न् तदप् प्र॑तिरथस्या, \\
प्रतिर थ॒त्वं यदप् प्र॑तिरथन् द्वि॒ती यो॒हो ता॒न्वाहा᳚ \\
प्र॒त्ये॑ वते न॒यज॑ मानो॒ भ्रातृ॑व्यां जय॒त्यथो॒,\\
अन॑ भिजितमे॒ वाभिज॑यति दश॒र्चं भ॑वति॒ दशा᳚क्षरा \\
वि॒राड् विराजे॒ मौ लो॒कौ विधृ॑ता व॒नयो᳚र् लो॒कयो॒र् विधृ॑त्या॒,\\
अथो॒ दशा᳚क्षरा वि॒राडन्नं॑ वि॒राड् वि॒राज् ये॒वान् नाद्ये॒ प्रति॑ तिष्ठ॒त्य स॑दिव॒वा,\\
अ॒न्तरि॑क्ष म॒न्तरि॑क्ष मि॒वाग्नी᳚ ध्र॒माग्नी᳚ध्ने\\
कवचाय हुं \\
\\
\subsection{\eng{Prathipurusha mithyunuvakaha}}
प्र॒ति॒ पू॒रु॒ष मेक॑ कपाला॒न् निर्व॑ प॒त्येक॒ मति॑रिक् \\
तं॒याव॑न्तो गृ॒ह्याः᳚ स्मस्तेभ्यः॒ कम॑करं\\
पशू॒नाꣳ शर्मा॑ऽसि॒ शर्म॒ यज॑मानस्य॒ शर्म॑ मे \\
य॒च्छैक॑ ए॒व रु॒द्रो नद् वि॒तीया॑य तस्थ \\
आ॒खुस्ते॑ रुद्र प॒शुस्तं जु॑षस्वै॒ष ते॑ रुद्र \\
भा॒गः स॒हस् वस्राऽम् बि॑कया॒ तं जु॑षस्व\\
भेष॒जं गवेऽश्वा॑य॒ पुरु॑षाय भे ष॒जमथो॑, \\
अ॒स्मभ्यं॑ भे ष॒जꣳ सुभे॑ष जं॒यथाऽ स॑ति।\\
\\
सु॒गं मे॒षाय॑ मे॒ष्या॑, अवा᳚म्ब रु॒द्रम॑दि म॒ह्यव॑ दे॒वं त्र्यं॑बकम्।\\
यथा॑ नः॒ श्रेय॑ सःकर॒द् यथा॑ नो॒\\
वस्य॑ सःकर॒द्यथा॑ नः पशु॒मतः॒ करद्यथा॑ नोव् व्यव सा॒यया᳚त्॥\\
\\
त्र्यं॑बकं यजामहे सुग॒न्धिंपु॑ष्टि॒वर्ध॑नम्। \\
उ॒र्वा॒रु॒कमि॑व॒ बन्ध॑नान्मृ॒त्योर्मु॑क्षीय॒माऽमृता᳚त्।\\
\\
ए॒षते॑ रुद्रभा॒गस्तं जु॑षस्व॒ तेना॑ऽव॒सेन॑\\
प॒रो मूज॑व॒तो ती॒ह्यव॑त त धन्वा॒ पिना॑कहस्तः॒ कृत्ति॑वासाः \\
प्र॒ति॒ पू॒रु॒ष मेक॑ कपा ला॒न् निर्व॑पति।\\
जा॒ता, ए॒व प्र॒जारु॒द्रान् नि॒रव॑दयते। एक॒मति॑रिक्तम्।\\
ज॒नि॒ष्य मा॑णा ए॒व प्र॒जारु॒द्रान् नि॒रव॑दयते। \\
एक॑ कपाला भवन्ति। ए॒क॒ धैवरु॒द्रन् नि॒रव॑दयते।\\
नाभि घा॑रयति। यद॑भि घा॒रये᳚त्‌। \\
अ॒न्त॒र॒व॒ चा॒रिणꣳ॑ रु॒द्रं कु॑र्यात्। \\
ए॒को॒ल् मु॒केन॑यन्ति। (1)\\
\\
तद्धि रु॒द्रस्य॑ भाग॒ धेयम्᳚। \\
इ॒मां दिश॑य्यन्ति। ए॒षावै रु॒द्रस्य॒दिक्।\\
स्वाया॑ मे॒वदि॒शि रु॒द्रन् नि॒रव॑दयते। \\
रु॒द्रोवा, अ॑प॒शुका॑या॒, आहु॑त्यै नाति॑ष्ठत।\\
अ॒सौते॑ प॒शुरिति॒ निर्दि॑ शे॒द्यं द्वि॒ष्यात्। \\
य॒मेवद् वेष्टि॑। तम॑स्मै प॒शुं निर्दि॑ शति। \\
यदि॒ नद् वि॒ष्यात्। आ॒खुस्ते॑ प॒शुरिति॑ ब्रूयात्। (2)\\
\\
नग्रा॒म्यान् प॒शून्, हि॒नस्ति॑। \\
नार॒ण्यान्। च॒तु॒ष्प॒थे जु॑होति।\\
ए॒षवा, अ॑ग्नी॒नां पड्वी॑ शो॒नाम॑। अ॒ग्नि॒वत् ये॒वजु॑होति। \\
म॒ध्य॒मेन॑ प॒र्णेन॑ जुहोति, स्रुग् घ्ये॑षा। \\
अथो॒खलु॑। अ॒न्त॒मे नै॒वहो॑ त॒व्यम्‌᳚। \\
अ॒न्त॒त ए॒वरु॒द्रन् नि॒र व॑दयते। (3)\\
\\
ए॒षते॑रुद्रभा॒गः स॒हस् वस्राऽम्बि॑क॒ येत्या॑ह। \\
श॒रद्वा, अ॒स्याम् बि॑का॒स् वसा᳚। \\
तया॒वा, ए॒षहि॑नस्ति।\\
यग्ं हि॒नस्ति॑। तयै॒ वैनग्ं स॒॑हश॑मयति। \\
भे॒ष॒जंगव॒ इत्या॑ह। याव॑न्त ए॒व ग्रा॒म्याःप॒शवः॑।\\
तेभ्यो॑ भेष॒जंक॑रोति। अवा᳚म्ब रु॒द्रम॑दि म॒हीत्या॑ह। \\
आ॒शिष॑ मे॒वै तामा शा᳚स्ते। (4)\\
\\
त्र्यं॑बकंयजामह॒ इत्या॑ह। \\
मृ॒त्योर्मु॑क्षीय॒माऽमृ॒ता दिति॒ वावै तदा॑ह। उत्कि॑रन्ति।\\
भग॑स्यलीफ् सन्ते। मूते॑ कृ॒त्वाऽऽ स॑जन्ति। \\
यथा॒ जन॑य्य॒ते॑ऽ वसंक॒रोति॑। ता॒दृ गे॒वतत्।\\
ए॒षते॑ रुद्र भा॒ग इत्या॑ह नि॒रव॑त्त्यै। अप्र॑ती क्ष॒माय॑न्ति। \\
अ॒पःपरि॑षिञ्चति। \\
रु॒द्रस् या॒न् तर् हि॑त्यै । प्रवा, ए॒ते᳚ऽस्माल्लो॒काच् च्य॑वन्ते। \\
येत्र्य॑म्ब कै॒श्चर॑न्ति। आ॒दि॒त्यं च॒रुं पुन॒रेत् य॒निर्व॑पति।\\
इ॒यंवा, अदि॑तिः। अ॒स्यामे॒व प्रति॑ तिष्ठन्ति॥ (5)\\
\\
वि॒भ्राड् बृहत् पि॒बतु सो॒म्यं \\
मध्वायु॒र् दधद् य॒॑ज्ञप॑ता॒ ववि॑हृतम्।\\
वात॑जू तो॒यो अ॑भि॒ रक्ष॑ ति॒त्मना᳚ \\
प्र॒जाः पु॑पोष पुरु॒धा विरा᳚जति॥\\
\\
नेत्रत्रयाय वौषट्॥\\
