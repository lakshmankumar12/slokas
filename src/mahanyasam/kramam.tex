\section{\eng{Kramam}}
\subsection{\eng{Ganapthi Dhyanam}}
{\centering
\begin{longtable}{|c|c|}
\hline
ओं ग॒णानां᳚ त्वा         & त्वा॒ ग॒णप॑तिं\\
\hline
ग॒णप॑तिꣳ हवामहे        & ग॒णप॑ति॒ मिति॑ ग॒ण - प॒तिं॒ >\\
\hline
ह॒वा॒म॒हे॒ क॒विं           & क॒विं क॑वी॒नां\\
\hline
क॒वी॒ना मु॑प॒मश्र॑ वस्तमं      & उ॒प॒मश्र॑ वस्त म॒मित् यु॑प॒मश्र॑वः - त॒मं॒ >\\
\hline
ज्ये॒ष्ठ॒राजं॒ ब्रह्म॑णां      & ज्ये॒ष्ठ॒राज॒मिति॑ ज्येष्ठ - राजं᳚ >\\
\hline
ब्रह्म॑णां ब्रह्मणः       & ब्र॒ह्म॒ण॒स्प॒ते॒ >\\
\hline
प॒त॒ आ                & आ नः॑\\
\hline
न॒श्शृ॒ण्वन्न्             & शृ॒ण्वन्नू॒तिभिः॑\\
\hline
ऊ॒ति भि॑स्सीद           & ऊ॒ति भि॒रित्यू॒ति - भिः॒\\
\hline
सी॒द॒ साद॑नं            & साद॑न॒ मिति॒ साद॑नं\\
\hline
\end{longtable}
}
\subsection{\eng{Anuvaka 1}}
ओं नमो भगवते रुद्राय
{\centering
\begin{longtable}{|c|c|}
\hline
ओं ॥  नम॑स्ते                & ते॒ रु॒द्र॒\\
\hline
रु॒द्र॒ म॒न्यवे᳚ >               & म॒न्यव॑ उ॒तो\\
\hline
उ॒तो ते᳚ >                  & उ॒तो, इत्यु॒तो\\
\hline
त॒ इष॑वे                    & इष॑वे॒ नमः॑\\
\hline
नम॒ इति॒ नमः॑               & नम॑स्ते\\
\hline
ते॒ अ॒स्तु॒                    & अ॒स्तु॒ धन्व॑ने\\
\hline
धन्व॑ने बा॒हुभ्यां᳚ >            & बा॒हुभ्या॑मु॒त\\
\hline
बा॒हुभ्या॒मिति॑ बा॒हु - भ्यां॒ >   & उ॒त ते᳚ >\\
\hline
ते॒ नमः॑                    & नम॒ इति॒ नमः॑\\
\hline
या ते᳚ >                   & त॒ इषुः॑\\
\hline
इषु॑श्शि॒वत॑मा                & शि॒वत॑मा शि॒वं\\
\hline
शि॒वत॒मेति॑ शि॒व - त॒मा॒ >      & शि॒वं ब॒भूव॑\\
\hline
ब॒भूव॑ ते                    & ते॒ धनुः॑\\
\hline
धनु॒रिति॒ धनुः॑               & शि॒वा श॑र॒व्या᳚ >\\
\hline
श॒र॒व्या॑ या                 & या तव॑\\
\hline
तव॒ तया᳚ >                 & तया॑ नः\\
\hline
नो॒ रु॒द्र॒                   & रु॒द्र॒ मृ॒ड॒य॒\\
\hline
मृ॒ड॒येति॑ मृडय                & या ते᳚ >\\
\hline
ते॒ रु॒द्र॒                    & रु॒द्र॒ शि॒वा\\
\hline
शि॒वा त॒नूः                 & त॒नूरघो॑रा\\
\hline
अघो॒राऽ पा॑प काशिनी         & अपा॑प काशि॒नीत्य पा॑प - का॒शि॒नी॒>\\
\hline
तया॑ नः                   & न॒स्त॒नुवा᳚ >\\
\hline
त॒नुवा॒ शन्त॑मया              & शन्त॑मया॒ गिरि॑शन्त\\
\hline
शन्त॑म॒येति॒ शं - त॒म॒या॒ >       & गिरि॑शन्ता॒भि\\
\hline
गिरि॑श॒न्तेति॒ गिरि॑-श॒न्त॒       & अ॒भिचा॑कशीहि\\
\hline
चा॒क॒शी॒हीति॑ चाकशीहि        & यामिषुं᳚ >\\
\hline
इषुं॑ गिरिशन्त               & गि॒रि॒श॒न्त॒ हस्ते᳚ >\\
\hline
गि॒रि॒श॒न्तेति॑ गिरि - श॒न्त॒     & हस्ते॒ बिभ॑र्.षि\\
\hline
बिभ॒र्.ष्यस्त॑वे               & अस्त॑व॒ इत्यस्त॑वे\\
\hline
शि॒वां गि॑रित्र              & गि॒रि॒त्र॒ तां\\
\hline
गि॒रि॒त्रेति॑ गिरि - त्र॒       & तां कु॑रु\\
\hline
कु॒रु॒ मा                    & मा हिꣳ॑सीः\\
\hline
हि॒ꣳ॒सीः॒ पुरु॑षं               & पुरु॑षं॒ जग॑त्\\
\hline
जग॒दिति॒ जग॑त्               & शि॒वेन॒ वच॑सा\\
\hline
वच॑सा त्वा                 & त्वा॒ गिरि॑श\\
\hline
गिरि॒शाच्छ॑                 & अच्छा॑वदामसि\\
\hline
व॒दा॒म॒सीति॑ वदामसि          & यथा॑ नः\\
\hline
नः॒ सर्वं᳚ >                 & सर्व॒मित्\\
\hline
इज्जग॑त्                    & जग॑दय॒क्ष्मं\\
\hline
अ॒य॒क्ष्मꣳ सु॒मनाः᳚ >           & सु॒मना॒ अस॑त्\\
\hline
सु॒मना॒ इति॑ सु - मनाः᳚ >      & अस॒दित्यस॑त्\\
\hline
अद्ध्य॑वोचत्                 & अ॒वो॒च॒ द॒धि॒ व॒क्ता\\
\hline
अ॒धि॒ व॒क्ता प्र॑थ॒मः            & अ॒धि॒ व॒क्तेत्य॑धि - व॒क्ता\\
\hline
प्र॒थ॒मो दैव्यः॑               & दैव्यो॑ भि॒षक्\\
\hline
भि॒षगिति॑ भि॒षक्             & अ॒हीꣲ॑श्च\\
\hline
च॒ सर्वान्॑                  & सर्वा᳚न् जं॒भयन्न्॑\\
\hline
जं॒भय॒न्थ् सर्वाः᳚>             & सर्वा᳚श्च\\
\hline
च॒ या॒तु॒धा॒न्यः॑               & या॒तु॒धा॒न्य॑ इति॑ यातु - धा॒न्यः॑\\
\hline
अ॒सौ यः                   & यस्ता॒म्रः\\
\hline
ता॒म्रो अ॑रु॒णः               & अ॒रु॒ण उ॒त\\
\hline
उ॒त ब॒भ्रुः                  & ब॒भ्रुः सु॑म॒ङ्गलः॑\\
\hline
सु॒म॒ङ्गल॒ इति॑ सु-म॒ङ्गलः॑        & ये च॑\\
\hline
चे॒मां                      & इ॒माꣳ रु॒द्राः\\
\hline
रु॒द्रा अ॒भितः॑               & अ॒भितो॑ दि॒क्षु\\
\hline
दि॒क्षु श्रि॒ताः              & श्रि॒ताः स॑हस्र॒शः\\
\hline
स॒ह॒स्र॒शोऽव॑                 & स॒ह॒स्र॒श इति॑ सहस्र - शः\\
\hline
अवै॑षां                     & ए॒षा॒ꣳ॒ हेडः॑\\
\hline
हेड॑ ईमहे                   & ई॒म॒ह॒ इती॑महे\\
\hline
अ॒सौ यः                   & यो॑ऽव॒सर्प॑ति\\
\hline
अ॒व॒सर्प॑ति॒ नील॑ग्रीवः         & अ॒व॒सर्प॒तीत्य॑व - सर्प॑ति\\
\hline
नील॑ग्रीवो॒ विलो॑हितः        & नील॑ग्रीव॒ इति॒ नील॑ - ग्री॒वः॒\\
\hline
विलो॑हित॒ इति॒ वि - लो॒हि॒तः॒  & उ॒तैनं᳚ >\\
\hline
ए॒नं॒ गो॒पाः                 & गो॒पा अ॑दृशन्न्\\
\hline
गो॒पा इति॑ गो-पाः          & अ॒दृ॒श॒न्नदृ॑शन्न्\\
\hline
अदृ॑शन्नुदहा॒र्यः॑              & उ॒द॒हा॒र्य॑ इत्यु॑द-हा॒र्यः॑\\
\hline
उ॒तैनं᳚ >                    & ए॒नं॒ विश्वा᳚ >\\
\hline
विश्वा॑ भू॒तानि॑              & भू॒तानि॒ सः\\
\hline
स दृ॒ष्टः                   & दृ॒ष्टो मृ॑डयाति\\
\hline
मृ॒ड॒या॒ति॒ नः॒                & न॒ इति॑ नः\\
\hline
नमो॑ अस्तु                  & अ॒स्तु॒ नील॑ग्रीवाय\\
\hline
नील॑ग्रीवाय सहस्रा॒क्षाय॑      & नील॑ग्रीवा॒येति॒ नील॑ - ग्री॒वा॒य॒\\
\hline
स॒ह॒स्रा॒क्षाय॑ मी॒ढुषे᳚ >         & स॒ह॒स्रा॒क्षायेति॑ सहस्र - अ॒क्षाय॑\\
\hline
मी॒ढुष॒ इति॑ मी॒ढुषे᳚ >          & अथो॒ ये\\
\hline
अथो॒ इत्यथो᳚ >              & ये अ॑स्य\\
\hline
अ॒स्य॒ सत्वा॑नः               & सत्वा॑नो॒ऽहं\\
\hline
अ॒हन्तेभ्यः॑                  & तेभ्यो॑ऽकरं\\
\hline
अ॒क॒र॒न्नमः॑                  & नम॒ इति॒ नमः॑\\
\hline
प्रमु॑ञ्च                    & मु॒ञ्च॒ धन्व॑नः\\
\hline
धन्व॑न॒स्त्वं                  & त्वमु॒भयोः᳚ >\\
\hline
उ॒भयो॒रार्त्नि॑योः            & आर्त्नि॑यो॒र्ज्यां\\
\hline
ज्यामिति॒ज्यां               & याश्च॑\\
\hline
च॒ ते॒ >                    & ते॒ हस्ते᳚ >\\
\hline
हस्त॒ इष॑वः                 & इष॑वः॒ परा᳚ >\\
\hline
परा॒ ताः                  & ता भ॑गवः\\
\hline
भ॒ग॒वो॒ व॒प॒                  & भ॒ग॒व॒ इति॑ भग - वः॒\\
\hline
व॒पेति॑ वप                  & अ॒व॒तत्य॒ धनुः॑\\
\hline
अ॒व॒तत्येत्य॑व - तत्य॑           & धनु॒स्त्वं\\
\hline
त्वꣳ सह॑स्राक्ष              & सह॑स्राक्ष॒ शते॑षुधे\\
\hline
सह॑स्रा॒क्षेति॒ सह॑स्र - अ॒क्ष॒     & शते॑षुध॒ इति॒ शत॑ - इ॒षु॒धे॒ >\\
\hline
नि॒शीर्य॑ श॒ल्यानां᳚ >          & नि॒शीर्येति॑ नि - शीर्य॑\\
\hline
श॒ल्यानां॒ मुखा᳚ >             & मुखा॑ शि॒वः\\
\hline
शि॒वो नः॑                  & नः॒ सु॒मनाः᳚ >\\
\hline
सु॒मना॑ भव                  & सु॒मना॒ इति॑ सु - मनाः᳚ >\\
\hline
भ॒वेति॑ भव                  & विज्यं॒ धनुः॑\\
\hline
विज्य॒मिति॒ वि - ज्यं॒ >       & धनुः॑ कप॒र्दिनः॑\\
\hline
क॒प॒र्दिनो॒ विश॑ल्यः           & विश॑ल्यो॒ बाण॑वान्\\
\hline
विश॑ल्य॒ इति॒ वि - श॒ल्यः॒      & बाण॑वाꣳ उ॒त\\
\hline
बाण॑वा॒निति॒ बाण॑ - वा॒न्॒      & उ॒तेत्यु॒त\\
\hline
अने॑शन्नस्य                  & अ॒स्येष॑वः\\
\hline
इष॑वः आ॒भुः                 & आ॒भुर॑स्य\\
\hline
अ॒स्य॒ नि॒ष॒ङ्गथिः॑             & नि॒ष॒ङ्गथि॒रिति॑ नि॒ष॒ङ्गथिः॑\\
\hline
या ते᳚ >                   & ते॒ हे॒तिः\\
\hline
हे॒तिर्मी॑ढुष्टम               & मी॒ढु॒ष्ट॒म॒ हस्ते᳚ >\\
\hline
मी॒ढु॒ष्ट॒मेति॑ मीढुः - त॒म॒       & हस्ते॑ ब॒भूव॑\\
\hline
ब॒भूव॑ ते                    & ते॒ धनुः॑\\
\hline
धनु॒रिति॒ धनुः॑               & तया॒ऽस्मान्\\
\hline
अ॒स्मान्. वि॒श्वतः॑            & वि॒श्वत॒स्त्वं\\
\hline
त्वम॑य॒क्ष्मया᳚ >              & अ॒य॒क्ष्मया॒ परि॑\\
\hline
परि॑ब्भुज                   & भु॒जेति॑ भुज\\
\hline
नम॑स्ते                     & ते॒ अ॒स्तु॒\\
\hline
अ॒स्त्वायु॑धाय                & आयु॑धा॒याना॑तताय\\
\hline
अना॑तताय धृ॒ष्णवे᳚ >           & अना॑तता॒येत्यना᳚ - त॒ता॒य॒\\
\hline
धृ॒ष्णव॒ इति॑ धृ॒ष्णवे᳚ >          & उ॒भाभ्या॑मु॒त\\
\hline
उ॒त ते᳚ >                   & ते॒ नमः॑\\
\hline
नमो॑ बा॒हुभ्यां᳚ >             & बा॒हुभ्या॒न्तव॑\\
\hline
बा॒हुभ्या॒मिति॑ बा॒हु - भ्यां॒ >   & तव॒ धन्व॑ने\\
\hline
धन्व॑न॒ इति॒ धन्व॑ने            & परि॑ ते\\
\hline
ते॒ धन्व॑नः                  & धन्व॑नो हे॒तिः\\
\hline
हे॒तिर॒स्मान्                 & अ॒स्मान् वृ॑णक्तु\\
\hline
वृ॒ण॒क्तु॒ वि॒श्वतः॑              & वि॒श्वत॒ इति॑ वि॒श्वतः॑\\
\hline
अथो॒ यः                   & अथो॒ इत्यथो᳚ >\\
\hline
य इ॑षु॒धिः                  & इ॒षु॒धिस्तव॑\\
\hline
इ॒षु॒धिरिती॑षु - धिः          & तवा॒रे\\
\hline
आ॒रे अ॒स्मत्                  & अ॒स्मन्नि\\
\hline
निधे॑हि                    & धे॒हि॒ तं\\
\hline
तमिति॒ तं                  & \\
\hline
\end{longtable}
}
\subsection{\eng{Anuvaka 2}}
ओं नमो भगवते रुद्राय
{\centering
\begin{longtable}{|c|c|}
\hline
नमो॒ हिर॑ण्यबाहवे            & हिर॑ण्यबाहवे सेना॒न्ये᳚ >\\
\hline
हिर॑ण्यबाहव॒ इति॒ हिर॑ण्य - बा॒ह॒वे॒ >   & से॒ना॒न्ये॑ दि॒शां\\
\hline
से॒ना॒न्य॑ इति॑ सेना - न्ये᳚ >     & दि॒शाञ्च॑\\
\hline
च॒ पत॑ये                    & पत॑ये॒ नमः॑\\
\hline
नमो॒ नमः॑                  & नमो॑ वृ॒क्षेभ्यः॑\\
\hline
वृ॒क्षेभ्यो॒ हरि॑केशेभ्यः          & हरि॑केशेभ्यः पशू॒नां\\
\hline
हरि॑केशेभ्य॒ इति॒ हरि॑ - के॒शे॒भ्यः॒  & प॒शू॒नां पत॑ये\\
\hline
पत॑ये॒ नमः॑                  & नमो॒ नमः॑\\
\hline
नमः॑ स॒स्पिञ्ज॑राय            & स॒स्पिञ्ज॑राय॒ त्विषी॑मते\\
\hline
त्विषी॑मते पथी॒नां            & त्विषी॑मत॒ इति॒ त्विषि॑ - म॒ते॒ >\\
\hline
प॒थी॒नां पत॑ये                & पत॑ये॒ नमः॑\\
\hline
नमो॒ नमः॑                  & नमो॑ बभ्लु॒शाय॑\\
\hline
ब॒भ्लु॒शाय॑ विव्या॒धिने᳚ >        & वि॒व्या॒धिनेऽन्ना॑नां\\
\hline
वि॒व्या॒धिन॒ इति॑ वि - व्या॒धिने᳚ > & अन्ना॑नां॒ पत॑ये\\
\hline
पत॑ये॒ नमः॑                  & नमो॒ नमः॑\\
\hline
नमो॒ हरि॑केशाय              & हरि॑केशायोपवी॒तिने᳚ >\\
\hline
हरि॑केशा॒येति॒ हरि॑ - के॒शा॒य॒     & उ॒प॒वी॒तिने॑ पु॒ष्टानां᳚ >\\
\hline
उ॒प॒वी॒तिन॒ इत्यु॑प - वी॒तिने᳚ >   & पु॒ष्टानां॒ पत॑ये\\
\hline
पत॑ये॒ नमः॑                  & नमो॒ नमः॑\\
\hline
नमो॑ भ॒वस्य॑                 & भ॒वस्य॑ हे॒त्यै\\
\hline
हे॒त्यै जग॑तां                 & जग॑तां॒ पत॑ये\\
\hline
पत॑ये॒ नमः॑                  & नमो॒ नमः॑\\
\hline
नमो॑ रु॒द्राय॑                & रु॒द्राया॑तता॒विने᳚ >\\
\hline
आ॒त॒ता॒विने॒ क्षेत्रा॑णां          & आ॒त॒ता॒विन॒ इत्या᳚ - त॒ता॒विने᳚ >\\
\hline
क्षेत्रा॑णां॒ पत॑ये              & पत॑ये॒ नमः॑\\
\hline
नमो॒ नमः॑                  & नमः॑ सू॒ताय॑\\
\hline
सू॒तायाह॑न्त्याय              & अह॑न्त्याय॒ वना॑नां\\
\hline
वना॑नां॒ पत॑ये                & पत॑ये॒ नमः॑\\
\hline
नमो॒ नमः॑                  & नमो॒ रोहि॑ताय\\
\hline
रोहि॑ताय स्थ॒पत॑ये            & स्थ॒पत॑ये वृ॒क्षाणां᳚ >\\
\hline
वृ॒क्षाणां॒ पत॑ये               & पत॑ये॒ नमः॑\\
\hline
नमो॒ नमः॑                  & नमो॑ म॒न्त्रिणे᳚ >\\
\hline
म॒न्त्रिणे॑ वाणि॒जाय॑           & वा॒णि॒जाय॒ कक्षा॑णां\\
\hline
कक्षा॑णां॒ पत॑ये               & पत॑ये॒ नमः॑\\
\hline
नमो॒ नमः॑                  & नमो॑ भुव॒न्तये᳚ >\\
\hline
भु॒व॒न्तये॑ वारिवस्कृ॒ताय॑         & वा॒रि॒व॒स्कृ॒तायौष॑धीनां\\
\hline
वा॒रि॒व॒स्कृ॒तायेति॑ वारिवः - कृ॒ताय॑ & ओष॑धीनां॒ पत॑ये\\
\hline
पत॑ये॒ नमः॑                  & नमो॒ नमः॑\\
\hline
नम॑ उ॒च्चैर्घो॑षाय             & उ॒च्चैर्घो॑षायाक्र॒न्दय॑ते\\
\hline
उ॒च्चैर्घो॑षा॒येत्यु॒च्चैः - घो॒षा॒य॒   & आ॒क्र॒न्दय॑ते पत्ती॒नां\\
\hline
आ॒क्र॒न्दय॑त॒ इत्या᳚ - क्र॒न्दय॑ते    & प॒त्ती॒नां पत॑ये\\
\hline
पत॑ये॒ नमः॑                  & नमो॒ नमः॑\\
\hline
नमः॑ कृथ्स्नवी॒ताय॑            & कृ॒थ्स्न॒वी॒ताय॒ धाव॑ते\\
\hline
कृ॒थ्स्न॒वी॒तायेति॑ कृथ्स्न - वी॒ताय॑ & धावे॑ते॒ सत्व॑नां\\
\hline
सत्व॑नां॒ पत॑ये                & पत॑ये॒ नमः॑\\
\hline
नम॒ इति॒ नमः॑               & \\
\hline
\end{longtable}
}
