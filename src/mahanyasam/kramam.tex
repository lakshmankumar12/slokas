\section{\eng{Kramam}}
\subsection{\eng{Ganapthi Dhyanam}}
{\centering
\begin{longtable}{|c|c|}
\hline
ओं ग॒णानां᳚ त्वा         & त्वा॒ ग॒णप॑तिं\\
\hline
ग॒णप॑तिꣳ हवामहे        & ग॒णप॑ति॒ मिति॑ ग॒ण - प॒तिं॒ >\\
\hline
ह॒वा॒म॒हे॒ क॒विं           & क॒विं क॑वी॒नां\\
\hline
क॒वी॒ना मु॑प॒मश्र॑ वस्तमं      & उ॒प॒मश्र॑ वस्त म॒मित् यु॑प॒मश्र॑वः - त॒मं॒ >\\
\hline
ज्ये॒ष्ठ॒राजं॒ ब्रह्म॑णां      & ज्ये॒ष्ठ॒राज॒मिति॑ ज्येष्ठ - राजं᳚ >\\
\hline
ब्रह्म॑णां ब्रह्मणः       & ब्र॒ह्म॒ण॒स्प॒ते॒ >\\
\hline
प॒त॒ आ                & आ नः॑\\
\hline
न॒श्शृ॒ण्वन्न्             & शृ॒ण्वन्नू॒तिभिः॑\\
\hline
ऊ॒ति भि॑स्सीद           & ऊ॒ति भि॒रित्यू॒ति - भिः॒\\
\hline
सी॒द॒ साद॑नं            & साद॑न॒ मिति॒ साद॑नं\\
\hline
\end{longtable}
}
\subsection{\eng{Anuvaka 1}}
ओं नमो भगवते रुद्राय\\
{\centering
{\small \eng{Sloka 1}} \\
\begin{longtable}{|c|c|}
\hline  
ओं ॥  नम॑स्ते                & ते॒ रु॒द्र॒\\
\hline
रु॒द्र॒ म॒न्यवे᳚ >               & म॒न्यव॑ उ॒तो\\
\hline
उ॒तो ते᳚ >                  & उ॒तो, इत्यु॒तो\\
\hline
त॒ इष॑वे                    & इष॑वे॒ नमः॑\\
\hline
नम॒ इति॒ नमः॑               & नम॑स्ते\\
\hline
ते॒ अ॒स्तु॒                    & अ॒स्तु॒ धन्व॑ने\\
\hline
धन्व॑ने बा॒हुभ्यां᳚ >            & बा॒हुभ्या॑मु॒त\\
\hline
बा॒हुभ्या॒मिति॑ बा॒हु - भ्यां॒ >   & उ॒त ते᳚ >\\
\hline
ते॒ नमः॑                    & नम॒ इति॒ नमः॑\\
\hline
\end{longtable}
}
{\centering
{\small \eng{Sloka 2}} \\
\begin{longtable}{|c|c|}
\hline
या ते᳚ >                   & त॒ इषुः॑\\
\hline
इषु॑श्शि॒वत॑मा                & शि॒वत॑मा शि॒वं\\
\hline
शि॒वत॒मेति॑ शि॒व - त॒मा॒ >      & शि॒वं ब॒भूव॑\\
\hline
ब॒भूव॑ ते                    & ते॒ धनुः॑\\
\hline
धनु॒रिति॒ धनुः॑               & शि॒वा श॑र॒व्या᳚ >\\
\hline
श॒र॒व्या॑ या                 & या तव॑\\
\hline
तव॒ तया᳚ >                 & तया॑ नः\\
\hline
नो॒ रु॒द्र॒                   & रु॒द्र॒ मृ॒ड॒य॒\\
\hline
मृ॒ड॒येति॑ मृडय                & या ते᳚ >\\
\hline
\end{longtable}
}
{\centering
{\small \eng{Sloka 3}} \\
\begin{longtable}{|c|c|}
\hline
ते॒ रु॒द्र॒                    & रु॒द्र॒ शि॒वा\\
\hline
शि॒वा त॒नूः                 & त॒नूरघो॑रा\\
\hline
अघो॒राऽ पा॑प काशिनी         & अपा॑प काशि॒नीत्य पा॑प - का॒शि॒नी॒>\\
\hline
तया॑ नः                   & न॒स्त॒नुवा᳚ >\\
\hline
त॒नुवा॒ शन्त॑मया              & शन्त॑मया॒ गिरि॑शन्त\\
\hline
शन्त॑म॒येति॒ शं - त॒म॒या॒ >       & गिरि॑शन्ता॒भि\\
\hline
गिरि॑श॒न्तेति॒ गिरि॑-श॒न्त॒       & अ॒भिचा॑कशीहि\\
\hline
चा॒क॒शी॒हीति॑ चाकशीहि        & यामिषुं᳚ >\\
\hline
\end{longtable}
}
{\centering
{\small \eng{Sloka 4}} \\
\begin{longtable}{|c|c|}
\hline
इषुं॑ गिरिशन्त               & गि॒रि॒श॒न्त॒ हस्ते᳚ >\\
\hline
गि॒रि॒श॒न्तेति॑ गिरि - श॒न्त॒     & हस्ते॒ बिभ॑र्.षि\\
\hline
बिभ॒र्.ष्यस्त॑वे               & अस्त॑व॒ इत्यस्त॑वे\\
\hline
शि॒वां गि॑रित्र              & गि॒रि॒त्र॒ तां\\
\hline
गि॒रि॒त्रेति॑ गिरि - त्र॒       & तां कु॑रु\\
\hline
कु॒रु॒ मा                    & मा हिꣳ॑सीः\\
\hline
हि॒ꣳ॒सीः॒ पुरु॑षं               & पुरु॑षं॒ जग॑त्\\
\hline
जग॒दिति॒ जग॑त्               & शि॒वेन॒ वच॑सा\\
\hline
\end{longtable}
}
{\centering
{\small \eng{Sloka 5}} \\
\begin{longtable}{|c|c|}
\hline
वच॑सा त्वा                 & त्वा॒ गिरि॑श\\
\hline
गिरि॒शाच्छ॑                 & अच्छा॑वदामसि\\
\hline
व॒दा॒म॒सीति॑ वदामसि          & यथा॑ नः\\
\hline
नः॒ सर्वं᳚ >                 & सर्व॒मित्\\
\hline
इज्जग॑त्                    & जग॑दय॒क्ष्मं\\
\hline
अ॒य॒क्ष्मꣳ सु॒मनाः᳚ >           & सु॒मना॒ अस॑त्\\
\hline
सु॒मना॒ इति॑ सु - मनाः᳚ >      & अस॒दित्यस॑त्\\
\hline
\end{longtable}
}
{\centering
{\small \eng{Sloka 6}} \\
\begin{longtable}{|c|c|}
\hline
अद्ध्य॑वोचत्                 & अ॒वो॒च॒ द॒धि॒ व॒क्ता\\
\hline
अ॒धि॒ व॒क्ता प्र॑थ॒मः            & अ॒धि॒ व॒क्तेत्य॑धि - व॒क्ता\\
\hline
प्र॒थ॒मो दैव्यः॑               & दैव्यो॑ भि॒षक्\\
\hline
भि॒षगिति॑ भि॒षक्             & अ॒हीꣲ॑श्च\\
\hline
च॒ सर्वान्॑                  & सर्वा᳚न् जं॒भयन्न्॑\\
\hline
जं॒भय॒न्थ् सर्वाः᳚>             & सर्वा᳚श्च\\
\hline
च॒ या॒तु॒धा॒न्यः॑               & या॒तु॒धा॒न्य॑ इति॑ यातु - धा॒न्यः॑\\
\hline
\end{longtable}
}
{\centering
{\small \eng{Sloka 7}} \\
\begin{longtable}{|c|c|}
\hline
अ॒सौ यः                   & यस्ता॒म्रः\\
\hline
ता॒म्रो, अ॑रु॒णः               & अ॒रु॒ण उ॒त\\
\hline
उ॒त ब॒भ्रुः                  & ब॒भ्रुः सु॑म॒ङ्गलः॑\\
\hline
सु॒म॒ङ्गल॒ इति॑ सु-म॒ङ्गलः॑        & ये च॑\\
\hline
चे॒मां                      & इ॒माꣳ रु॒द्राः\\
\hline
रु॒द्रा अ॒भितः॑               & अ॒भितो॑ दि॒क्षु\\
\hline
दि॒क्षु श्रि॒ताः              & श्रि॒ताः स॑हस्र॒शः\\
\hline
स॒ह॒स्र॒शोऽव॑                 & स॒ह॒स्र॒श इति॑ सहस्र - शः\\
\hline
अवै॑षां                     & ए॒षा॒ꣳ॒ हेडः॑\\
\hline
हेड॑ ईमहे                   & ई॒म॒ह॒ इती॑महे\\
\hline
\end{longtable}
}
{\centering
{\small \eng{Sloka 8}} \\
\begin{longtable}{|c|c|}
\hline
अ॒सौ यः                   & यो॑ऽव॒सर्प॑ति\\
\hline
अ॒व॒सर्प॑ति॒ नील॑ग्रीवः         & अ॒व॒सर्प॒तीत्य॑व - सर्प॑ति\\
\hline
नील॑ग्रीवो॒ विलो॑हितः        & नील॑ग्रीव॒ इति॒ नील॑ - ग्री॒वः॒\\
\hline
विलो॑हित॒ इति॒ वि - लो॒हि॒तः॒  & उ॒तैनं᳚ >\\
\hline
ए॒नं॒ गो॒पाः                 & गो॒पा, अ॑दृशन्न्\\
\hline
गो॒पा इति॑ गो-पाः          & अ॒दृ॒श॒न्नदृ॑शन्न्\\
\hline
अदृ॑शन्नुदहा॒र्यः॑              & उ॒द॒हा॒र्य॑ इत्यु॑द-हा॒र्यः॑\\
\hline
\end{longtable}
}
{\centering
{\small \eng{Sloka 9}} \\
\begin{longtable}{|c|c|}
\hline
उ॒तैनं᳚ >                    & ए॒नं॒ विश्वा᳚ >\\
\hline
विश्वा॑ भू॒तानि॑              & भू॒तानि॒ सः\\
\hline
स दृ॒ष्टः                   & दृ॒ष्टो मृ॑डयाति\\
\hline
मृ॒ड॒या॒ति॒ नः॒                & न॒ इति॑ नः\\
\hline
नमो॑ अस्तु                  & अ॒स्तु॒ नील॑ग्रीवाय\\
\hline
नील॑ग्रीवाय सहस्रा॒क्षाय॑      & नील॑ग्रीवा॒येति॒ नील॑ - ग्री॒वा॒य॒\\
\hline
स॒ह॒स्रा॒क्षाय॑ मी॒ढुषे᳚ >         & स॒ह॒स्रा॒क्षायेति॑ सहस्र - अ॒क्षाय॑\\
\hline
मी॒ढुष॒ इति॑ मी॒ढुषे᳚ >          & अथो॒ ये\\
\hline
\end{longtable}
}
{\centering
{\small \eng{Sloka 10}} \\
\begin{longtable}{|c|c|}
\hline
अथो॒, इत्यथो᳚ >              & ये अ॑स्य\\
\hline
अ॒स्य॒ सत्वा॑नः               & सत्वा॑नो॒ऽहं\\
\hline
अ॒हन्तेभ्यः॑                  & तेभ्यो॑ऽकरं\\
\hline
अ॒क॒र॒न्नमः॑                  & नम॒ इति॒ नमः॑\\
\hline
प्रमु॑ञ्च                    & मु॒ञ्च॒ धन्व॑नः\\
\hline
धन्व॑न॒स्त्वं                  & त्वमु॒भयोः᳚ >\\
\hline
उ॒भयो॒रार्त्नि॑योः            & आर्त्नि॑यो॒र्ज्यां\\
\hline
ज्यामि ति॒ज्यां               & याश्च॑\\
\hline
\end{longtable}
}
{\centering
{\small \eng{Sloka 11}} \\
\begin{longtable}{|c|c|}
\hline
च॒ ते॒ >                    & ते॒ हस्ते᳚ >\\
\hline
हस्त॒ इष॑वः                 & इष॑वः॒ परा᳚ >\\
\hline
परा॒ ताः                  & ता भ॑गवः\\
\hline
भ॒ग॒वो॒ व॒प॒                  & भ॒ग॒व॒ इति॑ भग - वः॒\\
\hline
व॒पेति॑ वप                  & अ॒व॒तत्य॒ धनुः॑\\
\hline
अ॒व॒तत्येत्य॑व - तत्य॑           & धनु॒स्त्वं\\
\hline
त्वꣳ सह॑स्राक्ष              & सह॑स्राक्ष॒ शते॑षुधे\\
\hline
सह॑स्रा॒क्षेति॒ सह॑स्र - अ॒क्ष॒     & शते॑षुध॒ इति॒ शत॑ - इ॒षु॒धे॒ >\\
\hline
\end{longtable}
}
{\centering
{\small \eng{Sloka 12}} \\
\begin{longtable}{|c|c|}
\hline
नि॒शीर्य॑ श॒ल्यानां᳚ >          & नि॒शीर्येति॑ नि - शीर्य॑\\
\hline
श॒ल्यानां॒ मुखा᳚ >             & मुखा॑ शि॒वः\\
\hline
शि॒वो नः॑                  & नः॒ सु॒मनाः᳚ >\\
\hline
सु॒मना॑ भव                  & सु॒मना॒ इति॑ सु - मनाः᳚ >\\
\hline
भ॒वेति॑ भव                  & विज्यं॒ धनुः॑\\
\hline
विज्य॒मिति॒ वि - ज्यं॒ >       & धनुः॑ कप॒र्दिनः॑\\
\hline
क॒प॒र्दिनो॒ विश॑ल्यः           & विश॑ल्यो॒ बाण॑वान्\\
\hline
विश॑ल्य॒ इति॒ वि - श॒ल्यः॒      & बाण॑वाꣳ उ॒त\\
\hline
बाण॑वा॒निति॒ बाण॑ - वा॒न्॒      & उ॒तेत्यु॒त\\
\hline
\end{longtable}
}
{\centering
{\small \eng{Sloka 13}} \\
\begin{longtable}{|c|c|}
\hline
अने॑शन्नस्य                  & अ॒स्ये ष॑वः\\
\hline
इष॑वः आ॒भुः                 & आ॒भुर॑स्य\\
\hline
अ॒स्य॒ नि॒ष॒ङ्गथिः॑             & नि॒ष॒ङ्गथि॒रिति॑ नि॒ष॒ङ्गथिः॑\\
\hline
या ते᳚ >                   & ते॒ हे॒तिः\\
\hline
हे॒तिर्मी॑ढुष्टम               & मी॒ढु॒ष्ट॒म॒ हस्ते᳚ >\\
\hline
मी॒ढु॒ष्ट॒मेति॑ मीढुः - त॒म॒       & हस्ते॑ ब॒भूव॑\\
\hline
ब॒भूव॑ ते                    & ते॒ धनुः॑\\
\hline
धनु॒रिति॒ धनुः॑               & तया॒ऽस्मान्\\
\hline
\end{longtable}
}
{\centering
{\small \eng{Sloka 14}} \\
\begin{longtable}{|c|c|}
\hline
अ॒स्मान्. वि॒श्वतः॑            & वि॒श्वत॒स्त्वं\\
\hline
त्वम॑य॒क्ष्मया᳚ >              & अ॒य॒क्ष्मया॒ परि॑\\
\hline
परि॑ब्भुज                   & भु॒जेति॑ भुज\\
\hline
नम॑स्ते                     & ते॒, अ॒स्तु॒\\
\hline
अ॒स्त्वायु॑धाय                & आयु॑धा॒याना॑ तताय\\
\hline
अना॑ तताय धृ॒ष्णवे᳚ >           & अना॑ तता॒येत्यना᳚ - त॒ता॒य॒\\
\hline
धृ॒ष्णव॒ इति॑ धृ॒ष्णवे᳚ >          & उ॒भाभ्या॑मु॒त\\
\hline
\end{longtable}
}
{\centering
{\small \eng{Sloka 15}} \\
\begin{longtable}{|c|c|}
\hline
उ॒त ते᳚ >                   & ते॒ नमः॑\\
\hline
नमो॑ बा॒हुभ्यां᳚ >             & बा॒हुभ्या॒न्तव॑\\
\hline
बा॒हुभ्या॒मिति॑ बा॒हु - भ्यां॒ >   & तव॒ धन्व॑ने\\
\hline
धन्व॑न॒ इति॒ धन्व॑ने            & परि॑ ते\\
\hline
ते॒ धन्व॑नः                  & धन्व॑नो हे॒तिः\\
\hline
हे॒तिर॒स्मान्                 & अ॒स्मान् वृ॑णक्तु\\
\hline
वृ॒ण॒क्तु॒ वि॒श्वतः॑              & वि॒श्वत॒ इति॑ वि॒श्वतः॑\\
\hline
\end{longtable}
}
{\centering
{\small \eng{Sloka 16}} \\
\begin{longtable}{|c|c|}
\hline
अथो॒ यः                   & अथो॒, इत्यथो᳚ >\\
\hline
य इ॑षु॒धिः                  & इ॒षु॒धिस्तव॑\\
\hline
इ॒षु॒धि रिती॑षु - धिः          & तवा॒रे\\
\hline
आ॒रे, अ॒स्मत्                  & अ॒स्मन्नि\\
\hline
निधे॑हि                    & धे॒हि॒ तं\\
\hline
तमिति॒ तं                  & \\
\hline
\end{longtable}
}
\subsection{\eng{Anuvaka 2}}
ओं नमो भगवते रुद्राय
{\centering
\begin{longtable}{|c|c|}
\hline
नमो॒ हिर॑ण्यबाहवे            & हिर॑ण्यबाहवे सेना॒न्ये᳚ >\\
\hline
हिर॑ण्यबाहव॒ इति॒ हिर॑ण्य - बा॒ह॒वे॒ >   & से॒ना॒न्ये॑ दि॒शां\\
\hline
से॒ना॒न्य॑ इति॑ सेना - न्ये᳚ >     & दि॒शाञ्च॑\\
\hline
च॒ पत॑ये                    & पत॑ये॒ नमः॑\\
\hline
नमो॒ नमः॑                  & नमो॑ वृ॒क्षेभ्यः॑\\
\hline
वृ॒क्षेभ्यो॒ हरि॑केशेभ्यः          & हरि॑केशेभ्यः पशू॒नां\\
\hline
हरि॑केशेभ्य॒ इति॒ हरि॑ - के॒शे॒भ्यः॒  & प॒शू॒नां पत॑ये\\
\hline
पत॑ये॒ नमः॑                  & नमो॒ नमः॑\\
\hline
नमः॑ स॒स्पिञ्ज॑राय            & स॒स्पिञ्ज॑राय॒ त्विषी॑मते\\
\hline
त्विषी॑मते पथी॒नां            & त्विषी॑मत॒ इति॒ त्विषि॑ - म॒ते॒ >\\
\hline
प॒थी॒नां पत॑ये                & पत॑ये॒ नमः॑\\
\hline
नमो॒ नमः॑                  & नमो॑ बभ्लु॒शाय॑\\
\hline
ब॒भ्लु॒शाय॑ विव्या॒धिने᳚ >        & वि॒व्या॒धिनेऽन् ना॑नां\\
\hline
वि॒व्या॒धिन॒ इति॑ वि - व्या॒धिने᳚ > & अन्ना॑नां॒ पत॑ये\\
\hline
पत॑ये॒ नमः॑                  & नमो॒ नमः॑\\
\hline
नमो॒ हरि॑केशाय              & हरि॑केशायो पवी॒तिने᳚ >\\
\hline
हरि॑केशा॒येति॒ हरि॑ - के॒शा॒य॒     & उ॒प॒वी॒तिने॑ पु॒ष्टानां᳚ >\\
\hline
उ॒प॒वी॒तिन॒ इत्यु॑प - वी॒तिने᳚ >   & पु॒ष्टानां॒ पत॑ये\\
\hline
पत॑ये॒ नमः॑                  & नमो॒ नमः॑\\
\hline
नमो॑ भ॒वस्य॑                 & भ॒वस्य॑ हे॒त्यै\\
\hline
हे॒त्यै जग॑तां                 & जग॑तां॒ पत॑ये\\
\hline
पत॑ये॒ नमः॑                  & नमो॒ नमः॑\\
\hline
नमो॑ रु॒द्राय॑                & रु॒द्राया॑तता॒विने᳚ >\\
\hline
आ॒त॒ता॒विने॒ क्षेत्रा॑णां          & आ॒त॒ता॒विन॒ इत्या᳚ - त॒ता॒विने᳚ >\\
\hline
क्षेत्रा॑णां॒ पत॑ये              & पत॑ये॒ नमः॑\\
\hline
नमो॒ नमः॑                  & नमः॑ सू॒ताय॑\\
\hline
सू॒तायाह॑न्त्याय              & अह॑न्त्याय॒ वना॑नां\\
\hline
वना॑नां॒ पत॑ये                & पत॑ये॒ नमः॑\\
\hline
नमो॒ नमः॑                  & नमो॒ रोहि॑ताय\\
\hline
रोहि॑ताय स्थ॒पत॑ये            & स्थ॒पत॑ये वृ॒क्षाणां᳚ >\\
\hline
वृ॒क्षाणां॒ पत॑ये               & पत॑ये॒ नमः॑\\
\hline
नमो॒ नमः॑                  & नमो॑ म॒न्त्रिणे᳚ >\\
\hline
म॒न्त्रिणे॑ वाणि॒जाय॑           & वा॒णि॒जाय॒ कक्षा॑णां\\
\hline
कक्षा॑णां॒ पत॑ये               & पत॑ये॒ नमः॑\\
\hline
नमो॒ नमः॑                  & नमो॑ भुव॒न्तये᳚ >\\
\hline
भु॒व॒न्तये॑ वारिवस्कृ॒ताय॑         & वा॒रि॒व॒स्कृ॒तायौष॑धीनां\\
\hline
वा॒रि॒व॒स्कृ॒तायेति॑ वारिवः - कृ॒ताय॑ & ओष॑धीनां॒ पत॑ये\\
\hline
पत॑ये॒ नमः॑                  & नमो॒ नमः॑\\
\hline
नम॑ उ॒च्चैर्घो॑षाय             & उ॒च्चैर्घो॑षायाक्र॒न्दय॑ते\\
\hline
उ॒च्चैर्घो॑षा॒येत्यु॒च्चैः - घो॒षा॒य॒   & आ॒क्र॒न्दय॑ते पत्ती॒नां\\
\hline
आ॒क्र॒न्दय॑त॒ इत्या᳚ - क्र॒न्दय॑ते    & प॒त्ती॒नां पत॑ये\\
\hline
पत॑ये॒ नमः॑                  & नमो॒ नमः॑\\
\hline
नमः॑ कृथ्स्नवी॒ताय॑            & कृ॒थ्स्न॒वी॒ताय॒ धाव॑ते\\
\hline
कृ॒थ्स्न॒वी॒तायेति॑ कृथ्स्न - वी॒ताय॑ & धावे॑ते॒ सत्व॑नां\\
\hline
सत्व॑नां॒ पत॑ये                & पत॑ये॒ नमः॑\\
\hline
नम॒ इति॒ नमः॑               & \\
\hline
\end{longtable}
}
\subsection{\eng{Anuvaka 3}}
{\centering
\begin{longtable}{|c|c|}
\hline
नमः॒ सह॑मानाय                & सह॑मानाय निव्या॒ धिने᳚ > \\
\hline
नि॒व्या॒धिन॑ आव्या॒धिनी॑नां        & नि॒व्या॒धिन॒ इति॑ नि - व्या॒धिने᳚ > \\
\hline
आ॒व्या॒धिनी॑नां॒ पत॑ये             & आ॒व्या॒धिनी॑ना॒मित्या᳚ - व्या॒धिनी॑नां \\
\hline
पत॑ये॒ नमः॑                    & नमो॒ नमः॑ \\
\hline
नमः॑ ककु॒भाय॑                  & क॒कु॒भाय॑ निष॒ङ्गिणे᳚ > \\
\hline
नि॒ष॒ङ्गिणे᳚ स्ते॒नानां᳚ >           & नि॒ष॒ङ्गिण॒ इति॑ नि - स॒ङ्गिने᳚ > \\
\hline
स्ते॒नानां॒ पत॑ये                 & पत॑ये॒ नमः॑ \\
\hline
नमो॒ नमः॑                    & नमो॑ निष॒ङ्गिणे᳚ > \\
\hline
नि॒ष॒ङ्गिण॑ इषुधि॒मते᳚ >           & नि॒ष॒ङ्गिण॒ इति॑ नि - स॒ङ्गिने᳚ > \\
\hline
इ॒षु॒धि॒मते॒ तस्क॑राणां             & इ॒षु॒धि॒मत॒ इती॑षुधि - मते᳚ > \\
\hline
तस्क॑राणां॒ पत॑ये                & पत॑ये॒ नमः॑ \\
\hline
नमो॒ नमः॑                    & नमो॒ वञ्च॑ते \\
\hline
वञ्च॑ते परि॒वञ्च॑ते               & प॒रि॒वञ्च॑ते स्तायू॒नां \\
\hline
प॒रि॒वञ्च॑त॒ इति॑ परि - वञ्च॑ते     & स्ता॒यू॒नां पत॑ये \\
\hline
पत॑ये॒ नमः॑                    & नमो॒ नमः॑ \\
\hline
नमो॑ निचे॒रवे᳚ >                & नि॒चे॒रवे॑ परिच॒राय॑ \\
\hline
नि॒चे॒रव॒ इति॑ नि - चे॒रवे᳚ >       & प॒रि॒च॒रायार॑ण्यानां \\
\hline
प॒रि॒च॒रायेति॑ परि - च॒राय॑       & अर॑ण्यानां॒ पत॑ये \\
\hline
पत॑ये॒ नमः॑                    & नमो॒ नमः॑ \\
\hline
नमः॑ सृका॒विभ्यः॑               & सृ॒का॒विभ्यो॒ जिघाꣳ॑सद्भ्यः \\
\hline
सृ॒का॒विभ्य॒ इति॑ सृका॒वि - भ्यः॒    & जिघाꣳ॑सद्भ्यो मुष्ण॒तां \\
\hline
जिघाꣳ॑सद्भ्य॒ इति॒ जिघाꣳ॑सत्-भ्यः॒  & मु॒ष्ण॒तां पत॑ये \\
\hline
पत॑ये॒ नमः॑                    & नमो॒ नमः॑ \\
\hline
नमो॑ऽसि॒मद्भ्यः॑                & अ॒सि॒मद्भ्यो॒ नक्तं᳚ > \\
\hline
अ॒सि॒मद्भ्य॒ इत्य॑सि॒मत् - भ्यः॒      & नक्त॒ञ्चर॑द्भ्यः \\
\hline
चर॑द्भ्यः प्रकृ॒न्तानां᳚ >          & चर॑द्भ्य॒ इति॒ चर॑त् - भ्यः॒ \\
\hline
प्र॒कृ॒न्तानां॒ पत॑ये               & प्र॒कृ॒न्ताना॒मिति॑ प्र - कृ॒न्तानां᳚ > \\
\hline
पत॑ये॒ नमः॑                    & नमो॒ नमः॑ \\
\hline
नम॑ उष्णी॒षिणे᳚ >               & उ॒ष्णी॒षिणे॑ गिरिच॒राय॑ \\
\hline
गि॒रि॒च॒राय॑ कुलु॒ञ्चानां᳚ >         & गि॒रि॒च॒रायेति॑ गिरि - च॒राय॑ \\
\hline
कु॒लु॒ञ्चानां॒ पत॑ये                & पत॑ये॒ नमः॑ \\
\hline
नमो॒ नमः॑                    & नम॒ इषु॑मद्भ्यः \\
\hline
इषु॑मद्भ्यो धन्वा॒विभ्यः॑          & इषु॑मद्भ्य॒ इतीषु॑मत् - भ्यः॒ \\
\hline
ध॒न्वा॒विभ्य॑श्च                 & ध॒न्वा॒विभ्य॒ इति॑ धन्वा॒वि - भ्यः॒ \\
\hline
च॒ वः॒                       & वो॒ नमः॑ \\
\hline
नमो॒ नमः॑                    & नम॑ आतन्वा॒नेभ्यः॑ \\
\hline
आ॒त॒न्वा॒नेभ्यः॑ प्रति॒दधा॑नेभ्यः      & आ॒त॒न्वा॒नेभ्य॒ इत्या᳚ - त॒न्वा॒नेभ्यः॑ \\
\hline
प्र॒ति॒दधा॑नेभ्यश्च               & प्र॒ति॒दधा॑नेभ्य॒ इति॑ प्रति - दधा॑नेभ्यः \\
\hline
च॒ वः॒                       & वो॒ नमः॑ \\
\hline
नमो॒ नमः॑                    & नम॑ आ॒यच्छ॑द्भ्यः \\
\hline
आ॒यच्छ॑द्भ्यो विसृ॒जद्भ्यः॑          & आ॒यच्छ॑द्भ्य॒ इत्या॒यच्छ॑त् - भ्यः॒ \\
\hline
वि॒सृ॒जद्भ्य॑श्च                  & वि॒सृ॒जद्भ्य॒ इति॑ विसृ॒जत् - भ्यः॒ \\
\hline
च॒ वः॒                       & वो॒ नमः॑ \\
\hline
नमो॒ नमः॑                    & नमोऽस्य॑द्भ्यः \\
\hline
अस्य॑द्भ्यो॒ विद्ध्य॑द्भ्यः          & अस्य॑द्भ्य॒ इत्यस्य॑त् - भ्यः॒ \\
\hline
विद्ध्य॑द्भ्यश्च                 & विद्ध्य॑द्भ्य॒ इति॒ विद्ध्य॑त् - भ्यः॒ \\
\hline
च॒ वः॒                       & वो॒ नमः॑ \\
\hline
नमो॒ नमः॑                    & नम॒ आसी॑नेभ्यः \\
\hline
आसी॑नेभ्यः॒ शया॑नेभ्यः            & शया॑नेभ्यश्च \\
\hline
च॒ वः॒                       & वो॒ नमः॑ \\
\hline
नमो॒ नमः॑                    & नमः॑ स्व॒पद्भ्यः॑ \\
\hline
स्व॒पद्भ्यो॒ जाग्र॑द्भ्यः           & स्व॒पद्भ्य॒ इति॑ स्व॒पत् - भ्यः॒ \\
\hline
जाग्र॑द्भ्यश्च                  & जाग्र॑द्भ्य॒ इति॒ जाग्र॑त् - भ्यः॒ \\
\hline
च॒ वः॒                       & वो॒ नमः॑ \\
\hline
नमो॒ नमः॑                    & नम॒स्तिष्ठ॑द्भ्यः \\
\hline
तिष्ठ॑द्भ्यो॒ धाव॑द्भ्यः           & तिष्ठ॑द्भ्य॒ इति॒ तिष्ठ॑त् - भ्यः॒ \\
\hline
धाव॑द्भ्यश्च                   & धाव॑द्भ्य॒ इति॒ धाव॑त् - भ्यः॒ \\
\hline
च॒ वः॒                       & वो॒ नमः॑ \\
\hline
नमो॒ नमः॑                    & नमः॑ स॒भाभ्यः॑ \\
\hline
स॒भाभ्यः॑ स॒भाप॑तिभ्यः           & स॒भाप॑तिभ्यश्च \\
\hline
स॒भाप॑तिभ्य॒ इति॑ स॒भाप॑ति - भ्यः॒  & च॒ वः॒ \\
\hline
वो॒ नमः॑                     & नमो॒ नमः॑ \\
\hline
नमो॒ अश्वे᳚भ्यः                 & अश्वे॒भ्योऽश्व॑पतिभ्यः \\
\hline
अश्व॑पतिभ्यश्च                 & अश्व॑पतिभ्य॒ इत्यश्व॑पति - भ्यः॒ \\
\hline
च॒ वः॒                       & वो॒ नमः॑ \\
\hline
नम॒ इति॒ नमः॑                 & \\
\hline
\end{longtable}
}
\subsection{\eng{Anuvaka 4}}
{\centering
\begin{longtable}{|c|c|}
\hline
नम॑ आव्या॒ धिनी᳚भ्यः               & आ॒व्या॒ धिनी᳚भ्यो वि॒विद्ध्य॑न्तीभ्यः\\
\hline
आ॒व्या॒ धिनी᳚भ्य॒ इत्या᳚ - व्या॒ धिनी᳚भ्यः   & वि॒विद्ध्य॑न्तीभ्यश्च\\
\hline
वि॒विद्ध्य॑न्तीभ्य॒ इति॑ वि -\\
विद्ध्य॑न्तीभ्यः                   & च॒ वः॒\\
\hline
वो॒ नमः॑                        & नमो॒ नमः॑\\
\hline
नम॒ उग॑णाभ्यः                    & उग॑णाभ्यस्तृꣳह॒तीभ्यः॑\\
\hline
तृ॒ꣳ॒ह॒तीभ्य॑श्च                     & च॒ वः॒\\
\hline
वो॒ नमः॑                        & नमो॒ नमः॑\\
\hline
नमो॑ गृ॒थ्सेभ्यः॑                    & गृ॒थ्सेभ्यो॑ गृ॒थ्सप॑तिभ्यः\\
\hline
गृ॒थ्सप॑तिभ्यश्च                    & गृ॒थ्सप॑तिभ्य॒ इति॑ गृ॒थ्सप॑ति - भ्यः॒\\
\hline
च॒ वः॒                          & वो॒ नमः॑\\
\hline
नमो॒ नमः॑                       & नमो॒ व्राते᳚भ्यः\\
\hline
व्राते᳚भ्यो॒ व्रात॑पतिभ्यः            & व्रात॑पतिभ्यश्च\\
\hline
व्रात॑पतिभ्य॒ इति॒ व्रात॑पति - भ्यः॒   & च॒ वः॒\\
\hline
वो॒ नमः॑                        & नमो॒ नमः॑\\
\hline
नमो॑ ग॒णेभ्यः॑                     & ग॒णेभ्यो॑ ग॒णप॑तिभ्यः\\
\hline
ग॒णप॑तिभ्यश्च                     & ग॒णप॑तिभ्य॒ इति॑ ग॒णप॑ति - भ्यः॒\\
\hline
च॒ वः॒                          & वो॒ नमः॑\\
\hline
नमो॒ नमः॑                       & नमो॒ विरू॑पेभ्यः\\
\hline
विरू॑पेभ्यो वि॒श्वरू॑पेभ्यः             & विरू॑पेभ्य॒ इति॒ वि - रू॒पे॒भ्यः॒\\
\hline
वि॒श्वरू॑पेभ्यश्च                    & वि॒श्वरू॑पेभ्य॒ इति॑ वि॒श्व - रू॒पे॒भ्यः॒\\
\hline
च॒ वः॒                          & वो॒ नमः॑\\
\hline
नमो॒ नमः॑                       & नमो॑ म॒हद्भ्यः॑\\
\hline
म॒हद्भ्यः॑, क्षुल्ल॒केभ्यः॑               & म॒हद्भ्य॒ इति॑ म॒हत् - भ्यः॒\\
\hline
क्षु॒ल्ल॒केभ्य॑श्च                     & च॒ वः॒\\
\hline
वो॒ नमः॑                        & नमो॒ नमः॑\\
\hline
नमो॑ र॒थिभ्यः॑                    & र॒थिभ्यो॑ऽर॒थेभ्यः॑\\
\hline
र॒थिभ्य॒ इति॑ र॒थि - भ्यः॒           & अ॒र॒थेभ्य॑श्च\\
\hline
च॒ वः॒                          & वो॒ नमः॑\\
\hline
नमो॒ नमः॑                       & नमो॒ रथे᳚भ्यः\\
\hline
रथे᳚भ्यो॒ रथ॑पतिभ्यः                & रथ॑पतिभ्यश्च\\
\hline
रथ॑पतिभ्य॒ इति॒ रथ॑पति - भ्यः॒       & च॒ वः॒\\
\hline
वो॒ नमः॑                        & नमो॒ नमः॑\\
\hline
नमः॒ सेना᳚भ्यः                    & सेना᳚भ्यः सेना॒निभ्यः॑\\
\hline
से॒ना॒निभ्य॑श्च                     & से॒ना॒निभ्य॒ इति॑ सेना॒नि - भ्यः॒\\
\hline
च॒ वः॒                          & वो॒ नमः॑\\
\hline
नमो॒ नमः॑                       & नमः॑ क्ष॒त्तृभ्यः॑\\
\hline
क्ष॒त्तृभ्यः॑ सङ्ग्रही॒तृभ्यः॑            & क्ष॒त्तृभ्यः॒ इति॑ क्ष॒त्तृ - भ्यः॒\\
\hline
स॒ङ्ग्र॒ही॒तृभ्य॑श्च                   & स॒ङ्ग्र॒ही॒तृभ्य॒ इति॑ सङ्ग्रही॒तृ - भ्यः॒\\
\hline
च॒ वः॒                          & वो॒ नमः॑\\
\hline
नमो॒ नमः॑                       & नम॒स्तक्ष॑भ्यः\\
\hline
तक्ष॑भ्यो रथका॒रेभ्यः॑               & तक्ष॑भ्य॒ इति॒ तक्ष॑ - भ्यः॒\\
\hline
र॒थ॒का॒रेभ्य॑श्च                     & र॒थ॒का॒रेभ्य॒ इति॑ रथ - का॒रेभ्यः॑\\
\hline
च॒ वः॒                          & वो॒ नमः॑\\
\hline
नमो॒ नमः॑                       & नमः॒ कुला॑लेभ्यः\\
\hline
कुला॑लेभ्यः क॒र्मारे᳚भ्यः              & क॒र्मारे᳚भ्यश्च\\
\hline
च॒ वः॒                          & वो॒ नमः॑\\
\hline
नमो॒ नमः॑                       & नमः॑ पु॒ञ्जिष्टे᳚भ्यः\\
\hline
पु॒ञ्जिष्टे᳚भ्यो निषा॒देभ्यः॑            & नि॒षा॒देभ्य॑श्च\\
\hline
च॒ वः॒                          & वो॒ नमः॑\\
\hline
नमो॒ नमः॑                       & नम॑ इषु॒कृद्भ्यः॑\\
\hline
इ॒षु॒कृद्भ्यो॑ धन्व॒कृद्भ्यः॑              & इ॒षु॒कृद्भ्य॒ इती॑षु॒कृत् - भ्यः॒\\
\hline
ध॒न्व॒कृद्भ्य॑श्च                     & ध॒न्व॒कृद्भ्य॒ इति॑ धन्व॒कृत् - भ्यः॒\\
\hline
च॒ वः॒                          & वो॒ नमः॑\\
\hline
नमो॒ नमः॑                       & नमो॑ मृग॒युभ्यः॑\\
\hline
मृ॒ग॒युभ्यः॑ श्व॒निभ्यः॑                & मृ॒ग॒युभ्य॒ इति॑ मृग॒यु - भ्यः॒\\
\hline
श्व॒निभ्य॑श्च                      & श्व॒निभ्य॒ इति॑ श्व॒नि - भ्यः॒\\
\hline
च॒ वः॒                          & वो॒ नमः॑\\
\hline
नमो॒ नमः॑                       & नमः॒ श्वभ्यः॑\\
\hline
श्वभ्य॒ श्श्वप॑तिभ्यः                & श्वभ्यः॒ इति॒ श्व - भ्यः॒\\
\hline
श्वप॑तिभ्यश्च                     & श्वप॑तिभ्य॒ इति॒ श्वप॑ति - भ्यः॒\\
\hline
च॒ वः॒                          & वो॒ नमः॑\\
\hline
नम॒ इति॒ नमः॑\\
\hline
\end{longtable}
}
\subsection{\eng{Anuvaka 5}}
{\centering
\begin{longtable}{|c|c|}
\hline
नमो॑ भ॒वाय॑                 & भ॒वाय॑ च\\
\hline
च॒ रु॒द्राय॑                  & रु॒द्राय॑ च\\
\hline
च॒ नमः॑                    & नम॑श्श॒र्वाय॑\\
\hline
श॒र्वाय॑ च                  & च॒ प॒शु॒ पत॑ये\\
\hline
प॒शु॒पत॑ये च                  & प॒शु॒पत॑य॒ इति॑ पशु - पत॑ये\\
\hline
च॒ नमः॑                    & नमो॒ नील॑ग्रीवाय\\
\hline
नील॑ग्रीवाय च              & नील॑ग्रीवा॒येति॒ नील॑ - ग्री॒वा॒य॒\\
\hline
च॒ शि॒ति॒कण्ठा॑य              & शि॒ति॒कण्ठा॑य च\\
\hline
शि॒ति॒कण्ठा॒येति॑ शिति - कण्ठा॑य & च॒ नमः॑\\
\hline
नमः॑ कप॒र्दिने᳚ >             & क॒प॒र्दिने॑ च\\
\hline
च॒ व्यु॑प्तकेशाय               & व्यु॑प्तकेशाय च\\
\hline
व्यु॑प्तकेशा॒येति॒ व्यु॑प्त - के॒शा॒य॒   & च॒ नमः॑\\
\hline
नमः॑ सहस्रा॒क्षाय॑            & स॒ह॒स्रा॒क्षाय॑ च\\
\hline
स॒ह॒स्रा॒क्षायेति॑ सहस्र - अ॒क्षाय॑ & च॒ श॒तध॑न्वने\\
\hline
श॒तध॑न्वने च                 & श॒तध॑न्वन॒ इति॑ श॒त - ध॒न्व॒ने॒ >\\
\hline
च॒ नमः॑                    & नमो॑ गिरि॒शाय॑\\
\hline
गि॒रि॒शाय॑ च                & च॒ शि॒पि॒वि॒ष्टाय॑\\
\hline
शि॒पि॒वि॒ष्टाय॑ च             & शि॒पि॒वि॒ष्टायेति॑ शिपि - वि॒ष्टाय॑\\
\hline
च॒ नमः॑                    & नमो॑ मी॒ढुष्ट॑माय\\
\hline
मी॒ढुष्ट॑माय च               & मी॒ढुष्ट॑मा॒येति॑ मी॒ढुः - त॒मा॒य॒\\
\hline
चेषु॑मते                     & इषु॑मते च\\
\hline
इषु॑मत॒ इतीषु॑ - म॒ते॒ >         & च॒ नमः॑\\
\hline
नमो᳚ ह्र॒स्वाय॑               & ह्र॒स्वाय॑ च\\
\hline
च॒ वा॒म॒नाय॑                 & वा॒म॒नाय॑ च\\
\hline
च॒ नमः॑                    & नमो॑ बृह॒ते\\
\hline
बृ॒ह॒ते च॑                    & च॒ वर्.षी॑यसे\\
\hline
वर्.षी॑यसे च                & च॒ नमः॑\\
\hline
नमो॑ वृ॒द्धाय॑                & वृ॒द्धाय॑ च\\
\hline
च॒ स॒म्ँवृद्ध्व॑ने                & स॒म्ँवृद्ध्व॑ने च\\
\hline
स॒म्ँवृद्ध्व॑न॒ इति॑ सं - वृद्ध्व॑ने    & च॒ नमः॑\\
\hline
नमो॒ अग्रि॑याय              & अग्रि॑याय च\\
\hline
च॒ प्र॒थ॒माय॑                 & प्र॒थ॒माय॑ च\\
\hline
च॒ नमः॑                    & नम॑ आ॒शवे᳚ >\\
\hline
आ॒शवे॑ च                    & चा॒जि॒राय॑\\
\hline
अ॒जि॒राय॑ च                 & च॒ नमः॑\\
\hline
नमः॒ शीघ्रि॑याय             & शीघ्रि॑याय च\\
\hline
च॒ शीभ्या॑य                 & शीभ्या॑य च\\
\hline
च॒ नमः॑                    & नम॑ ऊ॒र्म्या॑य\\
\hline
ऊ॒र्म्या॑य च                 & चा॒व॒स्व॒न्या॑य\\
\hline
अ॒व॒स्व॒न्या॑य च               & अ॒व॒स्व॒न्या॑येत्य॑व - स्व॒न्या॑य\\
\hline
च॒ नमः॑                    & नमः॑ स्रोत॒स्या॑य\\
\hline
स्रो॒त॒स्या॑य च               & च॒ द्वीप्या॑य\\
\hline
द्वीप्या॑य च                & चेति॑ च\\
\hline
\end{longtable}
}
\subsection{\eng{Anuvaka 6}}
{\centering
\begin{longtable}{|c|c|}
\hline
नमो᳚ ज्ये॒ष्ठाय॑                  & ज्ये॒ष्ठाय॑ च\\
\hline
च॒ क॒नि॒ष्ठाय॑                   & क॒नि॒ष्ठाय॑ च\\
\hline
च॒ नमः॑                       & नमः॑ पूर्व॒जाय॑\\
\hline
पू॒र्व॒जाय॑ च                    & पू॒र्व॒जायेति॑ पूर्व - जाय॑\\
\hline
चा॒प॒र॒जाय॑                     & अ॒प॒र॒जाय॑ च\\
\hline
अ॒प॒र॒जायेत्य॑पर - जाय॑            & च॒ नमः॑\\
\hline
नमो॑ मद्ध्य॒माय॑                 & म॒द्ध्य॒माय॑ च\\
\hline
चा॒प॒ग॒ल्भाय॑                    & अ॒प॒ग॒ल्भाय॑ च\\
\hline
अ॒प॒ग॒ल्भायेत्य॑प - ग॒ल्भाय॑          & च॒ नमः॑\\
\hline
नमो॑ जघ॒न्या॑य                  & ज॒घ॒न्या॑य च\\
\hline
च॒ बुद्ध्नि॑याय                  & बुद्ध्नि॑याय च\\
\hline
च॒ नमः॑                       & नम॑स्सो॒भ्या॑य\\
\hline
सो॒भ्या॑य च                    & च॒ प्र॒ति॒स॒र्या॑य\\
\hline
प्र॒ति॒स॒र्या॑य च                 & प्र॒ति॒स॒र्या॑येति॑ प्रति - स॒र्या॑य\\
\hline
च॒ नमः॑                       & नमो॒ याम्या॑य\\
\hline
याम्या॑य च                    & च॒ क्षेम्या॑य\\
\hline
क्षेम्या॑य च                    & च॒ नमः॑\\
\hline
नम॑ उ॒र्व॒र्या॑य                  & उ॒र्व॒र्या॑य च\\
\hline
च॒ खल्या॑य                     & खल्या॑य च\\
\hline
च॒ नमः॑                       & नमः॒ श्लोक्या॑य\\
\hline
श्लोक्या॑य च                   & चा॒व॒सा॒न्या॑य\\
\hline
अ॒व॒सा॒न्या॑य च                  & अ॒व॒सा॒न्या॑येत्य॑व - सा॒न्या॑य\\
\hline
च॒ नमः॑                       & नमो॒ वन्या॑य\\
\hline
वन्या॑य च                     & च॒ कक्ष्या॑य\\
\hline
कक्ष्या॑य च                    & च॒ नमः॑\\
\hline
नमः॑ श्र॒वाय॑                   & श्र॒वाय॑ च\\
\hline
च॒ प्र॒ति॒श्र॒वाय॑                 & प्र॒ति॒श्र॒वाय॑ च\\
\hline
प्र॒ति॒श्र॒वायेति॑ प्रति - श्र॒वाय॑    & च॒ नमः॑\\
\hline
नम॑ आ॒शुषे॑णाय                   & आ॒शुषे॑णाय च\\
\hline
आ॒शुषे॑णा॒येत्या॒शु - से॒ना॒य॒           & चा॒शुर॑थाय\\
\hline
आ॒शुर॑थाय च                    & आ॒शुर॑था॒येत्या॒शु - र॒था॒य॒\\
\hline
च॒ नमः॑                       & नमः॒ शूरा॑य\\
\hline
शूरा॑य च                      & चा॒व॒भि॒न्द॒ते\\
\hline
अ॒व॒भि॒न्द॒ते च॑                   & अ॒व॒भि॒न्द॒त इत्य॑व - भि॒न्द॒ते\\
\hline
च॒ नमः॑                       & नमो॑ व॒र्मिणे᳚ >\\
\hline
व॒र्मिणे॑ च                     & च॒ व॒रू॒थिने᳚ >\\
\hline
व॒रू॒थिने॑ च                     & च॒ नमः॑\\
\hline
नमो॑ बि॒ल्मिने᳚ >                & बि॒ल्मिने॑ च\\
\hline
च॒ क॒व॒चिने᳚ >                   & क॒व॒चिने॑ च\\
\hline
च॒ नमः॑                       & नमः॑ श्रु॒ताय॑\\
\hline
श्रु॒ताय॑ च                     & च॒ श्रु॒त॒से॒नाय॑\\
\hline
श्रु॒त॒से॒नाय॑ च                   & श्रु॒त॒से॒नायेति॑ श्रुत - से॒नाय॑\\
\hline
चेति॑ च                       & \\
\hline
\end{longtable}
}
\subsection{\eng{Anuvaka 7}}
{\centering
\begin{longtable}{|c|c|}
\hline
नमो॑ दुन्दु॒भ्या॑य              & दु॒न्दु॒भ्या॑य च\\
\hline
चा॒ह॒न॒न्या॑य                 & आ॒ह॒न॒न्या॑य च\\
\hline
आ॒ह॒न॒न्या॑येत्या᳚ - ह॒न॒न्या॑य      & च॒ नमः॑\\
\hline
नमो॑ धृ॒ष्णवे᳚                 & धृ॒ष्णवे॑ च\\
\hline
च॒ प्र॒मृ॒शाय॑                 & प्र॒मृ॒शाय॑ च\\
\hline
प्र॒मृ॒शायेति॑ प्र - मृ॒शाय॑       & च॒ नमः॑\\
\hline
नमो॑ दू॒ताय॑                 & दू॒ताय॑ च\\
\hline
च॒ प्रहि॑ताय                & प्रहि॑ताय च\\
\hline
प्रहि॑ता॒येति॒ प्र - हि॒ता॒य॒     & च॒ नमः॑\\
\hline
नमो॑ निष॒ङ्गिणे᳚ >            & नि॒ष॒ङ्गिणे॑ च\\
\hline
नि॒ष॒ङ्गिण॒ इति॑ नि - स॒ङ्गिने᳚ > & चे॒षु॒धि॒मते᳚ >\\
\hline
इ॒षु॒धि॒मते॑ च                 & इ॒षु॒धि॒मत॒ इती॑षुधि - मते᳚ >\\
\hline
च॒ नमः॑                    & नम॑ स्ती॒क्ष्णेष॑वे\\
\hline
ती॒क्ष्णेष॑वे च                & ती॒क्ष्णेष॑व॒ इति॑ ती॒क्ष्ण - इ॒ष॒वे॒ >\\
\hline
चा॒यु॒धिने᳚ >                 & आ॒यु॒धिने॑ च\\
\hline
च॒ नमः॑                    & नमः॑ स्वायु॒धाय॑\\
\hline
स्वा॒यु॒धाय॑ च                & स्वा॒यु॒धायेति॑ सु - आ॒यु॒धाय॑\\
\hline
च॒ सु॒धन्व॑ने                  & सु॒धन्व॑ने च\\
\hline
सु॒धन्व॑न॒ इति॑ सु - धन्व॑ने       & च॒ नमः॑\\
\hline
नम॒स्स्रुत्या॑य                & स्रुत्या॑य च\\
\hline
च॒ पथ्या॑य                  & पथ्या॑य च\\
\hline
च॒ नमः॑                    & नमः॑ का॒ट्या॑य\\
\hline
का॒ट्या॑य च                 & च॒ नी॒प्या॑य\\
\hline
नी॒प्या॑य च                 & च॒ नमः॑\\
\hline
नमः॒ सूद्या॑य                & सूद्या॑य च\\
\hline
च॒ स॒र॒स्या॑य                 & स॒र॒स्या॑य च\\
\hline
च॒ नमः॑                    & नमो॑ ना॒द्याय॑\\
\hline
ना॒द्याय॑ च                 & च॒ वै॒श॒न्ताय॑\\
\hline
वै॒श॒न्ताय॑ च                 & च॒ नमः॑\\
\hline
नमः॒ कूप्या॑य                & कूप्या॑य च\\
\hline
चा॒व॒ट्या॑य                  & अ॒व॒ट्या॑य च\\
\hline
च॒ नमः॑                    & नमो॒ वर्ष्या॑य\\
\hline
वर्ष्या॑य च                 & चा॒व॒र्ष्याय॑\\
\hline
अ॒व॒र्ष्याय॑ च                & च॒ नमः॑\\
\hline
नमो॑ मे॒घ्या॑य                & मे॒घ्या॑य च\\
\hline
च॒ वि॒द्यु॒त्या॑य               & वि॒द्यु॒त्या॑य च\\
\hline
वि॒द्यु॒त्या॑येति॑ वि - द्यु॒त्या॑य   & च॒ नमः॑\\
\hline
नम॑ ई॒द्ध्रिया॑य              & ई॒द्ध्रिया॑य च\\
\hline
चा॒त॒प्या॑य                  & आ॒त॒प्या॑य च\\
\hline
आ॒त॒प्या॑येत्या᳚ - त॒प्या॑य        & च॒ नमः॑\\
\hline
नमो॒ वात्या॑य               & वात्या॑य च\\
\hline
च॒ रेष्मि॑याय                & रेष्मि॑याय च\\
\hline
च॒ नमः॑                    & नमो॑ वास्त॒व्या॑य\\
\hline
वा॒स्त॒व्या॑य च               & च॒ वा॒स्तु॒पाय॑\\
\hline
वा॒स्तु॒पाय॑ च                & वा॒स्तु॒पायेति॑ वास्तु - पाय॑\\
\hline
चेति॑ च                    & \\
\hline
\end{longtable}
}
\subsection{\eng{Anuvaka 8}}
{\centering
\begin{longtable}{|c|c|}
\hline
नमः॒ सोमा॑य               & सोमा॑य च\\
\hline
च॒ रु॒ द्राय॑                & रु॒द्राय॑ च\\
\hline
च॒ नमः॑                   & नम॑स्ता॒म्राय॑\\
\hline
ता॒म्राय॑ च                & चा॒रु॒णाय॑\\
\hline
अ॒रु॒णाय॑ च                 & च॒ नमः॑\\
\hline
नम॑श्श॒ङ्गाय॑                & श॒ङ्गाय॑ च\\
\hline
च॒ प॒शु॒पत॑ये                 & प॒शु॒पत॑ये च\\
\hline
प॒शु॒पत॑य॒ इति॑ पशु - पत॑ये      & च॒ नमः॑\\
\hline
नम॑ उ॒ग्राय॑                & उ॒ग्राय॑ च\\
\hline
च॒ भी॒माय॑                 & भी॒माय॑ च\\
\hline
च॒ नमः॑                   & नमो॑ अग्रेव॒धाय॑\\
\hline
अ॒ग्रे॒व॒धाय॑ च               & अ॒ग्रे॒व॒धायेत्य॑ग्रे - व॒धाय॑\\
\hline
च॒ दू॒रे॒व॒धाय॑                & दू॒रे॒व॒धाय॑ च\\
\hline
दू॒रे॒व॒धायेति॑ दूरे - व॒धाय॑      & च॒ नमः॑\\
\hline
नमो॑ ह॒न्त्रे                & ह॒न्त्रे च॑\\
\hline
च॒ हनी॑यसे                 & हनी॑यसे च\\
\hline
च॒ नमः॑                   & नमो॑ वृ॒क्षेभ्यः॑\\
\hline
वृ॒क्षेभ्यो॒ हरि॑केशेभ्यः         & हरि॑केशेभ्यो॒ नमः॑\\
\hline
हरि॑केशेभ्य॒ इति॒ हरि॑ - के॒शे॒भ्यः॒ & नम॑स्ता॒राय॑\\
\hline
ता॒राय॒ नमः॑               & नमः॑ शं॒भवे᳚ >\\
\hline
शं॒भवे॑ च                   & शं॒भव॒ इति॑ शं - भवे᳚ >\\
\hline
च॒ म॒यो॒भवे᳚ >               & म॒यो॒भवे॑ च\\
\hline
म॒यो॒भव॒ इति॑ मयः - भवे᳚ >    & च॒ नमः॑\\
\hline
नम॑श्शङ्क॒राय॑               & श॒ङ्क॒राय॑ च\\
\hline
श॒ङ्क॒रायेति॑ शं - क॒राय॑       & च॒ म॒य॒स्क॒राय॑\\
\hline
म॒य॒स्क॒राय॑ च               & म॒य॒स्क॒रायेति॑ मयः - क॒राय॑\\
\hline
च॒ नमः॑                   & नम॑श्शि॒वाय॑\\
\hline
शि॒वाय॑ च                 & च॒ शि॒वत॑राय\\
\hline
शि॒वत॑राय च               & शि॒वत॑रा॒येति॑ शि॒व - त॒रा॒य॒\\
\hline
च॒ नमः॑                   & नम॒स्तीर्थ्या॑य\\
\hline
तीर्त्थ्या॑य च              & च॒ कूल्या॑य\\
\hline
कूल्या॑य च                 & च॒ नमः॑\\
\hline
नमः॑ पा॒र्या॑य              & पा॒र्या॑य च\\
\hline
चा॒वा॒र्या॑य                & अ॒वा॒र्या॑य च\\
\hline
च॒ नमः॑                   & नमः॑ प्र॒तर॑णाय\\
\hline
प्र॒तर॑णाय च               & प्र॒तर॑णा॒येति॑ प्र - तर॑णाय\\
\hline
चो॒त्तर॑णाय                & उ॒त्तर॑णाय च\\
\hline
उ॒त्तर॑णा॒येत्यु॑त् - तर॑णाय      & च॒ नमः॑\\
\hline
नम॑ आता॒र्या॑य              & आ॒ता॒र्या॑य च\\
\hline
आ॒ता॒र्या॑येत्या᳚ - ता॒र्या॑य     & चा॒ला॒द्या॑य\\
\hline
आ॒ला॒द्या॑य च               & आ॒ला॒द्या॑येत्या᳚ - ला॒द्या॑य\\
\hline
च॒ नमः॑                   & नम॒श्शष्प्या॑य\\
\hline
शष्प्या॑य च                & च॒ फेन्या॑य\\
\hline
फेन्या॑य च                 & च॒ नमः॑\\
\hline
नमः॑ सिक॒त्या॑य             & सि॒क॒त्या॑य च\\
\hline
च॒ प्र॒वा॒ह्या॑य              & प्र॒वा॒ह्या॑य च\\
\hline
प्र॒वा॒ह्या॑येति॑ प्र - वा॒ह्या॑य  & चेति॑ च\\
\hline
\end{longtable}
}
\subsection{\eng{Anuvaka 9}}
{\centering
\begin{longtable}{|c|c|}
\hline
नम॑ इरि॒ण्या॑य                   & इ॒रि॒ण्या॑य च\\
\hline
च॒ प्र॒प॒थ्या॑य                    & प्र॒प॒थ्या॑य च\\
\hline
प्र॒प॒थ्या॑येति॑ प्र - प॒थ्या॑य         & च॒ नमः॑\\
\hline
नमः॑ किꣳशि॒लाय॑                 & कि॒ꣳ॒शि॒लाय॑ च\\
\hline
च॒ क्षय॑णाय                     & क्षय॑णाय च\\
\hline
च॒ नमः॑                        & नमः॑ कप॒र्दिने᳚ >\\
\hline
क॒प॒र्दिने॑ च                     & च॒ पु॒ल॒स्तये᳚ >\\
\hline
पु॒ल॒स्तये॑ च                      & च॒ नमः॑\\
\hline
नमो॒ गोष्ठ्या॑य                  & गोष्ठ्या॑य च\\
\hline
गोष्ठ्या॒येति॒ गो - स्थ्या॒य॒         & च॒ गृह्या॑य\\
\hline
गृह्या॑य च                      & च॒ नमः॑\\
\hline
नम॒स्तल्प्या॑य                    & तल्प्या॑य च\\
\hline
च॒ गेह्या॑य                      & गेह्या॑य च\\
\hline
च॒ नमः॑                        & नमः॑ का॒ट्या॑य\\
\hline
का॒ट्या॑य च                     & च॒ ग॒ह्व॒रे॒ष्ठाय॑\\
\hline
ग॒ह्व॒रे॒ष्ठाय॑ च                   & ग॒ह्व॒रे॒ष्ठायेति॑ गह्वरे - स्थाय॑\\
\hline
च॒ नमः॑                        & नमो᳚ ह्रद॒य्या॑य\\
\hline
ह्र॒द॒य्या॑य च                    & च॒ नि॒वे॒ष्प्या॑य\\
\hline
नि॒वे॒ष्प्या॑य च                   & नि॒वे॒ष्प्या॑येति॑ नि - वे॒ष्प्या॑य\\
\hline
च॒ नमः॑                        & नमः॑ पाꣳस॒व्या॑य\\
\hline
पा॒ꣳ॒स॒व्या॑य च                   & च॒ र॒ज॒स्या॑य\\
\hline
र॒ज॒स्या॑य च                     & च॒ नमः॑\\
\hline
नम॒श्शुष्क्या॑य                    & शुष्क्या॑य च\\
\hline
च॒ ह॒रि॒त्या॑य                    & ह॒रि॒त्या॑य च\\
\hline
च॒ नमः॑                        & नमो॒ लोप्या॑य\\
\hline
लोप्या॑य च                     & चो॒ल॒प्या॑य\\
\hline
उ॒ल॒प्या॑य च                     & च॒ नमः॑\\
\hline
नम॑ ऊ॒र्व्या॑य                    & ऊ॒र्व्या॑य च\\
\hline
च॒ सू॒र्म्या॑य                     & सू॒र्म्या॑य च\\
\hline
च॒ नमः॑                        & नमः॑ प॒र्ण्या॑य\\
\hline
प॒र्ण्या॑य च                     & च॒ प॒र्ण॒श॒द्या॑य\\
\hline
प॒र्ण॒श॒द्या॑य च                   & प॒र्ण॒श॒द्या॑येति॑ पर्ण - श॒द्या॑य\\
\hline
च॒ नमः॑                        & नमो॑ऽपगु॒रमा॑णाय\\
\hline
अ॒प॒गु॒रमा॑णाय च                  & अ॒प॒गु॒रमा॑णा॒येत्य॑प - गु॒रमा॑णाय\\
\hline
चा॒भि॒घ्न॒ते                      & अ॒भि॒घ्न॒ते च॑\\
\hline
अ॒भि॒घ्न॒त इत्य॑भि - घ्न॒ते           & च॒ नमः॑\\
\hline
नम॑ आक्खिद॒ते                    & आ॒क्खि॒द॒ते च॑\\
\hline
आ॒क्खि॒द॒त इत्या᳚ - खि॒द॒ते           & च॒ प्र॒क्खि॒द॒ते\\
\hline
प्र॒क्खि॒द॒ते च॑                    & प्र॒क्खि॒द॒त इति॑ प्र - खि॒द॒ते\\
\hline
च॒ नमः॑                        & नमो॑ वः\\
\hline
वः॒ कि॒रि॒केभ्यः॑                  & कि॒रि॒केभ्यो॑ दे॒वानां᳚ >\\
\hline
दे॒वाना॒ꣳ॒ हृद॑येभ्यः                & हृद॑येभ्यो॒ नमः\\
\hline
नमो॑ विक्षीण॒केभ्यः॑               & वि॒क्षी॒ण॒केभ्यो॒ नमः॑\\
\hline
वि॒क्षी॒ण॒केभ्य॒ इति॑ वि - क्षी॒ण॒केभ्यः॑  & नमो॑ विचिन्व॒त्केभ्यः॑\\
\hline
वि॒चि॒न्व॒त्केभ्यो॒ नमः॑              & वि॒चि॒न्व॒त्केभ्य॒ इति॑ वि - चि॒न्व॒त्केभ्यः॑
\hline
नम॑ आनिर्.ह॒तेभ्यः॑                & आ॒नि॒र्॒.ह॒तेभ्यो॒ नमः॑\\
\hline
आ॒नि॒र्.॒ह॒तेभ्य॒ इत्या॑निः - ह॒तेभ्यः॑    & नम॑ आमीव॒त्केभ्यः॑\\
\hline
आ॒मी॒व॒त्केभ्य॒ इत्या᳚ - मी॒व॒त्केभ्यः॑     &
\hline
\end{longtable}
}
\subsection{\eng{Anuvaka 10}}
{\centering
{\small \eng{Sloka 1}} \\
\begin{longtable}{|c|c|}
\hline
द्रापे॒ अन्ध॑सः                   & अन्ध॑सस्पते\\
\hline
प॒ते॒ दरि॑द्रत्                    & दरि॑द्र॒न्नील॑लोहित\\
\hline
नील॑लोहि॒ तेति॒ नील॑ - लो॒हि॒त॒      & ए॒षां पुरु॑षाणां\\
\hline
पुरु॑षाणामे॒षां                    & ए॒षां प॑शू॒नां\\
\hline
प॒शू॒नां मा                      & मा भेः\\
\hline
भेर्मा                         & माऽरः॑\\
\hline
अ॒रो॒ मो                       & मो, ए॑षां\\
\hline
मो, इति॒ मो                    & ए॒षां॒ किं\\
\hline
किञ्च॒न                        & च॒ नाम॑मत्\\
\hline
आ॒म॒म॒ दित्या॑ ममत्                 & या ते᳚ >\\
\hline
\end{longtable}
}
{\centering
{\small \eng{Sloka 2}} \\
\begin{longtable}{|c|c|}
\hline
ते॒ रु॒द्र॒                        & रु॒द्र॒ शि॒वा\\
\hline
शि॒वा त॒नूः                     & त॒नूश्शि॒वा\\
\hline
शि॒वा वि॒श्वाह॑भेषजी              & वि॒श्वाह॑भेष॒जीति॑ वि॒श्वाह॑ - भे॒ष॒जी॒ >
\hline
शि॒वा रु॒द्रस्य॑                   & रु॒द्रस्य॑ भेष॒जी\\
\hline
भे॒ष॒जी तया᳚ >                   & तया॑ नः\\
\hline
नो॒ मृ॒ड॒                        & मृ॒ड॒ जी॒वसे᳚ >\\
\hline
जी॒वस॒ इति॑ जी॒वसे᳚ >              & इ॒माꣳ रु॒द्राय॑\\
\hline
\end{longtable}
}
{\centering
{\small \eng{Sloka 3}} \\
\begin{longtable}{|c|c|}
\hline
रु॒द्राय॑ त॒वसे᳚ >                  & त॒वसे॑ कप॒र्दिने᳚ >\\
\hline
क॒प॒र्दिने᳚ क्ष॒यद्वी॑राय             & क्ष॒यद्वी॑राय॒ प्र\\
\hline
क्ष॒यद्वी॑रा॒येति॑ क्ष॒यत् - वी॒रा॒य॒     & प्रभ॑रामहे\\
\hline
भ॒रा॒म॒हे॒ म॒तिं                    & म॒ति मिति॑ म॒तिं\\
\hline
यथा॑ नः                       & न॒श्शं\\
\hline
शमस॑त्                         & अस॑द् द्वि॒पदे᳚ >\\
\hline
द्वि॒पदे॒ चतु॑ष्पदे                  & द्वि॒पद॒ इति॑ द्वि - पदे᳚ >\\
\hline
चतु॑ष्पदे॒ विश्वं᳚ >                 & चतु॑ष्पद॒ इति॒ चतुः॑ - प॒दे॒ >\\
\hline
विश्वं॑ पु॒ष्टं                     & पु॒ष्टं ग्रामे᳚ >\\
\hline
ग्रामे॑, अ॒स्मिन्न्                  & अ॒स्मिन्नना॑तुरं\\
\hline
अना॑तुर॒मित्यना᳚ - तु॒रं॒ >           & मृ॒डा नः॑\\
\hline
\end{longtable}
}
{\centering
{\small \eng{Sloka 4}} \\
\begin{longtable}{|c|c|}
\hline
नो॒ रु॒द्र॒                       & रु॒द्रो॒त\\
\hline
उ॒त नः॑                        & नो॒ मयः॑\\
\hline
मय॑स्कृधि                       & कृ॒धि॒ क्ष॒यद्वी॑राय\\
\hline
क्ष॒यद्वी॑राय॒ नम॑सा               & क्ष॒यद्वी॑रा॒येति॑ क्ष॒यत् - वी॒रा॒य॒\\
\hline
नम॑सा विधेम                    & वि॒धे॒म॒ ते॒ >\\
\hline
त॒ इति॑ ते                      & यच्छं\\
\hline
शञ्च॑                          & च॒ योः\\
\hline
योश्च॑                         & च॒ मनुः॑\\
\hline
मनु॑राय॒जे                       & आ॒य॒जे पि॒ता\\
\hline
आ॒य॒ज इत्या᳚ - य॒जे                & पि॒ता तत्\\
\hline
तद॑श्याम                       & अ॒श्या॒म॒ तव॑\\
\hline
तव॑ रुद्र                       & रु॒द्र॒ प्रणी॑तौ\\
\hline
प्रणी॑ता॒विति॒ प्र - नी॒तौ॒ >       & मा नः॑\\
\hline
\end{longtable}
}
{\centering
{\small \eng{Sloka 5}} \\
\begin{longtable}{|c|c|}
\hline
नो॒ म॒हान्तं᳚ >                   & म॒हान्त॑मु॒त\\
\hline
उ॒त मा                        & मा नः॑\\
\hline
नो॒, अ॒र्भ॒कं                      & अ॒र्भ॒कं मा\\
\hline
मा नः॑                        & न॒ उक्ष॑न्तं\\
\hline
उक्ष॑न्तमु॒त                      & उ॒त मा\\
\hline
मा नः॑                        & न॒ उ॒क्षि॒तं\\
\hline
उ॒क्षि॒तमित्यु॑क्षि॒तं                & मा नः॑\\
\hline
नो॒ व॒धीः॒ >                    & व॒धीः॒ पि॒तरं᳚ >\\
\hline
पि॒तरं॒ मा                      & मोत\\
\hline
उ॒त मा॒तरं᳚ >                    & मा॒तरं॑ प्रि॒याः\\
\hline
प्रि॒या मा                     & मा नः॑\\
\hline
न॒स्त॒नुवः॑                       & त॒नुवो॑ रुद्र\\
\hline
रु॒द्र॒ री॒रि॒षः॒                   & री॒रि॒षः॒ इति॑ रीरिषः\\
\hline
\end{longtable}
}
{\centering
{\small \eng{Sloka 6}} \\
\begin{longtable}{|c|c|}
\hline
मा नः॑                        & न॒स्तो॒के\\
\hline
तो॒के तन॑ये                      & तन॑ये॒ मा\\
\hline
मा नः॑                        & न॒ आयु॑षि\\
\hline
आयु॑षि॒ मा                      & मा नः॑\\
\hline
नो॒ गोषु॑                       & गोषु॒ मा\\
\hline
मा नः॑                        & नो॒, अश्वे॑षु\\
\hline
अश्वे॑षु रीरिषः                  & री॒रि॒ष॒ इति॑ रीरिषः\\
\hline
वी॒रान्मा                      & मा नः॑\\
\hline
नो॒ रु॒द्र॒                       & रु॒द्र॒ भा॒मि॒तः\\
\hline
भा॒मि॒तो व॑धीः                  & व॒धी॒र्॒. ह॒विष्म॑न्तः\\
\hline
ह॒विष्म॑न्तो॒ नम॑सा                & नम॑सा विधेम\\
\hline
वि॒धे॒म॒ ते॒ >                     & त॒ इति॑ ते\\
\hline
\end{longtable}
}
{\centering
{\small \eng{Sloka 7}} \\
\begin{longtable}{|c|c|}
\hline
आ॒रात्ते᳚ >                      & ते॒ गो॒घ्ने\\
\hline
गो॒घ्न उ॒त                      & गो॒घ्न इति॑ गो - घ्ने\\
\hline
उ॒त पू॑रुष॒घ्ने                     & पू॒रु॒ष॒घ्ने क्ष॒यद्वी॑राय\\
\hline
पू॒रु॒ष॒घ्न इति॑ पूरुष - घ्ने           & क्ष॒यद्वी॑राय सु॒म्नं\\
\hline
क्ष॒यद्वी॑रा॒येति॑ क्ष॒यत् - वी॒रा॒य॒     & सु॒म्नम॒स्मे\\
\hline
अ॒स्मे ते᳚ >                      & अ॒स्मे इत्य॒स्मे\\
\hline
ते॒, अ॒स्तु॒                        & अ॒स्त्वित्य॑स्तु\\
\hline
रक्षा॑ च                       & च॒ नः॒\\
\hline
नो॒ अधि॑                       & अधि॑ च\\
\hline
च॒ दे॒व॒                         & दे॒व॒ ब्रू॒हि॒\\
\hline
ब्रू॒ह्यध॑                        & अधा॑ च\\
\hline
च॒ नः॒                         & नः॒ शर्म॑\\
\hline
शर्म॑ यच्छ                      & य॒च्छ॒ द्वि॒बर्.हाः᳚ >\\
\hline
द्वि॒बर्.हा॒ इति॑ द्वि - बर्.हाः᳚ >  & स्तु॒हि श्रु॒तं\\
\hline
\end{longtable}
}
{\centering
{\small \eng{Sloka 8}} \\
\begin{longtable}{|c|c|}
\hline
श्रु॒तं ग॑र्त्त॒सदं᳚ >                 & ग॒र्त्त॒सदंँ॒युवा॑नं\\
\hline
ग॒र्त्त॒सद॒मिति॑ गर्त्त - सदं᳚ >       & युवा॑नं मृ॒गं\\
\hline
मृ॒गन्न                         & न भी॒मं\\
\hline
भी॒म मु॑प ह॒त्नुं                     & उ॒प॒ ह॒त्नु मु॒ग्रं\\
\hline
उ॒ग्र मित्यु॒ग्रं                    & मृ॒डा ज॑रि॒त्रे\\
\hline
ज॒रि॒त्रे रु॑द्र                    & रु॒द्र॒ स्तवा॑नः\\
\hline
स्तवा॑नो, अ॒न्यं                   & अ॒न्यन्ते᳚ >\\
\hline
ते॒, अ॒स्मत्                       & अ॒स्मन्नि\\
\hline
नि व॑पन्तु                      & व॒प॒न्तु॒ सेनाः᳚ >\\
\hline
सेना॒ इति॒ सेनाः᳚ >               & परि॑ णः\\
\hline
\end{longtable}
}
{\centering
{\small \eng{Sloka 9}} \\
\begin{longtable}{|c|c|}
\hline
नो॒ रु॒द्रस्य॑                     & रु॒द्रस्य॑ हे॒तिः\\
\hline
हे॒ति र्वृ॑णक्तु                    & वृ॒ण॒क्तु॒ परि॑\\
\hline
परि॑ त्वे॒ षस्य॑                    & त्वे॒ षस्य॑ दुर्म॒तिः\\
\hline
दु॒र्म॒तिर॑घा॒योः                  & दु॒र्म॒तिरिति॑ दुः - म॒तिः\\
\hline
अ॒घा॒योरित्य॑घ - योः             & अव॑ स्थि॒रा\\
\hline
स्थि॒रा म॒घव॑द्भ्यः                & म॒घव॑द्भ्य स्तनुष्व\\
\hline
म॒घव॑द्भ्य॒ इति॑ म॒घव॑त् - भ्यः॒        & त॒नु॒ष्व॒ मीढ्वः॑\\
\hline
मीढ्व॑ स्तो॒काय॑                  & तो॒काय॒ तन॑याय\\
\hline
तन॑याय मृडय                    & मृ॒ड॒येति॑ मृडय\\
\hline
\end{longtable}
}
{\centering
{\small \eng{Sloka 10}} \\
\begin{longtable}{|c|c|}
\hline
मीढु॑ष्टम॒ शिव॑तम                 & मीढु॑ष्ट॒मेति॒ मीढुः॑ - त॒म॒\\
\hline
शिव॑तम शि॒वः                   & शिव॑त॒मेति॒ शिव॑ - त॒म॒\\
\hline
शि॒वो नः॑                      & न॒स्सु॒मनाः᳚ >\\
\hline
सु॒मना॑ भव                      & सु॒मना॒, इति॑ सु - मनाः᳚ >\\
\hline
भ॒वेति॑ भव                      & प॒र॒मे वृ॒क्षे\\
\hline
वृ॒क्ष आयु॑धं                      & आयु॑धं नि॒धाय॑\\
\hline
नि॒धाय॒ कृत्तिं᳚ >                 & नि॒धायेति॑ नि - धाय॑\\
\hline
कृत्तिंँ॒वसा॑नः                     & वसा॑न॒ आ\\
\hline
आ च॑र                         & च॒र॒ पिना॑कं\\
\hline
पिना॑कं॒ बिभ्र॑त्                  & बिभ्र॒दा\\
\hline
आ ग॑हि                        & ग॒हीति॑ गहि\\
\hline
\end{longtable}
}
{\centering
{\small \eng{Sloka 11}} \\
\begin{longtable}{|c|c|}
\hline
विकि॑रिद॒ विलो॑हित              & विकि॑रि॒देति॒ वि - कि॒रि॒द॒\\
\hline
विलो॑हित॒ नमः॑                  & विलो॑हि॒तेति॒ वि - लो॒हि॒त॒\\
\hline
नम॑स्ते                         & ते॒ अ॒स्तु॒\\
\hline
अ॒स्तु॒ भ॒ग॒वः॒                     & भ॒ग॒व॒ इति॑ भग - वः॒\\
\hline
यास्ते᳚ >                       & ते॒ स॒हस्रं᳚ >\\
\hline
स॒हस्रꣳ॑ हे॒तयः॑                   & हे॒तयो॒ऽन्यं\\
\hline
अ॒न्यम॒स्मत्                      & अ॒स्मन्नि\\
\hline
नि व॑पन्तु                      & व॒प॒न्तु॒ ताः\\
\hline
ता इति॒ ताः                   & स॒हस्रा॑णि सहस्र॒धा\\
\hline
\end{longtable}
}
{\centering
{\small \eng{Sloka 12}} \\
\begin{longtable}{|c|c|}
\hline
स॒ह॒स्र॒धा बा॑हु॒वोः                & स॒ह॒स्र॒धेति॑ सहस्र - धा\\
\hline
बा॒हु॒वोस्तव॑                     & तव॑ हे॒तयः॑\\
\hline
हे॒तय॒ इति॑ हे॒तयः॑                 & तासा॒मीशा॑नः\\
\hline
ईशा॑नो भगवः                   & भ॒ग॒वः॒ प॒रा॒चीना᳚ >\\
\hline
भ॒ग॒व॒ इति॑ भग - वः॒              & प॒रा॒चीना॒ मुखा᳚ >\\
\hline
मुखा॑ कृधि                      & कृ॒धीति॑ कृधि\\
\hline
\end{longtable}
}
\subsection{\eng{Anuvaka 11}}
{\centering
\begin{longtable}{|c|c|}
\hline
स॒हस्रा॑णि सहस्र॒शः                & स॒ह॒स्र॒शो ये\\
\hline
स॒ह॒स्र॒श इति॑ सहस्र - शः           & ये रु॒द्राः\\
\hline
रु॒द्रा अधि॑                      & अधि॒ भूम्यां᳚ >\\
\hline
भूम्या॒मिति॒ भूम्यां᳚ >               & तेषाꣳ॑ सहस्रयोज॒ने\\
\hline
स॒ह॒स्र॒यो॒ज॒नेऽव॑                    & स॒ह॒स्र॒यो॒ज॒न इति॑ सहस्र - यो॒ज॒ने\\
\hline
अव॒ धन्वा॑नि                     & धन्वा॑नि तन्मसि\\
\hline
त॒न्म॒सीति॑ तन्मसि                 & अ॒स्मिन् म॑ह॒ति\\
\hline
\end{longtable}
}
{\centering
\begin{longtable}{|c|c|}
\hline
म॒ह॒त्य॑र्ण॒वे                       & अ॒र्ण॒वे᳚ऽन्तरि॑क्षे\\
\hline
अ॒न्तरि॑क्षे भ॒वाः                  & भ॒वा अधि॑\\
\hline
अधीत्यधि॑                       & नील॑ग्रीवाः शिति॒कण्ठाः᳚ >\\
\hline
नील॑ग्रीवा॒ इति॒ नील॑ - ग्री॒वाः॒ >   & शि॒ति॒कण्ठाः᳚ श॒र्वाः\\
\hline
शि॒ति॒कण्ठा॒ इति॑ शिति - कण्ठाः᳚ >   & श॒र्वा अ॒धः\\
\hline
अ॒धः क्ष॑माच॒राः                  & क्ष॒मा॒च॒रा इति॑ क्षमाच॒राः\\
\hline
नील॑ग्रीवाः शिति॒कण्ठाः᳚ >         & नील॑ग्रीवा॒ इति॒ नील॑ - ग्री॒वाः॒ >\\
\hline
शि॒ति॒कण्ठा॒ दिवं᳚ >                & शि॒ति॒कण्ठा॒ इति॑ शिति - कण्ठाः᳚>\\
\hline
दिवꣳ॑ रु॒द्राः                    & रु॒द्रा उप॑श्रिताः\\
\hline
उप॑श्रिता॒ इत्युप॑ - श्रि॒ताः॒ >       & ये वृ॒क्षेषु॑\\
\hline
वृ॒क्षेषु॑ स॒स्पिञ्ज॑राः                & स॒स्पिञ्ज॑रा॒ नील॑ग्रीवाः\\
\hline
नील॑ग्रीवा॒ विलो॑हिताः            & नील॑ग्रीवा॒ इति॒ नील॑ - ग्री॒वाः॒ >\\
\hline
विलो॑हिता॒ इति॒ वि -लो॒हि॒ताः॒ >    & ये भू॒तानां᳚ >\\
\hline
भू॒ताना॒मधि॑पतयः                  & अधि॑पतयो विशि॒खासः॑\\
\hline
अधि॑पतय॒ इत्यधि॑ - प॒त॒यः॒           & वि॒शि॒खासः॑ कप॒र्दिनः॑\\
\hline
वि॒शि॒खास॒ इति॑ वि - शि॒खासः॑       & क॒प॒र्दिन॒ इति॑ कप॒र्दिनः॑\\
\hline
ये अन्ने॑षु                        & अन्ने॑षु वि॒विद्ध्य॑न्ति\\
\hline
वि॒विद्ध्य॑न्ति॒ पात्रे॑षु              & वि॒विद्ध्य॒न्तीति॑ वि - विद्ध्य॑न्ति\\
\hline
पात्रे॑षु॒ पिब॑तः                   & पिब॑तो॒ जनान्॑\\
\hline
जना॒निति॒ जनान्॑                  & ये प॒थां\\
\hline
प॒थां प॑थि॒रक्ष॑यः                  & प॒थि॒रक्ष॑य ऐलबृ॒दाः\\
\hline
प॒थि॒रक्ष॑य॒ इति॑ पथि - रक्ष॑यः       & ऐ॒ल॒बृ॒दा य॒व्युधः॑\\
\hline
य॒व्युध॒ इति॑ य॒व्युधः॑                & ये ती॒र्थानि॑\\
\hline
ती॒र्थानि॑ प्र॒चर॑न्ति               & प्र॒चर॑न्ति सृ॒काव॑न्तः\\
\hline
प्र॒चर॒न्तीति॑ प्र - चर॑न्ति          & सृ॒काव॑न्तो निष॒ङ्गिणः॑\\
\hline
सृ॒काव॑न्त॒ इति॑ सृ॒का - व॒न्तः॒         & नि॒ष॒ङ्गिण॒ इति॑ नि - स॒ङ्गिनः॑\\
\hline
य ए॒ताव॑न्तः                     & ए॒ताव॑न्तश्च\\
\hline
च॒ भूयाꣳ॑सः                      & भूयाꣳ॑सश्च\\
\hline
च॒ दिशः॑                        & दिशो॑ रु॒द्राः\\
\hline
रु॒द्रा वि॑तस्थि॒रे                  & वि॒त॒स्थि॒र इति॑ वि - त॒स्थि॒रे\\
\hline
तेषाꣳ॑ सहस्रयोज॒ने                 & स॒ह॒स्र॒यो॒ज॒नेऽव॑\\
\hline
स॒ह॒स्र॒यो॒ज॒न इति॑ सहस्र - यो॒ज॒ने      & अव॒ धन्वा॑नि\\
\hline
धन्वा॑नि तन्मसि                  & त॒न्म॒सीति॑ तन्मसि\\
\hline
\end{longtable}
}
{\centering
\begin{longtable}{|c|c|}
\hline
नमो॑ रु॒द्रेभ्यः॑                    & रु॒द्रेभ्यो॒ ये\\
\hline
ये पृ॑थि॒व्यां                      & पृ॒थि॒व्यांँये\\
\hline
ये᳚ऽन्तरि॑क्षे                      & अ॒न्तरि॑क्षे॒ ये\\
\hline
ये दि॒वि                        & दि॒वि येषां᳚ >\\
\hline
येषा॒मन्नं᳚ >                      & अन्नं॒ँवातः॑\\
\hline
वातो॑ व॒र्॒.षं                     & व॒र्॒.षमिष॑वः\\
\hline
इष॑व॒स्तेभ्यः॑                      & तेभ्यो॒ दश॑\\
\hline
दश॒ प्राचीः᳚ >                   & प्राची॒र्दश॑\\
\hline
दश॑ दक्षि॒णा                     & द॒क्षि॒णा दश॑\\
\hline
दश॑ प्र॒तीचीः᳚ >                  & प्र॒तीची॒र्दश॑\\
\hline
दशोदी॑चीः                      & उदी॑ची॒र्दश॑\\
\hline
दशो॒र्द्ध्वाः                     & ऊ॒र्द्ध्वास्तेभ्यः॑\\
\hline
तेभ्यो॒ नमः॑                      & नम॒स्ते\\
\hline
ते नः॑                          & नो॒ मृ॒ड॒य॒न्तु॒\\
\hline
मृ॒ड॒य॒न्तु॒ ते                       & ते यं\\
\hline
यं द्वि॒ष्मः                      & द्वि॒ष्मो यः\\
\hline
यश्च॑                           & च॒ नः॒\\
\hline
नो॒ द्वेष्टि॑                      & द्वेष्टि॒ तं\\
\hline
तंँवः॑                           & वो॒ जंभे᳚\\
\hline
जंभे॑ दधामि                      & द॒धा॒मीति॑ दधामि\\
\hline
\end{longtable}
}
\subsection{\eng{Triyambakam}}
{\centering
\begin{longtable}{|c|c|}
\hline
त्र्यं॑बकंँयजामहे             & त्र्यं॑बक॒मिति॒ त्रि - अं॒ब॒कं॒ >\\
\hline
य॒जा॒म॒हे॒ सु॒ग॒न्धिं            & सु॒ग॒न्धिं पु॑ष्टि॒वर्द्ध॑नं\\
\hline
सु॒ग॒न्धिमिति॑ सु - ग॒न्धिं     & पु॒ष्टि॒वर्द्ध॑न॒मिति॑ पुष्टि - वर्द्ध॑नं\\
\hline
उ॒र्वा॒रु॒कमि॑व              & इ॒व॒ बन्ध॑नात्\\
\hline
बन्ध॑नान् मृ॒त्योः           & मृ॒त्योर्मु॑क्षीय\\
\hline
मु॒क्षी॒य॒ मा               & माऽमृता᳚त् >\\
\hline
अ॒मृता॒तित्य॒मृता᳚त् >         & यो रु॒द्रः\\
\hline
रु॒द्रो अ॒ग्नौ              & अ॒ग्नौ यः\\
\hline
यो अ॒फ्सु                 & अ॒फ्सु यः\\
\hline
अ॒फ्स्वित्य॑प् - सु           & य ओष॑धीषु\\
\hline
ओष॑धीषु॒ यः               & यो रु॒द्रः\\
\hline
रु॒द्रो विश्वा᳚ >           & विश्वा॒ भुव॑ना\\
\hline
भुव॑नाऽऽवि॒वेश॑             & आ॒वि॒वेश॒ तस्मै᳚ >\\
\hline
आ॒वि॒वेशेत्या᳚ - वि॒वेश॑        & तस्मै॑ रु॒द्राय॑\\
\hline
रु॒द्राय॒ नमः॑              & नमो॑ अस्तु\\
\hline
अ॒स्त्वित्य॑स्तु              & \\
\hline
\end{longtable}
}