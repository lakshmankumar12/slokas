\subsection{\eng{Shiva Sankalpam}}
येने॒दं भू॒तं भुव॑नं भवि॒श्यत् परि॑-गृही तम॒ मृते॑ न॒सर्वम्᳚॥ \\
ये॑न य॒ज्ञस् त्रा॑यते॑ स॒प्त हो॑ता॒ तन्मे॒ मनः॑ शि॒वसं॑क॒ल्पम॑स्तु॥ (1)\\
\\
येन॒ कर्मा॑णिप् प्र॒चर॑न्ति॒ धीरा॒ यतो॑ वा॒चा मन॑सा चारु॒ यन्ति॑। \\
यत्सं मि॑तं॒ मनः॑ संचर॑न्ति॒ तन्मे॒ मनः॑ शि॒वसं॑क॒ल्पम॑स्तु॥ (2)\\
{\small यत्सं॒ मित॒ मनु॑सं॒ यन्ति॑प् प्रा॒णि न॒स् तन्मे॒ मनः॑ शि॒वसं॑क॒ल्पम॑स्तु॥}\\
\\
येन॒ कर्मा᳚ण्य॒ पसो॑ मनी॒षिणो॑ य॒ज्ञे कु॑ण्वन्ति वि॒दथे॑षु॒ धीराः᳚। \\
यद॑ पू॒र्वय् य॒क्ष मन्तं॑ प्र॒जानां॒ तन्मे॒ मनः॑ शि॒वसं॑क॒ल्पम॑स्तु॥ (3)\\
\\
यत् प्र॒ज्ञान॑ मु॒त चेतो॒ धृति॑श्च॒ यज् ज्योति॑-र॒न्त र॒मृतं॑ प्र॒जासु॑।\\
यस्मा॒न्न ऋ॒ते कि॑ञ्-च॒न कर्म॑क् क्रि॒यते॒ तन्मे॒ मनः॑ शि॒वसं॑क॒ल्पम॑स्तु॥ (4)\\
\\
सु॒षा॒ र॒थि रश्वा॑ निव॒यन् म॑नु॒ष्या᳚न् मे॒नि॒युते॑ प॒शुभि॑र् वा॒जिनी॑ वान्। \\
हृ॒त् प्र॒वि॒ष् ठय् यद च॑र॒य् यविष्ठं॒ तन्मे॒ मनः॑ शि॒वसं॑क॒ल्पम॑स्तु॥ (5)\\
{\small सु॒षा॒ र॒थि रश्वा॑ निव॒यन् म॑नु॒ष्या᳚न् नेनी॒यते॒ऽ भीशु॑भिर् वा॒जिन॑ इव।\\
हृत्प्रष्ठिं॒ यद॑जिरं॒ जबि॑ष्ठं॒ तन्मे॒ मनः॑ शि॒वसं॑क॒ल्पम॑स्तु॥}\\
\\
यस्मि॒न् नृच॒स् साम॒ यजूꣳ॑ षि॒ यस्मि॑न् प्रति॒ष्ठा र॑श॒ना भावि॒ भाराः᳚। \\
यस् मिग्ग्श् चि॒त्तꣳ सर्व॒ मोतं॑ प्र॒जानां॒ तन्मे॒ मनः॑ शि॒वसं॑क॒ल्पम॑स्तु॥ (6)\\
\\
यदत्र॑ ष॒ष्ठं त्रि॒शतꣳ॑ सु॒वीर्यं॑ य॒ज्ञस्य॑ गुह्यं॒ नव॑ना व॒माय्यं᳚। \\
दश॑ पञ्चत् त्रि॒ꣳ॒ शत॒य् यत् परं॒ तन्मे॒ मनः॑ शि॒वसं॑क॒ल्पम॑स्तु॥ (7)\\
\\
यज् जाग्र॑तो दू॒र मु॒दैतु॒ सर्वं॒ तत्-सु॒प्-तस्य॒-तथै॒ वेति॑।\\
दू॒रं॒ग॒ मं ज्योति॑ षां॒ ज्यो॒ति रेकं॒ तन्मे॒ मनः॑ शि॒वसं॑क॒ल्पम॑स्तु॥ (8)\\
\\
येने॒दं विश्वं॒ जग॑तो ब॒भूव॒ ये दे॒वापि॑ मह॒तो जा॒तवे॑दाः।\\
तदे॒वाग् निस् तद् वा॒युस् तत् सूर्य॒स् तदु॑-च॒न्द्र मा॒स्तन्मे॒ मनः॑\\
 शि॒वसं॑क॒ल्पम॑स्तु॥ (9)\\
{\small तदे॒वाग्निस् तम॑सो॒ ज्योति॒ रेकं॒ तन्मे॒ मन॑श्शि॒वसं॑क॒ल्पम॑स्तु॥}\\
\\
येन॒द् द्यौः पृ॑थि॒वी चा॒न् तरि॑क्षं च॒ ये पर्व॑ताः प्र॒दिशो॒ दिश॑श्च। \\
येने॒दं जग॒द् ध्याप्तं॑ प्र॒जानां॒ तन्मे॒ मनः॑ शि॒वसं॑क॒ल्पम॑स्तु॥ (10)\\
\\
ये मनो॒ हृद॑ यय्ये च॑ दे॒वा ये दि॒व्या, आपो॒ ये सूर्य॑ र॒श् मिः। \\
ते श्रोत्रे॒ चक्षु॑षी \textbf{सं॒चर॑न् तन्॒} तन्मे॒ मनः॑ शि॒वसं॑क॒ल्पम॑स्तु॥ (11)\\
\\
अचि॑न् त्यञ्॒चा प्र॑मे यंचव् व्य॒क्ता॒, व्यक्त॑ परं॒ चय॑त्। \\
सूक्ष्मा᳚त् सूक्ष्म त॑रन्, ज्ञे॒यं तन्मे॒ मनः॑ शि॒वसं॑क॒ल्पम॑स्तु॥ (12)\\
\\
एका॑ च द॒श च॑ श॒तं च॑ स॒हस्र॑ञ् चा॒ यु॑तञ्च। \\
नि॒यु तं॑च प्र॒यु त॒ञ्चार् बु॑दञ् च॒न् य॑र् बुदंच \\
{\small समु॒द्रश्च॒ मध्यं॒ चान्त॑श्च परा॒र्धश्च॒}\\
तन्मे॒ मनः॑ शि॒वसं॑क॒ल्पम॑स्तु॥ (13)\\
\\
ये प॑ञ्च पञ्चा॒द॒श श॒तꣳ स॒हस्र॑ म॒युतं॒ न्य॑र् बुदं च। \\
ए अ॑ग्नि चि॒त्तेष् ट॑का॒स्ताꣳ शरी॑रं॒ तन्मे॒ मनः॑ शि॒वसं॑क॒ल्पम॑स्तु ॥ (14)\\
\\
वेदा॒हमे॒तं पुरु॑षं म॒हान्त॑ मादि॒त्यव॑र्णं॒ तम॑सः॒ पर॑स्तात्। \\
यस्य॒ योनिं॒ परि॒ पश्य॑न्ति॒ धीरा॒स्तन्मे॒ मनः॑ शि॒वसं॑क॒ल्पम॑स्तु॥ (15)\\
\\
यस्यैतं धीराः᳚ पु॒नन्ति॑ क॒वयो᳚ ब्र॒ह्माण॑ मे॒तं त्वा॑ वृणुत॒ मिन्दुं᳚। \\
स्था॒व॒रं जङ्ग॑मं॒ द्यौरा॑का॒शं॒ तन्मे॒ मनः॑ शि॒वसं॑क॒ल्पम॑स्तु॥ (16)\\
\\
{\small \eng{17/18 may be swapped as well}}\\
परा᳚त् प॒रत॑रं ब्र॒ह्म॒ त॒त् परा᳚त् पर॒तो ह॑रिः। \\
य॒त् परा᳚त् पर॑तोऽ धी॒शं॒ तन्मे॒ मनः॑ शि॒वसं॑क॒ल्पम॑स्तु॥ (17)\\
\\
परा᳚त् प॒रत॑रं चैव त॒त् परा᳚च् चैव॒ यत् प॑रम्। \\
य॒त् परा᳚त् पर॑तो ज्ञे॒यं॒ तन्मे॒ मनः॑ शि॒वसं॑क॒ल्पम॑स्तु॥ (18)\\
\\
या वेदा दिषु॑-गाय॒त्री स॒र्वव् व्या॑पी म॒हेश्व॑री। \\
ऋग् य॑जु॒स् सामा॑ थर् वै॒श्च॒ तन्मे॒ मनः॑ शि॒वसं॑क॒ल्पम॑स्तु॥ (19)\\
\\
यो वै॑ दे॒वं म॑हादे॒वं॒ प्र॒य॒तः प्र॑णत॒श् शु॑चिः। \\
{\small यो वै॑ दे॒वं म॑हादे॒वं प्र॒णवं॑ पर॒मेश्व॑रम्।}\\
यस्सर्वे॑ सर्व॑ वेदै॒श्च तन्मे॒ मनः॑ शि॒वसं॑क॒ल्पम॑स्तु॥ (20)\\
\\
प्र॒य॒तः॒ प्रण॑वोंका॒रं प्र॒णवं॑ पुरु॒षोत्त॑मम्। \\
ओंका॑रं॒ प्रण॑वात्मा॒नं तन्मे॒ मनः॑ शि॒वसं॑क॒ल्पम॑स्तु॥ (21)\\
\\
योऽसौ॑ स॒र्वेषु॑ वेदे॒षु प॒ठ्यते᳚ ह्यय॒ मीश्व॑र:। \\
अकायो॑ निर्गु॑णो ह्या॒त्मा तन्मे॒ मनः॑ शि॒वसं॑क॒ल्पम॑स्तु॥ (22)\\
\\
गोभि॒र्जुष्टं॒ धने॑न॒ह् ह्यायु॑षा च॒ बले॑ नच। \\
प्र॒जया॑ प॒शुभिः॑ पुष्करा॒क्षं तन्मे॒ मनः॑ शि॒वसं॑क॒ल्पम॑स्तु॥ (23)\\
\\
{\small \eng{24/25 may be swapped as well}}\\
त्र्यं॑बकं यजामहे सुग॒न्धिं पु॑ष्ति॒वर्ध॑नम्। उ॒र्वा॒रु॒कमि॑व॒ \\
बन्ध॑नान्मृ॒त्योर्मु॑क्षीय॒ माऽमृता॒ तन्मे॒ मनः॑ शि॒वसं॑क॒ल्पम॑स्तु॥ (24)\\
\\
कैला॑स॒ शिख॑रे र॒म्ये॒ शङ्कर॑स्य शि॒वाल॑ये। \\
दे॒वता᳚स् तत्र॑ मोदन्ति॒ तन्मे॒ मनः॑ शि॒वसं॑क॒ल्पम॑स्तु॥ (25)\\
\\
{\small 26 may be left}\\
कैला॑स॒ शिखरा वा॒सं हि॒मव॑द् गिरि॒ संस्थिथं । \\
नी॒ल॒क॒ण्ठं त्रि॑नेत्रं च तन्मे॒ मनः॑ शि॒वसं॑क॒ल्पम॑स्तु॥ (26)\\
\\
वि॒श्व त॑श् चक्षुरु॒त वि॒श्व तो॑मुखो वि॒श्व तो॑हस्त उ॒त वि॒श्व त॑स्पात्।\\
सं बा॒हुभ्यां॒ नम॑ति॒-सं-पत॑त् त्रै॒र् द्यावा॑ पृथि॒वी ज॒नय॑न् दे॒व\\
 एक॒स्तन्मे॒ मनः॑ शि॒वसं॑क॒ल्पम॑स्तु॥ (27)\\
\\
चतुरो॑ वे॒दा न॑धीयी॒त स॒र्व शा᳚स् त्रम॒यं वि॑दुः।  \\
इ॒ति॒हा॒स पु॑राणा॒नां॒ तन्मे॒ मनः॑ शि॒वसं॑क॒ल्पम॑स्तु॥ (28)\\
\\
मा नो॑ म॒हान्त॑मु॒त मा नो॑, अर्भ॒कं मा न॒ उक्ष॑न्तमु॒त मा न॑ उक्षि॒तम्। \\
मा नो॑ऽवधीः पि॒तरं॒ मोत मा॒तरं॑ प्रि॒या मा न॑स्त॒नुवो॑ \\
रुद्र रीरिष॒स्तन्मे॒ मनः॑ शि॒वसं॑क॒ल्पम॑स्तु॥  (29)\\
\\
मान॑स्तो॒के तन॑ये॒ मा न॒ आयु॑षि॒ मा नो॒ गोषु॒ मा नो॒, अश्वे॑षु रीरिषः। \\
वी॒रान्मा नो॑ रुद्र भामि॒तो ऽव॑धीर् ह॒विष्म॑न्तो॒ नम॑सा \\
विधेम ते॒ तन्मे॒ मनः॑ शि॒वसं॑क॒ल्पम॑स्तु॥ (30)\\
\\
ऋ॒तꣳ स॒त्यं प॑रं ब्र॒ह्म॒ पु॒रुषं॑ कृष्ण॒पिङ्ग॑लम्। \\
ऊ॒र्ध्वरे॑तंवि॑रूपा॒क्षं॒ वि॒श्वरू॑पाय॒ वै नमो॒ नम॒स्तन्मे॒ मनः॑ शि॒वसं॑क॒ल्पम॑स्तु॥ (31)\\
\\
कद्रु॒द्राय॒ प्रचे॑तसे मी॒ढुष्ट॑माय॒ तव्य॑से। \\
वो॒चेम॒ शंत॑मꣳ हृ॒दे। सर्वो॒ ह्ये॑ष रु॒द्रस्तस्मै॑ रु॒द्राय॒ नमो॑ अस्तु॒ \\
तन्मे॒ मनः॑ शि॒वसं॑क॒ल्पम॑स्तु॥ (32)\\
\\
ब्रह्म॑ जज्ञा॒नं प्र॑थ॒मं पु॒रस्ता॒द् विसी॑ म॒तस् सु॒रुचो॑ वे॒न आ॑वः। \\
स बु॒ध्निया॑, उप॒मा, अ॑स्य वि॒ष्ठास् स॒तश्च॒ योनि॒मस॑तश्च॒ \\
विव॒स् तन्मे॒ मनः॑ शि॒वसं॑क॒ल्पम॑स्तु॥ (33)\\
\\
यः प्रा॑ण॒तो नि॑मिष॒तो म॑हि॒त्वै॒क इद्राजा॒ जग॑तो ब॒भूव॑।\\
य ईशे॑, अ॒स्यद् द्वि॒पद॒श्चतु॑ष्पदः॒ कस्मै॑ दे॒वाय॑ ह॒विषा॑ विधेम॒ \\
तन्मे॒ मनः॑ शि॒वसं॑क॒ल्पम॑स्तु॥ (34)\\
\\
य आ᳚त् म॒दा ब॑ल॒दा यस्य॒ विश्व॑ उ॒पास॑ते प्र॒शिषं॒ यस्य॑ दे॒वाः। \\
यस्य॑ छायाऽमृतं यस्य॑ मृ॒त्युः कस्मै॑ दे॒वाय॑ ह॒विषा॑ विधेम॒ \\
तन्मे॒ मनः॑ शि॒वसं॑क॒ल्पम॑स्तु॥ (35)\\
\\
यो रु॒द्रो, अ॒ग्नौ यो, अ॒प्सु य ओष॑धीषु॒ यो रु॒द्रो विश्वा॒ \\
भुव॑नाऽऽवि॒वेश॒ तस्मै॑ रु॒द्राय॒ नमो॑, अस्तु॒ तन्मे॒ \\
मनः॑ शि॒वसं॑क॒ल्पम॑स्तु॥ (36)\\
\\
ग॒न्ध॒द्वा॒रां दु॑राध॒र्षां॒ नि॒त्यपु॑ष्टां करी॒षिणी᳚म्। \\
ई॒श्वरीꣳ॑ सर्व॑भूता॒नां॒ त्वामि॒होप॑ह्रये॒ श्रियं॒  \\
तन्मे॒ मनः॑ शि॒वसं॑क॒ल्पम॑स्तु॥ (37)\\
\\
{\small 38 may be left}\\
नमकं॑ चम॑कं चै॒व पु॒रुषसू᳚क्तं च॒ यद् विदुः। \\
महादेवं च तत्तुल्यं॒ तन्मे॒ मनः॑ शि॒वसं॑क॒ल्पम॑स्तु॥ (38)\\
\\
य इ॒दꣳ शिव॑संक॒ल्प॒ꣳ॒ स॒दा ध्या॑यन्ति॒ब् ब्राह्म॑णाः। \\
ते परं॒ मोक्षं॑ गमिष्यन्ति॒ तन्मे॒ मनः॑ शि॒वसं॑क॒ल्पम॑स्तु ॥ (39)\\
हृदयाय नमः।\\
