\subsection{\eng{Panchanga sagrucchajapeth}}
स॒द्योजा॒तं प्र॑पद्या॒मि॒ स॒द्योजा॒ताय॒ वै नमो॒ नमः॑। \\
भ॒वे भ॑वे॒ नाति॑भवे भवस्व॒ माम्। भ॒वोद्भ॑वाय॒ नमः॑॥\\
\\
वा॒म॒दे॒वाय॒ नमो᳚ ज्ये॒ष्ठाय॒ नम॑श्श्रे॒ष्ठाय॒ नमो॑ रु॒द्राय॒ नमः॒ काला॑य॒ नमः॒ कल॑विकरणाय॒\\
नमो॒ बल॑विकरणाय॒ नमो॒ बला॑य॒ नमो॒ बल॑प्रमथनाय॒ नम॒स्सर्व॑भूतदमनाय॒ नमो॑\\
म॒नोन्म॑नाय॒ नमः॑॥\\
\\
अ॒घोरे᳚भ्योऽथ॒ घोरे᳚भ्यो घोर॒घोर॑तरेभ्यः। \\
स॒र्वे᳚तः॑ सर्व॒ शर्वे᳚भ्यो॒ नम॑स्ते अस्तु रु॒द्ररू॑पेभ्यः॥\\
\\
तत्पुरु॑षाय वि॒द्महे॑ महादे॒वाय॑ धीमहि। तन्नो॑ रुद्रः प्रचो॒दया᳚त्॥\\
\\
ईशानस्सर्व॑विद्या॒नां ईश्वरस्सर्व॑भूता॒नां॒ ब्रह्माधि॑पति॒-\\
रब्रह्म॒णोऽधि॑पति॒-रब्रह्मा॑ शि॒वोमे॑ अस्तु सदाशि॒वोम्॥\\
\subsubsection{\eng{Alternate}}
ह॒ग्ं॒ सश् शुचि॒ षद् वसुरन् तरि क्ष॒सद् धोता वेदि॒ष दति थिर् दरो ण॒सत्।\\
नृ॒षद् वर॒सद् रुत सब् यो मसदब् जा गोजा, ऋतजा, अद्रिजा, ऋतं बृहत्॥ \\
\\
प्रतद् विष्णु॑स्स्तवते वी॒र्या॑य । मृ॒गो न भी॒मः कु॑च॒रो गि॑रि॒ष्ठाः । \\
यस्यो॒रुषु॑ त्रि॒षु वि॒क्रम॑णेषु । अधि॑क्षि॒यन्ति॒ भुव॑नानि॒ विश्वा᳚  । \\
\\
त्र्य॑म्बकं यजामहे सुग॒न्धिं पु॑ष्टि॒वर्ध॑नम् ।\\
उ॒र्वा॒रु॒कमि॑व॒ बन्ध॑नान्मृ॒त्योर्मु॑क्षीय॒ माऽमृता᳚त् ।\\
\\
तत्स॑ वि॒तुर् वृ॑णीमहे । व॒यं दे॒वस्य॒ भोज॑नम् । \\
श्रेष्ठꣳ॑  सर्व॒ धात॑मं । तुरं॒ भग॑स्य धीमहि ॥\\
\\
विष्णु॒र् योनिं॑ कल्पयतु । त्वष्टा॑ रु॒पाणि॑ पिꣳशतु । \\
आसि॑ चतु प्र॒जाप॑तिः । धा॒ता गर्भं॑ दधातु मे ॥\\
\subsection{\eng{Ashtanga Pranamaha}}
हि॒र॒ण्य॒ग॒र्भः सम॑वर्त॒ताग्रे॑ भू॒तस्य॑ जा॒तः पति॒रेक॑ आसीत्।\\
स दाधा॑र पृथि॒वीं द्यामु॒तेमां कस्मै॑ दे॒वाय ह॒विषा॑ विधेम॥\\
ॐ उमा महेश्वराभ्यां नमः     (1)\\
\\
यः प्रा॑ण॒तो नि॑मिष॒तो म॑हि॒त् वै॒क इद्राजा॒ जग॑तो ब॒भूव॑।\\
य ईशे॑, अ॒स्यद् द्वि॒पद॒श् चतु॑ष्पदः॒ कस्मै॑ दे॒वाय॑ ह॒विषा॑ विधेम॥\\
ॐ उमा महेश्वराभ्यां नमः     (2)\\
\\
ब्रह्म॑ जज्ञा॒नं प्र॑थ॒मं पु॒रस्ता॒द् विसी॑ म॒तस् सु॒रुचो॑ वे॒न आ॑वः।\\
स बु॒ध् निया॑, उप॒मा, अ॑स्य वि॒ष्टास् स॒तश्च॒ यो॒नि मस॑ तश्च॒ विवः॑॥\\
ॐ उमा महेश्वराभ्यां नमः     (3)\\
\\
म॒ही द्यौः पृ॑थि॒वी च॑ न इ॒मं यज्ञं मि॑मिक्षताम्।\\
पि॒पृ॒तां नो॒-भरी॑ मभिः॥\\
ॐ उमा महेश्वराभ्यां नमः     (4)\\
\\
उप॑श् वासय पृथि॒वी मु॒तद् यां पु॑रु॒त्रा ते॑ मनुतां॒ विष्टि॑तं॒ जग॑त्।\\
स दुन्दु॑बे स॒जू रिन्द्रे॑ण दे॒वैर् दू॒राद् दवी॑यो॒, अप॑से ध॒शत् रून्॥\\
ॐ उमा महेश्वराभ्यां नमः     (5)\\
\\
अग्ने॒ नय॑ सु॒पता॑ रा॒ये, अ॒स्मान्. विश्वा॑नि देव व॒युना॑नि वि॒द्वान्।\\
यु॒यो॒ध् य॑स्मज् जु॑हु-रा॒ण-मेनो॒ भूयि॑ष् ठान्ते॒ नम॑ उक्तिं विधेम॥\\
ॐ उमा महेश्वराभ्यां नमः     (6)\\
 \\
या ते॑, अग्ने॒ रुद्रि॑या त॒नूस् तया॑नः पाहि॒ तस्या᳚स्ते॒ स्वाहा॒ याते॑,\\
अग्नेऽ याश॒या र॑जाश॒या ह॑राश॒या त॒नूर् वर्-षि॑ष्ठा-गह्वरे॒ष्-ठोग्रं वचो॒,\\
अपा वधीं त्वे॒षव् वचो॒, अपा॑ वधी॒ग् स्वाहा᳚॥   \\
{\small या ते॑ अग्ने॒ रुद्रि॑या त॒नूस्तया॑ नः पाहि॒ तस्या᳚स्ते॒ स्वाहा᳚।}\\
ॐ उमा महेश्वराभ्यां नमः     (7)\\
\\
इ॒मं य॑मप् प्रस् त॒र माहि सीदाङ्-गि॑रोभिः\eng{f} पि॒तृभिः॑स् संविदा॒नः।\\
आत्वा॒ मन्त्राः᳚ कवि श॒स्ता व॑हन्-त्वे॒ना रा॑जन्. ह॒विषा॑ माद यस्व॥\\
ॐ उमा महेश्वराभ्यां नमः     (8)\\
\\
उरसा शिरसा दृष्ट्या मनसा वचसा तथा।\\
पद्भ्यां  कराभ्यां कर्णाभ्यां प्रणा मोष्टाङ्ग उच्यते॥\\
\\
