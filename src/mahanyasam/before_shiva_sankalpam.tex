\subsection{\eng{Guhyadhi Mastakaantam Shadanganyasam}}
मनो॒ ज्योति॑र् जुष ता॒माज्यं॒ विच् छि॑न्नं य॒ज्ञꣳ समि॒मंद॑ धातु। \\
बृह॒स्पति॑ स्तनुता मि॒मंनो॒ विश्वे॑ दे॒वा, इ॒ह मा॑द यन्ताम्॥ गुह्याय नमः॥\\
{\small या इ॒ष्टा उ॒षसो॑ नि॒म्रुच॑श्च॒ तास् सन्द॑धामि ह॒विषा॑ घृ॒तेन॑॥ गुह्याय नमः॥}\\
\\
अबो᳚ध् य॒ग् निस् स॒मिधा जना॑नां॒ प्र॑ति धे॒नु मि॑वा य॒ती मु॒षासम्᳚। \\
य॒ह्वा, इ॑व॒प् प्र॒वया मु॒ज्जि हा॑नाः॒ प्रभा॒ नव॑स् सिस् रते॒ नाक॒ मच्छा॑॥ नाभ्यै नमः॥\\
\\
अ॒ग्निर् मू॒र्धा दि॒वः क॒कुत् पतिः॑ पृथि॒व्या, अ॒यम्। \\
अ॒पाꣳ रेताꣳ॑ सि जिन्वति । हृदयाय नमः ॥\\
\\
मू॒र्धानं॑ दि॒वो, अ॑र॒तिं पृ॑थि॒व्या वै᳚श्वान॒र मृ॒ताय॑ जा॒त म॒ग्निम्‌। \\
क॒विꣳ स॒म्राज॒ मति॑थिं॒ जना॑ना मा॒सन्ना पात्रं॑ जन यन्त दे॒वाः॥ कण्ठाय नमः॥ \\
\\
मर्मा॑णि ते॒ वर्म॑भिश्छादयामि॒ सोम॑स्त्वा॒ राजा॒ऽमृते॑ना॒भिऽव॑स्ताम्।\\
उ॒रोर्वरी॑यो॒ वरि॑वस्ते अस्तु॒ जय॑न्तं त्वामनु॑ मदन्तु दे॒वाः॥ मुखाय नमः।\\
\\
जा॒तवे॑दा॒ यदि॑ वा पाव॒कोऽसि॑। वै॒श्वा॒न॒रो यदि॑ वा वैद् युतोसि॑। \\
शं प्र॒जाभ्यो॒ यज॑मानाय लो॒कम्। ऊर्जं॒ पु॒ष् तिं दद॑ द॒भ्याव॑ वृथ् स्व॥ शिरसे नमः॥\\
\subsection{\eng{Atma Rakshaha}}
ब्रह्मा᳚त् म॒न् वद॑ सृजत। तद॑ कामयत। समा॒त् मना॑ पद् ये॒येति॑। \\
\\
आत् म॒न्नात् म॒न् नित्याम॑न् त्रयत। तस्मै॑ दश॒मꣳ हू॒तः प्रत्य॑श्रुणोत्।\\
स दश॑ हूतोऽभवत्। दश॑ हूतो ह॒ वै ना मै॒षः। \\
तं वा, ए॒तन् दश॑हू त॒ꣳ सन्तम्᳚। दश॑हो॒तेत् याच॑क्षते प॒रोक्षे॑ण। \\
प॒रोक्ष॑ प्रिया, इव॒ हि दे॒वाः॥\\
\\
आत् म॒न्नात् म॒न् नित्याम॑न् त्रयत। तस्मै॑ सप्त॒मꣳ हूतः प्रत्य॑श्रुणोत्। \\
स स॒प्त हू॑तोऽभवत्। स॒प्त हू॑तो ह॒ वै ना मै॒षः। \\
तं वा, ए॒तꣳ स॒प्तहू॑ त॒ꣳ सन्तम्᳚। स॒प्तहो॒तेत् याच॑क्षते प॒रोक्षे॑ण। \\
प॒रोक्ष॑ प्रिया, इव॒ हि दे॒वाः॥\\
\\
आत् म॒न्नात् म॒न् नित्याम॑न् त्रयत। तस्मै॑ ष॒ष्ठꣳ हू॒तः प्रत्य॑श्रुणोत्। \\
स षड् ढू॑तोऽ भवत्। षड् ढू॑तो ह वै नामै॒षः। \\
तं वा, ए॒तꣳ षड्ढू॒॑त॒ꣳ॒ सन्तम्᳚। षड्ढूो॒तेत् याच॑क्षते प॒रोक्षे॑ण। \\
प॒रोक्ष प्रिया, इव॒ हि दे॒वाः॥\\
\\
आत् म॒न्नात् म॒न् नित्याम॑न् त्रयत। तस्मै॑ पञ्च॒मꣳ हू॒तः प्रत्य॑श्रुणोत्। \\
स पञ्च॑ हूतोऽभवत्। पञ्च॑ हूतो ह॒ वै नामै॒षः। \\
तं वा, ए॒तं पञ्च॑हूत॒ꣳ॒ सन्तम्᳚। पञ्च॑हो॒तेत् याच॑क्षते प॒रोक्षे॑ण। \\
प॒रोक्ष॑ प्रिया, इव॒ हि दे॒वाः॥\\
\\
आत् म॒न्नात् म॒न् नित्याम॑न् त्रयत। तस्मै॑ चतु॒र्थꣳ हू॒तः प्रत्य॑श्रुणोत्। \\
स चतु॑र् हूतोऽभवत् । चतु॑र् हूतो ह॒ वै नामै॒षः। \\
तं वा,  ए॒तं चतु॑र् हू॒त॒ꣳ॒ सन्तम्᳚। चतु॑र् होतेत् याच॑क्षते प॒रोक्षे॑ण। \\
प॒रोक्ष॑ प्रिया, इव॒ हि दे॒वाः॥\\
\\
तम॑ब् ब्रवीत्। त्वं वै मे॒ नेदि॑ष्ठꣳ हूतः प्रत्य॑श् श्रौषीः। \\
त्वयै॑ नानाख् या॒तार॒ इति॑। तस्मा॒न्नु है॑ना॒हु॒श् चतु॑र् होतार॒ इत्या च॑क्षते। \\
तस्मा᳚च् छुश् रू॒षुः पु॒त्रा णा॒ꣳ॒ हृद् य॑तमः। ने दि॑ष्ठो॒ हृद् य॑तमः॒।\\
नेदि॑ष्ठो॒ ब्रह्म॑णो भवति। य ए॒वं वेद॑॥ आत्मने नमः ।\\
