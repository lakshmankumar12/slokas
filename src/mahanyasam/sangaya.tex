\subsection{\eng{Dik Samputa Nyasaha}}
ॐ भूर्भुव॒स्सुव॒रों ।\\
ॐ लं । त्रातारमिंन्द्र॑ मवि॒तार॒मिन्द्रꣳ हवे॑ हवे सु॒हव॒ꣳ॒ शूर॒मिन्द्रम्᳚ । \\
हु॒वे नु श॒क्रं पु॑रहू॒त मिन्द्रग्ग्॑ स्व॒स्ति नो॑ म॒घवा॑ धा॒त्विन्द्रः॑ ॥\\
\\
लं भर्व धिग्बागे, इन्द्राय वज्रर्हस्ताय देवाधिपतये, ऐरावत वाहनाय - \\
सांगाय सायुधाय सशक्तिस परिवाराय -  उमामहेश्वर पार्षदाय नमः।\\
लं इन्द्राय नमः ।  पूर्व दिग्भागे, इन्द्रः सुप्रीतो  वरदो भवतु ॥  (1)\\
{\small लं भूर्भुवस्सुवः इन्द्राय वज्रहस्ताय सुराधिपतय ऐरावतवाहनाय -\\
सांगाय सायुधाय सशक्ति परिवाराय - सर्वालंकार भूषिताय उमामहेश्वर पार्षदाय नमः।\\
पूर्व दिग्भागे ललाटस्थाने इन्द्रः सुप्रीतो सुप्रस्न्नो वरदो भवतु ॥}\\
ॐ भूर्भुव॒स्सुव॒रों ।\\
\\
रं । त्वन्नो॑, अग्ने॒ वरु॑णस्य वि॒द्वान् दे॒वस्य॒ हेडोऽव॑ यासि सीष्ठाः ।\\
यजि॑ष्ठो॒  वह्नि॑ तम॒श् शोशु॑ चानो॒ विश्वा॒, द्वेषाꣳ॑सि॒प् प्रमु॑ मुग् घ्य॒स् मत् ॥ \\
\\
रं आग् नेय धिग्बागे, अग्नये शक्ति हस्ताय तेजोऽधि पतयेऽ,\\
\hspace*{10cm} अज वाहनाय\\
सांगाय सायुधाय सशक्तिस परिवाराय - उमामहेश्वर पार्षदाय नमः।\\
रं अग्नये नमः । आग्नेय दिग्भागे अग्निः सुप्रीतो  वरदो भवतु ॥ (2)\\
{\small रं भूर्भुवस्सुवः अग्नये शक्ति हस्ताय तेजोऽधि पतयेऽ अज वाहनाय -\\
सांगाय सायुधाय सशक्ति परिवाराय - सर्वालंकार भूषिताय उमामहेश्वर पार्षदाय नमः ।\\
आग्नये दिग्भागे नेत्रस्थाने अग्निः सुप्रीतो सुप्रस्न्नो वरदो भवतु ॥}\\
ॐ भूर्भुव॒स्सुव॒रों ।\\
हं । सु॒गन्नः॒ पन्था॒ मभ॑यं कृणोतु । यस्मि॒न् नक्ष॑त्रे य॒म एति॒ राजा᳚ ।\\
यस्मि॑न् नेन म॒भ्य षिं॑ चन्त दे॒वाः । तद॑स्य चि॒त्रꣳ ह॒विषा॑ यजाम ॥\\
\\
हं दक्षिण धिग्बागे यमाय दण्ड हस्ताय धर्माधि पतये महिष वाहनाय\\
सांगाय सायुधाय सशक्तिस परिवाराय -  उमामहेश्वर पार्षदाय नमः।\\
हं यमाय नमः । दक्षिण दिग्भागे यमः सुप्रीतो  वरदो भवतु ॥ (3)\\
{\small हं भूर्भुवस्सुवः यमाय दण्ड हस्ताय धर्माधि पतये महिष वाहनाय -\\
सांगाय सायुधाय सशक्ति परिवाराय - सर्वालंकार भूषिताय उमामहेश्वर पार्षदाय नमः ।\\
दक्षिण दिग्भागे कर्णस्थाने यमः सुप्रीतो सुप्रस्न्नो वरदो भवतु॥}\\
ॐ भूर्भुव॒स्सुव॒रों ।\\
षं । असु॑न्वन्त मय॑जमान मिच् छस् ते॒ नस् ये॒त् यान् तस्क॑रस् यान् वे॑षि।\\
अ॒न्य म॒स्म दि॑च्छ॒ सात॑ इ॒त्या नमो॑ देवि-निर्ऋते॒ तुभ्य॑ मस्तु ॥\\
\\
षं निर्ऋति दिग्भागे निर्ऋतये खड्ग हस्ताय रक्षो धिपतये नर वाहनाय\\
सांगाय सायुधाय सशक्तिस परिवाराय -  उमामहेश्वर पार्षदाय नमः।\\
षं निर्ऋतये नमः । निर्ऋति दिग्भागे निर्ऋतिस्सुप्रीतो वरदो भवतु ॥ (4)\\
{\small षं भूर्भुवस्सुवः निऋतये खड्ग हस्ताय रक्षोधि पतये नर वाहनाय -\\
सांगाय सायुधाय सशक्ति परिवाराय - सर्वालंकार भूषिताय उमामहेश्वर पार्षदाय नमः ।\\
नैर्ऋतदिग्भागे मुखस्थाने निर्ऋतिस्सुप्रीतो सुप्रस्न्नो वरदो भवतु ॥}\\
ॐ भूर्भुव॒स्सुव॒रों ।\\
वं । तत्वा॑ यामि॒ ब्रह्म॑णा॒ वन्द॑ मा॒नस् तदा शा᳚स्ते॒ यज॑मानो ह॒विर्भिः॑ ।\\
अहे॑डमानो वरुणे॒ हबो॒ध् युरु॑शꣳ स॒मान॒ आयुः॒ प्रमो॑षीः ॥\\
\\
वं पश्चिम दिग्भागे वरुणाय पाश हस्ताय जलाधि पतये मकर वाहनाय\\
सांगाय सायुधाय सशक्तिस परिवाराय -  उमामहेश्वर पार्षदाय नमः।\\
वं वरुणाय नमः । पश्चिम दिग्भागे वरुणः सुप्रीतो वरदो भवतु ॥ (5)\\
{\small वं भूर्भुवस्सुवः वरुणाय पाशहस्ताय जलाधि पतये मकर वाहनाय -\\
सांगाय सायुधाय सशक्ति परिवाराय - सर्वालंकार भूषिताय उमामहेश्वर पार्षदाय नमः ।\\
पश्चिम दिग्भागे बाहुस्थाने वरुणः सुप्रीतो सुप्रस्न्नो वरदो भवतु ॥}\\
ॐ भूर्भुव॒स्सुव॒रों ।\\
यं । आ नो॑ नि॒युद् भि॑श् श॒तिनी॑ भिरध् व॒रम् । \\
स॒ह॒स् रिणी॑ भि॒रुप॑ याहि य॒ज्ञम् ।\\
वायो॑, अ॒स्मिन् ह॒विषि॑ मादयस्व । यू॒यं पा॑तस् स्व॒स्ति भि॒स् सदा॑ नः॥\\
\\
यं वायव्य दिग्भागे वायवे सांकु शध् वजहस्ताय प्राणाधिपतये मृगवाहनाय\\
सांगाय सायुधाय सशक्तिस परिवाराय -  उमामहेश्वर पार्षदाय नमः।\\
यं वायवे नमः । वायव्य दिग्भागे  वायुः सुप्रीतो  वरदो भवतु ॥ (6)\\
{\small यं भूर्भुवस्सुवः वायवे सांकुशध् वजहस्ताय प्राणाधिपतये मृगवाहनाय -\\
सांगाय सायुधाय सशक्ति परिवाराय - सर्वालंकार भूषिताय उमामहेश्वर पार्षदाय नमः ।\\
वायव्य दिग्भागे नासिकास्थाने वायुः सुप्रीतो सुप्रस्न्नो वरदो भवतु ॥}\\
ॐ भूर्भुव॒स्सुव॒रों ।\\
सं । व॒यꣳ सो॑ मव् व्र॒ते तव॑ । मन॑स्त॒ नू षु॒बिभ् र॑तः ।\\
प्र॒जा व॑न्तो, अशीमहि ।\\
सं उत्तर दिग्भागे सोमाय अमृत कलश \\
हस्ताय नक्षत्राधिपतये, अश्व वाहनाय सांगाय सायुधाय \\
सशक्तिस परिवाराय - उमामहेश्वर पार्षदाय नमः।\\
सं सोमाय नमः । उत्तर दिग्भागे  सोमः सुप्रीतो वरदो भवतु ॥ (7)\\
{\small इ॒न्द्राणी दे॒वी सु॒भगा॑ सु॒पत्नी᳚॥\\
सं भूर्भुवस्सुवः- सोमाय अमृत कलश हस्ताय नक्षत्राधि पतये अश्व वाहनाय -\\
सांगाय सायुधाय सशक्ति परिवाराय - सर्वालंकार भूषिताय उमामहेश्वर पार्षदाय नमः ।\\
उत्तर दिग्भागे ह्रद यस्थाने सोमः सुप्रीतो सुप्रस्न्नो वरदो भवतु ॥}\\
ॐ भूर्भुव॒स्सुव॒रों ।\\
शं । तमी शा᳚न॒ञ् जग॑तस् त॒स्थु ष॒स्पतिम्᳚ धि॒यं॒ जि॒न्व मव॑से हूमहे व॒यम्।\\
पू॒षा नो॒ यथा॒ वेद॑सा॒ मस॑द् वृ॒धे र॒क्षि॒ता पायु॒र द॑ब् शस् स्व॒स् तये᳚ ॥\\
\\
शं ईशान दिग्भागे, ईशानाय त्रिशूल हस्ताय भूताधि पतये वृषभ वाहनाय\\
सांगाय सायुधाय सशक्तिस परिवाराय -  उमामहेश्वर पार्षदाय नमः।\\
शं ईशानाय नमः  ईशान्य दिग्भागे, ईशानः सुप्रीतो  वरदो भवतु ॥ (8)\\
{\small शं भूर्भुवस्सुवः ईशानाय त्रिशूल हस्ताय विद्याधि पतये भूताधि पतये वृषभ वाहनाय -\\
सांगाय सायुधाय सशक्ति परिवाराय - सर्वालंकार भूषिताय उमामहेश्वर पार्षदाय नमः ।\\
ईशान दिग्भागे नाभिस्थाने ईशानः सुप्रीतो सुप्रस्न्नो वरदो भवतु ॥}\\
\\
ॐ भूर्भुव॒स्सुव॒रों ।\\
खं । अ॒स्मे रु॒द्रा मे॒हना॒ पर्व॑ तासो वृत्र॒ हत्ये॒ भर॑ हूतौ स॒जोषाः᳚ ॥\\
यश् शंस॑ते स्तुव॒ते धायि॑ प॒ज्र इन्द्र॑ज् ज्येष्ठा, अ॒स्माम् अ॑वन्तु दे॒वाः ॥\\
\\
खं ऊर्ध्व दिग्भागे ब्रह्मणे पद्महस्ताय प्रजाधिपतये हंस वाहनाय\\
सांगाय सायुधाय सशक्तिस परिवाराय -  उमामहेश्वर पार्षदाय नमः।\\
खं ब्रह्मणे नमः । ऊर्ध्व दिग्भागे ब्रह्मा सुप्रीतो  वरदो भवतु ॥  (9)\\
{\small खं भूर्भुवस्सुवः ब्रह्मणे पद्म हस्ताय लोकाधि पतये हंस वाहनाय -\\
सांगाय सायुधाय सशक्ति परिवाराय - सर्वालंकार भूषिताय उमामहेश्वर पार्षदाय नमः ।\\
ऊर्ध्व दिग्भागे मूर्धस्थाने ब्रह्मा सुप्रीतो सुप्रस्न्नो वरदो भवतु ॥}\\
\\
ॐ भूर्भुव॒स्सुव॒रों ।\\
ह्रीं । स्यो॒ना पृ॑थि विभवा॑ऽ नृक्ष॒रा नि॒वे श॑नि । यच्छा॑ न॒श् शर्म॑ स॒प्रथाः᳚ ।\\
\\
ह्रीं अधो दिग्भागे विष्णवे चक्र हस्ताय लोखाधिपतये गरुड वाहनाय\\
सांगाय सायुधाय सशक्तिस परिवाराय -  उमामहेश्वर पार्षदाय नमः।\\
ह्रीं विष्णवे नमः । अधो दिग्भागे वष्णुस्सुप्रीतो वरदो भवतु ॥  (10)\\
{\small ह्रीं भूर्भुवस्सुवः विष्णवे चक्र हस्ताय नागाधि पतये गरुडवाहनाय -\\
सांगाय सायुधाय सशक्ति परिवाराय - सर्वालंकार भूषिताय उमामहेश्वर पार्षदाय नमः ।\\
अधोदिग्भागे पादस्थाने विष्णुस्सुुप्रीतो सुप्रस्न्नो वरदो भवतु ॥}\\
