\section{\eng{Laghunyasam}}
\\
ॐ अथात्मानग्ं शिवात्मानं श्री रुद्ररूप-न्ध्यायेत् ॥\\
\\
शुद्धस्फटिक सङ्काशन् त्रिनेत्रम् पञ्च वक्त्रकम् ।\\
गङ्गाधरन् दशभुजं सर्वा भरण भूषितम् ॥\\
\\
नीलग्रीवं शशाङ्काङ्कन् नाग यज्ञोप वीतिनम् ।\\
व्याघ्र चर्मोत् तरीयञ्च वरेण्य मभय प्रदम् ॥\\
\\
कमण्डल्-वक्ष सूत्राणान् धारिणं शूल पाणिनम् ।\\
ज्वलन्तम् पिङ्ग लजटा शिखा मुद्द्योत धारिणम् ॥\\
\\
वृष स्कन्ध समारूढं उमा देहार्थ धारिणम् ।\\
अमृतेनाप् लुतं शान्तन् दिव्य भोग समन्वितम् ॥\\
\\
दिग् देवता समा युक्तं सुरासुर नमस्कृतम् ।\\
नित्यञ् चशाश्व तं शुद्धन् ध्रुव-मक्षर-मव्ययम् ।\\
सर्व व्यापिन-मीशानं रुद्रं-वै विश्वरूपिणम् ।\\
एवन् ध्यात्वा द्विजस् सम्यक् ततो यजनमारभेत् ॥\\
\\
अथातो रुद्रस् नानार् चना भिषेक विधिं-व्या᳚ क्ष्यास्यामः ।\\
आदित एव तीर्थेस् नात्वा,\\
उदेत्य शुचिः प्रयतो ब्रह्मचारी शुक्लवासा देवाभिमुख-स्स्थित्वा,\\
आत्मनि देवता-स्स्थापयेत् ॥\\
\\
प्रजनने ब्रह्मा तिष्ठतु ।\\
पादयोर् विष्णुस्तिष्ठतु ।\\
हस्तयोर्​ हरस्तिष्ठतु ।\\
बाह्वो रिन्द्रस्तिष्टतु ।\\
जठरे-ऽअग्निस्तिष्ठतु ।\\
हृद॑ये शिवस्तिष्ठतु ।\\
कण्ठे वसवस्तिष्ठन्तु ।\\
वक्त्रे सरस्वती तिष्ठतु ।\\
नासिकयोर्-वायुस्तिष्ठतु ।\\
नयनयोश्-चन्द्रा दित्यौ तिष्टेताम् ।\\
कर्णयो रश्विनौ तिष्टेताम् ।\\
ललाटे रुद्रास्तिष्ठन्तु ।\\
मूर्थ्-न्यादित्-यास्तिष्ठन्तु ।\\
शिरसि महादेवस्तिष्ठतु ।\\
शिखायां-वाँमदेवास्तिष्ठतु ।\\
पृष्ठे पिनाकी तिष्ठतु ।\\
पुरतश्-शूली तिष्ठतु ।\\
पार्​श् वयोश् शिवा शङ्करौ तिष्ठेताम् ।\\
सर्वतो वायुस्तिष्ठतु ।\\
ततो बहिस् सर्वतो-ऽग्निर् ज्वाला माला-परि वृतस्तिष्ठतु ।\\
सर्वेष् वङ्गेषु सर्वा देवता यथास्थानन्-तिष्ठन्तु ।\\
माग्ं रक्षन्तु ।\\
\\
अ॒ग्निर्मे॑ वा॒चि श्रि॒तः । वाघृद॑ये । हृद॑य॒-म्मयि॑ । अ॒हम॒मृते᳚ । अ॒मृत॒-म्ब्रह्म॑णि ।\\
वा॒युर्मे᳚ प्रा॒णे श्रि॒तः । प्रा॒णो हृद॑ये । हृद॑य॒-म्मयि॑ । अ॒हम॒मृते᳚ । अ॒मृत॒-म्ब्रह्म॑णि ।\\
सूर्यो॑ मे॒ चक्षुषि श्रि॒तः । चक्षु॒र्​ हृद॑ये । हृद॑य॒-म्मयि॑ । अ॒हम॒मृते᳚ । अ॒मृत॒-म्ब्रह्म॑णि ।\\
च॒न्द्रमा॑ मे॒ मन॑सि श्रि॒तः । मनो॒ हृद॑ये । हृद॑य॒-म्मयि॑ । अ॒हम॒मृते᳚ । अ॒मृत॒-म्ब्रह्म॑णि ।\\
दिशो॑ मे॒ श्रोत्रे᳚ श्रि॒ताः । श्रोत्र॒ग्ं॒ हृद॑ये । हृद॑य॒-म्मयि॑ । अ॒हम॒मृते᳚ । अ॒मृत॒-म्ब्रह्म॑णि ।\\
आपोमे॒ रेतसि श्रि॒ताः । रेतो हृद॑ये । हृद॑य॒-म्मयि॑ । अ॒हम॒मृते᳚ । अ॒मृत॒-म्ब्रह्म॑णि ।\\
पृ॒थि॒वी मे॒ शरी॑रे श्रि॒ता । शरी॑र॒ग्ं॒ हृद॑ये । हृद॑य॒-म्मयि॑ । अ॒हम॒मृते᳚ । अ॒मृत॒-म्ब्रह्म॑णि ।\\
ओ॒ष॒धि॒ व॒न॒स्पतयो॑ मे॒ लोम॑सु श्रि॒ताः । लोमा॑नि॒ हृद॑ये । हृद॑य॒-म्मयि॑ । \\
अ॒हम॒मृते᳚ । अ॒मृत॒-म्ब्रह्म॑णि ।\\
इन्द्रो॑ मे॒ बले᳚ श्रि॒तः । बल॒ग्ं॒ हृद॑ये । हृद॑य॒-म्मयि॑ । अ॒हम॒मृते᳚ । अ॒मृत॒-म्ब्रह्म॑णि ।\\
प॒र्जन्यो॑ मे॒ मू॒र्द्नि श्रि॒तः । मू॒र्धा हृद॑ये । हृद॑य॒-म्मयि॑ । अ॒हम॒मृते᳚ । अ॒मृत॒-म्ब्रह्म॑णि ।\\
ईशा॑नो मे॒ म॒न्यौ श्रि॒तः । म॒न्युर्​ हृद॑ये । हृद॑य॒-म्मयि॑ । अ॒हम॒मृते᳚ । अ॒मृत॒-म्ब्रह्म॑णि ।\\
आ॒त्मा म॑ आ॒त्मनि॑ श्रि॒तः । आ॒त्मा हृद॑ये । हृद॑य॒-म्मयि॑ । \\
अ॒हम॒मृते᳚ । अ॒मृत॒-म्ब्रह्म॑णि ।\\
पुन॑र्म आ॒त्मा पुन॒रा यु॒रागा᳚त् । पुनः॑ प्रा॒णः पुन॒रा कू॑त॒मागा᳚त् । \\
वै॒श्वा॒ न॒रो र॒श्मिभि॑र् वा   वृधा॒नः ।\\
अ॒न्तस् ति॑ष् ठ॒त्व मृत॑स्य गो॒पाः ॥\\
\\
अस्य श्री रुद्राध्याय प्रश्न महामन्त्रस्य,\\
अघोर ऋषिः,\\
अनुष्टु-प्छन्दः,\\
सङ्कर्​षण मूर्ति स्वरूपो यो-ऽसावादित्यः परमपुरुष-स्स एष रुद्रो देवता ।\\
नमश् शिवायेति बीजम् ।\\
शिवत रायेति शक्तिः ।\\
महा देवा येति कीलकम् ।\\
श्री साम्ब सदाशिव प्रसाद सिद्ध्यर्थे जपे विनियोगः ॥\\
\\
ॐ अग्निहोत्रात्मने अङ्गुष्ठाभ्या-न्नमः ।\\
दर्​शपूर्ण मासात्मने तर्जनीभ्या-न्नमः ।\\
चातुर्मास्यात्मने मध्यमाभ्या-न्नमः ।\\
निरूढ पशुबन्धात्मने अनामिकाभ्या-न्नमः ।\\
ज्योतिष्टोमात्मने कनिष्ठिकाभ्या-न्नमः ।\\
सर्वक्रत्वात्मने करतल करपृष्ठाभ्या-न्नमः ॥\\
\\
अग्निहोत्रात्मने हृदयाय नमः ।\\
दर्​शपूर्ण मासात्मने शिरसे स्वाहा ।\\
चातुर्मास्यात्मने शिखायै वषट् ।\\
निरूढ पशुबन्धात्मने कवचाय हुम् ।\\
ज्योतिष्टोमात्मने नेत्रत्रयाय वौषट् ।\\
सर्वक्रत्वात्मने अस्त्रायफट् । भूर्भुवस्सुवरोमिति दिग्बन्धः ॥\\
\\
ध्यानं\\
आपाताल-नभस्स्थलान्त-भुवन-ब्रह्माण्ड-माविस्फुरत्-\\
ज्योति-स्स्फाटिक-लिङ्ग-मौलि-विलसत्-पूर्णेन्दु-वान्तामृतैः ।\\
अस्तोकाप्लुत-मेक-मीश-मनिशं रुद्रानु-वाकाञ्जपन्\\
ध्याये-दीप्सित-सिद्धये ध्रुवपदं-विँप्रो-ऽभिषिञ्चे-च्चिवम् ॥\\
\\
ब्रह्माण्ड व्याप्तदेहा भसित हिमरुचा भासमाना भुजङ्गैः\\
कण्ठे कालाः कपर्दाः कलित-शशिकला-श्चण्ड कोदण्ड हस्ताः ।\\
त्र्यक्षा रुद्राक्षमालाः प्रकटितविभवा-श्शाम्भवा मूर्तिभेदाः\\
रुद्रा-श्श्रीरुद्रसूक्त-प्रकटितविभवा नः प्रयच्चन्तु सौख्यम् ॥\\
\\
ओ-ङ्ग॒णाना᳚-न्त्वा ग॒णप॑तिग्ं हवामहे क॒वि-ङ्क॑वी॒नामु॑प॒मश्र॑वस्तमम् ।\\
ज्ये॒ष्ठ॒राज॒-म्ब्रह्म॑णा-म्ब्रह्मणस्पद॒ आ नः॑ शृ॒ण्वन्नू॒तिभि॑स्सीद॒ साद॑नम् ॥\\
महागणपतये॒ नमः ॥\\
\\
श-ञ्च॑ मे॒ मय॑श्च मे प्रि॒य-ञ्च॑ मे-ऽनुका॒मश्च॑ \\
मे॒ काम॑श्च मे सौमनस॒श्च॑ मे भ॒द्र-ञ्च॑ मे॒ \\
श्रेय॑श्च मे॒ वस्य॑श्च मे॒ यश॑श्च मे॒ भग॑श्च मे॒ \\
द्रवि॑ण-ञ्च मे य॒न्ता च॑ मे ध॒र्ता च॑ मे॒ क्षेम॑श्च मे॒ \\
धृति॑श्च मे॒ विश्व॑-ञ्च मे॒ मह॑श्च मे सं॒​विँच्च॑ मे॒ \\
ज्ञात्र॑-ञ्च मे॒ सूश्च॑ मे प्र॒सूश्च॑ मे॒ सीर॑-ञ्च मे \\
ल॒यश्च॑ म ऋ॒त-ञ्च॑ मे॒-ऽमृत॑-ञ्च मे-ऽय॒क्ष्म-ञ्च॒ \\
मे-ऽना॑मयच्च मे जी॒वातु॑श्च मे दीर्घायु॒त्व-ञ्च॑ \\
मे-ऽनमि॒त्र-ञ्च॒ मे-ऽभ॑य-ञ्च मे सु॒ग-ञ्च॑ मे॒ \\
शय॑न-ञ्च मे सू॒षा च॑ मे॒ सु॒दिन॑-ञ्च मे ॥\\
\\
ॐ शान्ति॒-श्शान्ति॒-श्शान्तिः॑ ॥\\
