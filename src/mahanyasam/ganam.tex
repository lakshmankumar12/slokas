\section{\eng{Ganam}}
\subsection{\eng{Ganapthi Dhyanam}}
1. ग॒णाना᳚म् । त्वा॒ । ग॒णप॑तिम् ।\\
ग॒णाना᳚म् त्वा त्वा ग॒णाना᳚म् ग॒णाना᳚म् त्वा ग॒णप॑तिम् ग॒णप॑तिम् त्वा\\
ग॒णाना᳚म् ग॒णाना᳚म् त्वा ग॒णप॑तिम् ।\\
\\
2. त्वा॒ । ग॒णप॑तिम् । ह॒वा॒म॒हे॒ \\
त्वा॒ ग॒णप॑तिम् ग॒णप॑तिम् त्वा त्वा ग॒णप॑तिꣳ हवामहे हवामहे\\
ग॒णप॑तिम् त्वा त्वा ग॒णप॑तिꣳ हवामहे ।\\
\\
3. ग॒णप॑तिम् । ह॒वा॒म॒हे॒ । क॒विम् ।\\
ग॒णप॑तिꣳ हवामहे हवामहे ग॒णप॑तिम् ग॒णप॑तिꣳ हवामहे क॒विम्\\
क॒विꣳ ह॑वामहे ग॒णप॑तिम् ग॒णप॑तिꣳ हवामहे क॒विम् ।\\
\\
4. ग॒णप॑तिम् ।\\
ग॒णप॑ति॒मिति॑ ग॒ण - प॒ति॒म् ।\\
\\
5. ह॒वा॒म॒हे॒ । क॒विम् । क॒वी॒नाम् ।\\
ह॒वा॒म॒हे॒ क॒विम् क॒विꣳ ह॑वामहे हवामहे क॒विम् क॑वी॒नाम् क॑वी॒नाम्\\
क॒विꣳ ह॑वामहे हवामहे क॒विम् क॑वी॒नाम् ।\\
\\
6. क॒विम् । क॒वी॒नाम् । उ॒प॒मश्र॑वस्तमम् ॥\\
क॒विम् क॑वी॒नाम् क॑वी॒नाम् क॒विम् क॒विम् क॑वी॒ना मु॑प॒मश्र॑वस्तम\\
मुप॒मश्र॑वस्तमम् कवी॒नाम् क॒विम् क॒विम् क॑वी॒ना मु॑प॒मश्र॑वस्तमम् ।\\
\\
7. क॒वी॒नाम् । उ॒प॒मश्र॑वस्तमम् ॥\\
क॒वी॒ना मु॑प॒मश्र॑वस्तम मुप॒मश्र॑वस्तमम् कवी॒नाम् क॑वी॒ना\\
मु॑प॒मश्र॑वस्तमम् ।\\
\\
8. उ॒प॒मश्र॑वस्तमम् ॥\\
उ॒प॒मश्र॑वस्तम॒ मित्यु॑प॒मश्र॑वः - त॒म॒म् ।\\
\\
9. ज्ये॒ष्ठ॒राज᳚म् । ब्रह्म॑णाम् । ब्र॒ह्म॒णः॒ ।\\
ज्ये॒ष्ठ॒राज॒म् ब्रह्म॑णा॒म् ब्रह्म॑णाम् ज्येष्ठ॒राज॑म् ज्येष्ठ॒राज॒म् ब्रह्म॑णाम् ब्रह्मणो\\
ब्रह्मणो॒ ब्रह्म॑णाम् ज्येष्ठ॒राज॑म् ज्येष्ठ॒राज॒म् ब्रह्म॑णाम् ब्रह्मणः ।\\
\\
10. ज्ये॒ष्ठ॒राज᳚म् ।\\
ज्ये॒ष्ठ॒राज॒मिति॑ ज्येष्ठ - राज᳚म् ।\\
\\
11. ब्रह्म॑णाम् । ब्र॒ह्म॒णः॒ । प॒ते॒ ।\\
ब्रह्म॑णाम् ब्रह्मणो ब्रह्मणो॒ ब्रह्म॑णा॒म् ब्रह्म॑णाम् ब्रह्मण स्पते पते ब्रह्मणो॒\\
ब्रह्म॑णा॒म् ब्रह्म॑णाम् ब्रह्मण स्पते ।\\
\\
12. ब्र॒ह्म॒णः॒ । प॒ते॒ । आ ।\\
ब्र॒ह्म॒ण॒ स्प॒ते॒ प॒ते॒ ब्र॒ह्म॒णो॒ ब्र॒ह्म॒ण॒ स्प॒त॒ आप॑ते ब्रह्मणो ब्रह्मण स्पत॒ आ ।\\
\\
13. प॒ते॒ । आ । नः॒ ।\\
प॒त॒ आ प॑ते पत॒ आ नो॑ न॒ आ प॑ते पत॒ आ नः॑ ।\\
\\
14. आ । नः॒ । शृ॒ण्वन्न् ।\\
आ नो॑न॒ आनः॑ शृ॒ण्वन् छृ॒ण्वन् न॒ आनः॑ शृ॒ण्वन्न् ।\\
\\
15. नः॒ । शृ॒ण्वन्न् । ऊ॒तिभिः॑ ।\\
नः॒ शृ॒ण्वन् छृ॒ण्वन् नो॑नः शृ॒ण्वन् नू॒तिभि॑ रू॒तिभिः॑ शृ॒ण्वन् नो॑नः\\
शृ॒ण्वन् नू॒तिभिः॑ ।\\
\\
16. शृ॒ण्वन्न् । ऊ॒तिभिः॑ । सी॒द॒ ।\\
शृ॒ण्वन् नू॒तिभि॑ रू॒तिभिः॑ शृ॒ण्वन् छृ॒ण्वन् नू॒तिभिः॑ सीद सीदो॒तिभिः॑\\
शृ॒ण्वन् छृ॒ण्वन् नू॒तिभीः॑ सीद ।\\
\\
17. ऊ॒तिभिः॑ । सी॒द॒ । साद॑नम् ॥\\
ऊ॒तिभिः॑ सीद सीदो॒तिभि॑ रू॒तिभिः॑ सीद॒ साद॑न॒ꣳ॒ साद॑नꣳ\\
सीदो॒तिभि॑ रू॒तिभिः॑ सीद साद॑नम् ।\\
\\
18. ऊ॒तिभिः॑ ।\\
ऊ॒तिभि॒रित्यू॒ति - भिः॒ ।\\
\\
19. सी॒द॒ । साद॑नम् ॥\\
सी॒द॒ साद॑न॒ꣳ॒ साद॑नꣳ सीद सीद॒ साद॑नम् ।\\
\\
20. साद॑नम् ॥\\
साद॑न॒मिति॒ साद॑नम् ।\\
\\
(श्री महागणपतये नमः)\\
\subsection{\eng{Anuvaka 1}}
1. नमः॑ । ते॒ । रु॒द्र॒ ।\\
नम॑स्ते ते॒ नमो॒ नम॑स्ते रुद्र रुद्र ते॒ नमो॒ नम॑स्ते रुद्र ।\\
\\
2. ते॒ । रु॒द्र॒ । म॒न्यवे᳚ ।\\
ते॒ रु॒द्र॒ रु॒द्र॒ ते॒ ते॒ रु॒द्र॒ म॒न्यवे॑ म॒न्यवे॑ रुद्र ते ते रुद्र म॒न्यवे᳚ ।\\
\\
3. रु॒द्र॒ । म॒न्यवे᳚ । उ॒तो ।\\
रु॒द्र॒ म॒न्यवे॑ म॒न्यवे॑ रुद्र रुद्र म॒न्यव॑ उ॒तो उ॒तो म॒न्यवे॑ रुद्र\\
रुद्र म॒न्यव॑ उ॒तो ।\\
\\
4. म॒न्यवे᳚ । उ॒तो । ते॒ ।\\
म॒न्यव॑ उ॒तो उ॒तो म॒न्यवे॑ म॒न्यव॑ उ॒तो ते॑ त उ॒तो म॒न्यवे॑ म॒न्यव॑ उ॒तो ते᳚ ।\\
\\
5. उ॒तो । ते॒ । इष॑वे ।\\
उ॒तो ते॑ त उ॒तो उ॒तो त॒ इष॑व॒ इष॑वे त उ॒तो उ॒तो त॒ इष॑वे ।\\
\\
6. उ॒तो ।\\
उ॒तो इत्यु॒तो ।\\
\\
7. ते॒ । इष॑वे । नमः॑ ॥\\
त॒ इष॑व॒ इष॑वे ते त॒ इष॑वे॒ नमो॒ नम॒ इष॑वे ते त॒ इष॑वे॒ नमः॑ ।\\
\\
8. इष॑वे । नमः॑ ॥\\
इष॑वे॒ नमो॒ नम॒ इष॑व॒ इष॑वे॒ नमः॑ ।\\
\\
9. नमः॑ ॥\\
नम॒ इति॒ नमः॑ ।\\
\\
10. नमः॑ । ते॒ । अ॒स्तु॒ ।\\
नम॑स्ते ते॒ नमो॒ नम॑स्ते अस्त्वस्तु ते॒ नमो॒ नम॑स्ते अस्तु ।\\
\\
11. ते॒ । अ॒स्तु॒ । धन्व॑ने ।\\
ते॒ अ॒स्त्व॒स्तु॒ ते॒ ते॒ अ॒स्तु॒ धन्व॑ने॒ धन्व॑ने अस्तु ते ते अस्तु॒ धन्व॑ने ।\\
\\
12. अ॒स्तु॒ । धन्व॑ने । बा॒हुभ्या᳚म् ।\\
अ॒स्तु॒ धन्व॑ने॒ धन्व॑ने अस्त्वस्तु॒ धन्व॑ने बा॒हुभ्यां᳚ बा॒हुभ्यां॒ धन्व॑ने\\
अस्त्वस्तु॒ धन्व॑ने बा॒हुभ्या᳚म् ।\\
\\
13. धन्व॑ने । बा॒हुभ्या᳚म् । उ॒त ।\\
धन्व॑ने बा॒हुभ्यां᳚ बा॒हुभ्यां॒ धन्व॑ने॒ धन्व॑ने बा॒हुभ्या॑ मु॒तोत बा॒हुभ्यां॒ धन्व॑ने॒\\
धन्व॑ने बा॒हुभ्या॑ मु॒त ।\\
\\
14. बा॒हुभ्या᳚म् । उ॒त । ते॒ ।\\
बा॒हुभ्या॑ मु॒तोत बा॒हुभ्यां᳚ बा॒हुभ्या॑ मु॒त ते॑ त उ॒त बा॒हुभ्यां᳚ बा॒हुभ्या॑ मु॒त ते᳚ ।\\
\\
15. बा॒हुभ्या᳚म् ।\\
बा॒हुभ्या॒मिति॑ बा॒हु - भ्या॒म् ।\\
\\
16. उ॒त । ते॒ । नमः॑ ॥\\
उ॒त ते॑ त उ॒तोत ते॒ नमो॒ नम॑स्त उ॒तोत ते॒ नमः॑ ।\\
\\
17. ते॒ । नमः॑ ॥\\
ते॒ नमो॒ नम॑स्ते ते॒ नमः॑ ।\\
\\
18. नमः॑ ॥\\
नम॒ इति॒ नमः॑ ।\\
\\
19. या । ते॒ । इषुः॑ ।\\
या ते॑ ते॒ या या त॒ इषु ॒रिषु॑ स्ते॒ या या त॒ इषुः॑ ।\\
\\
20. ते॒ । इषुः॑ । शि॒वत॑मा ।\\
त॒ इषु॒ रिषु॑ स्ते त॒ इषुः॑ शि॒वत॑मा शि॒वत॒ मेषु॑ स्ते त॒ इषुः॑ शि॒वत॑मा ।\\
\\
21. इषुः॑ । शि॒वत॑मा । शि॒वम् ।\\
इषुः॑ शि॒वत॑मा शि॒वत॒ मेषु॒ रिषुः॑ शि॒वत॑मा शि॒वꣳ शि॒वꣳ शि॒वत॒\\
मेषु॒ रिषुः॑ शि॒वत॑मा शि॒वम् ।\\
\\
22. शि॒वत॑मा । शि॒वम् । ब॒भूव॑ ।\\
शि॒वत॑मा शि॒वꣳ शि॒वꣳ शि॒वत॑मा शि॒वत॑मा शि॒वं ब॒भूव॑ ब॒भूव॑\\
शि॒वꣳ शि॒वत॑मा शि॒वत॑मा शि॒वं ब॒भूव॑ ।\\
\\
23. शि॒वत॑मा ।\\
शि॒वत॒मेति॑ शि॒व - त॒मा॒ ।\\
\\
24. शि॒वम् । ब॒भूव॑ । ते॒ ।\\
शि॒वं ब॒भूव॑ ब॒भूव॑ शि॒वꣳ शि॒वं ब॒भूव॑ ते ते ब॒भूव॑ शि॒वꣳ\\
 शि॒वं ब॒भूव॑ ते ।\\
\\
25. ब॒भूव॑ । ते॒ । धनुः॑ ॥\\
ब॒भूव॑ ते ते ब॒भूव॑ ब॒भूव॑ ते॒ धनु॒र् धनु॑ स्ते ब॒भूव॑ ब॒भूव॑ ते॒ धनुः॑ ।\\
\\
26. ते॒ । धनुः॑ ॥\\
ते॒ धनु॒र् धनु॑ स्ते ते॒ धनुः॑ ।\\
\\
27. धनुः॑ ॥\\
धनु॒रिति॒ धनुः॑ ।\\
\\
28. शि॒वा । श॒र॒व्या᳚ । या ।\\
शि॒वा श॑र॒व्या॑ शर॒व्या॑ शि॒वा शि॒वा श॑र॒व्या॑ या या श॑र॒व्या॑ शि॒वा\\
शि॒वा श॑र॒व्या॑ या ।\\
\\
29. श॒र॒व्या᳚ । या । तव॑ ।\\
श॒र॒व्या॑ या या श॑र॒व्या॑ शर॒व्या॑ या तव॒ तव॒ या श॑र॒व्या॑ शर॒व्या॑ या तव॑ ।\\
\\
30. या । तव॑ । तया᳚ ।\\
या तव॒ तव॒ या या तव॒ तया॒ तया॒ तव॒ या या तव॒ तया᳚ ।\\
\\
31. तव॑ । तया᳚ । नः॒ ।\\
तव॒ तया॒ तया॒ तव॒ तव॒ तया॑ नो न॒ स्तया॒ तव॒ तव॒ तया॑ नः ।\\
\\
32. तया᳚ । नः॒ । रु॒द्र॒ ।\\
तया॑ नो न॒ स्तया॒ तया॑ नो रुद्र रुद्र न॒ स्तया॒ तया॑ नो रुद्र ।\\
\\
33. नः॒ । रु॒द्र॒ । मृ॒ड॒य॒ ॥\\
नो॒ रु॒द्र॒ रु॒द्र॒ नो॒ नो॒ रु॒द्र॒ मृ॒ड॒य॒ मृ॒ड॒य॒ रु॒द्र॒ नो॒ नो॒ रु॒द्र॒ मृ॒ड॒य॒ ।\\
\\
34. रु॒द्र॒ । मृ॒ड॒य॒ ॥\\
रु॒द्र॒ मृ॒ड॒य॒ मृ॒ड॒य॒ रु॒द्र॒ रु॒द्र॒ मृ॒ड॒य॒ ।\\
\\
35. मृ॒ड॒य॒ ॥\\
मृ॒ड॒येति॑ मृडय ।\\
\\
36. या । ते॒ । रु॒द्र॒ ।\\
या ते॑ ते॒ या या ते॑ रुद्र रुद्र ते॒ या या ते॑ रुद्र ।\\
\\
37. ते॒ । रु॒द्र॒ । शि॒वा ।\\
ते॒ रु॒द्र॒ रु॒द्र॒ ते॒ ते॒ रु॒द्र॒ शि॒वा शि॒वा रु॑द्र ते ते रुद्र शि॒वा ।\\
\\
38. रु॒द्र॒ । शि॒वा । त॒नूः ।\\
रु॒द्र॒ शि॒वा शि॒वा रु॑द्र रुद्र शि॒वा त॒नू स्त॒नूः शि॒वा रु॑द्र रुद्र शि॒वा त॒नूः ।\\
\\
39. शि॒वा । त॒नूः । अघो॑रा ।\\
शि॒वा त॒नू स्त॒नूः शि॒वा शि॒वा त॒नू रघो॒राऽघो॑रा त॒नूः शि॒वा शि॒वा\\
त॒नू रघो॑रा ।\\
\\
40. त॒नूः । अघो॑रा । अपा॑पकाशिनी ॥\\
त॒नू रघो॒राऽघो॑रा त॒नू स्त॒नू रघो॒राऽपा॑पकाशि॒ न्यपा॑पकाशि॒\\
न्यघो॑रा त॒नू स्त॒नू रघो॒राऽपा॑पकाशिनी ।\\
\\
41. अघो॑रा । अपा॑पकाशिनी ॥\\
अघो॒राऽपा॑पकाशि॒ न्यपा॑पकाशि॒ न्यघो॒राऽघो॒राऽपा॑पकाशिनी ।\\
\\
42. अपा॑पकाशिनी ॥\\
अपा॑पकाशि॒नीत्यपा॑प - का॒शि॒नी॒ ।\\
\\
43. तया᳚ । नः॒ । त॒नुवा᳚ ।\\
तया॑ नो न॒ स्तया॒ तया॑ न स्त॒नुवा॑ त॒नुवा॑ न॒ स्तया॒ तया॑ न स्त॒नुवा᳚ ।\\
\\
44. नः॒ । त॒नुवा᳚ । शन्त॑मया ।\\
न॒ स्त॒नुवा॑ त॒नुवा॑ नो न स्त॒नुवा॒ शन्त॑मया॒ शन्त॑मया त॒नुवा॑ नो न\\
स्त॒नुवा॒ शन्त॑मया ।\\
\\
45. त॒नुवा᳚ । शन्त॑मया । गिरि॑शन्त ।\\
त॒नुवा॒ शन्त॑मया॒ शन्त॑मया त॒नुवा॑ त॒नुवा॒ शन्त॑मया॒ गिरि॑शन्त॒ गिरि॑शन्त॒\\
शन्त॑मया त॒नुवा॑ त॒नुवा॒ शन्त॑मया॒ गिरि॑शन्त ।\\
\\
46. शन्त॑मया । गिरि॑शन्त । अ॒भि ।\\
शन्त॑मया॒ गिरि॑शन्त॒ गिरि॑शन्त॒ शन्त॑मया॒ शन्त॑मया॒ गिरि॑शन्ता॒ भ्य॑भि\\
गिरि॑शन्त॒ शन्त॑मया॒ शन्त॑मया॒ गिरि॑शन्ता॒ भि ।\\
\\
47. शन्त॑मया ।\\
शन्त॑म॒येति॒ शं - त॒म॒या॒ ।\\
\\
48. गिरि॑शन्त । अ॒भि । चा॒क॒शी॒हि॒ ॥\\
गिरि॑शन्ता॒ भ्य॑भि गिरि॑शन्त॒ गिरि॑शन्ता॒भि चा॑कशीहि चाकशी ह्य॒भि\\
गिरि॑शन्त॒ गिरि॑शन्ता॒ भिचा॑कशीहि ।\\
\\
49. गिरि॑शन्त ।\\
गिरि॑श॒न्तेति॒ गिरि॑ - श॒न्त॒ ।\\
\\
50. अ॒भि । चा॒क॒शी॒हि॒ ॥\\
अ॒भि चा॑कशीहि चाकशी ह्य॒भ्य॑भि चा॑कशीहि ।\\
\\
51. चा॒क॒शी॒हि॒ ॥\\
चा॒क॒शी॒हीति॑ चाकशीहि ।\\
\\
52. याम् । इषु᳚म् । गि॒रि॒श॒न्त॒ ।\\
या मिषु॒ मिषुं॒ँयांँया मिषुं॑ गिरिशन्त गिरिश॒ न्तेषुं॒ँयांँया मिषुं॑\\
गिरिशन्त ।\\
\\
53. इषु᳚म् । गि॒रि॒श॒न्त॒ । हस्ते᳚ ।\\
इषुं॑ गिरिशन्त गिरिश॒ न्तेषु॒ मिषुं॑ गिरिशन्त॒ हस्ते॒ हस्ते॑ गिरिश॒ न्तेषु॒ मिषुं॑\\
गिरिशन्त॒ हस्ते᳚ ।\\
\\
54. गि॒रि॒श॒न्त॒ । हस्ते᳚ । बिभ॑र्.षि ।\\
गि॒रि॒श॒न्त॒ हस्ते॒ हस्ते॑ गिरिशन्त गिरिशन्त॒ हस्ते॒ बिभ॑र्.षि॒ बिभ॑र्.षि॒ हस्ते॑\\
गिरिशन्त गिरिशन्त॒ हस्ते॒ बिभ॑र्.षि ।\\
\\
55. गि॒रि॒श॒न्त॒ ।\\
गि॒रि॒श॒न्तेति॑ गिरि - श॒न्त॒ ।\\
\\
56. हस्ते᳚ । बिभ॑र्.षि । अस्त॑वे ॥\\
हस्ते॒ बिभ॑र्.षि॒ बिभ॑र्.षि॒ हस्ते॒ हस्ते॒ बिभ॒र्ष्यस्त॑वे॒ अस्त॑वे॒ बिभ॑र्.षि॒\\
हस्ते॒ हस्ते॒ बिभ॒र्ष्यस्त॑वे ।\\
\\
57. बिभ॑र्.षि । अस्त॑वे ॥\\
बिभ॒र्ष्यस्त॑वे॒ अस्त॑वे॒ बिभ॑र्.षि॒ बिभ॒र्ष्यस्त॑वे ।\\
\\
58. अस्त॑वे ॥\\
अस्त॑व॒ इत्यस्त॑वे ।\\
\\
59. शि॒वाम् । गि॒रि॒त्र॒ । ताम् ।\\
शि॒वां गि॑रित्र गिरित्र शि॒वाꣳ शि॒वां गि॑रित्र॒ तां तां गि॑रित्र शि॒वाꣳ\\
शि॒वां गि॑रित्र॒ ताम् ।\\
\\
60. गि॒रि॒त्र॒ । ताम् । कु॒रु॒ ।\\
गि॒रि॒त्र॒ तां तां गि॑रित्र गिरित्र॒ तां कु॑रु कुरु॒ तां गि॑रित्र गिरित्र॒ तां कु॑रु ।\\
\\
61. गि॒रि॒त्र॒ ।\\
गि॒रि॒त्रेति॑ गिरि - त्र॒ ।\\
\\
62. ताम् । कु॒रु॒ । मा ।\\
तां कु॑रु कुरु॒ तां तां कु॑रु॒ मा मा कु॑रु॒ तां तां कु॑रु॒ मा ।\\
\\
63. कु॒रु॒ । मा । हि॒ꣳ॒सीः॒ ।\\
कु॒रु॒ मा मा कु॑रु कुरु॒ मा हिꣳ॑सीर्. हिꣳसी॒र् मा कु॑रु कुरु॒ मा हिꣳ॑सीः ।\\
\\
64. मा । हि॒ꣳ॒सीः॒ । पुरु॑षम् ।\\
मा हिꣳ॑सीर्. हिꣳसी॒र् मा मा हिꣳ॑सीः॒ पुरु॑षं॒ पुरु॑षꣳ हिꣳसी॒र् मा मा\\
हिꣳ॑सीः॒ पुरु॑षम् ।\\
\\
65. हि॒ꣳ॒सीः॒ । पुरु॑षम् । जग॑त् ॥\\
हि॒ꣳ॒सीः॒ पुरु॑ष॒म् पुरु॑षꣳ हिꣳसीर्. हिꣳसीः॒ पुरु॑षं॒ जग॒ज् जग॒त् पुरु॑षꣳ\\
हिꣳसीर्. हिꣳसीः॒ पुरु॑षं॒ जग॑त् ।\\
\\
66. पुरु॑षम् । जग॑त् ॥\\
पुरु॑षं॒ जग॒ज् जग॒त् पुरु॑षं॒ पुरु॑षं॒ जग॑त् ।\\
\\
67. जग॑त् ॥\\
जग॒दिति॒ जग॑त् ।\\
\\
68. शि॒वेन॑ । वच॑सा । त्वा॒ ।\\
शि॒वेन॒ वच॑सा॒ वच॑सा शि॒वेन॑ शि॒वेन॒ वच॑सा त्वा त्वा॒ वच॑सा शि॒वेन॑\\
शि॒वेन॒ वच॑सा त्वा ।\\
\\
69. वच॑सा । त्वा॒ । गिरि॑श ।\\
वच॑सा त्वा त्वा॒ वच॑सा॒ वच॑सा त्वा॒ गिरि॑श॒ गिरि॑श त्वा॒ वच॑सा॒ वच॑सा\\
त्वा॒ गिरि॑श ।\\
\\
70. त्वा॒ । गिरि॑श । अच्छ॑ ।\\
त्वा॒ गिरि॑श॒ गिरि॑श त्वा त्वा॒ गिरि॒शा च्छाच्छ॒ गिरि॑श त्वा त्वा॒ गिरि॒शाच्छ॑ ।\\
\\
71. गिरि॑श । अच्छ॑ । व॒दा॒म॒सि॒ ॥\\
गिरि॒शा च्छाच्छ॒ गिरि॑श॒ गिरि॒शाच्छा॑ वदामसि वदाम॒ स्यच्छ॒ गिरि॑श॒\\
गिरि॒शाच्छा॑ वदामसि ॥\\
\\
72. अच्छ॑ । व॒दा॒म॒सि॒ ॥\\
अच्छा॑ वदामसि वदाम॒ स्यच्छाच्छा॑ वदामसि ।\\
\\
73. व॒दा॒म॒सि॒ ।\\
व॒दा॒म॒सीति॑ वदामसि ।\\
\\
74. यथा᳚ । नः॒ । सर्व᳚म् ।\\
यथा॑ नो नो॒ यथा॒ यथा॑ नः॒ सर्व॒ꣳ॒ सर्व॑न्नो॒ यथा॒ यथा॑ नः॒ सर्व᳚म् ।\\
\\
75. नः॒ । सर्व᳚म् । इत् ।\\
नः॒ सर्व॒ꣳ॒ सर्व॑न्नो नः॒ सर्व॒ मिदिथ् सर्व॑न्नो नः॒ सर्व॒ मित् ।\\
\\
76. सर्व᳚म् । इत् । जग॑त् ।\\
सर्व॒ मिदिथ् सर्व॒ꣳ॒ सर्व॒ मिज् जग॒ज् जग॒दिथ् सर्व॒ꣳ॒ सर्व॒\\
मिज् जग॑त् ।\\
\\
77. इत् । जग॑त् । अ॒य॒क्ष्मम् ।\\
इज् जग॒ज् जग॒दिदिज् जग॑द य॒क्ष्म म॑य॒क्ष्मं जग॒दिदिज् जग॑द\\
य॒क्ष्मम् ।\\
\\
78. जग॑त् । अ॒य॒क्ष्मम् । सु॒मनाः᳚ ।\\
जग॑द य॒क्ष्म म॑य॒क्ष्मं जग॒ज् जग॑द य॒क्ष्मꣳ सु॒मनाः᳚ सु॒मना॑ अय॒क्ष्मं\\
जग॒ज् जग॑द य॒क्ष्मꣳ सु॒मनाः᳚ ।\\
\\
79. अ॒य॒क्ष्मम् । सु॒मनाः᳚ । अस॑त् ॥\\
अ॒य॒क्ष्मꣳ सु॒मनाः᳚ सु॒मना॑ अय॒क्ष्म म॑य॒क्ष्मꣳ सु॒मना॒ अस॒ दस॑थ्\\
सु॒मना॑ अय॒क्ष्म म॑य॒क्ष्मꣳ सु॒मना॒ अस॑त् ।\\
\\
80. सु॒मनाः᳚ । अस॑त् ॥\\
सु॒मना॒ अस॒दस॑थ् सु॒मनाः᳚ सु॒मना॒ अस॑त् ।\\
\\
81. सु॒मनाः᳚ ।\\
सु॒मना॒ इति॑ सु - मनाः᳚ ।\\
\\
82. अस॑त् ॥\\
अस॒दित्यस॑त् ।\\
\\
83. अधि॑ । अ॒वो॒च॒त् । अ॒धि॒व॒क्ता ।\\
अध्य॑ वोच दवोच॒ दध्यध्य॑ वोच दधिव॒क्ताऽधि॑व॒क्ताऽवो॑च॒ दध्यध्य॑ वोच\\
दधिव॒क्ता ।\\
\\
84. अ॒वो॒च॒त् । अ॒धि॒व॒क्ता । प्र॒थ॒मः ।\\
अ॒वो॒च॒ द॒धि॒व॒क्ताऽधि॑व॒क्ताऽवो॑च दवोच दधिव॒क्ता प्र॑थ॒मः प्र॑थ॒मो\\
अ॑धिव॒क्ताऽवो॑च दवोच दधिव॒क्ता प्र॑थ॒मः ।\\
\\
85. अ॒धि॒व॒क्ता । प्र॒थ॒मः । दैव्यः॑ ।\\
अ॒धि॒व॒क्ता प्र॑थ॒मः प्र॑थ॒मो अ॑धिव॒क्ताऽधि॑व॒क्ता प्र॑थ॒मो दैव्यो॒ दैव्यः॑ प्रथ॒मो\\
अ॑धिव॒क्ताऽधि॑व॒क्ता प्र॑थ॒मो दैव्यः॑ ।\\
\\
86. अ॒धि॒व॒क्ता ।\\
अ॒धि॒व॒क्तेत्य॑धि - व॒क्ता ।\\
\\
87. प्र॒थ॒मः । दैव्यः॑ । भि॒षक् ॥\\
प्र॒थ॒मो दैव्यो॒ दैव्यः॑ प्रथ॒मः प्र॑थ॒मो दैव्यो॑ भि॒षग् भि॒षग् दैव्यः॑ प्रथ॒मः\\
प्र॑थ॒मो दैव्यो॑ भि॒षक् ।\\
\\
88. दैव्यः॑ । भि॒षक् ॥\\
दैव्यो॑ भि॒षग् भि॒षग् दैव्यो॒ दैव्यो॑ भि॒षक् ।\\
\\
89. भि॒षक् ॥\\
भि॒षगिति॑ भि॒षक् ।\\
\\
90. अहीन्॑ । च॒ । सर्वान्॑ ।\\
अहीꣲ॑श्च॒ चाही॒ नहीꣲ॑श्च॒ सर्वा॒न् थ्सर्वा॒ꣲ॒ श्चाही॒ नहीꣲ॑श्च॒ सर्वान्॑ ।\\
\\
91. च॒ । सर्वा᳚न् । जं॒भयन्न्॑ ।\\
च॒ सर्वा॒न् थ्सर्वाꣲ॑श्च च॒ सर्वा᳚न् ज॒म्भय॑न् ज॒म्भय॒न् थ्सर्वाꣲ॑श्च च॒\\
सर्वा᳚न् ज॒म्भयन्न्॑ ।\\
\\
92. सर्वा᳚न् । जं॒भयन्न्॑ । सर्वाः᳚ ।\\
सर्वा᳚न् ज॒म्भय॑न् ज॒म्भय॒न् थ्सर्वा॒न् थ्सर्वा᳚न् ज॒म्भय॒न् थ्सर्वाः॒ सर्वा॑\\
ज॒म्भय॒न् थ्सर्वा॒न् थ्सर्वा᳚न् ज॒म्भय॒न् थ्सर्वाः᳚ ।\\
\\
93. जं॒भयन्न्॑ । सर्वाः᳚ । च॒ ।\\
ज॒म्भय॒न् थ्सर्वाः॒ सर्वा॑ ज॒म्भय॑न् ज॒म्भय॒न् थ्सर्वा᳚श्च च॒ सर्वा॑\\
ज॒म्भय॑न् ज॒म्भय॒न् थ्सर्वा᳚श्च ।\\
\\
94. सर्वाः᳚ । च॒ । या॒तु॒धा॒न्यः॑ ॥\\
सर्वा᳚श्च च॒ सर्वाः॒ सर्वा᳚श्च यातुधा॒न्यो॑ यातुधा॒न्य॑श्च॒ सर्वाः॒ सर्वा᳚श्च\\
यातुधा॒न्यः॑ ।\\
\\
95. च॒ । या॒तु॒धा॒न्यः॑ ॥\\
च॒ या॒तु॒धा॒न्यो॑ यातुधा॒न्य॑श्च च यातुधा॒न्यः॑ ।\\
\\
96. या॒तु॒धा॒न्यः॑ ॥\\
या॒तु॒धा॒न्य॑ इति॑ यातु - धा॒न्यः॑ ।\\
\\
97. अ॒सौ । यः । ता॒म्रः ।\\
अ॒सौ यो यो अ॒सा व॒सौ य स्ता॒म्र स्ता॒म्रो यो अ॒सा व॒सौ य स्ता॒म्रः ।\\
\\
98. यः । ता॒म्रः । अ॒रु॒णः ।\\
य स्ता॒म्र स्ता॒म्रो यो य स्ता॒म्रो अ॑रु॒णो अ॑रु॒ण स्ता॒म्रो यो य स्ता॒म्रो\\
अ॑रु॒णः ।\\
\\
99. ता॒म्रः । अ॒रु॒णः । उ॒त ।\\
ता॒म्रो अ॑रु॒णो अ॑रु॒ण स्ता॒म्र स्ता॒म्रो अ॑रु॒ण उ॒तोतारु॒ण स्ता॒म्र स्ता॒म्रो\\
अ॑रु॒ण उ॒त ।\\
\\
100. अ॒रु॒णः । उ॒त । ब॒भ्रुः ।\\
अ॒रु॒ण उ॒तोतारु॒णो अ॑रु॒ण उ॒त ब॒भ्रुर् ब॒भ्रु रु॒तारु॒णो अ॑रु॒ण उ॒त ब॒भ्रुः ।\\
\\
101. उ॒त । ब॒भ्रुः । सु॒म॒ङ्गलः॑ ॥\\
उ॒त ब॒भ्रुर् ब॒भ्रु रु॒तोत ब॒भ्रुः सु॑म॒ङ्गलः॑ सुम॒ङ्गलो॑ ब॒भ्रु रु॒तोत ब॒भ्रुः\\
सु॑म॒ङ्गलः॑ ।\\
\\
102. ब॒भ्रुः । सु॒म॒ङ्गलः॑ ॥\\
ब॒भ्रुः सु॑म॒ङ्गलः॑ सुम॒ङ्गलो॑ ब॒भ्रुर् ब॒भ्रुः सु॑म॒ङ्गलः॑ ।\\
\\
103. सु॒म॒ङ्गलः॑ ॥\\
सु॒म॒ङ्गल॒ इति॑ सु - म॒ङ्गलः॑ ।\\
\\
104. ये । च॒ । इ॒माम् ।\\
ये च॑ च॒ ये ये चे॒मा मि॒माञ्च॒ ये ये चे॒माम् ।\\
\\
105. च॒ । इ॒माम् । रु॒द्राः ।\\
चे॒मा मि॒माञ्च॑ चे॒माꣳ रु॒द्रा रु॒द्रा इ॒माञ्च॑ चे॒माꣳ रु॒द्राः ।\\
\\
106. इ॒माम् । रु॒द्राः । अ॒भितः॑ ।\\
इ॒माꣳ रु॒द्रा रु॒द्रा इ॒मा मि॒माꣳ रु॒द्रा अ॒भितो॑ अ॒भितो॑ रु॒द्रा इ॒मा मि॒माꣳ\\
रु॒द्रा अ॒भितः॑ ।\\
\\
107. रु॒द्राः । अ॒भितः॑ । दि॒क्षु ।\\
रु॒द्रा अ॒भितो॑ अ॒भितो॑ रु॒द्रा रु॒द्रा अ॒भिताे॑ दि॒क्षु दि॒क्ष्व॑भितो॑ रु॒द्रा रु॒द्रा\\
अ॒भितो॑ दि॒क्षु ।\\
\\
108. अ॒भितः॑ । दि॒क्षु । श्रि॒ताः ।\\
अ॒भितो॑ दि॒क्षु दि॒क्ष्व॑भितो॑ अ॒भितो॑ दि॒क्षु श्रि॒ताः श्रि॒ता दि॒क्ष्व॑भितो॑\\
अ॒भितो॑ दि॒क्षु श्रि॒ताः ।\\
\\
109. दि॒क्षु । श्रि॒ताः । स॒ह॒स्र॒शः ।\\
दि॒क्षु श्रि॒ताः श्रि॒ता दि॒क्षु दि॒क्षु श्रि॒ताः स॑हस्र॒शः स॑हस्र॒शः श्रि॒ता दि॒क्षु\\
दि॒क्षु श्रि॒ताः स॑हस्र॒शः ।\\
\\
110. श्रि॒ताः । स॒ह॒स्र॒शः । अव॑ ।\\
श्रि॒ताः स॑हस्र॒शः स॑हस्र॒शः श्रि॒ताः श्रि॒ताः स॑हस्र॒शोऽवाव॑ सहस्र॒शः\\
श्रि॒ताः श्रि॒ताः स॑हस्र॒शोऽव॑ ।\\
\\
111. स॒ह॒स्र॒शः । अव॑ । ए॒षा॒म् ।\\
स॒ह॒स्र॒शोऽवाव॑ सहस्र॒शः स॑हस्र॒शोऽवै॑षा मेषा॒ मव॑ सहस्र॒शः स॑हस्र॒शोऽवै॑षाम् ।\\
\\
112. स॒ह॒स्र॒शः ।\\
स॒ह॒स्र॒श इति॑ सहस्र - शः ।\\
\\
113. अव॑ । ए॒षा॒म् । हेडः॑ ।\\
अवै॑षा मेषा॒ मवा वै॑षा॒ꣳ॒ हेडो॒ हेड॑ एषा॒ मवा वै॑षा॒ꣳ॒ हेडः॑ ।\\
\\
114. ए॒षा॒म् । हेडः॑ । ई॒म॒हे॒ ॥\\
ए॒षा॒ꣳ॒ हेडो॒ हेड॑ एषा मेषा॒ꣳ॒ हेड॑ ईमह ईमहे॒ हेड॑ एषा मेषा॒ꣳ॒\\
हेड॑ ईमहे ।\\
\\
115. हेडः॑ । ई॒म॒हे॒ ॥\\
हेड॑ ईमह ईमहे॒ हेडो॒ हेड॑ ईमहे ।\\
\\
116. ई॒म॒हे॒ ॥\\
ई॒म॒ह॒ इती॑ महे ।\\
\\
117. अ॒सौ । यः । अ॒व॒सर्प॑ति ।\\
अ॒सौयो यो अ॒सा व॒सौ यो॑ऽव॒सर्प॑ त्यव॒सर्प॑ति॒यो अ॒सा व॒सौ यो॑ऽव॒सर्प॑ति ।\\
\\
118. यः । अ॒व॒सर्प॑ति । नील॑ग्रीवः ।\\
यो॑ऽव॒सर्प॑ त्यव॒सर्प॑ति॒ यो यो॑ऽव॒सर्प॑ति॒ नील॑ग्रीवो॒ नील॑ग्रीवोऽव॒सर्प॑ति॒ यो यो॑ऽव॒सर्प॑ति॒ नील॑ग्रीवः ।\\
\\
119. अ॒व॒सर्प॑ति । नील॑ग्रीवः । विलो॑हितः ॥\\
अ॒व॒सर्प॑ति॒ नील॑ग्रीवो॒ नील॑ग्रीवोऽव॒सर्प॑ त्यव॒सर्प॑ति॒ नील॑ग्रीवो॒ विलो॑हितो॒\\
विलो॑हितो॒ नील॑ग्रीवोऽव॒सर्प॑ त्यव॒सर्प॑ति॒ नील॑ग्रीवो॒ विलो॑हितः ।\\
\\
120. अ॒व॒सर्प॑ति ।\\
अ॒व॒सर्प॒तीत्य॑व - सर्प॑ति ।\\
\\
121. नील॑ग्रीवः । विलो॑हितः ॥\\
नील॑ग्रीवो॒ विलो॑हितो॒ विलो॑हितो॒ नील॑ग्रीवो॒ नील॑ग्रीवो॒ विलो॑हितः ।\\
\\
122. नील॑ग्रीवः ।\\
नील॑ग्रीव॒ इति॒ नील॑ - ग्री॒वः॒ ।\\
\\
123. विलो॑हितः ॥\\
विलो॑हित॒ इति॒ वि - लो॒हि॒तः॒ ।\\
\\
124. उ॒त । ए॒न॒म् । गो॒पाः ।\\
उ॒तैन॑ मेन मु॒तोतैनं॑ गो॒पा गो॒पा ए॑न मु॒तोतैनं॑ गो॒पाः ।\\
\\
125. ए॒न॒म् । गो॒पाः । अ॒दृ॒श॒न्न् ।\\
ए॒नं॒ गो॒पा गो॒पा ए॑न मेनं गो॒पा अ॑दृशन् नदृशन् गो॒पा ए॑न मेनं गो॒पा\\
अ॑दृशन्न् ।\\
\\
126. गो॒पाः । अ॒दृ॒श॒न्न् । अदृ॑शन्न् ।\\
गो॒पा अ॑दृशन् नदृशन् गो॒पा गो॒पा अ॑दृश॒न् नदृ॑श॒न् नदृ॑शन् नदृशन्\\
गो॒पा गो॒पा अ॑दृश॒न् नदृ॑शन्न् ।\\
\\
127. गो॒पाः ।\\
गो॒पा इति॑ गो - पाः ।\\
\\
128. अ॒दृ॒श॒न्न् । अदृ॑शन्न् । उ॒द॒हा॒र्यः॑ ॥\\
अ॒दृ॒श॒न् नदृ॑श॒न् नदृ॑शन् नदृशन् नदृश॒न् नदृ॑शन् नुदहा॒र्य॑\\
उदहा॒र्यो॑ अदृ॑शन् नदृशन् नदृश॒न् नदृ॑शन् नुदहा॒र्यः॑ ।\\
\\
129. अदृ॑शन्न् । उ॒द॒हा॒र्यः॑ ॥\\
अदृ॑शन् नुदहा॒र्य॑ उदहा॒र्यो॑ अदृ॑श॒न् नदृ॑शन् नुदहा॒र्यः॑ ।\\
\\
130. उ॒द॒हा॒र्यः॑ ॥\\
उ॒द॒हा॒र्य॑ इत्यु॑द - हा॒र्यः॑ ।\\
\\
131. उ॒त । ए॒न॒म् । विश्वा᳚ ।\\
उ॒तैन॑ मेन मु॒तोतैनं॒ँविश्वा॒ विश्वै॑न मु॒तोतैनं॒ँविश्वा᳚ ।\\
\\
132. ए॒न॒म् । विश्वा᳚ । भू॒तानि॑ ।\\
ए॒नं॒ँविश्वा॒ विश्वै॑न मेनं॒ँविश्वा॑ भू॒तानि॑ भू॒तानि॒ विश्वै॑न मेनं॒ँविश्वा॑ भू॒तानि॑ ।\\
\\
133. विश्वा᳚ । भू॒तानि॑ । सः ।\\
विश्वा॑ भू॒तानि॑ भू॒तानि॒ विश्वा॒ विश्वा॑ भू॒तानि॒ स स भू॒तानि॒ विश्वा॒ विश्वा॑\\
भू॒तानि॒ सः ।\\
\\
134. भू॒तानि॑ । सः । दृ॒ष्टः ।\\
भू॒तानि॒ स स भू॒तानि॑ भू॒तानि॒ स दृ॒ष्टो दृ॒ष्टः स भू॒तानि॑ भू॒तानि॒ स दृ॒ष्टः ।\\
\\
135. सः । दृ॒ष्टः । मृ॒ड॒या॒ति॒ ।\\
स दृ॒ष्टो दृ॒ष्टः स स दृ॒ष्टो मृ॑डयाति मृडयाति दृ॒ष्टः स स दृ॒ष्टो मृ॑डयाति ।\\
\\
136. दृ॒ष्टः । मृ॒ड॒या॒ति॒ । नः॒ ॥\\
दृ॒ष्टो मृ॑डयाति मृडयाति दृ॒ष्टो दृ॒ष्टो मृ॑डयाति नो नो मृडयाति दृ॒ष्टो\\
दृ॒ष्टो मृ॑डयाति नः ।\\
\\
137. मृ॒ड॒या॒ति॒ । नः॒ ॥\\
मृ॒ड॒या॒ति॒ नो॒ नो॒ मृ॒ड॒या॒ति॒ मृ॒ड॒या॒ति॒ नः॒ ।\\
\\
138. नः॒ ॥\\
न॒ इति॑ नः ।\\
\\
139. नमः॑ । अ॒स्तु॒ । नील॑ग्रीवाय ।\\
नमो॑ अस्त्वस्तु॒ नमो॒ नमो॑ अस्तु॒ नील॑ग्रीवाय॒ नील॑ग्रीवाया स्तु॒ नमो॒ नमो॑\\
अस्तु॒ नील॑ग्रीवाय ।\\
\\
140. अ॒स्तु॒ । नील॑ग्रीवाय । स॒ह॒स्रा॒क्षाय॑ ।\\
अ॒स्तु॒ नील॑ग्रीवाय॒ नील॑ग्रीवाया स्त्व स्तु॒ नील॑ग्रीवाय सहस्रा॒क्षाय॑\\
सहस्रा॒क्षाय॒ नील॑ग्रीवाया स्त्व स्तु॒ नील॑ग्रीवाय सहस्रा॒क्षाय॑ ।\\
\\
141. नील॑ग्रीवाय । स॒ह॒स्रा॒क्षाय॑ । मी॒ढुषे᳚ ॥\\
नील॑ग्रीवाय सहस्रा॒क्षाय॑ सहस्रा॒क्षाय॒ नील॑ग्रीवाय॒ नील॑ग्रीवाय\\
सहस्रा॒क्षाय॑ मी॒ढुषे॑ मी॒ढुषे॑ सहस्रा॒क्षाय॒ नील॑ग्रीवाय॒ नील॑ग्रीवाय\\
सहस्रा॒क्षाय॑ मी॒ढुषे᳚ ।\\
\\
142. नील॑ग्रीवाय ।\\
नील॑ग्रीवा॒येति॒ नील॑ - ग्री॒वा॒य॒ ।\\
\\
143. स॒ह॒स्रा॒क्षाय॑ । मी॒ढुषे᳚ ॥\\
स॒ह॒स्रा॒क्षाय॑ मी॒ढुषे॑ मी॒ढुषे॑ सहस्रा॒क्षाय॑ सहस्रा॒क्षाय॑ मी॒ढुषे᳚ ।\\
\\
144. स॒ह॒स्रा॒क्षाय॑ ।\\
स॒ह॒स्रा॒क्षायेति॑ सहस्र - अ॒क्षाय॑ ।\\
\\
145. मी॒ढुषे᳚ ॥\\
मी॒ढुष॒ इति॑ मी॒ढुषे᳚ ।\\
\\
146. अथो᳚ । ये । अ॒स्य॒ ।\\
अथो॒ ये येऽथो॒ अथो॒ ये अ॑स्या स्य॒ येऽथो॒ अथो॒ ये अ॑स्य ।\\
\\
147. अथो᳚ ।\\
अथो॒ इत्यथो᳚ ।\\
\\
148. ये । अ॒स्य॒ । सत्वा॑नः ।\\
ये अ॑स्यास्य॒ ये ये अ॑स्य॒ सत्वा॑नः॒ सत्वा॑नो अस्य॒ ये ये अ॑स्य॒ सत्वा॑नः ।\\
\\
149. अ॒स्य॒ । सत्वा॑नः । अ॒हम् ।\\
अ॒स्य॒ सत्वा॑नः॒ सत्वा॑नो अस्यास्य॒ सत्वा॑नो॒ऽह म॒हꣳ सत्वा॑नो अस्यास्य॒\\
सत्वा॑नो॒ऽहम् ।\\
\\
150. सत्वा॑नः । अ॒हम् । तेभ्यः॑ ।\\
सत्वा॑नो॒ऽह म॒हꣳ सत्वा॑नः॒ सत्वा॑नो॒ऽहं तेभ्य॒ स्तेभ्यो॒ऽहꣳ सत्वा॑नः॒\\
सत्वा॑नो॒ऽहं तेभ्यः॑ ।\\
\\
151. अ॒हम् । तेभ्यः॑ । अ॒क॒र॒म् ।\\
अ॒हं तेभ्य॒ स्तेभ्यो॒ऽहम॒हं तेभ्यो॑ऽकर मकर॒न् तेभ्यो॒ऽहम॒हं\\
तेभ्यो॑ऽकरम् ।\\
\\
152. तेभ्यः॑ । अ॒क॒र॒म् । नमः॑ ॥\\
तेभ्यो॑ऽकर मकर॒न् तेभ्य॒ स्तेभ्यो॑ऽकर॒न् नमो॒ नमो॑ऽकर॒न् तेभ्य॒ स्तेभ्यो॑ऽकर॒न् नमः॑ ।\\
\\
153. अ॒क॒र॒म् । नमः॑ ॥\\
अ॒क॒र॒न् नमो॒ नमो॑ऽकर मकर॒न् नमः॑ ।\\
\\
154. नमः॑ ॥\\
नम॒ इति॒ नमः॑ ।\\
\\
155. प्र । मु॒ञ्च॒ । धन्व॑नः ।\\
प्र मु॑ञ्च मुञ्च॒ प्र प्र मु॑ञ्च॒ धन्व॑नो॒ धन्व॑नो मुञ्च॒ प्र प्र मु॑ञ्च॒ धन्व॑नः ।\\
\\
156. मु॒ञ्च॒ । धन्व॑नः । त्वम् ।\\
मु॒ञ्च॒ धन्व॑नो॒ धन्व॑नो मुञ्च मुञ्च॒ धन्व॑न॒ स्त्वं त्वं धन्व॑नो मुञ्च मुञ्च॒\\
धन्व॑न॒ स्त्वम् ।\\
\\
157. धन्व॑नः । त्वम् । उ॒भयोः᳚ ।\\
धन्व॑न॒ स्त्वं त्वं धन्व॑नो॒ धन्व॑न॒ स्त्व मु॒भयो॑ रु॒भयो॒ स्त्वं धन्व॑नो॒\\
धन्व॑न॒ स्त्व मु॒भयोः᳚ ।\\
\\
158. त्वम् । उ॒भयोः᳚ । आर्त्नि॑योः ।\\
त्व मु॒भयो॑ रु॒भयो॒ स्त्वं त्व मु॒भयो॒ रार्त्नि॑यो॒ रार्त्नि॑यो रु॒भयो॒ स्त्वं त्व\\
मु॒भयो॒ रार्त्नि॑योः ।\\
\\
159. उ॒भयोः᳚ । आर्त्नि॑योः । ज्याम् ॥\\
उ॒भयो॒ रार्त्नि॑यो॒ रार्त्नि॑यो रु॒भयो॑ रु॒भयो॒ रार्त्नि॑यो॒र् ज्यां ज्या मार्त्नि॑यो\\
रु॒भयो॑ रु॒भयो॒ रार्त्नि॑यो॒र् ज्याम् ।\\
\\
160. आर्त्नि॑योः । ज्याम् ॥\\
आर्त्नि॑यो॒र् ज्यां ज्या मार्त्नि॑यो॒ रार्त्नि॑याे॒र् ज्याम् ।\\
\\
161. ज्याम् ॥\\
ज्यामिति॒ ज्याम् ।\\
\\
162. याः । च॒ । ते॒ ।\\
याश्च॑ च॒ या याश्च॑ ते ते च॒ या याश्च॑ ते ।\\
\\
163. च॒ । ते॒ । हस्ते᳚ ।\\
च॒ ते॒ ते॒ च॒ च॒ ते॒ हस्ते॒ हस्ते॑ ते च च ते॒ हस्ते᳚ ।\\
\\
164. ते॒ । हस्ते᳚ । इष॑वः ।\\
ते॒ हस्ते॒ हस्ते॑ ते ते॒ हस्त॒ इष॑व॒ इष॑वो॒ हस्ते॑ ते ते॒ हस्त॒ इष॑वः ।\\
\\
165. हस्ते᳚ । इष॑वः । परा᳚ ।\\
हस्त॒ इष॑व॒ इष॑वो॒ हस्ते॒ हस्त॒ इष॑वः॒ परा॒ परेष॑वो॒ हस्ते॒ हस्त॒ इष॑वः॒ परा᳚ ।\\
\\
166. इष॑वः । परा᳚ । ताः ।\\
इष॑वः॒ परा॒ परेष॑व॒ इष॑वः॒ परा॒ ता स्ताः परेष॑व॒ इष॑वः॒ परा॒ ताः ।\\
\\
167. परा᳚ । ताः । भ॒ग॒वः॒ ।\\
परा॒ ता स्ताः परा॒ परा॒ ता भ॑गवो भगव॒ स्ताः परा॒ परा॒ ता भ॑गवः ।\\
\\
168. ताः । भ॒ग॒वः॒ । व॒प॒ ॥\\
ता भ॑गवो भगव॒ स्ता स्ता भ॑गवो वप वप भगव॒ स्ता स्ता भ॑गवो वप ।\\
\\
169. भ॒ग॒वः॒ । व॒प॒ ॥\\
भ॒ग॒वो॒ व॒प॒ व॒प॒ भ॒ग॒वो॒ भ॒ग॒वो॒ व॒प॒ ।\\
\\
170. भ॒ग॒वः॒ ।\\
भ॒ग॒व॒ इति॑ भग - वः॒ ।\\
\\
171. व॒प॒ ॥\\
व॒पेति॑ वप ।\\
\\
172. अ॒व॒तत्य॑ । धनुः॑ । त्वम् ।\\
अ॒व॒तत्य॒ धनु॒र् धनु॑ रव॒तत्या॑ व॒तत्य॒ धनु॒ स्त्वं त्वं धनु॑ रव॒तत्या॑ व॒तत्य॒\\
धनु॒ स्त्वम् ।\\
\\
173. अ॒व॒तत्य॑ ।\\
अ॒व॒तत्येत्य॑व - तत्य॑ ।\\
\\
174. धनुः॑ । त्वम् । सह॑स्राक्ष ।\\
धनु॒स्त्वं त्वं धनु॒र् धनु॒ स्त्वꣳ सह॑स्राक्ष॒ सह॑स्राक्ष॒त्वं धनु॒र् धनु॒ स्त्वꣳ\\
सह॑स्राक्ष ।\\
\\
175. त्वम् । सह॑स्राक्ष । शते॑षुधे ॥\\
त्वꣳ सह॑स्राक्ष॒ सह॑स्राक्ष॒ त्वं त्वꣳ सह॑स्राक्ष॒ शते॑षुधे॒ शते॑षुधे॒ सह॑स्राक्ष॒\\
त्वं त्वꣳ सह॑स्राक्ष॒ शते॑षुधे ।\\
\\
176. सह॑स्राक्ष । शते॑षुधे ॥\\
सह॑स्राक्ष॒ शते॑षुधे॒ शते॑षुधे॒ सह॑स्राक्ष॒ सह॑स्राक्ष॒ शते॑षुधे ।\\
\\
177. सह॑स्राक्ष ।\\
सह॑स्रा॒क्षेति॒ सह॑स्र - अ॒क्ष॒ ।\\
\\
178. शते॑षुधे ॥\\
शते॑षुध॒ इति॒ शत॑ - इ॒षु॒धे॒ ।\\
\\
179. नि॒शीर्य॑ । श॒ल्याना᳚म् । मुखा᳚ ।\\
नि॒शीर्य॑ श॒ल्यानाꣳ॑ श॒ल्यानां᳚ नि॒शीर्य॑ नि॒शीर्य॑ श॒ल्यानां॒ मुखा॒ मुखा॑\\
श॒ल्यानां᳚ नि॒शीर्य॑ नि॒शीर्य॑ श॒ल्यानां॒ मुखा᳚ ।\\
\\
180. नि॒शिर्य॑ ।\\
नि॒शीर्येति॑ नि - शीर्य॑ ।\\
\\
181. श॒ल्याना᳚म् । मुखा᳚ । शि॒वः ।\\
श॒ल्यानां॒ मुखा॒ मुखा॑ श॒ल्यानाꣳ॑ श॒ल्यानां॒ मुखा॑ शि॒वः शि॒वो मुखा॑\\
श॒ल्यानाꣳ॑ श॒ल्यानां॒ मुखा॑ शि॒वः ।\\
\\
182. मुखा᳚ । शि॒वः । नः॒ ।\\
मुखा॑ शि॒वः शि॒वो मुखा॒ मुखा॑ शि॒वो नो॑ नः शि॒वो मुखा॒ मुखा॑\\
शि॒वो नः॑ ।\\
\\
183. शि॒वः । नः॒ । सु॒मनाः᳚ ।\\
शि॒वो नो॑ नः शि॒वः शि॒वो नः॑ सु॒मनाः᳚ सु॒मना॑ नः शि॒वः शि॒वो नः॑\\
सु॒मनाः᳚ ।\\
\\
184. नः॒ । सु॒मनाः᳚ । भ॒व॒ ॥\\
नः॒ सु॒मनाः᳚ सु॒मना॑ नो नः सु॒मना॑ भव भव सु॒मना॑ नो नः सु॒मना॑ भव ।\\
\\
185. सु॒मनाः᳚ । भ॒व॒ ॥\\
सु॒मना॑ भव भव सु॒मनाः᳚ सु॒मना॑ भव ।\\
\\
186. सु॒मनाः᳚ ।\\
सु॒मना॒ इति॑ सु - मनाः᳚ ।\\
\\
187. भ॒व॒ ॥\\
भ॒वेति॑ भव ।\\
\\
188. विज्य᳚म् । धनुः॑ । क॒प॒र्दिनः॑ ।\\
विज्यं॒ धनु॒र् धनु॒र् विज्यं॒ँविज्यं॒ धनुः॑ कप॒र्दिनः॑ कप॒र्दिनो॒ धनु॒र् विज्यं॒ँविज्यं॒ धनुः॑ कप॒र्दिनः॑ ।\\
\\
189. विज्य᳚म् ।\\
विज्य॒मिति॒ वि - ज्य॒म्॒ ।\\
\\
190. धनुः॑ । क॒प॒र्दिनः॑ । विश॑ल्यः ।\\
धनुः॑ कप॒र्दिनः॑ कप॒र्दिनो॒ धनु॒र् धनुः॑ कप॒र्दिनो॒ विश॑ल्यो॒ विश॑ल्यः\\
कप॒र्दिनो॒ धनु॒र् धनुः॑ कप॒र्दिनो॒ विश॑ल्यः ।\\
\\
191. क॒प॒र्दिनः॑ । विश॑ल्यः । बाण॑वान् ।\\
क॒प॒र्दिनो॒ विश॑ल्यो॒ विश॑ल्यः कप॒र्दिनः॑ कप॒र्दिनो॒ विश॑ल्यो॒ बाण॑वा॒न्\\
बाण॑वा॒न्॒. विश॑ल्यः कप॒र्दिनः॑ कप॒र्दिनो॒ विश॑ल्यो॒ बाण॑वान् ।\\
\\
192. विश॑ल्यः । बाण॑वान् । उ॒त ॥\\
विश॑ल्यो॒ बाण॑वा॒न् बाण॑वा॒न्॒. विश॑ल्यो॒ विश॑ल्यो॒ बाण॑वाꣳ उ॒तोत\\
बाण॑वा॒न्॒. विश॑ल्यो॒ विश॑ल्यो॒ बाण॑वाꣳ उ॒त ।\\
\\
193. विश॑ल्यः ।\\
विश॑ल्य॒ इति॒ वि - श॒ल्यः॒ ।\\
\\
194. बाण॑वान् । उ॒त ॥\\
बाण॑वाꣳ उ॒तोत बाण॑वा॒न् बाण॑वाꣳ उ॒त ।\\
\\
195. बाण॑वान् ।\\
बाण॑वा॒निति॒ बाण॑ - वा॒न् ।\\
\\
196. उ॒त ॥\\
उ॒तेत्यु॒त ।\\
\\
197. अने॑शन्न् । अ॒स्य॒ । इष॑वः ।\\
अने॑शन् नस्या॒ स्या ने॑श॒न् नने॑शन् न॒स्येष॑व॒ इष॑वो अ॒स्या ने॑श॒न् नने॑शन्\\
न॒स्येष॑वः ।\\
\\
198. अ॒स्य॒ । इष॑वः । आ॒भुः ।\\
अ॒स्येष॑व॒ इष॑वो अस्या॒ स्येष॑व आ॒भु रा॒भु रिष॑वो अस्या॒ स्येष॑व आ॒भुः ।\\
\\
199. इष॑वः । आ॒भुः । अ॒स्य॒ ।\\
इष॑व आ॒भु रा॒भु रिष॑व॒ इष॑व आ॒भु र॑स्या स्या॒भु रिष॑व॒ इष॑व आ॒भु र॑स्य ।\\
\\
200. आ॒भुः । अ॒स्य॒ । नि॒ष॒ङ्गथिः॑ ॥\\
आ॒भु र॑स्यास्या॒भु रा॒भु र॑स्य निष॒ङ्गथि॑र् निष॒ङ्गथि॑ रस्या॒भु रा॒भु र॑स्य\\
निष॒ङ्गथिः॑ ।\\
\\
201. अ॒स्य॒ । नि॒ष॒ङ्गथिः॑ ॥\\
अ॒स्य॒ नि॒ष॒ङ्गथि॑र् निष॒ङ्गथि॑ रस्यास्य निष॒ङ्गथिः॑ ।\\
\\
202. नि॒ष॒ङ्गथिः॑ ॥\\
नि॒ष॒ङ्गथि॒ रिति॑ निष॒ङ्गथिः॑ ।\\
\\
203. या । ते॒ । हे॒तिः ।\\
या ते॑ ते॒ या या ते॑ हे॒तिर्. हे॒ति स्ते॒ या या ते॑ हे॒तिः ।\\
\\
204. ते॒ । हे॒तिः । मी॒ढु॒ष्ट॒म॒ ।\\
ते॒ हे॒तिर्. हे॒ति स्ते॑ ते हे॒तिर् मी॑ढुष्टम मीढुष्टम हे॒ति स्ते॑ ते हे॒तिर्\\
मी॑ढुष्टम ।\\
\\
205. हे॒तिः । मी॒ढु॒ष्ट॒म॒ । हस्ते᳚ ।\\
हे॒तिर् मी॑ढुष्टम मीढुष्टम हे॒तिर्. हे॒तिर् मी॑ढुष्टम॒ हस्ते॒ हस्ते॑ मीढुष्टम\\
हे॒तिर्. हे॒तिर् मी॑ढुष्टम॒ हस्ते᳚ ।\\
\\
206. मी॒ढु॒ष्ट॒म॒ । हस्ते᳚ । ब॒भूव॑ ।\\
मी॒ढु॒ष्ट॒म॒ हस्ते॒ हस्ते॑ मीढुष्टम मीढुष्टम॒ हस्ते॑ ब॒भूव॑ ब॒भूव॒ हस्ते॑ मीढुष्टम\\
मीढुष्टम॒ हस्ते॑ ब॒भूव॑ ।\\
\\
207. मी॒ढु॒ष्ट॒म॒ ।\\
मी॒ढु॒ष्ट॒मेति॑ मीढुः - त॒म॒ ।\\
\\
208. हस्ते᳚ । ब॒भूव॑ । ते॒ ।\\
हस्ते॑ ब॒भूव॑ ब॒भूव॒ हस्ते॒ हस्ते॑ ब॒भूव॑ ते ते ब॒भूव॒ हस्ते॒ हस्ते॑ ब॒भूव॑ ते ।\\
\\
209. ब॒भूव॑ । ते॒ । धनुः॑ ॥\\
ब॒भूव॑ ते ते ब॒भूव॑ ब॒भूव॑ ते॒ धनु॒र् धनु॑ स्ते ब॒भूव॑ ब॒भूव॑ ते॒ धनुः॑ ।\\
\\
210. ते॒ । धनुः॑ ॥\\
ते॒ धनु॒र् धनु॑ स्ते ते॒ धनुः॑ ।\\
\\
211. धनुः॑ ॥\\
धनु॒ रिति॒ धनुः॑ ।\\
\\
212. तया᳚ । अ॒स्मान् । वि॒श्वतः॑ ।\\
तया॒ऽस्मा न॒स्मान् तया॒ तया॒ऽस्मान्. वि॒श्वतो॑ वि॒श्वतो॑ अ॒स्मान् तया॒\\
तया॒ऽस्मान्. वि॒श्वतः॑ ।\\
\\
213. अ॒स्मान् । वि॒श्वतः॑ । त्वम् ।\\
अ॒स्मान्. वि॒श्वतो॑ वि॒श्वतो॑ अ॒स्मा न॒स्मान्. वि॒श्वत॒ स्त्वं त्वंँवि॒श्वतो॑\\
अ॒स्मा न॒स्मान्. वि॒श्वत॒ स्त्वम् ।\\
\\
214. वि॒श्वतः॑ । त्वम् । अ॒य॒क्ष्मया᳚ ।\\
वि॒श्वत॒ स्त्वं त्वंँवि॒श्वतो॑ वि॒श्वत॒ स्त्व म॑य॒क्ष्मया॑ऽय॒क्ष्मया॒ त्वंँवि॒श्वतो॑ वि॒श्वत॒ स्त्व म॑य॒क्ष्मया᳚ ।\\
\\
215. त्वम् । अ॒य॒क्ष्मया᳚ । परि॑ ।\\
त्व म॑य॒क्ष्मया॑ऽय॒क्ष्मया॒ त्वं त्व म॑य॒क्ष्मया॒ परि॒ पर्य॑ य॒क्ष्मया॒ त्वं त्व\\
म॑य॒क्ष्मया॒ परि॑ ।\\
\\
216. अ॒य॒क्ष्मया᳚ । परि॑ । भु॒ज॒ ॥\\
अ॒य॒क्ष्मया॒ परि॒ पर्य॑ य॒क्ष्मया॑ऽय॒क्ष्मया॒ परि॑ब्भुज भुज॒ पर्य॑ य॒क्ष्मया॑ऽय॒क्ष्मया॒ परि॑ब्भुज ।\\
\\
217. परि॑ । भु॒ज॒ ॥\\
परि॑ब्भुज भुज॒ परि॒ परि॑ब्भुज ।\\
\\
218. भु॒ज॒ ॥\\
भु॒जेति॑ भुज ।\\
\\
219. नमः॑ । ते॒ । अ॒स्तु॒ ।\\
नम॑स्ते ते॒ नमो॒ नम॑स्ते अस्त्वस्तु ते॒ नमो॒ नम॑स्ते अस्तु ।\\
\\
220. ते॒ । अ॒स्तु॒ । आयु॑धाय ।\\
ते॒ अ॒स्त्व॒स्तु॒ ते॒ ते॒ अ॒स्त्वा यु॑धा॒या यु॑धायास्तु ते ते अ॒स्त्वायु॑धाय ।\\
\\
221. अ॒स्तु॒ । आयु॑धाय । अना॑तताय ।\\
अ॒स्त्वा यु॑धा॒या यु॑धाया स्त्व॒ स्त्वा यु॑धा॒या ना॑तता॒या ना॑तता॒या यु॑धाया\\
स्त्व॒ स्त्वा यु॑धा॒या ना॑तताय ।\\
\\
222. आयु॑धाय । अना॑तताय । धृ॒ष्णवे᳚ ॥\\
आयु॑धा॒या ना॑तता॒या ना॑तता॒या यु॑धा॒या यु॑धा॒या ना॑तताय धृ॒ष्णवे॑ धृ॒ष्णवेऽना॑तता॒या यु॑धा॒या यु॑धा॒या ना॑तताय धृ॒ष्णवे᳚ ।\\
\\
223. अना॑तताय । धृ॒ष्णवे᳚ ॥\\
अना॑तताय धृ॒ष्णवे॑ धृ॒ष्णवेऽना॑तता॒या ना॑तताय धृ॒ष्णवे᳚ ।\\
\\
224. अना॑तताय ।\\
अना॑तता॒येत्यना᳚ - त॒ता॒य॒ ।\\
\\
225. धृ॒ष्णवे᳚ ॥\\
धृ॒ष्णव॒ इति॑ धृ॒ष्णवे᳚ ।\\
\\
226. उ॒भाभ्या᳚म् । उ॒त । ते॒ ।\\
उ॒भाभ्या॑ मु॒तोतो भाभ्या॑ मु॒भाभ्या॑ मु॒तते॑ त उ॒तो भाभ्या॑ मु॒भाभ्या॑ मु॒तते᳚ ।\\
\\
227. उ॒त । ते॒ । नमः॑ ।\\
उ॒त ते॑त उ॒तो तते॒ नमो॒ नम॑स्त उ॒तो तते॒ नमः॑ ।\\
\\
228. ते॒ । नमः॑ । बा॒हुभ्या᳚म् ।\\
ते॒ नमो॒ नम॑स्ते ते॒ नमो॑ बा॒हुभ्यां᳚ बा॒हुभ्यां॒ नम॑स्ते ते॒ नमो॑ बा॒हुभ्या᳚म् ।\\
\\
229. नमः॑ । बा॒हुभ्या᳚म् । तव॑ ।\\
नमो॑ बा॒हुभ्यां᳚ बा॒हुभ्यां॒ नमो॒ नमो॑ बा॒हुभ्यां॒ तव॒ तव॑ बा॒हुभ्यां॒ नमो॒ नमो॑\\
बा॒हुभ्यां॒ तव॑ ।\\
\\
230. बा॒हुभ्या᳚म् । तव॑ । धन्व॑ने ॥\\
बा॒हुभ्यां॒ तव॒ तव॑ बा॒हुभ्यां᳚ बा॒हुभ्यां॒ तव॒ धन्व॑ने॒ धन्व॑ने॒ तव॑ बा॒हुभ्यां᳚\\
बा॒हुभ्यां॒ तव॒ धन्व॑ने ।\\
\\
231. बा॒हुभ्या᳚म् ।\\
बा॒हुभ्या॒मिति॑ बा॒हु - भ्या॒म् ।\\
\\
232. तव॑ । धन्व॑ने ॥\\
तव॒ धन्व॑ने॒ धन्व॑ने॒ तव॒ तव॒ धन्व॑ने ।\\
\\
233. धन्व॑ने ॥\\
धन्व॑न॒ इति॒ धन्व॑ने ।\\
\\
234. परि॑ । ते॒ । धन्व॑नः ।\\
परि॑ ते ते॒ परि॒ परि॑ ते॒ धन्व॑नो॒ धन्व॑न स्ते॒ परि॒ परि॑ ते॒ धन्व॑नः ।\\
\\
235. ते॒ । धन्व॑नः । हे॒तिः ।\\
ते॒ धन्व॑नो॒ धन्व॑न स्ते ते॒ धन्व॑नो हे॒तिर्. हे॒तिर् धन्व॑न स्ते ते॒\\
धन्व॑नो हे॒तिः ।\\
\\
236. धन्व॑नः । हे॒तिः । अ॒स्मान् ।\\
धन्व॑नो हे॒तिर्. हे॒तिर् धन्व॑नो॒ धन्व॑नो हे॒ति र॒स्मा न॒स्मान्. हे॒तिर् धन्व॑नो॒\\
धन्व॑नो हे॒ति र॒स्मान् ।\\
\\
237. हे॒तिः । अ॒स्मान् । वृ॒ण॒क्तु॒ ।\\
हे॒ति र॒स्मा न॒स्मान्. हे॒तिर्. हे॒ति र॒स्मान्. वृ॑णक्तु वृणक् त्व॒स्मान्.\\
हे॒तिर्. हे॒ति र॒स्मान्. वृ॑णक्तु ।\\
\\
238. अ॒स्मान् । वृ॒ण॒क्तु॒ । वि॒श्वतः॑ ॥\\
अ॒स्मान्. वृ॑णक्तु वृण क्त्व॒स्मा न॒स्मान्. वृ॑णक्तु वि॒श्वतो॑ वि॒श्वतो॑\\
वृण क्त्व॒स्मा न॒स्मान्. वृ॑णक्तु वि॒श्वतः॑ ।\\
\\
239. वृ॒ण॒क्तु॒ । वि॒श्वतः॑ ॥\\
वृ॒ण॒क्तु॒ वि॒श्वतो॑ वि॒श्वतो॑ वृणक्तु वृणक्तु वि॒श्वतः॑ ।\\
\\
240. वि॒श्वतः॑ ॥\\
वि॒श्वत॒ इति॑ वि॒श्वतः॑ ।\\
\\
241. अथो॒ । यः । इ॒षु॒धि ।\\
अथो॒ यो योऽथो॒ अथो॒ य इ॑षु॒धि रि॑षु॒धिर् योऽथो॒ अथो॒ य इ॑षु॒धिः ।\\
\\
242. अथो᳚ ।\\
अथो॒ इत्यथो᳚ ।\\
\\
243. यः । इ॒षु॒धिः । तव॑ ।\\
य इ॑षु॒धि रि॑षु॒धिर् यो य इ॑षु॒धि स्तव॒ तवे॑षु॒धिर् यो य इ॑षु॒धि स्तव॑ ।\\
\\
244. इ॒षु॒धिः । तव॑ । आ॒रे ।\\
इ॒षु॒धि स्तव॒ तवे॑षु॒धि रि॑षु॒धि स्तवा॒र आ॒रे तवे॑षु॒धि रि॑षु॒धि स्तवा॒रे ।\\
\\
245. इ॒षु॒धिः ।\\
इ॒षु॒धिरिती॑षु - धिः ।\\
\\
246. तव॑ । आ॒रे । अ॒स्मत् ।\\
तवा॒र आ॒रे तव॒ तवा॒रे अ॒स्म द॒स्म दा॒रे तव॒ तवा॒रे अ॒स्मत् ।\\
\\
247. आ॒रे । अ॒स्मत् । नि ।\\
आ॒रे अ॒स्म द॒स्म दा॒र आ॒रे अ॒स्मन् नि न्य॑स्म दा॒र आ॒रे अ॒स्मन् नि ।\\
\\
248. अ॒स्मत् । नि । धे॒हि॒ ।\\
अ॒स्मन् नि न्य॑स्म द॒स्मन् नि धे॑हि धेहि॒ न्य॑स्म द॒स्मन् नि धे॑हि ।\\
\\
249. नि । धे॒हि॒ । तम् ॥\\
नि धे॑हि धेहि॒ नि नि धे॑हि॒ तं तं धे॑हि॒ नि नि धे॑हि॒ तम् ।\\
\\
250. धे॒हि॒ । तम् ॥\\
धे॒हि॒ तं तं धे॑हि धेहि॒ तम् ।\\
\\
251. तम् ॥\\
तमिति॒ तं ।\\
\subsection{\eng{Anuvaka 2}}
\subsection{\eng{Anuvaka 3}}
\subsection{\eng{Anuvaka 4}}
\subsection{\eng{Anuvaka 5}}
\subsection{\eng{Anuvaka 6}}
\subsection{\eng{Anuvaka 7}}
\subsection{\eng{Anuvaka 8}}
\subsection{\eng{Anuvaka 9}}
\subsection{\eng{Anuvaka 10}}
\subsection{\eng{Anuvaka 11}}
\subsection{\eng{Triyambakam}}
