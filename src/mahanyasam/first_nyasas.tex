\section{\eng{Mahanyasam}}
\subsection{\eng{Panchanga Rudra Nyasaha}}
ओं नमो भगवते॑ रुद्रा॒य ॥ अथातः पंचाग रुद्राणां\\
न्यास पूर्वकं जपहोमा र्च नाभिषेक विधिं वया᳚ख्या॒स्यामः ॥\\
\\
ओं  ओंकार मन्त्र संयुक्तं नित्यं ध्यायन्ति योगिनः।\\
कामदं मोक्षदं तस्मै ओं नकाराय नमोनमः॥\\
\\
ओं भूर्भुवस्सुवः॑॥ ओं नं नम॑स्ते रुद्र म॒न्यव॑ उ॒तोत॒ इष॑वे॒ नम॑: ।\\
नम॑स्ते अस्तु॒ धन्व॑ने बा॒हुभ्या॑मु॒त ते॒ नम॑: ।\\
{\small\eng{(alternate)} या त॒ इषुः॑ शि॒वत॑मा शि॒वम् ब॒भूव॑ ते॒ धनुः॑।\\
शि॒वा श॑र॒व्या॑ या तव॒ तया॑ नो रुद्र मृडय॥}\\
{\small\eng{(East)}}\\
ओं कं खं गं घं डं । {\small यरलव शष सहों}\\
ओं नमो भगवते॑ रुद्रा॒य ॥ नं ओं - पूर्वाङ्ग रुद्राय नमः॥ 1 ॥\\
\\
महादेवं महात्मानं महा पातक नाशनम्।\\
महा पाप हरं वन्दे मकाराय नमोनमः॥\\
\\
ओं भूर्भुवस्सुवः॑॥ ओं मं \\
निध॑नपतये॒ नमः । निध॑नपतान्तिकाय॒ नमः ।\\
ऊर्ध्वाय॒ नमः । ऊर्ध्वलिङ्गाय॒ नमः ।\\
हिरण्याय॒ नमः । हिरण्यलिङ्गाय॒ नमः ।\\
सुवर्णाय॒ नमः । सुवर्णलिङ्गाय॒ नमः ।\\
दिव्याय॒ नमः । दिव्यलिङ्गाय॒ नमः ।\\
भवाय॒ नमः । भवलिङ्गाय॒ नमः ।\\
शर्वाय॒ नमः । शर्वलिङ्गाय॒ नमः ।\\
शिवाय॒ नमः । शिवलिङ्गाय॒ नमः ।\\
ज्वलाय॒ नमः । ज्वललिङ्गाय॒ नमः ।\\
आत्माय॒ नमः । आत्मलिङ्गाय॒ नमः ।\\
परमाय॒ नमः । परमलिङ्गाय॒ नमः ।\\
एतत्  सोमस्य॑ सूर् यस्य॒ सर्व\\
लिङ्ग॑ꣳस्था प॒य॒ति॒ पाणि मन्त्रं॑ पवि॒त्रम् ॥\\
\\
{\small\eng{(South)}}\\
ओं छं जं झं ञं नं । {\small यरलव शष सहों}\\
ओं नमो भगवते॑ रुद्रा॒य ॥ मं ओं - दक्षिणाङ्ग रुद्राय नमः॥ 2 ॥\\
\\
शिवं शान्तं जगन्नाधं लोकानुग्रह कारणम्।\\
शिवमेकं परं वन्दे शिकाराय नमोनमः॥\\
\\
ओं भूर्भुवस्सुवः॑॥ ओं शिं \\
अपै॑तु मृ॒त्यु र॒मृतं॑ न॒ आग॑न् वैवस् व॒तो नो॒ अभ॒यं कृणोतु।\\
प॒र्णं वन॒स्पते॑ रिवा॒भिन॑श् शीयताग्ं र॒यिस् सच॑तां न॒श्श ची पतिः॑॥\\
\\
{\small\eng{(West)}}\\
ओं  टं ठं डं ढं णं । {\small यरलव शष सहों}\\
ओं नमो भगवते॑ रुद्रा॒य ॥ शिं ओं - पश्चिमाङ्ग रुद्राय नमः॥ 3 ॥ \\
\\
वाहनं वृषभो यस्य वासुकिः कण्ठ भूषणम् ।\\
वामे शक्तिधरं वन्दे वकाराय नमोनमः॥\\
\\
ओं भूर्भुवस्सुवः॑॥ ओं वां \\
प्राणानां ग्रन्थिरसि रुद्रो मा॑ विशा॒न्तकः ।\\
तेनान्नेना᳚प्याय॒स्व ।\\
\\
{\small\eng{(alternate)} यो रु॒द्रो अ॒ग्नौ यो अ॒प्सु य ओष॑धीषु॒ यो
रु॒द्रो विश्वा॒ भुव॑नाऽऽवि॒वेश॒ तस्मै॑ रु॒द्राय॒ नमो॑ अस्तु॥\\
}
\\
{\small\eng{(North)}}\\
ओं  तं थं दं घं नं । {\small यरलव शष सहों}\\
ओं नमो भगवते॑ रुद्रा॒य ॥ वां औं - उत्तराङ्ग रुद्राय नमः॥ 4॥\\
\\
यत्र कुत्र स्थितं देवं सर्व व्यापिन मीश्वरम्।\\
यल्लिङ्गं पूज येन्नित्यं यकाराय नमोनमः॥।\\
\\
ओं भूर्भुवस्सुवः॑॥ ओं यं\\
यो रु॒द्रो अ॒ग्नौ यो अ॒प्सु य ओष॑धीषु॒ यो रु॒द्रो\\
विश्वा॒ भुव॑ना वि॒वेश॒ तस्मै॑ रु॒द्राय॒ नमो॑ अस्तु ।\\
\\
{\small\eng{(Heavenwards)}}\\
ओं पं फं बं भं मं । {\small यरलव शष सहों}\\
ओं नमो भगवते॑ रुद्रा॒य ॥ यं ओं - ऊर्ध्वाङ्ग रुद्राय नमः ॥ 5 ॥\\
\subsection{\eng{Panchamuka Nyasam}}
तत्पुरु॑षाय वि॒द्महे॑ महादे॒वाय॑ धीमहि । \\
तन्नो॑ रुद्रः प्रचो॒दया᳚त् ॥\\
\\
संवर्ताग्नि तटित्प्रदीप्त कनक प्रस्पर्दि तेजोमयं\\
गम्भीरध्वनि  मिश्रितोग्र दहन प्रोद् भा सिता म्राधरम्\\
अर्धेन् दुद् युति लोल पिङ्गळ जटाभार प्रबद्धोरगं\\
वन्दे सिद्ध सुरासुरेन्द्र नमितं पूर्वं मुखंशूलिनः \\
{\small
संवर्ताग्नि तटित्प्रदीप्त कनक प्रस्पर्धि तेजोरुणं\\
गम्भीरध्वनि \textbf{सामवेद जनकं ताम राधरं सुन्दरम्}।\\
अर्धेनदुद्युति लोल पिंगल जटा भार \textbf{प्रबोद्धोदकं} \\
वन्दे सिद्ध सुरा सुरेन्द्र नमितं पूर्वं मुखं शूलिनः॥\\
}
\\
ओं अं कं खं गं घं डं । आं ओं ओं नमो भगवते॑ रुद्रा॒य ॥\\
ओं नं - पूर्व मुखाय नमः॥\\
अ॒घोरे᳚भ्योऽथ॒ घोरे᳚भ्यो॒ घोर॒घोर॑तरेभ्यः । \\
स॒र्वे᳚तः॑ सर्व॒ शर्वे᳚भ्यो॒ नम॑स्ते अस्तु रु॒द्र रू॑पेभ्यः ॥\\
\\
कालाभ्र भ्रमराञ्ज नद्युतिनिभं व्यावृत्त पिङ्गेक्षणं\\
कर्णोद्भासित भोगि मस्तक मणिं प्रोद्भिन्न दंष्ट्राङ्कुरम् ।\\
सर्प प्रोतक पाल शुक्ति शकल व्याकीर्ण सञ्चारगं\\
वन्दे दक्षिण मीश्व रस्य कुटिल भ्रूभङ्गरौद्रं मुखम् \\
{\small
सर्पप् रोतक पाल शुक्ति शकल व्याकीर्ण \textbf{ताशेखरं}\\
वन्दे दक्षिण मीश्व रस्य \textbf{दक्षिण वदनं चाथर्वनादोदयम्}॥\\ 
}
\\
ओं इं छं जं झं ञं नं । ईं ओं ओं नमो भगवते॑ रुद्रा॒य ॥\\
ओं मं  - दक्षिण मुखाय नमः॥\\
\\
स॒द्योजा॒तं प्र॑पद्या॒मि॒ स॒द्योजा॒ताय॒ वै नमो॒ नमः॑ ।\\
भ॒वे भ॑वे॒ नाति॑भवे भवस्व॒ माम् । भ॒वोद्भ॑वाय॒ नमः ॥\\
\\
प्रालेयाचल चन्द्र कुन्द धवलं गोक्षीर फेन प्रभं\\
भस्माभ् यक्त मनङ्ग देह दहन ज्वाला वली लोचनम् ।\\
ब्रह्मेन्द्रादि मरुद्गणैः स्तुतिपरै रभ्यर्चितं योगि भिः\\
वन्देऽहं सकलं कलङ्क रहितं स्थाणोर्मुखं पश्चिमम् ॥\\
{\small
प्रालेयाचल \textbf{मिन्दुकुन्द} धवलं गोक्षीर फेनप्रभं\\
\textbf{भस्माभ् यंग} मनंग देह दहन ज्वाला वलीलोचनम्\\
\textbf{विष्णु ब्रह्म मरुद् गणार्चित पदं ऋग्वेद नादोदयं}\\
}
\\
ओं उं टं ठं डं ढं णं । ऊं ओं ओं नमो भगवते॑ रुद्रा॒य ॥\\
ओं  शिं - पश्चिम मुखाय नमः॥\\
\\
वा॒म॒दे॒वाय॒ नमो᳚ ज्ये॒ष्ठाय॒ नमः॑ श्रे॒ष्ठाय॒ नमो॑ रु॒द्राय॒ नमः॒ \\
काला॑य नमः॒ कल॑ विकरणाय॒ नमो॒ \\
बल॑ विकरणाय॒ नमो॒\\
बला॑य॒ नमो॒ बल॑प्रमथनाय॒ नमः॒ \\
सर्व॑ भूत दमनाय॒ नमो॑ म॒नोन्म॑नाय॒ नमः॒ ॥\\
\\
गौरं कुङ्कुम पङ्किलं सुतिलकं व्यापाण्डु गण्डस्थलं\\
भ्रूविक्षेप कटाक्ष वीक्षण लसत् संसक्त कर्णोत्पलम् ।\\
स्निग्धं बिम्ब फलाधर प्रहसितं नीलाल कालङ्कृतं\\
वन्दे पूर्ण शशाङ्क मण्डल निभं वक्त्रं हरस्योत्तरम् ॥\\
{\small
वन्दे \textbf{याजुष वेद घोष जनकं} वक्त्रं हरस्योत्तरम्॥\\
}
\\
ओं एं तं थं दं घं नं । ऐं ओं ओं नमो भगवते॑ रुद्रा॒य ॥\\
ओं वां - उत्तर मुखाय नमः॥\\
\\
ईशानः सर्व॑ विद्या॒ना॒ मीश्वरः सर्व॑भूता॒नां॒\\
ब्रह्माधि॑पति॒र्ब्रह्म॒णोऽधि॑पति॒ \\
र्ब्रह्मा॑ शि॒वो मे॑ अस्तु सदाशि॒वोम् ॥\\
\\
व्यक्ता व्यक्त गुणेतरं सुविमलं षट्त्रिं शतत् त्वात्मकं\\
तस्मा दुत्तर तत्त्व मक्षरमिति ध्येयं सदा योगिभिः ।\\
वन्दे तामस वर्जितं त्रिणयनं सूक्ष्मा तिसूक्ष्मात्परं\\
शान्तं पञ्चम मीश्वरस्य वदनं खव्यापि तेजोमयम् ॥\\
{\small
व्यक्ताव्यक्त \textbf{निरूपितं च परमं} षट्त्रिं \textbf{शतत्वाधिकं}\\
तस्मादुत्तर तत्व मक्षरमिति ध्येयं सदा योगिभिः।\\
\textbf{ओंकारादि समस्त मन्त्र जनकं} सूक्ष्मा तिसूक्ष्मं परं\\
\textbf{वन्दे} पंचम मीश्वरस्य वदनं खयापि तेजोमयम्॥\\
}
\\
ओं ओं पं फं बं भं मं । औं ओं ओं नमो भगवते॑ रुद्रा॒य ॥\\
ओं यं - ऊर्ध्व मुखाय नमः॥\\
\\
पूर्वे पशुपतिः पातु दक्षिणे पातु शंकरः।\\
पश्चिमे पातु विश्वेशो नीलकण्ठस्तथोत्तरे॥\\
\\
ऐशान्यां पातुमां शर्वो ह्याग् नेय्यां पार्वती पतिः।\\
नैर्ऋर्त्यां पातुमां रुद्रो वायव्यां नीललोहितः॥\\
ऊर्ध्वे त्रिलोचनः पात् अधरायां महेश्वरः।\\
एताभ्यो दश दिग् भ्यस्तु सर्वतः पातु शंकरः॥\\
\subsection{\eng{Keshadhi Padhanta nyasaha}}
ओं या ते॑ रुद्र शि॒वा त॒नूरघो॒राऽपा॑पकाशिनी ।\\
तया॑ नस्त॒नुवा॒ शन्त॑मया॒ गिरि॑शन्ता॒भिचा॑कशीहि ॥ शिखायै नमः॥ (1)\\
{\small\eng{Tuft}}\\
\\
अ॒स्मिन्म॑ह॒त्य॑र्ण॒वे᳚ऽन्तरि॑क्षे भ॒वा अधि॑ ।\\
तेषाग्ं॑ सहस्रयोज॒नेऽव॒धन्वा॑नि तन्मसि ॥ शिरसे नमः॥ (2)\\
{\small\eng{Top o\eng{f} Head}}\\
\\
स॒हस्रा॑णि सहस्र॒शो ये रु॒द्रा अधि॒ भूम्या᳚म् ।\\
तेषाग्ं॑ सहस्रयोज॒नेऽव॒धन्वा॑नि तन्मसि ॥ ललाटाय नमः॥ (3)\\
{\small\eng{Forehead}}\\
\\
ह॒ग्ं॒ सश् शुचि॒ षद् वसुरन् तरिक्ष॒सद् धोता वेदि॒ष दति थिर् दरो ण॒सत्।\\
नृ॒षद् वर॒सद्रु तसब् यो मसदब् जा गोजा ऋतजा अद्रिजा ऋतं बृहत्॥ \\
भ्रुवोर्मध्याय नमः॥ (4)\\
{\small\eng{Middle o\eng{f} Eyebrows}}\\
\\
त्र्य॑म्बकं यजामहे सुग॒न्धिं पु॑ष्टि॒वर्ध॑नम् ।\\
उ॒र्वा॒रु॒कमि॑व॒ बन्ध॑नान्मृ॒त्योर्मु॑क्षीय॒ माऽमृता᳚त् ॥  नेत्राभ्यां नमः॥ (5)\\
{\small\eng{Eyes}}\\
\\
नम॒: स्रुत्या॑य च॒ पथ्या॑य च॒    नम॑: का॒ट्या॑य च नी॒प्या॑य च॒ ॥ \\
कर्णाभ्यां नमः॥ (6)\\
{\small\eng{Ears}}\\
\\
मा न॑स्तो॒के तन॑ये॒ मा न॒ आयु॑षि॒ मा नो॒ गोषु॒ मा नो॒ अश्वे॑षु रीरिषः ।\\
वी॒रान्मा नो॑ रुद्र भामि॒तोऽव॑धीर्ह॒विष्म॑न्तो॒  नम॑सा विधेम ते ॥  \\
नासिकायै नमः॥ (7)\\
{\small\eng{Nose}}\\
\\
अ॒व॒तत्य॒ धनु॒स्तवग्ं सह॑स्राक्ष॒ शते॑षुधे ।\\
नि॒शीर्य॑ श॒ल्यानां॒ मुखा॑ शि॒वो न॑: सु॒मना॑ भव ॥ मुखाय नमः॥ (8)\\
{\small\eng{Face}}\\
\\
नील॑ग्रीवाः शिति॒कण्ठा᳚: श॒र्वा अ॒धः क्ष॑माच॒राः ।\\
तेषाग्ं॑ सहस्रयोज॒नेऽव॒धन्वा॑नि तन्मसि ॥ कण्ठाय नमः॥ (9)\\
{\small\eng{Throat}}\\
\\
नील॑ग्रीवाः शिति॒कण्ठा॒ दिवग्ं॑ रु॒द्रा उप॑श्रिताः ।\\
तेषाग्ं॑ सहस्रयोज॒नेऽव॒धन्वा॑नि तन्मसि ॥ उपकण्ठाय नमः। (10)\\
{\small\eng{Lower Neck}}\\
\\
नम॑स्ते अ॒स्त्वायु॑धा॒याना॑तताय धृ॒ष्णवे᳚ ।\\
उ॒भाभ्या॑मु॒त ते॒ नमो॑ बा॒हुभ्यां॒ तव॒ धन्व॑ने ॥ बाहुभ्यां नमः। (11)\\
{\small\eng{Shoulders}}\\
\\
या ते॑ हे॒तिर्मी॑ढुष्टम॒ हस्ते॑ ब॒भूव॑ ते॒ धनु॑: ।\\
तया॒ऽस्मान् वि॒श्वत॒स्त्वम॑य॒क्ष्मया॒ परि॑ब्भुज ॥ उपबाहुभ्यां नमः॥ (12)\\
{\small\eng{Elbow to Wrist}}\\
\\
{\small\eng{Not in challakere rendition}}\\
परि॑णो रु॒द्रस्य॑ हे॒तिर्वृ॑णक्तु॒ परि॑ त्वे॒षस्य॑ दुर्म॒तिर॑घा॒योः।\\
अव॑ स्थि॒रा म॒घव॑द्भ्यस्तनुष्व॒ मीढ्व॑स्तो॒काय॒ तन॑याय मृडय॥\\
मणिबन्धाभ्यां नमः॥ (13)\\
{\small\eng{Wrists}}\\
\\
ये ती॒र्थानि॑ प्र॒चर॑न्ति सृ॒काव॑न्तो निष॒ङ्गिण॑: ।\\
तेषाग्ं॑ सहस्रयोज॒नेऽव॒धन्वा॑नि तन्मसि ॥ हस्ताभ्यां नमः॥ (14)\\
{\small\eng{Hands}}\\
\\
स॒द्योजा॒तं प्र॑पद्या॒मि॒ स॒द्योजा॒ताय॒ वै नमो॒ नमः॑ ।\\
भ॒वे भ॑वे॒ नाति॑भवे भवस्व॒ माम् । भ॒वोद्भ॑वाय॒ नमः ॥ अङ्गुष्ठाभ्यां नमः॥ (15)\\
{\small\eng{Roll Ring Fingers on Thumb o\eng{f} each hand}}\\
\\
वा॒म॒दे॒वाय॒ नमो᳚ ज्ये॒ष्ठाय॒ नमः॑ श्रे॒ष्ठाय॒ नमो॑ रु॒द्राय॒ नमः॒ \\
काला॑य नमः॒ कल॑ विकरणाय॒ नमो॒ बल॑ विकरणाय॒ नमो॒\\
बला॑य॒ नमो॒ बल॑प्रमथनाय॒ नमः॒ सर्व॑ भूत दमनाय॒ \\
नमो॑ म॒नोन्म॑नाय॒ नमः॒ ॥ तर्जनीभ्यां नमः॥ (16)\\
{\small\eng{Roll Thumb on Ring Fingers o\eng{f} both Hands}}\\
\\
अ॒घोरे᳚भ्योऽथ॒ घोरे᳚भ्यो॒ घोर॒घोर॑तरेभ्यः । \\
स॒र्वे᳚तः॑ सर्व॒ शर्वे᳚भ्यो॒ नम॑स्ते अस्तु रु॒द्र रू॑पेभ्यः ॥ मध्यमाभ्यां नमः॥ (17)\\
{\small\eng{Roll Thumb on Middle Fingers o\eng{f} both Hands}}\\
\\
तत्पुरु॑षाय वि॒द्महे॑ महादे॒वाय॑ धीमहि । \\
तन्नो॑ रुद्रः प्रचो॒दया᳚त् ॥ अनामिकाभ्यां नमः॥ (18)\\
{\small\eng{Roll Thumb on Ring Fingers o\eng{f} both Hands}}\\
\\
ईशानः सर्व॑ विद्या॒ना॒ मीश्वरः सर्व॑भूता॒नां॒\\
ब्रह्माधि॑पति॒र्ब्रह्म॒णोऽधि॑पति॒ र्ब्रह्मा॑ शि॒वो मे॑ अस्तु सदाशि॒वोम् ॥\\
कनिष्ठिकाभ्यां नमः॥ (19)\\
{\small\eng{Roll Thumb on Little Fingers o\eng{f} both Hands}}\\
\\
{\small\eng{Not in challakere rendition}}\\
नमो हिरण्यबाहवे हिरण्यवर्णाय हिरण्यरूपाय \\
हिरण्यपतयेऽम्बिकापतय उमापतये\\
पशुपतये॑ नमो॒ नमः॥ करतल करपृष्ठाभ्यां नमः॥ (20)\\
{\small\eng{Rub each palm over other, front and back}}\\
\\
नमो॑ वः किरि॒केभ्यो॑ दे॒वाना॒ग्ं॒ हृद॑येभ्यः । हृदयाय नमः॥ (21)\\
{\small\eng{Touch Heart}}\\
\\
नमो॑ ग॒णेभ्यो॑ ग॒णप॑तिभ्यश्च वो॒ नमः॑ ॥ पृष्ठाय नमः॥ (22)\\
{\small\eng{Touch Back}}\\
\\
नम॒स्तक्ष॑भ्यो रथका॒रेभ्य॑श्च वो॒  नमः॑ ॥ कक्षाभ्यां नमः॥ (21)\\
{\small\eng{Armpit to Waist}}\\
\\
नमो॒ हिर॑ण्यबाहवे सेना॒न्ये॑ दि॒शां च॒ पत॑ये॒  नमः॑ ॥ पार्श्वाभ्यां नमः॥ (22)\\
{\small\eng{Trunk}}\\
\\
विज्यं॒ धनु॑: कप॒र्दिनो॒ विश॑ल्यो॒ बाण॑वाग्ं उ॒त ।\\
अने॑शन्न॒स्येष॑व आ॒भुर॑स्य निष॒ङ्गथि॑: ॥ जठराय नमः॥ (23)\\
{\small\eng{Stomach}}\\
\\
हि॒र॒ण्य॒ग॒र्भः सम॑वर्त॒ताग्रे॑ भू॒तस्य॑ जा॒तः पति॒रेक॑ आसीत्।\\
स दा॑धार पृथि॒वीं द्यामु॒तेमां कस्मै॑ दे॒वाय॑ ह॒विषा॑ विधेम॥ \\
नाभ्यै नमः। (24)	\\
{\small\eng{Navel}}\\
\\
मीढु॑ष्टम॒ शिव॑तम शि॒वो न॑: सु॒मना॑ भव । प॒र॒मे वृ॒क्ष \\
आयु॑धन्नि॒धाय॒ कृत्तिं॒ वसा॑न॒ आच॑र॒ पिना॑कं॒ बिभ्र॒दाग॑हि ॥ \\
कट्यै नमः॥ (25)\\
{\small\eng{Waist}}\\
    \\
ये भू॒ताना॒मधि॑पतयो विशि॒खास॑: कप॒र्दिन॑: ॥\\
तेषाग्ं॑ सहस्रयोज॒नेऽव॒धन्वा॑नि तन्मसि ॥ गुह्याय नमः॥ (26)\\
{\small\eng{Upper Reproductive Organs}}\\
\\
ये अन्ने॑षु वि॒विध्य॑न्ति॒ पात्रे॑षु॒ पिब॑तो॒ जनान्॑ ।\\
तेषाग्ं॑ सहस्रयोज॒नेऽव॒धन्वा॑नि तन्मसि ॥ अण्डाभ्यां नमः॥ (27)\\
{\small\eng{Lower Reproductive Organs}}\\
\\
स शि॑रा जा॒तवे॑दाः। अ॒क्षरं॑ पर॒मं प॒दं। वे॒दाना॒ꣳ॒ शिर॑ उत्त॒मम्।\\
जातवे॑दसे॒ शिर॑सि मा॒ता ब्रह्म॒ भूर्भुव॒स्सुवरोम्‌॥ अपानाय नमः॥ (28)\\
{\small\eng{Anus}}\\
\\
मा नो॑ म॒हान्त॑मु॒त मा नो॑ अर्भ॒कं मा न॒ उक्ष॑न्तमु॒त मा न॑ उक्षि॒तम् ।\\
मा नो॑ऽवधीः पि॒तरं॒ मोत मा॒तरं॑ प्रि॒या मा न॑स्त॒नुवो॑ रुद्र रीरिषः ॥\\
ऊरुभ्यां नमः॥ (29)\\
{\small\eng{Thighs}}\\
\\
एष ते रुद्र भागस्तं जुषस्व तेनाऽवसेन परो \\
मूर्जवतोऽ तीह्य वतत धन्वा पिनाक हस्तः कृत्तिवासाः॥ \\
जानुभ्यां नमः॥ (30)\\
{\small\eng{Knees}}\\
\\
स॒ꣳ॒सृ॒ष्ट॒ जित् सो॑ म॒पा बा॑हु श॒र्ध्यू᳚र्ध्व धन्वा॒ प्रति॑हिता भि॒रस्ता᳚ ।\\
बृह॑स्पते॒ परि॑दीया॒ रथे॑न रक्षो॒हाऽमित्राꣳ॑ अप॒बा ध॑मानः॥ \\
जङ्घाभ्यां नमः॥ (31)\\
{\small\eng{Knees to Ankles}}\\
\\
विश्वं॑ भू॒तं भुव॑नं चि॒त्रं ब॑हु॒धा जा॒तं जाय॑मानं च॒ यत्।\\
सर्वो॒ ह्ये॑ष रु॒द्रस्तस्मै॑ रु॒द्राय॒ नमो॑ अस्तु ॥ गुल्फाभ्यां नमः॥ (32)\\
{\small\eng{Ankles}}\\
\\
ये प॒थां प॑थि॒रक्ष॑य ऐलबृ॒दा य॒व्युधः॑।\\
तेषाꣳ॑ सहस्रऽयोज॒नेऽव॒धन्वा॑नि तन्मसि ॥ पादाभ्यां नमः॥ (33)\\
{\small\eng{Feet}}\\
\\
अध्य॑वोचदधिव॒क्ता प्र॑थ॒मो दैव्यो॑ भि॒षक्।\\
अहीग्॑श्च॒ सर्वा᳚ञ्ज॒म्भय॒न्थ्सर्वा᳚श्च यातुधा॒न्यः ॥ कवचाय नमः॥ (34)\\
{\small\eng{Cross hands across chest touching shoulder}}\\
\\
नमो॑ बि॒ल्मिने॑ च कव॒चिने॑ च॒नमः॑ श्रु॒ताय॑ च श्रुतसे॒नाय॑ च ॥ \\
उपकवचाय नमः ॥ (34)\\
{\small\eng{kavacha at elbow level}}\\
\\
नमो॑ अस्तु॒ नील॑ग्रीवाय सहस्रा॒क्षाय॑ मी॒ढुषे᳚ ।\\
अथो ये अस्य सत्वांनोऽहं तेभ्योऽकरन्नमः॥ तृतीय नेत्राय नमः॥ (35)\\
{\small\eng{Index/Middle/Ring at eyes/middle o\eng{f} eyebrows}}\\
\\
प्रमु॑ञ्च॒ धन्व॑न॒स्त्वमु॒भयो॒रार्त्रि॑यो॒र्ज्याम्। \\
याश्च॑ ते॒ हस्त॒ इष॑वः॒ परा॒ ता भ॑गवो वप ॥ अस्त्राय नमः ॥ (36)\\
{\small\eng{Slap index/middle o\eng{f} right on left palm}}\\
\\
य ए॒ताव॑न्तश्च॒ भूयाꣳ॑ सश्च॒ दिशो॑ रु॒द्रा वि॑तस्थि॒रे।\\
तेषाꣳ॑ सहस्रऽयोज॒नेऽव॒धन्वा॑नि तन्मसि ॥ दिग्बन्धाय नमः॥ (37)\\
{\small\eng{Snap middle/thumb with click sound across self}}\\
\\
ओं नमो भगवते॑ रुद्रा॒य ॥\\
\subsection{\eng{Dashanga Nyasaha}}
\\
आं मूर्ध्ने नमः ॥ नं नासिकायै नमः॒ ॥ मों ललाटाय नमः ॥\\
भं मुखाय नमः ॥ गं कण्ठाय नमः ॥  वं हृदयाय नमः॥\\
तें दक्षिण हस्ताय नमः ॥  रुं वाम हस्ताय नमः ॥\\
द्रां नाभ्यै नमः ॥ यं पादाभ्यां नमः॒॥\\
\\
\subsection{\eng{Panchanga nyasaha}}
\\
स॒द्योजा॒तं प्र॑पद्या॒मि॒ स॒द्योजा॒ताय॒ वै नमो॒ नमः॑ ।\\
भ॒वे भ॑वे॒ नाति॑भवे भवस्व॒ माम् । भ॒वोद्भ॑वाय॒ नमः । पादाभ्यां नमः ॥\\
\\
वा॒म॒दे॒वाय॒ नमो᳚ ज्ये॒ष्ठाय॒ नमः॑  श्रे॒ष्ठाय॒ नमो॑ रु॒द्राय॒ नमः॒ \\
काला॑य नमः॒ कल॑ विकरणाय॒ नमो॒  बल॑ विकरणाय॒ नमो॒\\
बला॑य॒ नमो॒ बल॑प्रमथनाय॒ नमः॒ \\
सर्व॑ भूत दमनाय॒ नमो॑ म॒नोन्म॑नाय॒ नमः॒ ॥ ऊरुमध्यमाभ्यां नमः ॥\\
\\
अ॒घोरे᳚भ्योऽथ॒ घोरे᳚भ्यो॒ घोर॒घोर॑तरेभ्यः । \\
स॒र्वे᳚तः॑ सर्व॒ शर्वे᳚भ्यो॒ नम॑स्ते अस्तु रु॒द्र रू॑पेभ्यः\\
हृदयाय नमः ॥\\
\\
तत्पुरु॑षाय वि॒द्महे॑ महादे॒वाय॑ धीमहि । \\
तन्नो॑ रुद्रः प्रचो॒दया᳚त्  ॥ मुखाय नमः ॥\\
\\
ईशानः सर्व॑ विद्या॒ना॒ मीश्वरः सर्व॑भूता॒नां॒\\
ब्रह्माधि॑पति॒र्ब्रह्म॒णोऽधि॑पति॒ \\
र्ब्रह्मा॑ शि॒वो मे॑ अस्तु सदाशि॒वोम् ॥ मूर्ध्ने नमः ॥\\
