\section{\eng{Narayana Sooktam}}
ॐ ॥ स॒ह॒स्र॒शीर्॑षं दे॒वं॒ वि॒श्वाक्षं॑ वि॒श्वशं॑भुवम् ।\\
विश्वं॑ ना॒राय॑णं दे॒व॒म॒क्षरं॑ पर॒मं पदम् ।\\
\\
वि॒श्वतः॒ पर॑मान् नित्यं वि॒श्वं ना॑राय॒णग्ं ह॑रिम् ।\\
विश्व॑मे॒ वेदं पुरु॑ष॒स् तद् विश्व-मुप॑ जीवति ।\\
\\
पतिं॒ विश्व॑स् यात्मेश् वर॒ग्ं॒ शाश्व॑तग्ं शि॒व-म॑च्युतम् ।\\
ना॒राय॒णं म॑हाज्ञे॒यं॒ वि॒श्वात्मा॑नं प॒राय॑णम् ।\\
\\
ना॒राय॒णप॑रो ज्यो॒ति॒ रा॒त्मा ना॑राय॒णः प॑रः ।\\
ना॒राय॒णपरं॑ ब्र॒ह्म॒ तत्त्वं ना॑राय॒णः प॑रः ।\\
\\
ना॒राय॒णप॑रो ध्या॒ता॒ ध्या॒नं ना॑राय॒णः प॑रः ।\\
यच्च॑ किं॒ चिज्जगत् सर्वं दृ॒श्यते᳚ श्रूय॒तेऽपि॑ वा ॥\\
\\
अन्त॑र्ब॒हिश्च॑ तत्स॒र्वं॒ व्या॒प्य ना॑राय॒णः स्थि॑तः ।\\
अनन्त॒ मव्ययं॑ क॒विग्ं स॑मु॒द्रेंऽतं॑ वि॒श्व शं॑भुवम् ।\\
\\
प॒द्म॒ को॒श-प्र॑ती का॒श॒ग्ं॒ हृ॒दयं॑ चाप्य॒ धोमु॑खम् ।\\
अधो॑ नि॒ष्ट्या वि॑तस् यांते॒ ना॒भ्या मु॑परि॒ तिष्ठ॑ति ।\\
\\
ज्वा॒ल॒ मा॒ला  कु॑लं भा॒ती॒ वि॒श्वस् यायत॒नं म॑हत् ।\\
सन्त॑तग्ं शि॒ला भि॑स् तुलं बत्याको श॒सन् निभं ।\\
\\
तस्यान्ते॑ सुषि॒रग्ं सू॒क्ष्मं तस्मिन्᳚ स॒र्वं प्रति॑ष्ठितम् ।\\
तस्य॒ मध्ये॑ म॒हान॑ग्नि-र्वि॒श्वार्चि॑-र्वि॒श्वतो॑मुखः ।\\
\\
सोऽग्र॑ भु॒ग्  विभ॑जन्  तिष्ठ॒न्  नाहा॑ रमज॒रः क॒विः ।\\
ति॒र्य॒ गू॒र्ध्व-म॑धश् शायी॒ र॒श्-मय॑स् तस्य॒ सन्त॑ता ।\\
\\
सं॒ ता॒प य॑ति स्वं दे॒ह मापा॑ दतल॒ मस्त॑कः ।\\
तस्य॒ मध्ये॒ वह्(न्) निशिखा अ॒णीयो᳚ र्ध्वा व्य॒वस्थि॑तः।\\
\\
नी॒लतो॑-यद॑ मध् यस्था॒-द्विध् युल्ले॑ खेव॒ भास्व॑रा ।\\
नी॒वार॒ शूक॑ वत् तन्वी॒ पी॒ताभा᳚ स्वत्य॒ णूप॑मा ।\\
\\
तस्याः᳚ शिखा॒या म॑ध्ये प॒रमा᳚ त्मा व्य॒वस्थि॑तः ।\\
स ब्रह्म॒ स शिवः॒ स हरिः॒ सेन्द्रः॒ सोऽक्ष॑रः पर॒मः स्व॒राट् ॥\\
\\
ऋतग्ं स॒त्यं प॑रं ब्र॒ह्म॒ पु॒रुषं॑ कृष्ण॒पिङ्ग॑लम् ।\\
ऊ॒र्ध्वरे॑तं वि॑रूपा॒क्षं॒ वि॒श्वरू॑पाय॒ वै नमो॒ नमः॑ ॥\\
\\
ना॒रा॒य॒णाय॑ वि॒द्महे॑ वासुदे॒वाय॑ धीमहि ।\\
तन्नो॑ विष्णुः प्रचो॒दया᳚त् ॥ \\
\\
ॐ शान्तिः॒ शान्तिः॒ शान्तिः॑ ॥\\
