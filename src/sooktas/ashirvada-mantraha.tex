\section{\eng{Ashirvada Mantraha}}
\subsubsection{\eng{TS 3-2-8-4, 3-2-8-5}}
प्री॑णात्य॒स्मे दे॑वासो॒ वपु॑षे चिकिथ्सत॒ यमा॒शिरा॒ दम्प॑ती वा॒मम॑श्ञु॒तः ।\\
पुमा᳚न् पु॒त्रो जा॑यते वि॒न्दते॒ वस्वथ॒ विश्वे॑ अर॒पा ए॑धते गृ॒हः ॥\\
आ॒शी॒र्दा॒या दम्प॑ती वा॒मम॑श्ञुता॒मरि॑ष्टो॒ रायः॑ सचता॒ꣳ॒ समो॑कसा ।\\
य आऽसि॑च॒थ् सं दु॑ग्धं कु॒म्भ्या स॒हेष्टेन॒ याम॒न्नम॑तिं जहातु॒ सः ॥\\
स॒र्पि॒र्ग्री॒वी\\
पीव॑र्यस्य जा॒या पीवा॑नः पु॒त्रा अकृ॑शासो अस्य ।\\
स॒हजा॑नि॒र्यः सु॑मख॒स्यमा॑न॒ इन्द्रा॑या॒ऽऽशिरꣳ॑ स॒ह कु॒म्भ्याऽदा᳚त् ॥\\
आ॒शीर्म॒ ऊर्ज॑मु॒त सु॑प्रजा॒स्त्वमिषं॑ दधातु॒ द्रवि॑ण॒ꣳ॒ सव॑र्चसं ।\\
सं॒जय॒न् क्षेत्रा॑णि॒ सह॑सा॒ऽहमि॑न्द्र कृण्वा॒नो अ॒न्याꣳ अध॑रान्थ्स॒पत्नान्॑ ॥\\
\subsubsection{\eng{TB 3-3-10-3, 3-3-10-4}}
आ॒शिष॑मे॒वैतामाशा᳚स्ते पूर्णपा॒त्रे । अ॒न्त॒तो॑ऽनु॒ष्टुभा᳚ ।\\
चतु॑ष्प॒द्वा ए॒तच्छन्दः॒ प्रति॑ष्ठितं॒ पत्नि॑यै पूर्णपा॒त्रे भ॑वति ।\\
अ॒स्मिन्ँलो॒के प्रति॑तिष्ठा॒नीति॑ । अ॒स्मिन्ने॒व लो॒के प्रति॑तिष्ठति ।\\
अथो॒ वाग्वा अ॑नु॒ष्टुक् । वाङ्मि॑थु॒नम् ।\\
आपो॒ रेतः॑ प्र॒जन॑नम् ।\\
ए॒तस्मा॒द्वै मि॑थु॒नाद्-वि॒द्योत॑मानः स्त॒नय॑न् वर्.षति ।\\
रेतः॑ सि॒ञ्चन्न् ।\\
प्र॒जाः प्र॑ज॒नयन्न्॑ ॥ यद्वै य॒ज्ञस्य॒ ब्रह्म॑णा यु॒ज्यते᳚ ।\\
ब्रह्म॑णा॒ वै तस्य॑ विमो॒कः । अ॒द्भिः शान्तिः॑ ।\\
विमु॑क्तं॒ँवा ए॒तर्.हि॒ योक्त्रं॒ ब्रह्म॑णा ।\\
आ॒दायै॑न॒त्पत्नी॑ स॒हाप उप॑गृह्णीते॒ शान्त्यै᳚ । अ॒ञ्ज॒लौ पू᳚र्णपा॒त्र-मान॑यति ।\\
रेत॑ ए॒वास्यां᳚ प्र॒जां द॑धाति । प्र॒जया॒ हि म॑नु॒ष्यः॑ पू॒र्णः ।\\
मुखं॒ँविमृ॑ष्टे । अ॒व॒भृ॒थस्यै॒व रू॒पं कृ॒त्वोत्ति॑ष्ठति ॥\\
\subsubsection{\eng{TB 3-9-11-3, 3-9-11-4}}
गो॒मृ॒ग॒क॒ण्ठेन॑ प्रथ॒मा-माहु॑तिं जुहोति ।\\
प॒शवो॒ वै गो॑मृ॒गः । रु॒द्रो᳚ऽग्निः स्वि॑ष्ट॒कृत् ।\\
रु॒द्रादे॒व प॒शून॒न्तर्द॑धाति । अथो॒ यत्रै॒षा-ऽऽहु॑तिर्. हू॒यते᳚ ।\\
न तत्र॑ रु॒द्रः प॒शून॒भिम॑न्यते ॥\\
अ॒श्व॒श॒फेन॑ द्वि॒तीया॒-माहु॑तिं जुहोति । प॒शवो॒ वा एक॑शफम् ।\\
रु॒द्रो᳚ऽग्निः स्वि॑ष्ट॒कृत् । रु॒द्रादे॒व प॒शू-न॒न्तर्द॑धाति ।\\
अथो॒ यत्रै॒षा-ऽऽहु॑तिर्. हू॒यते᳚ । न तत्र॑ रु॒द्रः प॒शू-न॒भिम॑न्यते ॥\\
अ॒य॒स्मये॑न कम॒ण्डलु॑ना तृ॒तीया᳚म् ।\\
आहु॑तिं जुहोत्याया॒स्यो॑ वै प्र॒जाः । रु॒द्रो᳚ऽग्निः स्वि॑ष्ट॒कृत् ।\\
रु॒द्रादे॒व प्र॒जा अ॒न्तर्द॑धाति । अथो॒ यत्रै॒षा-ऽऽहु॑तिर्. हू॒यते᳚ ।\\
न तत्र॑ रु॒द्रः प्र॒जा अ॒भिम॑न्यते ॥\\
\subsubsection{\eng{EK 1-11-1}}
अप॑श्यं त्वा॒ मन॑सा॒ चेकि॑तानं॒ तप॑सो जा॒तं तप॑सो॒ विभू॑तम् ।\\
इ॒ह प्र॒जामि॒ह र॒यिꣳ ररा॑णः॒ प्रजा॑यस्व प्र॒जया॑ पुत्रकाम ॥\\
अप॑श्यं त्वा॒ मन॑सा॒ दीद्ध्या॑ना॒ꣲ॒ स्वायां᳚ त॒नूꣳ ऋ॒त्विये॒ नाथ॑मानाम् ।\\
उप॒ मामु॒च्चा यु॑व॒तिर् बुभू॑याः॒ प्रजा॑यस्व प्र॒जया॑ पुत्रकामे ॥\\
\subsubsection{\eng{TS 5-4-12-3}}
सर्व॒स्याऽऽप्त्यै॒ सर्व॑स्य॒ जित्यै॒ सर्व॑मे॒व तेना᳚ऽऽप्नोति॒\\
सर्वं॑ जयति ॥\\
\subsubsection{\eng{TB 2-4-7-1, 2-4-7-2}}
आयु॑ष्मन्तं॒ँवर्च॑स्वन्तम् ।\\
अथो॒ अधि॑पतिंँवि॒शाम् ।\\
अ॒स्याः पृ॑थि॒व्या अद्ध्य॑क्षम् । इ॒ममि॑न्द्र वृष॒भं कृ॑णु ॥\\
\subsubsection{\eng{TB 1-2-1-19, 1-2-1-20}}
पू॒ष्णः पोषे॑ण॒ मह्य᳚म् ।\\
दी॒र्घा॒यु॒त्वाय॑ श॒तशा॑रदाय ।\\
श॒तꣳ श॒रद्भ्य॒ आयु॑षे॒ वर्च॑से ।\\
जी॒वात्वै पुण्या॑य ॥\\
\subsubsection{\eng{TB 2-7-14-3}}
ए॒तेन॒ वै दे॒वा जैत्वा॑नि जि॒त्वा ।\\
यं काम॒-मका॑मयन्त॒ तमा᳚प्नुवन्न् ।\\
यं कामं॑ का॒मय॑ते । तमे॒तेना᳚प्नोति ॥\\
\subsubsection{\eng{TB 3-1-1-12}}
पृ॒थ्वी सु॒वर्चा॑ युव॒तिः स॒जोषाः᳚ । पौ॒र्ण॒मा॒स्युद॑गा॒-च्छोभ॑माना ।\\
आ॒प्या॒यय॑न्ती दुरि॒तानि॒ विश्वा᳚ ।\\
उ॒रुं दुहां॒ँयज॑मानाय य॒ज्ञम्\\
\subsubsection{\eng{TB 3-1-2-1}}
ऋ॒द्ध्यास्म॑ ह॒व्यैर्-नम॑सोप॒ सद्य॑ । मि॒त्रं दे॒वं मि॑त्र॒धेय॑न्नो अस्तु ।\\
अ॒नू॒रा॒धान्. ह॒विषा॑ व॒र्द्धय॑न्तः । श॒तं जी॑वेम श॒रदः॒ सवी॑राः ॥\\
\subsubsection{\eng{TB 3-1-2-7}}
क्ष॒त्रस्य॒ राजा॒ वरु॑णोऽधिरा॒जः । नक्ष॑त्राणाꣳ श॒तभि॑ष॒ग् वसि॑ष्ठः ।\\
तौ दे॒वेभ्यः॑ कृणुतो दी॒र्घमायुः॑\\
\subsubsection{\eng{TS 2-4-14-1}}
नवो॑नवो भवति॒ जाय॑मा॒नोऽह्नां᳚ के॒तुरु॒षसा॑मे॒त्यग्रे᳚ ।\\
भा॒गं दे॒वेभ्यो॒ वि द॑धात्या॒यन् प्र च॒न्द्रमा᳚स्तिरति दी॒र्घमायुः॑ ॥\\
\subsubsection{\eng{Text from Vishnu Sooktam}}
पर्या᳚ प्त्या॒ अन॑न्तरायाय॒ सर्व॑स्तोमोऽति \\
रा॒त्र उ॑त्त॒म मह॑र्भवति सर्व॒स्याप्त्यै॒ सर्व॑स्य॒ \\
जित्त्यै॒ सर्व॑मे॒व तेना᳚प्नोति॒ सर्वं॑ जयति ॥\\
\subsubsection{\eng{TS 2-3-11-5}}
श॒तमा॑नं भवति श॒तायुः॒ पुरु॑षः श॒तेन्द्रि॑य॒ आयु॑ष्ये॒वेन्द्रि॒ये\\
प्रति॑ तिष्ठ॒त्यथो॒ खलु॒ याव॑तीः॒\\
\subsubsection{\eng{EK 1-9-3}}
सु॒म॒ङ्ग॒लीरि॒यंँव॒धूरि॒माꣳ स॑मेत॒ पश्य॑त ।\\
सौभा᳚ग्यम॒स्यै द॒त्वायाथास्तं॒ँविपरे॑तन ॥\\
\subsubsection{\eng{EK 1-4-1}}
इ॒मां त्वमि॑न्द्र मीढ्वः सुपु॒त्राꣳ सु॒भगां᳚ कुरु (कृणु) ।\\
दशा᳚स्यां पु॒त्रानाधे॑हि॒ पति॑मेकाद॒शं कृ॑धि ॥ 6 ॥\\
\subsubsection{\eng{Rig Veda Sri Suktam Kila}}
श्रीर्वर्चस्वमायुष्यमारोग्यमाविधाच्छोभमानं महीयते।\\
धनं धान्यं पशुं बहुपुत्रलाभं शतसंवत्सरं दीर्घमायुः ॥35॥\\
\subsubsection{\eng{Parishista Mantraha}}
शतं जीव शरदो वर्धमान इत्य मिथिकमो भवति शत मिति \\
शतं धीर्ग मायुर् मरुत येना वर्थ यन्थि शतमे न मेव \\
शथार्थ् मानं भवथि शत मनन्थं भवथि शतं ऐश्वर्यं भवथि \\
शथमिथि शतं दीर्घमायुः\\
\subsubsection{\eng{TB 1-8-9-1}}
श॒तमा॑नो भवति श॒तक्ष॑रः ( ) । श॒तायुः॒ पुरु॑षः श॒तेन्द्रि॑यः ।\\
आयु॑ष्ये॒वेन्द्रि॒ये प्रति॑ तिष्ठति । आयु॒र्वै हिर॑ण्यम् ।\\
आ॒यु॒ष्या॑ ए॒वैन॑-म॒भ्यति॑ क्षरन्ति । तेजो॒ वै हिर॑ण्यम् ।\\
ते॒ज॒स्या॑ ए॒वैन॑म॒भ्यति॑ क्षरन्ति । वर्चो॒ वै हिर॑ण्यम् ।\\
व॒र्च॒स्या॑ ए॒वैन॑म॒भ्यति॑ क्षरन्ति ॥\\
\subsubsection{\eng{TB 3-8-15-3}}
श॒ताय॒ स्वाहेत्या॑ह ।\\
श॒तायु॒र्वै पुरु॑षः श॒तवी᳚र्यः । आयु॑रे॒व वी॒र्य॑मव॑रुन्धे ।\\
स॒हस्रा॑य॒ स्वाहेत्या॑ह । आयु॒र्वै स॒हस्र᳚म् । आयु॑रे॒वा-व॑रुन्धे ॥\\
सर्व॑स्मै॒ स्वाहेत्या॑ह । अप॑रिमितमे॒वाव॑रुन्धे ॥\\
\subsubsection{\eng{TB 3-11-9-8}}
ते॒ज॒स्वी य॑श॒स्वी ब्र॑ह्मवर्च॒सी स्या॒मिति॑ ।\\
प्राङाहोतु॒र्द्धिष्ण्या॒-दुथ्स॑र्पेत् । येयं प्रागा॒द् यश॑स्वती ।\\
सा मा॒ प्रोर्णो॑तु । तेज॑सा॒ यश॑सा ब्रह्मवर्च॒सेनेति॑ ।\\
ते॒ज॒स्व्ये॑व य॑श॒स्वी ब्र॑ह्मवर्च॒सी भ॑वति ॥\\
\subsubsection{\eng{TB 3-7-1-1, 3-7-1-2}}
सर्वा॒न्॒. वा ए॒षो᳚ऽग्नौ कामा॒न् प्रवे॑शयति ।\\
यो᳚-ऽग्नीन॑न्वा॒धाय॑ व्र॒तमु॒पैति॑ ।\\
स यदनि॑ष्ट्वा प्रया॒यात् । अका॑मप्रीता एनं॒ कामा॒ नानु॒ प्रया॑युः ।\\
अ॒ते॒जा अ॑वी॒र्यः॑ स्यात् । स जु॑हुयात् । तुभ्यं॒ ता अ॑ङ्गिरस्तम ।\\
विश्वाः᳚ सुक्षि॒तयः॒ पृथ॑क् । अग्ने॒ कामा॑य येमिर॒ इति॑ ।\\
कामा॑ने॒वास्मि॑न् दधाति ।\\
काम॑प्रीता एनं॒ कामा॒ अनु॒प्रया᳚न्ति । ते॒ज॒स्वी वी॒र्या॑वान् भवति ॥\\
\subsubsection{\eng{TB 2-6-4-6}}
सि॒ꣳ॒हस्य॒ लोम॒ त्विषि॑-रिन्द्रि॒याणि॑ ॥\\
अङ्गा᳚न्या॒त्मन्-भि॒षजा॒ तद॒श्विना᳚ । आ॒त्मान॒मङ्गैः॒ सम॑धा॒थ् सर॑स्वती ।\\
इन्द्र॑स्य रू॒पꣳ श॒तमा॑न॒मायुः॑ ।\\
\subsubsection{\eng{TB 2-3-2-4, 2-3-2-5}}
प॒शुभ्योऽधीन्द्र᳚म् ।\\
तदिन्द्रं॒ँयश॑ आर्च्छत् । तदे॑नं॒ नाति॒ प्राच्य॑वत ।\\
इन्द्र॑ इव यश॒स्वी भ॑वति ।\\
\subsubsection{\eng{TS 1-6-1-1}}
सं त्वा॑ सिञ्चामि॒ यजु॑षा प्र॒जामायु॒र्द्धनं॑ च ।\\
\subsubsection{\eng{TB 3-12-5-4, 3-12-5-5}}
ए॒तैरायु॑ष्कामः । प्र॒जाप॒शुका॑मो वा\\
पु॒रस्ता॒-द्दश॑होतार॒-मुद॑ञ्च॒-मुप॑दधाति यावत्प॒दं ।\\
हृद॑यं॒ँयजु॑षी॒ पत्न्यौ॑ च । द॒क्षि॒ण॒तः प्राञ्चं॒ चतु॑र्.होतारं ।\\
प॒श्चादुद॑ञ्चं॒ पञ्च॑होतारं । उ॒त्त॒र॒तः प्राञ्च॒ꣳ॒ षड्ढो॑तारं ।तैत्तिरीय ब्राह्मणम्\\
उ॒परि॑ष्टा॒त् प्राञ्चꣳ॑ स॒प्तहो॑तारं । हृद॑यं॒ँयजूꣳ॑षि॒ पत्न्य॑श्च ।\\
य॒था॒व॒का॒शं ग्रहान्॑ । य॒था॒व॒का॒शं प्र॑तिग्र॒हान्ँलो॑कं पृ॒णाश्च॑ ।\\
सर्वा॑ हास्यै॒ता दे॒वताः᳚ प्री॒ता अ॒भीष्टा॑ भवन्ति । \\
\subsubsection{\eng{TB 3-10-1-1}}
रो॒च॒नो रोच॑मानः-शोभ॒नः-शोभ॑मानः क॒ल्याणः॑ ॥\\




