\section{\eng{Vishnu Sooktam}}
ॐ विष्णो॒र्नुकं॑ वी॒र्या॑णि॒ प्रवो॑चं॒ यः \\
पार्थि॑वानि विम॒मे रजाग्ं॑ सि॒यो \\
अस्क॑भाय॒दुत्त॑रग्ं स॒धस्थं॑ \\
विचक्रमा॒णस्त्रे॒धोरु॑गा॒यो\\
 विष्णो॑र॒राट॑मसि॒ विष्णोः᳚ \\
पृ॒ष्ठम॑सि॒ विष्णोः॒ श्नप्त्रे᳚स्थो॒ \\
विष्णो॒स्स्यूर॑सि॒ विष्णो᳚र्ध्रु॒वम॑सि\\
वैष्ण॒वम॑सि॒ विष्ण॑वे त्वा ॥\\
\\
तद॑स्य प्रि॒यम॒भिपाथो॑ अश्याम् । \\
नरो यत्र॑ देव॒यवो॒ मद॑न्ति । \\
उ॒रु॒क्र॒मस्य॒ स हि बन्धु॑रि॒त्था । \\
विष्णो᳚ प॒दे प॑र॒मे मध्व॒ उथ्सः॑ । \\
\\
प्रतद्विष्णु॑स्स्तवते वी॒र्या॑य । \\
मृ॒गो न भी॒मः कु॑च॒रो गि॑रि॒ष्ठाः । \\
यस्यो॒रुषु॑ त्रि॒षु वि॒क्रम॑णेषु । \\
अधि॑क्षि॒यन्ति॒ भुव॑नानि॒ विश्वा᳚  । \\
प॒रो मात्र॑या त॒नुवा॑ वृधान । \\
न ते॑ महि॒त्वमन्व॑श्नुवन्ति ॥\\
\\
उ॒भे ते॑ विद्म॒ रज॑सी पृथि॒व्या विष्णो॑ देव॒त्वम् । \\
प॒र॒मस्य॑ विथ्से । \\
विच॑क्रमे पृथि॒वीमे॒ष ए॒ताम् । \\
क्षेत्रा॑य॒ विष्णु॒र्मनु॑षे दश॒स्यन् । \\
ध्रु॒वासो॑ अस्य की॒रयो॒ जना॑सः । \\
ऊ॒रु॒क्षि॒तिग्ं सु॒जनि॑माचकार । \\
त्रिर्दे॒वः पृ॑थि॒वीमे॒ष ए॒ताम् । \\
विच॑क्रमे श॒तर्च॑सं महि॒त्वा । \\
प्रविष्णु॑रस्तु त॒वस॒स्तवी॑यान् । \\
त्वे॒षग्ग् ह्य॑स्य॒ स्थवि॑रस्य॒ नाम॑ ॥\\
\\
अतो᳚ दे॒वा अ॑वन्तु नो॒ यतो॒ विष्णु᳚र्विचक्र॒मे । \\
पृ॒थि॒व्याः स॒प्तधाम॑भिः । \\
इ॒दं विष्णु॒र्विच॑क्र॒मे त्रे॒धा निद॑धे प॒दम् । \\
समू॑ढमस्य पाग्ं सु॒रे ॥ \\
\\
त्रीणि॑ प॒दा विच॑क्रमे॒ विष्णु॑र्गो॒पा अदा᳚भ्यः । \\
ततो॒ धर्मा॑णि धा॒रयन्॑ । \\
विष्णोः॒ कर्मा॑णि पश्यत॒ यतो᳚ व्र॒तानि॑ पस्प॒शे । \\
इन्द्र॑स्य॒ युज्यः॒ सखा᳚ ॥\\
\\
तद्विष्णोः᳚ पर॒मं प॒दग्ं सदा॑ पश्यन्ति सू॒रयः॑ । \\
दि॒वीव॒ चक्षु॒रात॑तम् । \\
तद्विप्रा᳚सो विप॒न्यवो᳚ जागृ॒वाग्ं स॒स्समि᳚न्धते । \\
विष्णो॒र्यत्प॑र॒मं प॒दम् । \\
पर्या᳚ प्त्या॒ अन॑न्तरायाय॒ सर्व॑स्तोमोऽति \\
रा॒त्र उ॑त्त॒म मह॑र्भवति सर्व॒स्याप्त्यै॒ सर्व॑स्य॒ \\
जित्त्यै॒ सर्व॑मे॒व तेना᳚प्नोति॒ सर्वं॑ जयति ॥\\
\\
ॐ शान्तिः॒ शान्तिः॒ शान्तिः॑ ॥\\
