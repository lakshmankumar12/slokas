\section{\eng{Gosha Shanthi}}
शन्नो॒ वातः॑ पवतां मात॒ रिश्वाा॒ शन्न॑स् तपतु॒ सूर्यः॑ । \\
अहा॑नि॒ शं भ॑वन्तु न॒श् शꣳ रात्रिः॒ प्रति॑ धियतां ॥\\
शमु॒ षानो॒ व्यु॑च् छतु॒ शमा॑ दि॒त्य उदे॑ तुनः । \\
शि॒वा न॒श् शन्त॑ मा भव सुमृ डी॒का सर॑स्वति ।\\
माते॒ व्यो॑म स॒न्दृशि॑ ॥\\
इडा॑यै॒ वास् त्व॑सि वास्तु॒ मद्वा᳚स् तु॒मन्तो॑ भूयास् म॒मा \\
वास्तो᳚श् छिथ्स्म ह्यवा॒स्तुस् सभू॑या॒त्\\
यो᳚ऽस् मान् द्वेष्टि॒ यंच॑ व॒यं द्वि॒ष्मः ॥\\
प्र॒ति॒ष् ठाऽसि॑ प्रति॒ष् ठाव॑न्तो भूयास् म॒मा प्र॑ति॒ष् ठाया᳚\\
छिथ्स्म ह्य प्रति॒ष् ठस्स भू॑या॒त् \\
यो᳚ऽस् मान् द्वेष्टि॒ यंच॑ व॒यं द्वि॒ष्मः ॥\\
आवा॑तु वाहि भेष॒जव्ँ विवा॑त वाहि॒ यद् रपः॑ ।\\
त्वꣳहि वि॒श्व भे॑षजो दे॒वानां᳚ दू॒त ईय॑से ॥\\
द्वावि॒मौ वातौ वात॒ आ सिन्धो॒रा प॑रा॒ वतः॑ । \\
दक्षं॑ मे, अ॒न्य आ॒वातु॒ परा॒ऽन्यो वा॑तु॒ यद् रपः॑ ॥\\
यद॒दो वा॑त ते गृ॒हे॑ऽ मृत॑स्य नि॒धिर् हि॒तः । \\
ततो॑ नो देहि जी॒वसे॒ ततो॑ नो धेहि भेष॒जं ।\\
ततो॑ नो॒ मह॒ आव॑ह॒ वात॒ आवा॑तु भेष॒जं ॥\\
शं॒ भूर्म॑ यो॒भुर्नो॑ हृ॒दे प्रण॒ आयुꣳ॑षि तारिषत् । \\
इन्द्र॑स्य गृ॒हो॑ऽसि॒ तं त्वा॒ प्र॑पद्ये॒ सगु॒स् साश्वः॑ ।\\
स॒ह यन्मे॒ अस्ति॒ तेन॑ ॥\\
भूः प्रप॑द्ये॒ भुवः॒ प्रप॑द्ये॒ सुवः॒ प्रप॑द्ये॒ भूर्भुव॒स्सुवः॒ प्रप॑द्ये\\
वा॒युं प्रप॒द्येऽना᳚र्त्तां दे॒वतां॒ प्रप॒द् येऽश्मा॑ नमाख॒णं प्रप॑द्ये \\
प्र॒जाप॑तेर् ब्रह्म को॒शं ब्रम॒ प्रप॑द्य॒ ओं प्रप॑द्ये ॥\\
अ॒न्तरि॑क्षं म उ॒र्व॑न्तरं॑ बृ॒ह द॒ग् नयः॒ पर्व॑ ताश्च॒य या॒वातः॑ \\
स्व॒स्त्या स्वस्ति॒ मान्तया᳚ स्व॒स्त्या स्व॑स्ति॒ मान॑ सानि ॥\\
प्राणा॑ पानौ मृ॒त्योर्मा॑ पातं॒ प्राणा॑ पानौ॒ मा मा॑ हाशिष्ठं॒ मयि॑ \\
मे॒धां मयि॑ प्र॒जां मय्य॒ग्नि स्तेजो॑ दधातु॒ मयि॑ मे॒धां मयि॑ प्र॒जां मयीन्द्र॑ \\
इन्द्रि॒यं द॑धातु॒ मयि॑ मे॒धां मयि॑ प्र॒जां मयि॒ सूर्यो॒ भ्राजो॑ दधातु ॥\\
द्यु॒भि र॒त्युभिः॒ परि॑पात म॒स्मान रि॑ष्टे भिरश् विना॒ सौभ॑ गेभिः । \\
तन्नो॑ मि॒त्रो वरु॑णो माम हन्ता॒ मदि॑तिः॒ सिन्धुः॑ पृथि॒वी, उ॒तद्यौः ॥\\
कया॑ नश्चि॒त्र आभु॑व दू॒ती स॒दा वृ॑ध॒स् सखा᳚ । \\
कया॒ शचि॑ष् ठया वृ॒ता ॥\\
\\
कस्त्वा॑ स॒त्यो मदा॑नां॒ मꣳ हि॑ष्ठो मथ्स॒ दन्ध॑सः । \\
दृ॒ढा चि॑दा॒ रुजे॒ वसु॑ ॥\\
अ॒भी षुण॒स् सखी॑ना मवि॒ता ज॑रि त्रे॒णां । श॒तं भ॑वास् यू॒तिभिः॑ ॥\\
वय॑स् सुप॒र्णा, उप॑से दु॒रिन्द्रं॑ प्रि॒यमे॑धा॒ ऋष॑यो॒ नाध॑ मानाः । \\
अप॑द् ध्वा॒न्त मू᳚र् णु॒हि पू॒र्द्धि चक्षु॑र् मुमु॒ग् ध्य॑स्मान् नि॒ध ये॑व ब॒ध्दान् ॥\\
शन्नो॑ दे॒वी र॒भिष्ट॑य॒ आपो॑ भवन्तु पी॒तये᳚ । शांँय्यो र॒भिस्र॑ वन्तुनः ॥\\
ईशा॑ना॒ वार् या॑णां॒ क्षय॑न्तीः चर्ष णी॒नां । अ॒पो या॑चामि भेष॒जं ॥\\
सु॒मि॒त्रान॒ आप॒ ओष॑धयस् सन्तु दुर् मि॒त्रास् तस्मै॑ \\
भुयासु॒र् यो᳚ऽस्मान् द्वेष्टि॒ यंच॑ व॒यं द्वि॒ष्मः ॥\\
आपो॒ हिष्ठा म॑यो॒भुव॒स् तान॑ ऊ॒र्जे द॑धातन । म॒हे रणा॑य॒ चक्ष॑से । \\
यो वः॑ शि॒वत॑मो॒ रस॒स् तस्य॑ भाजय ते॒ह नः॑ । उ॒श॒तीरि॑व मा॒तरः॑ ।\\
तस्मा॒ अरं॑ गमामवो॒ यस्य॒ क्षया॑य॒ जिन्व॑थ । आपो॑ ज॒नय॑था चनः ॥\\
पृ॒थि॒वी शा॒न्ता साऽग्निना॑ शा॒न्ता सा मे॑ शा॒न्ता शुचꣳ॑ शमयतु । \\
अ॒न्तरि॑क्षꣳ शा॒न्तं तद् वा॒युना॑ शा॒न्तं तन्मे॑ शा॒न्तꣳ शुचꣳ॑ शमयतु । \\
\\
द्यौः शा॒न्ता साऽऽदि॒त्येन॑ शा॒न्ता सा मे॑ शा॒न्ता शुचꣳ॑ शमयतु ॥\\
पृ॒थि॒वी शान्ति॑ र॒न्तरि॑क्ष॒ꣳ॒॑ शान्ति॒र् द्यौर्श् शान्ति॒र् दिशः॒ शान्ति॑ \\
रवान्त रदि॒शाः शान्ति॑ र॒ग्निश् शान्ति॑र्\\
वा॒युश् शान्ति॑ रादि॒त्यश् शान्ति॑श् च॒न्द्रमा॒श् शान्ति॒र्\\
नक्ष॑त्राणि॒ शान्ति॒ राप॒श् शान्ति॒ रोष॑धय॒श् शान्ति॒र् वन॒स्पत॑य॒श् शान्ति॒र्\\
गौश् शान्ति॑ र॒जा शान्ति॒ रश्व॒श् शान्तिः॒ पुरु॑ष॒श् शान्ति॒र् ब्रम॒  शान्ति॑र्\\
ब्राह्म॒णश् शान्ति॒श् शान्ति॑ रे॒व शान्ति॒श् शान्ति॑र् मे, अस्तु शान्तिः॥\\
तया॒ऽहꣳ॑ शा॒न्त्या स॑र्व शा॒न्त्या मह्यं॑ द्वि॒पदे॒ चतु॑ष्पदे च॒ \\
शान्तिं॑ करोमि॒ शान्ति॑र्मे, अस्तु॒ शान्तिः॑ ॥\\
एह॒ श्रीश्च॒ ह्रीश्च॒ धृति॑श्च॒ तपो॑ मे॒धा प्र॑ति॒ष्ठा श्र॒द्धा स॒त्यं \\
धर्म॑श् चै॒तानि॒ मोत् ति॑ष्ठन्त॒ मनूत् ति॑ष्ठन्तु॒ मामा॒॒ꣳ॒ श्रीश्च॒ ह्रीश्च॒ \\
धृति॑श्च॒ तपो॑ मे॒धा प्र॑ति॒ष्ठा श्र॒द्धा स॒त्यं धर्म॑श् चै॒तानि॑ मा॒ मा हा॑सिषुः ॥\\
उदायु॑षा स्वा॒युषो दोष॑ धीना॒ꣳ॒ रसे॒नोत् \\
प॒र्जन्य॑स्य॒ शुष्मे॒ णोद॑स्था म॒मृताꣳ॒ अनु॑ ॥\\
तच् चक्षु॑र् दे॒वहि॑तं पु॒रस्ता᳚त् शु॒क्र मु॒च्चर॑त् ॥\\
पश्ये॑म श॒रद॑श्श॒तं जीवे॑म श॒रद॑श्श॒तं नन्दा॑म श॒रद॑श्श॒तं \\
मोदा॑म श॒रद॑श्श॒तं भवा॑म श॒रद॑श्श॒तꣳ शृ॒णवा॑म श॒रद॑श्श॒तं \\
प्रब्र॑वाम श॒रद॑श्श॒त मजी॑तास् याम श॒रद॑श्श॒तं जोक्च॒ सूर्यं॑ दृ॒शे ॥\\
य उद॑गान् मह॒तोऽर् णवा᳚न् वि॒ब्राज॑ मानस् सरि॒ रस्य॒ मद् ध्या॒थ् समा॑\\
वृष॒भो लो॑हिता॒क् षस् सूर्यो॑ विप॒श् चिन् मन॑सा पुनातु ॥\\
ब्रह्म॑ण॒श् चोत॑न्यसि॒ ब्रह्म॑ण आ॒णीस्थो॒ ब्रह्म॑ण आ॒व प॑नमसि धारि॒तेयं \\
पृ॑थि॒वी ब्रह्म॑णा म॒हि धा॑रि॒ तमे॑ नेन म॒हद॒न् तरि॑क्षं दि॒वं दा॑धार पृथि॒वीꣳ \\
सदे॑वाय् यद॒हंँ वेद॒ तद॒हं धा॑रयाणि॒ मामद् वेदोऽधि॒ विस् र॑सत् । \\
मे॒धा॒ म॒नी॒षे मावि॑ शताꣳ स॒मीची॑ भू॒तस्य॒ भव्य॒स् याव॑रुद् \\
ध्यै॒ सर्व॒ मायु॑र याणि॒ सर्व॒ मायु॑र याणि ॥\\
आ॒भिर् गी॒र् भिर् यदतो॑न ऊ॒न माप्या॑य यह रिवो॒ वर्द्ध॑मानः । \\
य॒दा स्तो॒ तृभ्यो॒ महि॑ गो॒त्रा रु॒जासि॑ भूयिष् ठ॒भाजो॒ , अध॑ते श्याम ।\\
ब्रह्म॒ प्रावा॑ दिष्म॒ तन्नो॒ माहा॑ सीत् ॥\\
॥ ओं शान्तिः॒ शान्तिः॒ शान्तिः॑ ॥\\
