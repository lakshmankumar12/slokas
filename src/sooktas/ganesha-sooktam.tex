\section{\eng{Ganesha Sooktam}}
आ तू न॑ इन्द्र क्षु॒मन्तं᳚ चि॒त्रं ग्रा॒भं सं गृ॑भाय ।\\
म॒हा॒ह॒स्ती दक्षि॑णेन ॥ 1 ॥\\
वि॒द्मा हि त्वा᳚ तुवि कू॒र्मिन् तु॒विदे᳚ष्णं तु॒वी म॑घम् ।\\
तु॒वि॒ मा॒त्र मवो᳚भिः ॥ 2 ॥\\
न॒हित्वा᳚ शूर दे॒वान मर्ता᳚ सो॒दित्स᳚न्तम् ।\\
भी॒मन्न गां वा॒रय᳚न्ते ॥ 3 ॥\\
एतो॒न् विन्द्रं॒ स्तवा॒ मेशा᳚ नं॒वस्व॑: स्व॒राजम्᳚ ।\\
नरा ध॑सा मर्धि षन्नः ॥ 4 ॥\\
प्रस्तो᳚ ष॒दुप॑ गासि ष॒च्छ्र व॒त्साम॑ गी॒य मा᳚नम् ।\\
अ॒भि राध॑ साजुगुरत् ॥ 5 ॥\\
आ नो᳚ भर॒ दक्षि॑णे ना॒भि स॒व्येन॒ प्रमृ॑श ।\\
इन्द्र॒ मानो॒ वसो॒र्निर्भा᳚क् ॥ 6 ॥\\
उप॑ क्रम॒स्वा भ॑र धृष॒ता धृ॑ष्णो॒ जना᳚नाम् ।\\
अदा᳚शूष्टरस्य॒ वेद॑: ॥ 7 ॥\\
इन्द्र॒य उ॒नुते॒ अस्ति॒ वाजो॒ विप्रे᳚भि॒: सनि॑त्वः ।\\
अ॒स्माभि॒: सुतं स॑नुहि ॥ 8 ॥\\
स॒द्यो॒ जुव॑स्ते॒ वाजा᳚, अ॒स्मभ्य᳚म् वि॒श्वश्च᳚न्द्राः ।\\
वशै᳚श्च म॒क्षू ज॑रन्ते ॥ 9 ॥\\
ग॒णानां᳚ त्वा ग॒णप॑तिं हवामहे\\
क॒विं क॑वी॒नामु॑प॒मश्र॑वस्तमम् ।\\
ज्ये॒ष्ठ॒राजं॒ ब्रह्म॑णां ब्रह्मणस्पत॒\\
आ न॑: शृ॒ण्वन्नू॒तिभि॑स्सीद॒ साद॑नम् ॥ 10 ॥\\
नि षु सी᳚द गणपते ग॒णेषु॒ त्वामा᳚हु॒र् विप्र॑तमं कवी॒नाम् ।\\
न ऋ॒ते त्वत् क्रि॑यते॒ किं च॒नारे म॒हा म॒र्कं म॑घवञ्चि॒त्र म॑र्च ॥ 11 ॥\\
अ॒भि॒ख् यानो᳚ मघव॒न् नाध॑ माना॒न्त् सखे᳚ बो॒धि व॑सुपते॒ सखी᳚नाम् ।\\
रणं᳚ कृधि रण कृत्सत् यशु॒ष् माभ॑क्ते चि॒दा भ॑जा रा॒ये, अ॒स्मान् ॥12॥\\
ॐ शान्ति॒-श्शान्ति॒-श्शान्तिः॑ ॥\\
