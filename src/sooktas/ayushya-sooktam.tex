\section{\eng{Ayushya Sooktam}}
यो ब्र॒ह्मा ब्रह्मण उ॑ज्जहा॒र॒ प्रा॒णैः शि॒रः कृत्तिवासाः᳚ पिना॒की ।\\
ईशानो देवस्सन आयु॑र्दधा॒तु॒ तस्मै जुहोमि हवि॑षा घृते॒न ॥ 1\\
\\
वि॒भ्रा॒ज॒मा॒न-स्सरिर॑स्य म॒द्ध्या॒द् रो॒च॒मा॒नो घर्मरुचि॑र्य आ॒गात् ।\\
स मृत्यु पाशान पनु॑द्य घो॒रा॒ नि॒हा॒यु॒ षे॒णो घृतम॑त्तु दे॒वः ॥ 2\\
\\
ब्रह्मज्योतिर् ब्रह्मपत्नी॑षु ग॒र्भं॒ य॒मा॒द॒धात् पुरुरूपं॑ जय॒न्तं ।\\
सुवर्णरंभ ग्रह म॑र्क म॒र्च्यं॒ त॒मा॒यु॒षे वर्द्ध यामो॑ घृते॒न ॥ 3\\
\\
श्रियंँलक्ष्मी-मौबलामं॑ बिकां॒ गां॒ ष॒ष्टीञ्च या॒मिन्द्र-सेने᳚त्युदा॒हुः ।\\
तांँ॒वि॒द्यां ब्रह्मयोनिꣳ॑ सरू॒पा॒-मि॒हा॒यु॒षे तर्पयामो॑ घृते॒न ॥ 4\\
\\
दाक्षायण्य-स्सर्वयोन्य॑ स्सुयो॒न्य॒-स्स॒ह॒स्र॒शो विश्व रूपा॑ विरू॒पाः ।\\
ससूनव-स्सपतयः॑ सयू॒थ्या॒ आ॒यु॒षे॒णो घृतमिदं॑ जुष॒न्तां ॥ 5\\
\\
दिव्या॑ गणा बहुरूपाः᳚ पु॒रा॒णा॒ आ॒यु॒श्छि॒दो॒ नः प्रमथ्न॑न्तु वी॒रान् ।\\
तेभ्यो जुहोमि बहुधा॑ घृ॒ते॒न॒ मा॒ नः॒ प्र॒जाꣳ रीरिषो मो॑त वी॒रान् ॥ 6\\
\\
ए॒कः॒ पु॒रस्ताद् य इदं॑ ब॒भू॒व॒ यतो बभूव भुवन॑स्य गो॒पाः ।\\
यमप्येति भुवनꣳ सां᳚परा॒ये॒ स नो हविर् घृत-मिहायुषे᳚ऽत्तु दे॒वः ॥ 7\\
\\
व॒सून् रुद्रा॑नादि॒त्या॒न् मरुतो॑ऽथ सा॒द्ध्या॒न् ऋभू॑न् य॒क्षा॒न्\\
गन्धर्वाꣲश्च पितृꣲ॑श्च वि॒श्वान् ।\\
भृ॒गू॒न्थ् सर्पाꣲ॑श्चाङ्गि रसो॑ऽथ स॒र्वा॒न्\\
घृ॒त॒ꣳ॒ हु॒त्वा स्वायुष्या महया॑म श॒श्वत् ॥ 8\\
\\
विष्णो॒ त्वन्नो॒ अन्त॑म॒-श्शर्म॑यच्छ सहन्त्य ।\\
प्रते॒ धारा॑ मधु॒श्चुत॒ उथ्सं॑दुह्रते॒ अक्षि॑तं ॥ 9\\