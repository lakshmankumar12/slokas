\section{\eng{Sri Sooktam}}
ओम् ॥ हिर॑ण्यवर्णां॒ हरि॑णीं सु॒वर्ण॑रज॒ तस्र॑जाम् ।\\
च॒न्द्रां हि॒रण्म॑यीं-ल॒क्ष्मीं जात॑वेदो म॒ आव॑ह ॥ 1 ॥\\
\\
तां म॒ आव॑ह॒ जात॑वेदो ल॒क्ष्मीमन॑पगा॒मिनी᳚म् ।\\
यस्यां॒ हिर॑ण्यं वि॒न्देयं॒ गामश्वं॒ पुरु॑षान॒हम् ॥ 2 ॥\\
\\
अ॒श्व॒पू॒र्वां र॑थम॒ध्यां ह॒स्तिना᳚द प्र॒बोधि॑नीम् ।\\
श्रियं॑ दे॒वीमुप॑ह्वये॒ श्रीर्मा᳚  दे॒वीर्जु॑षताम् ॥ 3 ॥\\
\\
कां॒ सो॑स्मि॒ तां हिर॑ण्यप्रा॒कारा॑मा॒र्द्रां ज्वल॑न्तीं तृ॒प्तां त॒र्पय॑न्तीम् ।\\
प॒द्मे॒ स्थि॒तां प॒द्मव॑र्णां॒ तामि॒होप॑ह्वये॒ श्रियम् ॥ 4 ॥\\
\\
च॒न्द्रां प्र॑भा॒सां य॒शसा॒ ज्वल॑न्तीं॒ श्रियं॑-लो॒के दे॒वजु॑ष्टामुदा॒राम् ।\\
तां प॒द्मिनी॑मीं॒ शर॑णम॒हं प्रप॑द्येऽल॒क्ष्मीर्मे॑ नश्यतां॒ त्वां-वृँ॑णे ॥ 5 ॥\\
\\
आ॒दि॒त्यव॑र्णे॒ तप॒सोऽधि॑जा॒तो वन॒स्पति॒स्तव॑ वृ॒क्षोऽथ॑ बि॒ल्वः ।\\
तस्य॒ फला᳚नि॒ तप॒सानु॑दन्तु मा॒यान्त॑रा॒याश्च॑ बा॒ह्या अ॑ल॒क्ष्मीः ॥ 6 ॥\\
\\
उपै॑तु॒ मां दे᳚वस॒खः की॒र्तिश्च॒ मणि॑ना स॒ह ।\\
प्रा॒दु॒र् भू॒तोऽस् मि॑राष् ट्रे॒ऽस्मिन् की॒र्ति॒मृ॑द्धिं द॒दातु॑ मे ॥ 7 ॥\\
\\
क्षु॒त्पि॒ पा॒सा म॑लां ज्ये॒ष्ठाम॒ल॒क्षी-र्ना॑श या॒म्यहम् ।\\
अभू॑ति॒मस॑मृद्धिं॒ च स॒र्वां॒ निर्णु॑द मे॒ गृहात् ॥ 8 ॥\\
\\
ग॒न्ध॒द्वा॒रां दु॑राध॒र्​षां॒ नि॒त्यपु॑ष्टां करी॒षिणी᳚म् ।\\
ई॒श्वरीग्ं॑ सर्व॑भूता॒नां॒ तामि॒होप॑ह्वये॒ श्रियम् ॥ 9 ॥\\
{\small श्री᳚र्मे भ॒जतु । अल॒क्षी᳚र्मे न॒श्यतु ।}\\
\\
मन॑सः॒ काम॒माकू᳚तिं-वा॒चः स॒त्यम॑शीमहि ।\\
प॒शू॒नां रू॒पमन्न॑स्य॒ मयि॒ श्रीः श्र॑यतां॒-यशाः॑ ॥ 10 ॥\\
\\
क॒र्दमे॑न प्र॑जाभू॒ता॒ म॒यि॒ सम्भ॑व क॒र्दम ।\\
श्रियं॑-वाँ॒सय॑ मे कु॒ले॒ मा॒तरं॑ पद्म॒मालि॑नीम् ॥ 11 ॥\\
\\
आपः॑ सृ॒जन्तु॑स् नि॒ग्धाा॒नि॒ चि॒क्ली॒ तव॑स मे॒ गृहे ।\\
नि च॑ दे॒वीं मा॒तरं॒ श्रियं॑-वा॒सय॑ मे कु॒ले ॥ 12 ॥\\
\\
आ॒र्द्रां पु॒ष्करि॑णीं पु॒ष्टिं॒ पि॒ङ्ग॒लां प॑द्ममा॒लिनीम् ।\\
च॒न्द्रां हि॒रण्म॑यीं-लँ॒क्ष्मीं जात॑वेदो म॒आव॑ह ॥ 13 ॥\\
\\
आ॒र्द्रां-यः॒ करि॑णीं-य॒ष्टिं॒ सु॒व॒र्णां हे॑ममा॒लिनीम् ।\\
सू॒र्यां हि॒रण्म॑यीं-ल॒क्ष्मीं॒ जात॑वेदो म॒आव॑ह ॥ 14 ॥\\
\\
तां म॒ आव॑ह॒ जात॑वेदो ल॒क्षीमन॑पगा॒मिनी᳚म् ।\\
यस्यां॒ हिर॑ण्यं॒ प्रभू॑तं॒ गावो॑ दा॒स्योऽश्वा॑न्, वि॒न्देयं॒ पुरु॑षान॒हम् ॥ 15 ॥\\
\\
ॐ म॒हा॒दे॒व्यै च॑ वि॒द्महे॑ विष्णुप॒त्नी च॑ धीमहि ।\\
तन्नो॑ लक्ष्मीः प्रचो॒दया᳚त् ॥ 16 ॥\\
\\
सक्तु मिव तित उना पुनन्तो यत्र धीरा मनसा वाच मक्रत । \\
अत्रा सखा᳚यः सख्यानी जानते भद्रैषां᳚ लक्ष्मी र्निहिता धिवाचि ॥\\
