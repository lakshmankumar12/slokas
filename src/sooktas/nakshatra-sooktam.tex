\section{\eng{Nakshatra Sooktam - Vakyam}}
\subsection{\eng{kruthika}}
{\small 1.नक्षत्रं - कृत्तिका   देवता - अग्निः}\\
अ॒ग्निर्नः॑ पातु॒ कृत्ति॑काः । नक्ष॑त्रं दे॒वमि॑न्द्रि॒यं । इ॒दमा॑सांँविचक्ष॒णं ।\\
ह॒विरा॒सं जु॑होतन । यस्य॒ भान्ति॑ र॒श्मयो॒ यस्य॑ के॒तवः॑ ।\\
यस्ये॒मा विश्वा॒ भुव॑नानि॒ सर्वा᳚ । स कृत्ति॑काभिर॒भि स॒म्ँवसा॑नः ।\\
अ॒ग्निर्नो॑ दे॒व स्सु॑वि॒ते द॑धातु ॥ 1\\
\subsection{\eng{rohini}}
{\small 2. नक्षत्रं - रोहिणी  देवता - प्रजापतिः}\\
प्र॒जाप॑ते रोहि॒णी वे॑तु॒ पत्नी᳚ । वि॒श्वरू॑पा बृह॒ती चि॒त्रभा॑नुः ।\\
सा नो॑ य॒ज्ञस्य॑ सुवि॒ते द॑धातु । यथा॒ जीवे॑म श॒रद॒ स्सवी॑राः ।\\
रो॒हि॒णी दे॒व्युद॑गात् पु॒रस्ता᳚त् । विश्वा॑ रू॒पाणि॑ प्रति॒मोद॑माना ।\\
प्र॒जाप॑तिꣳ ह॒विषा॑ व॒र्द्धय॑न्ती । प्रि॒या दे॒वाना॒-मुप॑यातु य॒ज्ञं ॥ 2\\
\subsection{\eng{mrugashirsha}}
{\small 3.नक्षत्रं - मृगशीर्.षः   देवता -सोमः}\\
सोमो॒ राजा॑ मृगशी॒र्॒.षेण॒ आगन्न्॑ । शि॒वं नक्ष॑त्रं प्रि॒यम॑स्य॒ धाम॑ ।\\
आ॒प्याय॑मानो बहु॒ धा जने॑षु । रेतः॑ प्र॒जांँयज॑माने दधातु ।\\
यत्ते॒ नक्ष॑त्रं मृगशी॒र्॒.षमस्ति॑ । प्रि॒यꣳ रा॑जन् प्रि॒यत॑मं प्रि॒याणां᳚ ।\\
तस्मै॑ ते सोम ह॒विषा॑ विधेम । शन्न॑ एधि द्वि॒पदे॒ शंचतु॑ष्पदे ॥ 3\\
\subsection{\eng{ardhra}}
{\small 4. नक्षत्रं - आर्द्रा  देवता - रुद्रः}\\
आ॒र्द्रया॑ रु॒द्रः प्रथ॑मान एति । श्रेष्ठो॑ दे॒वानां॒ पति॑रघ्नि॒यानां᳚ ।\\
नक्ष॑त्रमस्य ह॒विषा॑ विधेम । मा नः॑ प्र॒जाꣳ री॑रिष॒न् मोत वी॒रान् ।\\
हे॒ती रु॒द्रस्य॒ परि॑ णो वृणक्तु । आ॒र्द्रा नक्ष॑त्रं जुषताꣳ ह॒विर्नः॑ ।\\
प्र॒मु॒ञ्चमा॑नौ दुरि॒तानि॒ विश्वा᳚ । अपा॒घशꣳ॑स-न्नुदता॒मरा॑तिं ॥ 4\\
\subsection{\eng{punarvasu}}
{\small 5.नक्षत्रं - पुनर्वसु   देवता -अदितिः}\\
पुन॑र्नो दे॒व्यदि॑तिः स्पृणोतु । पुन॑र्वसू नः॒ पुन॒रेतां᳚ँय॒ज्ञं ।\\
पुन॑र्नो दे॒वा अ॒भिय॑न्तु॒ सर्वे᳚ । पुनः॑ पुनर्वो ह॒विषा॑ यजामः ।\\
ए॒वा न दे॒व्यदि॑ति-रन॒र्वा । विश्व॑स्य भ॒र्त्री जग॑तः प्रति॒ष्ठा ।\\
पुन॑र्वसू ह॒विषा॑ व॒र्द्धय॑न्ती । प्रि॒यं दे॒वाना॒-मप्ये॑तु॒ पाथः॑ ॥ 5\\
\subsection{\eng{pushya}}
{\small 6.नक्षत्रं - पुष्यः    देवता -बृहस्पतिः}\\
बृह॒स्पतिः॑ प्रथ॒मञ्जाय॑मानः । ति॒ष्यं॑ नक्ष॑त्रम॒भि संब॑भूव ।\\
श्रेष्ठो॑ दे॒वानां॒ पृत॑नासु जि॒ष्णुः । दिशोऽनु॒ सर्वा॒ अभ॑यन्नो अस्तु ।\\
ति॒ष्यः॑ पु॒रस्ता॑दु॒त म॑द्ध्य॒तो नः॑ । बृह॒स्पति॑र् नः॒ परि॑पातु प॒श्चात् ।\\
बाधे॑तां॒ द्वेषो॒ अभ॑यं कृणुतां । सु॒वीर्य॑स्य॒ पत॑यस्स्याम ॥ 6\\
\subsection{\eng{aashresham}}
{\small 7. नक्षत्रं -आश्रेषं    देवता -सर्पाः}\\
इ॒दꣳ स॒र्पेभ्यो॑ ह॒विर॑स्तु॒ जुष्टं᳚ । आ॒श्रे॒षा येषा॑-मनु॒यन्ति॒ चेतः॑ ।\\
ये अ॒न्तरि॑क्षं पृथि॒वीं क्षि॒यन्ति॑ । ते न॑ स्स॒र्पासो॒ हव॒माग॑मिष्ठाः ।\\
ये रो॑च॒ने सूर्य॒स्यापि॑ स॒र्पाः । ये दिवं॑ दे॒वी-मनु॑ स॒ञ्चर॑न्ति ।\\
येषा॑माश्रे॒षा अ॑नु॒यन्ति॒ कामं᳚ । तेभ्य॑ स्स॒र्पेभ्यो॒ मधु॑मज्जुहोमि ॥ 7\\
\subsection{\eng{magha}}
{\small 8.नक्षत्रं - मघा     देवता -पितरः}\\
उप॑हूताः पि॒तरो॒ ये म॒घासु॑ । मनो॑जवस स्सु॒कृत॑ स्सुकृ॒त्याः ।\\
ते नो॒ नक्ष॑त्रे॒ हव॒माग॑मिष्ठाः । स्व॒धाभि॑र् य॒ज्ञं प्रय॑तं जुषन्तां ।\\
ये अ॑ग्निद॒ग्धा येऽन॑ग्निदग्धाः । ये॑ऽमुँलो॒कं पि॒तरः॑ क्षि॒यन्ति॑ ।\\
याꣲश्च॑ वि॒द्म याꣳ उ॑ च॒ न प्र॑वि॒द्म ।\\
म॒घासु॑ य॒ज्ञꣳ सुकृ॑तं जुषन्तां ॥ 8\\
\subsection{\eng{purva phalguni}}
{\small 9.नक्षत्रं - पूर्वफल्गु॑नी  देवता -अर्यमा}\\
गवां॒ पतिः॒ फल्गु॑नीना-मसि॒ त्वं । तद॑र्यमन् वरुण मित्र॒ चारु॑ ।\\
तं त्वा॑ व॒यꣳ स॑नि॒तारꣳ॑ सनी॒नां । जी॒वा जीव॑न्त॒मुप॒ सम्ँवि॑शेम ।\\
येने॒मा विश्वा॒ भुव॑नानि॒ सञ्जि॑ता ।\\
यस्य॑ दे॒वा अ॑नुसँ॒य्यन्ति॒ चेतः॑ ।\\
अ॒र्य॒मा राजा॒ऽजर॒-स्तुवि॑ष्मान् । फल्गु॑नीना-मृष॒भो रो॑रवीति ॥ 9\\
\subsection{\eng{uttara phalguni}}
{\small 10.नक्षत्रं - उत्तर फ॑ल्गुनी  देवता - भगः}\\
श्रेष्ठो॑ दे॒वानां᳚ भगवो भगासि । तत्त्वा॑ विदुः॒ फल्गु॑नी॒-स्तस्य॑ वित्तात् ।\\
अ॒स्मभ्यं॑ क्ष॒त्रम॒जरꣳ॑ सु॒वीर्यं᳚ । गोम॒-दश्व॑-व॒दुप॒सन्नु॑दे॒ह ।\\
भगो॑ ह दा॒ता भग॒ इत्प्र॑दा॒ता । भगो॑ दे॒वीः फल्गु॑नी॒-रावि॑वेश ।\\
भग॒स्येत्तं प्र॑स॒वं ग॑मेम । यत्र॑ दे॒वै स्स॑ध॒मादं॑ मदेम । ॥ 10\\
\subsection{\eng{hasta}}
{\small 11.नक्षत्रं - हस्तः    देवता -सविता}\\
आया॑तु दे॒व स्स॑वि॒तोप॑यातु । हि॒र॒ण्यये॑न सु॒वृता॒ रथे॑न ।\\
वह॒न्॒. हस्तꣳ॑ सु॒भगंँ॑विद्म॒नाप॑सं । प्र॒यच्छ॑न्तं॒ पपु॑रिं॒ पुण्य॒मच्छ॑ ।\\
हस्तः॒ प्रय॑च्छत्व॒मृतंँ॒वसी॑यः । दक्षि॑णेन॒ प्रति॑गृभ्णीम एनत् ।\\
दा॒तार॑-म॒द्य स॑वि॒ता वि॑देय । यो नो॒ हस्ता॑य प्रसु॒वाति॑ य॒ज्ञं ॥ 11\\
\subsection{\eng{chitra}}
{\small 12. नक्षत्रं - चित्रा  देवता - त्वष्टा}\\
त्वष्टा॒ नक्ष॑त्र-म॒भ्ये॑ति चि॒त्रां । सु॒भꣳ स॑संँयुव॒तिꣳ रोच॑मानां ।\\
नि॒वे॒शय॑न्न॒-मृता॒न् मर्त्याꣲ॑श्च । रू॒पाणि॑ पि॒ꣳ॒शन् भुव॑नानि॒ विश्वा᳚ ।\\
तन्न॒स्त्वष्टा॒ तदु॑ चि॒त्रा विच॑ष्टां । तन्नक्ष॑त्रं भूरि॒दा अ॑स्तु॒ मह्यं᳚ ।\\
तन्नः॑ प्र॒जांँवी॒रव॑तीꣳ सनोतु । गोभि॑र्नो॒ अश्वै॒ स्सम॑नक्तु य॒ज्ञं ॥ 12\\
\subsection{\eng{swathi}}
{\small 13.नक्षत्रं - स्वाती     देवता -वायुः}\\
वा॒युर् नक्ष॑त्र-म॒भ्ये॑ति॒ निष्ट्यां᳚ । ति॒ग्मशृृं॑गो वृष॒भो रोरु॑वाणः ।\\
स॒मी॒रय॒न् भुव॑ना मात॒रिश्वा᳚ । अप॒ द्वेषाꣳ॑सि नुदता॒-मरा॑तीः ।\\
तन्नो॑ वा॒यस्तदु॒ निष्ट्या॑ श्रृृणोतु । तन्नक्ष॑त्रं भूरि॒दा अ॑स्तु॒ मह्यं᳚ ।\\
तन्नो॑ दे॒वासो॒ अनु॑ जानन्तु॒ कामं᳚ । यथा॒ तरे॑म दुरि॒तानि॒ विश्वा᳚ ॥ 13\\
\subsection{\eng{vishaka}}
{\small 14.नक्षत्रं - विशाखा    देवता -इन्द्राग्नीः}\\
दू॒र-म॒स्मच्छत्र॑वो यन्तु भी॒ताः । तदि॑न्द्रा॒ग्नी कृ॑णुतां॒ तद् विशा॑खे ।\\
तन्नो॑ दे॒वा अनु॑मदन्तु य॒ज्ञं । प॒श्चात् पु॒रस्ता॒दभ॑यन्नो अस्तु ।\\
नक्ष॑त्राणा॒-मधि॑पत्नी॒ विशा॑खे । श्रेष्ठा॑विन्द्रा॒ग्नी भुव॑नस्य गो॒पौ ।\\
विषू॑च॒ श्शत्रू॑-नप॒बाध॑मानौ । अप॒क्षुध॑-न्नुदता॒मरा॑तिं । ॥ 14\\
\subsection{\eng{pournamasi}}
{\small 15. पौर्णमासि}\\
पू॒र्णा प॒श्चादु॒त पू॒पु॒रस्ता᳚त् । उन्म॑द्ध्य॒तः पौ᳚र्णमा॒सी जि॑गाय ।\\
तस्यां᳚ दे॒वा अधि॑ स॒म्ँवस॑न्तः । उ॒त्त॒मे नाक॑ इ॒ह मा॑दयन्तां ।\\
पृ॒थ्वी सु॒वर्चा॑ युव॒ति स्स॒जोषाः᳚ । पौ॒र्ण॒मा॒स्युद॑गा॒-च्छोभ॑माना ।\\
आ॒प्या॒यय॑न्ती दुरि॒तानि॒ विश्वा᳚ ।\\
उ॒रुं दुहां॒ यँज॑मानाय य॒ज्ञं ॥ 15\\
\subsection{\eng{anuradha}}
{\small 16.नक्षत्रं - अनूराधा     देवता -मित्रः}\\
ऋ॒द्ध्यास्म॑ ह॒व्यैर् नम॑सोप॒सद्य॑ । मि॒त्रं दे॒वं मि॑त्र॒धेय॑न्नो अस्तु ।\\
अ॒नू॒रा॒धान्. ह॒विषा॑ व॒र्द्धय॑न्तः । श॒तं जी॑वेम श॒रद॒स्सवी॑राः ।\\
चि॒त्रं नक्ष॑त्र॒-मुद॑गात् पु॒रस्ता᳚त् । अ॒नू॒रा॒धास॒ इति॒ यद् वद॑न्ति ।\\
तन्मि॒त्र ए॑ति प॒थिभि॑र् देव॒यानैः᳚ । हि॒र॒ण्ययै॒र् वित॑तै-र॒न्तरि॑क्षे ॥ 16\\
\subsection{\eng{jyeshta}}
{\small 17.नक्षत्रं - ज्येष्ठा      देवता -इन्द्रः}\\
इन्द्रो᳚ ज्ये॒ष्ठामनु॒ नक्ष॑त्रमेति । यस्मि॑न् वृ॒त्रंँवृ॑त्र॒तूर्ये॑ त॒तार॑ ।\\
तस्मि॑न् व॒य-म॒मृतं॒ दुहा॑नाः । क्षुध॑न्तरेम॒ दुरि॑तिं॒ दुरि॑ष्टिं ।\\
पु॒र॒न्द॒राय॑ वृष॒भाय॑ धृ॒ष्णवे᳚ । अषा॑ढाय॒ सह॑मानाय मी॒ढुषे᳚ ।\\
इन्द्रा॑य ज्ये॒ष्ठा मधु॑म॒द् दुहा॑ना । उ॒रुं कृ॑णोतु॒ यज॑मानाय लो॒कं ॥ 17\\
\subsection{\eng{moolam}}
{\small 18.नक्षत्रं - मूलं         देवता -प्रजापतिः}\\
मूलं॑ प्र॒जांँवी॒रव॑तींँविदेय । परा᳚च्येतु॒ निर्.ऋ॑तिः परा॒चा ।\\
गोभि॒र् नक्ष॑त्रं प॒शुभि॒ स्सम॑क्तं । अह॑र् भूया॒द् यज॑मानाय॒ मह्यं᳚ ।\\
अह॑र्नो अ॒द्य सु॑वि॒ते द॑धातु । मूलं॒ नक्ष॑त्र॒मिति॒ यद् वद॑न्ति ।\\
परा॑चींँवा॒चा निर्.ऋ॑तिं नुदामि । शि॒वं प्र॒जायै॑ शि॒वम॑स्तु॒ मह्यं᳚ ॥ 18\\
\subsection{\eng{purva shada}}
{\small 19.नक्षत्रं - पूर्वाषाढा   देवता -आपः}\\
या दि॒व्या आपः॒ पय॑सा संबभू॒वुः । या अ॒न्तरि॑क्ष उ॒त पार्त्थि॑वी॒र्याः ।\\
यासा॑मषा॒ढा अ॑नु॒यन्ति॒ कामं᳚ । ता न॒ आपः॒ शꣲ स्यो॒ना भ॑वन्तु ।\\
याश्च॒ कूप्या॒ याश्च॑ ना॒द्या᳚ स्समु॒द्रियाः᳚ ।\\
याश्च॑ वैश॒न्तीरु॒त प्रा॑स॒चीर्याः ।\\
यासा॑मषा॒ढा मधु॑ भ॒क्षय॑न्ति । ता न॒ आपः॒ शꣲ स्यो॒ना भ॑वन्तु ॥ 19\\
\subsection{\eng{uttara shada}}
{\small 20.नक्षत्रं - उत्तराषाढा  देवता -विश्वेदेवाः}\\
तन्नो॒ विश्वे॒ उप॑शृण्वन्तु दे॒वाः । तद॑षा॒ढा अ॒भि सम्ँय॑न्तु य॒ज्ञं ।\\
तन्नक्ष॑त्रं प्रथतां प॒शुभ्यः॑ । कृ॒षिर् वृ॒ष्टिर् यज॑मानाय कल्पतां ।\\
शु॒भ्राः क॒न्या॑ युव॒तय॑ स्सु॒पेश॑सः । क॒र्म॒कृत॑ स्सु॒कृतो॑ वी॒र्या॑वतीः ।\\
विश्वा᳚न् दे॒वान्. ह॒विषा॑ व॒र्द्धय॑न्तीः ।\\
अ॒षा॒ढाः काम॒ मुप॑यान्तु य॒ज्ञं ॥ 20\\
\subsection{\eng{abhijit}}
{\small 21.नक्षत्रं - अभिजिद्      देवता -ब्रह्मा}\\
यस्मि॒न् ब्रह्मा॒ऽभ्यज॑य॒थ् सर्व॑मे॒तत् । अ॒मुञ्च॑ लो॒कमि॒दमू॑च॒ सर्वं᳚।\\
तन्नो॒ नक्ष॑त्र-मभि॒जिद् वि॒जित्य॑ । श्रियं॑ दधा॒त्वहृ॑णीयमानं ।\\
उ॒भौ लो॒कौ ब्रह्म॑णा॒ सञ्जि॑ते॒मौ । तन्नो॒ नक्ष॑त्र-मभि॒जिद् विच॑ष्टां ।\\
तस्मि॑न् व॒यं पृत॑ना॒ स्संज॑येम । तन्नो॑ दे॒वासो॒ अनु॑जानन्तु॒ कामं᳚ ॥ 21\\
\subsection{\eng{shravanam}}
{\small 22.नक्षत्रं - श्रवणं      देवता -विष्णुः}\\
शृृ॒ण्वन्ति॑ श्रो॒णा-म॒मृत॑स्य गो॒पां ।\\
पुण्या॑मस्या॒ उप॑शृृणोमि॒ वाचं᳚ ।\\
म॒हीं दे॒वींँविष्णु॑पत्नी-मजू॒र्यां । प्र॒तीची॑-मेनाꣳ ह॒विषा॑ यजामः ।\\
त्रे॒धा विष्णु॑-रुरुगा॒यो विच॑क्रमे । म॒हीं दिवं॑ पृथि॒वी-म॒न्तरि॑क्षं ।\\
तच्छ्रो॒णैति॒ श्रव॑ इ॒च्छमा॑ना ।\\
पुण्य॒ꣲ॒ श्लोकंँ॒यज॑मानाय कृण्व॒ती ॥ 22\\
\subsection{\eng{shravishta}}
{\small 23.नक्षत्रं - श्रविष्टा     देवता -वसवः}\\
अ॒ष्टौ दे॒वा वस॑व स्सो॒म्यासः॑ ।\\
चत॑स्रो दे॒वीर॒जराः॒ श्रवि॑ष्ठाः ।\\
ते य॒ज्ञं पा᳚न्तु॒ रज॑सः प॒रस्ता᳚त् ।\\
स॒म्ँव॒थ्स॒रीण॑-म॒मृतꣲ॑ स्व॒स्ति ।\\
य॒ज्ञं नः॑ पान्तु॒ वस॑वः पु॒रस्ता᳚त् । द॒क्षि॒ण॒तो॑-ऽभिय॑न्तु॒ श्रवि॑ष्ठाः ।\\
पुण्य॒न्नक्ष॑त्र-म॒भि सम्ँवि॑शाम । मा नो॒ अरा॑ति-र॒घश॒ꣳ॒सागन्न्॑ ॥ 23\\
\subsection{\eng{shatabishak}}
{\small 24.नक्षत्रं - शतभिषक्      देवता -वरुणः}\\
क्ष॒त्रस्य॒ राजा॒ वरु॑णोऽधिरा॒जः । नक्ष॑त्राणाꣳ श॒तभि॑ष॒ग् वसि॑ष्ठः ।\\
तौ दे॒वेभ्यः॑ कृणुतो दी॒र्घमायुः॑ । श॒तꣳ स॒हस्रा॑ भेष॒जानि॑ धत्तः ।\\
य॒ज्ञन्नो॒ राजा॒ वरु॑ण॒ उप॑यातु । तन्नो॒ विश्वे॑ अ॒भि सम्ँय॑न्तु दे॒वाः ।\\
तन्नो॒ नक्ष॑त्रꣳ श॒तभि॑षग् जुषा॒णं । दी॒र्घमायुः॒ प्रति॑रद् भेष॒जानि॑ ॥ 24\\
\subsection{\eng{purva proshtapada}}
{\small 25.नक्षत्रं - पूर्वप्रोष्ठपदा   देवता -अजयेकपादः}\\
अ॒ज एक॑पा॒-दुद॑गात् पु॒रस्ता᳚त् । विश्वा॑ भू॒तानि॑ प्रति॒मोद॑मानः ।\\
तस्य॑ दे॒वाः प्र॑स॒वंँय॑न्ति॒ सर्वे᳚ । प्रो॒ष्ठ॒प॒दासो॑ अ॒मृत॑स्य गो॒पाः ।\\
वि॒भ्राज॑मान स्समिधा॒न उ॒ग्रः । आऽन्तरि॑क्ष-मरुह॒दग॒न् द्यां ।\\
तꣳ सूर्यं॑ दे॒व-म॒ज-मेक॑पादं ।\\
प्रो॒ष्ठ॒प॒दासो॒ अनु॑यन्ति॒ सर्वे᳚ ॥ 25\\
\subsection{\eng{uttara proshtapada}}
{\small 26.नक्षत्रं - उत्तरप्रोष्ठपदा   देवता -अहिर्बुद्ध्नियः}\\
अहि॑ र्बु॒द्ध्नियः॒ प्रथ॑मान एति । श्रेष्ठो॑ दे॒वाना॑मु॒त मानु॑षाणां ।\\
तं ब्रा᳚ह्म॒णा स्सो॑म॒पा स्सो॒म्यासः॑ । प्रो॒ष्ठ॒प॒दासो॑ अ॒भि र॑क्षन्ति॒ सर्वे᳚ ।\\
च॒त्वार॒ एक॑म॒भि कर्म॑ दे॒वाः । प्रो॒ष्ठ॒प॒दास॒ इति॒ यान्. वद॑न्ति ।\\
ते बु॒द्धनियं॑ परि॒षद्यꣲ॑ स्तु॒वन्तः॑ । अहिꣳ॑ रक्षन्ति॒ नम॑सोप॒सद्य॑ ॥ 26\\
\subsection{\eng{revathi}}
{\small 27.नक्षत्रं - रेवती         देवता -पूषा}\\
पू॒षा रे॒वत्यन्वे॑ति॒ पन्थां᳚ । पु॒ष्टि॒पती॑ पशु॒पा वाज॑बस्त्यौ ।\\
इ॒मानि॑ ह॒व्या प्रय॑ता जुषा॒णा । सु॒गैर्नो॒ यानै॒रुप॑यातांँय॒ज्ञं ।\\
क्षु॒द्रान् प॒शून् र॑क्षतु रे॒वती॑ नः । गावो॑ नो॒ अश्वा॒ꣳ॒ अन्वे॑तु पू॒षा ।\\
अन्न॒ꣳ॒ रक्ष॑न्तौ बहु॒ धा विरू॑पं । वाजꣳ॑ सनुतांँ॒यज॑मानाय य॒ज्ञं ॥ 27\\
\subsection{\eng{ashwini}}
{\small 28 नक्षत्रं - अश्विनी  देवता - अश्विनी देवौ}\\
तद॒श्विना॑-वश्व॒युजोप॑यातां । शुभ॒ङ्गमि॑ष्ठौ सु॒यमे॑भि॒रश्वैः᳚ ।\\
स्वन्नक्ष॑त्रꣳ ह॒विषा॒ यज॑न्तौ । मध्वा॒ संपृ॑क्तौ॒ यजु॑षा॒ सम॑क्तौ ।\\
यौ दे॒वानां᳚ भि॒षजौ॑ हव्यवा॒हौ । विश्व॑स्य दू॒ता-व॒मृत॑स्य गो॒पौ ।\\
तौ नक्ष॑त्रं जुजुषा॒णोप॑यातां । नमो॒ऽश्विभ्यां᳚ कृणुमो-ऽश्व॒युग्भ्यां᳚ ॥ 28\\
\subsection{\eng{apabharani}}
{\small 29.नक्षत्रं - अपभरणी      देवता -यमः}\\
अप॑ पा॒प्मानं॒ भर॑णीर् भरन्तु । तद् य॒मो राजा॒ भग॑वा॒न्॒. विच॑ष्टां ।\\
लो॒कस्य॒ राजा॑ मह॒तो म॒हान्. हि । सु॒गन्नः॒ पन्था॒मभ॑यं कृणोतु ।\\
यस्मि॒न् नक्ष॑त्रे य॒म एति॒ राजा᳚ । यस्मि॑न्नेन-म॒भ्यषि॑ञ्चन्त दे॒वाः ।\\
तद॑स्य चि॒त्रꣳ ह॒विषा॑ यजाम । अप॑ पा॒प्मानं॒ भर॑णीर् भरन्तु ॥ 29\\
\subsection{\eng{amavasi}}
{\small 30.अमावासि }\\
नि॒वेश॑नी स॒ङ्गम॑नी॒ वसू॑नांँ॒विश्वा॑ रू॒पाणि॒ वसू᳚न्यावे॒शय॑न्ती ।\\
स॒ह॒स्र॒पो॒षꣳ सु॒भगा॒ ररा॑णा॒ सा न॒ आ ग॒न् वर्च॑सा सम्ँविदा॒ना ।\\
यत्ते॑ दे॒वा अद॑धुर् भाग॒धेय॒-ममा॑वास्ये स॒म्ँवस॑न्तो महि॒त्वा ।\\
सा नो॑ य॒ज्ञं पि॑पृहि विश्ववारे र॒यिन्नो॑ धेहि सुभगे सु॒वीरं᳚ ॥ 30\\
\subsection{\eng{chandrama}}
{\small 31.चन्द्रमा }\\
नवो॑नवो भवति॒ जाय॑मा॒नोऽह्नां᳚ के॒तु-रु॒षसा॑मे॒त्यग्रे᳚ ।\\
भा॒गं दे॒वेभ्यो॒ वि द॑धात्या॒यन् प्रच॒न्द्रमा᳚स्तिरति दी॒र्घमायुः॑ ।\\
यमा॑दि॒त्या अ॒ꣳ॒शु-मा᳚प्या॒यय॑न्ति॒ यमक्षि॑त॒-मक्षि॑तयः॒ पिब॑न्ति ।\\
तेन॑ नो॒ राजा॒ वरु॑णो॒ बृह॒स्पति॒रा प्या॑ययन्तु॒ भुव॑नस्य गो॒पाः ।\\
\subsection{\eng{aho ratri}}
{\small 32. अहो रात्री }\\
ये विरू॑पे॒ सम॑नसा स॒म्ँव्यय॑न्ती । स॒मा॒नं तन्तुं॑ परि तात॒नाते᳚ ।\\
वि॒भू प्र॒भू अ॑नु॒भू वि॒श्वतो॑ हुवे । ते नो॒ नक्ष॑त्रे॒ हव॒माग॑मेतं ।\\
व॒यं दे॒वी ब्रह्म॑णा सम्ँविदा॒नाः । सु॒रत्ना॑सो दे॒ववी॑तिं॒ दधा॑नाः ।\\
अ॒हो॒रा॒त्रे ह॒विषा॑ व॒र्द्धय॑न्तः । अति॑ पा॒प्मान॒-मति॑मुक्त्या गमेम ॥\\
\subsection{\eng{usha}}
{\small 33. उषा }\\
प्रत्यु॑व दृश्याय॒ती । व्यु॒च्छन्ती॑ दुहि॒ता दि॒वः ।\\
अ॒पो म॒ही वृ॑णुते॒ चक्षु॑षा । तमो॒ ज्योति॑ष्कृणोति सू॒नरी᳚ ।\\
उदु॒स्त्रिया᳚ स्सचते॒ सूर्यः॑ । सचा॑ उ॒द्यन्नक्ष॑त्र-मर्चि॒मत् ।\\
तवेदु॑षो॒ व्युषि॒ सूर्य॑स्य च । सं भ॒क्तेन॑ गमेमहि ॥\\
\subsection{\eng{nakshatra}}
{\small 34. नक्षत्रः }\\
तन्नो॒ नक्ष॑त्र-मर्चि॒मत् । भा॒नु॒मत्तेज॑ उ॒च्छर॑त् ।\\
उप॑ य॒ज्ञमि॒हाग॑मत् । प्र नक्ष॑त्राय दे॒वाय॑ । इन्द्रा॒येन्दुꣳ॑ हवामहे ।\\
स न॑ स्सवि॒ता सु॑वथ् स॒निं । पु॒ष्टि॒दांँवी॒रव॑त्तमं ॥\\
\subsection{\eng{surya}}
{\small 35. सूर्यः }\\
उदु॒ त्यं जा॒तवे॑दसं दे॒वंँव॑हन्ति के॒तवः॑ ।\\
दृ॒शे विश्वा॑य॒ सूर्यं᳚ ।\\
चि॒त्रं दे॒वाना॒-मुद॑गा॒दनी॑कं॒ चक्षु॑र् मि॒त्रस्य॒ वरु॑णस्या॒ग्नेः ।\\
आऽप्रा॒ द्यावा॑पृ॒थिवी अ॒न्तरि॑क्ष॒ó्॒ सूर्य॑ आ॒त्मा\\
जग॑त स्त॒स्थुष॑श्च ॥\\
\subsection{\eng{adithi}}
{\small 36. अदितिः }\\
अदि॑तिर्न उरुष्य॒-त्वदि॑तिः॒ शर्म॑ यच्छतु ।\\
अदि॑तिः पा॒त्वꣳ ह॑सः ॥\\
म॒हीमू॒ षु मा॒तरꣳ॑ सुव्र॒ताना॑-मृ॒तस्य॒ पत्नी॒ मव॑से हुवेम ।\\
तु॒वि॒क्ष॒त्रा-म॒जर॑न्ती-मुरू॒चीꣳ सु॒शर्मा॑ण॒-मदि॑तिꣳ सु॒प्रणी॑तिं ।\\
\subsection{\eng{vishnu}}
{\small 37. विष्णुः }\\
इ॒दंँविष्णु॒र् वि च॑क्रमे त्रे॒धा नि द॑धे प॒दं ।\\
समू॑ढमस्य पाꣳ सु॒रे ॥\\
प्र तद्विष्णुः॑ स्तवते वी॒र्या॑य । मृ॒गो न भी॒मः कु॑च॒रो गि॑रि॒ष्ठाः।\\
यस्यो॒रुषु॑ त्रि॒षु वि॒क्रम॑णेषु । अधि॑क्षि॒यन्ति॒ भुव॑नानि॒ विश्वा᳚ ॥\\
\subsection{\eng{agni}}
{\small 38. अग्निः }\\
अ॒ग्निर् मू॒र्द्धा दि॒वः क॒कुत्पतिः॑ पृथि॒व्या अ॒यं ।\\
अ॒पाꣳ रेताꣳ॑सि जिन्वति ॥\\
भुवो॑ य॒ज्ञस्य॒ रज॑सश्च ने॒ता यत्रा॑ नि॒युद्भिः॒ सच॑से\\
शि॒वाभिः॑ ।\\
दि॒वि मू॒र्द्धानं॑ दधिषे सुव॒र्॒.षां जि॒ह्वाम॑ग्ने\\
चकृषे हव्य॒वाहं᳚ ।\\
\subsection{\eng{anumathi}}
{\small 39. अनुमती }\\
अनु॑ नो॒ऽद्यानु॑मतिर्य॒ज्ञं दे॒वेषु॑ मन्यतां ।\\
अ॒ग्निश्च॑ हव्य॒वाह॑नो॒ भव॑तां दा॒शुषे॒ मयः॑ ।\\
अन्विद॑नुमते॒ त्वं मन्या॑सै॒ शं च॑ नः कृधि ।\\
क्रत्वे॒ दक्षा॑य नो हिनु॒ प्र ण॒ आयूꣳ॑षि तारिषः ।\\
\subsection{\eng{havyavaha}}
{\small 40. हव्यवाहः (अग्निः) }\\
ह॒व्य॒वाह॑मभि-माति॒षाहं᳚ । र॒क्षो॒हणं॒ पृत॑नासु जि॒ष्णुं ।\\
ज्योति॑ष्मन्तं॒ दीद्य॑तं॒ पुर॑न्धिं । अ॒ग्निꣲस्वि॑ष्ट॒कृत॒-माहु॑वेम ।\\
स्वि॑ष्टमग्ने अ॒भितत् पृ॑णाहि । विश्वा॑ देव॒ पृत॑ना अ॒भिष्य ।\\
उ॒रुन्नः॒ पन्थां᳚ प्रदि॒शन् विभा॑हि ।\\
ज्योति॑ष्मद्धेह्य॒ जर॑न्न॒ आयुः॑ ॥ 28\\
