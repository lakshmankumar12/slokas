\section{\eng{Navagraha Sooktam}}
1. (देवता - आदित्या: , अधिदेवता- अग्नि:, प्रत्यधिदेवता - पशुपति:)\\
\\
ओं आस॒त्येन॒ रज॑सा॒ वर्त्त॑मानो निवे॒शय॑न्न॒मृतं॒ मर्त्त्य॑ञ्च ।\\
हि॒र॒ण्यये॑न सवि॒ता रथे॒नाऽदे॒वो या॑ति॒ भुव॑ना वि॒पश्यन्न्॑ ॥\\
\\
अ॒ग्निं दू॒तंँवृ॑णीमहे॒ होता॑रंँवि॒श्ववे॑दसं ।\\
अ॒स्य य॒ज्ञस्य॑ सु॒क्रतुं᳚ ॥ \\
\\
येषा॒मीशे॑ पशु॒पतिः॑ पशू॒नां चतु॑ष्पदामु॒त च॑ द्वि॒पदां᳚ ।\\
निष्क्री॑तो॒ऽयंँय॒ज्ञियं॑ भा॒गमे॑तु रा॒यस्पोषा॒ यज॑मानस्य सन्तु ॥\\
(ओं अधिदेवता प्रत्यधिदेवता सहिताय भगवते आदित्याय नमः) 1\\
\\
2. (देवता - सोमः अधिदेवता- आपः ,  प्रत्यधिदेवता - गौरी)\\
\\
ओं आप्या॑यस्व॒ समे॑तु ते वि॒श्वत॑-स्सोम॒ वृष्णि॑यं ।\\
भवा॒ वाज॑स्य संग॒थे ॥\\
\\
अ॒फ्सु मे॒ सोमो॑ अब्रवीद॒न्तर् विश्वा॑नि भेष॒जा ।\\
अ॒ग्निञ्च॑ वि॒श्वशं॑भुव॒-माप॑श्च वि॒श्वभे॑षजीः ॥\\
\\
गौ॒री मि॑माय सलि॒लानि॒ तक्ष॒त्येक॑पदी द्वि॒पदी॒ सा चतु॑ष्पदी ।\\
अ॒ष्टाप॑दी॒ नव॑पदी बभू॒वुषी॑ स॒हस्रा᳚क्षरा पर॒मे व्यो॑मन्न् ॥\\
(ओं अधिदेवता प्रत्यधिदेवता सहिताय भगवते सोमाय नमः) 2\\
\\
3. (देवता - अङ्गारक: , अधिदेवता - पृथिवी ,\\
प्रत्यधिदेवता - क्षेत्रपति:)\\
ओं अ॒ग्निर्मू॒र्द्धा दि॒वः क॒कुत्पतिः॑ पृथि॒व्या अ॒यं ।\\
अ॒पाꣳ रेताꣳ॑सि जिन्वति ॥\\
\\
स्यो॒ना पृ॑थिवि॒ भवा॑ऽनृक्ष॒रा नि॒वेश॑नी ।\\
यच्छा॑न॒-श्शर्म॑ स॒प्रथाः᳚॥\\
\\
क्षेत्र॑स्य॒ पति॑ना व॒यꣳ हि॒तेने॑व जयामसि ।\\
गामश्वं॑ पोषयि॒त्न्वा स नो॑ मृडाती॒दृशे᳚ ॥\\
(ओं अधिदेवता प्रत्यधिदेवता सहिताय भगवते अङ्गारकाय नमः) 3\\
\\
4. (देवता - बुधः, अधिदेवता - विष्णु: ,\\
प्रत्यधिदेवता - पुरुष नारायणः)\\
ओं उद्बु॑द्ध्यस्वाग्ने॒ प्रति॑ जागृह्येन-मिष्टापू॒र्त्ते सꣳ सृ॑जेथाम॒यञ्च॑ ।\\
पुनः॑ कृ॒ण्वꣲस्त्वा॑ पि॒तरं॒ँयुवा॑न-म॒न्वाताꣳ॑सी॒त् त्वयि॒ तन्तु॑मे॒तं ॥\\
\\
इ॒दंँविष्णु॒र् वि च॑क्रमे त्रे॒धा नि द॑धे प॒दं ।\\
समू॑ढमस्य पाꣳसु॒रे ॥\\
\\
विष्णो॑ र॒राट॑मसि॒ विष्णोः᳚ पृ॒ष्ठम॑सि॒ विष्णो॒-श्ञप्त्रे᳚स्थो॒ विष्णो॒\\
स्स्यूर॑सि॒ विष्णो᳚र् ध्रु॒वम॑सि वैष्ण॒वम॑सि॒ विष्ण॑वे त्वा ॥\\
(ओं अधिदेवता प्रत्यधिदेवता सहिताय भगवते बुधाय नमः) 4\\
\\
5. (देवता - बुहस्पतिः, अधिदेवता - इन्द्र-मरुत,\\
प्रत्यधिदेवता - ब्रह्म)\\
ओं बृह॑स्पते॒ अति॒ यद॒र्यो अर्.ह᳚द् द्यु॒मद् वि॒भाति॒ क्रतु॑ म॒ज्जने॑षु ।\\
यद् दी॒दय॒च्छव॑सर्त्त प्रजात॒ तद॒स्मासु॒ द्रवि॑णं धेहि चि॒त्रं ॥\\
\\
इन्द्र॑ मरुत्व इ॒ह पा॑हि॒ सोमंँ॒यथा॑ शार्या॒ते अपि॑ब स्सु॒तस्य॑ ।\\
तव॒ प्रणी॑ती॒ तव॑ शूर॒ शर्म॒न्ना वि॑वासन्ति क॒वयः॑ सुय॒ज्ञाः ॥\\
\\
ब्रह्म॑जज्ञा॒नं प्र॑थ॒मं पु॒रस्ता॒द् वि सी॑म॒त-स्सु॒रुचो॑ वे॒न आ॑वः ।\\
सबु॒द्ध्निया॑ उप॒मा अ॑स्य वि॒ष्ठा स्स॒तश्च॒ योनि॒मस॑तश्च॒ विवः॑ ॥\\
(ओं अधिदेवता प्रत्यधिदेवता सहिताय भगवते बृहस्पतये नमः) 5
\\
6. (देवता - शुक्रः, अधिदेवता - इन्द्राणी, प्रत्यधिदेवता - इन्द्रः)\\
ओं प्र व॑श्शु॒क्राय॑ भा॒नवे॑ भरध्वꣳ ह॒व्यं म॒तिं चा॒ग्नये॒ सुपू॑तं ।\\
यो दैव्या॑नि॒ मानु॑षा ज॒नूꣲष्य॒न्तर् विश्वा॑नि वि॒द्मना॒ जिगा॑ति ॥\\
\\
इ॒न्द्रा॒णी मा॒सु नारि॑षु सु॒पत्नी॑म॒हम॑श्रवं ।\\
न ह्य॑स्या अप॒रं च॒न ज॒रसा॒ मर॑ते॒ पतिः॑ ॥\\
\\
इन्द्रं॑ँवो वि॒श्वत॒स्परि॒ हवा॑महे॒ जने᳚भ्यः ।\\
अ॒स्माक॑मस्तु॒ केव॑लः ॥\\
(ओं अधिदेवता प्रत्यधिदेवता सहिताय भगवते शुक्राय नमः) 6\\
\\
7. (देवता - शनैश्चरः, अधिदेवता - प्रजापतिः,
प्रत्यधिदेवता - यमः)
ओं शन्नो॑ दे॒वी-र॒भिष्ट॑य॒ आपो॑ भवन्तु पी॒तये᳚ ।\\
शंँयोर॒भि-स्र॑वन्तु नः ॥\\
\\
प्रजा॑पते॒ न त्वदे॒तान्य॒न्यो विश्वा॑ जा॒तानि॒ परि॒ ता ब॑भूव ।\\
यत्का॑मास्ते जुहु॒मस्तन्नो॑ अस्तु व॒यꣲ स्या॑म॒ पत॑यो रयी॒णां ॥\\
\\
इ॒मंँय॑म प्रस्त॒रमा हि सीदाङ्गि॑रोभिः पि॒तृभिः॑ सम्ँविदा॒नः ।\\
आ त्वा॒ मन्त्राः᳚ कविश॒स्ता व॑हन्त्वे॒ना रा॑जन् ह॒विषा॑ मादयस्व ॥\\
(ओं अधिदेवता प्रत्यधिदेवता सहिताय भगवते शनैश्चराय नमः) 7\\
\\
8. (देवता - राहुः, अधिदेवता- सर्पः, प्रत्यधिदेवता - निर्.ऋति)\\
ओं कया॑ नश्चि॒त्र आ भु॑वदू॒ती स॒दावृ॑ध॒ स्सखा᳚ ।\\
कया॒ शचि॑ष्ठया वृ॒ता ॥\\
\\
आऽयङ्कौः पृश्ञि॑-रक्रमी॒दस॑नन् मा॒तरं॒ पुनः॑ ।\\
पि॒तर॑ञ्च प्र॒यन्थ्सुवः॑ ॥\\
यत्ते॑ दे॒वी निर्.ऋति॑राब॒बन्ध॒ दाम॑ ग्री॒वास्व॑विच॒र्त्यं ।\\
\\
इ॒दन्ते॒ तद्विष्या॒म्यायु॑षो॒ न मद्ध्या॒दथा॑ जी॒वः पि॒तुम॑द्धि॒ प्रमु॑क्तः ॥\\
(ओं अधिदेवता प्रत्यधिदेवता सहिताय भगवते राहवे नमः) 8\\
\\
9. (देवता - केतु: , अधिदेवता - ब्रह्म,\\
प्रत्यधिदेवता - चित्रगुप्तः)\\
ओं के॒तुङ् कृ॒ण्वन्न॑के॒तवे॒ पेशो॑ मर्या, अपे॒शसे᳚ ।\\
समु॒षद्भि॑रजायथाः ॥\\
\\
ब्र॒ह्मा दे॒वानां᳚ पद॒वीः क॑वी॒ना-मृषि॒र्विप्रा॑णां महि॒षो मृ॒गाणां᳚ ।\\
श्ये॒नो गृद्ध्रा॑णा॒ꣲ॒ स्वधि॑ति॒र् वना॑ना॒ꣳ॒ सोमः॑\\
प॒वित्र॒मत्ये॑ति॒ रेभन्न्॑ ॥\\
\\
स चि॑त्र चि॒त्रं चि॒तय॑न्त-म॒स्मे चित्र॑क्षत्र चि॒त्रत॑मंँवयो॒धां ।\\
च॒न्द्रं र॒यिं पु॑रु॒वीरं॑ बृ॒हन्तं॒ चन्द्र॑ च॒न्द्राभि॑र् गृण॒ते यु॑वस्व ॥\\
(ओं अधिदेवता प्रत्यधिदेवता सहितेभ्यो भगवद्भ्यः केतुभ्यो नमः) 9\\
\\
ओं आदित्यादि नवग्रह देवताभ्यो नमो नमः॥\\
(ओं शान्तिः॒ शान्तिः॒ शान्तिः॑ ॥)\\




