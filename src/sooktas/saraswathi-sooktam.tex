\section{\eng{Saraswathi Sooktam}}
इ॒य म॑द दाद् रभ॒स मृ॑ण॒च्युतं॒ दिवो᳚ दासं वध्र्य॒श् वाय॑ दा॒शुषे᳚ ।\\
या शश्व᳚न् तमाच॒ खशदा᳚ व॒सं प॒णिं ताते᳚ दा॒त्राणि॑ तवि॒षा स॑रस्वति ॥ १ ॥\\
इ॒यं शुष्मे᳚ भिर्बिस॒खा इ॑वारु ज॒त्सानु॑ गिरी॒णां त॑वि॒षे भि॑रू॒र् मिभिः॑ ।\\
पा॒रा॒ व॒त॒घ्नी मव॑से सुवृ॒क्ति भि॑स्सर॑स्वती॒ मा वि॑वासे मधी॒ तिभिः॑ ॥ २ ॥\\
सर॑स्वतिदे व॒निदो॒ निब॑र् हय प्र॒जां विश्व॑स्य॒ बृस॑यस्य मा॒यिनः॑ ।\\
उ॒तक् षि॒तिभ्यो॒ऽ वनी᳚ रविन्दो वि॒षमे᳚भ्यो, अस्रवो वाजिनीवति ॥ ३ ॥\\
प्रणो᳚ दे॒वी सर॑स्वती॒ वाजे᳚ भिर् वा॒जिनी᳚वती । धी॒नाम॑वि॒त्र्य॑वतु ॥ ४ ॥\\
यस्त्वा᳚ देवि सरस्वत् युपब्रू॒ते धने᳚ हि॒ते । इन्द्रं॒ न वृ॑त्र॒तूर्ये᳚ ॥ ५ ॥\\
त्वं दे᳚वि सरस्व॒त् यवा॒ वाजे᳚षु वाजिनि ।\\
रदा᳚ पू॒षेव॑ नः स॒निम् ॥ ६ ॥\\
उ॒तस् यानः॒ सर॑स्वती घो॒रा हिर᳚ण्यवर्तनिः ।\\
वृ॒त्र॒घ्नी व॑ष्टि सुष्टु॒तिम् ॥ ७ ॥\\
यस्या᳚ अन॒न्तो अह्रु॑तस् त्वे॒षश् च॑रि॒ष्णुर᳚र् ण॒वः ।\\
अम॒श् चर॑ति॒ रोरु॑वत् ॥ ८ ॥\\
सानो॒ विश्वा॒ अति॒द् विषः॒स् वसृ᳚ र॒न्या, ऋ॒ताव॑री । अत॒न्नहे᳚ व॒सूर्यः॑ ॥ ९ ॥\\
उ॒तनः॑ प्रि॒या प्रि॒यासु॑ स॒प्तस् व॑सा॒ सुजु॑ष्टा ।\\
सर॑स्वती॒स् तोम्या᳚ भूत् ॥ १० ॥\\
आ॒प॒प् रुषी॒ पार्थि॑वान् यु॒रुरजो᳚ अ॒न्तरि॑क्षम् । सर॑स्वती नि॒दस्पा᳚तु ॥ ११ ॥\\
त्रि॒ष॒ धस्था᳚ स॒प्तधा᳚तु॒ पञ्च॑ जा॒ता व॒र्धय᳚न्ती । वाजे᳚वाजे॒ हव्या᳚ भूत् ॥ १२ ॥\\
प्रया म॑हि॒म्ना म॒हिना᳚सु॒ चेकि॑ते द्यु॒म्ने भि॑र॒न्या अ॒पसा᳚ म॒पस्त॑मा ।\\
रथ॑ इव बृह॒ती वि॒भ्वने᳚ कृ॒तो प॒स्तुत्या᳚ चिकि॒ तुषा॒ सर॑स्वती ॥ १३ ॥\\
सर॑स्वत् य॒भिनो᳚ नेषि॒वस्यो॒ माप॑स् फरीः॒ पय॑सा॒ मान॒ आध॑क् ।\\
जु॒षस् व॑नः स॒ख्या वे॒श्या᳚ च॒मात् वत् क्षेत्रा॒ण् यर॑णानि गन्म ॥ १४ ॥\\
प्रक्षो द॑सा॒ धाय॑सा सस्र ए॒षा सर॑स्वती ध॒रुण॒ माय॑सी॒पूः ।\\
प्र॒बा ब॑धाना र॒थ्ये᳚व याति॒ विश्वा᳚ अ॒पो म॑हि॒ना सिन्धु॑र॒न्याः ॥ १५ ॥\\
एका᳚ चेत॒त् सर॑स्वती न॒दीनां॒ शुचि᳚र् य॒ती गि॒रिभ्य॒ आस॑ मु॒द्रात् ।\\
रा॒यश् चेत᳚न्ती॒ भुव॑नस्य॒ भूरे᳚र् घृ॒तं पयो᳚ दुदुहे॒ नाहु॑षाय ॥ १६ ॥\\
स वा᳚वृधे॒ नर्यो॒ योष॑णासु॒ वृषा॒ शिशु᳚र्वृष॒भो य॒ज्ञिया᳚सु ।\\
स वा॒जिनं᳚ म॒घव॑द्भ्यो दधाति॒ वि सा॒तये᳚ त॒न्वं᳚ मामृजीत ॥ १७ ॥\\
उ॒त स्या न॒: सर॑स्वती जुषा॒णोप॑ श्रवत्सु॒भगा᳚ य॒ज्ञे अ॒स्मिन् ।\\
मि॒तज्ञु॑भिर्नम॒स्यै᳚रिया॒ना रा॒या यु॒जा चि॒दुत्त॑रा॒ सखि॑भ्यः ॥ १८ ॥\\
इ॒मा जुह्वा᳚ना यु॒ष्मदा नमो᳚भि॒: प्रति॒ स्तोमं᳚ सरस्वति जुषस्व ।\\
तव॒ शर्म᳚न्प्रि॒यत॑मे॒ दधा᳚ना॒ उप॑ स्थेयाम शर॒णं न वृ॒क्षम् ॥ १९ ॥\\
अ॒यमु॑ ते सरस्वति॒ वसि॑ष्ठो॒ द्वारा᳚वृ॒तस्य॑ सुभगे॒ व्या᳚वः ।\\
वर्ध॑ शुभ्रे स्तुव॒ते रा᳚सि॒ वाजा॑न्यू॒यं पा᳚त स्व॒स्तिभि॒: सदा᳚ नः ॥ २० ॥\\
–(ऋ।वे।७।९६)\\
बृ॒हदु॑ गायिषे॒ वचो᳚ऽसु॒र्या᳚ न॒दीना᳚म् ।\\
सर॑स्वती॒मिन्म॑हया सुवृ॒क्तिभि॒स्स्तोमै᳚र्वसिष्ठ॒ रोद॑सी ॥ २१ ॥\\
उ॒भे यत्ते᳚ महि॒ना शु॑भ्रे॒ अन्ध॑सी अधिक्षि॒यन्ति॑ पू॒रव॑: ।\\
सा नो᳚ बोध्यवि॒त्री म॒रुत्स॑खा॒ चोद॒ राधो᳚ म॒घोना᳚म् ॥ २२ ॥\\
भ॒द्रमिद्भ॒द्रा कृ॑णव॒त्सर॑स्व॒त्यक॑वारी चेतति वा॒जिनी᳚वती ।\\
गृ॒णा॒ना ज॑मदग्नि॒वत्स्तु॑वा॒ना च॑ वसिष्ठ॒वत् ॥ २३ ॥\\
ज॒नी॒यन्तो॒ न्वग्र॑वः पुत्री॒यन्त॑: सु॒दान॑वः ।\\
सर॑स्वन्तं हवामहे ॥ २४ ॥\\
ये ते᳚ सरस्व ऊ॒र्मयो॒ मधु॑मन्तो घृत॒श्चुत॑: ।\\
तेभि᳚र्नोऽवि॒ता भ॒व ॥ २५ ॥\\
पी॒पि॒वांसं॒ सर॑स्वत॒: स्तनं॒ यो वि॒श्वद॑र्शतः ।\\
भ॒क्षी॒महि॑ प्र॒जामिषम्᳚ ॥ २६ ॥\\
अम्बि॑तमे॒ नदी᳚तमे॒ देवि॑तमे॒ सर॑स्वति ।\\
अ॒प्र॒श॒स्ता इ॑व स्मसि॒ प्रश॑स्तिमम्ब नस्कृधि ॥ २७ ॥\\
त्वे विश्वा᳚ सरस्वति श्रि॒तायूं᳚षि दे॒व्याम् ।\\
शु॒नहो᳚त्रेषु मत्स्व प्र॒जां दे᳚वि दिदिड्ढि नः ॥ २८ ॥\\
इ॒मा ब्रह्म॑ सरस्वति जु॒षस्व॑ वाजिनीवति ।\\
या ते॒ मन्म॑ गृत्सम॒दा ऋ॑तावरि प्रि॒या दे॒वेषु॒ जुह्व॑ति ॥ २९ ॥\\
पा॒व॒का न॒: सर॑स्वती॒ वाजे᳚भिर्वा॒जिनी᳚वती ।\\
य॒ज्ञं व॑ष्टु धि॒याव॑सुः ॥ ३० ॥\\
चो॒द॒यि॒त्री सू॒नृता᳚नां॒ चेत᳚न्ती सुमती॒नाम् ।\\
य॒ज्ञं द॑धे॒ सर॑स्वती ॥ ३१ ॥\\
म॒हो अर्ण॒: सर॑स्वती॒ प्र चे᳚तयति के॒तुना᳚ ।\\
धियो॒ विश्वा॒ वि रा᳚जति ॥ ३२ ॥\\
सर॑स्वतीं देव॒यन्तो᳚ हवन्ते॒ सर॑स्वतीमध्व॒रे ता॒यमा᳚ने ।\\
सर॑स्वतीं सु॒कृतो᳚ अह्वयन्त॒ सर॑स्वती दा॒शुषे॒ वार्यं᳚ दात् ॥ ३३ ॥\\
सर॑स्वति॒ या स॒रथं᳚ य॒याथ॑ स्व॒धाभि॑र्देवि पि॒तृभि॒र्मद᳚न्ती ।\\
आ॒सद्या॒स्मिन्ब॒र्हिषि॑ मादयस्वानमी॒वा इष॒ आ धे᳚ह्य॒स्मे ॥ ३४ ॥\\
सर॑स्वतीं॒ यां पि॒तरो॒ हव᳚न्ते दक्षि॒णा य॒ज्ञम॑भि॒नक्ष॑माणाः ।\\
स॒ह॒स्रा॒र्घमि॒लो अत्र॑ भा॒गं रा॒यस्पोषं॒ यज॑मानेषु धेहि ॥ ३५ ॥\\
आ नो᳚ दि॒वो बृ॑ह॒तः पर्व॑ता॒दा सर॑स्वती यज॒ता ग᳚न्तु य॒ज्ञम् ।\\
हवं᳚ दे॒वी जु॑जुषा॒णा घृ॒ताची᳚ श॒ग्मां नो॒ वाच॑मुश॒ती शृ॑णोतु ॥ ३६ ॥\\
रा॒काम॒हं सु॒हवां᳚ सुष्टु॒ती हु॑वे शृ॒णोतु॑ नः सु॒भगा॒ बोध॑तु॒ त्मना᳚ ।\\
सीव्य॒त्वप॑: सू॒च्याच्छि॑द्यमानया॒ ददा᳚तु वी॒रं श॒तदा᳚यमु॒क्थ्यम्᳚ ॥ ३७ ॥\\
यास्ते᳚ राके सुम॒तय॑: सु॒पेश॑सो॒ याभि॒र्ददा᳚सि दा॒शुषे॒ वसू᳚नि ।\\
ताभि᳚र्नो अ॒द्य सु॒मना᳚ उ॒पाग॑हि सहस्रपो॒षं सु॑भगे॒ ररा᳚णा ॥ ३८ ॥\\
सिनी᳚वालि॒ पृथु॑ष्टुके॒ या दे॒वाना॒मसि॒ स्वसा᳚ ।\\
जु॒षस्व॑ ह॒व्यमाहु॑तं प्र॒जां दे᳚वि दिदिड्ढि नः ॥ ३९ ॥\\
या सु॑बा॒हुः स्व᳚ङ्गु॒रिः सु॒षूमा᳚ बहु॒सूव॑री ।\\
तस्यै᳚ वि॒श्पत्न्यै᳚ ह॒विः सि॑नीवा॒ल्यै जु॑होतन ॥ ४० ॥\\
या गु॒ङ्गूर्या सि॑नीवा॒ली या रा॒का या सर॑स्वती ।\\
इ॒न्द्रा॒णीम॑ह्व ऊ॒तये᳚ वरुणा॒नीं स्व॒स्तये᳚ ॥ ४१ ॥\\
ओं शान्ति॒: शान्ति॒: शान्ति॑: ॥\\
