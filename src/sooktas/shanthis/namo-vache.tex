नमो॑ वा॒चे याचो॑ दि॒ता या चानु॑ दिता॒\\
तस्यै॑ वा॒चे नमो॒ नमो॑ वा॒चे नमो॑ वा॒चस् \\
पत॑ये॒ नम॒ ऋषि॑भ्यो मन्त्र॒ कृद्भ्यो॒\\
मन्त्र॑ पतिभ्यो॒ मामा मृष॑यो मन्त्र॒ कृतो॑ मन्त्र॒ पत॑यः॒\\
परा॑ दु॒र्माऽह मृषी᳚न् मन्त्र॒ कृतो॑\\
मन्त्र॒ पती॒न् परा॑दांँ वैश्व दे॒वींँ वाच॑  \\
मुद्यासꣳ शि॒वा मद॑स् ता॒ञ्जुष् \\
टां᳚ दे॒वेभ्यः॒ शर्म॑ मे॒ द्यौः शर्म॑ पृथि॒वी शर्म॒ विश्व॑ मि॒दं जग॑त् ।\\
शर्म॑ च॒न्द्रश्च॒ सूर्य॑श्च॒ शर्म॑ ब्रह्म प्रजाप॒ती ।\\
भू॒तंँ व॑दिष्ये॒ भुव॑नंँ वदिष्ये॒ तेजो॑ वदिष्ये॒ यशो॑ वदिष्ये॒ तपो॑ वदिष्ये॒\\
ब्रह्म॑ वदिष्ये स॒त्यंँ व॑दिष्ये॒ तस्मा॑ अ॒हमि॒द-मु॑प॒स्तर॑ण॒-मुप॑स्तृण उप॒स् \\
तर॑णं मे प्र॒जायै॑ पशू॒नां भू॑या दुप॒स्तर॑ण \\
म॒हं प्र॒जायै॑ पशू॒नां भू॑या सं॒प्राणा॑ पानौ \\
मृ॒त्योर् मा॑पातं॒ प्राणा॑ पानौ॒ मा मा॑ हा \\
सिष्टं॒ मधु॑ मनिष्ये॒ मधु॑ जनिष्ये॒ मधु॑ वक्ष्यामि॒ मधु॑ \\
वदिष्यामि॒ मधु॑मतीं दे॒वेभ्यो॒ वाच॑मुद्यासꣳ शुश्रू॒षेण्यां᳚ \\
मनु॒ष्ये᳚भ्य॒स्तं मा॑ दे॒वा अ॑वन्तु शो॒भायै॑ पि॒तरोऽनु॑मदन्तु ॥ \\
ओं शान्तिः॒ शान्तिः॒ शान्तिः॑ ॥\\