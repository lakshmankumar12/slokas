\section{\eng{Durga Sooktam}}
जा॒तवे॑दसे सुनवा म॒सोम॑मराती य॒तो निद॑हाति॒ वेदः॑ ।\\
स नः॑ पर्-ष॒दति॑ दु॒र्गाणि॒ विश्वा॑ ना॒वेव॒ सिन्धुं॑ दुरि॒ताऽत्य॒ग्निः ॥ (1)\\
\\
ताम॒ग्निव॑र्णां॒ तप॑सा ज्वल॒न्तीं-वैँ॑रोच॒नीं क॑र्मफ॒लेषु॒ जुष्टा᳚म् ।\\
दु॒र्गां दे॒वीग्ं शर॑णम॒हं प्रप॑द्ये सु॒तर॑ सितर से॒ नमः॑ ॥ (2)\\
\\
अग्ने॒ त्वं पा॑रया॒ नव्यो॑ अ॒स्मान्​थ् स्व॒स्ति भि॒रति॑ दु॒र्गाणि॒ विश्वा᳚ ।\\
पूश्च॑ पृ॒थ्वी ब॑हु॒ला न॑ उ॒र्वी भवा॑ तो॒काय॒ तन॑याय॒ शं​योँः ॥ (3)\\
\\
विश्वा॑ निनो दु॒र्गहा॑ जातवे दः॒सिन्धु॒न्न ना॒वा दु॑रि॒ताऽति॑ पर्-षि ।\\
अग्ने॑ अत्रि॒वन् मन॑सा गृणा॒नो᳚ऽ स्माकं॑ बोध्य वि॒ता त॒नूना᳚म् ॥ (4)\\
\\
पृ॒त॒ना॒ जित॒ग्ं॒ सह॑मान मु॒ग्र म॒ग्निग्ं हु॑वेम पर॒माथ् स॒धस्था᳚त् ।\\
स नः॑ पर्-ष॒दति॑ दु॒र्गाणि॒ विश्वा॒ क्षाम॑द् दे॒वो अति॑ दुरि॒ताऽत् य॒ग्निः ॥ (5)\\
\\
प्र॒त्नोषि॑ क॒मीड्यो॑ अध्व॒रेषु॑ स॒नाच्च॒हो ता॒नव्य॑श्च॒ सत्सि॑ ।\\
स्वाञ्चा᳚ऽग्ने त॒नुवं॑ पि॒प्रय॑स् वा॒स् मभ्यं॑ च॒सौ भ॑ग॒माय॑ जस्व ॥ (6)\\
\\
गोभि॒र्जुष्ट॑ मयुजो॒ निषि॑क् तं॒तवे᳚न्द्र विष्णो॒ रनु॒सञ्च॑रेम ।\\
नाक॑स्य पृ॒ष्ठ म॒भि सं॒वँसा॑ नो॒वैष्ण॑वीं-लोँ॒क इ॒ह मा॑दयन्ताम् ॥ (7)\\
\\
ॐ का॒त्या॒य॒नाय॑ वि॒द्महे॑ कन्यकु॒मारि॑ धीमहि । \\
तन्नो॑ दुर्गिः प्रचो॒दया᳚त् ॥\\
ॐ शान्तिः॒ शान्तिः॒ शान्तिः॑ ॥\\
