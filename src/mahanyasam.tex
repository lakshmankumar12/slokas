\documentclass[12pt]{article}
\title{Mahanyasam}
\author{Ancient Vedic Text}
\date{\today}
% you can set margin=0cm if you absolutely want no margin
\usepackage[a4paper, margin=1cm, right=0.5cm]{geometry}
\usepackage{fontspec}
\usepackage{setspace}
\usepackage{hyperref}
\usepackage{bookmark}
\usepackage{longtable}

% Remove headers and footers
\pagestyle{empty}

% Set Devanagari as main font
\setmainfont{Noto Sans Devanagari}[Script=Devanagari]
\newfontfamily\englishfont{Noto Sans}
\newfontfamily\tamilfont{Noto Sans Tamil}[Script=Tamil]
\newfontfamily\symbolfont{Noto Sans Symbols}
\newcommand{\eng}[1]{{\englishfont#1}}
\newcommand{\tamil}[1]{{\tamilfont#1}}
\newcommand{\symbol}[1]{{\symbolfont#1}}

\let\oldlongtable\longtable
\let\endoldlongtable\endlongtable
\renewenvironment{longtable}
  {\fontsize{20}{30}\selectfont\oldlongtable}
  {\endoldlongtable}

\begin{document}
% The first number (20) is the font size in points
% The second number (30) is the baseline skip - the distance between lines, also in points
\fontsize{20}{30}\selectfont

\section{\eng{Mahanyasam}}
\subsection{\eng{Panchanga Rudra Nyasaha}}
ओं नमो भगवते॑ रुद्रा॒य ॥ अथातः पंचाग रुद्राणां\\
न्यास पूर्वकं जपहोमा र्च नाभिषेक विधिं वया᳚ख्या॒स्यामः ॥\\
\\
ओं  ओंकार मन्त्र संयुक्तं नित्यं ध्यायन्ति योगिनः।\\
कामदं मोक्षदं तस्मै ओं नकाराय नमोनमः॥\\
\\
ओं भूर्भुवस्सुवः॑॥ ओं नं नम॑स्ते रुद्र म॒न्यव॑ उ॒तोत॒ इष॑वे॒ नम॑: ।\\
नम॑स्ते अस्तु॒ धन्व॑ने बा॒हुभ्या॑मु॒त ते॒ नम॑: ।\\
{\small\eng{(alternate)} या त॒ इषुः॑ शि॒वत॑मा शि॒वम् ब॒भूव॑ ते॒ धनुः॑।\\
शि॒वा श॑र॒व्या॑ या तव॒ तया॑ नो रुद्र मृडय॥}\\
{\small\eng{(East)}}\\
ओं कं खं गं घं डं । {\small यरलव शष सहों}\\
ओं नमो भगवते॑ रुद्रा॒य ॥ नं ओं - पूर्वाङ्ग रुद्राय नमः॥ 1 ॥\\
\\
महादेवं महात्मानं महा पातक नाशनम्।\\
महा पाप हरं वन्दे मकाराय नमोनमः॥\\
\\
ओं भूर्भुवस्सुवः॑॥ ओं मं \\
निध॑नपतये॒ नमः । निध॑नपतान्तिकाय॒ नमः ।\\
ऊर्ध्वाय॒ नमः । ऊर्ध्वलिङ्गाय॒ नमः ।\\
हिरण्याय॒ नमः । हिरण्यलिङ्गाय॒ नमः ।\\
सुवर्णाय॒ नमः । सुवर्णलिङ्गाय॒ नमः ।\\
दिव्याय॒ नमः । दिव्यलिङ्गाय॒ नमः ।\\
भवाय॒ नमः । भवलिङ्गाय॒ नमः ।\\
शर्वाय॒ नमः । शर्वलिङ्गाय॒ नमः ।\\
शिवाय॒ नमः । शिवलिङ्गाय॒ नमः ।\\
ज्वलाय॒ नमः । ज्वललिङ्गाय॒ नमः ।\\
आत्माय॒ नमः । आत्मलिङ्गाय॒ नमः ।\\
परमाय॒ नमः । परमलिङ्गाय॒ नमः ।\\
एतत्  सोमस्य॑ सूर् यस्य॒ सर्व\\
लिङ्ग॑ꣳस्था प॒य॒ति॒ पाणि मन्त्रं॑ पवि॒त्रम् ॥\\
\\
{\small\eng{(South)}}\\
ओं छं जं झं ञं नं । {\small यरलव शष सहों}\\
ओं नमो भगवते॑ रुद्रा॒य ॥ मं ओं - दक्षिणाङ्ग रुद्राय नमः॥ 2 ॥\\
\\
शिवं शान्तं जगन्नाधं लोकानुग्रह कारणम्।\\
शिवमेकं परं वन्दे शिकाराय नमोनमः॥\\
\\
ओं भूर्भुवस्सुवः॑॥ ओं शिं \\
अपै॑तु मृ॒त्यु र॒मृतं॑ न॒ आग॑न् वैवस् व॒तो नो॒ अभ॒यं कृणोतु।\\
प॒र्णं वन॒स्पते॑ रिवा॒भिन॑श् शीयताग्ं र॒यिस् सच॑तां न॒श्श ची पतिः॑॥\\
\\
{\small\eng{(West)}}\\
ओं  टं ठं डं ढं णं । {\small यरलव शष सहों}\\
ओं नमो भगवते॑ रुद्रा॒य ॥ शिं ओं - पश्चिमाङ्ग रुद्राय नमः॥ 3 ॥ \\
\\
वाहनं वृषभो यस्य वासुकिः कण्ठ भूषणम् ।\\
वामे शक्तिधरं वन्दे वकाराय नमोनमः॥\\
\\
ओं भूर्भुवस्सुवः॑॥ ओं वां \\
प्राणानां ग्रन्थिरसि रुद्रो मा॑ विशा॒न्तकः ।\\
तेनान्नेना᳚प्याय॒स्व ।\\
\\
{\small\eng{(alternate)} यो रु॒द्रो अ॒ग्नौ यो अ॒प्सु य ओष॑धीषु॒ यो
रु॒द्रो विश्वा॒ भुव॑नाऽऽवि॒वेश॒ तस्मै॑ रु॒द्राय॒ नमो॑ अस्तु॥\\
}
\\
{\small\eng{(North)}}\\
ओं  तं थं दं घं नं । {\small यरलव शष सहों}\\
ओं नमो भगवते॑ रुद्रा॒य ॥ वां औं - उत्तराङ्ग रुद्राय नमः॥ 4॥\\
\\
यत्र कुत्र स्थितं देवं सर्व व्यापिन मीश्वरम्।\\
यल्लिङ्गं पूज येन्नित्यं यकाराय नमोनमः॥।\\
\\
ओं भूर्भुवस्सुवः॑॥ ओं यं\\
यो रु॒द्रो अ॒ग्नौ यो अ॒प्सु य ओष॑धीषु॒ यो रु॒द्रो\\
विश्वा॒ भुव॑ना वि॒वेश॒ तस्मै॑ रु॒द्राय॒ नमो॑ अस्तु ।\\
\\
{\small\eng{(Heavenwards)}}\\
ओं पं फं बं भं मं । {\small यरलव शष सहों}\\
ओं नमो भगवते॑ रुद्रा॒य ॥ यं ओं - ऊर्ध्वाङ्ग रुद्राय नमः ॥ 5 ॥\\
\subsection{\eng{Panchamuka Nyasam}}
तत्पुरु॑षाय वि॒द्महे॑ महादे॒वाय॑ धीमहि । \\
तन्नो॑ रुद्रः प्रचो॒दया᳚त् ॥\\
\\
संवर्ताग्नि तटित्प्रदीप्त कनक प्रस्पर्दि तेजोमयं\\
गम्भीरध्वनि  मिश्रितोग्र दहन प्रोद् भा सिता म्राधरम्\\
अर्धेन् दुद् युति लोल पिङ्गळ जटाभार प्रबद्धोरगं\\
वन्दे सिद्ध सुरासुरेन्द्र नमितं पूर्वं मुखंशूलिनः \\
{\small
संवर्ताग्नि तटित्प्रदीप्त कनक प्रस्पर्धि तेजोरुणं\\
गम्भीरध्वनि \textbf{सामवेद जनकं ताम राधरं सुन्दरम्}।\\
अर्धेनदुद्युति लोल पिंगल जटा भार \textbf{प्रबोद्धोदकं} \\
वन्दे सिद्ध सुरा सुरेन्द्र नमितं पूर्वं मुखं शूलिनः॥\\
}
\\
ओं अं कं खं गं घं डं । आं ओं ओं नमो भगवते॑ रुद्रा॒य ॥\\
ओं नं - पूर्व मुखाय नमः॥\\
अ॒घोरे᳚भ्योऽथ॒ घोरे᳚भ्यो॒ घोर॒घोर॑तरेभ्यः । \\
स॒र्वे᳚तः॑ सर्व॒ शर्वे᳚भ्यो॒ नम॑स्ते अस्तु रु॒द्र रू॑पेभ्यः ॥\\
\\
कालाभ्र भ्रमराञ्ज नद्युतिनिभं व्यावृत्त पिङ्गेक्षणं\\
कर्णोद्भासित भोगि मस्तक मणिं प्रोद्भिन्न दंष्ट्राङ्कुरम् ।\\
सर्प प्रोतक पाल शुक्ति शकल व्याकीर्ण सञ्चारगं\\
वन्दे दक्षिण मीश्व रस्य कुटिल भ्रूभङ्गरौद्रं मुखम् \\
{\small
सर्पप् रोतक पाल शुक्ति शकल व्याकीर्ण \textbf{ताशेखरं}\\
वन्दे दक्षिण मीश्व रस्य \textbf{दक्षिण वदनं चाथर्वनादोदयम्}॥\\ 
}
\\
ओं इं छं जं झं ञं नं । ईं ओं ओं नमो भगवते॑ रुद्रा॒य ॥\\
ओं मं  - दक्षिण मुखाय नमः॥\\
\\
स॒द्योजा॒तं प्र॑पद्या॒मि॒ स॒द्योजा॒ताय॒ वै नमो॒ नमः॑ ।\\
भ॒वे भ॑वे॒ नाति॑भवे भवस्व॒ माम् । भ॒वोद्भ॑वाय॒ नमः ॥\\
\\
प्रालेयाचल चन्द्र कुन्द धवलं गोक्षीर फेन प्रभं\\
भस्माभ् यक्त मनङ्ग देह दहन ज्वाला वली लोचनम् ।\\
ब्रह्मेन्द्रादि मरुद्गणैः स्तुतिपरै रभ्यर्चितं योगि भिः\\
वन्देऽहं सकलं कलङ्क रहितं स्थाणोर्मुखं पश्चिमम् ॥\\
{\small
प्रालेयाचल \textbf{मिन्दुकुन्द} धवलं गोक्षीर फेनप्रभं\\
\textbf{भस्माभ् यंग} मनंग देह दहन ज्वाला वलीलोचनम्\\
\textbf{विष्णु ब्रह्म मरुद् गणार्चित पदं ऋग्वेद नादोदयं}\\
}
\\
ओं उं टं ठं डं ढं णं । ऊं ओं ओं नमो भगवते॑ रुद्रा॒य ॥\\
ओं  शिं - पश्चिम मुखाय नमः॥\\
\\
वा॒म॒दे॒वाय॒ नमो᳚ ज्ये॒ष्ठाय॒ नमः॑ श्रे॒ष्ठाय॒ नमो॑ रु॒द्राय॒ नमः॒ \\
काला॑य नमः॒ कल॑ विकरणाय॒ नमो॒ \\
बल॑ विकरणाय॒ नमो॒\\
बला॑य॒ नमो॒ बल॑प्रमथनाय॒ नमः॒ \\
सर्व॑ भूत दमनाय॒ नमो॑ म॒नोन्म॑नाय॒ नमः॒ ॥\\
\\
गौरं कुङ्कुम पङ्किलं सुतिलकं व्यापाण्डु गण्डस्थलं\\
भ्रूविक्षेप कटाक्ष वीक्षण लसत् संसक्त कर्णोत्पलम् ।\\
स्निग्धं बिम्ब फलाधर प्रहसितं नीलाल कालङ्कृतं\\
वन्दे पूर्ण शशाङ्क मण्डल निभं वक्त्रं हरस्योत्तरम् ॥\\
{\small
वन्दे \textbf{याजुष वेद घोष जनकं} वक्त्रं हरस्योत्तरम्॥\\
}
\\
ओं एं तं थं दं घं नं । ऐं ओं ओं नमो भगवते॑ रुद्रा॒य ॥\\
ओं वां - उत्तर मुखाय नमः॥\\
\\
ईशानः सर्व॑ विद्या॒ना॒ मीश्वरः सर्व॑भूता॒नां॒\\
ब्रह्माधि॑पति॒र्ब्रह्म॒णोऽधि॑पति॒ \\
र्ब्रह्मा॑ शि॒वो मे॑ अस्तु सदाशि॒वोम् ॥\\
\\
व्यक्ता व्यक्त गुणेतरं सुविमलं षट्त्रिं शतत् त्वात्मकं\\
तस्मा दुत्तर तत्त्व मक्षरमिति ध्येयं सदा योगिभिः ।\\
वन्दे तामस वर्जितं त्रिणयनं सूक्ष्मा तिसूक्ष्मात्परं\\
शान्तं पञ्चम मीश्वरस्य वदनं खव्यापि तेजोमयम् ॥\\
{\small
व्यक्ताव्यक्त \textbf{निरूपितं च परमं} षट्त्रिं \textbf{शतत्वाधिकं}\\
तस्मादुत्तर तत्व मक्षरमिति ध्येयं सदा योगिभिः।\\
\textbf{ओंकारादि समस्त मन्त्र जनकं} सूक्ष्मा तिसूक्ष्मं परं\\
\textbf{वन्दे} पंचम मीश्वरस्य वदनं खव्यापि तेजोमयम्॥\\
}
\\
ओं ओं पं फं बं भं मं । औं ओं ओं नमो भगवते॑ रुद्रा॒य ॥\\
ओं यं - ऊर्ध्व मुखाय नमः॥\\
\\
पूर्वे पशुपतिः पातु दक्षिणे पातु शंकरः।\\
पश्चिमे पातु विश्वेशो नीलकण्ठस्तथोत्तरे॥\\
\\
ऐशान्यां पातुमां शर्वो ह्याग् नेय्यां पार्वती पतिः।\\
नैर्ऋर्त्यां पातुमां रुद्रो वायव्यां नीललोहितः॥\\
ऊर्ध्वे त्रिलोचनः पात् अधरायां महेश्वरः।\\
एताभ्यो दश दिग् भ्यस्तु सर्वतः पातु शंकरः॥\\
\subsection{\eng{Keshadhi Padhanta nyasaha}}
ओं या ते॑ रुद्र शि॒वा त॒नूरघो॒राऽपा॑पकाशिनी ।\\
तया॑ नस्त॒नुवा॒ शन्त॑मया॒ गिरि॑शन्ता॒भिचा॑कशीहि ॥ शिखायै नमः॥ (1)\\
{\small\eng{Tuft}}\\
\\
अ॒स्मिन्म॑ह॒त्य॑र्ण॒वे᳚ऽन्तरि॑क्षे भ॒वा अधि॑ ।\\
तेषाग्ं॑ सहस्रयोज॒नेऽव॒धन्वा॑नि तन्मसि ॥ शिरसे नमः॥ (2)\\
{\small\eng{Top o\eng{f} Head}}\\
\\
स॒हस्रा॑णि सहस्र॒शो ये रु॒द्रा अधि॒ भूम्या᳚म् ।\\
तेषाग्ं॑ सहस्रयोज॒नेऽव॒धन्वा॑नि तन्मसि ॥ ललाटाय नमः॥ (3)\\
{\small\eng{Forehead}}\\
\\
ह॒ग्ं॒ सश् शुचि॒ षद् वसुरन् तरिक्ष॒सद् धोता वेदि॒ष दति थिर् दरो ण॒सत्।\\
नृ॒षद् वर॒सद्रु तसब् यो मसदब् जा गोजा, ऋतजा, अद्रिजा, ऋतं बृहत्॥ \\
भ्रुवोर्मध्याय नमः॥ (4)\\
{\small\eng{Middle o\eng{f} Eyebrows}}\\
\\
त्र्य॑म्बकं यजामहे सुग॒न्धिं पु॑ष्टि॒वर्ध॑नम् ।\\
उ॒र्वा॒रु॒कमि॑व॒ बन्ध॑नान्मृ॒त्योर्मु॑क्षीय॒ माऽमृता᳚त् ॥  नेत्राभ्यां नमः॥ (5)\\
{\small\eng{Eyes}}\\
\\
नम॒: स्रुत्या॑य च॒ पथ्या॑य च॒    नम॑: का॒ट्या॑य च नी॒प्या॑य च॒ ॥ \\
कर्णाभ्यां नमः॥ (6)\\
{\small\eng{Ears}}\\
{\small नमः॒ स्रुत्या॑य च॒ पथ्या॑य च॒ नमः॑ का॒ट्या॑य च नी॒प्या॑य च॒ नमः॒\\
सूद्या॑य च सर॒स्या॑य च॒ नमो॑ ना॒द्याय॑ च वैश॒न्ताय॑ च॥ कर्णाभ्यां नमः॥}\\
\\
मा न॑स्तो॒के तन॑ये॒ मा न॒ आयु॑षि॒ मा नो॒ गोषु॒ मा नो॒ अश्वे॑षु रीरिषः ।\\
वी॒रान्मा नो॑ रुद्र भामि॒तोऽव॑धीर्ह॒विष्म॑न्तो॒  नम॑सा विधेम ते ॥  \\
नासिकायै नमः॥ (7)\\
{\small\eng{Nose}}\\
\\
अ॒व॒तत्य॒ धनु॒स्तवग्ं सह॑स्राक्ष॒ शते॑षुधे ।\\
नि॒शीर्य॑ श॒ल्यानां॒ मुखा॑ शि॒वो न॑: सु॒मना॑ भव ॥ मुखाय नमः॥ (8)\\
{\small\eng{Face}}\\
\\
नील॑ग्रीवाः शिति॒कण्ठा᳚: श॒र्वा अ॒धः क्ष॑माच॒राः ।\\
तेषाग्ं॑ सहस्रयोज॒नेऽव॒धन्वा॑नि तन्मसि ॥ कण्ठाय नमः॥ (9)\\
{\small\eng{Throat}}\\
\\
नील॑ग्रीवाः शिति॒कण्ठा॒ दिवग्ं॑ रु॒द्रा उप॑श्रिताः ।\\
तेषाग्ं॑ सहस्रयोज॒नेऽव॒धन्वा॑नि तन्मसि ॥ उपकण्ठाय नमः। (10)\\
{\small\eng{Lower Neck}}\\
\\
नम॑स्ते अ॒स्त्वायु॑धा॒याना॑तताय धृ॒ष्णवे᳚ ।\\
उ॒भाभ्या॑मु॒त ते॒ नमो॑ बा॒हुभ्यां॒ तव॒ धन्व॑ने ॥ बाहुभ्यां नमः। (11)\\
{\small\eng{Shoulders}}\\
\\
या ते॑ हे॒तिर्मी॑ढुष्टम॒ हस्ते॑ ब॒भूव॑ ते॒ धनु॑: ।\\
तया॒ऽस्मान् वि॒श्वत॒स्त्वम॑य॒क्ष्मया॒ परि॑ब्भुज ॥ उपबाहुभ्यां नमः॥ (12)\\
{\small\eng{Elbow to Wrist}}\\
\\
{\small\eng{\textbf{Not in challakere rendition}}}\\
परि॑णो रु॒द्रस्य॑ हे॒तिर्वृ॑णक्तु॒ परि॑ त्वे॒षस्य॑ दुर्म॒तिर॑घा॒योः।\\
अव॑ स्थि॒रा म॒घव॑द्भ्यस्तनुष्व॒ मीढ्व॑स्तो॒काय॒ तन॑याय मृडय॥\\
मणिबन्धाभ्यां नमः॥ (13)\\
{\small\eng{Wrists}}\\
\\
ये ती॒र्थानि॑ प्र॒चर॑न्ति सृ॒काव॑न्तो निष॒ङ्गिण॑: ।\\
तेषाग्ं॑ सहस्रयोज॒नेऽव॒धन्वा॑नि तन्मसि ॥ हस्ताभ्यां नमः॥ (14)\\
{\small\eng{Hands}}\\
\\
स॒द्योजा॒तं प्र॑पद्या॒मि॒ स॒द्योजा॒ताय॒ वै नमो॒ नमः॑ ।\\
भ॒वे भ॑वे॒ नाति॑भवे भवस्व॒ माम् । भ॒वोद्भ॑वाय॒ नमः ॥ अङ्गुष्ठाभ्यां नमः॥ (15)\\
{\small\eng{Roll Ring Fingers on Thumb o\eng{f} each hand}}\\
\\
वा॒म॒दे॒वाय॒ नमो᳚ ज्ये॒ष्ठाय॒ नमः॑ श्रे॒ष्ठाय॒ नमो॑ रु॒द्राय॒ नमः॒ \\
काला॑य नमः॒ कल॑ विकरणाय॒ नमो॒ बल॑ विकरणाय॒ नमो॒\\
बला॑य॒ नमो॒ बल॑प्रमथनाय॒ नमः॒ सर्व॑ भूत दमनाय॒ \\
नमो॑ म॒नोन्म॑नाय॒ नमः॒ ॥ तर्जनीभ्यां नमः॥ (16)\\
{\small\eng{Roll Thumb on Ring Fingers o\eng{f} both Hands}}\\
\\
अ॒घोरे᳚भ्योऽथ॒ घोरे᳚भ्यो॒ घोर॒घोर॑तरेभ्यः । \\
स॒र्वे᳚तः॑ सर्व॒ शर्वे᳚भ्यो॒ नम॑स्ते अस्तु रु॒द्र रू॑पेभ्यः ॥ मध्यमाभ्यां नमः॥ (17)\\
{\small\eng{Roll Thumb on Middle Fingers o\eng{f} both Hands}}\\
\\
तत्पुरु॑षाय वि॒द्महे॑ महादे॒वाय॑ धीमहि । \\
तन्नो॑ रुद्रः प्रचो॒दया᳚त् ॥ अनामिकाभ्यां नमः॥ (18)\\
{\small\eng{Roll Thumb on Ring Fingers o\eng{f} both Hands}}\\
\\
ईशानः सर्व॑ विद्या॒ना॒ मीश्वरः सर्व॑भूता॒नां॒\\
ब्रह्माधि॑पति॒र्ब्रह्म॒णोऽधि॑पति॒ र्ब्रह्मा॑ शि॒वो मे॑ अस्तु सदाशि॒वोम् ॥\\
कनिष्ठिकाभ्यां नमः॥ (19)\\
{\small\eng{Roll Thumb on Little Fingers o\eng{f} both Hands}}\\
\\
{\small\eng{\textbf{Not in challakere rendition}}}\\
नमो हिरण्यबाहवे हिरण्यवर्णाय हिरण्यरूपाय \\
हिरण्यपतयेऽम्बिकापतय उमापतये\\
पशुपतये॑ नमो॒ नमः॥ करतल करपृष्ठाभ्यां नमः॥ (20)\\
{\small\eng{Rub each palm over other, front and back}}\\
\\
नमो॑ वः किरि॒केभ्यो॑ दे॒वाना॒ग्ं॒ हृद॑येभ्यः । हृदयाय नमः॥ (21)\\
{\small\eng{Touch Heart}}\\
\\
नमो॑ ग॒णेभ्यो॑ ग॒णप॑तिभ्यश्च वो॒ नमः॑ ॥ पृष्ठाय नमः॥ (22)\\
{\small\eng{Touch Back}}\\
\\
नम॒स्तक्ष॑भ्यो रथका॒रेभ्य॑श्च वो॒  नमः॑ ॥ कक्षाभ्यां नमः॥ (21)\\
{\small\eng{Armpit to Waist}}\\
\\
नमो॒ हिर॑ण्यबाहवे सेना॒न्ये॑ दि॒शां च॒ पत॑ये॒  नमः॑ ॥ पार्श्वाभ्यां नमः॥ (22)\\
{\small\eng{Trunk}}\\
\\
विज्यं॒ धनु॑: कप॒र्दिनो॒ विश॑ल्यो॒ बाण॑वाग्ं उ॒त ।\\
अने॑शन्न॒स्येष॑व आ॒भुर॑स्य निष॒ङ्गथि॑: ॥ जठराय नमः॥ (23)\\
{\small\eng{Stomach}}\\
\\
हि॒र॒ण्य॒ग॒र्भः सम॑वर्त॒ताग्रे॑ भू॒तस्य॑ जा॒तः पति॒रेक॑ आसीत्।\\
स दा॑धार पृथि॒वीं द्यामु॒तेमां कस्मै॑ दे॒वाय॑ ह॒विषा॑ विधेम॥ \\
नाभ्यै नमः। (24)	\\
{\small\eng{Navel}}\\
\\
मीढु॑ष्टम॒ शिव॑तम शि॒वो न॑: सु॒मना॑ भव । प॒र॒मे वृ॒क्ष \\
आयु॑धन्नि॒धाय॒ कृत्तिं॒ वसा॑न॒ आच॑र॒ पिना॑कं॒ बिभ्र॒दाग॑हि ॥ \\
कट्यै नमः॥ (25)\\
{\small\eng{Waist}}\\
    \\
ये भू॒ताना॒मधि॑पतयो विशि॒खास॑: कप॒र्दिन॑: ॥\\
तेषाग्ं॑ सहस्रयोज॒नेऽव॒धन्वा॑नि तन्मसि ॥ गुह्याय नमः॥ (26)\\
{\small\eng{Upper Reproductive Organs}}\\
\\
ये अन्ने॑षु वि॒विध्य॑न्ति॒ पात्रे॑षु॒ पिब॑तो॒ जनान्॑ ।\\
तेषाग्ं॑ सहस्रयोज॒नेऽव॒धन्वा॑नि तन्मसि ॥ अण्डाभ्यां नमः॥ (27)\\
{\small\eng{Lower Reproductive Organs}}\\
\\
स शि॑रा जा॒तवे॑दाः। अ॒क्षरं॑ पर॒मं प॒दं। वे॒दाना॒ꣳ॒ शिर॑ उत्त॒मम्।\\
जातवे॑दसे॒ शिर॑सि मा॒ता ब्रह्म॒ भूर्भुव॒स्सुवरोम्‌॥ अपानाय नमः॥ (28)\\
{\small\eng{Anus}}\\
{\small स शि॒रा जा॒तवे॑दा अ॒क्षरं॑ पर॒मं प॒दम्।\\
वेदा॑ना॒ꣳ॒ शिर॑सि मा॒ता॒ आ॒युष्मन्तं॑ करोतु॒ माम्॥ अपानाय नमः॥ }\\
\\
मा नो॑ म॒हान्त॑मु॒त मा नो॑ अर्भ॒कं मा न॒ उक्ष॑न्तमु॒त मा न॑ उक्षि॒तम् ।\\
मा नो॑ऽवधीः पि॒तरं॒ मोत मा॒तरं॑ प्रि॒या मा न॑स्त॒नुवो॑ रुद्र रीरिषः ॥\\
ऊरुभ्यां नमः॥ (29)\\
{\small\eng{Thighs}}\\
\\
एष ते रुद्र भागस्तं जुषस्व तेनाऽवसेन परो \\
मूर्जवतोऽ तीह्य वतत धन्वा पिनाक हस्तः कृत्तिवासाः॥ \\
जानुभ्यां नमः॥ (30)\\
{\small\eng{Knees}}\\
\\
स॒ꣳ॒सृ॒ष्ट॒ जित् सो॑ म॒पा बा॑हु श॒र्ध्यू᳚र्ध्व धन्वा॒ प्रति॑हिता भि॒रस्ता᳚ ।\\
बृह॑स्पते॒ परि॑दीया॒ रथे॑न रक्षो॒हाऽमित्राꣳ॑ अप॒बा ध॑मानः॥ \\
जङ्घाभ्यां नमः॥ (31)\\
{\small\eng{Knees to Ankles}}\\
\\
विश्वं॑ भू॒तं भुव॑नं चि॒त्रं ब॑हु॒धा जा॒तं जाय॑मानं च॒ यत्।\\
सर्वो॒ ह्ये॑ष रु॒द्रस्तस्मै॑ रु॒द्राय॒ नमो॑ अस्तु ॥ गुल्फाभ्यां नमः॥ (32)\\
{\small\eng{Ankles}}\\
\\
ये प॒थां प॑थि॒रक्ष॑य ऐलबृ॒दा य॒व्युधः॑।\\
तेषाꣳ॑ सहस्रऽयोज॒नेऽव॒धन्वा॑नि तन्मसि ॥ पादाभ्यां नमः॥ (33)\\
{\small\eng{Feet}}\\
\\
अध्य॑वोचदधिव॒क्ता प्र॑थ॒मो दैव्यो॑ भि॒षक्।\\
अहीग्॑श्च॒ सर्वा᳚ञ्ज॒म्भय॒न्थ्सर्वा᳚श्च यातुधा॒न्यः ॥ कवचाय नमः॥ (34)\\
{\small  कवचाय हुम्॥}\\
{\small\eng{Cross hands across chest touching shoulder}}\\
\\
नमो॑ बि॒ल्मिने॑ च कव॒चिने॑ च॒नमः॑ श्रु॒ताय॑ च श्रुतसे॒नाय॑ च ॥ \\
उपकवचाय नमः ॥ {\small उपकवचाय हुम्॥ } (34)\\
{\small\eng{kavacha at elbow level}}\\
\\
नमो॑ अस्तु॒ नील॑ग्रीवाय सहस्रा॒क्षाय॑ मी॒ढुषे᳚ ।\\
अथो ये अस्य सत्वांनोऽहं तेभ्योऽकरन्नमः॥ तृतीय नेत्राय नमः॥ (35)\\
{\small नेत्रत्रयाय वौषट्॥}\\
{\small\eng{Index/Middle/Ring at eyes/middle o\eng{f} eyebrows}}\\
\\
प्रमु॑ञ्च॒ धन्व॑न॒स्त्वमु॒भयो॒रार्त्रि॑यो॒र्ज्याम्। \\
याश्च॑ ते॒ हस्त॒ इष॑वः॒ परा॒ ता भ॑गवो वप ॥ अस्त्राय नमः ॥ (36)\\
{\small अस्त्राय फट्॥}\\
{\small\eng{Slap index/middle o\eng{f} right on left palm}}\\
\\
य ए॒ताव॑न्तश्च॒ भूयाꣳ॑ सश्च॒ दिशो॑ रु॒द्रा वि॑तस्थि॒रे।\\
तेषाꣳ॑ सहस्रऽयोज॒नेऽव॒धन्वा॑नि तन्मसि ॥ दिग्बन्धाय नमः॥ (37)\\
{\small  इति दिग्बन्धः॥}\\
{\small\eng{Snap middle/thumb with click sound across self}}\\
\\
ओं नमो भगवते॑ रुद्रा॒य ॥\\
\subsection{\eng{Dashanga Nyasaha}}
\\
आं मूर्ध्ने नमः ॥ नं नासिकायै नमः॒ ॥ मों ललाटाय नमः ॥\\
भं मुखाय नमः ॥ गं कण्ठाय नमः ॥  वं हृदयाय नमः॥\\
तें दक्षिण हस्ताय नमः ॥  रुं वाम हस्ताय नमः ॥\\
द्रां नाभ्यै नमः ॥ यं पादाभ्यां नमः॒॥\\
\\
\subsection{\eng{Panchanga nyasaha}}
\\
स॒द्योजा॒तं प्र॑पद्या॒मि॒ स॒द्योजा॒ताय॒ वै नमो॒ नमः॑ ।\\
भ॒वे भ॑वे॒ नाति॑भवे भवस्व॒ माम् । भ॒वोद्भ॑वाय॒ नमः । पादाभ्यां नमः ॥\\
\\
वा॒म॒दे॒वाय॒ नमो᳚ ज्ये॒ष्ठाय॒ नमः॑  श्रे॒ष्ठाय॒ नमो॑ रु॒द्राय॒ नमः॒ \\
काला॑य नमः॒ कल॑ विकरणाय॒ नमो॒  बल॑ विकरणाय॒ नमो॒\\
बला॑य॒ नमो॒ बल॑प्रमथनाय॒ नमः॒ \\
सर्व॑ भूत दमनाय॒ नमो॑ म॒नोन्म॑नाय॒ नमः॒ ॥ ऊरुमध्यमाभ्यां नमः ॥\\
\\
अ॒घोरे᳚भ्योऽथ॒ घोरे᳚भ्यो॒ घोर॒घोर॑तरेभ्यः । \\
स॒र्वे᳚तः॑ सर्व॒ शर्वे᳚भ्यो॒ नम॑स्ते अस्तु रु॒द्र रू॑पेभ्यः\\
हृदयाय नमः ॥\\
\\
तत्पुरु॑षाय वि॒द्महे॑ महादे॒वाय॑ धीमहि । \\
तन्नो॑ रुद्रः प्रचो॒दया᳚त्  ॥ मुखाय नमः ॥\\
\\
ईशानः सर्व॑ विद्या॒ना॒ मीश्वरः सर्व॑भूता॒नां॒\\
ब्रह्माधि॑पति॒र्ब्रह्म॒णोऽधि॑पति॒ र्ब्रह्मा॑ शि॒वो मे॑\\
अस्तु सदाशि॒वोम् ॥ {\small हंस हंसः} मूर्ध्ने नमः ॥\\

\subsection{\eng{Hamsa gayathri stotram}}
अस्य श्री हंस गायत्री स्तोत्र महामन्त्रस्य। अव्यक्त पर ब्रह्म ऋषिः ।\\
{\small अनुष्टुप् छन्दः।} अव्यक्त गायत्रि छन्दः। \\
परम हंसो देवता । हंसां बीजं। हंसीं शक्तिः।\\
हंसों कीलकं। परम हंस प्रसाद सिध्द्यर्थे जपे विनियोगः॥\\
\\
हंसां आङ्गुष्ठाभ्यां नमः ।\\
हंसीं तर्जनीभ्यां नमः ।\\
हंसूं मध्यमाभ्यां नमः ।\\
हंसैं अनामिकाभ्यां नमः ।\\
हंसौं कनिष्ठिकाभ्यां नमः ।\\
हंसः करतलकर पृष्ठाभ्यां नमः ॥\\
\\
हंसां हृदयाय नमः ।\\
हंसीं शिरसे स्वाहा।\\
हंसूं शिखायै वषट्।\\
हंसैं कवचायहम्।\\
हंसौं नेत्रत्रयाय वषट्।\\
हंसः अस्त्राय फट्।\\
\\
भूर्भुव॒ स्सुव॒रोमिति दिग्बन्धः ॥\\
\\
\subsubsection{\eng{Dhyanam}}
ध्यानं।\\
गमा गमस्थं गमनादि शून्यं चिद्रूपदीपं तिमिरापहारम्।\\
पश्यामि ते सर्वजनान्तरस्थं नमामि हंसं परमात्मरूपम्॥\\
देहो देवालयः प्रोक्तो जीवो देवस्सनातनः।\\
त्यजे दज्ञान निर्माल्यं सोऽहं भावेन पूजयेत्॥\\
\\
हंसो हंसः परम हंसः  \\
हंसस् सोऽहं सोऽहं हंसः॥\\
\\
हं॒स॒ हंसा॒य॑ वि॒द्महे॑ परमहंसा॒य॑ धीमही।\\
तन्नो॑ हंसः प्रचो॒दया᳚त्॥\\
\\
हंस हंसेति योब्रूयाध् दं॑सो ना॑म स॒दाशि॑वः।\\
एवं न्यास विधिं॒ कृत्वा ततस् संपुट मारभेत्॥\\

\subsection{\eng{Dik Samputa Nyasaha}}
ॐ भूर्भुव॒स्सुव॒रों ।\\
ॐ लं । त्रातारमिंन्द्र॑ मवि॒तार॒मिन्द्रꣳ हवे॑ हवे सु॒हव॒ꣳ॒ शूर॒मिन्द्रम्᳚ । \\
हु॒वे नु श॒क्रं पु॑रहू॒त मिन्द्रग्ग्॑ स्व॒स्ति नो॑ म॒घवा॑ धा॒त्विन्द्रः॑ ॥\\
\\
लं भर्व धिग्बागे, इन्द्राय वज्रर्हस्ताय देवाधिपतये, ऐरावत वाहनाय - \\
सांगाय सायुधाय सशक्तिस परिवाराय -  उमामहेश्वर पार्षदाय नमः।\\
लं इन्द्राय नमः ।  पूर्व दिग्भागे, इन्द्रः सुप्रीतो  वरदो भवतु ॥  (1)\\
{\small लं भूर्भुवस्सुवः इन्द्राय वज्रहस्ताय सुराधिपतय ऐरावतवाहनाय -\\
सांगाय सायुधाय सशक्ति परिवाराय - सर्वालंकार भूषिताय उमामहेश्वर पार्षदाय नमः।\\
पूर्व दिग्भागे ललाटस्थाने इन्द्रः सुप्रीतो सुप्रस्न्नो वरदो भवतु ॥}\\
ॐ भूर्भुव॒स्सुव॒रों ।\\
\\
रं । त्वन्नो॑, अग्ने॒ वरु॑णस्य वि॒द्वान् दे॒वस्य॒ हेडोऽव॑ यासि सीष्ठाः ।\\
यजि॑ष्ठो॒  वह्नि॑ तम॒श् शोशु॑ चानो॒ विश्वा॒, द्वेषाꣳ॑सि॒प् प्रमु॑ मुग् घ्य॒स् मत् ॥ \\
\\
रं आग् नेय धिग्बागे, अग्नये शक्ति हस्ताय तेजोऽधि पतयेऽ,\\
\hspace*{10cm} अज वाहनाय\\
सांगाय सायुधाय सशक्तिस परिवाराय - उमामहेश्वर पार्षदाय नमः।\\
रं अग्नये नमः । आग्नेय दिग्भागे अग्निः सुप्रीतो  वरदो भवतु ॥ (2)\\
{\small रं भूर्भुवस्सुवः अग्नये शक्ति हस्ताय तेजोऽधि पतयेऽ अज वाहनाय -\\
सांगाय सायुधाय सशक्ति परिवाराय - सर्वालंकार भूषिताय उमामहेश्वर पार्षदाय नमः ।\\
आग्नये दिग्भागे नेत्रस्थाने अग्निः सुप्रीतो सुप्रस्न्नो वरदो भवतु ॥}\\
ॐ भूर्भुव॒स्सुव॒रों ।\\
हं । सु॒गन्नः॒ पन्था॒ मभ॑यं कृणोतु । यस्मि॒न् नक्ष॑त्रे य॒म एति॒ राजा᳚ ।\\
यस्मि॑न् नेन म॒भ्य षिं॑ चन्त दे॒वाः । तद॑स्य चि॒त्रꣳ ह॒विषा॑ यजाम ॥\\
\\
हं दक्षिण धिग्बागे यमाय दण्ड हस्ताय धर्माधि पतये महिष वाहनाय\\
सांगाय सायुधाय सशक्तिस परिवाराय -  उमामहेश्वर पार्षदाय नमः।\\
हं यमाय नमः । दक्षिण दिग्भागे यमः सुप्रीतो  वरदो भवतु ॥ (3)\\
{\small हं भूर्भुवस्सुवः यमाय दण्ड हस्ताय धर्माधि पतये महिष वाहनाय -\\
सांगाय सायुधाय सशक्ति परिवाराय - सर्वालंकार भूषिताय उमामहेश्वर पार्षदाय नमः ।\\
दक्षिण दिग्भागे कर्णस्थाने यमः सुप्रीतो सुप्रस्न्नो वरदो भवतु॥}\\
ॐ भूर्भुव॒स्सुव॒रों ।\\
षं । असु॑न्वन्त मय॑जमान मिच् छस् ते॒ नस् ये॒त् यान् तस्क॑रस् यान् वे॑षि।\\
अ॒न्य म॒स्म दि॑च्छ॒ सात॑ इ॒त्या नमो॑ देवि-निर्ऋते॒ तुभ्य॑ मस्तु ॥\\
\\
षं निर्ऋति दिग्भागे निर्ऋतये खड्ग हस्ताय रक्षो धिपतये नर वाहनाय\\
सांगाय सायुधाय सशक्तिस परिवाराय -  उमामहेश्वर पार्षदाय नमः।\\
षं निर्ऋतये नमः । निर्ऋति दिग्भागे निर्ऋतिस्सुप्रीतो वरदो भवतु ॥ (4)\\
{\small षं भूर्भुवस्सुवः निऋतये खड्ग हस्ताय रक्षोधि पतये नर वाहनाय -\\
सांगाय सायुधाय सशक्ति परिवाराय - सर्वालंकार भूषिताय उमामहेश्वर पार्षदाय नमः ।\\
नैर्ऋतदिग्भागे मुखस्थाने निर्ऋतिस्सुप्रीतो सुप्रस्न्नो वरदो भवतु ॥}\\
ॐ भूर्भुव॒स्सुव॒रों ।\\
वं । तत्वा॑ यामि॒ ब्रह्म॑णा॒ वन्द॑ मा॒नस् तदा शा᳚स्ते॒ यज॑मानो ह॒विर्भिः॑ ।\\
अहे॑डमानो वरुणे॒ हबो॒ध् युरु॑शꣳ स॒मान॒ आयुः॒ प्रमो॑षीः ॥\\
\\
वं पश्चिम दिग्भागे वरुणाय पाश हस्ताय जलाधि पतये मकर वाहनाय\\
सांगाय सायुधाय सशक्तिस परिवाराय -  उमामहेश्वर पार्षदाय नमः।\\
वं वरुणाय नमः । पश्चिम दिग्भागे वरुणः सुप्रीतो वरदो भवतु ॥ (5)\\
{\small वं भूर्भुवस्सुवः वरुणाय पाशहस्ताय जलाधि पतये मकर वाहनाय -\\
सांगाय सायुधाय सशक्ति परिवाराय - सर्वालंकार भूषिताय उमामहेश्वर पार्षदाय नमः ।\\
पश्चिम दिग्भागे बाहुस्थाने वरुणः सुप्रीतो सुप्रस्न्नो वरदो भवतु ॥}\\
ॐ भूर्भुव॒स्सुव॒रों ।\\
यं । आ नो॑ नि॒युद् भि॑श् श॒तिनी॑ भिरध् व॒रम् । \\
स॒ह॒स् रिणी॑ भि॒रुप॑ याहि य॒ज्ञम् ।\\
वायो॑, अ॒स्मिन् ह॒विषि॑ मादयस्व । यू॒यं पा॑तस् स्व॒स्ति भि॒स् सदा॑ नः॥\\
\\
यं वायव्य दिग्भागे वायवे सांकु शध् वजहस्ताय प्राणाधिपतये मृगवाहनाय\\
सांगाय सायुधाय सशक्तिस परिवाराय -  उमामहेश्वर पार्षदाय नमः।\\
यं वायवे नमः । वायव्य दिग्भागे  वायुः सुप्रीतो  वरदो भवतु ॥ (6)\\
{\small यं भूर्भुवस्सुवः वायवे सांकुशध् वजहस्ताय प्राणाधिपतये मृगवाहनाय -\\
सांगाय सायुधाय सशक्ति परिवाराय - सर्वालंकार भूषिताय उमामहेश्वर पार्षदाय नमः ।\\
वायव्य दिग्भागे नासिकास्थाने वायुः सुप्रीतो सुप्रस्न्नो वरदो भवतु ॥}\\
ॐ भूर्भुव॒स्सुव॒रों ।\\
सं । व॒यꣳ सो॑ मव् व्र॒ते तव॑ । मन॑स्त॒ नू षु॒बिभ् र॑तः ।\\
प्र॒जा व॑न्तो, अशीमहि ।\\
सं उत्तर दिग्भागे सोमाय अमृत कलश \\
हस्ताय नक्षत्राधिपतये, अश्व वाहनाय सांगाय सायुधाय \\
सशक्तिस परिवाराय - उमामहेश्वर पार्षदाय नमः।\\
सं सोमाय नमः । उत्तर दिग्भागे  सोमः सुप्रीतो वरदो भवतु ॥ (7)\\
{\small इ॒न्द्राणी दे॒वी सु॒भगा॑ सु॒पत्नी᳚॥\\
सं भूर्भुवस्सुवः- सोमाय अमृत कलश हस्ताय नक्षत्राधि पतये अश्व वाहनाय -\\
सांगाय सायुधाय सशक्ति परिवाराय - सर्वालंकार भूषिताय उमामहेश्वर पार्षदाय नमः ।\\
उत्तर दिग्भागे ह्रद यस्थाने सोमः सुप्रीतो सुप्रस्न्नो वरदो भवतु ॥}\\
ॐ भूर्भुव॒स्सुव॒रों ।\\
शं । तमी शा᳚न॒ञ् जग॑तस् त॒स्थु ष॒स्पतिम्᳚ धि॒यं॒ जि॒न्व मव॑से हूमहे व॒यम्।\\
पू॒षा नो॒ यथा॒ वेद॑सा॒ मस॑द् वृ॒धे र॒क्षि॒ता पायु॒र द॑ब् शस् स्व॒स् तये᳚ ॥\\
\\
शं ईशान दिग्भागे, ईशानाय त्रिशूल हस्ताय भूताधि पतये वृषभ वाहनाय\\
सांगाय सायुधाय सशक्तिस परिवाराय -  उमामहेश्वर पार्षदाय नमः।\\
शं ईशानाय नमः  ईशान्य दिग्भागे, ईशानः सुप्रीतो  वरदो भवतु ॥ (8)\\
{\small शं भूर्भुवस्सुवः ईशानाय त्रिशूल हस्ताय विद्याधि पतये भूताधि पतये वृषभ वाहनाय -\\
सांगाय सायुधाय सशक्ति परिवाराय - सर्वालंकार भूषिताय उमामहेश्वर पार्षदाय नमः ।\\
ईशान दिग्भागे नाभिस्थाने ईशानः सुप्रीतो सुप्रस्न्नो वरदो भवतु ॥}\\
\\
ॐ भूर्भुव॒स्सुव॒रों ।\\
खं । अ॒स्मे रु॒द्रा मे॒हना॒ पर्व॑ तासो वृत्र॒ हत्ये॒ भर॑ हूतौ स॒जोषाः᳚ ॥\\
यश् शंस॑ते स्तुव॒ते धायि॑ प॒ज्र इन्द्र॑ज् ज्येष्ठा, अ॒स्माम् अ॑वन्तु दे॒वाः ॥\\
\\
खं ऊर्ध्व दिग्भागे ब्रह्मणे पद्महस्ताय प्रजाधिपतये हंस वाहनाय\\
सांगाय सायुधाय सशक्तिस परिवाराय -  उमामहेश्वर पार्षदाय नमः।\\
खं ब्रह्मणे नमः । ऊर्ध्व दिग्भागे ब्रह्मा सुप्रीतो  वरदो भवतु ॥  (9)\\
{\small खं भूर्भुवस्सुवः ब्रह्मणे पद्म हस्ताय लोकाधि पतये हंस वाहनाय -\\
सांगाय सायुधाय सशक्ति परिवाराय - सर्वालंकार भूषिताय उमामहेश्वर पार्षदाय नमः ।\\
ऊर्ध्व दिग्भागे मूर्धस्थाने ब्रह्मा सुप्रीतो सुप्रस्न्नो वरदो भवतु ॥}\\
\\
ॐ भूर्भुव॒स्सुव॒रों ।\\
ह्रीं । स्यो॒ना पृ॑थि विभवा॑ऽ नृक्ष॒रा नि॒वे श॑नि । यच्छा॑ न॒श् शर्म॑ स॒प्रथाः᳚ ।\\
\\
ह्रीं अधो दिग्भागे विष्णवे चक्र हस्ताय लोखाधिपतये गरुड वाहनाय\\
सांगाय सायुधाय सशक्तिस परिवाराय -  उमामहेश्वर पार्षदाय नमः।\\
ह्रीं विष्णवे नमः । अधो दिग्भागे वष्णुस्सुप्रीतो वरदो भवतु ॥  (10)\\
{\small ह्रीं भूर्भुवस्सुवः विष्णवे चक्र हस्ताय नागाधि पतये गरुडवाहनाय -\\
सांगाय सायुधाय सशक्ति परिवाराय - सर्वालंकार भूषिताय उमामहेश्वर पार्षदाय नमः ।\\
अधोदिग्भागे पादस्थाने विष्णुस्सुुप्रीतो सुप्रस्न्नो वरदो भवतु ॥}\\

\subsection{\eng{Shodashanga Roudri Karanam}}\\
{\small ॐ भूर्भुव॒स्सुवः॑। ॐ अं । नम॑श्शं॒भवे॑ च मयो॒भवे॑ च॒ नमः॑ शंक॒राय॑ च मयस्क॒राय॑ च॒ नमः॑ शि॒वाय॑ च शि॒वत॑राय च॥}\\
वि॒भू॑रसिप् प्र॒वाह॑ णो॒ रौद्रे॒णानी॑केन पा॒हि माऽ᳚ग्ने पिपृ॒हि\\
मा॒ मा मा॑ हिꣳसीः ॥\\
{\small अं ॐ भूर्भुव॒स्सुवरोम्। शिखास्थाने रुद्राय नमः॥}\\
\\
{\small ॐ भूर्भुव॒स्सुवः॑। ॐ आं। नम॑श्शं॒भवे॑ च मयो॒भवे॑ च॒ नमः॑ शंक॒राय॑ च मयस्क॒राय॑ च॒ नमः॑ शि॒वाय॑ च शि॒वत॑राय च॥}\\
वह्नि॑रसि हव्य॒ वाह॑नो॒ रौद्रे॒णानी॑केन पा॒हि माऽ᳚ग्ने पिपृ॒हि\\
मा॒ मा मा॑ हिꣳसीः ॥\\
{\small आं ॐ भूर्भुव॒स्सुवरोम्। शिरस्थाने रुद्राय नमः॥}\\
\\
{\small ॐ भूर्भुव॒स्सुवः॑। ॐ इं। नम॑श्शं॒भवे॑ च मयो॒भवे॑ च॒ नमः॑ शंक॒राय॑ च मयस्क॒राय॑ च॒ नमः॑ शि॒वाय॑ च शि॒वत॑राय च॥}\\
श्वा॒ त्रो॑सि॒ प्रचे॑ता॒ रौद्रे॒णानी॑केन पा॒हि माऽ᳚ग्ने पिपृ॒हि\\
मा॒ मा मा॑ हिꣳसीः ॥\\
{\small इं ॐ भूर्भुव॒स्सुवरोम्। मूर्धस्थाने रुद्राय नमः॥}\\
\\
{\small ॐ भूर्भुव॒स्सुवः॑। ॐ ईं। नम॑श्शं॒भवे॑ च मयो॒भवे॑ च॒ नमः॑ शंक॒राय॑ च मयस्क॒राय॑ च॒ नमः॑ शि॒वाय॑ च शि॒वत॑राय च॥}\\
तु॒थो॑सि वि॒श्व वे॑दा॒ रौद्रे॒णानी॑केन पा॒हि माऽ᳚ग्ने पिपृ॒हि\\
मा॒ मा मा॑ हिꣳसीः ॥\\
{\small ईं ॐ भूर्भुव॒स्सुवरोम्। ललाटस्थाने रुद्राय नमः॥}\\
\\
{\small ॐ भूर्भुव॒स्सुवः॑। ॐ उं। नम॑श्शं॒भवे॑ च मयो॒भवे॑ च॒ नमः॑ शंक॒राय॑ च मयस्क॒राय॑ च॒ नमः॑ शि॒वाय॑ च शि॒वत॑राय च॥}\\
उ॒शिग॑ सिक॒वी रौद्रे॒णानी॑केन पा॒हि माऽ᳚ग्ने पिपृ॒हि\\
मा॒ मा मा॑ हिꣳसीः ॥  (5)\\
{\small उं ॐ भूर्भुव॒स्सुवरोम्। भ्रूस्थाने रुद्राय नमः॥}\\
\\
{\small ॐ भूर्भुव॒स्सुवः॑। ॐ ऊं। नम॑श्शं॒भवे॑ च मयो॒भवे॑ च॒ नमः॑ शंक॒राय॑ च मयस्क॒राय॑ च॒ नमः॑ शि॒वाय॑ च शि॒वत॑राय च॥}\\
अंघा॑रि रसि॒ बंभा॑री॒ रौद्रे॒णानी॑केन पा॒हि माऽ᳚ग्ने पिपृ॒हि\\
मा॒ मा मा॑ हिꣳसीः ॥\\
{\small ऊं ॐ भूर्भुव॒स्सुवरोम्। मुखस्थाने रुद्राय नमः॥}\\
\\
{\small ॐ भूर्भुव॒स्सुवः॑। ॐ ऋं । नम॑श्शं॒भवे॑ च मयो॒भवे॑ च॒ नमः॑ शंक॒राय॑ च मयस्क॒राय॑ च॒ नमः॑ शि॒वाय॑ च शि॒वत॑राय च॥}\\
अ॒व॒स् युर॑सि॒ दुव॑स् वा॒न् रौद्रे॒णानी॑केन पा॒हि माऽ᳚ग्ने पिपृ॒हि\\
मा॒ मा मा॑ हिꣳसीः ॥\\
{\small ऋं ॐ भूर्भुव॒स्सुवरोम्। कण्ठस्थाने रुद्राय नमः॥}\\
\\
{\small ॐ भूर्भुव॒स्सुवः॑। ॐ ऋृं । नम॑श्शं॒भवे॑ च मयो॒भवे॑ च॒ नमः॑ शंक॒राय॑ च मयस्क॒राय॑ च॒ नमः॑ शि॒वाय॑ च शि॒वत॑राय च॥}\\
शु॒न्ध्यू र॑सि मार् जा॒लीयो॒ रौद्रे॒णानी॑केन पा॒हि माऽ᳚ग्ने पिपृ॒हि\\
मा॒ मा मा॑ हिꣳसीः ॥\\
{\small ऋृं ॐ भूर्भुव॒स्सुवरोम्। बाहुस्थाने रुद्राय नमः॥}\\
\\
{\small ॐ भूर्भुव॒स्सुवः॑। ॐ लृं । नम॑श्शं॒भवे॑ च मयो॒भवे॑ च॒ नमः॑ शंक॒राय॑ च मयस्क॒राय॑ च॒ नमः॑ शि॒वाय॑ च शि॒वत॑राय च॥}\\
सं॒म्राड॑सि कृ॒शा नू॒ रौद्रे॒णानी॑केन पा॒हि माऽ᳚ग्ने पिपृ॒हि\\
मा॒ मा मा॑ हिꣳसीः ॥\\
{\small लृं  ॐ भूर्भुव॒स्सुवरोम्। ऊरुस्थाने रुद्राय नमः॥}\\
\\
{\small ॐ भूर्भुव॒स्सुवः॑। ॐ लृंृ । नम॑श्शं॒भवे॑ च मयो॒भवे॑ च॒ नमः॑ शंक॒राय॑ च मयस्क॒राय॑ च॒ नमः॑ शि॒वाय॑ च शि॒वत॑राय च॥}\\
प॒रि॒ षद्यो॑सि॒ पव॑ मानो॒ रौद्रे॒णानी॑केन पा॒हि माऽ᳚ग्ने पिपृ॒हि\\
मा॒ मा मा॑ हिꣳसीः ॥ (10)\\
{\small लृंृ  ॐ भूर्भुव॒स्सुवरोम्। हृदयस्थाने रुद्राय नमः॥}\\
\\
{\small ॐ भूर्भुव॒स्सुवः॑। ॐ एं। नम॑श्शं॒भवे॑ च मयो॒भवे॑ च॒ नमः॑ शंक॒राय॑ च मयस्क॒राय॑ च॒ नमः॑ शि॒वाय॑ च शि॒वत॑राय च॥}\\
प्र॒तक् वा॑सि॒ नभ॑स् वा॒न् रौद्रे॒णानी॑केन पा॒हि माऽ᳚ग्ने पिपृ॒हि\\
मा॒ मा मा॑ हिꣳसीः ॥\\
{\small एं ॐ भूर्भुव॒स्सुवरोम्। नाभिस्थाने रुद्राय नमः॥}\\
\\
{\small ॐ भूर्भुव॒स्सुवः॑। ॐ ऐं। नम॑श्शं॒भवे॑ च मयो॒भवे॑ च॒ नमः॑ शंक॒राय॑ च मयस्क॒राय॑ च॒ नमः॑ शि॒वाय॑ च शि॒वत॑राय च॥}\\
असं॑ मृष् टोसि हव्य॒ सूदो॒ रौद्रे॒णानी॑केन पा॒हि माऽ᳚ग्ने पिपृ॒हि\\
मा॒ मा मा॑ हिꣳसीः ॥\\
{\small ऐं ॐ भूर्भुव॒स्सुवरोम्। कटिस्थाने रुद्राय नमः॥}\\
\\
{\small ॐ भूर्भुव॒स्सुवः॑। ॐ ॐ। नम॑श्शं॒भवे॑ च मयो॒भवे॑ च॒ नमः॑ शंक॒राय॑ च मयस्क॒राय॑ च॒ नमः॑ शि॒वाय॑ च शि॒वत॑राय च॥}\\
ऋ॒त धा॑मासि॒ सुव॑र् ज्योती॒ रौद्रे॒णानी॑केन पा॒हि माऽ᳚ग्ने पिपृ॒हि\\
मा॒ मा मा॑ हिꣳसीः ॥\\
{\small ॐ ॐ भूर्भुव॒स्सुवरोम्। ऊरुस्थाने रुद्राय नमः॥}\\
\\
{\small ॐ भूर्भुव॒स्सुवः॑। ॐ औं। नम॑श्शं॒भवे॑ च मयो॒भवे॑ च॒ नमः॑ शंक॒राय॑ च मयस्क॒राय॑ च॒ नमः॑ शि॒वाय॑ च शि॒वत॑राय च॥}\\
ब्रह्म॑ ज्योति रसि॒ सुव॑र् धामा॒ रौद्रे॒णानी॑केन पा॒हि माऽ᳚ग्ने पिपृ॒हि\\
मा॒ मा मा॑ हिꣳसीः ॥\\
{\small औं ॐ भूर्भुव॒स्सुवरोम्। जानुस्थाने रुद्राय नमः॥}\\
\\
{\small ॐ भूर्भुव॒स्सुवः॑। ॐ अं। नम॑श्शं॒भवे॑ च मयो॒भवे॑ च॒ नमः॑ शंक॒राय॑ च मयस्क॒राय॑ च॒ नमः॑ शि॒वाय॑ च शि॒वत॑राय च॥}\\
अ॒जो᳚स् येकपा॒द् रौद्रे॒णानी॑केन पा॒हि माऽ᳚ग्ने पिपृ॒हि\\
मा॒ मा मा॑ हिꣳसीः ॥ (15)\\
{\small अं ॐ भूर्भुव॒स्सुवरोम्। जंघास्थाने रुद्राय नमः॥}\\
\\
{\small ॐ भूर्भुव॒स्सुवः॑। ॐ अः। नम॑श्शं॒भवे॑ च मयो॒भवे॑ च॒ नमः॑ शंक॒राय॑ च मयस्क॒राय॑ च॒ नमः॑ शि॒वाय॑ च शि॒वत॑राय च॥}\\
अहि॑रसि बु॒ध् नियो॒ रौद्रे॒णानी॑केन पा॒हि माऽ᳚ग्ने पिपृ॒हि\\
मा॒ मा मा॑ हिꣳसीः ॥\\
{\small अः ॐ भूर्भुव॒स्सुवरोम्। पादस्थाने रुद्राय नमः॥}\\
\\
त्व गस् थिग तैः सर्व पापैः प्रमुच्यते । सर्व भूतेष्व परा जितो भवति ।\\
ततो भूतप् प्रेत पिशाचब् ब्रह्मराक्ष सयक्ष यमदूत शाकिनी डाकिनी \\
सर्पश् श्वापद तस्करद् ज्वरा द्युपद् रवो पघाताः । \\
{\small सर्प श्वापद वृश्चिक तस्करा द्युपद्रवा द्युपघाताः।}\\
सर्वे ज्वलन्तं पश्यन्तु । मां रक्षन्तु ।\\
यजमानं सक कुटुम्बं सर्वे बक्त महाजनानाम्  च  रक्षन्तु ॥\\
{\small यजमानं रक्षन्तु। सर्वान् महाजनानाम् रक्षन्तु॥}\\

\subsection{\eng{Guhyadhi Mastakaantam Shadanganyasam}}
मनो॒ ज्योति॑र् जुष ता॒माज्यं॒ विच् छि॑न्नं य॒ज्ञꣳ समि॒मंद॑ धातु। \\
बृह॒स्पति॑ स्तनुता मि॒मंनो॒ विश्वे॑ दे॒वा, इ॒ह मा॑द यन्ताम्॥ गुह्याय नमः॥\\
{\small या इ॒ष्टा उ॒षसो॑ नि॒म्रुच॑श्च॒ तास् सन्द॑धामि ह॒विषा॑ घृ॒तेन॑॥ गुह्याय नमः॥}\\
\\
अबो᳚ध् य॒ग् निस् स॒मिधा जना॑नां॒ प्र॑ति धे॒नु मि॑वा य॒ती मु॒षासम्᳚। \\
य॒ह्वा, इ॑व॒प् प्र॒वया मु॒ज्जि हा॑नाः॒ प्रभा॒ नव॑स् सिस् रते॒ नाक॒ मच्छा॑॥ नाभ्यै नमः॥\\
\\
अ॒ग्निर् मू॒र्धा दि॒वः क॒कुत् पतिः॑ पृथि॒व्या, अ॒यम्। \\
अ॒पाꣳ रेताꣳ॑ सि जिन्वति । हृदयाय नमः ॥\\
\\
मू॒र्धानं॑ दि॒वो, अ॑र॒तिं पृ॑थि॒व्या वै᳚श्वान॒र मृ॒ताय॑ जा॒त म॒ग्निम्‌। \\
क॒विꣳ स॒म्राज॒ मति॑थिं॒ जना॑ना मा॒सन्ना पात्रं॑ जन यन्त दे॒वाः॥ कण्ठाय नमः॥ \\
\\
मर्मा॑णि ते॒ वर्म॑भिश्छादयामि॒ सोम॑स्त्वा॒ राजा॒ऽमृते॑ना॒भिऽव॑स्ताम्।\\
उ॒रोर्वरी॑यो॒ वरि॑वस्ते अस्तु॒ जय॑न्तं त्वामनु॑ मदन्तु दे॒वाः॥ मुखाय नमः।\\
\\
जा॒तवे॑दा॒ यदि॑ वा पाव॒कोऽसि॑। वै॒श्वा॒न॒रो यदि॑ वा वैद् युतोसि॑। \\
शं प्र॒जाभ्यो॒ यज॑मानाय लो॒कम्। ऊर्जं॒ पु॒ष् तिं दद॑ द॒भ्याव॑ वृथ् स्व॥ शिरसे नमः॥\\
\subsection{\eng{Atma Rakshaha}}
ब्रह्मा᳚त् म॒न् वद॑ सृजत। तद॑ कामयत। समा॒त् मना॑ पद् ये॒येति॑। \\
\\
आत् म॒न्नात् म॒न् नित्याम॑न् त्रयत। तस्मै॑ दश॒मꣳ हू॒तः प्रत्य॑श्रुणोत्।\\
स दश॑ हूतोऽभवत्। दश॑ हूतो ह॒ वै ना मै॒षः। \\
तं वा, ए॒तन् दश॑हू त॒ꣳ सन्तम्᳚। दश॑हो॒तेत् याच॑क्षते प॒रोक्षे॑ण। \\
प॒रोक्ष॑ प्रिया, इव॒ हि दे॒वाः॥\\
\\
आत् म॒न्नात् म॒न् नित्याम॑न् त्रयत। तस्मै॑ सप्त॒मꣳ हूतः प्रत्य॑श्रुणोत्। \\
स स॒प्त हू॑तोऽभवत्। स॒प्त हू॑तो ह॒ वै ना मै॒षः। \\
तं वा, ए॒तꣳ स॒प्तहू॑ त॒ꣳ सन्तम्᳚। स॒प्तहो॒तेत् याच॑क्षते प॒रोक्षे॑ण। \\
प॒रोक्ष॑ प्रिया, इव॒ हि दे॒वाः॥\\
\\
आत् म॒न्नात् म॒न् नित्याम॑न् त्रयत। तस्मै॑ ष॒ष्ठꣳ हू॒तः प्रत्य॑श्रुणोत्। \\
स षड् ढू॑तोऽ भवत्। षड् ढू॑तो ह वै नामै॒षः। \\
तं वा, ए॒तꣳ षड्ढू॒॑त॒ꣳ॒ सन्तम्᳚। षड्ढूो॒तेत् याच॑क्षते प॒रोक्षे॑ण। \\
प॒रोक्ष प्रिया, इव॒ हि दे॒वाः॥\\
\\
आत् म॒न्नात् म॒न् नित्याम॑न् त्रयत। तस्मै॑ पञ्च॒मꣳ हू॒तः प्रत्य॑श्रुणोत्। \\
स पञ्च॑ हूतोऽभवत्। पञ्च॑ हूतो ह॒ वै नामै॒षः। \\
तं वा, ए॒तं पञ्च॑हूत॒ꣳ॒ सन्तम्᳚। पञ्च॑हो॒तेत् याच॑क्षते प॒रोक्षे॑ण। \\
प॒रोक्ष॑ प्रिया, इव॒ हि दे॒वाः॥\\
\\
आत् म॒न्नात् म॒न् नित्याम॑न् त्रयत। तस्मै॑ चतु॒र्थꣳ हू॒तः प्रत्य॑श्रुणोत्। \\
स चतु॑र् हूतोऽभवत् । चतु॑र् हूतो ह॒ वै नामै॒षः। \\
तं वा,  ए॒तं चतु॑र् हू॒त॒ꣳ॒ सन्तम्᳚। चतु॑र् होतेत् याच॑क्षते प॒रोक्षे॑ण। \\
प॒रोक्ष॑ प्रिया, इव॒ हि दे॒वाः॥\\
\\
तम॑ब् ब्रवीत्। त्वं वै मे॒ नेदि॑ष्ठꣳ हूतः प्रत्य॑श् श्रौषीः। \\
त्वयै॑ नानाख् या॒तार॒ इति॑। तस्मा॒न्नु है॑ना॒हु॒श् चतु॑र् होतार॒ इत्या च॑क्षते। \\
तस्मा᳚च् छुश् रू॒षुः पु॒त्रा णा॒ꣳ॒ हृद् य॑तमः। ने दि॑ष्ठो॒ हृद् य॑तमः॒।\\
नेदि॑ष्ठो॒ ब्रह्म॑णो भवति। य ए॒वं वेद॑॥ आत्मने नमः ।\\

\subsection{\eng{Shiva Sankalpam}}
येने॒दं भू॒तं भुव॑नं भवि॒श्यत् परि॑-गृही तम॒ मृते॑ न॒सर्वम्᳚॥ \\
ये॑न य॒ज्ञस् त्रा॑यते॑ स॒प्त हो॑ता॒ तन्मे॒ मनः॑ शि॒वसं॑क॒ल्पम॑स्तु॥ (1)\\
\\
येन॒ कर्मा॑णिप् प्र॒चर॑न्ति॒ धीरा॒ यतो॑ वा॒चा मन॑सा चारु॒ यन्ति॑। \\
यत्सं मि॑तं॒ मनः॑ संचर॑न्ति॒ तन्मे॒ मनः॑ शि॒वसं॑क॒ल्पम॑स्तु॥ (2)\\
{\small यत्सं॒ मित॒ मनु॑सं॒ यन्ति॑प् प्रा॒णि न॒स् तन्मे॒ मनः॑ शि॒वसं॑क॒ल्पम॑स्तु॥}\\
\\
येन॒ कर्मा᳚ण्य॒ पसो॑ मनी॒षिणो॑ य॒ज्ञे कु॑ण्वन्ति वि॒दथे॑षु॒ धीराः᳚। \\
यद॑ पू॒र्वय् य॒क्ष मन्तं॑ प्र॒जानां॒ तन्मे॒ मनः॑ शि॒वसं॑क॒ल्पम॑स्तु॥ (3)\\
\\
यत् प्र॒ज्ञान॑ मु॒त चेतो॒ धृति॑श्च॒ यज् ज्योति॑-र॒न्त र॒मृतं॑ प्र॒जासु॑।\\
यस्मा॒न्न ऋ॒ते कि॑ञ्-च॒न कर्म॑क् क्रि॒यते॒ तन्मे॒ मनः॑ शि॒वसं॑क॒ल्पम॑स्तु॥ (4)\\
\\
सु॒षा॒ र॒थि रश्वा॑ निव॒यन् म॑नु॒ष्या᳚न् मे॒नि॒युते॑ प॒शुभि॑र् वा॒जिनी॑ वान्। \\
हृ॒त् प्र॒वि॒ष् ठय् यद च॑र॒य् यविष्ठं॒ तन्मे॒ मनः॑ शि॒वसं॑क॒ल्पम॑स्तु॥ (5)\\
{\small सु॒षा॒ र॒थि रश्वा॑ निव॒यन् म॑नु॒ष्या᳚न् नेनी॒यते॒ऽ भीशु॑भिर् वा॒जिन॑ इव।\\
हृत्प्रष्ठिं॒ यद॑जिरं॒ जबि॑ष्ठं॒ तन्मे॒ मनः॑ शि॒वसं॑क॒ल्पम॑स्तु॥}\\
\\
यस्मि॒न् नृच॒स् साम॒ यजूꣳ॑ षि॒ यस्मि॑न् प्रति॒ष्ठा र॑श॒ना भावि॒ भाराः᳚। \\
यस् मिग्ग्श् चि॒त्तꣳ सर्व॒ मोतं॑ प्र॒जानां॒ तन्मे॒ मनः॑ शि॒वसं॑क॒ल्पम॑स्तु॥ (6)\\
\\
यदत्र॑ ष॒ष्ठं त्रि॒शतꣳ॑ सु॒वीर्यं॑ य॒ज्ञस्य॑ गुह्यं॒ नव॑ना व॒माय्यं᳚। \\
दश॑ पञ्चत् त्रि॒ꣳ॒ शत॒य् यत् परं॒ तन्मे॒ मनः॑ शि॒वसं॑क॒ल्पम॑स्तु॥ (7)\\
\\
यज् जाग्र॑तो दू॒र मु॒दैतु॒ सर्वं॒ तत्-सु॒प्-तस्य॒-तथै॒ वेति॑।\\
दू॒रं॒ग॒ मं ज्योति॑ षां॒ ज्यो॒ति रेकं॒ तन्मे॒ मनः॑ शि॒वसं॑क॒ल्पम॑स्तु॥ (8)\\
\\
येने॒दं विश्वं॒ जग॑तो ब॒भूव॒ ये दे॒वापि॑ मह॒तो जा॒तवे॑दाः।\\
तदे॒वाग् निस् तद् वा॒युस् तत् सूर्य॒स् तदु॑-च॒न्द्र मा॒स्तन्मे॒ मनः॑\\
 शि॒वसं॑क॒ल्पम॑स्तु॥ (9)\\
{\small तदे॒वाग्निस् तम॑सो॒ ज्योति॒ रेकं॒ तन्मे॒ मन॑श्शि॒वसं॑क॒ल्पम॑स्तु॥}\\
\\
येन॒द् द्यौः पृ॑थि॒वी चा॒न् तरि॑क्षं च॒ ये पर्व॑ताः प्र॒दिशो॒ दिश॑श्च। \\
येने॒दं जग॒द् ध्याप्तं॑ प्र॒जानां॒ तन्मे॒ मनः॑ शि॒वसं॑क॒ल्पम॑स्तु॥ (10)\\
\\
ये मनो॒ हृद॑ यय्ये च॑ दे॒वा ये दि॒व्या, आपो॒ ये सूर्य॑ र॒श् मिः। \\
ते श्रोत्रे॒ चक्षु॑षी \textbf{सं॒चर॑न् तन्॒} तन्मे॒ मनः॑ शि॒वसं॑क॒ल्पम॑स्तु॥ (11)\\
\\
अचि॑न् त्यञ्॒चा प्र॑मे यंचव् व्य॒क्ता॒, व्यक्त॑ परं॒ चय॑त्। \\
सूक्ष्मा᳚त् सूक्ष्म त॑रन्, ज्ञे॒यं तन्मे॒ मनः॑ शि॒वसं॑क॒ल्पम॑स्तु॥ (12)\\
\\
एका॑ च द॒श च॑ श॒तं च॑ स॒हस्र॑ञ् चा॒ यु॑तञ्च। \\
नि॒यु तं॑च प्र॒यु त॒ञ्चार् बु॑दञ् च॒न् य॑र् बुदंच \\
{\small समु॒द्रश्च॒ मध्यं॒ चान्त॑श्च परा॒र्धश्च॒}\\
तन्मे॒ मनः॑ शि॒वसं॑क॒ल्पम॑स्तु॥ (13)\\
\\
ये प॑ञ्च पञ्चा॒द॒श श॒तꣳ स॒हस्र॑ म॒युतं॒ न्य॑र् बुदं च। \\
ए अ॑ग्नि चि॒त्तेष् ट॑का॒स्ताꣳ शरी॑रं॒ तन्मे॒ मनः॑ शि॒वसं॑क॒ल्पम॑स्तु ॥ (14)\\
\\
वेदा॒हमे॒तं पुरु॑षं म॒हान्त॑ मादि॒त्यव॑र्णं॒ तम॑सः॒ पर॑स्तात्। \\
यस्य॒ योनिं॒ परि॒ पश्य॑न्ति॒ धीरा॒स्तन्मे॒ मनः॑ शि॒वसं॑क॒ल्पम॑स्तु॥ (15)\\
\\
यस्यैतं धीराः᳚ पु॒नन्ति॑ क॒वयो᳚ ब्र॒ह्माण॑ मे॒तं त्वा॑ वृणुत॒ मिन्दुं᳚। \\
स्था॒व॒रं जङ्ग॑मं॒ द्यौरा॑का॒शं॒ तन्मे॒ मनः॑ शि॒वसं॑क॒ल्पम॑स्तु॥ (16)\\
\\
{\small \eng{17/18 may be swapped as well}}\\
परा᳚त् प॒रत॑रं ब्र॒ह्म॒ त॒त् परा᳚त् पर॒तो ह॑रिः। \\
य॒त् परा᳚त् पर॑तोऽ धी॒शं॒ तन्मे॒ मनः॑ शि॒वसं॑क॒ल्पम॑स्तु॥ (17)\\
\\
परा᳚त् प॒रत॑रं चैव त॒त् परा᳚च् चैव॒ यत् प॑रम्। \\
य॒त् परा᳚त् पर॑तो ज्ञे॒यं॒ तन्मे॒ मनः॑ शि॒वसं॑क॒ल्पम॑स्तु॥ (18)\\
\\
या वेदा दिषु॑-गाय॒त्री स॒र्वव् व्या॑पी म॒हेश्व॑री। \\
ऋग् य॑जु॒स् सामा॑ थर् वै॒श्च॒ तन्मे॒ मनः॑ शि॒वसं॑क॒ल्पम॑स्तु॥ (19)\\
\\
यो वै॑ दे॒वं म॑हादे॒वं॒ प्र॒य॒तः प्र॑णत॒श् शु॑चिः। \\
{\small यो वै॑ दे॒वं म॑हादे॒वं प्र॒णवं॑ पर॒मेश्व॑रम्।}\\
यस्सर्वे॑ सर्व॑ वेदै॒श्च तन्मे॒ मनः॑ शि॒वसं॑क॒ल्पम॑स्तु॥ (20)\\
\\
प्र॒य॒तः॒ प्रण॑वोंका॒रं प्र॒णवं॑ पुरु॒षोत्त॑मम्। \\
ओंका॑रं॒ प्रण॑वात्मा॒नं तन्मे॒ मनः॑ शि॒वसं॑क॒ल्पम॑स्तु॥ (21)\\
\\
योऽसौ॑ स॒र्वेषु॑ वेदे॒षु प॒ठ्यते᳚ ह्यय॒ मीश्व॑र:। \\
अकायो॑ निर्गु॑णो ह्या॒त्मा तन्मे॒ मनः॑ शि॒वसं॑क॒ल्पम॑स्तु॥ (22)\\
\\
गोभि॒र्जुष्टं॒ धने॑न॒ह् ह्यायु॑षा च॒ बले॑ नच। \\
प्र॒जया॑ प॒शुभिः॑ पुष्करा॒क्षं तन्मे॒ मनः॑ शि॒वसं॑क॒ल्पम॑स्तु॥ (23)\\
\\
{\small \eng{24/25 may be swapped as well}}\\
त्र्यं॑बकं यजामहे सुग॒न्धिं पु॑ष्ति॒वर्ध॑नम्। उ॒र्वा॒रु॒कमि॑व॒ \\
बन्ध॑नान्मृ॒त्योर्मु॑क्षीय॒ माऽमृता॒ तन्मे॒ मनः॑ शि॒वसं॑क॒ल्पम॑स्तु॥ (24)\\
\\
कैला॑स॒ शिख॑रे र॒म्ये॒ शङ्कर॑स्य शि॒वाल॑ये। \\
दे॒वता᳚स् तत्र॑ मोदन्ति॒ तन्मे॒ मनः॑ शि॒वसं॑क॒ल्पम॑स्तु॥ (25)\\
\\
{\small 26 may be left}\\
कैला॑स॒ शिखरा वा॒सं हि॒मव॑द् गिरि॒ संस्थिथं । \\
नी॒ल॒क॒ण्ठं त्रि॑नेत्रं च तन्मे॒ मनः॑ शि॒वसं॑क॒ल्पम॑स्तु॥ (26)\\
\\
वि॒श्व त॑श् चक्षुरु॒त वि॒श्व तो॑मुखो वि॒श्व तो॑हस्त उ॒त वि॒श्व त॑स्पात्।\\
सं बा॒हुभ्यां॒ नम॑ति॒-सं-पत॑त् त्रै॒र् द्यावा॑ पृथि॒वी ज॒नय॑न् दे॒व\\
 एक॒स्तन्मे॒ मनः॑ शि॒वसं॑क॒ल्पम॑स्तु॥ (27)\\
\\
चतुरो॑ वे॒दा न॑धीयी॒त स॒र्व शा᳚स् त्रम॒यं वि॑दुः।  \\
इ॒ति॒हा॒स पु॑राणा॒नां॒ तन्मे॒ मनः॑ शि॒वसं॑क॒ल्पम॑स्तु॥ (28)\\
\\
मा नो॑ म॒हान्त॑मु॒त मा नो॑, अर्भ॒कं मा न॒ उक्ष॑न्तमु॒त मा न॑ उक्षि॒तम्। \\
मा नो॑ऽवधीः पि॒तरं॒ मोत मा॒तरं॑ प्रि॒या मा न॑स्त॒नुवो॑ \\
रुद्र रीरिष॒स्तन्मे॒ मनः॑ शि॒वसं॑क॒ल्पम॑स्तु॥  (29)\\
\\
मान॑स्तो॒के तन॑ये॒ मा न॒ आयु॑षि॒ मा नो॒ गोषु॒ मा नो॒, अश्वे॑षु रीरिषः। \\
वी॒रान्मा नो॑ रुद्र भामि॒तो ऽव॑धीर् ह॒विष्म॑न्तो॒ नम॑सा \\
विधेम ते॒ तन्मे॒ मनः॑ शि॒वसं॑क॒ल्पम॑स्तु॥ (30)\\
\\
ऋ॒तꣳ स॒त्यं प॑रं ब्र॒ह्म॒ पु॒रुषं॑ कृष्ण॒पिङ्ग॑लम्। \\
ऊ॒र्ध्वरे॑तंवि॑रूपा॒क्षं॒ वि॒श्वरू॑पाय॒ वै नमो॒ नम॒स्तन्मे॒ मनः॑ शि॒वसं॑क॒ल्पम॑स्तु॥ (31)\\
\\
कद्रु॒द्राय॒ प्रचे॑तसे मी॒ढुष्ट॑माय॒ तव्य॑से। \\
वो॒चेम॒ शंत॑मꣳ हृ॒दे। सर्वो॒ ह्ये॑ष रु॒द्रस्तस्मै॑ रु॒द्राय॒ नमो॑ अस्तु॒ \\
तन्मे॒ मनः॑ शि॒वसं॑क॒ल्पम॑स्तु॥ (32)\\
\\
ब्रह्म॑ जज्ञा॒नं प्र॑थ॒मं पु॒रस्ता॒द् विसी॑ म॒तस् सु॒रुचो॑ वे॒न आ॑वः। \\
स बु॒ध्निया॑, उप॒मा, अ॑स्य वि॒ष्ठास् स॒तश्च॒ योनि॒मस॑तश्च॒ \\
विव॒स् तन्मे॒ मनः॑ शि॒वसं॑क॒ल्पम॑स्तु॥ (33)\\
\\
यः प्रा॑ण॒तो नि॑मिष॒तो म॑हि॒त्वै॒क इद्राजा॒ जग॑तो ब॒भूव॑।\\
य ईशे॑, अ॒स्यद् द्वि॒पद॒श्चतु॑ष्पदः॒ कस्मै॑ दे॒वाय॑ ह॒विषा॑ विधेम॒ \\
तन्मे॒ मनः॑ शि॒वसं॑क॒ल्पम॑स्तु॥ (34)\\
\\
य आ᳚त् म॒दा ब॑ल॒दा यस्य॒ विश्व॑ उ॒पास॑ते प्र॒शिषं॒ यस्य॑ दे॒वाः। \\
यस्य॑ छायाऽमृतं यस्य॑ मृ॒त्युः कस्मै॑ दे॒वाय॑ ह॒विषा॑ विधेम॒ \\
तन्मे॒ मनः॑ शि॒वसं॑क॒ल्पम॑स्तु॥ (35)\\
\\
यो रु॒द्रो, अ॒ग्नौ यो, अ॒प्सु य ओष॑धीषु॒ यो रु॒द्रो विश्वा॒ \\
भुव॑नाऽऽवि॒वेश॒ तस्मै॑ रु॒द्राय॒ नमो॑, अस्तु॒ तन्मे॒ \\
मनः॑ शि॒वसं॑क॒ल्पम॑स्तु॥ (36)\\
\\
ग॒न्ध॒द्वा॒रां दु॑राध॒र्षां॒ नि॒त्यपु॑ष्टां करी॒षिणी᳚म्। \\
ई॒श्वरीꣳ॑ सर्व॑भूता॒नां॒ त्वामि॒होप॑ह्रये॒ श्रियं॒  \\
तन्मे॒ मनः॑ शि॒वसं॑क॒ल्पम॑स्तु॥ (37)\\
\\
{\small 38 may be left}\\
नमकं॑ चम॑कं चै॒व पु॒रुषसू᳚क्तं च॒ यद् विदुः। \\
महादेवं च तत्तुल्यं॒ तन्मे॒ मनः॑ शि॒वसं॑क॒ल्पम॑स्तु॥ (38)\\
\\
य इ॒दꣳ शिव॑संक॒ल्प॒ꣳ॒ स॒दा ध्या॑यन्ति॒ब् ब्राह्म॑णाः। \\
ते परं॒ मोक्षं॑ गमिष्यन्ति॒ तन्मे॒ मनः॑ शि॒वसं॑क॒ल्पम॑स्तु ॥ (39)\\
हृदयाय नमः।\\

\subsection{\eng{Purusha Suktam}}
स॒हस्र॑शीर्षा॒ पुरु॑षः । स॒ह॒स्रा॒क्षः स॒हस्र॑पात् ।\\
स भूमिं॑ वि॒श्वतो॑ वृ॒त्वा । अत्य॑तिष्ठद्दशाङ्गु॒लम् ।\\
पुरु॑ष ए॒वेदग्ं सर्वम्᳚ । यद्भू॒तं यच्च॒ भव्यम्᳚ ।\\
उ॒तामृ॑त॒त्वस्येशा॑नः । य॒दन्ने॑नाति॒रोह॑ति ।\\
ए॒तावा॑नस्य महि॒मा ।\\
अतो॒ ज्यायाग्॑श्च॒ पूरु॑षः ॥ १ ॥\\
\\
पादो᳚ऽस्य॒ विश्वा॑ भू॒तानि॑ । त्रि॒पाद॑स्या॒मृतं॑ दि॒वि ।\\
त्रि॒पादू॒र्ध्व उदै॒त्पुरु॑षः ।\\
पादो᳚ऽस्ये॒हाऽऽभ॑वा॒त्पुन॑: ।\\
ततो॒ विष्व॒ङ्व्य॑क्रामत् ।\\
सा॒श॒ना॒न॒श॒ने अ॒भि । तस्मा᳚द्वि॒राड॑जायत ।\\
वि॒राजो॒ अधि॒ पूरु॑षः । स जा॒तो अत्य॑रिच्यत ।\\
प॒श्चाद्भूमि॒मथो॑ पु॒रः ॥ २ ॥\\
\\
यत्पुरु॑षेण ह॒विषा᳚ । दे॒वा य॒ज्ञमत॑न्वत ।\\
व॒स॒न्तो अ॑स्यासी॒दाज्यम्᳚ । ग्री॒ष्म इ॒ध्मश्श॒रद्ध॒विः ।\\
स॒प्तास्या॑सन्परि॒धय॑: । त्रिः स॒प्त स॒मिध॑: कृ॒ताः ।\\
दे॒वा यद्य॒ज्ञं त॑न्वा॒नाः ।\\
अब॑ध्न॒न्पुरु॑षं प॒शुम् ।\\
तं य॒ज्ञं ब॒र्हिषि॒ प्रौक्षन्॑ ।\\
पुरु॑षं जा॒तम॑ग्र॒तः ॥ ३ ॥\\
\\
तेन॑ दे॒वा अय॑जन्त । सा॒ध्या ऋष॑यश्च॒ ये ।\\
तस्मा᳚द्य॒ज्ञात्स॑र्व॒हुत॑: । सम्भृ॑तं पृषदा॒ज्यम् ।\\
प॒शूग्‍स्ताग्‍श्च॑क्रे वाय॒व्यान्॑ । आ॒र॒ण्यान्ग्रा॒म्याश्च॒ ये ।\\
तस्मा᳚द्य॒ज्ञात्स॑र्व॒हुत॑: । ऋच॒: सामा॑नि जज्ञिरे ।\\
छन्दाग्ं॑सि जज्ञिरे॒ तस्मा᳚त् । यजु॒स्तस्मा॑दजायत ॥ ४ ॥\\
\\
तस्मा॒दश्वा॑ अजायन्त । ये के चो॑भ॒याद॑तः ।\\
गावो॑ ह जज्ञिरे॒ तस्मा᳚त् । तस्मा᳚ज्जा॒ता अ॑जा॒वय॑: ।\\
यत्पुरु॑षं॒ व्य॑दधुः । क॒ति॒धा व्य॑कल्पयन् ।\\
मुखं॒ किम॑स्य॒ कौ बा॒हू । कावू॒रू पादा॑वुच्येते ।\\
ब्रा॒ह्म॒णो᳚ऽस्य॒ मुख॑मासीत् । बा॒हू रा॑ज॒न्य॑: कृ॒तः ॥ ५ ॥\\
\\
ऊ॒रू तद॑स्य॒ यद्वैश्य॑: । प॒द्भ्याग्ं शू॒द्रो अ॑जायत ।\\
च॒न्द्रमा॒ मन॑सो जा॒तः । चक्षो॒: सूर्यो॑ अजायत ।\\
मुखा॒दिन्द्र॑श्चा॒ग्निश्च॑ । प्रा॒णाद्वा॒युर॑जायत ।\\
नाभ्या॑ आसीद॒न्तरि॑क्षम् । शी॒र्ष्णो द्यौः सम॑वर्तत ।\\
प॒द्भ्यां भूमि॒र्दिश॒: श्रोत्रा᳚त् ।\\
तथा॑ लो॒काग्ं अ॑कल्पयन् ॥ ६ ॥\\
\\
वेदा॒हमे॒तं पुरु॑षं म॒हान्तम्᳚ ।\\
आ॒दि॒त्यव॑र्णं॒ तम॑स॒स्तु पा॒रे ।\\
सर्वा॑णि रू॒पाणि॑ वि॒चित्य॒ धीर॑: ।\\
नामा॑नि कृ॒त्वाऽभि॒वद॒न्॒ यदास्ते᳚ ।\\
धा॒ता पु॒रस्ता॒द्यमु॑दाज॒हार॑ ।\\
श॒क्रः प्रवि॒द्वान्प्र॒दिश॒श्चत॑स्रः ।\\
तमे॒वं वि॒द्वान॒मृत॑ इ॒ह भ॑वति ।\\
नान्यः पन्था॒ अय॑नाय विद्यते ।\\
य॒ज्ञेन॑ य॒ज्ञम॑यजन्त दे॒वाः ।\\
तानि॒ धर्मा॑णि प्रथ॒मान्या॑सन् ।\\
ते ह॒ नाकं॑ महि॒मान॑: सचन्ते ।\\
यत्र॒ पूर्वे॑ सा॒ध्याः सन्ति॑ दे॒वाः ॥ ७ ॥\\
शिरसे स्वाहा ॥\\
\subsubsection{\eng{Uttara Narayanam}}
अ॒द्भ्यः सम्भू॑तः पृथि॒व्यै रसा᳚च्च ।\\
वि॒श्वक॑र्मण॒: सम॑वर्त॒ताधि॑ ।\\
तस्य॒ त्वष्टा॑ वि॒दध॑द्रू॒पमे॑ति ।\\
तत्पुरु॑षस्य॒ विश्व॒माजा॑न॒मग्रे᳚ ।\\
वेदा॒हमे॒तं पुरु॑षं म॒हान्तम्᳚ ।\\
आ॒दि॒त्यव॑र्णं॒ तम॑स॒: पर॑स्तात् ।\\
तमे॒वं वि॒द्वान॒मृत॑ इ॒ह भ॑वति ।\\
नान्यः पन्था॑ विद्य॒तेय॑ऽनाय ।\\
प्र॒जाप॑तिश्चरति॒ गर्भे॑ अ॒न्तः ।\\
अ॒जाय॑मानो बहु॒धा विजा॑यते ॥ ८ ॥\\
\\
तस्य॒ धीरा॒: परि॑जानन्ति॒ योनिम्᳚ ।\\
मरी॑चीनां प॒दमि॑च्छन्ति वे॒धस॑: ।\\
यो दे॒वेभ्य॒ आत॑पति ।\\
यो दे॒वानां᳚ पु॒रोहि॑तः ।\\
पूर्वो॒ यो दे॒वेभ्यो॑ जा॒तः ।\\
नमो॑ रु॒चाय॒ ब्राह्म॑ये ।\\
रुचं॑ ब्रा॒ह्मं ज॒नय॑न्तः ।\\
दे॒वा अग्रे॒ तद॑ब्रुवन् ।\\
यस्त्वै॒वं ब्रा᳚ह्म॒णो वि॒द्यात् ।\\
तस्य॑ दे॒वा अस॒न् वशे᳚ ॥ ९ ॥\\
\\
ह्रीश्च॑ ते ल॒क्ष्मीश्च॒ पत्॒न्यौ᳚ ।\\
अ॒हो॒रा॒त्रे पा॒र्श्वे । नक्ष॑त्राणि रू॒पम् ।\\
अ॒श्विनौ॒ व्यात्तम्᳚ । इ॒ष्टं म॑निषाण ।\\
अ॒मुं म॑निषाण । सर्वं॑ मनिषाण ॥ १० ॥\\
शिकायै वषट् ॥\\

\subsection{\eng{Aprathiratham}}
आ॒शुः शिशा॑नो वृष॒भो न यु॒ध्मो घ॒नाघ॒नः क्षोभ॑णश्चर्षणी॒नाम्।\\
सं॒क्त्रन्द॑ नोऽ निमि॒ष ए॑कवी॒रः श॒तꣳ सेना॑, अजयथ् सा॒कमिन्द्रः॑॥ (1)\\
\\
सं॒क्रन्द॑ नेना निमि॒षेण॑ जि॒ष्णुना॑ युत्का॒रेण॑ दुश् च्य॒वनेन॑ धृ॒ष्णुना᳚।\\
तदिन्द्रे॑ण जयत॒-तथ् स॑हध्वं॒ युधो॑ नर॒ इषु॑हस्तेन॒ वृष्णा᳚॥ (2)\\
\\
स इषु॑ हस् तैः॒सनि॑ ष॒ङ्गि भि॑र् व॒शी सग्ग् स्र॑ष्टा॒ सयुध॒ इन्द्रो॑ ग॒णेन॑।\\
स॒ꣳ॒ सृ॒ष्ट॒ जिथ्सो॑ म॒पा बा॑हु श॒ध्यू᳚र्ध्व ध॑न्वा॒, प्रति॑हिता भि॒रस्ता᳚॥ (3)\\
\\
बृह॑स्पते॒ परि॑ दीया॒ रथे॑न रक्षो॒हाऽ मित्राꣳ॑ अप॒बा ध॑मानः।\\
प्र॒भञ् जन्थ् सेनाः᳚ प्र॒मृणो यु॒धा जय॑न् न॒स्माक॑ मेध्य वि॒ता रथा॑नाम्॥ (4)\\
\\
गो॒त्र॒भिदं॑ गो॒विदं॒ वज्र॑बाहुं॒ जय॑न्त॒ मज्म॑ प्रमृ॒णन्त॒ मोज॑सा।\\
इ॒मꣳ स॑जाता॒, अनु॑ वीरयध्व॒ मिन्द्रꣳ॑ सखा॒योऽ नु॒सꣳ र॑भध्वम्॥ (5)\\
\\
ब॒ल॒वि॒ज्ञा॒यः स्थविरः॒ प्रवीरः॒ सह॑स्वान् , वा॒जी सह॑मान उ॒ग्रः।\\
अ॒भिवी॑रो, अ॒भिस॑त्वा सहो॒जा जैत्र॑मिन्द्र॒ रथ॒मा ति॑ष्ठ गो॒वित्॥ (6)\\
\\
अ॒भिगो॒त्राणि॒ सह॑सा॒ गाह॑मानोऽ दा॒यो वी॒रः श॒तम॑न्यु॒ रिन्द्रः॑|\\
दु॒श् च्य॒व॒नः पृ॑तना॒ षाड॑ यु॒ध्द्यो᳚ऽ स्माक॒ꣳ॒ सेना॑, अवतु॒ प्र यु॒थ्सु॥ (7)\\
\\
ड्न्द्र॑ आसां ने॒ता बृह॒स्पति॒र् दक्षि॑णा य॒ज्ञः पु॒र एतु॒ सोमः॑।\\
दे॒व॒से॒नाना॑ मभिभञ्ज ती॒नां जय॑न्तीनां म॒रुतो॑ य॒न् त्वग्रे᳚ ॥ (8)\\
\\
इन्द्र॑स्य॒ वृष्णो॒ वरु॑णस्य॒ राज्ञ॑ आदि॒त्याना᳚ म॒रुता॒ꣳ॒ शर्ध्ध॑ उ॒ग्रम्।\\
म॒हाम॑नसां भुवनच्य॒ वानां॒ घोषो॑ दे॒वानां॒ जय॑ता॒ मुद॑स्थात्। (9)\\
\\
अ॒स्माक॒ मिन्द्रः॒ समृ॑ते षुध् व॒जेष् व॒स्मा कंया, इष॑व॒स्ता ज॑यन्तु।\\
अ॒स्माकं॑ वी॒रा, उत्त॑रे भवन्त् व॒स्मानु॑ देवा, अवता॒ हवे॑षु॥ (10)\\
\\
उद् ध॑र्षय मघव॒न् नायु॑धा॒न् युथ्सत्व॑नां माम॒कानां॒ महाꣳ॑ सि।\\
उद् वृ॑त् रहन् वा॒जिनां॒ वाजि॑ना॒न् युद्रथा॑नां॒ जय॑तामेतु॒ घोषः॑॥ (11)\\
\\
उप॒ प्रेत॒ जय॑ता नरः स्थि॒रा वः॑ सन्तु बा॒हवः॑।\\
इन्द्रो॑ वः॒ शर्म॑ यच् छत्वना धृ॒ष्याय थाऽस॑थ। (12)\\
\\
अव॑सृष्टा॒ परा॑ पत॒ शर॑व्ये॒, ब्रह्म॑सꣳ शिता।\\
गच्छा॒ मित्रा॒न् प्रवि॑श॒ मैषां॒ कं च॒नोच् छि॑षः॥ (13)\\
\\
मर्मा॑णि ते॒ वर्म॑भिश् छादयामि॒ सोम॑स्त्वा॒ राजा॒मृते॑ ना॒भिऽ व॑स्तां।\\
उ॒रोर् वरी॑यो॒ वरि॑वस्ते, अस्तु॒ जय॑न् तं॒त्वा-मनु॑ मदन्तु दे॒वाः॥ (14)\\
\\
यत्र॑ बा॒णाः स॒म्पत॑न्ति कुमा॒रा वि॑शि॒खा, इ॑व।\\
इन्द्रो॑ न॒स्तत्र॑ वृत्र॒हा वि॑श्वा॒हा शर्म॑ यच्छतु। (15)\\
\\
{\small असु॑रा नजय॒न् तदप् प्र॑तिरथस्या, \\
प्रतिर थ॒त्वं यदप् प्र॑तिरथन् द्वि॒ती यो॒हो ता॒न्वाहा᳚ \\
प्र॒त्ये॑ वते न॒यज॑ मानो॒ भ्रातृ॑व्यां जय॒त्यथो॒,\\
अन॑ भिजितमे॒ वाभिज॑यति दश॒र्चं भ॑वति॒ दशा᳚क्षरा \\
वि॒राड् विराजे॒ मौ लो॒कौ विधृ॑ता व॒नयो᳚र् लो॒कयो॒र् विधृ॑त्या॒,\\
अथो॒ दशा᳚क्षरा वि॒राडन्नं॑ वि॒राड् वि॒राज् ये॒वान् नाद्ये॒ प्रति॑ तिष्ठ॒त्य स॑दिव॒वा,\\
अ॒न्तरि॑क्ष म॒न्तरि॑क्ष मि॒वाग्नी᳚ ध्र॒माग्नी᳚ध्ने }\\
\textbf{कवचाय हुं} \\
\\
\subsection{\eng{Prathipurusha mithyunuvakaha}}
प्र॒ति॒ पू॒रु॒ष मेक॑ कपाला॒न् निर्व॑ प॒त्येक॒ मति॑रिक् \\
तं॒याव॑न्तो गृ॒ह्याः᳚ स्मस्तेभ्यः॒ कम॑करं\\
पशू॒नाꣳ शर्मा॑ऽसि॒ शर्म॒ यज॑मानस्य॒ शर्म॑ मे \\
य॒च्छैक॑ ए॒व रु॒द्रो नद् वि॒तीया॑य तस्थ \\
आ॒खुस्ते॑ रुद्र प॒शुस्तं जु॑षस्वै॒ष ते॑ रुद्र \\
भा॒गः स॒हस् वस्राऽम् बि॑कया॒ तं जु॑षस्व\\
भेष॒जं गवेऽश्वा॑य॒ पुरु॑षाय भे ष॒जमथो॑, \\
अ॒स्मभ्यं॑ भे ष॒जꣳ सुभे॑ष जं॒यथाऽ स॑ति।\\
\\
सु॒गं मे॒षाय॑ मे॒ष्या॑, अवा᳚म्ब रु॒द्रम॑दि म॒ह्यव॑ दे॒वं त्र्यं॑बकम्।\\
यथा॑ नः॒ श्रेय॑ सःकर॒द् यथा॑ नो॒\\
वस्य॑ सःकर॒द्यथा॑ नः पशु॒मतः॒ करद्यथा॑ नोव् व्यव सा॒यया᳚त्॥\\
\\
त्र्यं॑बकं यजामहे सुग॒न्धिंपु॑ष्टि॒वर्ध॑नम्। \\
उ॒र्वा॒रु॒कमि॑व॒ बन्ध॑नान्मृ॒त्योर्मु॑क्षीय॒माऽमृता᳚त्।\\
\\
ए॒षते॑ रुद्रभा॒गस्तं जु॑षस्व॒ तेना॑ऽव॒सेन॑\\
प॒रो मूज॑व॒तो ती॒ह्यव॑त त धन्वा॒ पिना॑कहस्तः॒ कृत्ति॑वासाः \\
प्र॒ति॒ पू॒रु॒ष मेक॑ कपा ला॒न् निर्व॑पति।\\
जा॒ता, ए॒व प्र॒जारु॒द्रान् नि॒रव॑दयते। एक॒मति॑रिक्तम्।\\
ज॒नि॒ष्य मा॑णा ए॒व प्र॒जारु॒द्रान् नि॒रव॑दयते। \\
एक॑ कपाला भवन्ति। ए॒क॒ धैवरु॒द्रन् नि॒रव॑दयते।\\
नाभि घा॑रयति। यद॑भि घा॒रये᳚त्‌। \\
अ॒न्त॒र॒व॒ चा॒रिणꣳ॑ रु॒द्रं कु॑र्यात्। \\
ए॒को॒ल् मु॒केन॑यन्ति। (1)\\
\\
तद्धि रु॒द्रस्य॑ भाग॒ धेयम्᳚। \\
इ॒मां दिश॑य्यन्ति। ए॒षावै रु॒द्रस्य॒दिक्।\\
स्वाया॑ मे॒वदि॒शि रु॒द्रन् नि॒रव॑दयते। \\
रु॒द्रोवा, अ॑प॒शुका॑या॒, आहु॑त्यै नाति॑ष्ठत।\\
अ॒सौते॑ प॒शुरिति॒ निर्दि॑ शे॒द्यं द्वि॒ष्यात्। \\
य॒मेवद् वेष्टि॑। तम॑स्मै प॒शुं निर्दि॑ शति। \\
यदि॒ नद् वि॒ष्यात्। आ॒खुस्ते॑ प॒शुरिति॑ ब्रूयात्। (2)\\
\\
नग्रा॒म्यान् प॒शून्, हि॒नस्ति॑। \\
नार॒ण्यान्। च॒तु॒ष्प॒थे जु॑होति।\\
ए॒षवा, अ॑ग्नी॒नां पड्वी॑ शो॒नाम॑। अ॒ग्नि॒वत् ये॒वजु॑होति। \\
म॒ध्य॒मेन॑ प॒र्णेन॑ जुहोति, स्रुग् घ्ये॑षा। \\
अथो॒खलु॑। अ॒न्त॒मे नै॒वहो॑ त॒व्यम्‌᳚। \\
अ॒न्त॒त ए॒वरु॒द्रन् नि॒र व॑दयते। (3)\\
\\
ए॒षते॑रुद्रभा॒गः स॒हस् वस्राऽम्बि॑क॒ येत्या॑ह। \\
श॒रद्वा, अ॒स्याम् बि॑का॒स् वसा᳚। \\
तया॒वा, ए॒षहि॑नस्ति।\\
यग्ं हि॒नस्ति॑। तयै॒ वैनग्ं स॒॑हश॑मयति। \\
भे॒ष॒जंगव॒ इत्या॑ह। याव॑न्त ए॒व ग्रा॒म्याःप॒शवः॑।\\
तेभ्यो॑ भेष॒जंक॑रोति। अवा᳚म्ब रु॒द्रम॑दि म॒हीत्या॑ह। \\
आ॒शिष॑ मे॒वै तामा शा᳚स्ते। (4)\\
\\
त्र्यं॑बकंयजामह॒ इत्या॑ह। \\
मृ॒त्योर्मु॑क्षीय॒माऽमृ॒ता दिति॒ वावै तदा॑ह। उत्कि॑रन्ति।\\
भग॑स्यलीफ् सन्ते। मूते॑ कृ॒त्वाऽऽ स॑जन्ति। \\
यथा॒ जन॑य्य॒ते॑ऽ वसंक॒रोति॑। ता॒दृ गे॒वतत्।\\
ए॒षते॑ रुद्र भा॒ग इत्या॑ह नि॒रव॑त्त्यै। अप्र॑ती क्ष॒माय॑न्ति। \\
अ॒पःपरि॑षिञ्चति। \\
रु॒द्रस् या॒न् तर् हि॑त्यै । प्रवा, ए॒ते᳚ऽस्माल्लो॒काच् च्य॑वन्ते। \\
येत्र्य॑म्ब कै॒श्चर॑न्ति। आ॒दि॒त्यं च॒रुं पुन॒रेत् य॒निर्व॑पति।\\
इ॒यंवा, अदि॑तिः। अ॒स्यामे॒व प्रति॑ तिष्ठन्ति॥ (5)\\
\\
{\small वि॒भ्राड् बृहत् पि॒बतु सो॒म्यं \\
मध्वायु॒र् दधद् य॒॑ज्ञप॑ता॒ ववि॑हृतम्।\\
वात॑जू तो॒यो अ॑भि॒ रक्ष॑ ति॒त्मना᳚ \\
प्र॒जाः पु॑पोष पुरु॒धा विरा᳚जति॥}\\
\\
\textbf{नेत्रत्रयाय वौषट्॥}\\

\subsection{\eng{Tvamagne Rudro Anuvakaha}}
त्वम॑ऽग्ने रु॒द्रो, असु॑रो म॒हो दि॒वस् त्वग्ं शर्धो॒ मारु॑तं पृ॒क्ष ई॑शिषे।\\
त्वं वातै॑ ररु॒णैर् या॑सि शङ्ख॒ यस्त्वं पू॒षा वि॑ध॒तः पा॑सि॒ नुत्मना᳚॥ \\
\\
आ वो॒ राजा॑ नमध् व॒रस्य॑ रु॒द्रग्ं होता॑रग्ं सत्य॒ यज॒ग्ं॒ रोद॑स् योः।\\
अ॒ग्निं पु॒रा त॑न यि॒त्नो र॒चित् ता॒द् धिर॑ण्य रूप॒ मव॑से कृणुध्वम्॥\\
\\
अग्निर् होता निष॑सा दा॒ यजी॑य् यानु॒ पस्थे॑ मा॒तुः सु॑र॒भावु॑ लो॒के। \\
युवा॑ क॒विः पु॑रुनि॒ष्ठ ऋ॒तावा॑ ध॒र्ता  कृ॑ष्टी॒ नामु॒त मध्य॑ इ॒द्धः।\\
\\
सा॒ध्वी म॑कर् दे॒ववी॑ तिन्नो, अ॒द्य य॒ज्ञस्य॑ जि॒ह्वाम॑ विदाम॒ गुह्या᳚म्\\
स आयु॒राऽगा᳚त् सुर॒भिर्वसा॑नो भ॒द्राम॑कर् दे॒वहू॑ तिन्नो, अ॒द्य॥\\
\\
अक्र॑न्द द॒ग्निः स्त॒न य॑न्नि व॒द्यौः क्षामा॒ रेरि॑हद् वी॒रुधः॑ सम॒ञन्न्।\\
स॒द्यो ज॑ज्ञा॒नो विही मि॒द्धो, अख्य॒ दारो द॑सी भा॒नुना॑ भात्य॒न्तः॥\\
\\
त्वे वसू॑नि पुर्वणी कहोतर् दो॒षा वस्तो॒ रेरि॑रे य॒ज्ञिया॑सः।\\
क्षामे॑ व॒विश्वा॒ भुव॑नानि॒ यस् मि॒न्थ् सꣳ सौ भ॑गानि दधि॒रे पा॑व॒के॥\\
\\
तुभ्यं ता, अ॑ङ्गिरस्तम॒ विश्वाः᳚ सुक् क्षि॒तयः॒ पृथ॑क्।\\
अग्ने॒ कामा॑य येमिरे॥\\
\\
अ॒श्याम॒ तंकाम॑ मऽग्ने॒ तवो॒त् य॑श्याम॑ र॒यिꣳ र॑यिवः सु॒वीरम्᳚\\
अ॒श्याम॒ वाज॑म॒भि वा॒जय॑न् तो॒ऽश्याम॑द्  द्यु॒म्न म॑ज रा॒जर॑न्ते॥\\
\\
श्रेष्ठं॑य विष्ठ भार॒ ताऽग्ने᳚ द्युमन्त॒ माभ॑र।\\
वसो॑ पुरु॒स् पृहꣳ॑ र॒यिम्॥\\
 \\
सश्वि॑ ता॒नस् त॑न्य॒तू रो॑च न॒स्था, अ॒जरे॑भि॒र् नान॑दद् भि॒र् यवि॑ष्ठः।\\
यः पा॑व॒कः पु॑रु॒तमः॑ पुरू॒णि॑ पृ॒थून् य॒ग्नि र॑नु॒याति॒ भर्वन्न्॥\\
\\
आयु॑ष्टे वि॒श्वतो॑ दध द॒य म॒ग्निर् वरे᳚ण्यः।\\
पुन॑स्ते प्रा॒ण आऽय॑ति॒ परा॒ यक्ष्मꣳ॑ सुवामि ते॥\\
\\
आ॒यु॒र्दा, अ॑ग्ने ह॒विषो॑ जुषा॒णो घृ॒त प्र॑तीको घृ॒तयो॑ निरेधि।\\
घृ॒तं पी॒त्वा मधु॒ चारु॒ गव्यं॑ पि॒तेव॑ पु॒त्रम॒भि र॑क्ष तादि॒मम्।॥\\
\\
तस्मै॑ ते प्रति॒हर्य॑ते॒ जात॑वेदो॒ विच॑र्षणे।\\
अग्ने॒ जना॑मि सुष्टु॒ तिम्॥\\
\\
दि॒वस्परि॑ प्रथमं ज॑ज्ञे, अ॒ग्नि र॒स्मद् द्वि॒तीयं॒ परि॑ जा॒तवे॑दाः।\\
तृ॒तीय॑ म॒प्सु नृ॒मणा॒, अज॑स्र॒ मिन्धा॑न एनं जरते स्वा॒धीः॥\\
\\
शुचिः॑ पावक॒ वन्द्योऽग्ने॑ बृहद् विरो॑चसे।\\
त्वं घृ॒ते भि॒राहु॑तः॥\\
\\
दृ॒शा॒नो रु॒क्म उ॒र्व्याव् य॑द् यौद् दु॒र्मर् ष॒ मायुः॑ श्रि॒ये रु॑चा॒नः।\\
अ॒ग्नि र॒मृतो॑, अभव॒द् वयो॑भिः यदे॑ नं॒द्यौर ज॑नयत् सु॒रेताः᳚॥\\
\\
आ यदि॒षे नृ॒पतिं॒ तेज॒ आन॒ट् शुचि॒ रेतो॒ निषिक्तं॒ द्यौर॒भीके᳚।\\
अ॒ग्निः शर्ध मनव॒द्यं युवा॑नग्ग् स्वा॒ धियं जनयत् सू॒दय॑च्च ॥\\
\\
सते जीयसा॒ मन॑सा॒ त्वोत॑ उ॒त शि॑क्षस्-वप्-प्र॒त्-यस्य॑ शि॒क्षोः।\\
अग्ने॑ रा॒यो नृत॑ मस्य॒ प्रभू॑तौ भू॒याम॑ ते सुष्टुत य॑श्च॒ वस्वः॑।\\
\\
अग्ने॒ सह॑न्त॒ माभ॑र द्यु॒म्-नस्य॑  प्रा॒सहा॑ र॒यिम्।\\
विश्वा॒ यश् च॑र्ष॒णी॒ रभ्या॑सा वाजे॑षु सा॒सह॑त्।\\
\\
तम॑ग्ने पृतना॒सहꣳ॑ र॒यिꣳ स॑हस्व॒ आ भ॑र।\\
त्वꣳ हि स॒त्यो, अद्भु॑तो दा॒ता वाज॑स्य गोम॑तः॥\\
\\
उ॒क्षान्ना॑य व॒शान्ना॑य॒ सोम॑ पृष्ठाय वे॒धसे᳚।\\
स्तोमै᳚र् विधे मा॒ऽग् नये᳚॥\\
\\
व॒द्मा हि सू॑नो॒, अस्य॑द् म॒सद् वा॑ च॒क्रे, अ॒ग्निर् ज॒नु षाऽज् मान्नम्᳚।\\
सत्वन्न॑ ऊर् जसन॒ ऊर्जं॑ धा॒ राजे॑ वजे रवृ॒के क्षे᳚ष् य॒न्तः॥\\
\\
अग्न॒ आयूꣳ॑ षि पवस॒ आसु॒ वोर् ज॒मिषं॑ च नः।\\
आ॒रे बा॑धस्व दु॒च्छु ना᳚म्॥\\
\\
अग्ने॒ पव॑स् व॒स् वपा॑, अ॒स्मे वर्चः॑॑ सु॒वीर्यम्᳚\\
दध॒त्पोषꣳ॑ र॒यिं मयि॑॥\\
\\
अग्ने॑ पावक रो॒चिषा॑ म॒न्द्रया॑ देव जिह्वया᳚।\\
आ दे॒वान् व॑क्षि॒ यक्षि॑ च॥\\
\\
स नः॑ पावक दीदि॒ वोऽग्ने॑ दे॒वाꣳ इ॒हाऽऽव॑ह।\\
उप॑ य॒ज्ञꣳ ह॒विश् च॑नः॥\\
\\
अ॒ग्निः शुचि॑व् व्रत तमः॒ शुचि॒र् विप्रः॒ शुचिः॑ क॒विः।\\
शुची॑ रोचत॒ आहु॑तः॥\\
\\
उद॑ग्ने॒ शुच॑ य॒स्तव॑ शु॒क्रा, भ्राज॑न्त ईरते।\\
तव॒ ज्योतीग्ग्॑ष् य॒र्चयः॑॥\\
\\
त्वम॑ग्ने रु॒द्रो असु॑रो म॒हो दि॒वः। त्वꣳ शर्धो॒ मारु॑तं पृ॒क्ष ई॑शिषे।\\
त्वं वातै॑ररु॒णैर्या॑सि शङ्ग॒यः। त्वं पू॒षा वि॑ध॒तः पा॑सि॒ नु त्मना᳚।\\
\\
देवा॑ दे॒वेषु॑ श्रयद्ध्वम्। प्रथ॑मा द्वि॒तीये॑षु श्रयद्ध्वम्। \\
द्विती॑यास् तृ॒तीये॑षु श्रयद्ध्वम्। तृती॑याश् चतु॒र्थेषु॑ श्रयद्ध्वम्। \\
च॒तु॒र्थाः प॑श्च॒मेषु॑ श्रयद्ध्वम्। प॒ञ्च॒माः ष॒ष् ठेषु॑ श्रयद्ध्वम्।\\
\\
ष॒ष्ठाः स॑प्त॒मेषु॑ श्रयद्ध्वम्। स॒प्त॒मा, अ॑ष्ट॒मेषु॑ श्रयद्ध्वम्। \\
अ॒ष्ट॒मा-न॑व॒मेषु॑ श्रयद्ध्वम्। न॒व॒मा-द॑श॒मेषु॑ श्रयद्ध्वम्। \\
द॒श॒मा, ए॑का द॒शेषु॑ श्रयद्ध्वम्। ए॒का॒द॒शा द्वा॑ द॒शेषु॑ श्रयद्ध्वम्। \\
द्वा॒ द॒शास् त्र॑यो द॒शेषु॒॑ श्रयद्ध्वम्। त्र॒यो॒ द॒शाश् च॑तर् द॒शेषु॑ श्रयद्ध्वम्। \\
च॒त॒र् द॒शाः प॑ञ्च द॒शेषु॑ श्रयद्ध्वम्। प॒ञ्च॒ द॒शा: षो॑ड॒शेषु॑ श्रयद्ध्वम्। \\
षो॒ड॒शाः स॑प्त द॒शेषु॑ श्रयद्ध्वम्। स॒प्त॒ द॒शा, अ॑ष्टा द॒शेषु॑ श्रयद्ध्वम्।\\
\\
अ॒ष्टा॒द॒शा, ए॑कान् नवि॒ꣳ॒ शेषु॑ श्रयद्ध्वम्। ए॒का॒न् न॒विꣳ॒ शा वि॒ꣳ॒शेषु॑ श्रयद्ध्वम्। \\
वि॒ꣳ॒शा, ए॑क वि॒ꣳ॒ शेषु॑ श्रयद्ध्वम्। ए॒क॒वि॒ꣳ॒ शा द्वा॑ वि॒ꣳ॒शेषु॑ श्रयद्ध्वम्। \\
द्वा॒ वि॒ꣳ॒शास् त्र॑योवि॒ꣳ॒ शेषु॑ श्रयद्ध्वम्। त्र॒यो॒वि॒ꣳ॒ शाश् च॑तर् वि॒ꣳ॒ शेषु॑ श्रयद्ध्वम्। \\
च॒त॒र् वि॒ꣳ॒शाः प॑ञ्चवि॒ꣳ॒ शेषु॑ श्रयद्ध्वम्। प॒ञ्च॒वि॒ꣳ॒ शाः ष॑ड् वि॒ꣳ॒शेषु॑ श्रयद्ध्वम्। \\
ष॒ड् विꣳ॒शाः स॑प्त वि॒ꣳ॒ शेषु॑ श्रयद्ध्वम्। स॒प्त॒ विꣳ॒शा, अ॑ष्टा वि॒ꣳ॒ शेषु॑ श्रयद्ध्वम्। \\
\\
अ॒ष्टा॒ वि॒ꣳ॒ शा, ए॑कान् नत्रि॒ꣳ॒ शेषु॑ श्रयद्ध्वम्। ए॒का॒न् न॒त्रि॒ꣳ शास् त्रि॒ꣳ॒ शेषु॑ श्रयद्ध्वम्। \\
त्रि॒ꣳ॒ शा, ए॑कत्रि॒ꣳ॒ शेषु॑ श्रयद्ध्वम्। ए॒क॒त्रि॒ꣳ॒ शा द्वा᳚ त्रि॒ꣳ॒ शेषु॑ श्रयद्ध्वम्।\\
द्वा॒त्रि॒ꣳ॒ शास् त्र॑यस् त्रि॒ꣳ॒ शेषु॑ श्रयद्ध्वम् । \\
\\
देवा᳚स् त्रिरेका दशा॒स् त्रिस् त्र॑यस् त्रिꣳशाः। \\
उत्त॑रे भवत । उत्त॑र वर्त् मान॒ उत्त॑र सत्वानः। यत्का॑म इ॒दं जु॒होमि॑।\\
तन्मे॒ समृ॑द् यताम्। व॒यग्ग् स्या॑म॒ पत॑यो रयी॒ णाम्। भूर्भुवः॒ स्वः॑ स्वाहा᳚॥\\
\\
अस्त्रायफट्\\

\subsection{\eng{Panchanga sagrucchajapeth}}
स॒द्योजा॒तं प्र॑पद्या॒मि॒ स॒द्योजा॒ताय॒ वै नमो॒ नमः॑। \\
भ॒वे भ॑वे॒ नाति॑भवे भवस्व॒ माम्। भ॒वोद्भ॑वाय॒ नमः॑॥\\
\\
वा॒म॒दे॒वाय॒ नमो᳚ ज्ये॒ष्ठाय॒ नम॑श्श्रे॒ष्ठाय॒ नमो॑ रु॒द्राय॒ नमः॒ काला॑य॒ नमः॒ कल॑विकरणाय॒\\
नमो॒ बल॑विकरणाय॒ नमो॒ बला॑य॒ नमो॒ बल॑प्रमथनाय॒ नम॒स्सर्व॑भूतदमनाय॒ नमो॑\\
म॒नोन्म॑नाय॒ नमः॑॥\\
\\
अ॒घोरे᳚भ्योऽथ॒ घोरे᳚भ्यो घोर॒घोर॑तरेभ्यः। \\
स॒र्वे᳚तः॑ सर्व॒ शर्वे᳚भ्यो॒ नम॑स्ते अस्तु रु॒द्ररू॑पेभ्यः॥\\
\\
तत्पुरु॑षाय वि॒द्महे॑ महादे॒वाय॑ धीमहि। तन्नो॑ रुद्रः प्रचो॒दया᳚त्॥\\
\\
ईशानस्सर्व॑विद्या॒नां ईश्वरस्सर्व॑भूता॒नां॒ ब्रह्माधि॑पति॒-\\
रब्रह्म॒णोऽधि॑पति॒-रब्रह्मा॑ शि॒वोमे॑ अस्तु सदाशि॒वोम्॥\\
\subsubsection{\eng{Alternate}}
ह॒ग्ं॒ सश् शुचि॒ षद् वसुरन् तरि क्ष॒सद् धोता वेदि॒ष दति थिर् दरो ण॒सत्।\\
नृ॒षद् वर॒सद् रुत सब् यो मसदब् जा गोजा, ऋतजा, अद्रिजा, ऋतं बृहत्॥ \\
\\
प्रतद् विष्णु॑स्स्तवते वी॒र्या॑य । मृ॒गो न भी॒मः कु॑च॒रो गि॑रि॒ष्ठाः । \\
यस्यो॒रुषु॑ त्रि॒षु वि॒क्रम॑णेषु । अधि॑क्षि॒यन्ति॒ भुव॑नानि॒ विश्वा᳚  । \\
\\
त्र्य॑म्बकं यजामहे सुग॒न्धिं पु॑ष्टि॒वर्ध॑नम् ।\\
उ॒र्वा॒रु॒कमि॑व॒ बन्ध॑नान्मृ॒त्योर्मु॑क्षीय॒ माऽमृता᳚त् ।\\
\\
तत्स॑ वि॒तुर् वृ॑णीमहे । व॒यं दे॒वस्य॒ भोज॑नम् । \\
श्रेष्ठꣳ॑  सर्व॒ धात॑मं । तुरं॒ भग॑स्य धीमहि ॥\\
\\
विष्णु॒र् योनिं॑ कल्पयतु । त्वष्टा॑ रु॒पाणि॑ पिꣳशतु । \\
आसि॑ चतु प्र॒जाप॑तिः । धा॒ता गर्भं॑ दधातु मे ॥\\
\subsection{\eng{Ashtanga Pranamaha}}
हि॒र॒ण्य॒ग॒र्भः सम॑वर्त॒ताग्रे॑ भू॒तस्य॑ जा॒तः पति॒रेक॑ आसीत्।\\
स दाधा॑र पृथि॒वीं द्यामु॒तेमां कस्मै॑ दे॒वाय ह॒विषा॑ विधेम॥\\
ॐ उमा महेश्वराभ्यां नमः     (1)\\
\\
यः प्रा॑ण॒तो नि॑मिष॒तो म॑हि॒त् वै॒क इद्राजा॒ जग॑तो ब॒भूव॑।\\
य ईशे॑, अ॒स्यद् द्वि॒पद॒श् चतु॑ष्पदः॒ कस्मै॑ दे॒वाय॑ ह॒विषा॑ विधेम॥\\
ॐ उमा महेश्वराभ्यां नमः     (2)\\
\\
ब्रह्म॑ जज्ञा॒नं प्र॑थ॒मं पु॒रस्ता॒द् विसी॑ म॒तस् सु॒रुचो॑ वे॒न आ॑वः।\\
स बु॒ध् निया॑, उप॒मा, अ॑स्य वि॒ष्टास् स॒तश्च॒ यो॒नि मस॑ तश्च॒ विवः॑॥\\
ॐ उमा महेश्वराभ्यां नमः     (3)\\
\\
म॒ही द्यौः पृ॑थि॒वी च॑ न इ॒मं यज्ञं मि॑मिक्षताम्।\\
पि॒पृ॒तां नो॒-भरी॑ मभिः॥\\
ॐ उमा महेश्वराभ्यां नमः     (4)\\
\\
उप॑श् वासय पृथि॒वी मु॒तद् यां पु॑रु॒त्रा ते॑ मनुतां॒ विष्टि॑तं॒ जग॑त्।\\
स दुन्दु॑बे स॒जू रिन्द्रे॑ण दे॒वैर् दू॒राद् दवी॑यो॒, अप॑से ध॒शत् रून्॥\\
ॐ उमा महेश्वराभ्यां नमः     (5)\\
\\
अग्ने॒ नय॑ सु॒पता॑ रा॒ये, अ॒स्मान्. विश्वा॑नि देव व॒युना॑नि वि॒द्वान्।\\
यु॒यो॒ध् य॑स्मज् जु॑हु-रा॒ण-मेनो॒ भूयि॑ष् ठान्ते॒ नम॑ उक्तिं विधेम॥\\
ॐ उमा महेश्वराभ्यां नमः     (6)\\
 \\
या ते॑, अग्ने॒ रुद्रि॑या त॒नूस् तया॑नः पाहि॒ तस्या᳚स्ते॒ स्वाहा॒ याते॑,\\
अग्नेऽ याश॒या र॑जाश॒या ह॑राश॒या त॒नूर् वर्-षि॑ष्ठा-गह्वरे॒ष्-ठोग्रं वचो॒,\\
अपा वधीं त्वे॒षव् वचो॒, अपा॑ वधी॒ग् स्वाहा᳚॥   \\
{\small या ते॑ अग्ने॒ रुद्रि॑या त॒नूस्तया॑ नः पाहि॒ तस्या᳚स्ते॒ स्वाहा᳚।}\\
ॐ उमा महेश्वराभ्यां नमः     (7)\\
\\
इ॒मं य॑मप् प्रस् त॒र माहि सीदाङ्-गि॑रोभिः\eng{f} पि॒तृभिः॑स् संविदा॒नः।\\
आत्वा॒ मन्त्राः᳚ कवि श॒स्ता व॑हन्-त्वे॒ना रा॑जन्. ह॒विषा॑ माद यस्व॥\\
ॐ उमा महेश्वराभ्यां नमः     (8)\\
\\
उरसा शिरसा दृष्ट्या मनसा वचसा तथा।\\
पद्भ्यां  कराभ्यां कर्णाभ्यां प्रणा मोष्टाङ्ग उच्यते॥\\
\\

\section{\eng{Kalasheshu Dhyanam}}
ध्यायेन्निरा मयं वस्तु सर्गस् थितिल यादिकं ।\\
निर्गुणन् निष्कळन् नित्यं मनो वाचा मगोचरं ॥ 1\\
\\
गंगाधरं शशिधरं जटामकुट शोभितं ।\\
श्वेत भूति त्रिपुण्ड्रेण विरा जित ललाटकं ॥ 2\\
\\
लोचन त्रय संपन्नं स्वर्ण कुण्डल शोभितं\\
स्मेरा ननञ् चतुर्बाहुं मुक्ता हारोप शोभितं ॥ 3\\
\\
अक्ष मालां सुधा कुंभञ् चिन्मयीं मुद्रिका मपि\\
पुस्तकं च भुजैर् दिव्यैर् दधानं पार्वती पतिं ॥ 4\\
\\
श्वेतां बरधरं श्वेतं रत्न सिंहास नस्थितं\\
सर्वा भीष्ट प्रदा तारं वट मूल निवासिनं ॥ 5\\
\\
वामांगे संस्थितां गौरीं बालार्कायुत सन्निभां\\
जपा कुसुम साहस्र समा नश्रिय-मीश्वरीं ॥ 6\\
\\
सुवर्ण रत्न खचित मकुटेन विराजितां\\
ललाटप॒ संराजत् सल्लग्नति लकाञ्चितां ॥ 7\\
\\
राजीवायत नेत्रान्तां नीलोत्पल दळेक्षणां\\
सन्तप्त हेम रचित ताटङ्का भरणान्वितां ॥ 8\\
\\
तांबूल चर्व णरत रक्त जिह्वा विराजितां\\
पताका भरणो पेतां मुक्ता हारोप शोभितां ॥ 9\\
\\
स्वर्ण कङ्कण सय्युक्तैश् चतुर् भिर् बाहु भिर्युतां ।\\
सुवर्ण रत्न खचित काञ्ची दाम विराजितां ॥ 10\\
\\
कद लील लितस्तंभ सन्नि भोरुयु गान्वितां\\
श्रिया विराजित पदां भक्त त्राण परायणां ॥ 11\\
\\
अन्योन्याश् लिष्ट हृद् बाहु गौरी शङ्कर संज्ञकं\\
सनातनं परं ब्रह्म परमात्मान मव्ययं ॥ 12\\
\\
आवाह यामि जगता मीश्वरं परमेश्वरं ।\\
मंगला यतनं देवं युवान मतिसुन्दरं ।\\
ध्यायेत् कल्पत रोर्मूले सुखासीनं सहोमया ॥ 13\\
\\
आगच्छाऽऽ गच्छ भगवन् देवेश परमेश्वर ।\\
सच्चिदानन्द भूतेश पार्वती च नमोऽस्तुते\\
\\
आत्वा॑ वहन्तु॒ हर॑य॒स्सचे॑तसः श्वे॒तैरश्वै᳚ स्स॒ह के॑तु॒मद्भिः॑ ।\\
वाता॑जितै॒ र्बल॑वद्भि॒ र्मनो॑जवै॒ राया॑हि शी॒घ्रं मम॑ ह॒व्याय॑ श॒र्वों ।\\
\\
त्र्यं॑बकंँयजामहे सुग॒न्धिं पु॑ष्टि॒वर्ध॑नं ।\\
उ॒र्वा॒रु॒कमि॑व॒ बन्ध॑नान् मृ॒त्योर्मु॑क्षीय॒ माऽमृता᳚त् ।\\
\\
गौ॒रीमि॑माय सलि॒लानि॒ तक्ष॒त्येक॑पदी द्वि॒पदि॒ सा चतु॑ष्पदी ।\\
अ॒ष्टाप॑दी॒ नव॑पदी बभू॒वुषी॑ स॒हस्रा᳚क्षरा पर॒मे व्यो॑मन् ।\\
\section{\eng{Laghunyasam}}
\\
ॐ अथात्मानग्ं शिवात्मानं श्री रुद्ररूप-न्ध्यायेत् ॥\\
\\
शुद्धस्फटिक सङ्काशन् त्रिनेत्रम् पञ्च वक्त्रकम् ।\\
गङ्गाधरन् दशभुजं सर्वा भरण भूषितम् ॥\\
\\
नीलग्रीवं शशाङ्काङ्कन् नाग यज्ञोप वीतिनम् ।\\
व्याघ्र चर्मोत् तरीयञ्च वरेण्य मभय प्रदम् ॥\\
\\
कमण्डल्-वक्ष सूत्राणान् धारिणं शूल पाणिनम् ।\\
ज्वलन्तम् पिङ्ग लजटा शिखा मुद्द्योत धारिणम् ॥\\
\\
वृष स्कन्ध समारूढं उमा देहार्थ धारिणम् ।\\
अमृतेनाप् लुतं शान्तन् दिव्य भोग समन्वितम् ॥\\
\\
दिग् देवता समा युक्तं सुरासुर नमस्कृतम् ।\\
नित्यञ् चशाश्व तं शुद्धन् ध्रुव-मक्षर-मव्ययम् ।\\
सर्व व्यापिन-मीशानं रुद्रं-वै विश्वरूपिणम् ।\\
एवन् ध्यात्वा द्विजस् सम्यक् ततो यजनमारभेत् ॥\\
\\
अथातो रुद्रस् नानार् चना भिषेक विधिं-व्या᳚ क्ष्यास्यामः ।\\
आदित एव तीर्थेस् नात्वा,\\
उदेत्य शुचिः प्रयतो ब्रह्मचारी शुक्लवासा देवाभिमुख-स्स्थित्वा,\\
आत्मनि देवता-स्स्थापयेत् ॥\\
\\
प्रजनने ब्रह्मा तिष्ठतु ।\\
पादयोर् विष्णुस्तिष्ठतु ।\\
हस्तयोर्​ हरस्तिष्ठतु ।\\
बाह्वो रिन्द्रस्तिष्टतु ।\\
जठरे-ऽअग्निस्तिष्ठतु ।\\
हृद॑ये शिवस्तिष्ठतु ।\\
कण्ठे वसवस्तिष्ठन्तु ।\\
वक्त्रे सरस्वती तिष्ठतु ।\\
नासिकयोर्-वायुस्तिष्ठतु ।\\
नयनयोश्-चन्द्रा दित्यौ तिष्टेताम् ।\\
कर्णयो रश्विनौ तिष्टेताम् ।\\
ललाटे रुद्रास्तिष्ठन्तु ।\\
मूर्थ्-न्यादित्-यास्तिष्ठन्तु ।\\
शिरसि महादेवस्तिष्ठतु ।\\
शिखायां-वाँमदेवास्तिष्ठतु ।\\
पृष्ठे पिनाकी तिष्ठतु ।\\
पुरतश्-शूली तिष्ठतु ।\\
पार्​श् वयोश् शिवा शङ्करौ तिष्ठेताम् ।\\
सर्वतो वायुस्तिष्ठतु ।\\
ततो बहिस् सर्वतो-ऽग्निर् ज्वाला माला-परि वृतस्तिष्ठतु ।\\
सर्वेष् वङ्गेषु सर्वा देवता यथास्थानन्-तिष्ठन्तु ।\\
माग्ं रक्षन्तु ।\\
\\
अ॒ग्निर्मे॑ वा॒चि श्रि॒तः । वाघृद॑ये । हृद॑य॒-म्मयि॑ । अ॒हम॒मृते᳚ । अ॒मृत॒-म्ब्रह्म॑णि ।\\
वा॒युर्मे᳚ प्रा॒णे श्रि॒तः । प्रा॒णो हृद॑ये । हृद॑य॒-म्मयि॑ । अ॒हम॒मृते᳚ । अ॒मृत॒-म्ब्रह्म॑णि ।\\
सूर्यो॑ मे॒ चक्षुषि श्रि॒तः । चक्षु॒र्​ हृद॑ये । हृद॑य॒-म्मयि॑ । अ॒हम॒मृते᳚ । अ॒मृत॒-म्ब्रह्म॑णि ।\\
च॒न्द्रमा॑ मे॒ मन॑सि श्रि॒तः । मनो॒ हृद॑ये । हृद॑य॒-म्मयि॑ । अ॒हम॒मृते᳚ । अ॒मृत॒-म्ब्रह्म॑णि ।\\
दिशो॑ मे॒ श्रोत्रे᳚ श्रि॒ताः । श्रोत्र॒ग्ं॒ हृद॑ये । हृद॑य॒-म्मयि॑ । अ॒हम॒मृते᳚ । अ॒मृत॒-म्ब्रह्म॑णि ।\\
आपोमे॒ रेतसि श्रि॒ताः । रेतो हृद॑ये । हृद॑य॒-म्मयि॑ । अ॒हम॒मृते᳚ । अ॒मृत॒-म्ब्रह्म॑णि ।\\
पृ॒थि॒वी मे॒ शरी॑रे श्रि॒ता । शरी॑र॒ग्ं॒ हृद॑ये । हृद॑य॒-म्मयि॑ । अ॒हम॒मृते᳚ । अ॒मृत॒-म्ब्रह्म॑णि ।\\
ओ॒ष॒धि॒ व॒न॒स्पतयो॑ मे॒ लोम॑सु श्रि॒ताः । लोमा॑नि॒ हृद॑ये । हृद॑य॒-म्मयि॑ । \\
अ॒हम॒मृते᳚ । अ॒मृत॒-म्ब्रह्म॑णि ।\\
इन्द्रो॑ मे॒ बले᳚ श्रि॒तः । बल॒ग्ं॒ हृद॑ये । हृद॑य॒-म्मयि॑ । अ॒हम॒मृते᳚ । अ॒मृत॒-म्ब्रह्म॑णि ।\\
प॒र्जन्यो॑ मे॒ मू॒र्द्नि श्रि॒तः । मू॒र्धा हृद॑ये । हृद॑य॒-म्मयि॑ । अ॒हम॒मृते᳚ । अ॒मृत॒-म्ब्रह्म॑णि ।\\
ईशा॑नो मे॒ म॒न्यौ श्रि॒तः । म॒न्युर्​ हृद॑ये । हृद॑य॒-म्मयि॑ । अ॒हम॒मृते᳚ । अ॒मृत॒-म्ब्रह्म॑णि ।\\
आ॒त्मा म॑ आ॒त्मनि॑ श्रि॒तः । आ॒त्मा हृद॑ये । हृद॑य॒-म्मयि॑ । \\
अ॒हम॒मृते᳚ । अ॒मृत॒-म्ब्रह्म॑णि ।\\
पुन॑र्म आ॒त्मा पुन॒रा यु॒रागा᳚त् । पुनः॑ प्रा॒णः पुन॒रा कू॑त॒मागा᳚त् । \\
वै॒श्वा॒ न॒रो र॒श्मिभि॑र् वा   वृधा॒नः ।\\
अ॒न्तस् ति॑ष् ठ॒त्व मृत॑स्य गो॒पाः ॥\\
\\
अस्य श्री रुद्राध्याय प्रश्न महामन्त्रस्य,\\
अघोर ऋषिः,\\
अनुष्टु-प्छन्दः,\\
सङ्कर्​षण मूर्ति स्वरूपो यो-ऽसावादित्यः परमपुरुष-स्स एष रुद्रो देवता ।\\
नमश् शिवायेति बीजम् ।\\
शिवत रायेति शक्तिः ।\\
महा देवा येति कीलकम् ।\\
श्री साम्ब सदाशिव प्रसाद सिद्ध्यर्थे जपे विनियोगः ॥\\
\\
ॐ अग्निहोत्रात्मने अङ्गुष्ठाभ्या-न्नमः ।\\
दर्​शपूर्ण मासात्मने तर्जनीभ्या-न्नमः ।\\
चातुर्मास्यात्मने मध्यमाभ्या-न्नमः ।\\
निरूढ पशुबन्धात्मने अनामिकाभ्या-न्नमः ।\\
ज्योतिष्टोमात्मने कनिष्ठिकाभ्या-न्नमः ।\\
सर्वक्रत्वात्मने करतल करपृष्ठाभ्या-न्नमः ॥\\
\\
अग्निहोत्रात्मने हृदयाय नमः ।\\
दर्​शपूर्ण मासात्मने शिरसे स्वाहा ।\\
चातुर्मास्यात्मने शिखायै वषट् ।\\
निरूढ पशुबन्धात्मने कवचाय हुम् ।\\
ज्योतिष्टोमात्मने नेत्रत्रयाय वौषट् ।\\
सर्वक्रत्वात्मने अस्त्रायफट् । भूर्भुवस्सुवरोमिति दिग्बन्धः ॥\\
\\
ध्यानं\\
आपाताल-नभस्स्थलान्त-भुवन-ब्रह्माण्ड-माविस्फुरत्-\\
ज्योति-स्स्फाटिक-लिङ्ग-मौलि-विलसत्-पूर्णेन्दु-वान्तामृतैः ।\\
अस्तोकाप्लुत-मेक-मीश-मनिशं रुद्रानु-वाकाञ्जपन्\\
ध्याये-दीप्सित-सिद्धये ध्रुवपदं-विँप्रो-ऽभिषिञ्चे-च्चिवम् ॥\\
\\
ब्रह्माण्ड व्याप्तदेहा भसित हिमरुचा भासमाना भुजङ्गैः\\
कण्ठे कालाः कपर्दाः कलित-शशिकला-श्चण्ड कोदण्ड हस्ताः ।\\
त्र्यक्षा रुद्राक्षमालाः प्रकटितविभवा-श्शाम्भवा मूर्तिभेदाः\\
रुद्रा-श्श्रीरुद्रसूक्त-प्रकटितविभवा नः प्रयच्चन्तु सौख्यम् ॥\\
\\
ओ-ङ्ग॒णाना᳚-न्त्वा ग॒णप॑तिग्ं हवामहे क॒वि-ङ्क॑वी॒नामु॑प॒मश्र॑वस्तमम् ।\\
ज्ये॒ष्ठ॒राज॒-म्ब्रह्म॑णा-म्ब्रह्मणस्पद॒ आ नः॑ शृ॒ण्वन्नू॒तिभि॑स्सीद॒ साद॑नम् ॥\\
महागणपतये॒ नमः ॥\\
\\
श-ञ्च॑ मे॒ मय॑श्च मे प्रि॒य-ञ्च॑ मे-ऽनुका॒मश्च॑ \\
मे॒ काम॑श्च मे सौमनस॒श्च॑ मे भ॒द्र-ञ्च॑ मे॒ \\
श्रेय॑श्च मे॒ वस्य॑श्च मे॒ यश॑श्च मे॒ भग॑श्च मे॒ \\
द्रवि॑ण-ञ्च मे य॒न्ता च॑ मे ध॒र्ता च॑ मे॒ क्षेम॑श्च मे॒ \\
धृति॑श्च मे॒ विश्व॑-ञ्च मे॒ मह॑श्च मे सं॒​विँच्च॑ मे॒ \\
ज्ञात्र॑-ञ्च मे॒ सूश्च॑ मे प्र॒सूश्च॑ मे॒ सीर॑-ञ्च मे \\
ल॒यश्च॑ म ऋ॒त-ञ्च॑ मे॒-ऽमृत॑-ञ्च मे-ऽय॒क्ष्म-ञ्च॒ \\
मे-ऽना॑मयच्च मे जी॒वातु॑श्च मे दीर्घायु॒त्व-ञ्च॑ \\
मे-ऽनमि॒त्र-ञ्च॒ मे-ऽभ॑य-ञ्च मे सु॒ग-ञ्च॑ मे॒ \\
शय॑न-ञ्च मे सू॒षा च॑ मे॒ सु॒दिन॑-ञ्च मे ॥\\
\\
ॐ शान्ति॒-श्शान्ति॒-श्शान्तिः॑ ॥\\

\section{श्री रुद्रप्रश्नः – नमकप्रश्नः}
ओं नमो भगवते॑ रुद्रा॒य ॥\\
\subsection{॥ प्रथम अनुवाक ॥}
ओं नम॑स्ते रुद्र म॒न्यव॑ उ॒तोत॒ इष॑वे॒ नम॑: ।\\
नम॑स्ते अस्तु॒ धन्व॑ने बा॒हुभ्या॑मु॒त ते॒ नम॑: ।\\
\\
या त॒ इषु॑: शि॒वत॑मा शि॒वं ब॒भूव॑ ते॒ धनु॑: ।\\
शि॒वा श॑र॒व्या॑ या तव॒ तया॑ नो रुद्र मृडय ।\\
\\
या ते॑ रुद्र शि॒वा त॒नूरघो॒राऽपा॑पकाशिनी ।\\
तया॑ नस्त॒नुवा॒ शन्त॑मया॒ गिरि॑शन्ता॒भिचा॑कशीहि ।\\
\\
यामिषुं॑ गिरिशन्त॒ हस्ते॒ बिभ॒र्ष्यस्त॑वे ।\\
शि॒वां गि॑रित्र॒ तां कु॑रु॒ मा हिग्ं॑सी॒: पुरु॑षं॒ जग॑त् ।\\
\\
शि॒वेन॒ वच॑सा त्वा॒ गिरि॒शाच्छा॑वदामसि ।\\
यथा॑ न॒: सर्व॒मिज्जग॑दय॒क्ष्मग्ं सु॒मना॒ अस॑त् ।\\
\\
अध्य॑वोचदधिव॒क्ता प्र॑थ॒मो दैव्यो॑ भि॒षक् ।\\
अहीग्॑श्च॒ सर्वा᳚ञ्ज॒म्भय॒न्त्सर्वा᳚श्च यातुधा॒न्य॑: ।\\
\\
अ॒सौ यस्ता॒म्रो अ॑रु॒ण उ॒त ब॒भ्रुः सु॑म॒ङ्गल॑: ।\\
ये चे॒माग्ं रु॒द्रा अ॒भितो॑ दि॒क्षु श्रि॒ताः\\
स॑हस्र॒शोऽवै॑षा॒ग्ं॒ हेड॑ ईमहे ।\\
\\
अ॒सौ यो॑ऽव॒सर्प॑ति॒ नील॑ग्रीवो॒ विलो॑हितः ।\\
उ॒तैनं॑ गो॒पा अ॑दृश॒न्नदृ॑शन्नुदहा॒र्य॑: ।\\
\\
उ॒तैनं॒ विश्वा॑ भू॒तानि॒ स दृ॒ष्टो मृ॑डयाति नः ।\\
नमो॑ अस्तु॒ नील॑ग्रीवाय सहस्रा॒क्षाय॑ मी॒ढुषे᳚ ।\\
\\
अथो॒ ये अ॑स्य॒ सत्त्वा॑नो॒ऽहं तेभ्यो॑ऽकर॒न्नम॑: ।\\
प्रमु॑ञ्च॒ धन्व॑न॒स्त्वमु॒भयो॒रार्त्नि॑यो॒र्ज्याम् ।\\
\\
याश्च॑ ते॒ हस्त॒ इष॑व॒: परा॒ ता भ॑गवो वप ।\\
अ॒व॒तत्य॒ धनु॒स्तवग्ं सह॑स्राक्ष॒ शते॑षुधे ।\\
\\
नि॒शीर्य॑ श॒ल्यानां॒ मुखा॑ शि॒वो न॑: सु॒मना॑ भव ।\\
विज्यं॒ धनु॑: कप॒र्दिनो॒ विश॑ल्यो॒ बाण॑वाग्ं उ॒त ।\\
\\
अने॑शन्न॒स्येष॑व आ॒भुर॑स्य निष॒ङ्गथि॑: ।\\
या ते॑ हे॒तिर्मी॑ढुष्टम॒ हस्ते॑ ब॒भूव॑ ते॒ धनु॑: ।\\
\\
तया॒ऽस्मान् वि॒श्वत॒स्त्वम॑य॒क्ष्मया॒ परि॑ब्भुज ।\\
नम॑स्ते अ॒स्त्वायु॑धा॒याना॑तताय धृ॒ष्णवे᳚ ।\\
\\
उ॒भाभ्या॑मु॒त ते॒ नमो॑ बा॒हुभ्यां॒ तव॒ धन्व॑ने ।\\
परि॑ते॒ धन्व॑नो हे॒तिर॒स्मान्वृ॑णक्तु वि॒श्वत॑: ।\\
\\
अथो॒ य इ॑षु॒धिस्तवा॒रे अ॒स्मन्निधे॑हि॒ तम् ॥\\
\\
नम॑स्ते अस्तु भगवन्विश्वेश्व॒राय॑ महादे॒वाय॑\\
त्र्यम्ब॒काय॑ त्रिपुरान्त॒काय॑ त्रिकाग्निका॒लाय॑\\
कालाग्निरु॒द्राय॑ नीलक॒ण्ठाय॑ मृत्युञ्ज॒याय॑\\
सर्वेश्व॒राय॑ सदाशि॒वाय॑ श्रीमन्महादे॒वाय॒ नम॑: ॥ 1 ॥\\
\subsection{॥ द्वितीय अनुवाक ॥}
नमो॒ हिर॑ण्यबाहवे सेना॒न्ये॑ दि॒शां च॒ पत॑ये॒ नमो॒{\small 1}\\
नमो॑ वृ॒क्षेभ्यो॒ हरि॑केशेभ्यः पशू॒नां पत॑ये॒ नमो॒{\small 2}\\
नम॑: स॒स्पिञ्ज॑राय॒ त्विषी॑मते पथी॒नां पत॑ये॒ नमो॒{\small 3}\\
नमो॑ बभ्लु॒शाय॑ विव्या॒धिनेऽन्ना॑नां॒ पत॑ये॒ नमो॒{\small 4}\\
नमो॒ हरि॑केशायोपवी॒तिने॑ पु॒ष्टानां॒ पत॑ये॒ नमो॒{\small 5}\\
नमो॑ भ॒वस्य॑ हे॒त्यै जग॑तां॒ पत॑ये॒ नमो॒{\small 6}\\
नमो॑ रु॒द्राया॑तता॒विने॒ क्षेत्रा॑णां॒ पत॑ये॒ नमो॒{\small 7}\\
नम॑: सू॒तायाह॑न्त्याय॒ वना॑नां॒ पत॑ये॒ नमो॒{\small 8}\\
नमो॒ रोहि॑ताय स्थ॒पत॑ये वृ॒क्षाणां॒ पत॑ये॒ नमो॒{\small 9}\\
नमो॑ म॒न्त्रिणे॑ वाणि॒जाय॒ कक्षा॑णां॒ पत॑ये॒ नमो॒{\small 10}\\
नमो॑ भुव॒न्तये॑ वारिवस्कृ॒तायौष॑धीनां॒ पत॑ये॒ नमो॒{\small 11}\\
नम॑ उ॒च्चैर्घो॑षायाक्र॒न्दय॑ते पत्ती॒नां पत॑ये॒ नमो॒{\small 12}\\
नम॑: कृत्स्नवी॒ताय॒ धाव॑ते॒ सत्त्व॑नां॒ पत॑ये॒ नम॑:{\small 13}॥2॥\\
\subsection{॥ तृतीय अनुवाक ॥}
नम॒: सह॑मानाय निव्या॒धिन॑ आव्या॒धिनी॑नां॒ पत॑ये॒ नमो॒{\small 1}\\
नम॑: ककु॒भाय॑ निष॒ङ्गिणे᳚ स्ते॒नानां॒ पत॑ये॒ नमो॒{\small 2}\\
नमो॑ निष॒ङ्गिण॑ इषुधि॒मते॒ तस्क॑राणां॒ पत॑ये॒ नमो॒{\small 3}\\
नमो॒ वञ्च॑ते परि॒वञ्च॑ते स्तायू॒नां पत॑ये॒ नमो॒{\small 4}\\
नमो॑ निचे॒रवे॑ परिच॒रायार॑ण्यानां॒ पत॑ये॒ नमो॒{\small 5}\\
नम॑: सृका॒विभ्यो॒ जिघाग्ं॑सद्भ्यो मुष्ण॒तां पत॑ये॒ नमो॒{\small 6}\\
नमो॑ऽसि॒मद्भ्यो॒ नक्त॒ञ्चर॑द्भ्यः प्रकृ॒न्तानां॒ पत॑ये॒ नमो॒{\small 7}\\
नम॑ उष्णी॒षिणे॑ गिरिच॒राय॑ कुलु॒ञ्चानां॒ पत॑ये॒ नमो॒{\small 8}\\
नम॒ इषु॑मद्भ्यो धन्वा॒विभ्य॑श्च वो॒ नमो॒{\small 9}\\
नम॑ आतन्वा॒नेभ्य॑: प्रति॒दधा॑नेभ्यश्च वो॒ नमो॒{\small 10}\\
नम॑ आ॒यच्छ॑द्भ्यो विसृ॒जद्भ्य॑श्च वो॒ नमो॒{\small 11}\\
नमो ऽस्य॑द्भ्यो॒ विध्य॑द्भ्यश्च वो॒ नमो॒{\small 12}\\
नम॒ आसी॑नेभ्य॒: शया॑नेभ्यश्च वो॒ नमो॒{\small 13}\\
नम॑: स्व॒पद्भ्यो॒ जाग्र॑द्भ्यश्च वो॒ नमो॒{\small 14}\\
नमस्ति॒ष्ठ॑द्भ्यो॒ धाव॑द्भ्यश्च वो॒ नमो॒{\small 15}\\
नम॑: स॒भाभ्य॑: स॒भाप॑तिभ्यश्च वो॒ नमो॒{\small 16}\\
नमो॒ अश्वे॒भ्योऽश्व॑पतिभ्यश्च वो॒ नम॑:{\small 17} ॥ 3 ॥\\
\subsection{॥ चतुर्थ अनुवाक ॥}
नम॑ आव्या॒धिनी᳚भ्यो वि॒विध्य॑न्तीभ्यश्च वो॒ नमो॒{\small 1}\\
नम॒ उग॑णाभ्यस्तृग्ंह॒तीभ्य॑श्च वो॒ नमो॒{\small 2}\\
नमो॑ गृ॒त्सेभ्यो॑ गृ॒त्सप॑तिभ्यश्च वो॒ नमो॒{\small 3}\\
नमो॒ व्राते᳚भ्यो॒ व्रात॑पतिभ्यश्च वो॒ नमो॒{\small 4}\\
नमो॑ ग॒णेभ्यो॑ ग॒णप॑तिभ्यश्च वो॒ नमो॒{\small 5}\\
नमो॒ विरू॑पेभ्यो वि॒श्वरू॑पेभ्यश्च वो॒ नमो॒{\small 6}\\
नमो॑ म॒हद्भ्य॑:, क्षुल्ल॒केभ्य॑श्च वो॒ नमो॒{\small 7}\\
नमो॑ र॒थिभ्यो॑ऽर॒थेभ्य॑श्च वो॒ नमो॒{\small 8}\\
नमो॒ रथे᳚भ्यो॒ रथ॑पतिभ्यश्च वो॒ नमो॒{\small 9}\\
नम॒: सेना᳚भ्यः सेना॒निभ्य॑श्च वो॒ नमो॒{\small 10}\\
नम॑:, क्ष॒त्तृभ्य॑: सङ्ग्रही॒तृभ्य॑श्च वो॒ नमो॒{\small 11}\\
नम॒स्तक्ष॑भ्यो रथका॒रेभ्य॑श्च वो॒ नमो॒{\small 12}\\
नम॒: कुला॑लेभ्यः क॒र्मारे᳚भ्यश्च वो॒ नमो॒{\small 13}\\
नम॑: पु॒ञ्जिष्टे᳚भ्यो निषा॒देभ्य॑श्च वो॒ नमो॒{\small 14}\\
नम॑ इषु॒कृद्भ्यो॑ धन्व॒कृद्भ्य॑श्च वो॒ नमो॒{\small 15}\\
नमो॑ मृग॒युभ्य॑: श्व॒निभ्य॑श्च वो॒ नमो॒{\small 16}\\
नम॒: श्वभ्य॒: श्वप॑तिभ्यश्च वो॒ नम॑:{\small 17} ॥ 4 ॥\\
\subsection{॥ पञ्चम अनुवाक ॥}
नमो॑ भ॒वाय॑ च रु॒द्राय॑ च॒{\small 1} \\
नम॑: श॒र्वाय॑ च पशु॒पत॑ये च॒{\small 2}\\
नमो॒ नील॑ग्रीवाय च शिति॒कण्ठा॑य च॒{\small 3}\\
नम॑: कप॒र्दिने॑ च॒ व्यु॑प्तकेशाय च॒{\small 4}\\
नम॑: सहस्रा॒क्षाय॑ च श॒तध॑न्वने च॒{\small 5}\\
नमो॑ गिरि॒शाय॑ च शिपिवि॒ष्टाय॑ च॒{\small 6}\\
नमो॑ मी॒ढुष्ट॑माय॒ चेषु॑मते च॒{\small 7}\\
नमो᳚ ह्र॒स्वाय॑ च वाम॒नाय॑ च॒{\small 8}\\
नमो॑ बृह॒ते च॒ वर्षी॑यसे च॒{\small 9}\\
नमो॑ वृ॒द्धाय॑ च सं॒वृध्व॑ने च॒{\small 10}\\
नमो॒ अग्रि॑याय च प्रथ॒माय॑ च॒{\small 11}\\
नम॑ आ॒शवे॑ चाजि॒राय॑ च॒{\small 12}\\
नम॒: शीघ्रि॑याय च॒ शीभ्या॑य च॒{\small 13}\\
नम॑ ऊ॒र्म्या॑य चावस्व॒न्या॑य च॒{\small 14}\\
नम॑: स्रोत॒स्या॑य च॒ द्वीप्या॑य च{\small 15} ॥ 5 ॥\\
\subsection{॥ षष्ठम अनुवाक ॥}
नमो᳚ ज्ये॒ष्ठाय॑ च कनि॒ष्ठाय॑ च॒{\small 1}\\
नम॑: पूर्व॒जाय॑ चापर॒जाय॑ च॒{\small 2}\\
नमो॑ मध्य॒माय॑ चापग॒ल्भाय॑ च॒{\small 3}\\
नमो॑ जघ॒न्या॑य च॒ बुध्नि॑याय च॒{\small 4}\\
नम॑: सो॒भ्या॑य च प्रतिस॒र्या॑य च॒{\small 5}\\
नमो॒ याम्या॑य च॒ क्षेम्या॑य च॒{\small 6}\\
नम॑ उर्व॒र्या॑य च॒ खल्या॑य च॒{\small 7}\\
नम॒: श्लोक्या॑य चाऽवसा॒न्या॑य च॒{\small 8}\\
नमो॒ वन्या॑य च॒ कक्ष्या॑य च॒{\small 9}\\
नम॑: श्र॒वाय॑ च प्रतिश्र॒वाय॑ च॒{\small 10}\\
नम॑ आ॒शुषे॑णाय चा॒शुर॑थाय च॒{\small 11}\\
नम॒: शूरा॑य चावभिन्द॒ते च॒{\small 12}\\
नमो॑ व॒र्मिणे॑ च वरू॒थिने॑ च॒{\small 13}\\
नमो॑ बि॒ल्मिने॑ च कव॒चिने॑ च॒{\small 14}\\
नम॑: श्रु॒ताय॑ च श्रुतसे॒नाय॑ च{\small 15} ॥ 6 ॥\\
\subsection{॥ सप्तम अनुवाक ॥}
नमो॑ दुन्दु॒भ्या॑य चाहन॒न्या॑य च॒{\small 1}\\
नमो॑ धृ॒ष्णवे॑ च प्रमृ॒शाय॑ च॒{\small 2}\\
नमो॑ दू॒ताय॑ च॒ प्रहि॑ताय च॒{\small 3}\\
नमो॑ निष॒ङ्गिणे॑ चेषुधि॒मते॑ च॒{\small 4}\\
नम॑स्ती॒क्ष्णेष॑वे चायु॒धिने॑ च॒{\small 5}\\
नम॑: स्वायु॒धाय॑ च सु॒धन्व॑ने च॒{\small 6}\\
नम॒: स्रुत्या॑य च॒ पथ्या॑य च॒{\small 7}\\
नम॑: का॒ट्या॑य च नी॒प्या॑य च॒{\small 8}\\
नम॒: सूद्या॑य च सर॒स्या॑य च॒{\small 9}\\
नमो॑ ना॒द्याय॑ च वैश॒न्ताय॑ च॒{\small 10}\\
नम॒: कूप्या॑य चाव॒ट्या॑य च॒{\small 11}\\
नमो॒ वर्ष्या॑य चाव॒र्ष्याय॑ च॒{\small 12}\\
नमो॑ मे॒घ्या॑य च विद्यु॒त्या॑य च॒{\small 13}\\
नम॑ ई॒ध्रिया॑य चात॒प्या॑य च॒{\small 14}\\
नमो॒ वात्या॑य च॒ रेष्मि॑याय च॒{\small 15}\\
नमो॑ वास्त॒व्या॑य च वास्तु॒ पाय॑ च{\small 16} ॥ 7 ॥\\
\subsection{॥ अष्टम अनुवाक ॥}
नम॒: सोमा॑य च रु॒द्राय॑ च॒{\small 1}\\
नम॑स्ता॒म्राय॑ चारु॒णाय॑ च॒{\small 2}\\
नम॑: श॒ङ्गाय॑ च पशु॒पत॑ये च॒{\small 3}\\
नम॑ उ॒ग्राय॑ च भी॒माय॑ च॒{\small 4}\\
नमो॑ अग्रेव॒धाय॑ च दूरेव॒धाय॑ च॒{\small 5}\\
नमो॑ ह॒न्त्रे च॒ हनी॑यसे च॒{\small 6}\\
नमो॑ वृ॒क्षेभ्यो॒ हरि॑केशेभ्यो॒{\small 7}\\
नम॑स्ता॒राय॒ नम॑श्श॒म्भवे॑ च मयो॒भवे॑ च॒{\small 8}\\
नम॑: शङ्क॒राय॑ च मयस्क॒राय॑ च॒{\small 9}\\
नम॑: शि॒वाय॑ च शि॒वत॑राय च॒{\small 10}\\
नम॒स्तीर्थ्या॑य च॒ कूल्या॑य च॒{\small 11}\\
नम॑: पा॒र्या॑य चावा॒र्या॑य च॒{\small 12}\\
नम॑: प्र॒तर॑णाय चो॒त्तर॑णाय च॒{\small 13}\\
नम॑ आता॒र्या॑य चाला॒द्या॑य च॒{\small 14}\\
नम॒: शष्प्या॑य च॒ फेन्या॑य च॒{\small 15}\\
नम॑: सिक॒त्या॑य च प्रवा॒ह्या॑य च{\small 16} ॥ 8 ॥\\
\subsection{॥ नवम अनुवाक ॥}
नम॑ इरि॒ण्या॑य च प्रप॒थ्या॑य च॒{\small 1}\\
नम॑: किग्ंशि॒लाय॑ च॒ क्षय॑णाय च॒{\small 2}\\
नम॑: कप॒र्दिने॑ च पुल॒स्तये॑ च॒{\small 3}\\
नमो॒ गोष्ठ्या॑य च॒ गृह्या॑य च॒{\small 4}\\
नम॒स्तल्प्या॑य च॒ गेह्या॑य च॒{\small 5}\\
नम॑: का॒ट्या॑य च गह्वरे॒ष्ठाय॑ च॒{\small 6}\\
नमो᳚ ह्रद॒य्या॑य च निवे॒ष्प्या॑य च॒{\small 7}\\
नम॑: पाग्ं स॒व्या॑य च रज॒स्या॑य च॒{\small 8}\\
नम॒: शुष्क्या॑य च हरि॒त्या॑य च॒{\small 9}\\
नमो॒ लोप्या॑य चोल॒प्या॑य च॒{\small 10}\\
नम॑ ऊ॒र्व्या॑य च सू॒र्म्या॑य च॒{\small 11}\\
नम॑: प॒र्ण्या॑य च पर्णश॒द्या॑य च॒{\small 12}\\
नमो॑ऽपगु॒रमा॑णाय चाभिघ्न॒ते च॒{\small 13}\\
नम॑ आख्खिद॒ते च॑ प्रख्खिद॒ते च॒{\small 14}\\
नमो॑ वः किरि॒केभ्यो॑ दे॒वाना॒ग्ं॒ हृद॑येभ्यो॒{\small 15}\\
नमो॑ विक्षीण॒केभ्यो॒ नमो॑ विचिन्व॒त्केभ्यो॒{\small 16}\\
नम॑ आनिर्_ह॒तेभ्यो॒ नम॑ आमीव॒त्केभ्य॑:{\small 17} ॥ 9 ॥\\
\subsection{॥ दशम अनुवाक ॥}
द्रापे॒ अन्ध॑सस्पते॒ दरि॑द्र॒न्नील॑लोहित ।\\
ए॒षां पुरु॑षाणामे॒षां प॑शू॒नां मा भेर्माऽरो॒\\
मो ए॑षां॒ किञ्च॒नाम॑मत् ।\\
\\
या ते॑ रुद्र शि॒वा त॒नूः शि॒वा वि॒श्वाह॑भेषजी ।\\
शि॒वा रु॒द्रस्य॑ भेष॒जी तया॑ नो मृड जी॒वसे᳚ ।\\
\\
इ॒माग्ं रु॒द्राय॑ त॒वसे॑ कप॒र्दिने᳚\\
क्ष॒यद्वी॑राय॒ प्रभ॑रामहे म॒तिम् ।\\
यथा॑ न॒: शमस॑द्द्वि॒पदे॒ चतु॑ष्पदे॒\\
विश्वं॑ पु॒ष्टं ग्रामे॑ अ॒स्मिन्नना॑तुरम् ।\\
\\
मृ॒डा नो॑ रुद्रो॒त नो॒ मय॑स्कृधि\\
क्ष॒यद्वी॑राय॒ नम॑सा विधेम ते ।\\
यच्छं च॒ योश्च॒ मनु॑राय॒जे\\
पि॒ता तद॑श्याम॒ तव॑ रुद्र॒ प्रणी॑तौ ।\\
\\
मा नो॑ म॒हान्त॑मु॒त मा नो॑ अर्भ॒कं\\
मा न॒ उक्ष॑न्तमु॒त मा न॑ उक्षि॒तम् ।\\
मा नो॑ऽवधीः पि॒तरं॒ मोत मा॒तरं॑\\
प्रि॒या मा न॑स्त॒नुवो॑ रुद्र रीरिषः ।\\
\\
मा न॑स्तो॒के तन॑ये॒ मा न॒ आयु॑षि॒\\
मा नो॒ गोषु॒ मा नो॒ अश्वे॑षु रीरिषः ।\\
वी॒रान्मा नो॑ रुद्र भामि॒तोऽव॑धीर्ह॒विष्म॑न्तो॒\\
नम॑सा विधेम ते ।\\
\\
आ॒रात्ते॑ गो॒घ्न उ॒त पू॑रुष॒घ्ने क्ष॒यद्वी॑राय\\
सु॒म्नम॒स्मे ते॑ अस्तु ।\\
रक्षा॑ च नो॒ अधि॑ च देव ब्रू॒ह्यधा॑ च न॒:\\
शर्म॑ यच्छ द्वि॒बर्हा᳚: ।\\
\\
स्तु॒हि श्रु॒तं ग॑र्त॒सदं॒ युवा॑नं मृ॒गन्न\\
भी॒ममु॑पह॒त्नुमु॒ग्रम् ।\\
मृ॒डा ज॑रि॒त्रे रु॑द्र॒ स्तवा॑नो अ॒न्यन्ते॑\\
अ॒स्मन्निव॑पन्तु॒ सेना᳚: ।\\
\\
परि॑णो रु॒द्रस्य॑ हे॒तिर्वृ॑णक्तु॒ परि॑ त्वे॒षस्य॑\\
दुर्म॒ति र॑घा॒योः ।\\
अव॑ स्थि॒रा म॒घव॑द्भ्यस्तनुष्व॒ मीढ्व॑स्तो॒काय॒\\
तन॑याय मृडय ।\\
\\
मीढु॑ष्टम॒ शिव॑तम शि॒वो न॑: सु॒मना॑ भव ।\\
प॒र॒मे वृ॒क्ष आयु॑धन्नि॒धाय॒ कृत्तिं॒ वसा॑न॒\\
आच॑र॒ पिना॑कं॒ बिभ्र॒दाग॑हि ।\\
\\
विकि॑रिद॒ विलो॑हित॒ नम॑स्ते अस्तु भगवः ।\\
यास्ते॑ स॒हस्रग्ं॑ हे॒तयो॒न्यम॒स्मन्निव॑पन्तु॒ ताः ।\\
\\
स॒हस्रा॑णि सहस्र॒धा बा॑हु॒वोस्तव॑ हे॒तय॑: ।\\
तासा॒मीशा॑नो भगवः परा॒चीना॒ मुखा॑ कृधि ॥ 10 ॥\\
\subsection{॥ एकादश अनुवाक ॥}
स॒हस्रा॑णि सहस्र॒शो ये रु॒द्रा अधि॒ भूम्या᳚म् ।\\
तेषाग्ं॑ सहस्रयोज॒नेऽव॒धन्वा॑नि तन्मसि ।\\
\\
अ॒स्मिन्म॑ह॒त्य॑र्ण॒वे᳚ऽन्तरि॑क्षे भ॒वा अधि॑ ।\\
नील॑ग्रीवाः शिति॒कण्ठा᳚: श॒र्वा अ॒धः क्ष॑माच॒राः ।\\
नील॑ग्रीवाः शिति॒कण्ठा॒ दिवग्ं॑ रु॒द्रा उप॑श्रिताः ।\\
ये वृ॒क्षेषु॑ स॒स्पिञ्ज॑रा॒ नील॑ग्रीवा॒ विलो॑हिताः ।\\
ये भू॒ताना॒मधि॑पतयो विशि॒खास॑: कप॒र्दिन॑: ।\\
ये अन्ने॑षु वि॒विध्य॑न्ति॒ पात्रे॑षु॒ पिब॑तो॒ जनान्॑ ।\\
ये प॒थां प॑थि॒रक्ष॑य ऐलबृ॒दा य॒व्युध॑: ।\\
ये ती॒र्थानि॑ प्र॒चर॑न्ति सृ॒काव॑न्तो निष॒ङ्गिण॑: ।\\
य ए॒ताव॑न्तश्च॒ भूयाग्ं॑सश्च॒ दिशो॑ रु॒द्रा वि॑तस्थि॒रे ।\\
तेषाग्ं॑ सहस्रयोज॒नेऽव॒धन्वा॑नि तन्मसि ।\\
\\
नमो॑ रु॒द्रेभ्यो॒ ये पृ॑थि॒व्यां ये᳚ऽन्तरि॑क्षे॒ ये दि॒वि\\
येषा॒मन्नं॒ वातो॑ व॒र्॒षमिष॑व॒स्तेभ्यो॒ दश॒ प्राची॒र्दश॑\\
दक्षि॒णा दश॑ \\
प्र॒तीची॒र्दशोदी॑ची॒र्दशो॒र्ध्वास्तेभ्यो॒ नम॒स्ते नो॑\\
मृडयन्तु॒ ते यं द्वि॒ष्मो यश्च॑ नो॒ द्वेष्टि॒ तं\\
वो॒ जम्भे॑ दधामि ॥ 11 ॥\\
\\
त्र्य॑म्बकं यजामहे सुग॒न्धिं पु॑ष्टि॒वर्ध॑नम् ।\\
उ॒र्वा॒रु॒कमि॑व॒ बन्ध॑नान्मृ॒त्योर्मु॑क्षीय॒ माऽमृता᳚त् ।\\
\\
यो रु॒द्रो अ॒ग्नौ यो अ॒प्सु य ओष॑धीषु॒ यो रु॒द्रो\\
विश्वा॒ भुव॑ना वि॒वेश॒ तस्मै॑ रु॒द्राय॒ नमो॑ अस्तु ।\\
\\
तमु॑ ष्टु॒हि॒ यः स्वि॒षुः सु॒धन्वा॒\\
यो विश्व॑स्य॒ क्षय॑ति भेष॒जस्य॑ ।\\
यक्ष्वा᳚म॒हे सौ᳚मन॒साय॑ रु॒द्रं\\
नमो᳚भिर्दे॒वमसु॑रं दुवस्य ।\\
\\
अ॒यं मे॒ हस्तो॒ भग॑वान॒यं मे॒ भग॑वत्तरः ।\\
अ॒यं मे᳚ वि॒श्वभे᳚षजो॒ऽयग्ं शि॒वाभि॑मर्शनः ।\\
\\
ये ते॑ स॒हस्र॑म॒युतं॒ पाशा॒ मृत्यो॒ मर्त्या॑य॒ हन्त॑वे ।\\
तान् य॒ज्ञस्य॑ मा॒यया॒ सर्वा॒नव॑यजामहे ।\\
मृ॒त्यवे॒ स्वाहा॑ मृ॒त्यवे॒ स्वाहा᳚ ।\\
ओं नमो भगवते रुद्राय विष्णवे मृत्यु॑र्मे पा॒हि ॥\\
\\
प्राणानां ग्रन्थिरसि रुद्रो मा॑ विशा॒न्तकः ।\\
तेनान्नेना᳚प्याय॒स्व । सदाशि॒वोम् ॥\\
\\
ओं शान्ति॒: शान्ति॒: शान्ति॑: ॥\\

\section{॥ श्री रुद्रप्रश्नः – चमकप्रश्नः ॥}
\subsection{॥ प्रथम अनुवाक ॥}
ओं अग्ना॑विष्णू स॒जोष॑से॒मा व॑र्धन्तु वां॒ गिर॑: ।\\
द्यु॒म्नैर्वाजे॑भि॒राग॑तम् ।\\
\\
वाज॑श्च मे प्रस॒वश्च॑ मे॒ प्रय॑तिश्च मे॒ प्रसि॑तिश्च मे\\
धी॒तिश्च॑ मे॒ क्रतु॑श्च मे॒ स्वर॑श्च मे॒ श्लोक॑श्च मे\\
श्रा॒वश्च॑ मे॒ श्रुति॑श्च मे॒ ज्योति॑श्च मे॒ सुव॑श्च मे\\
प्रा॒णश्च॑ मेऽपा॒नश्च॑ मे व्या॒नश्च॒ मेऽसु॑श्च मे\\
चि॒त्तं च॑ म॒ आधी॑तं च मे॒ वाक्च॑ मे॒ मन॑श्च मे॒\\
चक्षु॑श्च मे॒ श्रोत्रं॑ च मे॒ दक्ष॑श्च मे॒ बलं॑ च म॒\\
ओज॑श्च मे॒ सह॑श्च म॒ आयु॑श्च मे ज॒रा च॑ म\\
आ॒त्मा च॑ मे त॒नूश्च॑ मे॒ शर्म॑ च मे॒ वर्म॑ च\\
मे॒ऽङ्गा॑नि च मे॒ऽस्थानि॑ च मे॒ परूग्ं॑षि च मे॒\\
शरी॑राणि च मे ॥ 1 ॥\\
\subsection{॥ द्वितीय अनुवाक ॥}
ज्यैष्ठ्यं॑ च म॒ आधि॑पत्यं च मे म॒न्युश्च॑ मे॒\\
भाम॑श्च॒ मेऽम॑श्च॒ मेऽम्भ॑श्च मे जे॒मा च॑ मे\\
महि॒मा च॑ मे वरि॒मा च॑ मे प्रथि॒मा च॑ मे\\
व॒र्ष्मा च॑ मे द्राघु॒या च॑ मे वृ॒द्धं च॑ मे॒\\
वृद्धि॑श्च मे स॒त्यं च॑ मे श्र॒द्धा च॑ मे॒ जग॑च्च मे॒\\
धनं॑ च मे॒ वश॑श्च मे॒ त्विषि॑श्च मे क्री॒डा च॑ मे॒\\
मोद॑श्च मे जा॒तं च॑ मे जनि॒ष्यमा॑णं च मे\\
सू॒क्तं च॑ मे सुकृ॒तं च॑ मे वि॒त्तं च॑ मे॒\\
वेद्यं॑ च मे भू॒तं च॑ मे भवि॒ष्यच्च॑ मे\\
सु॒गं च॑ मे सु॒पथं॑ च म ऋ॒द्धं च॑ म॒\\
ऋद्धि॑श्च मे क्लु॒प्तं च॑ मे॒ क्लुप्ति॑श्च मे\\
म॒तिश्च॑ मे सुम॒तिश्च॑ मे ॥ 2 ॥\\
\subsection{॥ तृतीय अनुवाक ॥}
शं च॑ मे॒ मय॑श्च मे प्रि॒यं च॑ मेऽनुका॒मश्च॑ मे॒\\
काम॑श्च मे सौमन॒सश्च॑ मे भ॒द्रं च॑ मे॒ श्रेय॑श्च मे॒\\
वस्य॑श्च मे॒ यश॑श्च मे॒ भग॑श्च मे॒ द्रवि॑णं च मे\\
य॒न्ता च॑ मे ध॒र्ता च॑ मे॒ क्षेम॑श्च मे॒ धृति॑श्च मे॒\\
विश्वं॑ च मे॒ मह॑श्च मे सं॒विच्च॑ मे॒ ज्ञात्रं॑ च मे॒\\
सूश्च॑ मे प्र॒सूश्च॑ मे॒ सीरं॑ च मे ल॒यश्च॑ म\\
ऋ॒तं च॑ मे॒ऽमृतं॑ च मेऽय॒क्ष्मं च॒\\
मेऽना॑मयच्च मे जी॒वातु॑श्च मे दीर्घायु॒त्वं च॑\\
मेऽनमि॒त्रं च॒ मेऽभ॑यं च मे सु॒गं च॑ मे॒\\
शय॑नं च मे सू॒षा च॑ मे सु॒दिनं॑ च मे ॥ 3 ॥\\
\subsection{॥ चतुर्थ अनुवाक ॥}
ऊर्क्च॑ मे सू॒नृता॑ च मे॒ पय॑श्च मे॒ रस॑श्च मे\\
घृ॒तं च॑ मे॒ मधु॑ च मे॒ सग्धि॑श्च मे॒ सपी॑तिश्च मे\\
कृ॒षिश्च॑ मे॒ वृष्टि॑श्च मे॒ जैत्रं॑ च म॒ औद्भि॑द्यं च मे\\
र॒यिश्च॑ मे॒ राय॑श्च मे पु॒ष्टं च॑ मे॒ पुष्टि॑श्च मे\\
वि॒भु च॑ मे प्र॒भु च॑ मे ब॒हु च॑ मे॒ भूय॑श्च मे\\
पू॒र्णं च॑ मे पू॒र्णत॑रं च॒ मेऽक्षि॑तिश्च मे॒\\
कूय॑वाश्च॒ मेऽन्नं॑ च॒ मेऽक्षु॑च्च मे व्री॒हय॑श्च मे॒\\
यवा᳚श्च मे॒ माषा᳚श्च मे॒ तिला᳚श्च मे मु॒द्गाश्च॑ मे\\
ख॒ल्वा᳚श्च मे गो॒धूमा᳚श्च मे म॒सुरा᳚श्च मे\\
प्रि॒यङ्ग॑वश्च॒ मेऽण॑वश्च मे श्या॒मका᳚श्च मे\\
नी॒वारा᳚श्च मे ॥ 4 ॥\\
\subsection{॥ पञ्चम अनुवाक ॥}
अश्मा॑ च मे॒ मृत्ति॑का च मे गि॒रय॑श्च मे॒ पर्व॑ताश्च मे॒\\
सिक॑ताश्च मे॒ वन॒स्पत॑यश्च मे॒ हिर॑ण्यं च॒\\
मेऽय॑श्च मे॒ सीसं॑ च मे॒ त्रपु॑श्च मे श्या॒मं च॑ मे\\
लो॒हं च॑ मे॒ऽग्निश्च॑ म॒ आप॑श्च मे वी॒रुध॑श्च म॒\\
ओष॑धयश्च मे कृष्टप॒च्यं च॑ मेऽकृष्टप॒च्यं च॑ मे\\
ग्रा॒म्याश्च॑ मे प॒शव॑ आर॒ण्याश्च॑ य॒ज्ञेन॑ कल्पन्तां\\
वि॒त्तं च मे॒ वित्ति॑श्च मे भू॒तं च॑ मे॒ भूति॑श्च मे॒\\
वसु॑ च मे वस॒तिश्च॑ मे॒ कर्म॑ च मे॒ शक्ति॑श्च॒\\
मेऽर्थ॑श्च म॒ एम॑श्च म॒ इति॑श्च मे॒ गति॑श्च मे ॥ 5 ॥\\
\subsection{॥ षष्ठम अनुवाक ॥}
अ॒ग्निश्च॑ म॒ इन्द्र॑श्च मे॒ सोम॑श्च म॒ इन्द्र॑श्च मे\\
सवि॒ता च॑ म॒ इन्द्र॑श्च मे॒ सर॑स्वती च म॒ इन्द्र॑श्च मे\\
पू॒षा च॑ म॒ इन्द्र॑श्च मे॒ बृह॒स्पति॑श्च म॒ इन्द्र॑श्च मे\\
मि॒त्रश्च॑ म॒ इन्द्र॑श्च मे॒ वरु॑णश्च म॒ इन्द्र॑श्च मे॒\\
त्वष्टा॑ च म॒ इन्द्र॑श्च मे धा॒ता च॑ म॒ इन्द्र॑श्च मे॒\\
विष्णु॑श्च म॒ इन्द्र॑श्च मे॒ऽश्विनौ॑ च म॒ इन्द्र॑श्च मे\\
म॒रुत॑श्च म॒ इन्द्र॑श्च मे॒ विश्वे॑ च मे दे॒वा इन्द्र॑श्च मे\\
पृथि॒वी च॑ म॒ इन्द्र॑श्च मे॒ऽन्तरि॑क्षं च म॒ इन्द्र॑श्च मे॒\\
द्यौश्च॑ म॒ इन्द्र॑श्च मे॒ दिश॑श्च म॒ इन्द्र॑श्च मे\\
मू॒र्धा च॑ म॒ इन्द्र॑श्च मे प्र॒जाप॑तिश्च म॒ इन्द्र॑श्च मे ॥ 6 ॥\\
\subsection{॥ सप्तम अनुवाक ॥}
अ॒ग्ं॒शुश्च॑ मे र॒श्मिश्च॒ मेऽदा᳚भ्यश्च॒ मेऽधि॑पतिश्च म\\
उपा॒ग्ं॒शुश्च॑ मेऽन्तर्या॒मश्च॑ म ऐन्द्रवाय॒वश्च॑ मे\\
मैत्रावरु॒णश्च॑ म आश्वि॒नश्च॑ मे प्रतिप्र॒स्थान॑श्च मे\\
शु॒क्रश्च॑ मे म॒न्थी च॑ म आग्रय॒णश्च॑ मे वैश्वदे॒वश्च॑ मे\\
ध्रु॒वश्च॑ मे वैश्वान॒रश्च॑ म ऋतुग्र॒हाश्च॑\\
मेऽतिग्रा॒ह्या᳚श्च म ऐन्द्रा॒ग्नश्च॑ मे वैश्वदे॒वश्च॑ मे\\
मरुत्व॒तीया᳚श्च मे माहे॒न्द्रश्च॑ म आदि॒त्यश्च॑ मे\\
सावि॒त्रश्च॑ मे सारस्व॒तश्च॑ मे पौ॒ष्णश्च॑ मे\\
पात्नीव॒तश्च॑ मे हारियोज॒नश्च॑ मे ॥ 7 ॥\\
\subsection{॥ अष्टम अनुवाक ॥}
इ॒ध्मश्च॑ मे ब॒र्हिश्च॑ मे॒ वेदि॑श्च मे॒ धिष्णि॑याश्च मे॒\\
स्रुच॑श्च मे चम॒साश्च॑ मे॒ ग्रावा॑णश्च मे॒ स्वर॑वश्च म\\
उपर॒वाश्च॑ मेऽधि॒षव॑णे च मे द्रोणकल॒शश्च॑ मे\\
वाय॒व्या॑नि च मे पूत॒भृच्च॑ म आधव॒नीय॑श्च म॒\\
आग्नी᳚ध्रं च मे हवि॒र्धानं॑ च मे गृ॒हाश्च॑ मे॒\\
सद॑श्च मे पुरो॒डाशा᳚श्च मे पच॒ताश्च॑\\
मेऽवभृ॒थश्च॑ मे स्वगाका॒रश्च॑ मे ॥ 8 ॥\\
\subsection{॥ नवम अनुवाक ॥}
अ॒ग्निश्च॑ मे घ॒र्मश्च॑ मे॒ऽर्कश्च॑ मे॒ सूर्य॑श्च मे\\
प्रा॒णश्च॑ मेऽश्वमे॒धश्च॑ मे पृथि॒वी च॒ मेऽदि॑तिश्च मे॒\\
दिति॑श्च मे॒ द्यौश्च॑ मे॒ शक्क्व॑रीर॒ङ्गुल॑यो॒ दिश॑श्च मे\\
य॒ज्ञेन॑ कल्पन्ता॒मृक्च॑ मे॒ साम॑ च मे॒ स्तोम॑श्च मे॒\\
यजु॑श्च मे दी॒क्षा च॑ मे॒ तप॑श्च म ऋ॒तुश्च॑ मे\\
व्र॒तं च॑ मेऽहोरा॒त्रयो᳚र्वृ॒ष्ट्या बृ॑हद्रथन्त॒रे च॑ मे\\
य॒ज्ञेन॑ कल्पेताम् ॥ 9 ॥\\
\subsection{॥ दशम अनुवाक ॥}
गर्भा᳚श्च मे व॒त्साश्च॑ मे॒ त्र्यवि॑श्च मे त्र्य॒वी च॑ मे\\
दित्य॒वाट् च॑ मे दित्यौ॒ही च॑ मे॒ पञ्चा॑विश्च मे\\
पञ्चा॒वी च॑ मे त्रिव॒त्सश्च॑ मे त्रिव॒त्सा च॑ मे\\
तुर्य॒वाट् च॑ मे तुर्यौ॒ही च॑ मे पष्ठ॒वाट् च॑ मे\\
पष्ठौ॒ही च॑ म उ॒क्षा च॑ मे व॒शा च॑ म ऋष॒भश्च॑ मे\\
वे॒हच्च॑ मेऽन॒ड्वाञ्च॑ मे धे॒नुश्च॑ म॒\\
आयु॑र्य॒ज्ञेन॑ कल्पतां प्रा॒णो य॒ज्ञेन॑ कल्पतामपा॒नो\\
य॒ज्ञेन॑ कल्पतां व्या॒नो य॒ज्ञेन॑ कल्पतां॒\\
चक्षु॑र्य॒ज्ञेन॑ कल्पता॒ग्॒ श्रोत्रं॑ य॒ज्ञेन॑ कल्पतां॒\\
मनो॑ य॒ज्ञेन॑ कल्पतां॒ वाग्य॒ज्ञेन॑ कल्पतामा॒त्मा\\
य॒ज्ञेन॑ कल्पतां य॒ज्ञो य॒ज्ञेन॑ कल्पताम् ॥ 10 ॥\\
\subsection{॥ एकादश अनुवाक ॥}
एका॑ च मे ति॒स्रश्च॑ मे॒ पञ्च॑ च मे स॒प्त च॑ मे॒\\
नव॑ च म॒ एका॑दश च मे॒ त्रयो॑दश च मे॒\\
पञ्च॑दश च मे स॒प्तद॑श च मे॒ नव॑दश च म॒\\
एक॑विग्ंशतिश्च मे॒ त्रयो॑विग्ंशतिश्च मे॒\\
पञ्च॑विग्ंशतिश्च मे स॒प्तविग्ं॑शतिश्च मे॒ नव॑विग्ंशतिश्च म॒\\
एक॑त्रिग्ंशच्च मे॒ त्रय॑स्त्रिग्ंशच्च मे॒ चत॑स्रश्च\\
मे॒ऽष्टौ च॑ मे॒ द्वाद॑श च मे॒ षोड॑श च मे\\
विग्ंश॒तिश्च॑ मे॒ चतु॑र्विग्ंशतिश्च मे॒ऽष्टाविग्ं॑शतिश्च मे॒\\
द्वात्रिग्ं॑शच्च मे॒ षट्त्रिग्ं॑शच्च मे चत्वरि॒ग्ं॒शच्च॑ मे॒\\
चतु॑श्चत्वारिग्ंशच्च मे॒ऽष्टाच॑त्वारिग्ंशच्च मे॒ वाज॑श्च\\
प्रस॒वश्चा॑पि॒जश्च॒ क्रतु॑श्च॒ सुव॑श्च मू॒र्धा च॒\\
व्यश्नि॑यश्चान्त्याय॒नश्चान्त्य॑श्च भौव॒नश्च॒\\
भुव॑न॒श्चाधि॑पतिश्च ॥ 11॥\\
\\
ओं इडा॑ देव॒हूर्मनु॑र्यज्ञ॒नीर्बृह॒स्पति॑रुक्थाम॒दानि॑\\
शग्ंसिष॒द्विश्वे॑दे॒वाः सू᳚क्त॒वाच॒: पृथि॑वीमात॒र्मा मा॑\\
हिग्ंसी॒र्मधु॑ मनिष्ये॒ मधु॑ जनिष्ये॒ मधु॑ वक्ष्यामि॒\\
मधु॑ वदिष्यामि॒ मधु॑मतीं दे॒वेभ्यो॒ वाच॑मुद्यासग्ं\\
शुश्रू॒षेण्यां᳚ मनु॒ष्ये᳚भ्य॒स्तं मा॑ दे॒वा अ॑वन्तु\\
शो॒भायै॑ पि॒तरोऽनु॑मदन्तु ॥\\
\\
ओं शान्ति॒: शान्ति॒: शान्ति॑: ॥\\

\section{श्रीरुद्रनाम त्रिशती}
नमो॒ हिर॑ण्यबाहवे॒ नमः॑ । से॒ना॒न्ये॑  नमः॑ ।\\
दि॒शां च॒ पत॑ये॒ नमः॑ । नमो॑ वृ॒क्षेभ्यो॒ नमः॑ ।\\
हरि॑केशेभ्यो॒  नमः॑ । प॒शू॒नां पत॑ये॒  नमः॑ ।\\
नमः॑ स॒स्पिञ्ज॑राय॒ नमः॑ । त्विषी॑मते॒ नमः॑ ।\\
प॒थी॒नां पत॑ये॒ नमः॑ । नमो॑ बभ्लु॒शाय॒ नमः॑ ।\\
वि॒व्या॒धिने॒ नमः॑ । अन्ना॑नां॒ पत॑ये॒ नमः॑ ।\\
नमो॒ हरि॑केशाय॒ नमः॑ । उ॒प॒वी॒तिने॒ नमः॑ ।\\
पु॒ष्टानां॒ पत॑ये नमः॑ । नमो॑ भ॒वस्य॑ हे॒त्यै नमः॑ ।\\
जग॑तां॒ पत॑ये॒ नमः॑ । नमो॑ रु॒द्राय॒ नमः॑ ।\\
आ॒त॒ ता॒विने॒ नमः॑ । क्षेत्रा॑णां॒ पत॑ये॒ नमः॑ ।\\
नमः॑ सू॒ताय॒ नमः॑ । \textbf{अह॑न्त्याय॒} नमः॑ ।\\
वना॑नां॒  पत॑ये॒ नमः॑ । नमो॒ रोहि॑ताय॒ नमः॑ ।\\
\textbf{स्थ॒पत॑ये} नमः॑ । वृ॒क्षाणां॒ पत॑ये॒ नमः॑ ।\\
नमो॑ म॒न्त्रिणे॒ नमः॑ । वा॒णि॒जाय॒ नमः॑ ।\\
कक्षा॑णां॒ पत॑ये नमः॑ । नमो॑ भुव॒न्तये॒ नमः॑ ।\\
वा॒रि॒व॒स्कृ॒ताय॒ नमः॑ । ओष॑धीनां॒ पत॑ये॒ नमः॑ ।\\
नम॑ उ॒च्चैर्घो॑षाय॒ नमः॑ । आ॒क्र॒न्दय॑ते॒ नमः॑ ।\\
प॒त्ती॒नाम् पत॑ये॒ नमः॑ । नमः॑ कृत्स्नवी॒ताय॒ नमः॑ ।\\
धाव॑ते॒ नमः॑ । सत्त्व॑नां॒ पत॑ये॒ नमः॑ ॥\\
\\
नमः॒ सह॑मानाय॒ नमः॑ । नि॒व्या॒धिने॒ नमः॑ ।\\
आ॒व्या॒धि नी॑नां॒ पत॑ये॒ नमः॑ । नमः॑ ककु॒भाय॒ नमः॑ ।\\
नि॒षं॒गिणे॒ नमः॑ । स्ते॒नानां॒ पत॑ये॒ नमः॑ ।\\
नमो॑ निषंङ्गिणे॒ नमः॑ । इ॒षु॒धि॒मते॒ नमः॑ ।\\
तस्क॑राणां॒ पत॑ये॒ नमः॑ । नमो॒ वञ्च॑ते॒ नमः॑ ।\\
प॒रि॒वञ्च॑ते॒ नमः॑ । स्ता॒यू॒नां पत॑ये॒ नमः॑ ।\\
नमो॑ निचे॒रवे॒ नमः॑ । प॒रि॒च॒राय॒ नमः॑ ।\\
अर॑ण्यानां॒ पत॑ये॒ नमः॑ । नमः॑ सृका॒विभ्यो॒ नमः॑ ।\\
जिघाꣳ॑सद्भ्यो॒ नमः॑ । मु॒ष्ण॒तां पत॑ये॒ नमः॑ ।\\
नमो॑ऽसि॒मद्भ्यो॒ नमः॑ । नक्त॒ञ् चर॑द्भ्यो॒ नमः॑ ।\\
प्र॒कृ॒न्तानां॒ पत॑ये॒ नमः॑ । नम॑ उष्णी॒षिणे॒ नमः॑ ।\\
गि॒रि॒च॒राय॒ नमः॑ । कु॒लुं॒चानां॒ पत॑ये॒  नमः॑ ।\\
\\
नम॒ इषु॑मद्भ्यो॒  नमः॑ । ध॒न्वा॒विभ्य॑श्च॒  नमः॑ । वो॒  नमः॑ ।\\
नम॑ आतन्वा॒नेभ्यो॒ नमः॑। प्र॒ति॒दधा॑नेभ्यश्च  नमः॑ । वो॒ नमः॑ ।\\
नम॑ आ॒यच्छ॑द्भ्यो॒  नमः॑ ।  वि॒सृ॒जद्भ्य॑श्च॒  नमः॑ ।  वो॒ नमः॑ ।\\
\textbf{नमोऽस् य॑द्भ्यो॒} नमः॑ । विध्य॑द्भ्यश्च॒  नमः॑ ।  वो॒ नमः॑ ।\\
नम॒ आसी॑नेभ्यो॒  नमः॑ । शया॑नेभ्यश्च॒  नमः॑ । वो॒ नमः॑ ।\\
नमः॑ स्व॒पद्भ्यो॒  नमः॑ । जाग्र॑द्भ्यश्च॒  नमः॑ ।  वो॒ नमः॑ ।\\
नम॒स्तिष्ठ॑द्भ्यो॒  नमः॑ । धाव॑द्भ्यश्च॒  नमः॑ ।  वो॒ नमः॑ ।\\
नम॑स्स॒भाभ्यो॒  नमः॑ । स॒भाप॑तिभ्यश्च॒ नमः॑ । वो॒ नमः॑ ।\\
नमो॒ अश्वे᳚भ्यो॒  नमः॑ । अश्व॑पतिभ्यश्च॒  नमः॑ । वो॒ नमः॑ ।\\
\\
नम॑ आव्य॒धिनी᳚भ्यो॒  नमः॑ । वि॒विध्य॑न्तीभ्यश्च॒  नमः॑ ।  वो॒ नमः॑ ।\\
नम॒ उग॑णाभ्यो॒ नमः॑ । तृ॒ꣳहतीभ्य॑श्च॒ नमः॑ । वो॒ नमः॑ ।\\
नमो॑ गृ॒त्सेभ्यो॒ नमः॑ । गृ॒त्सप॑तिभ्यश्च॒ नमः॑ । वो॒ नमः॑ ।\\
नमो॒ व्राते᳚भ्यो॒ नमः॑ । व्रात॑पतिभ्यश्च॒ नमः॑ । वो॒ नमः॑ ।\\
नमो॑ ग॒णेभ्यो॒ नमः॑ । ग॒णप॑तिभ्यश्च॒ नमः॑ । वो॒ नमः॑ ।\\
\\
नमो॒ विरू॑पेभ्यो॒ नमः॑ । वि॒श्वरुपेभ्यश्च॒ नमः॑ । वो॒ नमः॑ ।\\
नमो॑ म॒हद्भ्यो॒ नमः॑ । क्षु॒ल्ल॒केभ्य॑श्च॒ नमः॑ । वो॒ नमः॑ ।\\
नमो॑ र॒थिभ्यो॒ नमः॑ । अ॒र॒थेभ्य॑श्च॒ नमः॑ । वो॒ नमः॑ ।\\
नमो॒ रथे᳚भ्यो॒ नमः॑ । रथ॑पतिभ्यश्च॒ नमः॑ । वो॒ नमः॑ ।\\
नम॒स्सेना᳚भ्यो॒ नमः॑ । से॒ना॒निभ्य॑श्च॒ नमः॑ । वो॒ नमः॑ ।\\
\textbf{नमः॑ क्ष॒त्तृभ्यो॒} नमः॑ । सं॒ग्र॒ही॒तृभ्य॑श्च॒ नमः॑ । वो॒ नमः॑ ।\\
नम॒स्तक्ष॑भ्यो॒ नमः॑ । र॒थ॒का॒रेभ्य॑श्च॒ नमः॑ । वो॒ नमः॑ ।\\
नमः॒ कुला॑लेभ्यो॒ नमः॑ । क॒र्मारे᳚भ्यश्च॒ नमः॑ । वो॒ नमः॑ ।\\
नमः॑ पुं॒जिष्टे᳚भ्यो॒ नमः॑ । नि॒षा॒देभ्य॑श्च॒ नमः॑ । वो॒ नमः॑ ।\\
नम॑ इषु॒कृद्भ्यो॒ नमः॑ । ध॒न्व॒कृद्भ्य॑श्च॒ नमः॑ । वो॒ नमः॑ ।\\
नमो॑ मृग॒युभ्यो॒ नमः॑ । श्व॒निभ्य॑श्च॒ नमः॑ । वो॒ नमः॑ ।\\
\textbf{नमः॒ श्वभ् यो॒} नमः॑ । श्वप॑तिभ् यश्च॒ नमः॑ । वो॒ नमः॑\\
\\
नमो॑ भ॒वाय॑ च॒ नमः॑ । रु॒द्राय॑ च॒ नमः॑ ।\\
नम॑श्श॒र्वाय॑ च॒ नमः॑ । प॒शु॒पत॑ये च॒ नमः॑ ।\\
नमो॒ नील॑ग्रीवाय च॒ नमः॑ । शि॒ति॒कण्ठा॑य च॒ नमः॑ ।\\
नमः॑ कप॒र्दिने॑ च॒ नमः॑ । \textbf{व्यु॑प्त} केशाय च॒ नमः॑ ।\\
नम॑स्सहस्रा॒क्षाय॑ च॒ नमः॑ । श॒तध॑न्वने च॒ नमः॑ ।\\
नमो॑ गिरि॒शाय॑ च॒ नमः॑ । शि॒पि॒वि॒ष्टाय॑ च॒ नमः॑ ।\\
नमो॑ मी॒ढुष्ट॑माय च॒ नमः॑ । इषु॑मते च॒ नमः॑ ।\\
नमो᳚ ह्रस्वाय॑ च॒ नमः॑ । वा॒म॒नाय॑ च॒ नमः॑ ।\\
नमो॑ बृह॒ते च॒ नमः॑ । वर्षी॑यसे च॒ नमः॑ ।\\
नमो॑ वृ॒द्धाय॑ च॒ नमः॑ । सं॒वृध्व॑ने च॒ नमः॑ ।\\
नमो॒, अग्रि॑याय च॒ नमः॑ । प्र॒थ॒माय॑ च॒ नमः॑ ।\\
नम॑ आ॒शवे॑ च॒ नमः॑ । \textbf{अ॒जि॒राय॑} च॒ नमः॑ ।\\
नमः॒ शीघ्रि॑याय च॒ नमः॑ । शीभ्या॑य च॒ नमः॑ ।\\
नम॑ ऊ॒र्म्या॑य च॒ नमः॑ । अ॒व॒स्व॒न्या॑य च॒ नमः॑ ।\\
नमः॑ स्त्रो त॒स्या॑य च॒ नमः॑ । द्वीप्या॑य च॒ नमः॑ ।\\
\\
नमो᳚ ज्ये॒ष्ठाय॑ च॒ नमः॑ । क॒नि॒ष्ठाय॑ च॒ नमः॑ ।\\
नमः॑ पूर्व॒जाय॑ च॒ नमः॑ । अ॒प॒र॒जाय॑ च॒ नमः॑ ।\\
नमो॑ मध्य॒माय॑ च॒ नमः॑ । अ॒प॒ग॒ल्भाय॑ च॒ नमः॑ ।\\
नमो॑ जघ॒न्या॑य च॒ नमः॑ । बुध्नि॑याय च॒ नमः॑ ।\\
नमः॑ सो॒भ्या॑य च॒ नमः॑ । प्र॒ति॒स॒र्या॑य च॒ नमः॑ ।\\
नमो॒ याम्या॑य च॒ नमः॑ । क्षेम्या॑य च॒ नमः॑ ।\\
नम॑ उर्व॒र्या॑य च॒ नमः॑ । खल्या॑य च॒ नमः॑ ।\\
नमः॒ श्लोक्या॑य च॒ नमः॑ । अ॒व॒सा॒न्या॑य च॒ नमः॑ ।\\
नमो॒ वन्या॑य च॒ नमः॑ । कक्ष्या॑य च॒ नमः॑ ।\\
नमः॑ श्र॒वाय॑ च॒ नमः॑ । प्र॒ति॒श्र॒वाय॑ च॒ नमः॑ ।\\
नम॑ आ॒शुषे॑णाय च॒ नमः॑ । आ॒शुर॑थाय च॒ नमः॑ ।\\
नमः॒ शूरा॑य च॒ नमः॑ । अ॒व॒ भि॒न्द॒ते च॒ नमः॑ ।\\
नमो॑ व॒र्मिणे॑ च॒ नमः॑ । व॒रू॒थिने॑ च॒ नमः॑ ।\\
नमो॑ बि॒ल्मिने॑ च॒ नमः॑ । क॒व॒चिने॑ च॒ नमः॑ ।\\
नम॑श्श्रु॒ताय॑ च॒ नमः॑ । श्रु॒त॒से॒नाय॑ च॒ नमः॑ ।\\
\\
नमो॑ दुन्दु॒भ्या॑य च॒ नमः॑ । आ॒ह॒न॒न्या॑य च॒ नमः॑ ।\\
नमो॑ धृ॒ष्णवे॑ च॒ नमः॑ । प्र॒मृ॒शाय॑ च॒ नमः॑ ।\\
नमो॑ दू॒ताय॑ च॒ नमः॑ । प्रहि॑ताय च॒ नमः॑ ।\\
नमो॑ निष॒ङ्गिणे॑ च॒ नमः॑ । इ॒षु॒धि॒मते॑ च॒ नमः॑ ।\\
नम॑स्ती॒क्ष्णेष॑वे च॒ नमः॑ । आ॒यु॒धिने॑ च॒ नमः॑ ।\\
नमः॑ स्वा यु॒धाय॑ च॒ नमः॑ । सु॒धन्व॑ने च॒ नमः॑ ।\\
नमः॒ स्रुत्या॑य च॒ नमः॑ । पथ्या॑य च॒ नमः॑ ।\\
नमः॑ का॒ट्या॑य च॒ नमः॑ । नी॒प्या॑य च॒ नमः॑ ।\\
नम॒स्सूद्या॑य च॒ नमः॑ । स॒र॒स्या॑य च॒ नमः॑ ।\\
नमो॑ ना॒द्याय॑ च॒ नमः॑ । वै॒श॒न्ताय॑ च॒ नमः॑ ।\\
नमः॒ कूप्या॑य च॒ नमः॑ । अ॒व॒ट्या॑य च॒ नमः॑ ।\\
नमो॒ वर्ष्या॑य च॒ नमः॑ । अ॒व॒र्ष्याय॑ च॒ नमः॑ ।\\
नमो॑ मे॒घ्या॑य च॒ नमः॑ । वि॒द्यु॒त्या॑य च॒ नमः॑ ।\\
नम॑ ई॒ध्रिया॑य च॒ नमः॑ । आ॒त॒प्या॑य च॒ नमः॑ ।\\
नमो॒ वात्या॑य च॒ नमः॑ । रेष्मि॑याय च॒ नमः॑ ।\\
नमो॑ वास्त॒व्या॑य च॒ नमः॑ । वास्तु॒पाय॑ च॒ नमः॑ ।\\
\\
नम॒स्सोमा॑य च॒ नमः॑ । रु॒द्राय॑ च॒ नमः॑ ।\\
नम॑स्ता॒म्राय॑ च॒ नमः॑ । अ॒रु॒णाय॑ च॒ नमः॑ ।\\
नमः॑ श॒ङ्गाय॑ च॒ नमः॑ । प॒शु॒पत॑ये च॒ नमः॑ ।\\
नम॑ उ॒ग्राय॑ च॒ नमः॑ । भी॒माय॑ च॒ नमः॑ ।\\
नमो॑, अग्रेव॒धाय॑ च॒ नमः॑ । दू॒रे॒व॒धाय॑ च॒ नमः॑ ।\\
नमो॑ ह॒न्त्रे च॒ नमः॑ । हनी॑यसे च॒ नमः॑ ।\\
नमो॑ वृ॒क्षेभ्यो॒ नमः॑ । हरि॑केशेभ्यो॒ नमः॑ ।\\
नम॑स्ता॒राय॒ नमः॑ । नम॑श्शं॒भवे॑ च॒ नमः॑ ।\\
म॒यो॒भवे॑ च॒ नमः॑ । नम॑श्शंक॒राय॑ च॒ नमः॑ ।\\
म॒य॒स्क॒राय॑ च॒ नमः॑ । नमः॑ शि॒वाय॑  च॒ नमः॑ ।\\
शि॒वत॑राय च॒ नमः॑ । नम॒स्तीर्थ्या॑य च॒ नमः॑ ।\\
कूल्या॑य च॒ नमः॑ । नमः॑ पा॒र्या॑य च॒ नमः॑ ।\\
अ॒वा॒र्या॑य च॒ नमः॑ । नमः॑ प्र॒तर॑णाय च॒ नमः॑ ।\\
उ॒त्तर॑णाय च॒ नमः॑ । नम॑ आता॒र्या॑य च॒ नमः॑ ।\\
आ॒ला॒द्या॑य च॒ नमः॑ । नमः॒ शष्प्या॑य च॒ नमः॑ ।\\
फेन्या॑य च॒ नमः॑ । नमः॑ सिक॒त्या॑य च॒ नमः॑ ।\\
प्र॒वा॒ह्या॑य च॒ नमः॑ ।\\
\\
नमः॑ इरि॒ण्या॑य च॒ नमः॑ । प्र॒प॒थ्या॑य च॒ नमः॑ ।\\
नमः॑ किꣳशि॒लाय॑ च॒ नमः॑ । क्षय॑णाय च॒ नमः॑ ।\\
नमः॑ कप॒र्दिने॑ च॒ नमः॑ । पु॒ल॒स्तये॑ च॒ नमः॑ ।\\
नमो॒ गोष्ठ्या॑य च॒ नमः॑ । गृह्या॑य च॒ नमः॑ ।\\
नम॒स्तल्प्या॑य च॒ नमः॑ । गेह्या॑य च॒ नमः॑ ।\\
नमः॑ का॒ट्या॑य च॒ नमः॑ । ग॒ह्व॒रे॒ष्ठाय॑ च॒ नमः॑ ।\\
नमो᳚ ह्रद॒य्या॑य च॒ नमः॑ । नि॒वे॒ष्प्या॑य च॒ नमः॑ ।\\
नमः॑ पाꣳस॒व्या॑य च॒ नमः॑ । र॒ज॒स्या॑य च॒ नमः॑ ।\\
नमः॒ शुष्क्या॑य च॒ नमः॑ । ह॒रि॒त्या॑य च॒ नमः॑ ।\\
नमो॒ लोप्या॑य च॒ नमः॑ । उ॒ल॒प्या॑य च॒ नमः॑ ।\\
नम॑ ऊ॒र्व्या॑य च॒ नमः॑ । सू॒र्म्या॑य च॒ नमः॑ ।\\
नमः॑ प॒र्ण्या॑य च॒ नमः॑ । प॒र्ण॒श॒द्या॑य च॒ नमः॑ ।\\
नमो॑पगु॒रमा॑णाय च॒ नमः॑ । \textbf{अ॒भि॒घ्न॒ते} च॒ नमः॑ ।\\
नम॑ आक्खिद॒ते च॒ नमः॑ । प्र॒क्खि॒द॒ते च॒ नमः॑ । नमो॒ वो॒ नमः॑ ।\\
कि॒रि॒केभ्यो॒ नमः॑ । दे॒वाना॒ꣳ॒ हृद॑येभ्यो॒ नमः॑ ।\\
नमो॑ विक्षीण॒केभ्यो॒ नमः॑ । नमो॑ विचिन्व॒त्केभ्यो॒ नमः॑ ।\\
नम॑ आनिर्ह॒तेभ्यो॒ नमः॑ । नम॑ आमी व॒त्केभ्यो॒ नमः॑ ।\\
\\
॥ ॐ नमो भगवते रुद्राय ॥\\

\section{\eng{Shivopasana Mantram}}
निध॑नपतये॒ नमः । निध॑नपतान्तिकाय॒ नमः ।\\
ऊर्ध्वाय॒ नमः । ऊर्ध्वलिङ्गाय॒ नमः ।\\
हिरण्याय॒ नमः । हिरण्यलिङ्गाय॒ नमः ।\\
सुवर्णाय॒ नमः । सुवर्णलिङ्गाय॒ नमः ।\\
दिव्याय॒ नमः । दिव्यलिङ्गाय॒ नमः ।\\
भवाय॒ नमः । भवलिङ्गाय॒ नमः ।\\
शर्वाय॒ नमः । शर्वलिङ्गाय॒ नमः ।\\
शिवाय॒ नमः । शिवलिङ्गाय॒ नमः ।\\
ज्वलाय॒ नमः । ज्वललिङ्गाय॒ नमः ।\\
आत्माय॒ नमः । आत्मलिङ्गाय॒ नमः ।\\
परमाय॒ नमः । परमलिङ्गाय॒ नमः ।\\
एतत्  सोमस्य॑ सूर् यस्य॒ सर्व\\
लिङ्ग॑ꣳस्था प॒य॒ति॒ पाणि मन्त्रं॑ पवि॒त्रम् ॥ १॥\\
\\
स॒द्योजा॒तं प्र॑पद्या॒मि॒\\
स॒द्योजा॒ताय॒ वै नमो॒ नमः॑ ।\\
भ॒वे भ॑वे॒ नाति॑भवे भवस्व॒ माम् ।\\
भ॒वोद्भ॑वाय॒ नमः ॥ १॥\\
\\
वा॒म॒दे॒वाय॒ नमो᳚ ज्ये॒ष्ठाय॒ नमः॑\\
श्रे॒ष्ठाय॒ नमो॑ रु॒द्राय॒ नमः॒\\
काला॑य नमः॒ कल॑ विकरणाय॒ नमो॒\\
बल॑ विकरणाय॒ नमो॒\\
बला॑य॒ नमो॒ बल॑प्रमथनाय॒ नमः॒\\
सर्व॑ भूत दमनाय॒ नमो॑ म॒नोन्म॑नाय॒ नमः॒ ॥ १॥\\
\\
अ॒घोरे᳚भ्योऽथ॒ घोरे᳚भ्यो॒ घोर॒घोर॑तरेभ्यः ।\\
स॒र्वे᳚तः॑ सर्व॒ शर्वे᳚भ्यो॒ नम॑स्ते अस्तु रु॒द्र रू॑पेभ्यः ॥ १॥\\
\\
तत्पुरु॑षाय वि॒द्महे॑ महादे॒वाय॑ धीमहि ।\\
तन्नो॑ रुद्रः प्रचो॒दया᳚त् ॥ १॥\\
\\
ईशानः सर्व॑ विद्या॒ना॒ मीश्वरः सर्व॑भूता॒नां॒\\
ब्रह्माधि॑पति॒र्ब्रह्म॒णोऽधि॑पति॒\\
र्ब्रह्मा॑ शि॒वो मे॑ अस्तु सदाशि॒वोम् ॥ १॥\\
\\
नमो हिरण्यबाहवे हिरण्यवर्णाय\\
हिरण्यरूपाय हिरण्यपतये ।\\
अम्बिकापतय उमापतये पशुपतये॑ नमो॒ नमः ॥ १॥\\
\\
ऋ॒तꣳ स॒त्यं प॑रं ब्र॒ह्म॒ पु॒रुषं॑ कृष्ण॒ पिङ्ग॑लम् ।\\
ऊ॒र्ध्व रे॑तं वि॑रूपा॒क्षं॒ वि॒श्वरू॑पाय॒ वै नमो॒ नमः॑ ॥ १॥\\
\\
सर्वो॒ वै रु॒द्रस्तस्मै॑ रु॒द्राय॒ नमो॑ अस्तु ।\\
पुरु॑षो॒ वै रु॒द्रः सन्म॒हो नमो॒ नमः॑ ।\\
विश्वं॑ भू॒तं भुव॑नं चि॒त्रं ब॑हु॒धा जा॒तं जाय॑मानं च॒ यत् ।\\
सर्वो॒ ह्ये॑ष रु॒द्रस्तस्मै॑ रु॒द्राय॒ नमो॑ अस्तु ॥ १॥\\
\\
कद्रु॒द्राय॒ प्रचे॑तसे मी॒ढुष्ट॑माय॒ तव्य॑से ।\\
वो॒चेम॒ शन्त॑मꣳ हृ॒दे ।\\
सर्वो॒ह्ये॑ष रु॒द्रस्तस्मै॑ रु॒द्राय॒ नमो॑ अस्तु ॥ १॥\\

\section{\eng{Kramam}}
\subsection{\eng{Ganapthi Dhyanam}}
{\centering
\begin{longtable}{|c|c|}
\hline
ओं ग॒णानां᳚ त्वा         & त्वा॒ ग॒णप॑तिं\\
\hline
ग॒णप॑तिꣳ हवामहे        & ग॒णप॑ति॒ मिति॑ ग॒ण - प॒तिं॒ >\\
\hline
ह॒वा॒म॒हे॒ क॒विं           & क॒विं क॑वी॒नां\\
\hline
क॒वी॒ना मु॑प॒मश्र॑ वस्तमं      & उ॒प॒मश्र॑ वस्त म॒मित् यु॑प॒मश्र॑वः - त॒मं॒ >\\
\hline
ज्ये॒ष्ठ॒राजं॒ ब्रह्म॑णां      & ज्ये॒ष्ठ॒राज॒मिति॑ ज्येष्ठ - राजं᳚ >\\
\hline
ब्रह्म॑णां ब्रह्मणः       & ब्र॒ह्म॒ण॒स्प॒ते॒ >\\
\hline
प॒त॒ आ                & आ नः॑\\
\hline
न॒श्शृ॒ण्वन्न्             & शृ॒ण्वन्नू॒तिभिः॑\\
\hline
ऊ॒ति भि॑स्सीद           & ऊ॒ति भि॒रित्यू॒ति - भिः॒\\
\hline
सी॒द॒ साद॑नं            & साद॑न॒ मिति॒ साद॑नं\\
\hline
\end{longtable}
}
\subsection{\eng{Anuvaka 1}}
ओं नमो भगवते रुद्राय
{\centering
\begin{longtable}{|c|c|}
\hline
ओं ॥  नम॑स्ते                & ते॒ रु॒द्र॒\\
\hline
रु॒द्र॒ म॒न्यवे᳚ >               & म॒न्यव॑ उ॒तो\\
\hline
उ॒तो ते᳚ >                  & उ॒तो इत्यु॒तो\\
\hline
त॒ इष॑वे                    & इष॑वे॒ नमः॑\\
\hline
नम॒ इति॒ नमः॑               & नम॑स्ते\\
\hline
ते॒ अ॒स्तु॒                    & अ॒स्तु॒ धन्व॑ने\\
\hline
धन्व॑ने बा॒हुभ्यां᳚ >            & बा॒हुभ्या॑मु॒त\\
\hline
बा॒हुभ्या॒मिति॑ बा॒हु - भ्यां॒ >   & उ॒त ते᳚ >\\
\hline
ते॒ नमः॑                    & नम॒ इति॒ नमः॑\\
\hline
या ते᳚ >                   & त॒ इषुः॑\\
\hline
इषु॑श्शि॒वत॑मा                & शि॒वत॑मा शि॒वं\\
\hline
शि॒वत॒मेति॑ शि॒व - त॒मा॒ >      & शि॒वं ब॒भूव॑\\
\hline
ब॒भूव॑ ते                    & ते॒ धनुः॑\\
\hline
धनु॒रिति॒ धनुः॑               & शि॒वा श॑र॒व्या᳚ >\\
\hline
श॒र॒व्या॑ या                 & या तव॑\\
\hline
तव॒ तया᳚ >                 & तया॑ नः\\
\hline
नो॒ रु॒द्र॒                   & रु॒द्र॒ मृ॒ड॒य॒\\
\hline
मृ॒ड॒येति॑ मृडय                & या ते᳚ >\\
\hline
ते॒ रु॒द्र॒                    & रु॒द्र॒ शि॒वा\\
\hline
शि॒वा त॒नूः                 & त॒नूरघो॑रा\\
\hline
अघो॒राऽपा॑पकाशिनी          & अपा॑पकाशि॒नीत्यपा॑प - का॒शि॒नी॒>\\
\hline
तया॑ नः                   & न॒स्त॒नुवा᳚ >\\
\hline
त॒नुवा॒ शन्त॑मया              & शन्त॑मया॒ गिरि॑शन्त\\
\hline
शन्त॑म॒येति॒ शं - त॒म॒या॒ >       & गिरि॑शन्ता॒भि\\
\hline
गिरि॑श॒न्तेति॒ गिरि॑-श॒न्त॒       & अ॒भिचा॑कशीहि\\
\hline
चा॒क॒शी॒हीति॑ चाकशीहि        & यामिषुं᳚ >\\
\hline
इषुं॑ गिरिशन्त               & गि॒रि॒श॒न्त॒ हस्ते᳚ >\\
\hline
गि॒रि॒श॒न्तेति॑ गिरि - श॒न्त॒     & हस्ते॒ बिभ॑र्.षि\\
\hline
बिभ॒र्.ष्यस्त॑वे               & अस्त॑व॒ इत्यस्त॑वे\\
\hline
शि॒वां गि॑रित्र              & गि॒रि॒त्र॒ तां\\
\hline
गि॒रि॒त्रेति॑ गिरि - त्र॒       & तां कु॑रु\\
\hline
कु॒रु॒ मा                    & मा हिꣳ॑सीः\\
\hline
हि॒ꣳ॒सीः॒ पुरु॑षं               & पुरु॑षं॒ जग॑त्\\
\hline
जग॒दिति॒ जग॑त्               & शि॒वेन॒ वच॑सा\\
\hline
वच॑सा त्वा                 & त्वा॒ गिरि॑श\\
\hline
गिरि॒शाच्छ॑                 & अच्छा॑वदामसि\\
\hline
व॒दा॒म॒सीति॑ वदामसि          & यथा॑ नः\\
\hline
नः॒ सर्वं᳚ >                 & सर्व॒मित्\\
\hline
इज्जग॑त्                    & जग॑दय॒क्ष्मं\\
\hline
अ॒य॒क्ष्मꣳ सु॒मनाः᳚ >           & सु॒मना॒ अस॑त्\\
\hline
सु॒मना॒ इति॑ सु - मनाः᳚ >      & अस॒दित्यस॑त्\\
\hline
अद्ध्य॑वोचत्                 & अ॒वो॒च॒द॒धि॒व॒क्ता\\
\hline
अ॒धि॒व॒क्ता प्र॑थ॒मः            & अ॒धि॒व॒क्तेत्य॑धि - व॒क्ता\\
\hline
प्र॒थ॒मो दैव्यः॑               & दैव्यो॑ भि॒षक्\\
\hline
भि॒षगिति॑ भि॒षक्             & अ॒हीꣲ॑श्च\\
\hline
च॒ सर्वान्॑                  & सर्वा᳚न् जं॒भयन्न्॑\\
\hline
जं॒भय॒न्थ् सर्वाः᳚>             & सर्वा᳚श्च\\
\hline
च॒ या॒तु॒धा॒न्यः॑               & या॒तु॒धा॒न्य॑ इति॑ यातु - धा॒न्यः॑\\
\hline
अ॒सौ यः                   & यस्ता॒म्रः\\
\hline
ता॒म्रो अ॑रु॒णः               & अ॒रु॒ण उ॒त\\
\hline
उ॒त ब॒भ्रुः                  & ब॒भ्रुः सु॑म॒ङ्गलः॑\\
\hline
सु॒म॒ङ्गल॒ इति॑ सु-म॒ङ्गलः॑        & ये च॑\\
\hline
चे॒मां                      & इ॒माꣳ रु॒द्राः\\
\hline
रु॒द्रा अ॒भितः॑               & अ॒भितो॑ दि॒क्षु\\
\hline
दि॒क्षु श्रि॒ताः              & श्रि॒ताः स॑हस्र॒शः\\
\hline
स॒ह॒स्र॒शोऽव॑                 & स॒ह॒स्र॒श इति॑ सहस्र - शः\\
\hline
अवै॑षां                     & ए॒षा॒ꣳ॒ हेडः॑\\
\hline
हेड॑ ईमहे                   & ई॒म॒ह॒ इती॑महे\\
\hline
अ॒सौ यः                   & यो॑ऽव॒सर्प॑ति\\
\hline
अ॒व॒सर्प॑ति॒ नील॑ग्रीवः         & अ॒व॒सर्प॒तीत्य॑व - सर्प॑ति\\
\hline
नील॑ग्रीवो॒ विलो॑हितः        & नील॑ग्रीव॒ इति॒ नील॑ - ग्री॒वः॒\\
\hline
विलो॑हित॒ इति॒ वि - लो॒हि॒तः॒  & उ॒तैनं᳚ >\\
\hline
ए॒नं॒ गो॒पाः                 & गो॒पा अ॑दृशन्न्\\
\hline
गो॒पा इति॑ गो-पाः          & अ॒दृ॒श॒न्नदृ॑शन्न्\\
\hline
अदृ॑शन्नुदहा॒र्यः॑              & उ॒द॒हा॒र्य॑ इत्यु॑द-हा॒र्यः॑\\
\hline
उ॒तैनं᳚ >                    & ए॒नं॒ विश्वा᳚ >\\
\hline
विश्वा॑ भू॒तानि॑              & भू॒तानि॒ सः\\
\hline
स दृ॒ष्टः                   & दृ॒ष्टो मृ॑डयाति\\
\hline
मृ॒ड॒या॒ति॒ नः॒                & न॒ इति॑ नः\\
\hline
नमो॑ अस्तु                  & अ॒स्तु॒ नील॑ग्रीवाय\\
\hline
नील॑ग्रीवाय सहस्रा॒क्षाय॑      & नील॑ग्रीवा॒येति॒ नील॑ - ग्री॒वा॒य॒\\
\hline
स॒ह॒स्रा॒क्षाय॑ मी॒ढुषे᳚ >         & स॒ह॒स्रा॒क्षायेति॑ सहस्र - अ॒क्षाय॑\\
\hline
मी॒ढुष॒ इति॑ मी॒ढुषे᳚ >          & अथो॒ ये\\
\hline
अथो॒ इत्यथो᳚ >              & ये अ॑स्य\\
\hline
अ॒स्य॒ सत्वा॑नः               & सत्वा॑नो॒ऽहं\\
\hline
अ॒हन्तेभ्यः॑                  & तेभ्यो॑ऽकरं\\
\hline
अ॒क॒र॒न्नमः॑                  & नम॒ इति॒ नमः॑\\
\hline
प्रमु॑ञ्च                    & मु॒ञ्च॒ धन्व॑नः\\
\hline
धन्व॑न॒स्त्वं                  & त्वमु॒भयोः᳚ >\\
\hline
उ॒भयो॒रार्त्नि॑योः            & आर्त्नि॑यो॒र्ज्यां\\
\hline
ज्यामिति॒ज्यां               & याश्च॑\\
\hline
च॒ ते॒ >                    & ते॒ हस्ते᳚ >\\
\hline
हस्त॒ इष॑वः                 & इष॑वः॒ परा᳚ >\\
\hline
परा॒ ताः                  & ता भ॑गवः\\
\hline
भ॒ग॒वो॒ व॒प॒                  & भ॒ग॒व॒ इति॑ भग - वः॒\\
\hline
व॒पेति॑ वप                  & अ॒व॒तत्य॒ धनुः॑\\
\hline
अ॒व॒तत्येत्य॑व - तत्य॑           & धनु॒स्त्वं\\
\hline
त्वꣳ सह॑स्राक्ष              & सह॑स्राक्ष॒ शते॑षुधे\\
\hline
सह॑स्रा॒क्षेति॒ सह॑स्र - अ॒क्ष॒     & शते॑षुध॒ इति॒ शत॑ - इ॒षु॒धे॒ >\\
\hline
नि॒शीर्य॑ श॒ल्यानां᳚ >          & नि॒शीर्येति॑ नि - शीर्य॑\\
\hline
श॒ल्यानां॒ मुखा᳚ >             & मुखा॑ शि॒वः\\
\hline
शि॒वो नः॑                  & नः॒ सु॒मनाः᳚ >\\
\hline
सु॒मना॑ भव                  & सु॒मना॒ इति॑ सु - मनाः᳚ >\\
\hline
भ॒वेति॑ भव                  & विज्यं॒ धनुः॑\\
\hline
विज्य॒मिति॒ वि - ज्यं॒ >       & धनुः॑ कप॒र्दिनः॑\\
\hline
क॒प॒र्दिनो॒ विश॑ल्यः           & विश॑ल्यो॒ बाण॑वान्\\
\hline
विश॑ल्य॒ इति॒ वि - श॒ल्यः॒      & बाण॑वाꣳ उ॒त\\
\hline
बाण॑वा॒निति॒ बाण॑ - वा॒न्॒      & उ॒तेत्यु॒त\\
\hline
अने॑शन्नस्य                  & अ॒स्येष॑वः\\
\hline
इष॑वः आ॒भुः                 & आ॒भुर॑स्य\\
\hline
अ॒स्य॒ नि॒ष॒ङ्गथिः॑             & नि॒ष॒ङ्गथि॒रिति॑ नि॒ष॒ङ्गथिः॑\\
\hline
या ते᳚ >                   & ते॒ हे॒तिः\\
\hline
हे॒तिर्मी॑ढुष्टम               & मी॒ढु॒ष्ट॒म॒ हस्ते᳚ >\\
\hline
मी॒ढु॒ष्ट॒मेति॑ मीढुः - त॒म॒       & हस्ते॑ ब॒भूव॑\\
\hline
ब॒भूव॑ ते                    & ते॒ धनुः॑\\
\hline
धनु॒रिति॒ धनुः॑               & तया॒ऽस्मान्\\
\hline
अ॒स्मान्. वि॒श्वतः॑            & वि॒श्वत॒स्त्वं\\
\hline
त्वम॑य॒क्ष्मया᳚ >              & अ॒य॒क्ष्मया॒ परि॑\\
\hline
परि॑ब्भुज                   & भु॒जेति॑ भुज\\
\hline
नम॑स्ते                     & ते॒ अ॒स्तु॒\\
\hline
अ॒स्त्वायु॑धाय                & आयु॑धा॒याना॑तताय\\
\hline
अना॑तताय धृ॒ष्णवे᳚ >           & अना॑तता॒येत्यना᳚ - त॒ता॒य॒\\
\hline
धृ॒ष्णव॒ इति॑ धृ॒ष्णवे᳚ >          & उ॒भाभ्या॑मु॒त\\
\hline
उ॒त ते᳚ >                   & ते॒ नमः॑\\
\hline
नमो॑ बा॒हुभ्यां᳚ >             & बा॒हुभ्या॒न्तव॑\\
\hline
बा॒हुभ्या॒मिति॑ बा॒हु - भ्यां॒ >   & तव॒ धन्व॑ने\\
\hline
धन्व॑न॒ इति॒ धन्व॑ने            & परि॑ ते\\
\hline
ते॒ धन्व॑नः                  & धन्व॑नो हे॒तिः\\
\hline
हे॒तिर॒स्मान्                 & अ॒स्मान् वृ॑णक्तु\\
\hline
वृ॒ण॒क्तु॒ वि॒श्वतः॑              & वि॒श्वत॒ इति॑ वि॒श्वतः॑\\
\hline
अथो॒ यः                   & अथो॒ इत्यथो᳚ >\\
\hline
य इ॑षु॒धिः                  & इ॒षु॒धिस्तव॑\\
\hline
इ॒षु॒धिरिती॑षु - धिः          & तवा॒रे\\
\hline
आ॒रे अ॒स्मत्                  & अ॒स्मन्नि\\
\hline
निधे॑हि                    & धे॒हि॒ तं\\
\hline
तमिति॒ तं                  & \\
\hline
\end{longtable}
}
\subsection{\eng{Anuvaka 2}}
ओं नमो भगवते रुद्राय
{\centering
\begin{longtable}{|c|c|}
\hline
नमो॒ हिर॑ण्यबाहवे            &          हिर॑ण्यबाहवे सेना॒न्ये᳚ >

हिर॑ण्यबाहव॒ इति॒ हिर॑ण्य - बा॒ह॒वे॒ >   &         से॒ना॒न्ये॑ दि॒शां

से॒ना॒न्य॑ इति॑ सेना - न्ये᳚ >     &      दि॒शाञ्च॑

च॒ पत॑ये                    &              पत॑ये॒ नमः॑

नमो॒ नमः॑                  &              नमो॑ वृ॒क्षेभ्यः॑

वृ॒क्षेभ्यो॒ हरि॑केशेभ्यः          &       हरि॑केशेभ्यः पशू॒नां

हरि॑केशेभ्य॒ इति॒ हरि॑ - के॒शे॒भ्यः॒  & प॒शू॒नां पत॑ये

पत॑ये॒ नमः॑                  &            नमो॒ नमः॑

नमः॑ स॒स्पिञ्ज॑राय            &            स॒स्पिञ्ज॑राय॒ त्विषी॑मते

त्विषी॑मते पथी॒नां            &          त्विषी॑मत॒ इति॒ त्विषि॑ - म॒ते॒ >

प॒थी॒नां पत॑ये                &          पत॑ये॒ नमः॑

नमो॒ नमः॑                  &              नमो॑ बभ्लु॒शाय॑

ब॒भ्लु॒शाय॑ विव्या॒धिने᳚ >        &      वि॒व्या॒धिनेऽन्ना॑नां

वि॒व्या॒धिन॒ इति॑ वि - व्या॒धिने᳚ > &  अन्ना॑नां॒ पत॑ये

पत॑ये॒ नमः॑                  &           नमो॒ नमः॑

नमो॒ हरि॑केशाय              &             हरि॑केशायोपवी॒तिने᳚ >

हरि॑केशा॒येति॒ हरि॑ - के॒शा॒य॒     &    उ॒प॒वी॒तिने॑ पु॒ष्टानां᳚ >

उ॒प॒वी॒तिन॒ इत्यु॑प - वी॒तिने᳚ >   &     पु॒ष्टानां॒ पत॑ये

पत॑ये॒ नमः॑                  &              नमो॒ नमः॑

नमो॑ भ॒वस्य॑                 &              भ॒वस्य॑ हे॒त्यै

हे॒त्यै जग॑तां                 &             जग॑तां॒ पत॑ये

पत॑ये॒ नमः॑                  &              नमो॒ नमः॑

नमो॑ रु॒द्राय॑                &                रु॒द्राया॑तता॒विने᳚ >

आ॒त॒ता॒विने॒ क्षेत्रा॑णां          &         आ॒त॒ता॒विन॒ इत्या᳚ - त॒ता॒विने᳚ >

क्षेत्रा॑णां॒ पत॑ये              &             पत॑ये॒ नमः॑

नमो॒ नमः॑                  &                नमः॑ सू॒ताय॑

सू॒तायाह॑न्त्याय              &              अह॑न्त्याय॒ वना॑नां

वना॑नां॒ पत॑ये                &            पत॑ये॒ नमः॑

नमो॒ नमः॑                  &                नमो॒ रोहि॑ताय

रोहि॑ताय स्थ॒पत॑ये            &           स्थ॒पत॑ये वृ॒क्षाणां᳚ >

वृ॒क्षाणां॒ पत॑ये               &            पत॑ये॒ नमः॑

नमो॒ नमः॑                  &                नमो॑ म॒न्त्रिणे᳚ >

म॒न्त्रिणे॑ वाणि॒जाय॑           &             वा॒णि॒जाय॒ कक्षा॑णां

कक्षा॑णां॒ पत॑ये               &              पत॑ये॒ नमः॑

नमो॒ नमः॑                  &                 नमो॑ भुव॒न्तये᳚ >

भु॒व॒न्तये॑ वारिवस्कृ॒ताय॑         &       वा॒रि॒व॒स्कृ॒तायौष॑धीनां

वा॒रि॒व॒स्कृ॒तायेति॑ वारिवः - कृ॒ताय॑ &  ओष॑धीनां॒ पत॑ये

पत॑ये॒ नमः॑                  &               नमो॒ नमः॑

नम॑ उ॒च्चैर्घो॑षाय             &              उ॒च्चैर्घो॑षायाक्र॒न्दय॑ते

उ॒च्चैर्घो॑षा॒येत्यु॒च्चैः - घो॒षा॒य॒   &      आ॒क्र॒न्दय॑ते पत्ती॒नां

आ॒क्र॒न्दय॑त॒ इत्या᳚ - क्र॒न्दय॑ते    &          प॒त्ती॒नां पत॑ये

पत॑ये॒ नमः॑                  &               नमो॒ नमः॑

नमः॑ कृथ्स्नवी॒ताय॑            &             कृ॒थ्स्न॒वी॒ताय॒ धाव॑ते

कृ॒थ्स्न॒वी॒तायेति॑ कृथ्स्न - वी॒ताय॑ &    धावे॑ते॒ सत्व॑नां

सत्व॑नां॒ पत॑ये                &             पत॑ये॒ नमः॑

नम॒ इति॒ नमः॑

\hline
\end{longtable}
}

\section{\eng{Ganam}}
\subsection{\eng{Ganapthi Dhyanam}}
1. ग॒णाना᳚म् । त्वा॒ । ग॒णप॑तिम् ।\\
ग॒णाना᳚न् त्वा त्वा ग॒णाना᳚ङ् ग॒णाना᳚न् त्वा ग॒णप॑तिङ् ग॒णप॑तिन् त्वा\\
ग॒णाना᳚ङ् ग॒णाना᳚न् त्वा ग॒णप॑तिम् ।\\
\\
2. त्वा॒ । ग॒णप॑तिम् । ह॒वा॒म॒हे॒ \\
त्वा॒ ग॒णप॑तिङ् ग॒णप॑तिन् त्वा त्वा ग॒णप॑तिꣳ हवामहे हवामहे\\
ग॒णप॑तिन् त्वा त्वा ग॒णप॑तिꣳ हवामहे ।\\
\\
3. ग॒णप॑तिम् । ह॒वा॒म॒हे॒ । क॒विम् ।\\
ग॒णप॑तिꣳ हवामहे हवामहे ग॒णप॑तिङ् ग॒णप॑तिꣳ हवामहे क॒विङ्\\
क॒विꣳ ह॑वामहे ग॒णप॑तिङ् ग॒णप॑तिꣳ हवामहे क॒विम् ।\\
\\
4. ग॒णप॑तिम् ।\\
ग॒णप॑ति॒मिति॑ ग॒ण - प॒ति॒म् ।\\
\\
5. ह॒वा॒म॒हे॒ । क॒विम् । क॒वी॒नाम् ।\\
ह॒वा॒म॒हे॒ क॒विङ् क॒विꣳ ह॑वामहे हवामहे क॒विङ् क॑वी॒नाङ् क॑वी॒नाङ्\\
क॒विꣳ ह॑वामहे हवामहे क॒विङ् क॑वी॒नाम् ।\\
\\
6. क॒विम् । क॒वी॒नाम् । उ॒प॒मश्र॑वस्तमम् ॥\\
क॒विङ् क॑वी॒नाङ् क॑वी॒नाङ् क॒विङ् क॒विङ् क॑वी॒ना मु॑प॒मश्र॑वस्तम\\
मुप॒मश्र॑वस्तमङ् कवी॒नाङ् क॒विङ् क॒विङ् क॑वी॒ना मु॑प॒मश्र॑वस्तमम् ।\\
\\
7. क॒वी॒नाम् । उ॒प॒मश्र॑वस्तमम् ॥\\
क॒वी॒ना मु॑प॒मश्र॑वस्तम मुप॒मश्र॑वस्तमङ् कवी॒नाङ् क॑वी॒ना\\
मु॑प॒मश्र॑वस्तमम् ।\\
\\
8. उ॒प॒मश्र॑वस्तमम् ॥\\
उ॒प॒मश्र॑वस्तम॒ मित्यु॑प॒मश्र॑वः - त॒म॒म् ।\\
\\
9. ज्ये॒ष्ठ॒राज᳚म् । ब्रह्म॑णाम् । ब्र॒ह्म॒णः॒ ।\\
ज्ये॒ष्ठ॒राज॒म् ब्रह्म॑णा॒म् ब्रह्म॑णाञ् ज्येष्ठ॒राज॑ञ् ज्येष्ठ॒राज॒म् ब्रह्म॑णाम् ब्रह्मणो\\
ब्रह्मणो॒ ब्रह्म॑णाञ् ज्येष्ठ॒राज॑ञ् ज्येष्ठ॒राज॒म् ब्रह्म॑णाम् ब्रह्मणः ।\\
\\
10. ज्ये॒ष्ठ॒राज᳚म् ।\\
ज्ये॒ष्ठ॒राज॒मिति॑ ज्येष्ठ - राज᳚म् ।\\
\\
11. ब्रह्म॑णाम् । ब्र॒ह्म॒णः॒ । प॒ते॒ ।\\
ब्रह्म॑णाम् ब्रह्मणो ब्रह्मणो॒ ब्रह्म॑णा॒म् ब्रह्म॑णाम् ब्रह्मण स्पते पते ब्रह्मणो॒\\
ब्रह्म॑णा॒म् ब्रह्म॑णाम् ब्रह्मण स्पते ।\\
\\
12. ब्र॒ह्म॒णः॒ । प॒ते॒ । आ ।\\
ब्र॒ह्म॒ण॒ स्प॒ते॒ प॒ते॒ ब्र॒ह्म॒णो॒ ब्र॒ह्म॒ण॒ स्प॒त॒ आप॑ते ब्रह्मणो ब्रह्मण स्पत॒ आ ।\\
\\
13. प॒ते॒ । आ । नः॒ ।\\
प॒त॒ आ प॑ते पत॒ आ नो॑ न॒ आ प॑ते पत॒ आ नः॑ ।\\
\\
14. आ । नः॒ । शृ॒ण्वन्न् ।\\
आ नो॑न॒ आनः॑ शृ॒ण्वन् शृ॒ण्वन् न॒ आनः॑ शृ॒ण्वन्न् ।\\
\\
15. नः॒ । शृ॒ण्वन्न् । ऊ॒तिभिः॑ ।\\
नः॒ शृ॒ण्वन् शृ॒ण्वन् नो॑नः शृ॒ण्वन् नू॒तिभि॑ रू॒तिभिः॑ शृ॒ण्वन् नो॑नः\\
शृ॒ण्वन् नू॒तिभिः॑ ।\\
\\
16. शृ॒ण्वन्न् । ऊ॒तिभिः॑ । सी॒द॒ ।\\
शृ॒ण्वन् नू॒तिभि॑ रू॒तिभिः॑ शृ॒ण्वन् शृ॒ण्वन् नू॒तिभिः॑ सीद सीदो॒तिभिः॑\\
शृ॒ण्वन् शृ॒ण्वन् नू॒तिभीः॑ सीद ।\\
\\
17. ऊ॒तिभिः॑ । सी॒द॒ । साद॑नम् ॥\\
ऊ॒तिभिः॑ सीद सीदो॒तिभि॑ रू॒तिभिः॑ सीद॒ साद॑न॒ꣳ॒ साद॑नꣳ\\
सीदो॒तिभि॑ रू॒तिभिः॑ सीद साद॑नम् ।\\
\\
18. ऊ॒तिभिः॑ ।\\
ऊ॒तिभि॒रित्यू॒ति - भिः॒ ।\\
\\
19. सी॒द॒ । साद॑नम् ॥\\
सी॒द॒ साद॑न॒ꣳ॒ साद॑नꣳ सीद सीद॒ साद॑नम् ।\\
\\
20. साद॑नम् ॥\\
साद॑न॒मिति॒ साद॑नम् ।\\
\\
(ॐ श्री महागणपतये नमः)\\
\subsection{\eng{Anuvaka 1}}
\subsubsection{\eng{sloka 1}}
1. नमः॑ । ते॒ । रु॒द्र॒ ।\\
नम॑स्ते ते॒ नमो॒ नम॑स्ते रुद्र रुद्र ते॒ नमो॒ नम॑स्ते रुद्र ।\\
\\
2. ते॒ । रु॒द्र॒ । म॒न्यवे᳚ ।\\
ते॒ रु॒द्र॒ रु॒द्र॒ ते॒ ते॒ रु॒द्र॒ म॒न्यवे॑ म॒न्यवे॑ रुद्र ते ते रुद्र म॒न्यवे᳚ ।\\
\\
3. रु॒द्र॒ । म॒न्यवे᳚ । उ॒तो ।\\
रु॒द्र॒ म॒न्यवे॑ म॒न्यवे॑ रुद्र रुद्र म॒न्यव॑ उ॒तो, उ॒तो म॒न्यवे॑ रुद्र\\
रुद्र म॒न्यव॑ उ॒तो ।\\
\\
4. म॒न्यवे᳚ । उ॒तो । ते॒ ।\\
म॒न्यव॑ उ॒तो, उ॒तो म॒न्यवे॑ म॒न्यव॑ उ॒तो ते॑ त उ॒तो म॒न्यवे॑ म॒न्यव॑ उ॒तो ते᳚ ।\\
\\
5. उ॒तो । ते॒ । इष॑वे ।\\
उ॒तो ते॑ त उ॒तो, उ॒तो त॒ इष॑व॒ इष॑वे त उ॒तो, उ॒तो त॒ इष॑वे ।\\
\\
6. उ॒तो ।\\
उ॒तो, इत्यु॒तो ।\\
\\
7. ते॒ । इष॑वे । नमः॑ ॥\\
त॒ इष॑व॒ इष॑वे ते त॒ इष॑वे॒ नमो॒ नम॒ इष॑वे ते त॒ इष॑वे॒ नमः॑ ।\\
\\
8. इष॑वे । नमः॑ ॥\\
इष॑वे॒ नमो॒ नम॒ इष॑व॒ इष॑वे॒ नमः॑ ।\\
\\
9. नमः॑ ॥\\
नम॒ इति॒ नमः॑ ।\\
\\
10. नमः॑ । ते॒ । अ॒स्तु॒ ।\\
नम॑स्ते ते॒ नमो॒ नम॑स्ते, अस्त्वस्तु ते॒ नमो॒ नम॑स्ते, अस्तु ।\\
\\
11. ते॒ । अ॒स्तु॒ । धन्व॑ने ।\\
ते॒, अ॒स्त्व॒स्तु॒ ते॒ ते॒, अ॒स्तु॒ धन्व॑ने॒ धन्व॑ने, अस्तु ते ते, अस्तु॒ धन्व॑ने ।\\
\\
12. अ॒स्तु॒ । धन्व॑ने । बा॒हुभ्या᳚म् ।\\
अ॒स्तु॒ धन्व॑ने॒ धन्व॑ने, अस्त्वस्तु॒ धन्व॑ने बा॒हुभ्यां᳚ बा॒हुभ्यां॒ धन्व॑ने,\\
अस्त्वस्तु॒ धन्व॑ने बा॒हुभ्या᳚म् ।\\
\\
13. धन्व॑ने । बा॒हुभ्या᳚म् । उ॒त ।\\
धन्व॑ने बा॒हुभ्यां᳚ बा॒हुभ्यां॒ धन्व॑ने॒ धन्व॑ने बा॒हुभ्या॑ मु॒तोत बा॒हुभ्यां॒ धन्व॑ने॒\\
धन्व॑ने बा॒हुभ्या॑ मु॒त ।\\
\\
14. बा॒हुभ्या᳚म् । उ॒त । ते॒ ।\\
बा॒हुभ्या॑ मु॒तोत बा॒हुभ्यां᳚ बा॒हुभ्या॑ मु॒त ते॑ त उ॒त बा॒हुभ्यां᳚ बा॒हुभ्या॑ मु॒त ते᳚ ।\\
\\
15. बा॒हुभ्या᳚म् ।\\
बा॒हुभ्या॒मिति॑ बा॒हु - भ्या॒म् ।\\
\\
16. उ॒त । ते॒ । नमः॑ ॥\\
उ॒त ते॑ त उ॒तोत ते॒ नमो॒ नम॑स्त उ॒तोत ते॒ नमः॑ ।\\
\\
17. ते॒ । नमः॑ ॥\\
ते॒ नमो॒ नम॑स्ते ते॒ नमः॑ ।\\
\\
18. नमः॑ ॥\\
नम॒ इति॒ नमः॑ ।\\
\\
\subsubsection{\eng{sloka 2}}
19. या । ते॒ । इषुः॑ ।\\
या ते॑ ते॒ या या त॒ इषु॒ रिषु॑स्ते॒ या या त॒ इषुः॑ ।\\
\\
20. ते॒ । इषुः॑ । शि॒वत॑मा ।\\
त॒ इषु॒ रिषु॑स्ते त॒ इषुः॑ शि॒वत॑मा शि॒वत॒ मेषु॑स्ते त॒ इषुः॑ शि॒वत॑मा ।\\
\\
21. इषुः॑ । शि॒वत॑मा । शि॒वम् ।\\
इषुः॑ शि॒वत॑मा शि॒वत॒ मेषु॒ रिषुः॑ शि॒वत॑मा शि॒वꣳ शि॒वꣳ शि॒वत॒\\
मेषु॒ रिषुः॑ शि॒वत॑मा शि॒वम् ।\\
\\
22. शि॒वत॑मा । शि॒वम् । ब॒भूव॑ ।\\
शि॒वत॑मा शि॒वꣳ शि॒वꣳ शि॒वत॑मा शि॒वत॑मा शि॒वं ब॒भूव॑ ब॒भूव॑\\
शि॒वꣳ शि॒वत॑मा शि॒वत॑मा शि॒वं ब॒भूव॑ ।\\
\\
23. शि॒वत॑मा ।\\
शि॒वत॒मेति॑ शि॒व - त॒मा॒ ।\\
\\
24. शि॒वम् । ब॒भूव॑ । ते॒ ।\\
शि॒वं ब॒भूव॑ ब॒भूव॑ शि॒वꣳ शि॒वं ब॒भूव॑ ते ते ब॒भूव॑ शि॒वꣳ\\
शि॒वं ब॒भूव॑ ते ।\\
\\
25. ब॒भूव॑ । ते॒ । धनुः॑ ॥\\
ब॒भूव॑ ते ते ब॒भूव॑ ब॒भूव॑ ते॒ धनु॒र् धनु॑स्ते ब॒भूव॑ ब॒भूव॑ ते॒ धनुः॑ ।\\
\\
26. ते॒ । धनुः॑ ॥\\
ते॒ धनु॒र् धनु॑स्ते ते॒ धनुः॑ ।\\
\\
27. धनुः॑ ॥\\
धनु॒रिति॒ धनुः॑ ।\\
\\
28. शि॒वा । श॒र॒व्या᳚ । या ।\\
शि॒वा श॑र॒व्या॑ शर॒व्या॑ शि॒वा शि॒वा श॑र॒व्या॑ या या श॑र॒व्या॑ शि॒वा\\
शि॒वा श॑र॒व्या॑ या ।\\
\\
29. श॒र॒व्या᳚ । या । तव॑ ।\\
श॒र॒व्या॑ या या श॑र॒व्या॑ शर॒व्या॑ या तव॒ तव॒ या श॑र॒व्या॑ शर॒व्या॑ या तव॑ ।\\
\\
30. या । तव॑ । तया᳚ ।\\
या तव॒ तव॒ या या तव॒ तया॒ तया॒ तव॒ या या तव॒ तया᳚ ।\\
\\
31. तव॑ । तया᳚ । नः॒ ।\\
तव॒ तया॒ तया॒ तव॒ तव॒ तया॑ नो न॒ स्तया॒ तव॒ तव॒ तया॑ नः ।\\
\\
32. तया᳚ । नः॒ । रु॒द्र॒ ।\\
तया॑ नो न॒स्तया॒ तया॑ नो रुद्र रुद्र न॒स्तया॒ तया॑ नो रुद्र ।\\
\\
33. नः॒ । रु॒द्र॒ । मृ॒ड॒य॒ ॥\\
नो॒ रु॒द्र॒ रु॒द्र॒ नो॒ नो॒ रु॒द्र॒ मृ॒ड॒य॒ मृ॒ड॒य॒ रु॒द्र॒ नो॒ नो॒ रु॒द्र॒ मृ॒ड॒य॒ ।\\
\\
34. रु॒द्र॒ । मृ॒ड॒य॒ ॥\\
रु॒द्र॒ मृ॒ड॒य॒ मृ॒ड॒य॒ रु॒द्र॒ रु॒द्र॒ मृ॒ड॒य॒ ।\\
\\
35. मृ॒ड॒य॒ ॥\\
मृ॒ड॒येति॑ मृडय ।\\
\\
\subsubsection{\eng{sloka 3}}
36. या । ते॒ । रु॒द्र॒ ।\\
या ते॑ ते॒ या या ते॑ रुद्र रुद्र ते॒ या या ते॑ रुद्र ।\\
\\
37. ते॒ । रु॒द्र॒ । शि॒वा ।\\
ते॒ रु॒द्र॒ रु॒द्र॒ ते॒ ते॒ रु॒द्र॒ शि॒वा शि॒वा रु॑द्र ते ते रुद्र शि॒वा ।\\
\\
38. रु॒द्र॒ । शि॒वा । त॒नूः ।\\
रु॒द्र॒ शि॒वा शि॒वा रु॑द्र रुद्र शि॒वा त॒नू स्त॒नूः शि॒वा रु॑द्र रुद्र शि॒वा त॒नूः ।\\
\\
39. शि॒वा । त॒नूः । अघो॑रा ।\\
शि॒वा त॒नू स्त॒नूः शि॒वा शि॒वा त॒नूर घो॒राऽ घो॑रा त॒नूः शि॒वा शि॒वा\\
त॒नूर घो॑रा ।\\
\\
40. त॒नूः । अघो॑रा । अपा॑पकाशिनी ॥\\
त॒नूर घो॒राऽघो॑रा त॒नूस्त॒नूर घो॒राऽ पा॑पकाशि॒न्य\\
पा॑पकाशि॒न्य घो॑रा त॒नूस्त॒नूर घो॒राऽ पा॑पकाशिनी ।\\
\\
41. अघो॑रा । अपा॑पकाशिनी ॥\\
अघो॒राऽ पा॑पकाशि॒न्य पा॑पकाशि॒न्य घो॒राऽ घो॒राऽ पा॑पकाशिनी ।\\
\\
42. अपा॑पकाशिनी ॥\\
अपा॑प काशि॒ नीत्य पा॑प - का॒शि॒नी॒ ।\\
\\
43. तया᳚ । नः॒ । त॒नुवा᳚ ।\\
तया॑ नो न॒स्तया॒ तया॑ नस्त॒नुवा॑ त॒नुवा॑ न॒स्तया॒ तया॑ नस्त॒नुवा᳚ ।\\
\\
44. नः॒ । त॒नुवा᳚ । शन्त॑मया ।\\
न॒स्त॒नुवा॑ त॒नुवा॑ नो नस्त॒नुवा॒ शन्त॑मया॒ शन्त॑मया त॒नुवा॑\\
नो नस्त॒नुवा॒ शन्त॑मया ।\\
\\
45. त॒नुवा᳚ । शन्त॑मया । गिरि॑शन्त ।\\
त॒नुवा॒ शन्त॑मया॒ शन्त॑मया त॒नुवा॑ त॒नुवा॒ शन्त॑मया॒ गिरि॑शन्त॒ गिरि॑शन्त॒\\
शन्त॑मया त॒नुवा॑ त॒नुवा॒ शन्त॑मया॒ गिरि॑शन्त ।\\
\\
46. शन्त॑मया । गिरि॑शन्त । अ॒भि ।\\
शन्त॑मया॒ गिरि॑शन्त॒ गिरि॑शन्त॒ शन्त॑मया॒ शन्त॑मया॒ गिरि॑शन् ता॒भ्य॑भि\\
गिरि॑शन्त॒ शन्त॑मया॒ शन्त॑मया॒ गिरि॑शन् ता॒भि ।\\
\\
47. शन्त॑मया ।\\
शन्त॑म॒येति॒ शं - त॒म॒या॒ ।\\
\\
48. गिरि॑शन्त । अ॒भि । चा॒क॒शी॒हि॒ ॥\\
गिरि॑ शन् ता॒भ्य॑भि गिरि॑ शन्त॒ गिरि॑ शन्ता॒भि चा॑कशीहि चाकशी\\
ह्य॒भि गिरि॑शन्त॒ गिरि॑शन्ता॒ भिचा॑कशीहि ।\\
\\
49. गिरि॑शन्त ।\\
गिरि॑श॒न्तेति॒ गिरि॑ - श॒न्त॒ ।\\
\\
50. अ॒भि । चा॒क॒शी॒हि॒ ॥\\
अ॒भि चा॑कशीहि चाकशी ह्य॒भ्य॑भि चा॑कशीहि ।\\
\\
51. चा॒क॒शी॒हि॒ ॥\\
चा॒क॒शी॒हीति॑ चाकशीहि ।\\
\\
\subsubsection{\eng{sloka 4}}
52. याम् । इषु᳚म् । गि॒रि॒श॒न्त॒ ।\\
या मिषु॒ मिषुं॒ यांँया मिषुं॑ गिरिशन्त गिरिश॒ न्तेषुं॒ यांँया मिषुं॑\\
गिरिशन्त ।\\
\\
53. इषु᳚म् । गि॒रि॒श॒न्त॒ । हस्ते᳚ ।\\
इषुं॑ गिरिशन्त गिरिश॒ न्तेषु॒ मिषुं॑ गिरिशन्त॒ हस्ते॒ हस्ते॑ गिरिश॒ न्तेषु॒ मिषुं॑\\
गिरिशन्त॒ हस्ते᳚ ।\\
\\
54. गि॒रि॒श॒न्त॒ । हस्ते᳚ । बिभ॑र्.षि ।\\
गि॒रि॒श॒न्त॒ हस्ते॒ हस्ते॑ गिरिशन्त गिरिशन्त॒ हस्ते॒ बिभ॑र्.षि॒ बिभ॑र्.षि॒ हस्ते॑\\
गिरिशन्त गिरिशन्त॒ हस्ते॒ बिभ॑र्.षि ।\\
\\
55. गि॒रि॒श॒न्त॒ ।\\
गि॒रि॒श॒न्तेति॑ गिरि - श॒न्त॒ ।\\
\\
56. हस्ते᳚ । बिभ॑र्.षि । अस्त॑वे ॥\\
हस्ते॒ बिभ॑र्.षि॒ बिभ॑र्.षि॒ हस्ते॒ हस्ते॒ बिभ॒र्ष्यस्त॑वे॒, अस्त॑वे॒ बिभ॑र्.षि॒\\
हस्ते॒ हस्ते॒ बिभ॒र्ष्यस्त॑वे ।\\
\\
57. बिभ॑र्.षि । अस्त॑वे ॥\\
बिभ॒र्ष्यस्त॑वे॒, अस्त॑वे॒ बिभ॑र्.षि॒ बिभ॒र्ष्यस्त॑वे ।\\
\\
58. अस्त॑वे ॥\\
अस्त॑व॒ इत्यस्त॑वे ।\\
\\
59. शि॒वाम् । गि॒रि॒त्र॒ । ताम् ।\\
शि॒वां गि॑रित्र गिरित्र शि॒वाꣳ शि॒वां गि॑रित्र॒ तां तां गि॑रित्र शि॒वाꣳ\\
शि॒वां गि॑रित्र॒ ताम् ।\\
\\
60. गि॒रि॒त्र॒ । ताम् । कु॒रु॒ ।\\
गि॒रि॒त्र॒ तां तां गि॑रित्र गिरित्र॒ तां कु॑रु कुरु॒ तां गि॑रित्र गिरित्र॒ तां कु॑रु ।\\
\\
61. गि॒रि॒त्र॒ ।\\
गि॒रि॒त्रेति॑ गिरि - त्र॒ ।\\
\\
62. ताम् । कु॒रु॒ । मा ।\\
तां कु॑रु कुरु॒ तां तां कु॑रु॒ मा मा कु॑रु॒ तां तां कु॑रु॒ मा ।\\
\\
63. कु॒रु॒ । मा । हि॒ꣳ॒सीः॒ ।\\
कु॒रु॒ मा मा कु॑रु कुरु॒ मा हिꣳ॑सीर्. हिꣳसी॒र् मा कु॑रु कुरु॒ मा हिꣳ॑सीः ।\\
\\
64. मा । हि॒ꣳ॒सीः॒ । पुरु॑षम् ।\\
मा हिꣳ॑सीर्. हिꣳसी॒र् मा मा हिꣳ॑सीः॒ पुरु॑षं॒ पुरु॑षꣳ हिꣳसी॒र् मा मा\\
हिꣳ॑सीः॒ पुरु॑षम् ।\\
\\
65. हि॒ꣳ॒सीः॒ । पुरु॑षम् । जग॑त् ॥\\
हि॒ꣳ॒सीः॒ पुरु॑ष॒म् पुरु॑षꣳ हिꣳसीर्. हिꣳसीः॒ पुरु॑षं॒ जग॒ज् जग॒त् पुरु॑षꣳ\\
हिꣳसीर्. हिꣳसीः॒ पुरु॑षं॒ जग॑त् ।\\
\\
66. पुरु॑षम् । जग॑त् ॥\\
पुरु॑षं॒ जग॒ज् जग॒त् पुरु॑षं॒ पुरु॑षं॒ जग॑त् ।\\
\\
67. जग॑त् ॥\\
जग॒दिति॒ जग॑त् ।\\
\\
\subsubsection{\eng{sloka 5}}
68. शि॒वेन॑ । वच॑सा । त्वा॒ ।\\
शि॒वेन॒ वच॑सा॒ वच॑सा शि॒वेन॑ शि॒वेन॒ वच॑सा त्वा त्वा॒ वच॑सा शि॒वेन॑\\
शि॒वेन॒ वच॑सा त्वा ।\\
\\
69. वच॑सा । त्वा॒ । गिरि॑श ।\\
वच॑सा त्वा त्वा॒ वच॑सा॒ वच॑सा त्वा॒ गिरि॑श॒ गिरि॑श त्वा॒ वच॑सा॒ वच॑सा\\
त्वा॒ गिरि॑श ।\\
\\
70. त्वा॒ । गिरि॑श । अच्छ॑ ।\\
त्वा॒ गिरि॑श॒ गिरि॑श त्वा त्वा॒ गिरि॒ शाच् छाच्छ॒\\
गिरि॑श त्वा त्वा॒ गिरि॒ शाच्छ॑ ।\\
\\
71. गिरि॑श । अच्छ॑ । व॒दा॒म॒सि॒ ॥\\
गिरि॒ शाच् छाच्छ॒ गिरि॑श॒ गिरि॒ शाच्छा॑ वदामसि\\
वदा म॒स्यच्छ॒ गिरि॑श॒ गिरि॒शाच्छा॑ वदामसि ॥\\
\\
72. अच्छ॑ । व॒दा॒म॒सि॒ ॥\\
अच्छा॑ वदामसि वदा म॒स्यच् छाच्छा॑ वदामसि ।\\
\\
73. व॒दा॒म॒सि॒ ।\\
व॒दा॒म॒ सीति॑ वदामसि ।\\
\\
74. यथा᳚ । नः॒ । सर्व᳚म् ।\\
यथा॑ नो नो॒ यथा॒ यथा॑ नः॒ सर्व॒ꣳ॒ सर्व॑न्नो॒ यथा॒ यथा॑ नः॒ सर्व᳚म् ।\\
\\
75. नः॒ । सर्व᳚म् । इत् ।\\
नः॒ सर्व॒ꣳ॒ सर्व॑न्नो नः॒ सर्व॒ मिदिथ् सर्व॑न्नो नः॒ सर्व॒ मित् ।\\
\\
76. सर्व᳚म् । इत् । जग॑त् ।\\
सर्व॒ मिदिथ् सर्व॒ꣳ॒ सर्व॒ मिज् जग॒ज् जग॒ दिथ् सर्व॒ꣳ॒ सर्व॒\\
मिज् जग॑त् ।\\
\\
77. इत् । जग॑त् । अ॒य॒क्ष्मम् ।\\
इज् जग॒ज् जग॒ दिदिज् जग॑\textbf{द} य॒क्ष्मम॑ य॒क्ष्मं जग॒ दिदिज्\\
जग॑\textbf{द} य॒क्ष्मम् ।\\
\\
78. जग॑त् । अ॒य॒क्ष्मम् । सु॒मनाः᳚ ।\\
जग॑द य॒क्ष्मम॑ य॒क्ष्मं जग॒ज् जग॑द य॒क्ष्मꣳ सु॒मनाः᳚ सु॒मना॑,\\
अय॒क्ष्मं जग॒ज् जग॑द य॒क्ष्मꣳ सु॒मनाः᳚ ।\\
\\
79. अ॒य॒क्ष्मम् । सु॒मनाः᳚ । अस॑त् ॥\\
अ॒य॒क्ष्मꣳ सु॒मनाः᳚ सु॒मना॑, अय॒क्ष्मम॑ य॒क्ष्मꣳ सु॒मना॒,\\
अस॒ दस॑थ् सु॒मना॑, अय॒क्ष्मम॑ य॒क्ष्मꣳ सु॒मना॒, अस॑त् ।\\
\\
80. सु॒मनाः᳚ । अस॑त् ॥\\
सु॒मना॒, अस॒ दस॑थ् सु॒मनाः᳚ सु॒मना॒, अस॑त् ।\\
\\
81. सु॒मनाः᳚ ।\\
सु॒मना॒, इति॑ सु - मनाः᳚ ।\\
\\
82. अस॑त् ॥\\
अस॒ दित्यस॑त् ।\\
\\
\subsubsection{\eng{sloka 6}}
83. अधि॑ । अ॒वो॒च॒त् । अ॒धि॒व॒क्ता ।\\
अध्य॑वो चदवो च॒दध् यध्य॑वो चदधि व॒क्ताऽ धि॑व॒क्ताऽ वो॑ च॒दध्\\
यध्य॑ वो चदधि व॒क्ता ।\\
\\
84. अ॒वो॒च॒त् । अ॒धि॒व॒क्ता । प्र॒थ॒मः ।\\
अ॒वो॒ च॒द॒धि॒ व॒क्ताऽ धि॑व॒क्ताऽ वो॑ चदवो चदधि व॒क्ता प्र॑थ॒मः प्र॑थ॒मो,\\
अ॑धि व॒क्ताऽ वो॑ चदवो चदधि व॒क्ता प्र॑थ॒मः ।\\
\\
85. अ॒धि॒व॒क्ता । प्र॒थ॒मः । दैव्यः॑ ।\\
अ॒धि॒ व॒क्ता प्र॑थ॒मः प्र॑थ॒मो, अ॑धि व॒क्ताऽ धि॑व॒क्ता प्र॑थ॒मो दैव्यो॒ दैव्यः॑ प्रथ॒मो,\\
अ॑धि व॒क्ताऽ धि॑व॒क्ता प्र॑थ॒मो दैव्यः॑ ।\\
\\
86. अ॒धि॒व॒क्ता ।\\
अ॒धि॒ व॒क्तेत्य॑धि - व॒क्ता ।\\
\\
87. प्र॒थ॒मः । दैव्यः॑ । भि॒षक् ॥\\
प्र॒थ॒मो दैव्यो॒ दैव्यः॑ प्रथ॒मः प्र॑थ॒मो दैव्यो॑ भि॒षग् भि॒षग् दैव्यः॑ प्रथ॒मः\\
प्र॑थ॒मो दैव्यो॑ भि॒षक् ।\\
\\
88. दैव्यः॑ । भि॒षक् ॥\\
दैव्यो॑ भि॒षग् भि॒षग् दैव्यो॒ दैव्यो॑ भि॒षक् ।\\
\\
89. भि॒षक् ॥\\
भि॒षगिति॑ भि॒षक् ।\\
\\
90. अहीन्॑ । च॒ । सर्वान्॑ ।\\
अहीꣲ॑श्च॒ चाही॒ नहीꣲ॑श्च॒ सर्वा॒न् थ्सर्वा॒ꣲ॒ श्चाही॒ नहीꣲ॑श्च॒ सर्वान्॑ ।\\
\\
91. च॒ । सर्वा᳚न् । जं॒भयन्न्॑ ।\\
च॒ सर्वा॒न् थ्सर्वाꣲ॑श्च च॒ सर्वा᳚न् ज॒म्भय॑न् ज॒म्भय॒न् थ्सर्वाꣲ॑श्च च॒\\
सर्वा᳚न् ज॒म्भयन्न्॑ ।\\
\\
92. सर्वा᳚न् । जं॒भयन्न्॑ । सर्वाः᳚ ।\\
सर्वा᳚न् ज॒म्भय॑न् ज॒म्भय॒न् थ्सर्वा॒न् थ्सर्वा᳚न् ज॒म्भय॒न् थ्सर्वाः॒ सर्वा॑\\
ज॒म्भय॒न् थ्सर्वा॒न् थ्सर्वा᳚न् ज॒म्भय॒न् थ्सर्वाः᳚ ।\\
\\
93. जं॒भयन्न्॑ । सर्वाः᳚ । च॒ ।\\
ज॒म्भय॒न् थ्सर्वाः॒ सर्वा॑ ज॒म्भय॑न् ज॒म्भय॒न् थ्सर्वा᳚श्च च॒ सर्वा॑\\
ज॒म्भय॑न् ज॒म्भय॒न् थ्सर्वा᳚श्च ।\\
\\
94. सर्वाः᳚ । च॒ । या॒तु॒धा॒न्यः॑ ॥\\
सर्वा᳚श्च च॒ सर्वाः॒ सर्वा᳚श्च यातुधा॒न्यो॑ यातु धा॒न्य॑श्च॒ सर्वाः॒ सर्वा᳚श्च\\
यातुधा॒न्यः॑ ।\\
\\
95. च॒ । या॒तु॒धा॒न्यः॑ ॥\\
च॒ या॒तु॒धा॒न्यो॑ यातुधा॒न्य॑श्च च यातुधा॒न्यः॑ ।\\
\\
96. या॒तु॒धा॒न्यः॑ ॥\\
या॒तु॒धा॒न्य॑ इति॑ यातु - धा॒न्यः॑ ।\\
\\
\subsubsection{\eng{sloka 7}}
97. अ॒सौ । यः । ता॒म्रः ।\\
अ॒सौ यो यो, अ॒सा व॒सौ यस्तां॒ रस्ता॒म्रो यो, अ॒सा व॒सौ यस्ता॒म्रः ।\\
\\
98. यः । ता॒म्रः । अ॒रु॒णः ।\\
यस्तां॒ रस्ता॒म्रो यो यस्ता॒म्रो, अ॑रु॒णो, अ॑रु॒ णस्ता॒म्रो यो यस्ता॒म्रो,\\
अ॑रु॒णः ।\\
\\
99. ता॒म्रः । अ॒रु॒णः । उ॒त ।\\
ता॒म्रो, अ॑रु॒णो, अ॑रु॒ णस्तां॒ रस्ता॒म्रो, अ॑रु॒ण उ॒तो तारु॒ णस्तां॒ रस्ता॒म्रो,\\
अ॑रु॒ण उ॒त ।\\
\\
100. अ॒रु॒णः । उ॒त । ब॒भ्रुः ।\\
अ॒रु॒ण उ॒तो तारु॒णो, अ॑रु॒ण उ॒त ब॒भ्रुर् ब॒भ्रु रु॒ता रु॒णो, अ॑रु॒ण उ॒त ब॒भ्रुः ।\\
\\
101. उ॒त । ब॒भ्रुः । सु॒म॒ङ्गलः॑ ॥\\
उ॒त ब॒भ्रुर् ब॒भ्रु रु॒तोत ब॒भ्रुः सु॑म॒ङ्गलः॑ सुम॒ङ्गलो॑ ब॒भ्रु रु॒तोत ब॒भ्रुः\\
सु॑म॒ङ्गलः॑ ।\\
\\
102. ब॒भ्रुः । सु॒म॒ङ्गलः॑ ॥\\
ब॒भ्रुः सु॑म॒ङ्गलः॑ सुम॒ङ्गलो॑ ब॒भ्रुर् ब॒भ्रुः सु॑म॒ङ्गलः॑ ।\\
\\
103. सु॒म॒ङ्गलः॑ ॥\\
सु॒म॒ङ्गल॒ इति॑ सु - म॒ङ्गलः॑ ।\\
\\
104. ये । च॒ । इ॒माम् ।\\
ये च॑ च॒ ये ये चे॒मा मि॒माञ्च॒ ये ये चे॒माम् ।\\
\\
105. च॒ । इ॒माम् । रु॒द्राः ।\\
चे॒मा मि॒माञ्च॑ चे॒माꣳ रु॒द्रा रु॒द्रा, इ॒माञ्च॑ चे॒माꣳ रु॒द्राः ।\\
\\
106. इ॒माम् । रु॒द्राः । अ॒भितः॑ ।\\
इ॒माꣳ रु॒द्रा रु॒द्रा, इ॒मा मि॒माꣳ रु॒द्रा, अ॒भितो॑, अ॒भितो॑ रु॒द्रा,\\
इ॒मा मि॒माꣳ रु॒द्रा, अ॒भितः॑ ।\\
\\
107. रु॒द्राः । अ॒भितः॑ । दि॒क्षु ।\\
रु॒द्रा, अ॒भितो॑, अ॒भितो॑ रु॒द्रा रु॒द्रा, अ॒भिताे॑ दि॒क्षु दि॒क्ष्व॑ भितो॑ रु॒द्रा रु॒द्रा,\\
अ॒भितो॑ दि॒क्षु ।\\
\\
108. अ॒भितः॑ । दि॒क्षु । श्रि॒ताः ।\\
अ॒भितो॑ दि॒क्षु दि॒क्ष्व॑ भितो॑, अ॒भितो॑ दि॒क्षु श्रि॒ताः श्रि॒ता दि॒क्ष्व॑ भितो॑,\\
अ॒भितो॑ दि॒क्षु श्रि॒ताः ।\\
\\
109. दि॒क्षु । श्रि॒ताः । स॒ह॒स्र॒शः ।\\
दि॒क्षु श्रि॒ताः श्रि॒ता दि॒क्षु दि॒क्षु श्रि॒ताः स॑हस्र॒शः स॑हस्र॒शः श्रि॒ता दि॒क्षु\\
दि॒क्षु श्रि॒ताः स॑हस्र॒शः ।\\
\\
110. श्रि॒ताः । स॒ह॒स्र॒शः । अव॑ ।\\
श्रि॒ताः स॑हस्र॒शः स॑हस्र॒शः श्रि॒ताः श्रि॒ताः स॑हस्र॒शोऽ वाव॑ सहस्र॒शः\\
श्रि॒ताः श्रि॒ताः स॑हस्र॒ शोऽव॑ ।\\
\\
111. स॒ह॒स्र॒शः । अव॑ । ए॒षा॒म् ।\\
स॒ह॒स्र॒शोऽ वाव॑ सहस्र॒शः स॑हस्र॒शोऽ वै॑षा मेषा॒ मव॑ सहस्र॒शः स॑हस्र॒शोऽ वै॑षाम् ।\\
\\
112. स॒ह॒स्र॒शः ।\\
स॒ह॒स्र॒श इति॑ सहस्र - शः ।\\
\\
113. अव॑ । ए॒षा॒म् । हेडः॑ ।\\
अवै॑षा मेषा॒ मवा वै॑षा॒ꣳ॒ हेडो॒ हेड॑ एषा॒ मवा वै॑षा॒ꣳ॒ हेडः॑ ।\\
\\
114. ए॒षा॒म् । हेडः॑ । ई॒म॒हे॒ ॥\\
ए॒षा॒ꣳ॒ हेडो॒ हेड॑ एषा मेषा॒ꣳ॒ हेड॑ ईमह ईमहे॒ हेड॑ एषा मेषा॒ꣳ॒\\
हेड॑ ईमहे ।\\
\\
115. हेडः॑ । ई॒म॒हे॒ ॥\\
हेड॑ ईमह ईमहे॒ हेडो॒ हेड॑ ईमहे ।\\
\\
116. ई॒म॒हे॒ ॥\\
ई॒म॒ह॒ इती॑ महे ।\\
\\
\subsubsection{\eng{sloka 8}}
117. अ॒सौ । यः । अ॒व॒सर्प॑ति ।\\
अ॒सौयो यो, अ॒सा व॒सौ यो॑ऽव॒ सर्प॑त्यव॒ सर्प॑ति॒यो, अ॒सा व॒सौ यो॑ऽव॒ सर्प॑ति ।\\
\\
118. यः । अ॒व॒सर्प॑ति । नील॑ग्रीवः ।\\
यो॑ऽव॒ सर्प॑त्यव॒ सर्प॑ति॒ यो यो॑ऽव॒ सर्प॑ति॒ नील॑ग्रीवो॒ नील॑ग्रीवोऽव॒ सर्प॑ति॒ यो यो॑ऽव॒ सर्प॑ति॒ नील॑ग्रीवः ।\\
\\
119. अ॒व॒सर्प॑ति । नील॑ग्रीवः । विलो॑हितः ॥\\
अ॒व॒ सर्प॑ति॒ नील॑ग्रीवो॒ नील॑ग्रीवोऽव॒ सर्प॑त्यव॒ सर्प॑ति॒ नील॑ग्रीवो॒ विलो॑हितो॒\\
विलो॑हितो॒ नील॑ग्रीवोऽव॒ सर्प॑त्यव॒ सर्प॑ति॒ नील॑ग्रीवो॒ विलो॑हितः ।\\
\\
120. अ॒व॒सर्प॑ति ।\\
अ॒व॒ सर्प॒तीत्य॑व - सर्प॑ति ।\\
\\
121. नील॑ग्रीवः । विलो॑हितः ॥\\
नील॑ग्रीवो॒ विलो॑हितो॒ विलो॑हितो॒ नील॑ग्रीवो॒ नील॑ग्रीवो॒ विलो॑हितः ।\\
\\
122. नील॑ग्रीवः ।\\
नील॑ग्रीव॒ इति॒ नील॑ - ग्री॒वः॒ ।\\
\\
123. विलो॑हितः ॥\\
विलो॑हित॒ इति॒ वि - लो॒हि॒तः॒ ।\\
\\
124. उ॒त । ए॒न॒म् । गो॒पाः ।\\
उ॒तैन॑ मेन मु॒तो तैनं॑ गो॒पा गो॒पा, ए॑न मु॒तोतैनं॑ गो॒पाः ।\\
\\
125. ए॒न॒म् । गो॒पाः । अ॒दृ॒श॒न्न् ।\\
ए॒नं॒ गो॒पा गो॒पा, ए॑न मेनं गो॒पा, अ॑दृशन् नदृशन् गो॒पा,\\
ए॑न मेनं गो॒पा, अ॑दृशन्न् ।\\
\\
126. गो॒पाः । अ॒दृ॒श॒न्न् । अदृ॑शन्न् ।\\
गो॒पा, अ॑दृशन् नदृशन् गो॒पा गो॒पा, अ॑दृश॒न् नदृ॑श॒न् नदृ॑शन् नदृशन्\\
गो॒पा गो॒पा, अ॑दृश॒न् नदृ॑शन्न् ।\\
\\
127. गो॒पाः ।\\
गो॒पा इति॑ गो - पाः ।\\
\\
128. अ॒दृ॒श॒न्न् । अदृ॑शन्न् । उ॒द॒हा॒र्यः॑ ॥\\
अ॒दृ॒श॒न् नदृ॑श॒न् नदृ॑शन् नदृशन् नदृश॒न् नदृ॑शन् नुदहा॒र्य॑\\
उद हा॒र्यो॑, अदृ॑शन् नदृशन् नदृश॒न् नदृ॑शन् नुदहा॒र्यः॑ ।\\
\\
129. अदृ॑शन्न् । उ॒द॒हा॒र्यः॑ ॥\\
अदृ॑शन् नुदहा॒र्य॑ उद हा॒र्यो॑, अदृ॑श॒न् नदृ॑शन् नुदहा॒र्यः॑ ।\\
\\
130. उ॒द॒हा॒र्यः॑ ॥\\
उ॒द॒ हा॒र्य॑ इत्यु॑द - हा॒र्यः॑ ।\\
\\
\subsubsection{\eng{sloka 9}}
131. उ॒त । ए॒न॒म् । विश्वा᳚ ।\\
उ॒तैन॑ मेन मु॒तोतैनंँ॒ विश्वा॒ विश्वै॑न मु॒तोतैनंँ॒ विश्वा᳚ ।\\
\\
132. ए॒न॒म् । विश्वा᳚ । भू॒तानि॑ ।\\
ए॒नंँ॒ विश्वा॒ विश्वै॑न मेनंँ॒ विश्वा॑ भू॒तानि॑ भू॒तानि॒ विश्वै॑न मेनंँ॒ विश्वा॑ भू॒तानि॑ ।\\
\\
133. विश्वा᳚ । भू॒तानि॑ । सः ।\\
विश्वा॑ भू॒तानि॑ भू॒तानि॒ विश्वा॒ विश्वा॑ भू॒तानि॒ स स भू॒तानि॒ विश्वा॒\\
विश्वा॑ भू॒तानि॒ सः ।\\
\\
134. भू॒तानि॑ । सः । दृ॒ष्टः ।\\
भू॒तानि॒ स स भू॒तानि॑ भू॒तानि॒ स दृ॒ष्टो दृ॒ष्टः स भू॒तानि॑ भू॒तानि॒ स दृ॒ष्टः ।\\
\\
135. सः । दृ॒ष्टः । मृ॒ड॒या॒ति॒ ।\\
स दृ॒ष्टो दृ॒ष्टः स स दृ॒ष्टो मृ॑डयाति मृडयाति दृ॒ष्टः स स दृ॒ष्टो मृ॑डयाति ।\\
\\
136. दृ॒ष्टः । मृ॒ड॒या॒ति॒ । नः॒ ॥\\
दृ॒ष्टो मृ॑डयाति मृडयाति दृ॒ष्टो दृ॒ष्टो मृ॑डयाति नो नो मृडयाति\\
दृ॒ष्टो दृ॒ष्टो मृ॑डयाति नः ।\\
\\
137. मृ॒ड॒या॒ति॒ । नः॒ ॥\\
मृ॒ड॒या॒ति॒ नो॒ नो॒ मृ॒ड॒या॒ति॒ मृ॒ड॒या॒ति॒ नः॒ ।\\
\\
138. नः॒ ॥\\
न॒ इति॑ नः ।\\
\\
139. नमः॑ । अ॒स्तु॒ । नील॑ग्रीवाय ।\\
नमो॑, अस्त्वस्तु॒ नमो॒ नमो॑, अस्तु॒ नील॑ग्रीवाय॒ नील॑ग्रीवा यास्तु॒\\
नमो॒ नमो॑, अस्तु॒ नील॑ग्रीवाय ।\\
\\
140. अ॒स्तु॒ । नील॑ग्रीवाय । स॒ह॒स्रा॒क्षाय॑ ।\\
अ॒स्तु॒ नील॑ग्रीवाय॒ नील॑ग्रीवा यास्त्वस्तु॒ नील॑ग्रीवाय सहस्रा॒क्षाय॑\\
सहस्रा॒क्षाय॒ नील॑ग्रीवा यास्त्वस्तु॒ नील॑ग्रीवाय सहस्रा॒क्षाय॑ ।\\
\\
141. नील॑ग्रीवाय । स॒ह॒स्रा॒क्षाय॑ । मी॒ढुषे᳚ ॥\\
नील॑ग्रीवाय सहस्रा॒क्षाय॑ सहस्रा॒क्षाय॒ नील॑ग्रीवाय॒ नील॑ग्रीवाय\\
सहस्रा॒क्षाय॑ मी॒ढुषे॑ मी॒ढुषे॑ सहस्रा॒क्षाय॒ नील॑ग्रीवाय॒ नील॑ग्रीवाय\\
सहस्रा॒क्षाय॑ मी॒ढुषे᳚ ।\\
\\
142. नील॑ग्रीवाय ।\\
नील॑ग्रीवा॒येति॒ नील॑ - ग्री॒वा॒य॒ ।\\
\\
143. स॒ह॒स्रा॒क्षाय॑ । मी॒ढुषे᳚ ॥\\
स॒ह॒स्रा॒क्षाय॑ मी॒ढुषे॑ मी॒ढुषे॑ सहस्रा॒क्षाय॑ सहस्रा॒क्षाय॑ मी॒ढुषे᳚ ।\\
\\
144. स॒ह॒स्रा॒क्षाय॑ ।\\
स॒ह॒स्रा॒क्षायेति॑ सहस्र - अ॒क्षाय॑ ।\\
\\
145. मी॒ढुषे᳚ ॥\\
मी॒ढुष॒ इति॑ मी॒ढुषे᳚ ।\\
\\
\subsubsection{\eng{sloka 10}}
146. अथो᳚ । ये । अ॒स्य॒ ।\\
अथो॒ ये येऽथो॒, अथो॒ ये, अ॑स्यास्य॒ येऽथो॒, अथो॒ ये, अ॑स्य ।\\
\\
147. अथो᳚ ।\\
अथो॒ इत्यथो᳚ ।\\
\\
148. ये । अ॒स्य॒ । सत्वा॑नः ।\\
ये अ॑स्यास्य॒ ये ये, अ॑स्य॒ सत्वा॑नः॒ सत्वा॑नो, अस्य॒ ये ये, अ॑स्य॒ सत्वा॑नः ।\\
\\
149. अ॒स्य॒ । सत्वा॑नः । अ॒हम् ।\\
अ॒स्य॒ सत्वा॑नः॒ सत्वा॑नो, अस्यास्य॒ सत्वा॑नो॒ऽह म॒हꣳ सत्वा॑नो,\\
अस्यास्य॒ सत्वा॑नो॒ऽहम् ।\\
\\
150. सत्वा॑नः । अ॒हम् । तेभ्यः॑ ।\\
सत्वा॑नो॒ऽह म॒हꣳ सत्वा॑नः॒ सत्वा॑नो॒ऽहं तेभ्य॒स् तेभ्यो॒ऽहꣳ\\
सत्वा॑नः॒ सत्वा॑नो॒ऽहं तेभ्यः॑ ।\\
\\
151. अ॒हम् । तेभ्यः॑ । अ॒क॒र॒म् ।\\
अ॒हं तेभ्य॒स् तेभ्यो॒ऽ हम॒हं तेभ्यो॑ऽ कर मकर॒न्\\
तेभ्यो॒ऽ हम॒हं तेभ्यो॑ऽ करम् ।\\
\\
152. तेभ्यः॑ । अ॒क॒र॒म् । नमः॑ ॥\\
तेभ्यो॑ऽ कर मकर॒न् तेभ्य॒स् तेभ्यो॑ऽ कर॒न्नमो॒ नमो॑ऽ कर॒न् तेभ्य॒स् तेभ्यो॑ऽ कर॒न्नमः॑ ।\\
\\
153. अ॒क॒र॒म् । नमः॑ ॥\\
अ॒क॒र॒न् नमो॒ नमो॑ऽकर मकर॒न् नमः॑ ।\\
\\
154. नमः॑ ॥\\
नम॒ इति॒ नमः॑ ।\\
\\
155. प्र । मु॒ञ्च॒ । धन्व॑नः ।\\
प्र मु॑ञ्च मुञ्च॒ प्र प्र मु॑ञ्च॒ धन्व॑नो॒ धन्व॑नो मुञ्च॒ प्र प्र मु॑ञ्च॒ धन्व॑नः ।\\
\\
156. मु॒ञ्च॒ । धन्व॑नः । त्वम् ।\\
मु॒ञ्च॒ धन्व॑नो॒ धन्व॑नो मुञ्च मुञ्च॒ धन्व॑ न॒स्त्वं त्वं धन्व॑नो मुञ्च मुञ्च॒\\
धन्व॑ न॒स्त्वम् ।\\
\\
157. धन्व॑नः । त्वम् । उ॒भयोः᳚ ।\\
धन्व॑ न॒स्त्वं त्वं धन्व॑नो॒ धन्व॑ न॒स्त्व मु॒भयो॑ रु॒भ यो॒स्त्वं धन्व॑नो॒\\
धन्व॑ न॒स्त्व मु॒भयोः᳚ ।\\
\\
158. त्वम् । उ॒भयोः᳚ । आर्त्नि॑योः ।\\
त्व मु॒भयो॑ रु॒भ यो॒स्त्वं त्व मु॒भयो॒ रार्त्नि॑यो॒ रार्त्नि॑यो रु॒भ यो॒स्त्वं त्व\\
मु॒भयो॒ रार्त्नि॑योः ।\\
\\
159. उ॒भयोः᳚ । आर्त्नि॑योः । ज्याम् ॥\\
उ॒भयो॒ रार्त्नि॑यो॒ रार्त्नि॑यो रु॒भयो॑ रु॒भयो॒ रार्त्नि॑यो॒र् ज्यां ज्या मार्त्नि॑यो\\
रु॒भयो॑ रु॒भयो॒ रार्त्नि॑यो॒र् ज्याम् ।\\
\\
160. आर्त्नि॑योः । ज्याम् ॥\\
आर्त्नि॑यो॒र् ज्यां ज्या मार्त्नि॑यो॒ रार्त्नि॑याे॒र् ज्याम् ।\\
\\
161. ज्याम् ॥\\
ज्यामिति॒ ज्याम् ।\\
\\
\subsubsection{\eng{sloka 11}}
162. याः । च॒ । ते॒ ।\\
याश्च॑ च॒ या याश्च॑ ते ते च॒ या याश्च॑ ते ।\\
\\
163. च॒ । ते॒ । हस्ते᳚ ।\\
च॒ ते॒ ते॒ च॒ च॒ ते॒ हस्ते॒ हस्ते॑ ते च च ते॒ हस्ते᳚ ।\\
\\
164. ते॒ । हस्ते᳚ । इष॑वः ।\\
ते॒ हस्ते॒ हस्ते॑ ते ते॒ हस्त॒ इष॑व॒ इष॑वो॒ हस्ते॑ ते ते॒ हस्त॒ इष॑वः ।\\
\\
165. हस्ते᳚ । इष॑वः । परा᳚ ।\\
हस्त॒ इष॑व॒ इष॑वो॒ हस्ते॒ हस्त॒ इष॑वः॒ परा॒ परे ष॑वो॒ हस्ते॒ हस्त॒ इष॑वः॒ परा᳚ ।\\
\\
166. इष॑वः । परा᳚ । ताः ।\\
इष॑वः॒ परा॒ परे ष॑व॒ इष॑वः॒ परा॒ तास्ताः परे ष॑व॒ इष॑वः॒ परा॒ ताः ।\\
\\
167. परा᳚ । ताः । भ॒ग॒वः॒ ।\\
परा॒ तास्ताः परा॒ परा॒ ता भ॑गवो भग व॒स्ताः परा॒ परा॒ ता भ॑गवः ।\\
\\
168. ताः । भ॒ग॒वः॒ । व॒प॒ ॥\\
ता भ॑गवो भग व॒स्तास्ता भ॑गवो वप वप भग व॒स्तास्ता भ॑गवो वप ।\\
\\
169. भ॒ग॒वः॒ । व॒प॒ ॥\\
भ॒ग॒वो॒ व॒प॒ व॒प॒ भ॒ग॒वो॒ भ॒ग॒वो॒ व॒प॒ ।\\
\\
170. भ॒ग॒वः॒ ।\\
भ॒ग॒व॒ इति॑ भग - वः॒ ।\\
\\
171. व॒प॒ ॥\\
व॒पेति॑ वप ।\\
\\
172. अ॒व॒तत्य॑ । धनुः॑ । त्वम् ।\\
अ॒व॒तत्य॒ धनु॒र् धनु॑ रव॒ तत्या॑व॒ तत्य॒ धनु॒स्त्वं त्वं धनु॑ रव॒ तत्या॑व॒\\
तत्य॒ धनु॒स्त्वम् ।\\
\\
173. अ॒व॒तत्य॑ ।\\
अ॒व॒ तत्येत्य॑व - तत्य॑ ।\\
\\
174. धनुः॑ । त्वम् । सह॑स्राक्ष ।\\
धनु॒स्त्वं त्वं धनु॒र् धनु॒स्त्वꣳ सह॑स्राक्ष॒ सह॑स्राक्ष॒त्वं धनु॒र् धनु॒स्त्वꣳ\\
सह॑स्राक्ष ।\\
\\
175. त्वम् । सह॑स्राक्ष । शते॑षुधे ॥\\
त्वꣳ सह॑स्राक्ष॒ सह॑स्राक्ष॒ त्वं त्वꣳ सह॑स्राक्ष॒ शते॑षुधे॒ शते॑षुधे॒ सह॑स्राक्ष॒\\
त्वं त्वꣳ सह॑स्राक्ष॒ शते॑षुधे ।\\
\\
176. सह॑स्राक्ष । शते॑षुधे ॥\\
सह॑स्राक्ष॒ शते॑षुधे॒ शते॑षुधे॒ सह॑स्राक्ष॒ सह॑स्राक्ष॒ शते॑षुधे ।\\
\\
177. सह॑स्राक्ष ।\\
सह॑स्रा॒क्षेति॒ सह॑स्र - अ॒क्ष॒ ।\\
\\
178. शते॑षुधे ॥\\
शते॑षुध॒ इति॒ शत॑ - इ॒षु॒धेः॒ ।\\
\\
\subsubsection{\eng{sloka 12}}
179. नि॒शीर्य॑ । श॒ल्याना᳚म् । मुखा᳚ ।\\
नि॒शीर्य॑ श॒ल्यानाꣳ॑ श॒ल्यानां᳚ नि॒शीर्य॑ नि॒शीर्य॑ श॒ल्यानां॒ मुखा॒ मुखा॑\\
श॒ल्यानां᳚ नि॒शीर्य॑ नि॒शीर्य॑ श॒ल्यानां॒ मुखा᳚ ।\\
\\
180. नि॒शिर्य॑ ।\\
नि॒शीर्येति॑ नि - शीर्य॑ ।\\
\\
181. श॒ल्याना᳚म् । मुखा᳚ । शि॒वः ।\\
श॒ल्यानां॒ मुखा॒ मुखा॑ श॒ल्यानाꣳ॑ श॒ल्यानां॒ मुखा॑ शि॒वः शि॒वो मुखा॑\\
श॒ल्यानाꣳ॑ श॒ल्यानां॒ मुखा॑ शि॒वः ।\\
\\
182. मुखा᳚ । शि॒वः । नः॒ ।\\
मुखा॑ शि॒वः शि॒वो मुखा॒ मुखा॑ शि॒वो नो॑ नः शि॒वो मुखा॒ मुखा॑\\
शि॒वो नः॑ ।\\
\\
183. शि॒वः । नः॒ । सु॒मनाः᳚ ।\\
शि॒वो नो॑ नः शि॒वः शि॒वो नः॑ सु॒मनाः᳚ सु॒मना॑ नः शि॒वः शि॒वो नः॑\\
सु॒मनाः᳚ ।\\
\\
184. नः॒ । सु॒मनाः᳚ । भ॒व॒ ॥\\
नः॒ सु॒मनाः᳚ सु॒मना॑ नो नः सु॒मना॑ भव भव सु॒मना॑ नो नः सु॒मना॑ भव ।\\
\\
185. सु॒मनाः᳚ । भ॒व॒ ॥\\
सु॒मना॑ भव भव सु॒मनाः᳚ सु॒मना॑ भव ।\\
\\
186. सु॒मनाः᳚ ।\\
सु॒मना॒ इति॑ सु - मनाः᳚ ।\\
\\
187. भ॒व॒ ॥\\
भ॒वेति॑ भव ।\\
\\
188. विज्य᳚म् । धनुः॑ । क॒प॒र्दिनः॑ ।\\
विज्यं॒ धनु॒र् धनु॒र् विज्यंँ॒ विज्यं॒ धनुः॑ कप॒र्दिनः॑ कप॒र्दिनो॒ धनु॒र् विज्यंँ॒ विज्यं॒ धनुः॑ कप॒र्दिनः॑ ।\\
\\
189. विज्य᳚म् ।\\
विज्य॒मिति॒ वि - ज्य॒म्॒ ।\\
\\
190. धनुः॑ । क॒प॒र्दिनः॑ । विश॑ल्यः ।\\
धनुः॑ कप॒र्दिनः॑ कप॒र्दिनो॒ धनु॒र् धनुः॑ कप॒र्दिनो॒ विश॑ल्यो॒ विश॑ल्यः\\
कप॒र्दिनो॒ धनु॒र् धनुः॑ कप॒र्दिनो॒ विश॑ल्यः ।\\
\\
191. क॒प॒र्दिनः॑ । विश॑ल्यः । बाण॑वान् ।\\
क॒प॒र्दिनो॒ विश॑ल्यो॒ विश॑ल्यः कप॒र्दिनः॑ कप॒र्दिनो॒ विश॑ल्यो॒ बाण॑वा॒न्\\
बाण॑वा॒न्॒. विश॑ल्यः कप॒र्दिनः॑ कप॒र्दिनो॒ विश॑ल्यो॒ बाण॑वान् ।\\
\\
192. विश॑ल्यः । बाण॑वान् । उ॒त ॥\\
विश॑ल्यो॒ बाण॑वा॒न् बाण॑वा॒न्॒. विश॑ल्यो॒ विश॑ल्यो॒ बाण॑वाꣳ उ॒तोत\\
बाण॑वा॒न्॒. विश॑ल्यो॒ विश॑ल्यो॒ बाण॑वाꣳ उ॒त ।\\
\\
193. विश॑ल्यः ।\\
विश॑ल्य॒ इति॒ वि - श॒ल्यः॒ ।\\
\\
194. बाण॑वान् । उ॒त ॥\\
बाण॑वाꣳ उ॒तोत बाण॑वा॒न् बाण॑वाꣳ उ॒त ।\\
\\
195. बाण॑वान् ।\\
बाण॑वा॒ निति॒ बाण॑ - वा॒न् ।\\
\\
196. उ॒त ॥\\
उ॒तेत्यु॒त ।\\
\\
\subsubsection{\eng{sloka 13}}
197. अने॑शन्न् । अ॒स्य॒ । इष॑वः ।\\
अने॑शन् नस्या॒स्या ने॑श॒न्न ने॑शन्न॒स्ये ष॑व॒ इष॑वो, अ॒स्या ने॑श॒न्न\\
ने॑शन्न॒स्ये ष॑वः ।\\
\\
198. अ॒स्य॒ । इष॑वः । आ॒भुः ।\\
अ॒स्ये ष॑व॒ इष॑वो, अस्या॒स्ये ष॑व आ॒भु रा॒भु रिष॑वो, अस्या॒स्ये ष॑व आ॒भुः ।\\
\\
199. इष॑वः । आ॒भुः । अ॒स्य॒ ।\\
इष॑व आ॒भु रा॒भु रि ष॑व॒ इष॑व आ॒भु र॑स्यास्या॒भु रि ष॑व॒ इष॑व आ॒भु र॑स्य ।\\
\\
200. आ॒भुः । अ॒स्य॒ । नि॒ष॒ङ्गथिः॑ ॥\\
आ॒भु र॑स्यास्या॒भु रा॒भु र॑स्य निष॒ङ्गथि॑र् निष॒ङ्गथि॑ रस्या॒भु रा॒भु र॑स्य\\
निष॒ङ्गथिः॑ ।\\
\\
201. अ॒स्य॒ । नि॒ष॒ङ्गथिः॑ ॥\\
अ॒स्य॒ नि॒ष॒ङ्गथि॑र् निष॒ङ्गथि॑ रस्यास्य निष॒ङ्गथिः॑ ।\\
\\
202. नि॒ष॒ङ्गथिः॑ ॥\\
नि॒ष॒ङ्गथि॒ रिति॑ निष॒ङ्गथिः॑ ।\\
\\
203. या । ते॒ । हे॒तिः ।\\
या ते॑ ते॒ या या ते॑ हे॒तिर्. हे॒तिस्ते॒ या या ते॑ हे॒तिः ।\\
\\
204. ते॒ । हे॒तिः । मी॒ढु॒ष्ट॒म॒ ।\\
ते॒ हे॒तिर्. हे॒तिस्ते॑ ते हे॒तिर् मी॑ढुष्टम मीढुष्टम हे॒तिस्ते॑ ते हे॒तिर्\\
मी॑ढुष्टम ।\\
\\
205. हे॒तिः । मी॒ढु॒ष्ट॒म॒ । हस्ते᳚ ।\\
हे॒तिर् मी॑ढुष्टम मीढुष्टम हे॒तिर्. हे॒तिर् मी॑ढुष्टम॒ हस्ते॒ हस्ते॑ मीढुष्टम\\
हे॒तिर्. हे॒तिर् मी॑ढुष्टम॒ हस्ते᳚ ।\\
\\
206. मी॒ढु॒ष्ट॒म॒ । हस्ते᳚ । ब॒भूव॑ ।\\
मी॒ढु॒ष्ट॒म॒ हस्ते॒ हस्ते॑ मीढुष्टम मीढुष्टम॒ हस्ते॑ ब॒भूव॑ ब॒भूव॒ हस्ते॑ मीढुष्टम\\
मीढुष्टम॒ हस्ते॑ ब॒भूव॑ ।\\
\\
207. मी॒ढु॒ष्ट॒म॒ ।\\
मी॒ढु॒ष्ट॒मेति॑ मीढुः - त॒म॒ ।\\
\\
208. हस्ते᳚ । ब॒भूव॑ । ते॒ ।\\
हस्ते॑ ब॒भूव॑ ब॒भूव॒ हस्ते॒ हस्ते॑ ब॒भूव॑ ते ते ब॒भूव॒ हस्ते॒ हस्ते॑ ब॒भूव॑ ते ।\\
\\
209. ब॒भूव॑ । ते॒ । धनुः॑ ॥\\
ब॒भूव॑ ते ते ब॒भूव॑ ब॒भूव॑ ते॒ धनु॒र् धनु॑स्ते ब॒भूव॑ ब॒भूव॑ ते॒ धनुः॑ ।\\
\\
210. ते॒ । धनुः॑ ॥\\
ते॒ धनु॒र् धनु॑स्ते ते॒ धनुः॑ ।\\
\\
211. धनुः॑ ॥\\
धनु॒ रिति॒ धनुः॑ ।\\
\\
\subsubsection{\eng{sloka 14}}
212. तया᳚ । अ॒स्मान् । वि॒श्वतः॑ ।\\
तया॒ऽस्मा न॒स्मान् तया॒ तया॒ऽस्मान्. वि॒श्वतो॑ वि॒श्वतो॑,\\
अ॒स्मान् तया॒ तया॒ऽस्मान्. वि॒श्वतः॑ ।\\
\\
213. अ॒स्मान् । वि॒श्वतः॑ । त्वम् ।\\
अ॒स्मान्. वि॒श्वतो॑ वि॒श्वतो॑, अ॒स्मा न॒स्मान्. वि॒श्व त॒स्त्वं त्वंँवि॒श्वतो॑,\\
अ॒स्मा न॒स्मान्. वि॒श्व त॒स्त्वम् ।\\
\\
214. वि॒श्वतः॑ । त्वम् । अ॒य॒क्ष्मया᳚ ।\\
वि॒श्व त॒स्त्वं त्वंँ वि॒श्वतो॑ वि॒श्व त॒स्त्व म॑य॒क्ष्मया॑ऽ य॒क्ष्मया॒ त्वंँ वि॒श्वतो॑ वि॒श्व त॒स्त्व म॑य॒क्ष्मया᳚ ।\\
\\
215. त्वम् । अ॒य॒क्ष्मया᳚ । परि॑ ।\\
त्व म॑य॒क्ष्मया॑ऽ य॒क्ष्मया॒ त्वं त्व म॑य॒क्ष्मया॒ परि॒ पर्य॑ य॒क्ष्मया॒ त्वं त्व\\
म॑य॒क्ष्मया॒ परि॑ ।\\
\\
216. अ॒य॒क्ष्मया᳚ । परि॑ । भु॒ज॒ ॥\\
अ॒य॒क्ष्मया॒ परि॒ पर्य॑ य॒क्ष्मया॑ऽ य॒क्ष्मया॒ परि॑ब्भुज भुज॒ पर्य॑ य॒क्ष्मया॑ऽ\\
य॒क्ष्मया॒ परि॑ब्भुज ।\\
\\
217. परि॑ । भु॒ज॒ ॥\\
परि॑ब्भुज भुज॒ परि॒ परि॑ब्भुज ।\\
\\
218. भु॒ज॒ ॥\\
भु॒जेति॑ भुज ।\\
\\
219. नमः॑ । ते॒ । अ॒स्तु॒ ।\\
नम॑स्ते ते॒ नमो॒ नम॑स्ते, अस्त्वस्तु ते॒ नमो॒ नम॑स्ते, अस्तु ।\\
\\
220. ते॒ । अ॒स्तु॒ । आयु॑धाय ।\\
ते॒, अ॒स्त्व॒स्तु॒ ते॒ ते॒, अ॒स्त्वायु॑ धा॒या यु॑धायास्तु ते ते, अ॒स्त्वायु॑ धाय ।\\
\\
221. अ॒स्तु॒ । आयु॑धाय । अना॑तताय ।\\
अ॒स्त्वायु॑ धा॒या यु॑धाया स्त्व॒ स्त्वा यु॑धा॒या ना॑त ता॒या ना॑त ता॒या यु॑धाया\\
स्त्व॒ स्त्वा यु॑धा॒या ना॑त ताय ।\\
\\
222. आयु॑धाय । अना॑तताय । धृ॒ष्णवे᳚ ॥\\
आयु॑धा॒या ना॑तता॒या ना॑तता॒या यु॑धा॒या यु॑धा॒या ना॑तताय धृ॒ष्णवे॑\\
धृ॒ष्णवेऽ ना॑तता॒या यु॑धा॒या यु॑धा॒या ना॑तताय धृ॒ष्णवे᳚ ।\\
\\
223. अना॑तताय । धृ॒ष्णवे᳚ ॥\\
अना॑त ताय धृ॒ष्णवे॑ धृ॒ष्णवेऽ ना॑तता॒या ना॑तताय धृ॒ष्णवे᳚ ।\\
\\
224. अना॑तताय ।\\
अना॑त ता॒येत्यना᳚ - त॒ता॒य॒ ।\\
\\
225. धृ॒ष्णवे᳚ ॥\\
धृ॒ष्णव॒ इति॑ धृ॒ष्णवे᳚ ।\\
\\
\subsubsection{\eng{sloka 15}}
226. उ॒भाभ्या᳚म् । उ॒त । ते॒ ।\\
उ॒भाभ्या॑ मु॒तोतो भाभ्या॑ मु॒भाभ्या॑ मु॒तते॑ त उ॒तो भाभ्या॑ मु॒भाभ्या॑ मु॒तते᳚ ।\\
\\
227. उ॒त । ते॒ । नमः॑ ।\\
उ॒त ते॑त उ॒तो तते॒ नमो॒ नम॑स्त उ॒तो तते॒ नमः॑ ।\\
\\
228. ते॒ । नमः॑ । बा॒हुभ्या᳚म् ।\\
ते॒ नमो॒ नम॑स्ते ते॒ नमो॑ बा॒हुभ्यां᳚ बा॒हुभ्यां॒ नम॑स्ते ते॒ नमो॑ बा॒हुभ्या᳚म् ।\\
\\
229. नमः॑ । बा॒हुभ्या᳚म् । तव॑ ।\\
नमो॑ बा॒हुभ्यां᳚ बा॒हुभ्यां॒ नमो॒ नमो॑ बा॒हुभ्यां॒ तव॒ तव॑ बा॒हुभ्यां॒ नमो॒ नमो॑\\
बा॒हुभ्यां॒ तव॑ ।\\
\\
230. बा॒हुभ्या᳚म् । तव॑ । धन्व॑ने ॥\\
बा॒हुभ्यां॒ तव॒ तव॑ बा॒हुभ्यां᳚ बा॒हुभ्यां॒ तव॒ धन्व॑ने॒ धन्व॑ने॒ तव॑ बा॒हुभ्यां᳚\\
बा॒हुभ्यां॒ तव॒ धन्व॑ने ।\\
\\
231. बा॒हुभ्या᳚म् ।\\
बा॒हुभ्या॒मिति॑ बा॒हु - भ्या॒म् ।\\
\\
232. तव॑ । धन्व॑ने ॥\\
तव॒ धन्व॑ने॒ धन्व॑ने॒ तव॒ तव॒ धन्व॑ने ।\\
\\
233. धन्व॑ने ॥\\
धन्व॑न॒ इति॒ धन्व॑ने ।\\
\\
234. परि॑ । ते॒ । धन्व॑नः ।\\
परि॑ ते ते॒ परि॒ परि॑ ते॒ धन्व॑नो॒ धन्व॑ नस्ते॒ परि॒ परि॑ ते॒ धन्व॑नः ।\\
\\
235. ते॒ । धन्व॑नः । हे॒तिः ।\\
ते॒ धन्व॑नो॒ धन्व॑ नस्ते ते॒ धन्व॑नो हे॒तिर्. हे॒तिर् धन्व॑ नस्ते ते॒\\
धन्व॑नो हे॒तिः ।\\
\\
236. धन्व॑नः । हे॒तिः । अ॒स्मान् ।\\
धन्व॑नो हे॒तिर्. हे॒तिर् धन्व॑नो॒ धन्व॑नो हे॒ति र॒स्मा न॒स्मान्. हे॒तिर् धन्व॑नो॒\\
धन्व॑नो हे॒ति र॒स्मान् ।\\
\\
237. हे॒तिः । अ॒स्मान् । वृ॒ण॒क्तु॒ ।\\
हे॒ति र॒स्मा न॒स्मान्. हे॒तिर्. हे॒ति र॒स्मान्. वृ॑णक्तु वृणक् त्व॒स्मान्.\\
हे॒तिर्. हे॒ति र॒स्मान्. वृ॑णक्तु ।\\
\\
238. अ॒स्मान् । वृ॒ण॒क्तु॒ । वि॒श्वतः॑ ॥\\
अ॒स्मान्. वृ॑णक्तु वृणक् त्व॒स्मा न॒स्मान्. वृ॑णक्तु वि॒श्वतो॑ वि॒श्वतो॑\\
वृणक् त्व॒स्मा न॒स्मान्. वृ॑णक्तु वि॒श्वतः॑ ।\\
\\
239. वृ॒ण॒क्तु॒ । वि॒श्वतः॑ ॥\\
वृ॒ण॒क्तु॒ वि॒श्वतो॑ वि॒श्वतो॑ वृणक्तु वृणक्तु वि॒श्वतः॑ ।\\
\\
240. वि॒श्वतः॑ ॥\\
वि॒श्वत॒ इति॑ वि॒श्वतः॑ ।\\
\\
\subsubsection{\eng{sloka 16}}
241. अथो॒ । यः । इ॒षु॒धि ।\\
अथो॒ यो योऽ थो॒, अथो॒ य इ॑षु॒धि रि॑षु॒ धिर् योऽ थो॒, अथो॒ य इ॑षु॒धिः ।\\
\\
242. अथो᳚ ।\\
अथो॒ इत्यथो᳚ ।\\
\\
243. यः । इ॒षु॒धिः । तव॑ ।\\
य इ॑षु॒धि रि॑षु॒ धिर् यो य इ॑षु॒ धिस्तव॒ तवे॑ षु॒धिर् यो य इ॑षु॒ धिस्तव॑ ।\\
\\
244. इ॒षु॒धिः । तव॑ । आ॒रे ।\\
इ॒षु॒ धिस्तव॒ तवे॑ षु॒धि रि॑षु॒ धिस्तवा॒र आ॒रे तवे॑ षु॒धि रि॑षु॒ धिस्त वा॒रे ।\\
\\
245. इ॒षु॒धिः ।\\
इ॒षु॒धि रिती॑षु - धिः ।\\
\\
246. तव॑ । आ॒रे । अ॒स्मत् ।\\
तवा॒र आ॒रे तव॒ तवा॒रे, अ॒स्म द॒स्म दा॒रे तव॒ तवा॒रे, अ॒स्मत् ।\\
\\
247. आ॒रे । अ॒स्मत् । नि ।\\
आ॒रे, अ॒स्म द॒स्म दा॒र आ॒रे अ॒स्मन् निन्य॑स्म दा॒र आ॒रे, अ॒स्मन्नि ।\\
\\
248. अ॒स्मत् । नि । धे॒हि॒ ।\\
अ॒स्मन् निन्य॑स्म द॒स्मन्नि धे॑हि धेहि॒न्य॑स्म द॒स्मन्नि धे॑हि ।\\
\\
249. नि । धे॒हि॒ । तम् ॥\\
नि धे॑हि धेहि॒ नि नि धे॑हि॒ तं तं धे॑हि॒ नि नि धे॑हि॒ तम् ।\\
\\
250. धे॒हि॒ । तम् ॥\\
धे॒हि॒ तं तं धे॑हि धेहि॒ तम् ।\\
\\
251. तम् ॥\\
तमिति॒ तं ।\\
\subsection{\eng{Anuvaka 2}}
1. नमः॑ । हिर॑ण्यबाहवे । से॒ना॒न्ये᳚ ।\\
नमो॒ हिर॑ण्यबाहवे॒ हिर॑ण्यबाहवे॒ नमो॒ नमो॒ हिर॑ण्यबाहवे सेना॒न्ये॑\\
सेना॒न्ये॑ हिर॑ण्यबाहवे॒ नमो॒ नमो॒ हिर॑ण्यबाहवे सेना॒न्ये᳚ ।\\
\\
2. हिर॑ण्यबाहवे । से॒ना॒न्ये᳚ । दि॒शाम् ।\\
हिर॑ण्यबाहवे सेना॒न्ये॑ सेना॒न्ये॑ हिर॑ण्यबाहवे॒ हिर॑ण्यबाहवे सेना॒न्ये॑\\
दि॒शां दि॒शाꣳ से॑ना॒न्ये॑ हिर॑ण्यबाहवे॒ हिर॑ण्यबाहवे सेना॒न्ये॑ दि॒शाम् ।\\
\\
3. हिर॑ण्यबाहवे ।\\
हिर॑ण्यबाहव॒ इति॒ हिर॑ण्य - बा॒ह॒वे॒ ।\\
\\
4. से॒ना॒न्ये᳚ । दि॒शाम् । च॒ ।\\
से॒ना॒न्ये॑ दि॒शां दि॒शाꣳ से॑ना॒न्ये॑ सेना॒न्ये॑ दि॒शाञ्च॑ च दि॒शाꣳ से॑ना॒न्ये॑\\
सेना॒न्ये॑ दि॒शाञ्च॑ ।\\
\\
5. से॒ना॒न्ये᳚ ।\\
से॒ना॒न्य॑ इति॑ सेना - न्ये᳚ ।\\
\\
6. दि॒शाम् । च॒ । पत॑ये ।\\
दि॒शाञ्च॑ च दि॒शां दि॒शाञ्च॒ पत॑ये॒ पत॑ये च दि॒शां दि॒शाञ्च॒ पत॑ये ।\\
\\
7. च॒ । पत॑ये । नमः॑ ।\\
च॒ पत॑ये॒ पत॑ये च च॒ पत॑ये॒ नमो॒ नम॒ स्पत॑ये च च॒ पत॑ये॒ नमः॑ ।\\
\\
8. पत॑ये । नमः॑ ।\\
पत॑ये॒ नमो॒ नम॒ स्पत॑ये॒ पत॑ये॒ नमो॒ नमः॑ ।\\
\\
9. नमः॑ । नमः॑ ।\\
नमो॒ नमः॑ ।\\
\\
10. नमः॑ । वृ॒क्षेभ्यः॑ । हरि॑केशेभ्यः ।\\
नमो॑ वृ॒क्षेभ्यो॑ वृ॒क्षेभ्यो॒ नमो॒ नमो॑ वृ॒क्षेभ्यो॒ हरि॑केशेभ्यो॒ हरि॑केशेभ्यो\\
वृ॒क्षेभ्यो॒ नमो॒ नमो॑ वृ॒क्षेभ्यो॒ हरि॑केशेभ्यः ।\\
\\
11. वृ॒क्षेभ्यः॑ । हरि॑केशेभ्यः । प॒शू॒नाम् ।\\
वृ॒क्षेभ्यो॒ हरि॑केशेभ्यो॒ हरि॑केशेभ्यो वृ॒क्षेभ्यो॑ वृ॒क्षेभ्यो॒ हरि॑केशेभ्यः\\
पशू॒नां प॑शू॒नाꣳ हरि॑केशेभ्यो वृ॒क्षेभ्यो॑ वृ॒क्षेभ्यो॒ हरि॑केशेभ्यः पशू॒नाम् ।\\
\\
12. हरि॑केशेभ्यः । प॒शू॒नाम् । पत॑ये ।\\
हरि॑केशेभ्यः पशू॒नां प॑शू॒नाꣳ हरि॑केशेभ्यो॒ हरि॑केशेभ्यः पशू॒नां पत॑ये॒\\
पत॑ये पशू॒नाꣳ हरि॑केशेभ्यो॒ हरि॑केशेभ्यः पशू॒नां पत॑ये ।\\
\\
13. हरि॑केशेभ्य ।\\
हरि॑केशेभ्य॒ इति॒ हरि॑ - के॒शे॒भ्यः॒ ।\\
\\
14. प॒शू॒नाम् । पत॑ये । नमः॑ ।\\
प॒शू॒नां पत॑ये॒ पत॑ये पशू॒नां प॑शू॒नां पत॑ये॒ नमो॒ नम॒ स्पत॑ये पशू॒नां प॑शू॒नां\\
पत॑ये॒ नमः॑ ।\\
\\
15. पत॑ये । नमः॑ ।\\
पत॑ये॒ नमो॒ नम॒ स्पत॑ये॒ पत॑ये॒ नमो॒ नमः॑ ।\\
\\
16. नमः॑ । नमः॑ ।\\
नमो॒ नमः॑ ।\\
\\
17. नमः॑ । स॒स्पिञ्ज॑राय । त्विषी॑मते ।\\
नमः॑ स॒स्पिञ्ज॑राय स॒स्पिञ्ज॑राय॒ नमो॒ नमः॑ स॒स्पिञ्ज॑राय॒ त्विषी॑मते॒\\
त्विषी॑मते स॒स्पिञ्ज॑राय॒ नमो॒ नमः॑ स॒स्पिञ्ज॑राय॒ त्विषी॑मते ।\\
\\
18. स॒स्पिञ्ज॑राय । त्विषी॑मते । प॒थी॒नाम् ।\\
स॒स्पिञ्ज॑राय॒ त्विषी॑मते॒ त्विषी॑मते स॒स्पिञ्ज॑राय स॒स्पिञ्ज॑राय॒ त्विषी॑मते\\
पथी॒नां प॑थी॒नां त्विषी॑मते स॒स्पिञ्ज॑राय स॒स्पिञ्ज॑राय॒ त्विषी॑मते पथी॒नाम् ।\\
\\
19. त्विषी॑मते । प॒थी॒नाम् । पत॑ये ।\\
त्विषी॑मते पथी॒नां प॑थी॒नां त्विषी॑मते॒ त्विषी॑मते पथी॒नां पत॑ये॒ पत॑ये पथी॒नां\\
त्विषी॑मते॒ त्विषी॑मते पथी॒नां पत॑ये ।\\
\\
20. त्विषी॑मते ।\\
त्विषी॑मत॒ इति॒ त्विषि - म॒ते॒ ।\\
\\
21. प॒थी॒नाम् । पत॑ये । नमः॑ ।\\
प॒थी॒नां पत॑ये॒ पत॑ये पथी॒नां प॑थी॒नां पत॑ये॒ नमो॒ नम॒ स्पत॑ये पथी॒नां\\
प॑थी॒नां पत॑ये॒ नमः॑ ।\\
\\
22. पत॑ये । नमः॑ ।\\
पत॑ये॒ नमो॒ नम॒ स्पत॑ये॒ पत॑ये॒ नमो॒ नमः॑ ।\\
\\
23. नमः॑ । नमः॑ ।\\
नमो॒ नमः॑ ।\\
\\
24. नमः॑ । ब॒भ्लु॒शाय॑ । वि॒व्या॒धिने᳚ ।\\
नमो॑ बभ्लु॒शाय॑ बभ्लु॒शाय॒ नमो॒ नमो॑ बभ्लु॒शाय॑ विव्या॒धिने॑\\
विव्या॒धिने॑ बभ्लु॒शाय॒ नमो॒ नमो॑ बभ्लु॒शाय॑ विव्या॒धिने᳚ ।\\
\\
25. ब॒भ्लु॒शाय॑ । वि॒व्या॒धिने᳚ । अन्ना॑नाम् ।\\
ब॒भ्लु॒शाय॑ विव्या॒धिने॑ विव्या॒धिने॑ बभ्लु॒शाय॑ बभ्लु॒शाय॑ विव्या॒धिनेऽन्ना॑ना॒ मन्ना॑नांँविव्या॒धिने॑ बभ्लु॒शाय॑ बभ्लु॒शाय॑ विव्या॒धिनेऽन्ना॑नाम् ।\\
\\
26. वि॒व्या॒धिने᳚ । अन्ना॑नाम् । पत॑ये ।\\
वि॒व्या॒धिनेऽन्ना॑ना॒ मन्ना॑नांँविव्या॒धिने॑ विव्या॒धिनेऽन्ना॑नां॒ पत॑ये॒ पत॒येऽन्ना॑नांँविव्या॒धिने॑ विव्या॒धिनेऽन्ना॑नां॒ पत॑ये ।\\
\\
27. वि॒व्या॒धिने᳚ ।\\
वि॒व्या॒धिन॒ इति॑ वि - व्या॒धिने᳚ ।\\
\\
28. अन्ना॑नाम् । पत॑ये । नमः॑ ।\\
अन्ना॑नां॒ पत॑ये॒ पत॒येऽन्ना॑ना॒ मन्ना॑नां॒ पत॑ये॒ नमो॒ नम॒ स्पत॒येऽन्ना॑ना॒\\
मन्ना॑नां॒ पत॑ये॒ नमः॑ ।\\
\\
29. पत॑ये । नमः॑ ।\\
पत॑ये॒ नमो॒ नम॒ स्पत॑ये॒ पत॑ये॒ नमो॒ नमः॑ ।\\
\\
30. नमः॑ । नमः॑ ।\\
नमो॒ नमः॑ ।\\
\\
31. नमः॑ । हरि॑केशाय । उ॒प॒वी॒तिने᳚ ।\\
नमो॒ हरि॑केशाय॒ हरि॑केशाय॒ नमो॒ नमो॒ हरि॑केशा योपवी॒तिन॑\\
उपवी॒तिने॒ हरि॑केशाय॒ नमो॒ नमो॒ हरि॑केशा योपवी॒तिने᳚ ।\\
\\
32. हरि॑केशाय । उ॒प॒वी॒तिने᳚ । पु॒ष्टाना᳚म् ।\\
हरि॑केशा योपवी॒तिन॑ उपवी॒तिने॒ हरि॑केशाय॒ हरि॑केशा योपवी॒तिने॑\\
पु॒ष्टानां᳚ पु॒ष्टाना॑ मुपवी॒तिने॒ हरि॑केशाय॒ हरि॑केशा योपवी॒तिने॑ पु॒ष्टाना᳚म् ।\\
\\
33. हरि॑केशाय ।\\
हरि॑केशा॒येति॒ हरि॑ - के॒शा॒य॒ ।\\
\\
34. उ॒प॒वी॒तिने᳚ । पु॒ष्टाना᳚म् । पत॑ये ।\\
उ॒प॒वी॒तिने॑ पु॒ष्टानां᳚ पु॒ष्टाना॑ मुपवी॒तिन॑ उपवी॒तिने॑ पु॒ष्टानां॒ पत॑ये॒ पत॑ये\\
पु॒ष्टाना॑ मुपवी॒तिन॑ उपवी॒तिने॑ पु॒ष्टानां॒ पत॑ये ।\\
\\
35. उ॒प॒वी॒तिने᳚ ।\\
उ॒प॒वी॒तिन॒ इत्यु॑प - वी॒तिने᳚ ।\\
\\
36. पु॒ष्टाना᳚म् । पत॑ये । नमः॑ ।\\
पु॒ष्टानां॒ पत॑ये॒ पत॑ये पु॒ष्टानां᳚ पु॒ष्टानां॒ पत॑ये॒ नमो॒ नम॒ स्पत॑ये पु॒ष्टानां᳚\\
पु॒ष्टानां॒ पत॑ये॒ नमः॑ ।\\
\\
37. पत॑ये । नमः॑ ।\\
पत॑ये॒ नमो॒ नम॒ स्पत॑ये॒ पत॑ये॒ नमो॒ नमः॑ ।\\
\\
38. नमः॑ । नमः॑ ।\\
नमो॒ नमः॑ ।\\
\\
39. नमः॑ । भ॒वस्य॑ । हे॒त्यै ।\\
नमो॑ भ॒वस्य॑ भ॒वस्य॒ नमो॒ नमो॑ भ॒वस्य॑ हे॒त्यै हे॒त्यै भ॒वस्य॒ नमो॒ नमो॑\\
भ॒वस्य॑ हे॒त्यै ।\\
\\
40. भ॒वस्य॑ । हे॒त्यै । जग॑ताम् ।\\
भ॒वस्य॑ हे॒त्यै हे॒त्यै भ॒वस्य॑ भ॒वस्य॑ हे॒त्यै जग॑तां॒ जग॑ताꣳ हे॒त्यै\\
भ॒वस्य॑ भ॒वस्य॑ हे॒त्यै जग॑ताम् ।\\
\\
41. हे॒त्यै । जग॑ताम् । पत॑ये ।\\
हे॒त्यै जग॑तां॒ जग॑ताꣳ हे॒त्यै हे॒त्यै जग॑तां॒ पत॑ये॒ पत॑ये॒ जग॑ताꣳ हे॒त्यै हे॒त्यै\\
जग॑तां॒ पत॑ये ।\\
\\
42. जग॑ताम् । पत॑ये । नमः॑ ।\\
जग॑तां॒ पत॑ये॒ पत॑ये॒ जग॑तां॒ जग॑तां॒ पत॑ये॒ नमो॒ नम॒ स्पत॑ये॒ जग॑तां॒ जग॑तां॒\\
पत॑ये॒ नमः॑ ।\\
\\
43. पत॑ये । नमः॑ ।\\
पत॑ये॒ नमो॒ नम॒स्पत॑ये॒ पत॑ये॒ नमो॒ नमः॑ ।\\
\\
44. नमः॑ । नमः॑ ।\\
नमो॒ नमः॑ ।\\
\\
45. नमः॑ । रु॒द्राय॑ । आ॒त॒ता॒विने᳚ ।\\
नमो॑ रु॒द्राय॑ रु॒द्राय॒ नमो॒ नमो॑ रु॒द्राया॑ तता॒विन॑ आतता॒विने॑ रु॒द्राय॒\\
नमो॒ नमो॑ रु॒द्राया॑ तता॒विने᳚ ।\\
\\
46. रु॒द्राय॑ । आ॒त॒ता॒विने᳚ । क्षेत्रा॑णाम् ।\\
रु॒द्राया॑ तता॒विन॑ आतता॒विने॑ रु॒द्राय॑ रु॒द्राया॑ तता॒विने॒ क्षेत्रा॑णां॒ क्षेत्रा॑णा\\
मातता॒विने॑ रु॒द्राय॑ रु॒द्राया॑ तता॒विने॒ क्षेत्रा॑णाम् ।\\
\\
47. आ॒त॒ता॒विने᳚ । क्षेत्रा॑णाम् । पत॑ये ।\\
आ॒त॒ता॒विने॒ क्षेत्रा॑णां॒ क्षेत्रा॑णा मातता॒विन॑ आतता॒विने॒ क्षेत्रा॑णां॒ पत॑ये॒ पत॑ये॒\\
क्षेत्रा॑णा मातता॒विन॑ आतता॒विने॒ क्षेत्रा॑णां॒ पत॑ये ।\\
\\
48. आ॒त॒ता॒विने᳚ ।\\
आ॒त॒ता॒विन॒ इत्या᳚ - त॒ता॒विने᳚ ।\\
\\
49. क्षेत्रा॑णाम् । पत॑ये । नमः॑ ।\\
क्षेत्रा॑णां॒ पत॑ये॒ पत॑ये॒ क्षेत्रा॑णां॒ क्षेत्रा॑णां॒ पत॑ये॒ नमो॒ नम॒ स्पत॑ये॒ क्षेत्रा॑णां॒\\
क्षेत्रा॑णां॒ पत॑ये॒ नमः॑ ।\\
\\
50. पत॑ये । नमः॑ ।\\
पत॑ये॒ नमो॒ नम॒स्पत॑ये॒ पत॑ये॒ नमो॒ नमः॑ ।\\
\\
51. नमः॑ । नमः॑ ।\\
नमो॒ नमः॑ ।\\
\\
52. नमः॑ । सू॒ताय॑ । अह॑न्त्याय ।\\
नमः॑ सू॒ताय॑ सू॒ताय॒ नमो॒ नमः॑ सू॒ताया ह॑न्त्या॒या ह॑न्त्याय सू॒ताय॒ नमो॒\\
नमः॑ सू॒ताया ह॑न्त्याय ।\\
\\
53. सू॒ताय॑ । अह॑न्त्याय । वना॑नाम् ।\\
सू॒ताया ह॑न्त्या॒या ह॑न्त्याय सू॒ताय॑ सू॒ताया ह॑न्त्याय॒ वना॑नां॒ँवना॑ना॒ मह॑न्त्याय सू॒ताय॑ सू॒ताया ह॑न्त्याय॒ वना॑नाम् ।\\
\\
54. अह॑न्त्याय । वना॑नाम् । पत॑ये ।\\
अह॑न्त्याय॒ वना॑नां॒ँवना॑ना॒ मह॑न्त्या॒या ह॑न्त्याय॒ वना॑नां॒ पत॑ये॒ पत॑ये॒\\
वना॑ना॒ मह॑न्त्या॒या ह॑न्त्याय॒ वना॑नां॒ पत॑ये ।\\
\\
55. वना॑नाम् । पत॑ये । नमः॑ ।\\
वना॑नां॒ पत॑ये॒ पत॑ये॒ वना॑नां॒ँवना॑नां॒ पत॑ये॒ नमो॒ नम॒ स्पत॑ये॒ वना॑नां॒ँवना॑नां॒\\
पत॑ये॒ नमः॑ ।\\
\\
56. पत॑ये । नमः॑ ।\\
पत॑ये॒ नमो॒ नम॒स्पत॑ये॒ पत॑ये॒ नमो॒ नमः॑ ।\\
\\
57. नमः॑ । नमः॑ ।\\
नमो॒ नमः॑ ।\\
\\
58. नमः॑ । रोहि॑ताय । स्थ॒पत॑ये ।\\
नमो॒ रोहि॑ताय॒ रोहि॑ताय॒ नमो॒ नमो॒ रोहि॑ताय स्थ॒पत॑ये स्थ॒पत॑ये॒ रोहि॑ताय॒\\
नमो॒ नमो॒ रोहि॑ताय स्थ॒पत॑ये ।\\
\\
59. रोहि॑ताय । स्थ॒पत॑ये । वृ॒क्षाणा᳚म् ।\\
रोहि॑ताय स्थ॒पत॑ये स्थ॒पत॑ये॒ रोहि॑ताय॒ रोहि॑ताय स्थ॒पत॑ये वृ॒क्षाणां᳚ँवृ॒क्षाणाꣲ॑ स्थ॒पत॑ये॒ रोहि॑ताय॒ रोहि॑ताय स्थ॒पत॑ये वृ॒क्षाणा᳚म् ।\\
\\
60. स्थ॒पत॑ये । वृ॒क्षाणा᳚म् । पत॑ये ।\\
स्थ॒पत॑ये वृ॒क्षाणां᳚ँवृ॒क्षाणाꣲ॑ स्थ॒पत॑ये स्थ॒पत॑ये वृ॒क्षाणां॒ पत॑ये॒ पत॑ये\\
वृ॒क्षाणाꣲ॑ स्थ॒पत॑ये स्थ॒पत॑ये वृ॒क्षाणां॒ पत॑ये ।\\
\\
61. वृ॒क्षाणा᳚म् । पत॑ये । नमः॑ ।\\
वृ॒क्षाणां॒ पत॑ये॒ पत॑ये वृ॒क्षाणां᳚ँवृ॒क्षाणां॒ पत॑ये॒ नमो॒ नम॒स्पत॑ये वृ॒क्षाणां᳚ँवृ॒क्षाणां॒ पत॑ये॒ नमः॑ ।\\
\\
62. पत॑ये । नमः॑ ।\\
पत॑ये॒ नमो॒ नम॒स्पत॑ये॒ पत॑ये॒ नमो॒ नमः॑ ।\\
\\
63. नमः॑ । नमः॑ ।\\
नमो॒ नमः॑ ।\\
\\
64. नमः॑ । म॒न्त्रिणे᳚ । वा॒णि॒जाय॑ ।\\
नमो॑ म॒न्त्रिणे॑ म॒न्त्रिणे॒ नमो॒ नमो॑ म॒न्त्रिणे॑ वाणि॒जाय॑ वाणि॒जाय॑ म॒न्त्रिणे॒\\
नमो॒ नमो॑ म॒न्त्रिणे॑ वाणि॒जाय॑ ।\\
\\
65. म॒न्त्रिणे᳚ । वा॒णि॒जाय॑ । कक्षा॑णाम् ।\\
म॒न्त्रिणे॑ वाणि॒जाय॑ वाणि॒जाय॑ म॒न्त्रिणे॑ म॒न्त्रिणे॑ वाणि॒जाय॒ कक्षा॑णां॒\\
कक्षा॑णांँवाणि॒जाय॑ म॒न्त्रिणे॑ म॒न्त्रिणे॑ वाणि॒जाय॒ कक्षा॑णाम् ।\\
\\
66. वा॒णि॒जाय॑ । कक्षा॑णाम् । पत॑ये ।\\
वा॒णि॒जाय॒ कक्षा॑णां॒ कक्षा॑णांँवाणि॒जाय॑ वाणि॒जाय॒ कक्षा॑णां॒ पत॑ये॒ पत॑ये॒\\
कक्षा॑णांँवाणि॒जाय॑ वाणि॒जाय॒ कक्षा॑णां॒ पत॑ये ।\\
\\
67. कक्षा॑णाम् । पत॑ये । नमः॑ ।\\
कक्षा॑णां॒ पत॑ये॒ पत॑ये॒ कक्षा॑णां॒ कक्षा॑णां॒ पत॑ये॒ नमो॒ नम॒स्पत॑ये॒ कक्षा॑णां॒\\
कक्षा॑णां॒ पत॑ये॒ नमः॑ ।\\
\\
68. पत॑ये । नमः॑ ।\\
पत॑ये॒ नमो॒ नम॒स्पत॑ये॒ पत॑ये॒ नमो॒ नमः॑ ।\\
\\
69. नमः॑ । नमः॑ ।\\
नमो॒ नमः॑ ।\\
\\
70. नमः॑ । भु॒व॒न्तये᳚ । वा॒रि॒व॒स्कृ॒ताय॑ ।\\
नमो॑ भुव॒न्तये॑ भुव॒न्तये॒ नमो॒ नमो॑ भुव॒न्तये॑ वारिवस्कृ॒ताय॑\\
वारिवस्कृ॒ताय॑ भुव॒न्तये॒ नमो॒ नमो॑ भुव॒न्तये॑ वारिवस्कृ॒ताय॑ ।\\
\\
71. भु॒व॒न्तये᳚ । वा॒रि॒व॒स्कृ॒ताय॑ । ओष॑धीनाम् ।\\
भु॒व॒न्तये॑ वारिवस्कृ॒ताय॑ वारिवस्कृ॒ताय॑ भुव॒न्तये॑ भुव॒न्तये॑ वारिवस्कृ॒ता\\
यौष॑धीना॒ मोष॑धीनांँवारिवस्कृ॒ताय॑ भुव॒न्तये॑ भुव॒न्तये॑ वारिवस्कृ॒ता\\
यौष॑धीनाम् ।\\
\\
72. वा॒रि॒व॒स्कृ॒ताय॑ । ओष॑धीनाम् । पत॑ये ।\\
वा॒रि॒व॒स्कृ॒ता यौष॑धीना॒ मोष॑धीनांँवारिवस्कृ॒ताय॑ वारिवस्कृ॒ता यौष॑धीनां॒\\
पत॑ये॒ पत॑य॒ ओष॑धीनांँवारिवस्कृ॒ताय॑ वारिवस्कृ॒ता यौष॑धीनां॒ पत॑ये ।\\
\\
73. वा॒रि॒व॒स्कृ॒ताय॑ ।\\
वा॒रि॒व॒स्कृ॒ता येति॑ वारिवः - कृ॒ताय॑ ।\\
\\
74. ओष॑धीनाम् । पत॑ये । नमः॑ ।\\
ओष॑धीनां॒ पत॑ये॒ पत॑य॒ ओष॑धीना॒ मोष॑धीनां॒ पत॑ये॒ नमो॒ नम॒स्पत॑य॒\\
ओष॑धीना॒ मोष॑धीनां॒ पत॑ये॒ नमः॑ ।\\
\\
75. पत॑ये । नमः॑ ।\\
पत॑ये॒ नमो॒ नम॒स्पत॑ये॒ पत॑ये॒ नमो॒ नमः॑ ।\\
\\
76. नमः॑ । नमः॑ ।\\
नमो॒ नमः॑ ।\\
\\
77. नमः॑ । उ॒च्चैर्घो॑षाय । आ॒क्र॒न्दय॑ते ।\\
नम॑ उ॒च्चैर्घो॑षा यो॒च्चैर्घो॑षाय॒ नमो॒ नम॑ उ॒च्चैर्घो॑षाया क्र॒न्दय॑त\\
आक्र॒न्दय॑त उ॒च्चैर्घो॑षाय॒ नमो॒ नम॑ उ॒च्चैर्घो॑षाया क्र॒न्दय॑ते ।\\
\\
78. उ॒च्चैर्घो॑षाय । आ॒क्र॒न्दय॑ते । प॒त्ती॒नाम् ।\\
उ॒च्चैर्घो॑षाया क्र॒न्दय॑त आक्र॒न्दय॑त उ॒च्चैर्घो॑षा यो॒च्चैर्घो॑षाया क्र॒न्दय॑ते\\
पत्ती॒नां प॑त्ती॒ना मा᳚क्र॒न्दय॑त उ॒च्चैर्घो॑षा यो॒च्चैर्घो॑षाया क्र॒न्दय॑ते पत्ती॒नाम् ।\\
\\
79. उ॒च्चैर्घो॑षाय ।\\
उ॒च्चैर्घो॑षा॒येत्यु॒च्चैः - घो॒षा॒य॒ ।\\
\\
80. आ॒क्र॒न्दय॑ते । प॒त्ती॒नाम् । पत॑ये ।\\
आ॒क्र॒न्दय॑ते पत्ती॒नां प॑त्ती॒ना मा᳚क्र॒न्दय॑त आक्र॒न्दय॑ते पत्ती॒नां पत॑ये॒ पत॑ये\\
पत्ती॒ना मा᳚क्र॒न्दय॑त आक्र॒न्दय॑ते पत्ती॒नां पत॑ये ।\\
\\
81. आ॒क्र॒न्दय॑ते ।\\
आ॒क्र॒न्दय॑त॒ इत्या᳚ - क्र॒न्दय॑ते ।\\
\\
82. प॒त्ती॒नाम् । पत॑ये । नमः॑ ।\\
प॒त्ती॒नां पत॑ये॒ पत॑ये पत्ती॒नां प॑त्ती॒नां पत॑ये॒ नमो॒ नम॒स्पत॑ये पत्ती॒नां प॑त्ती॒नां\\
पत॑ये॒ नमः॑ ।\\
\\
83. पत॑ये । नमः॑ ।\\
पत॑ये॒ नमो॒ नम॒स्पत॑ये॒ पत॑ये॒ नमो॒ नमः॑ ।\\
\\
84. नमः॑ । नमः॑ ।\\
नमो॒ नमः॑ ।\\
\\
85. नमः॑ । कृ॒थ्स्न॒वी॒ताय॑ । धाव॑ते ।\\
नमः॑ कृथ्स्नवी॒ताय॑ कृथ्स्नवी॒ताय॒ नमो॒ नमः॑ कृथ्स्नवी॒ताय॒ धाव॑ते॒ धाव॑ते\\
कृथ्स्नवी॒ताय॒ नमो॒ नमः॑ कृथ्स्नवी॒ताय॒ धाव॑ते ।\\
\\
86. कृ॒थ्स्न॒वी॒ताय॑ । धाव॑ते । सत्त्व॑नाम् ।\\
कृ॒थ्स्न॒वी॒ताय॒ धाव॑ते॒ धाव॑ते कृथ्स्नवी॒ताय॑ कृथ्स्नवी॒ताय॒ धाव॑ते॒ सत्त्व॑ना॒ꣳ॒\\
सत्त्व॑नां॒ धाव॑ते कृथ्स्नवी॒ताय॑ कृथ्स्नवी॒ताय॒ धाव॑ते॒ सत्त्व॑नाम् ।\\
\\
87. कृ॒थ्स्न॒वी॒ताय॑ ।\\
कृ॒थ्स्न॒वी॒तायेति॑ कृथ्स्न - वी॒ताय॑ ।\\
\\
88. धाव॑ते । सत्व॑नाम् । पत॑ये ॥\\
धाव॑ते॒ सत्व॑ना॒ꣳ॒ सत्व॑नां॒ धाव॑ते॒ धाव॑ते॒ सत्व॑नां॒ पत॑ये॒ पत॑ये॒ सत्व॑नां॒\\
धाव॑ते॒ धाव॑ते॒ सत्व॑नां॒ पत॑ये ।\\
\\
89. सत्व॑नाम् । पत॑ये । नमः॑ ।\\
सत्व॑नां॒ पत॑ये॒ पत॑ये॒ सत्व॑ना॒ꣳ॒ सत्व॑नां॒ पत॑ये॒ नमो॒ नम॒स्पत॑ये॒ सत्व॑ना॒ꣳ॒\\
सत्व॑नां॒ पत॑ये॒ नमः॑ ।\\
\\
90. पत॑ये । नमः॑ ॥\\
पत॑ये॒ नमो॒ नम॒स्पत॑ये॒ पत॑ये॒ नमः॑ ।\\
\\
91. नमः॑ ॥\\
नम॒ इति॒ नमः॑ ।\\
\subsection{\eng{Anuvaka 3}}
1. नमः॑ । सह॑मानाय । नि॒व्या॒धिने᳚ ।\\
नमः॒ सह॑मानाय॒ सह॑मानाय॒ नमो॒ नमः॒ सह॑मानाय निव्या॒धिने॑\\
निव्या॒धिने॒ सह॑मानाय॒ नमो॒ नमः॒ सह॑मानाय निव्या॒धिने᳚ ।\\
\\
2. सह॑मानाय । नि॒व्या॒धिने᳚ । आ॒व्या॒धिनी॑नाम् ।\\
सह॑मानाय निव्या॒धिने॑ निव्या॒धिने॒ सह॑मानाय॒ सह॑मानाय निव्या॒धिन॑\\
आव्या॒धिनी॑ना माव्या॒धिनी॑नां निव्या॒धिने॒ सह॑मानाय॒ सह॑मानाय\\
निव्या॒धिन॑ आव्या॒धिनी॑नाम् ।\\
\\
3. नि॒व्या॒धिने᳚ । आ॒व्या॒धिनी॑नाम् । पत॑ये ।\\
नि॒व्या॒धिन॑ आव्या॒धिनी॑ना माव्या॒धिनी॑नां निव्या॒धिने॑ निव्या॒धिन॑\\
आव्या॒धिनी॑नां॒ पत॑ये॒ पत॑य आव्या॒धिनी॑नां निव्या॒धिने॑ निव्या॒धिन॑\\
आव्या॒धिनी॑नां॒ पत॑ये ।\\
\\
4. नि॒व्या॒धिने᳚ ।\\
नि॒व्या॒धिन॒ इति॑ नि - व्या॒धिने᳚ ।\\
\\
5. आ॒व्या॒धिनी॑नाम् । पत॑ये । नमः॑ ।\\
आ॒व्या॒धिनी॑नां॒ पत॑ये॒ पत॑य आव्या॒धिनी॑ना माव्या॒धिनी॑नां॒ पत॑ये॒ नमो॒\\
नम॒स्पत॑य आव्या॒धिनी॑ना माव्या॒धिनी॑नां॒ पत॑ये॒ नमः॑ ।\\
\\
6. आ॒व्या॒धिनी॑नाम् ।\\
आ॒व्या॒धिनी॑ना॒ मित्या᳚ - व्या॒धिनी॑नाम् ।\\
\\
7. पत॑ये । नमः॑ ।\\
पत॑ये॒ नमो॒ नम॒स्पत॑ये॒ पत॑ये॒ नमो॒ नमः॑ ।\\
\\
8. नमः॑ । नमः॑ ।\\
नमो॒ नमः॑ ।\\
\\
9. नमः॑ । क॒कु॒भाय॑ । नि॒ष॒ङ्गिणे᳚ ।\\
नमः॑ ककु॒भाय॑ ककु॒भाय॒ नमो॒ नमः॑ ककु॒भाय॑ निष॒ङ्गिणे॑ निष॒ङ्गिणे॑\\
ककु॒भाय॒ नमो॒ नमः॑ ककु॒भाय॑ निष॒ङ्गिणे᳚ ।\\
\\
10. क॒कु॒भाय॑ । नि॒ष॒ङ्गिणे᳚ । स्ते॒नाना᳚म् ।\\
क॒कु॒भाय॑ निष॒ङ्गिणे॑ निष॒ङ्गिणे॑ ककु॒भाय॑ ककु॒भाय॑ निष॒ङ्गिणे᳚\\
स्ते॒नानाꣲ॑ स्ते॒नानां᳚ निष॒ङ्गिणे॑ ककु॒भाय॑ ककु॒भाय॑ निष॒ङ्गिणे᳚ स्ते॒नाना᳚म् ।\\
\\
11. नि॒ष॒ङ्गिणे᳚ । स्ते॒नाना᳚म् । पत॑ये ।\\
नि॒ष॒ङ्गिणे᳚ स्ते॒नानाꣲ॑ स्ते॒नानां᳚ निष॒ङ्गिणे॑ निष॒ङ्गिणे᳚ स्ते॒नानां॒ पत॑ये॒ पत॑ये\\
स्ते॒नानां᳚ निष॒ङ्गिणे॑ निष॒ङ्गिणे᳚ स्ते॒नानां॒ पत॑ये ।\\
\\
12. नि॒ष॒ङ्गिणे᳚ ।\\
नि॒ष॒ङ्गिण॒ इति॑ नि - स॒ङ्गिने᳚ ।\\
\\
13. स्ते॒नाना᳚म् । पत॑ये । नमः॑ ।\\
स्ते॒नानां॒ पत॑ये॒ पत॑ये स्ते॒नानाꣲ॑ स्ते॒नानां॒ पत॑ये॒ नमो॒ नम॒स्पत॑ये\\
स्ते॒नानाꣲ॑ स्ते॒नानां॒ पत॑ये॒ नमः॑ ।\\
\\
14. पत॑ये । नमः॑ ।\\
पत॑ये॒ नमो॒ नम॒स्पत॑ये॒ पत॑ये॒ नमो॒ नमः॑ ।\\
\\
15. नमः॑ । नमः॑ ।\\
नमो॒ नमः॑ ।\\
\\
16. नमः॑ । नि॒ष॒ङ्गिणे᳚ । इ॒षु॒धि॒मते᳚ ।\\
नमो॑ निष॒ङ्गिणे॑ निष॒ङ्गिणे॒ नमो॒ नमो॑ निष॒ङ्गिण॑ इषुधि॒मत॑ इषुधि॒मते॑\\
निष॒ङ्गिणे॒ नमो॒ नमो॑ निष॒ङ्गिण॑ इषुधि॒मते᳚ ।\\
\\
17. नि॒ष॒ङ्गिणे᳚ । इ॒षु॒धि॒मते᳚ । तस्क॑राणाम् ।\\
नि॒ष॒ङ्गिण॑ इषुधि॒मत॑ इषुधि॒मते॑ निष॒ङ्गिणे॑ निष॒ङ्गिण॑ इषुधि॒मते॒ तस्क॑राणां॒\\
तस्क॑राणा मिषुधि॒मते॑ निष॒ङ्गिणे॑ निष॒ङ्गिण॑ इषुधि॒मते॒ तस्क॑राणाम् ।\\
\\
18. नि॒ष॒ङ्गिणे᳚ ।\\
नि॒ष॒ङ्गिण॒ इति॑ नि - स॒ङ्गिने᳚ ।\\
\\
19. इ॒षु॒धि॒मते᳚ । तस्क॑राणाम् । पत॑ये ।\\
इ॒षु॒धि॒मते॒ तस्क॑राणां॒ तस्क॑राणा मिषुधि॒मत॑ इषुधि॒मते॒ तस्क॑राणां॒ पत॑ये॒\\
पत॑ये॒ तस्क॑राणा मिषुधि॒मत॑ इषुधि॒मते॒ तस्क॑राणां॒ पत॑ये ।\\
\\
20. इ॒षु॒धि॒मते᳚ ।\\
इ॒षु॒धि॒मत॒ इती॑षुधि - मते᳚ ।\\
\\
21. तस्क॑राणाम् । पत॑ये । नमः॑ ।\\
तस्क॑राणां॒ पत॑ये॒ पत॑ये॒ तस्क॑राणां॒ तस्क॑राणां॒ पत॑ये॒ नमो॒ नम॒स्पत॑ये॒\\
तस्क॑राणां॒ तस्क॑राणां॒ पत॑ये॒ नमः॑ ।\\
\\
22. पत॑ये । नमः॑ ।\\
पत॑ये॒ नमो॒ नम॒स्पत॑ये॒ पत॑ये॒ नमो॒ नमः॑ ।\\
\\
23. नमः॑ । नमः॑ ।\\
नमो॒ नमः॑ ।\\
\\
24. नमः॑ । वञ्च॑ते । प॒रि॒वञ्च॑ते ।\\
नमो॒ वञ्च॑ते॒ वञ्च॑ते॒ नमो॒ नमो॒ वञ्च॑ते परि॒वञ्च॑ते परि॒वञ्च॑ते॒ वञ्च॑ते॒\\
नमो॒ नमो॒ वञ्च॑ते परि॒वञ्च॑ते ।\\
\\
25. वञ्च॑ते । प॒रि॒वञ्च॑ते । स्ता॒यू॒नाम् ।\\
वञ्च॑ते परि॒वञ्च॑ते परि॒वञ्च॑ते॒ वञ्च॑ते॒ वञ्च॑ते परि॒वञ्च॑ते स्तायू॒नाꣲ\\
स्ता॑यू॒नां प॑रि॒वञ्च॑ते॒ वञ्च॑ते॒ वञ्च॑ते परि॒वञ्च॑ते स्तायू॒नाम् ।\\
\\
26. प॒रि॒वञ्च॑ते । स्ता॒यू॒नाम् । पत॑ये ।\\
प॒रि॒वञ्च॑ते स्तायू॒नाꣲ स्ता॑यू॒नां प॑रि॒वञ्च॑ते परि॒वञ्च॑ते स्तायू॒नां पत॑ये॒\\
पत॑ये स्तायू॒नां प॑रि॒वञ्च॑ते परि॒वञ्च॑ते स्तायू॒नां पत॑ये ।\\
\\
27. प॒रि॒वञ्च॑ते ।\\
प॒रि॒वञ्च॑त॒ इति॑ परि - वञ्च॑ते ।\\
\\
28. स्ता॒यू॒नाम् । पत॑ये । नमः॑ ।\\
स्ता॒यू॒नां पत॑ये॒ पत॑ये स्तायू॒नाꣲ स्ता॑यू॒नां पत॑ये॒ नमो॒ नम॒स्पत॑ये\\
स्तायू॒नाꣲ स्ता॑यू॒नां पत॑ये॒ नमः॑ ।\\
\\
29. पत॑ये । नमः॑ ।\\
पत॑ये॒ नमो॒ नम॒स्पत॑ये॒ पत॑ये॒ नमो॒ नमः॑ ।\\
\\
30. नमः॑ । नमः॑ ।\\
नमो॒ नमः॑ ।\\
\\
31. नमः॑ । नि॒चे॒रवे᳚ । प॒रि॒च॒राय॑ ।\\
नमो॑ निचे॒रवे॑ निचे॒रवे॒ नमो॒ नमो॑ निचे॒रवे॑ परिच॒राय॑ परिच॒राय॑ निचे॒रवे॒\\
नमो॒ नमो॑ निचे॒रवे॑ परिच॒राय॑ ।\\
\\
32. नि॒चे॒रवे᳚ । प॒रि॒च॒राय॑ । अर॑ण्यानाम् ।\\
नि॒चे॒रवे॑ परिच॒राय॑ परिच॒राय॑ निचे॒रवे॑ निचे॒रवे॑ परिच॒राया र॑ण्याना॒ मर॑ण्यानां\\
परिच॒राय॑ निचे॒रवे॑ निचे॒रवे॑ परिच॒राया र॑ण्यानाम् ।\\
\\
33. नि॒चे॒रवे᳚ ।\\
नि॒चे॒रव॒ इति॑ नि - चे॒रवे᳚ ।\\
\\
34. प॒रि॒च॒राय॑ । अर॑ण्यानाम् । पत॑ये ।\\
प॒रि॒च॒राया र॑ण्याना॒ मर॑ण्यानां परिच॒राय॑ परिच॒राया र॑ण्यानां॒ पत॑ये॒ पत॒येऽर॑ण्यानाम् परिच॒राय॑ परिच॒राया र॑ण्यानां॒ पत॑ये ।\\
\\
35. प॒रि॒च॒राय॑ ।\\
प॒रि॒च॒रायेति॑ परि - च॒राय॑ ।\\
\\
36. अर॑ण्यानाम् । पत॑ये । नमः॑ ।\\
अर॑ण्यानां॒ पत॑ये॒ पत॒येऽर॑ण्याना॒ मर॑ण्यानां॒ पत॑ये॒ नमो॒ नम॒स्पत॒येऽर॑ण्याना॒\\
मर॑ण्यानां॒ पत॑ये॒ नमः॑ ।\\
\\
37. पत॑ये । नमः॑ ।\\
पत॑ये॒ नमो॒ नम॒स्पत॑ये॒ पत॑ये॒ नमो॒ नमः॑ ।\\
\\
38. नमः॑ । नमः॑ ।\\
नमो॒ नमः॑ ।\\
\\
39. नमः॑ । सृ॒का॒विभ्यः॑ । जिघाꣳ॑सद्भ्यः ।\\
नमः॑ सृका॒विभ्यः॑ सृका॒विभ्यो॒ नमो॒ नमः॑ सृका॒विभ्यो॒ जिघाꣳ॑सद्भ्यो॒\\
जिघाꣳ॑सद्भ्यः सृका॒विभ्यो॒ नमो॒ नमः॑ सृका॒विभ्यो॒ जिघाꣳ॑सद्भ्यः ।\\
\\
40. सृ॒का॒विभ्यः॑ । जिघाꣳ॑सद्भ्यः । मु॒ष्ण॒ताम् ।\\
सृ॒का॒विभ्यो॒ जिघाꣳ॑सद्भ्यो॒ जिघाꣳ॑सद्भ्यः सृका॒विभ्यः॑ सृका॒विभ्यो॒\\
जिघाꣳ॑सद्भ्यो मुष्ण॒तां मु॑ष्ण॒तां जिघाꣳ॑सद्भ्यः सृका॒विभ्यः॑\\
सृका॒विभ्यो॒ जिघाꣳ॑सद्भ्यो मुष्ण॒ताम् ।\\
\\
41. सृ॒का॒विभ्यः॑ ।\\
सृ॒का॒विभ्य॒ इति॑ सृका॒वि - भ्यः॒ ।\\
\\
42. जिघाꣳ॑सद्भ्यः । मु॒ष्ण॒ताम् । पत॑ये ।\\
जिघाꣳ॑सद्भ्यो मुष्ण॒तां मु॑ष्ण॒तां जिघाꣳ॑सद्भ्यो॒ जिघाꣳ॑सद्भ्यो मुष्ण॒तां\\
पत॑ये॒ पत॑ये मुष्ण॒तां जिघाꣳ॑सद्भ्यो॒ जिघाꣳ॑सद्भ्यो मुष्ण॒तां पत॑ये ।\\
\\
43. जिघाꣳ॑सद्भ्यः ।\\
जिघाꣳ॑सद्भ्य॒ इति॒ जिघाꣳ॑सत् - भ्यः॒ ।\\
\\
44. मु॒ष्ण॒ताम् । पत॑ये । नमः॑ ।\\
मु॒ष्ण॒तां पत॑ये॒ पत॑ये मुष्ण॒तां मु॑ष्ण॒तां पत॑ये॒ नमो॒ नम॒स्पत॑ये मुष्ण॒तां\\
मु॑ष्ण॒तां पत॑ये॒ नमः॑ ।\\
\\
45. पत॑ये । नमः॑ ।\\
पत॑ये॒ नमो॒ नम॒स्पत॑ये॒ पत॑ये॒ नमो॒ नमः॑ ।\\
\\
46. नमः॑ । नमः॑ ।\\
नमो॒ नमः॑ ।\\
\\
47. नमः॑ । अ॒सि॒मद्भ्यः॑ । नक्त᳚म् ।\\
नमो॑ऽसि॒मद्भ्यो॑ऽसि॒मद्भ्यो॒ नमो॒ नमो॑ऽसि॒मद्भ्यो॒ नक्तं॒ नक्त॑\\
मसि॒मद्भ्यो॒ नमो॒ नमो॑ऽसि॒मद्भ्यो॒ नक्त᳚म् ।\\
\\
48. अ॒सि॒मद्भ्यः॑ । नक्त᳚म् । चर॑द्भ्यः ।\\
अ॒सि॒मद्भ्यो॒ नक्तं॒ नक्त॑ मसि॒मद्भ्यो॑ऽसि॒मद्भ्यो॒ नक्त॒म् चर॑द्भ्य॒ श्चर॑द्भ्यो॒\\
नक्त॑ मसि॒मद्भ्यो॑ऽसि॒मद्भ्यो॒ नक्त॒म् चर॑द्भ्यः ।\\
\\
49. अ॒सि॒मद्भ्यः॑ ।\\
अ॒सि॒मद्भ्य॒ इत्य॑सि॒मत् - भ्यः॒ ।\\
\\
50. नक्त᳚म् । चर॑द्भ्यः । प्र॒कृ॒न्ताना᳚म् ।\\
नक्त॒म् चर॑द्भ्य॒ श्चर॑द्भ्यो॒ नक्तं॒ नक्त॒म् चर॑द्भ्यः प्रकृ॒न्तानां᳚\\
प्रकृ॒न्ताना॒म् चर॑द्भ्यो॒ नक्तं॒ नक्त॒म् चर॑द्भ्यः प्रकृ॒न्ताना᳚म् ।\\
\\
51. चर॑द्भ्यः । प्र॒कृ॒न्ताना᳚म् । पत॑ये ।\\
चर॑द्भ्यः प्रकृ॒न्तानां᳚ प्रकृ॒न्ताना॒म् चर॑द्भ्य॒ श्चर॑द्भ्यः प्रकृ॒न्तानां॒ पत॑ये॒\\
पत॑ये प्रकृ॒न्ताना॒म् चर॑द्भ्य॒ श्चर॑द्भ्यः प्रकृ॒न्तानां॒ पत॑ये ।\\
\\
52. चर॑द्भ्यः ।\\
चर॑द्भ्य॒ इति॒ चर॑त् - भ्यः॒ ।\\
\\
53. प्र॒कृ॒न्ताना᳚म् । पत॑ये । नमः॑ ।\\
प्र॒कृ॒न्तानां॒ पत॑ये॒ पत॑ये प्रकृ॒न्तानां᳚ प्रकृ॒न्तानां॒ पत॑ये॒ नमो॒ नम॒स्पत॑ये\\
प्रकृ॒न्ताना᳚म् प्रकृ॒न्तानां॒ पत॑ये॒ नमः॑ ।\\
\\
54. प्र॒कृ॒न्ताना᳚म् ।\\
प्र॒कृ॒न्ताना॒ मिति॑ प्र - कृ॒न्ताना᳚म् ।\\
\\
55. पत॑ये । नमः॑ ।\\
पत॑ये॒ नमो॒ नम॒स्पत॑ये॒ पत॑ये॒ नमो॒ नमः॑ ।\\
\\
56. नमः॑ । नमः॑ ।\\
नमो॒ नमः॑ ।\\
\\
57. नमः॑ । उ॒ष्णी॒षिणे᳚ । गि॒रि॒च॒राय॑ ।\\
नम॑ उष्णी॒षिण॑ उष्णी॒षिणे॒ नमो॒ नम॑ उष्णी॒षिणे॑ गिरिच॒राय॑\\
गिरिच॒रायो᳚ ष्णी॒षिणे॒ नमो॒ नम॑ उष्णी॒षिणे॑ गिरिच॒राय॑ ।\\
\\
58. उ॒ष्णी॒षिणे᳚ । गि॒रि॒च॒राय॑ । कु॒लु॒ञ्चाना᳚म् ।\\
उ॒ष्णी॒षिणे॑ गिरिच॒राय॑ गिरिच॒रायो᳚ ष्णी॒षिण॑ उष्णी॒षिणे॑ गिरिच॒राय॑\\
कुलु॒ञ्चानां᳚ कुलु॒ञ्चानां᳚ गिरिच॒रायो᳚ ष्णी॒षिण॑ उष्णी॒षिणे॑ गिरिच॒राय॑\\
कुलु॒ञ्चाना᳚म् ।\\
\\
59. गि॒रि॒च॒राय॑ । कु॒लु॒ञ्चाना᳚म् । पत॑ये ।\\
गि॒रि॒च॒राय॑ कुलु॒ञ्चानां᳚ कुलु॒ञ्चानां᳚ गिरिच॒राय॑ गिरिच॒राय॑ कुलु॒ञ्चानां॒\\
पत॑ये॒ पत॑ये कुलु॒ञ्चानां᳚ गिरिच॒राय॑ गिरिच॒राय॑ कुलु॒ञ्चानां॒ पत॑ये ।\\
\\
60. गि॒रि॒च॒राय॑ ।\\
गि॒रि॒च॒रायेति॑ गिरि - च॒राय॑ ।\\
\\
61. कु॒लु॒ञ्चाना᳚म् । पत॑ये । नमः॑ ।\\
कु॒लु॒ञ्चानां॒ पत॑ये॒ पत॑ये कुलु॒ञ्चानां᳚ कुलु॒ञ्चानां॒ पत॑ये॒ नमो॒\\
नम॒स्पत॑ये कुलु॒ञ्चानां᳚ कुलु॒ञ्चानां॒ पत॑ये॒ नमः॑ ।\\
\\
62. पत॑ये । नमः॑ ।\\
पत॑ये॒ नमो॒ नम॒स्पत॑ये॒ पत॑ये॒ नमो॒ नमः॑ ।\\
\\
63. नमः॑ । नमः॑ ।\\
नमो॒ नमः॑ ।\\
\\
64. नमः॑ । इषु॑मद्भ्यः । ध॒न्वा॒विभ्यः॑ ।\\
नम॒ इषु॑मद्भ्य॒ इषु॑मद्भ्यो॒ नमो॒ नम॒ इषु॑मद्भ्यो धन्वा॒विभ्यो॑\\
धन्वा॒विभ्य॒ इषु॑मद्भ्यो॒ नमो॒ नम॒ इषु॑मद्भ्यो धन्वा॒विभ्यः॑ ।\\
\\
65. इषु॑मद्भ्यः । ध॒न्वा॒विभ्यः॑ । च॒ ।\\
इषु॑मद्भ्यो धन्वा॒विभ्यो॑ धन्वा॒विभ्य॒ इषु॑मद्भ्य॒ इषु॑मद्भ्यो\\
धन्वा॒विभ्य॑श्च च धन्वा॒विभ्य॒ इषु॑मद्भ्य॒ इषु॑मद्भ्यो धन्वा॒विभ्य॑श्च ।\\
\\
66. इषु॑मद्भ्यः ।\\
इषु॑मद्भ्य॒ इतीषु॑मत् - भ्यः॒ ।\\
\\
67. ध॒न्वा॒विभ्यः॑ । च॒ । वः॒ ।\\
ध॒न्वा॒विभ्य॑श्च च धन्वा॒विभ्यो॑ धन्वा॒विभ्य॑श्च वो वश्च धन्वा॒विभ्यो॑\\
धन्वा॒विभ्य॑श्च वः ।\\
\\
68. ध॒न्वा॒विभ्यः॑ ।\\
ध॒न्वा॒विभ्य॒ इति॑ धन्वा॒वि - भ्यः॒ ।\\
\\
69. च॒ । वः॒ । नमः॑ ।\\
च॒ वो॒ व॒श्च॒ च॒ वो॒ नमो॒ नमो॑ वश्च च वो॒ नमः॑ ।\\
\\
70. वः॒ । नमः॑ ।\\
वो॒ नमो॒ नमो॑ वो वो॒ नमो॒ नमः॑ ।\\
\\
71. नमः॑ । नमः॑ ।\\
नमो॒ नमः॑ ।\\
\\
72. नमः॑ । आ॒त॒न्वा॒नेभ्यः॑ । प्र॒ति॒दधा॑नेभ्यः ।\\
नम॑ आतन्वा॒नेभ्य॑ आतन्वा॒नेभ्यो॒ नमो॒ नम॑ आतन्वा॒नेभ्यः॑ प्रति॒दधा॑नेभ्यः\\
प्रति॒दधा॑नेभ्य आतन्वा॒नेभ्यो॒ नमो॒ नम॑ आतन्वा॒नेभ्यः॑ प्रति॒दधा॑नेभ्यः ।\\
\\
73. आ॒त॒न्वा॒नेभ्यः॑ । प्र॒ति॒दधा॑नेभ्यः । च॒ ।\\
आ॒त॒न्वा॒नेभ्यः॑ प्रति॒दधा॑नेभ्यः प्रति॒दधा॑नेभ्य आतन्वा॒नेभ्य॑ आतन्वा॒नेभ्यः॑\\
प्रति॒दधा॑नेभ्यश्च च प्रति॒दधा॑नेभ्य आतन्वा॒नेभ्य॑ आतन्वा॒नेभ्यः॑\\
प्रति॒दधा॑नेभ्यश्च ।\\
\\
74. आ॒त॒न्वा॒नेभ्यः॑ ।\\
आ॒त॒न्वा॒नेभ्य॒ इत्या᳚ - त॒न्वा॒नेभ्यः॑ ।\\
\\
75. प्र॒ति॒दधा॑नेभ्यः । च॒ । वः॒ ।\\
प्र॒ति॒दधा॑नेभ्यश्च च प्रति॒दधा॑नेभ्यः प्रति॒दधा॑नेभ्यश्च वो वश्च\\
प्रति॒दधा॑नेभ्यः प्रति॒दधा॑नेभ्यश्च वः ।\\
\\
76. प्र॒ति॒दधा॑नेभ्यः ।\\
प्र॒ति॒दधा॑नेभ्य॒ इति॑ प्रति - दधा॑नेभ्यः ।\\
\\
77. च॒ । वः॒ । नमः॑ ।\\
च॒ वो॒ व॒श्च॒ च॒ वो॒ नमो॒ नमो॑ वश्च च वो॒ नमः॑ ।\\
\\
78. वः॒ । नमः॑ ।\\
वो॒ नमो॒ नमो॑ वो वो॒ नमो॒ नमः॑ ।\\
\\
79. नमः॑ । नमः॑ ।\\
नमो॒ नमः॑ ।\\
\\
80. नमः॑ । आ॒यच्छ॑द्भ्यः । वि॒सृ॒जद्भ्यः॑ ।\\
नम॑ आ॒यच्छ॑द्भ्य आ॒यच्छ॑द्भ्यो॒ नमो॒ नम॑ आ॒यच्छ॑द्भ्यो विसृ॒जद्भ्यो॑\\
विसृ॒जद्भ्य॑ आ॒यच्छ॑द्भ्यो॒ नमो॒ नम॑ आ॒यच्छ॑द्भ्यो विसृ॒जद्भ्यः॑ ।\\
\\
81. आ॒यच्छ॑द्भ्यः । वि॒सृ॒जद्भ्यः॑ । च॒ ।\\
आ॒यच्छ॑द्भ्यो विसृ॒जद्भ्यो॑ विसृ॒जद्भ्य॑ आ॒यच्छ॑द्भ्य आ॒यच्छ॑द्भ्यो\\
विसृ॒जद्भ्य॑श्च च विसृ॒जद्भ्य॑ आ॒यच्छ॑द्भ्य आ॒यच्छ॑द्भ्यो विसृ॒जद्भ्य॑श्च ।\\
\\
82. आ॒यच्छ॑द्भ्यः ।\\
आ॒यच्छ॑द्भ्य॒ इत्या॒यच्छ॑त् - भ्यः॒ ।\\
\\
83. वि॒सृ॒जद्भ्यः॑ । च॒ । वः॒ ।\\
वि॒सृ॒जद्भ्य॑श्च च विसृ॒जद्भ्यो॑ विसृ॒जद्भ्य॑श्च वो वश्च विसृ॒जद्भ्यो॑\\
विसृ॒जद्भ्य॑श्च वः ।\\
\\
84. वि॒सृ॒जद्भ्यः॑ ।\\
वि॒सृ॒जद्भ्य॒ इति॑ विसृ॒जत् - भ्यः॒ ।\\
\\
85. च॒ । वः॒ । नमः॑ ।\\
च॒ वो॒ व॒श्च॒ च॒ वो॒ नमो॒ नमो॑ वश्च च वो॒ नमः॑ ।\\
\\
86. वः॒ । नमः॑ ।\\
वो॒ नमो॒ नमो॑ वो वो॒ नमो॒ नमः॑ ।\\
\\
87. नमः॑ । नमः॑ ।\\
नमो॒ नमः॑ ।\\
\\
88. नमः॑ । अस्य॑द्भ्यः । विद्ध्य॑द्भ्यः ।\\
नमोऽस्य॒द्भ्योऽस्य॑द्भ्यो॒ नमो॒ नमोऽस्य॑द्भ्यो॒ विद्ध्य॑द्भ्यो॒\\
विद्ध्य॒द्भ्योऽस्य॑द्भ्यो॒ नमो॒ नमोऽस्य॑द्भ्यो॒ विद्ध्य॑द्भ्यः ।\\
\\
89. अस्य॑द्भ्यः । विद्ध्य॑द्भ्यः । च॒ ।\\
अस्य॑द्भ्यो॒ विद्ध्य॑द्भ्यो॒ विद्ध्य॒द्भ्योऽस्य॒द्भ्योऽस्य॑द्भ्यो॒ विद्ध्य॑द्भ्यश्च\\
च॒ विद्ध्य॒द्भ्योऽस्य॒द्भ्योऽस्य॑द्भ्यो॒ विद्ध्य॑द्भ्यश्च ।\\
\\
90. अस्य॑द्भ्यः ।\\
अस्य॑द्भ्य॒ इत्यस्य॑त् - भ्यः॒ ।\\
\\
91. विद्ध्य॑द्भ्यः । च॒ । वः॒ ।\\
विद्ध्य॑द्भ्यश्च च॒ विद्ध्य॑द्भ्यो॒ विद्ध्य॑द्भ्यश्च वो वश्च॒ विद्ध्य॑द्भ्यो॒\\
विद्ध्य॑द्भ्यश्च वः ।\\
\\
92. विद्ध्य॑द्भ्यः ।\\
विद्ध्य॑द्भ्य॒ इति॒ विद्ध्य॑त् - भ्यः॒ ।\\
\\
93. च॒ । वः॒ । नमः॑ ।\\
च॒ वो॒ व॒श्च॒ च॒ वो॒ नमो॒ नमो॑ वश्च च वो॒ नमः॑ ।\\
\\
94. वः॒ । नमः॑ ।\\
वो॒ नमो॒ नमो॑ वो वो॒ नमो॒ नमः॑ ।\\
\\
95. नमः॑ । नमः॑ ।\\
नमो॒ नमः॑ ।\\
\\
96. नमः॑ । आसी॑नेभ्यः । शया॑नेभ्यः ।\\
नम॒ आसी॑नेभ्य॒ आसी॑नेभ्यो॒ नमो॒ नम॒ आसी॑नेभ्यः॒ शया॑नेभ्यः॒\\
शया॑नेभ्य॒ आसी॑नेभ्यो॒ नमो॒ नम॒ आसी॑नेभ्यः॒ शया॑नेभ्यः ।\\
\\
97. आसी॑नेभ्यः । शया॑नेभ्यः । च॒ ।\\
आसी॑नेभ्यः॒ शया॑नेभ्यः॒ शया॑नेभ्य॒ आसी॑नेभ्य॒ आसी॑नेभ्यः॒\\
शया॑नेभ्यश्च च॒ शया॑नेभ्य॒ आसी॑नेभ्य॒ आसी॑नेभ्यः॒ शया॑नेभ्यश्च ।\\
\\
98. शया॑नेभ्यः । च॒ । वः॒ ।\\
शया॑नेभ्यश्च च॒ शया॑नेभ्यः॒ शया॑नेभ्यश्च वो वश्च॒ शया॑नेभ्यः॒\\
शया॑नेभ्यश्च वः ।\\
\\
99. च॒ । वः॒ । नमः॑ ।\\
च॒ वो॒ व॒श्च॒ च॒ वो॒ नमो॒ नमो॑ वश्च च वो॒ नमः॑ ।\\
\\
100. वः॒ । नमः॑ ।\\
वो॒ नमो॒ नमो॑ वो वो॒ नमो॒ नमः॑ ।\\
\\
101. नमः॑ । नमः॑ ।\\
नमो॒ नमः॑ ।\\
\\
102. नमः॑ । स्व॒पद्भ्यः॑ । जाग्र॑द्भ्यः ।\\
नमः॑ स्व॒पद्भ्यः॑ स्व॒पद्भ्यो॒ नमो॒ नमः॑ स्व॒पद्भ्यो॒ जाग्र॑द्भ्यो॒\\
जाग्र॑द्भ्यः स्व॒पद्भ्यो॒ नमो॒ नमः॑ स्व॒पद्भ्यो॒ जाग्र॑द्भ्यः ।\\
\\
103. स्व॒पद्भ्यः॑ । जाग्र॑द्भ्यः । च॒ ।\\
स्व॒पद्भ्यो॒ जाग्र॑द्भ्यो॒ जाग्र॑द्भ्यः स्व॒पद्भ्यः॑ स्व॒पद्भ्यो॒ जाग्र॑द्भ्यश्च\\
च॒ जाग्र॑द्भ्यः स्व॒पद्भ्यः॑ स्व॒पद्भ्यो॒ जाग्र॑द्भ्यश्च ।\\
\\
104. स्व॒पद्भ्यः॑ ।\\
स्व॒पद्भ्य॒ इति॑ स्व॒पत् - भ्यः॒ ।\\
\\
105. जाग्र॑द्भ्यः । च॒ । वः॒ ।\\
जाग्र॑द्भ्यश्च च॒ जाग्र॑द्भ्यो॒ जाग्र॑द्भ्यश्च वो वश्च॒ जाग्र॑द्भ्यो॒\\
जाग्र॑द्भ्यश्च वः ।\\
\\
106. जाग्र॑द्भ्यः ।\\
जाग्र॑द्भ्य॒ इति॒ जाग्र॑त् - भ्यः॒ ।\\
\\
107. च॒ । वः॒ । नमः॑ ।\\
च॒ वो॒ व॒श्च॒ च॒ वो॒ नमो॒ नमो॑ वश्च च वो॒ नमः॑ ।\\
\\
108. वः॒ । नमः॑ ।\\
वो॒ नमो॒ नमो॑ वो वो॒ नमो॒ नमः॑ ।\\
\\
109. नमः॑ । नमः॑ ।\\
नमो॒ नमः॑ ।\\
\\
110. नमः॑ । तिष्ठ॑द्भ्यः । धाव॑द्भ्यः ।\\
नम॒ स्तिष्ठ॑द्भ्य॒ स्तिष्ठ॑द्भ्यो॒ नमो॒ नम॒ स्तिष्ठ॑द्भ्यो॒ धाव॑द्भ्यो॒\\
धाव॑द्भ्य॒ स्तिष्ठ॑द्भ्यो॒ नमो॒ नम॒ स्तिष्ठ॑द्भ्यो॒ धाव॑द्भ्यः ।\\
\\
111. तिष्ठ॑द्भ्यः । धाव॑द्भ्यः । च॒ ।\\
तिष्ठ॑द्भ्यो॒ धाव॑द्भ्यो॒ धाव॑द्भ्य॒ स्तिष्ठ॑द्भ्य॒ स्तिष्ठ॑द्भ्यो॒ धाव॑द्भ्यश्च\\
च॒ धाव॑द्भ्य॒ स्तिष्ठ॑द्भ्य॒ स्तिष्ठ॑द्भ्यो॒ धाव॑द्भ्यश्च ।\\
\\
112. तिष्ठ॑द्भ्यः ।\\
तिष्ठ॑द्भ्य॒ इति॒ तिष्ठ॑त् - भ्यः॒ ।\\
\\
113. धाव॑द्भ्यः । च॒ । वः॒ ।\\
धाव॑द्भ्यश्च च॒ धाव॑द्भ्यो॒ धाव॑द्भ्यश्च वो वश्च॒ धाव॑द्भ्यो॒\\
धाव॑द्भ्यश्च वः ।\\
\\
114. धाव॑द्भ्यः ।\\
धाव॑द्भ्य॒ इति॒ धाव॑त् - भ्यः॒ ।\\
\\
115. च॒ । वः॒ । नमः॑ ।\\
च॒ वो॒ व॒श्च॒ च॒ वो॒ नमो॒ नमो॑ वश्च च वो॒ नमः॑ ।\\
\\
116. वः॒ । नमः॑ ।\\
वो॒ नमो॒ नमो॑ वो वो॒ नमो॒ नमः॑ ।\\
\\
117. नमः॑ । नमः॑ ।\\
नमो॒ नमः॑ ।\\
\\
118. नमः॑ । स॒भाभ्यः॑ । स॒भाप॑तिभ्यः ।\\
नमः॑ स॒भाभ्यः॑ स॒भाभ्यो॒ नमो॒ नमः॑ स॒भाभ्यः॑ स॒भाप॑तिभ्यः\\
स॒भाप॑तिभ्यः स॒भाभ्यो॒ नमो॒ नमः॑ स॒भाभ्यः॑ स॒भाप॑तिभ्यः ।\\
\\
119. स॒भाभ्यः॑ । स॒भाप॑तिभ्यः । च॒ ।\\
स॒भाभ्यः॑ स॒भाप॑तिभ्यः स॒भाप॑तिभ्यः स॒भाभ्यः॑ स॒भाभ्यः॑\\
स॒भाप॑तिभ्यश्च च स॒भाप॑तिभ्यः स॒भाभ्यः॑ स॒भाभ्यः॑ स॒भाप॑तिभ्यश्च ।\\
\\
120. स॒भाप॑तिभ्यः । च॒ । वः॒ ।\\
स॒भाप॑तिभ्यश्च च स॒भाप॑तिभ्यः स॒भाप॑तिभ्यश्च वो वश्च स॒भाप॑तिभ्यः\\
स॒भाप॑तिभ्यश्च वः ।\\
\\
121. स॒भाप॑तिभ्यः ।\\
स॒भाप॑तिभ्य॒ इति॑ स॒भाप॑ति - भ्यः॒ ।\\
\\
122. च॒ । वः॒ । नमः॑ ।\\
च॒ वो॒ व॒श्च॒ च॒ वो॒ नमो॒ नमो॑ वश्च च वो॒ नमः॑ ।\\
\\
123. वः॒ । नमः॑ ।\\
वो॒ नमो॒ नमो॑ वो वो॒ नमो॒ नमः॑ ।\\
\\
124. नमः॑ । नमः॑ ।\\
नमो॒ नमः॑ ।\\
\\
125. नमः॑ । अश्वे᳚भ्यः । अश्व॑पतिभ्यः ।\\
नमो॒ अश्वे॒भ्योऽश्वे᳚भ्यो॒ नमो॒ नमो॒ अश्वे॒भ्योऽश्व॑पति॒भ्योऽश्व॑पति॒भ्योऽश्वे᳚भ्यो॒ नमो॒ नमो॒ अश्वे॒भ्योऽश्व॑पतिभ्यः ।\\
\\
126. अश्वे᳚भ्यः । अश्व॑पतिभ्यः । च॒ ।\\
अश्वे॒भ्योऽश्व॑पति॒भ्योऽश्व॑पति॒भ्योऽश्वे॒भ्योऽश्वे॒भ्योऽश्व॑पतिभ्यश्च॒ चा\\
श्व॑पति॒भ्योऽश्वे॒भ्योऽश्वे॒भ्योऽश्व॑पतिभ्यश्च ।\\
\\
127. अश्व॑पतिभ्यः । च॒ । वः॒ ।\\
अश्व॑पतिभ्यश्च॒ चा श्व॑पति॒भ्योऽश्व॑पतिभ्यश्च वो व॒श्चा श्व॑पति॒भ्योऽश्व॑पतिभ्यश्च वः ।\\
\\
128. अश्व॑पतिभ्यः ।\\
अश्व॑पतिभ्य॒ इत्यश्व॑पति - भ्यः॒ ।\\
\\
129. च॒ । वः॒ । नमः॑ ॥\\
च॒ वो॒ व॒श्च॒ च॒ वो॒ नमो॒ नमो॑ वश्च च वो॒ नमः॑ ।\\
\\
130. वः॒ । नमः॑ ॥\\
वो॒ नमो॒ नमो॑ वो वो॒ नमः॑ ।\\
\\
131. नमः॑ ॥\\
नम॒ इति॒ नमः॑ ।\\
\subsection{\eng{Anuvaka 4}}
1. नमः॑ । आ॒व्या॒धिनी᳚भ्यः । वि॒विद्ध्य॑न्तीभ्यः ।\\
नम॑ आव्या॒धिनी᳚भ्य आव्या॒धिनी᳚भ्यो॒ नमो॒ नम॑ आव्या॒धिनी᳚भ्यो\\
वि॒विद्ध्य॑न्तीभ्यो वि॒विद्ध्य॑न्तीभ्य आव्या॒धिनी᳚भ्यो॒ नमो॒ नम॑\\
आव्या॒धिनी᳚भ्यो वि॒विद्ध्य॑न्तीभ्यः ।\\
\\
2. आ॒व्या॒धिनी᳚भ्यः । वि॒विद्ध्य॑न्तीभ्यः । च॒ ।\\
आ॒व्या॒धिनी᳚भ्यो वि॒विद्ध्य॑न्तीभ्यो वि॒विद्ध्य॑न्तीभ्य आव्या॒धिनी᳚भ्य\\
आव्या॒धिनी᳚भ्यो वि॒विद्ध्य॑न्तीभ्यश्च च वि॒विद्ध्य॑न्तीभ्य आव्या॒धिनी᳚भ्य\\
आव्या॒धिनी᳚भ्यो वि॒विद्ध्य॑न्तीभ्यश्च ।\\
\\
3. आ॒व्या॒धिनी᳚भ्यः ।\\
आ॒व्या॒धिनी᳚भ्य॒ इत्या᳚ - व्या॒धिनी᳚भ्यः ।\\
\\
4. वि॒विद्ध्य॑न्तीभ्यः । च॒ । वः॒ ।\\
वि॒विद्ध्य॑न्तीभ्यश्च च वि॒विद्ध्य॑न्तीभ्यो वि॒विद्ध्य॑न्तीभ्यश्च वो वश्च\\
वि॒विद्ध्य॑न्तीभ्यो वि॒विद्ध्य॑न्तीभ्यश्च वः ।\\
\\
5. वि॒विद्ध्य॑न्तीभ्यः ।\\
वि॒विद्ध्य॑न्तीभ्य॒ इति॑ वि - विद्ध्य॑न्तीभ्यः ।\\
\\
6. च॒ । वः॒ । नमः॑ ।\\
च॒ वो॒ व॒श्च॒ च॒ वो॒ नमो॒ नमो॑ वश्च च वो॒ नमः॑ ।\\
\\
7. वः॒ । नमः॑ ।\\
वो॒ नमो॒ नमो॑ वो वो॒ नमो॒ नमः॑ ।\\
\\
8. नमः॑ । नमः॑ ।\\
नमो॒ नमः॑ ।\\
\\
9. नमः॑ । उग॑णाभ्यः । तृ॒ꣳ॒ह॒तीभ्यः॑ ।\\
नम॒ उग॑णाभ्य॒ उग॑णाभ्यो॒ नमो॒ नम॒ उग॑णाभ्य स्तृꣳह॒तीभ्य॑\\
स्तृꣳह॒तीभ्य॒ उग॑णाभ्यो॒ नमो॒ नम॒ उग॑णाभ्य स्तृꣳह॒तीभ्यः॑ ।\\
\\
10. उग॑णाभ्यः । तृ॒ꣳ॒ह॒तीभ्यः॑ । च॒ ।\\
उग॑णाभ्य स्तृꣳह॒तीभ्य॑ स्तृꣳह॒तीभ्य॒ उग॑णाभ्य॒ उग॑णाभ्य\\
स्तृꣳह॒तीभ्य॑श्च च तृꣳह॒तीभ्य॒ उग॑णाभ्य॒ उग॑णाभ्य स्तृꣳह॒तीभ्य॑श्च ।\\
\\
11. तृ॒ꣳ॒ह॒तीभ्यः॑ । च॒ । वः॒ ।\\
तृ॒ꣳ॒ह॒तीभ्य॑श्च च तृꣳह॒तीभ्य॑ स्तृꣳह॒तीभ्य॑श्च वो वश्च तृꣳह॒तीभ्य॑\\
स्तृꣳह॒तीभ्य॑श्च वः ।\\
\\
12. च॒ । वः॒ । नमः॑ ।\\
च॒ वो॒ व॒श्च॒ च॒ वो॒ नमो॒ नमो॑ वश्च च वो॒ नमः॑ ।\\
\\
13. वः॒ । नमः॑ ।\\
वो॒ नमो॒ नमो॑ वो वो॒ नमो॒ नमः॑ ।\\
\\
14. नमः॑ । नमः॑ ।\\
नमो॒ नमः॑ ।\\
\\
15. नमः॑ । गृ॒थ्सेभ्यः॑ । गृ॒थ्सप॑तिभ्यः ।\\
नमो॑ गृ॒थ्सेभ्यो॑ गृ॒थ्सेभ्यो॒ नमो॒ नमो॑ गृ॒थ्सेभ्यो॑ गृ॒थ्सप॑तिभ्यो\\
गृ॒थ्सप॑तिभ्यो गृ॒थ्सेभ्यो॒ नमो॒ नमो॑ गृ॒थ्सेभ्यो॑ गृ॒थ्सप॑तिभ्यः ।\\
\\
16. गृ॒थ्सेभ्यः॑ । गृ॒थ्सप॑तिभ्यः । च॒ ।\\
गृ॒थ्सेभ्यो॑ गृ॒थ्सप॑तिभ्यो गृ॒थ्सप॑तिभ्यो गृ॒थ्सेभ्यो॑ गृ॒थ्सेभ्यो॑\\
गृ॒थ्सप॑तिभ्यश्च च गृ॒थ्सप॑तिभ्यो गृ॒थ्सेभ्यो॑ गृ॒थ्सेभ्यो॑ गृ॒थ्सप॑तिभ्यश्च ।\\
\\
17. गृ॒थ्सप॑तिभ्यः । च॒ । वः॒ ।\\
गृ॒थ्सप॑तिभ्यश्च च गृ॒थ्सप॑तिभ्यो गृ॒थ्सप॑तिभ्यश्च वो वश्च गृ॒थ्सप॑तिभ्यो\\
गृ॒थ्सप॑तिभ्यश्च वः ।\\
\\
18. गृ॒थ्सप॑तिभ्यः ।\\
गृ॒थ्सप॑तिभ्य॒ इति॑ गृ॒थ्सप॑ति - भ्यः॒ ।\\
\\
19. च॒ । वः॒ । नमः॑ ।\\
च॒ वो॒ व॒श्च॒ च॒ वो॒ नमो॒ नमो॑ वश्च च वो॒ नमः॑ ।\\
\\
20. वः॒ । नमः॑ ।\\
वो॒ नमो॒ नमो॑ वो वो॒ नमो॒ नमः॑ ।\\
\\
21. नमः॑ । नमः॑ ।\\
नमो॒ नमः॑ ।\\
\\
22. नमः॑ । व्राते᳚भ्यः । व्रात॑पतिभ्यः ।\\
नमो॒ व्राते᳚भ्यो॒ व्राते᳚भ्यो॒ नमो॒ नमो॒ व्राते᳚भ्यो॒ व्रात॑पतिभ्यो॒\\
व्रात॑पतिभ्यो॒ व्राते᳚भ्यो॒ नमो॒ नमो॒ व्राते᳚भ्यो॒ व्रात॑पतिभ्यः ।\\
\\
23. व्राते᳚भ्यः । व्रात॑पतिभ्यः । च॒ ।\\
व्राते᳚भ्यो॒ व्रात॑पतिभ्यो॒ व्रात॑पतिभ्यो॒ व्राते᳚भ्यो॒ व्राते᳚भ्यो॒ व्रात॑पतिभ्यश्च\\
च॒ व्रात॑पतिभ्यो॒ व्राते᳚भ्यो॒ व्राते᳚भ्यो॒ व्रात॑पतिभ्यश्च ।\\
\\
24. व्रात॑पतिभ्यः । च॒ । वः॒ ।\\
व्रात॑पतिभ्यश्च च॒ व्रात॑पतिभ्यो॒ व्रात॑पतिभ्यश्च वो वश्च॒ व्रात॑पतिभ्यो॒\\
व्रात॑पतिभ्यश्च वः ।\\
\\
25. व्रात॑पतिभ्यः ।\\
व्रात॑पतिभ्य॒ इति॒ व्रात॑पति - भ्यः॒ ।\\
\\
26. च॒ । वः॒ । नमः॑ ।\\
च॒ वो॒ व॒श्च॒ च॒ वो॒ नमो॒ नमो॑ वश्च च वो॒ नमः॑ ।\\
\\
27. वः॒ । नमः॑ ।\\
वो॒ नमो॒ नमो॑ वो वो॒ नमो॒ नमः॑ ।\\
\\
28. नमः॑ । नमः॑ ।\\
नमो॒ नमः॑ ।\\
\\
29. नमः॑ । ग॒णेभ्यः॑ । ग॒णप॑तिभ्यः ।\\
नमो॑ ग॒णेभ्यो॑ ग॒णेभ्यो॒ नमो॒ नमो॑ ग॒णेभ्यो॑ ग॒णप॑तिभ्यो ग॒णप॑तिभ्यो\\
ग॒णेभ्यो॒ नमो॒ नमो॑ ग॒णेभ्यो॑ ग॒णप॑तिभ्यः ।\\
\\
30. ग॒णेभ्यः॑ । ग॒णप॑तिभ्यः । च॒ ।\\
ग॒णेभ्यो॑ ग॒णप॑तिभ्यो ग॒णप॑तिभ्यो ग॒णेभ्यो॑ ग॒णेभ्यो॑ ग॒णप॑तिभ्यश्च च\\
ग॒णप॑तिभ्यो ग॒णेभ्यो॑ ग॒णेभ्यो॑ ग॒णप॑तिभ्यश्च ।\\
\\
31. ग॒णप॑तिभ्यः । च॒ । वः॒ ।\\
ग॒णप॑तिभ्यश्च च ग॒णप॑तिभ्यो ग॒णप॑तिभ्यश्च वो वश्च ग॒णप॑तिभ्यो\\
ग॒णप॑तिभ्यश्च वः ।\\
\\
32. ग॒णप॑तिभ्यः ।\\
ग॒णप॑तिभ्य॒ इति॑ ग॒णप॑ति - भ्यः॒ ।\\
\\
33. च॒ । वः॒ । नमः॑ ।\\
च॒ वो॒ व॒श्च॒ च॒ वो॒ नमो॒ नमो॑ वश्च च वो॒ नमः॑ ।\\
\\
34. वः॒ । नमः॑ ।\\
वो॒ नमो॒ नमो॑ वो वो॒ नमो॒ नमः॑ ।\\
\\
35. नमः॑ । नमः॑ ।\\
नमो॒ नमः॑ ।\\
\\
36. नमः॑ । विरू॑पेभ्यः । वि॒श्वरू॑पेभ्यः ।\\
नमो॒ विरू॑पेभ्यो॒ विरू॑पेभ्यो॒ नमो॒ नमो॒ विरू॑पेभ्यो वि॒श्वरू॑पेभ्यो\\
वि॒श्वरू॑पेभ्यो॒ विरू॑पेभ्यो॒ नमो॒ नमो॒ विरू॑पेभ्यो वि॒श्वरू॑पेभ्यः ।\\
\\
37. विरू॑पेभ्यः । वि॒श्वरू॑पेभ्यः । च॒ ।\\
विरू॑पेभ्यो वि॒श्वरू॑पेभ्यो वि॒श्वरू॑पेभ्यो॒ विरू॑पेभ्यो॒ विरू॑पेभ्यो\\
वि॒श्वरू॑पेभ्यश्च च वि॒श्वरू॑पेभ्यो॒ विरू॑पेभ्यो॒ विरू॑पेभ्यो वि॒श्वरू॑पेभ्यश्च ।\\
\\
38. विरू॑पेभ्यः ।\\
विरू॑पेभ्य॒ इति॒ वि - रू॒पे॒भ्यः॒ ।\\
\\
39. वि॒श्वरू॑पेभ्यः । च॒ । वः॒ ।\\
वि॒श्वरू॑पेभ्यश्च च वि॒श्वरू॑पेभ्यो वि॒श्वरू॑पेभ्यश्च वो वश्च वि॒श्वरू॑पेभ्यो\\
वि॒श्वरू॑पेभ्यश्च वः ।\\
\\
40. वि॒श्वरू॑पेभ्यः ।\\
वि॒श्वरू॑पेभ्य॒ इति॑ वि॒श्व - रू॒पे॒भ्यः॒ ।\\
\\
41. च॒ । वः॒ । नमः॑ ।\\
च॒ वो॒ व॒श्च॒ च॒ वो॒ नमो॒ नमो॑ वश्च च वो॒ नमः॑ ।\\
\\
42. वः॒ । नमः॑ ।\\
वो॒ नमो॒ नमो॑ वो वो॒ नमो॒ नमः॑ ।\\
\\
43. नमः॑ । नमः॑ ।\\
नमो॒ नमः॑ ।\\
\\
44. नमः॑ । म॒हद्भ्यः॑ । क्षु॒ल्ल॒केभ्यः॑ ।\\
नमो॑ म॒हद्भ्यो॑ म॒हद्भ्यो॒ नमो॒ नमो॑ म॒हद्भ्यः॑ क्षुल्ल॒केभ्यः॑\\
क्षुल्ल॒केभ्यो॑ म॒हद्भ्यो॒ नमो॒ नमो॑ म॒हद्भ्यः॑ क्षुल्ल॒केभ्यः॑ ।\\
\\
45. म॒हद्भ्यः॑ । क्षु॒ल्ल॒केभ्यः॑ । च॒ ।\\
म॒हद्भ्यः॑ क्षुल्ल॒केभ्यः॑ क्षुल्ल॒केभ्यो॑ म॒हद्भ्यो॑ म॒हद्भ्यः॑\\
क्षुल्ल॒केभ्य॑श्च च क्षुल्ल॒केभ्यो॑ म॒हद्भ्यो॑ म॒हद्भ्यः॑ क्षुल्ल॒केभ्य॑श्च ।\\
\\
46. म॒हद्भ्यः॑ ।\\
म॒हद्भ्य॒ इति॑ म॒हत् - भ्यः॒ ।\\
\\
47. क्षु॒ल्ल॒केभ्यः॑ । च॒ । वः॒ ।\\
क्षु॒ल्ल॒केभ्य॑श्च च क्षुल्ल॒केभ्यः॑ क्षुल्ल॒केभ्य॑श्च वो वश्च क्षुल्ल॒केभ्यः॑\\
क्षुल्ल॒केभ्य॑श्च वः ।\\
\\
48. च॒ । वः॒ । नमः॑ ।\\
च॒ वो॒ व॒श्च॒ च॒ वो॒ नमो॒ नमो॑ वश्च च वो॒ नमः॑ ।\\
\\
49. वः॒ । नमः॑ ।\\
वो॒ नमो॒ नमो॑ वो वो॒ नमो॒ नमः॑ ।\\
\\
50. नमः॑ । नमः॑ ।\\
नमो॒ नमः॑ ।\\
\\
51. नमः॑ । र॒थिभ्यः॑ । अ॒र॒थेभ्यः॑ ।\\
नमो॑ र॒थिभ्यो॑ र॒थिभ्यो॒ नमो॒ नमो॑ र॒थिभ्यो॑ऽर॒थेभ्यो॑ऽर॒थेभ्यो॑ र॒थिभ्यो॒\\
नमो॒ नमो॑ र॒थिभ्यो॑ऽर॒थेभ्यः॑ ।\\
\\
52. र॒थिभ्यः॑ । अ॒र॒थेभ्यः॑ । च॒ ।\\
र॒थिभ्यो॑ऽर॒थेभ्यो॑ऽर॒थेभ्यो॑ र॒थिभ्यो॑ र॒थिभ्यो॑ऽर॒थेभ्य॑श्च चा र॒थेभ्यो॑\\
र॒थिभ्यो॑ र॒थिभ्यो॑ऽर॒थेभ्य॑श्च ।\\
\\
53. र॒थिभ्यः॑ ।\\
र॒थिभ्य॒ इति॑ र॒थि - भ्यः॒ ।\\
\\
54. अ॒र॒थेभ्यः॑ । च॒ । वः॒ ।\\
अ॒र॒थेभ्य॑श्च चा र॒थेभ्यो॑ऽर॒थेभ्य॑श्च वो वश्चा र॒थेभ्यो॑ऽर॒थेभ्य॑श्च वः ।\\
\\
55. च॒ । वः॒ । नमः॑ ।\\
च॒ वो॒ व॒श्च॒ च॒ वो॒ नमो॒ नमो॑ वश्च च वो॒ नमः॑ ।\\
\\
56. वः॒ । नमः॑ ।\\
वो॒ नमो॒ नमो॑ वो वो॒ नमो॒ नमः॑ ।\\
\\
57. नमः॑ । नमः॑ ।\\
नमो॒ नमः॑ ।\\
\\
58. नमः॑ । रथे᳚भ्यः । रथ॑पतिभ्यः ।\\
नमो॒ रथे᳚भ्यो॒ रथे᳚भ्यो॒ नमो॒ नमो॒ रथे᳚भ्यो॒ रथ॑पतिभ्यो॒ रथ॑पतिभ्यो॒\\
रथे᳚भ्यो॒ नमो॒ नमो॒ रथे᳚भ्यो॒ रथ॑पतिभ्यः ।\\
\\
59. रथे᳚भ्यः । रथ॑पतिभ्यः । च॒ ।\\
रथे᳚भ्यो॒ रथ॑पतिभ्यो॒ रथ॑पतिभ्यो॒ रथे᳚भ्यो॒ रथे᳚भ्यो॒ रथ॑पतिभ्यश्च च॒\\
रथ॑पतिभ्यो॒ रथे᳚भ्यो॒ रथे᳚भ्यो॒ रथ॑पतिभ्यश्च ।\\
\\
60. रथ॑पतिभ्यः । च॒ । वः॒ ।\\
रथ॑पतिभ्यश्च च॒ रथ॑पतिभ्यो॒ रथ॑पतिभ्यश्च वो वश्च॒ रथ॑पतिभ्यो॒\\
रथ॑पतिभ्यश्च वः ।\\
\\
61. रथ॑पतिभ्यः ।\\
रथ॑पतिभ्य॒ इति॒ रथ॑पति - भ्यः॒ ।\\
\\
62. च॒ । वः॒ । नमः॑ ।\\
च॒ वो॒ व॒श्च॒ च॒ वो॒ नमो॒ नमो॑ वश्च च वो॒ नमः॑ ।\\
\\
63. वः॒ । नमः॑ ।\\
वो॒ नमो॒ नमो॑ वो वो॒ नमो॒ नमः॑ ।\\
\\
64. नमः॑ । नमः॑ ।\\
नमो॒ नमः॑ ।\\
\\
65. नमः॑ । सेना᳚भ्यः । से॒ना॒निभ्यः॑ ।\\
नमः॒ सेना᳚भ्यः॒ सेना᳚भ्यो॒ नमो॒ नमः॒ सेना᳚भ्यः सेना॒निभ्यः॑\\
सेना॒निभ्यः॒ सेना᳚भ्यो॒ नमो॒ नमः॒ सेना᳚भ्यः सेना॒निभ्यः॑ ।\\
\\
66. सेना᳚भ्यः । से॒ना॒निभ्यः॑ । च॒ ।\\
सेना᳚भ्यः सेना॒निभ्यः॑ सेना॒निभ्यः॒ सेना᳚भ्यः॒ सेना᳚भ्यः सेना॒निभ्य॑श्च\\
च सेना॒निभ्यः॒ सेना᳚भ्यः॒ सेना᳚भ्यः सेना॒निभ्य॑श्च ।\\
\\
67. से॒ना॒निभ्यः॑ । च॒ । वः॒ ।\\
से॒ना॒निभ्य॑श्च च सेना॒निभ्यः॑ सेना॒निभ्य॑श्च वो वश्च सेना॒निभ्यः॑\\
सेना॒निभ्य॑श्च वः ।\\
\\
68. से॒ना॒निभ्यः॑ ।\\
से॒ना॒निभ्य॒ इति॑ सेना॒नि - भ्यः॒ ।\\
\\
69. च॒ । वः॒ । नमः॑ ।\\
च॒ वो॒ व॒श्च॒ च॒ वो॒ नमो॒ नमो॑ वश्च च वो॒ नमः॑ ।\\
\\
70. वः॒ । नमः॑ ।\\
वो॒ नमो॒ नमो॑ वो वो॒ नमो॒ नमः॑ ।\\
\\
71. नमः॑ । नमः॑ ।\\
नमो॒ नमः॑ ।\\
\\
72. नमः॑ । क्ष॒त्तृभ्यः॑ । स॒ङ्ग्र॒ही॒तृभ्यः॑ ।\\
नमः॑ क्ष॒त्तृभ्यः॑ क्ष॒त्तृभ्यो॒ नमो॒ नमः॑ क्ष॒त्तृभ्यः॑ सङ्ग्रही॒तृभ्यः॑\\
सङ्ग्रही॒तृभ्यः॑ क्ष॒त्तृभ्यो॒ नमो॒ नमः॑ क्ष॒त्तृभ्यः॑ सङ्ग्रही॒तृभ्यः॑ ।\\
\\
73. क्ष॒त्तृभ्यः॑ । स॒ङ्ग्र॒ही॒तृभ्यः॑ । च॒ ।\\
क्ष॒त्तृभ्यः॑ सङ्ग्रही॒तृभ्यः॑ सङ्ग्रही॒तृभ्यः॑ क्ष॒त्तृभ्यः॑ क्ष॒त्तृभ्यः॑\\
सङ्ग्रही॒तृभ्य॑श्च च सङ्ग्रही॒तृभ्यः॑ क्ष॒त्तृभ्यः॑ क्ष॒त्तृभ्यः॑ सङ्ग्रही॒तृभ्य॑श्च ।\\
\\
74. क्ष॒त्तृभ्यः॑ ।\\
क्ष॒त्तृभ्य॒ इति॑ क्ष॒त्तृ - भ्यः॒ ।\\
\\
75. स॒ङ्ग्र॒ही॒तृभ्यः॑ । च॒ । वः॒ ।\\
स॒ङ्ग्र॒ही॒तृभ्य॑श्च च सङ्ग्रही॒तृभ्यः॑ सङ्ग्रही॒तृभ्य॑श्च वो वश्च\\
सङ्ग्रही॒तृभ्यः॑ सङ्ग्रही॒तृभ्य॑श्च वः ।\\
\\
76. स॒ङ्ग्र॒ही॒तृभ्यः॑ ।\\
स॒ङ्ग्र॒ही॒तृभ्य॒ इति॑ सङ्ग्रही॒तृ - भ्यः॒ ।\\
\\
77. च॒ । वः॒ । नमः॑ ।\\
च॒ वो॒ व॒श्च॒ च॒ वो॒ नमो॒ नमो॑ वश्च च वो॒ नमः॑ ।\\
\\
78. वः॒ । नमः॑ ।\\
वो॒ नमो॒ नमो॑ वो वो॒ नमो॒ नमः॑ ।\\
\\
79. नमः॑ । नमः॑ ।\\
नमो॒ नमः॑ ।\\
\\
80. नमः॑ । तक्ष॑भ्यः । र॒थ॒का॒रेभ्यः॑ ।\\
नम॒ स्तक्ष॑भ्य॒ स्तक्ष॑भ्यो॒ नमो॒ नम॒ स्तक्ष॑भ्यो रथका॒रेभ्यो॑ रथका॒रेभ्य॒\\
स्तक्ष॑भ्यो॒ नमो॒ नम॒ स्तक्ष॑भ्यो रथका॒रेभ्यः॑ ।\\
\\
81. तक्ष॑भ्यः । र॒थ॒का॒रेभ्यः॑ । च॒ ।\\
तक्ष॑भ्यो रथका॒रेभ्यो॑ रथका॒रेभ्य॒ स्तक्ष॑भ्य॒ स्तक्ष॑भ्यो रथका॒रेभ्य॑श्च च\\
रथका॒रेभ्य॒ स्तक्ष॑भ्य॒ स्तक्ष॑भ्यो रथका॒रेभ्य॑श्च ।\\
\\
82. तक्ष॑भ्यः ।\\
तक्ष॑भ्य॒ इति॒ तक्ष॑ - भ्यः॒ ।\\
\\
83. र॒थ॒का॒रेभ्यः॑ । च॒ । वः॒ ।\\
र॒थ॒का॒रेभ्य॑श्च च रथका॒रेभ्यो॑ रथका॒रेभ्य॑श्च वो वश्च रथका॒रेभ्यो॑\\
रथका॒रेभ्य॑श्च वः ।\\
\\
84. र॒थ॒का॒रेभ्यः॑ ।\\
र॒थ॒का॒रेभ्य॒ इति॑ रथ - का॒रेभ्यः॑ ।\\
\\
85. च॒ । वः॒ । नमः॑ ।\\
च॒ वो॒ व॒श्च॒ च॒ वो॒ नमो॒ नमो॑ वश्च च वो॒ नमः॑ ।\\
\\
86. वः॒ । नमः॑ ।\\
वो॒ नमो॒ नमो॑ वो वो॒ नमो॒ नमः॑ ।\\
\\
87. नमः॑ । नमः॑ ।\\
नमो॒ नमः॑ ।\\
\\
88. नमः॑ । कुला॑लेभ्यः । क॒र्मारे᳚भ्यः ।\\
नमः॒ कुला॑लेभ्यः॒ कुला॑लेभ्यो॒ नमो॒ नमः॒ कुला॑लेभ्यः क॒र्मारे᳚भ्यः\\
क॒र्मारे᳚भ्यः॒ कुला॑लेभ्यो॒ नमो॒ नमः॒ कुला॑लेभ्यः क॒र्मारे᳚भ्यः ।\\
\\
89. कुला॑लेभ्यः । क॒र्मारे᳚भ्यः । च॒ ।\\
कुला॑लेभ्यः क॒र्मारे᳚भ्यः क॒र्मारे᳚भ्यः॒ कुला॑लेभ्यः॒ कुला॑लेभ्यः\\
क॒र्मारे᳚भ्यश्च च क॒र्मारे᳚भ्यः॒ कुला॑लेभ्यः॒ कुला॑लेभ्यः क॒र्मारे᳚भ्यश्च ।\\
\\
90. क॒र्मारे᳚भ्यः । च॒ । वः॒ ।\\
क॒र्मारे᳚भ्यश्च च क॒र्मारे᳚भ्यः क॒र्मारे᳚भ्यश्च वो वश्च क॒र्मारे᳚भ्यः\\
क॒र्मारे᳚भ्यश्च वः ।\\
\\
91. च॒ । वः॒ । नमः॑ ।\\
च॒ वो॒ व॒श्च॒ च॒ वो॒ नमो॒ नमो॑ वश्च च वो॒ नमः॑ ।\\
\\
92. वः॒ । नमः॑ ।\\
वो॒ नमो॒ नमो॑ वो वो॒ नमो॒ नमः॑ ।\\
\\
93. नमः॑ । नमः॑ ।\\
नमो॒ नमः॑ ।\\
\\
94. नमः॑ । पु॒ञ्जिष्टे᳚भ्यः । नि॒षा॒देभ्यः॑ ।\\
नमः॑ पु॒ञ्जिष्टे᳚भ्यः पु॒ञ्जिष्टे᳚भ्यो॒ नमो॒ नमः॑ पु॒ञ्जिष्टे᳚भ्यो निषा॒देभ्यो॑\\
निषा॒देभ्यः॑ पु॒ञ्जिष्टे᳚भ्यो॒ नमो॒ नमः॑ पु॒ञ्जिष्टे᳚भ्यो निषा॒देभ्यः॑ ।\\
\\
95. पु॒ञ्जिष्टे᳚भ्यः । नि॒षा॒देभ्यः॑ । च॒ ।\\
पु॒ञ्जिष्टे᳚भ्यो निषा॒देभ्यो॑ निषा॒देभ्यः॑ पु॒ञ्जिष्टे᳚भ्यः पु॒ञ्जिष्टे᳚भ्यो\\
निषा॒देभ्य॑श्च च निषा॒देभ्यः॑ पु॒ञ्जिष्टे᳚भ्यः पु॒ञ्जिष्टे᳚भ्यो निषा॒देभ्य॑श्च ।\\
\\
96. नि॒षा॒देभ्यः॑ । च॒ । वः॒ ।\\
नि॒षा॒देभ्य॑श्च च निषा॒देभ्यो॑ निषा॒देभ्य॑श्च वो वश्च निषा॒देभ्यो॑\\
निषा॒देभ्य॑श्च वः ।\\
\\
97. च॒ । वः॒ । नमः॑ ।\\
च॒ वो॒ व॒श्च॒ च॒ वो॒ नमो॒ नमो॑ वश्च च वो॒ नमः॑ ।\\
\\
98. वः॒ । नमः॑ ।\\
वो॒ नमो॒ नमो॑ वो वो॒ नमो॒ नमः॑ ।\\
\\
99. नमः॑ । नमः॑ ।\\
नमो॒ नमः॑ ।\\
\\
100. नमः॑ । इ॒षु॒कृद्भ्यः॑ । ध॒न्व॒कृद्भ्यः॑ ।\\
नम॑ इषु॒कृद्भ्य॑ इषु॒कृद्भ्यो॒ नमो॒ नम॑ इषु॒कृद्भ्यो॑ धन्व॒कृद्भ्यो॑\\
धन्व॒कृद्भ्य॑ इषु॒कृद्भ्यो॒ नमो॒ नम॑ इषु॒कृद्भ्यो॑ धन्व॒कृद्भ्यः॑ ।\\
\\
101. इ॒षु॒कृद्भ्यः॑ । ध॒न्व॒कृद्भ्यः॑ । च॒ ।\\
इ॒षु॒कृद्भ्यो॑ धन्व॒कृद्भ्यो॑ धन्व॒कृद्भ्य॑ इषु॒कृद्भ्य॑ इषु॒कृद्भ्यो॑\\
धन्व॒कृद्भ्य॑श्च च धन्व॒कृद्भ्य॑ इषु॒कृद्भ्य॑ इषु॒कृद्भ्यो॑ धन्व॒कृद्भ्य॑श्च ।\\
\\
102. इ॒षु॒कृद्भ्यः॑ ।\\
इ॒षु॒कृद्भ्य॒ इती॑षु॒कृत् - भ्यः॒ ।\\
\\
103. ध॒न्व॒कृद्भ्यः॑ । च॒ । वः॒ ।\\
ध॒न्व॒कृद्भ्य॑श्च च धन्व॒कृद्भ्यो॑ धन्व॒कृद्भ्य॑श्च वो वश्च धन्व॒कृद्भ्यो॑\\
धन्व॒कृद्भ्य॑श्च वः ।\\
\\
104. ध॒न्व॒कृद्भ्यः॑ ।\\
ध॒न्व॒कृद्भ्य॒ इति॑ धन्व॒कृत् - भ्यः॒ ।\\
\\
105. च॒ । वः॒ । नमः॑ ।\\
च॒ वो॒ व॒श्च॒ च॒ वो॒ नमो॒ नमो॑ वश्च च वो॒ नमः॑ ।\\
\\
106. वः॒ । नमः॑ ।\\
वो॒ नमो॒ नमो॑ वो वो॒ नमो॒ नमः॑ ।\\
\\
107. नमः॑ । नमः॑ ।\\
नमो॒ नमः॑ ।\\
\\
108. नमः॑ । मृ॒ग॒युभ्यः॑ । श्व॒निभ्यः॑ ।\\
नमो॑ मृग॒युभ्यो॑ मृग॒युभ्यो॒ नमो॒ नमो॑ मृग॒युभ्यः॑ श्व॒निभ्यः॑ श्व॒निभ्यो॑\\
मृग॒युभ्यो॒ नमो॒ नमो॑ मृग॒युभ्यः॑ श्व॒निभ्यः॑ ।\\
\\
109. मृ॒ग॒युभ्यः॑ । श्व॒निभ्यः॑ । च॒ ।\\
मृ॒ग॒युभ्यः॑ श्व॒निभ्यः॑ श्व॒निभ्यो॑ मृग॒युभ्यो॑ मृग॒युभ्यः॑ श्व॒निभ्य॑श्च च\\
श्व॒निभ्यो॑ मृग॒युभ्यो॑ मृग॒युभ्यः॑ श्व॒निभ्य॑श्च ।\\
\\
110. मृ॒ग॒युभ्यः॑ ।\\
मृ॒ग॒युभ्य॒ इति॑ मृग॒यु - भ्यः॒ ।\\
\\
111. श्व॒निभ्यः॑ । च॒ । वः॒ ।\\
श्व॒निभ्य॑श्च च श्व॒निभ्यः॑ श्व॒निभ्य॑श्च वो वश्च श्व॒निभ्यः॑ श्व॒निभ्य॑श्च वः ।\\
\\
112. श्व॒निभ्यः॑ ।\\
श्व॒निभ्य॒ इति॑ श्व॒नि - भ्यः॒ ।\\
\\
113. च॒ । वः॒ । नमः॑ ।\\
च॒ वो॒ व॒श्च॒ च॒ वो॒ नमो॒ नमो॑ वश्च च वो॒ नमः॑ ।\\
\\
114. वः॒ । नमः॑ ।\\
वो॒ नमो॒ नमो॑ वो वो॒ नमो॒ नमः॑ ।\\
\\
115. नमः॑ । नमः॑ ।\\
नमो॒ नमः॑ ।\\
\\
116. नमः॑ । श्वभ्यः॑ । श्वप॑तिभ्यः ।\\
नमः॒ श्वभ्यः॒ श्वभ्यो॒ नमो॒ नमः॒ श्वभ्यः॒ श्वप॑तिभ्यः॒ श्वप॑तिभ्यः॒ श्वभ्यो॒\\
नमो॒ नमः॒ श्वभ्यः॒ श्वप॑तिभ्यः ।\\
\\
117. श्वभ्यः॑ । श्वप॑तिभ्यः । च॒ ।\\
श्वभ्यः॒ श्वप॑तिभ्यः॒ श्वप॑तिभ्यः॒ श्वभ्यः॒ श्वभ्यः॒ श्वप॑तिभ्यश्च च॒\\
श्वप॑तिभ्यः॒ श्वभ्यः॒ श्वभ्यः॒ श्वप॑तिभ्यश्च ।\\
\\
118. श्वभ्यः॑ ।\\
श्वभ्य॒ इति॒ श्व - भ्यः॒ ।\\
\\
119. श्वप॑तिभ्यः । च॒ । वः॒ ।\\
श्वप॑तिभ्यश्च च॒ श्वप॑तिभ्यः॒ श्वप॑तिभ्यश्च वो वश्च॒ श्वप॑तिभ्यः॒\\
श्वप॑तिभ्यश्च वः ।\\
\\
120. श्वप॑तिभ्यः ।\\
श्वप॑तिभ्य॒ इति॒ श्वप॑ति - भ्यः॒ ।\\
\\
121. च॒ । वः॒ । नमः॑ ॥\\
च॒ वो॒ व॒श्च॒ च॒ वो॒ नमो॒ नमो॑ वश्च च वो॒ नमः॑ ।\\
\\
122. वः॒ । नमः॑ ॥\\
वो॒ नमो॒ नमो॑ वो वो॒ नमः॑ ।\\
\\
123. नमः॑ ॥\\
नम॒ इति॒ नमः॑ ।\\
\subsection{\eng{Anuvaka 5}}
1. नमः॑ । भ॒वाय॑ । च॒ ।\\
नमो॑ भ॒वाय॑ भ॒वाय॒ नमो॒ नमो॑ भ॒वाय॑ च च भ॒वाय॒ नमो॒ नमो॑ भ॒वाय॑ च ।\\
\\
2. भ॒वाय॑ । च॒ । रु॒द्राय॑ ।\\
भ॒वाय॑ च च भ॒वाय॑ भ॒वाय॑ च रु॒द्राय॑ रु॒द्राय॑ च भ॒वाय॑ भ॒वाय॑ च रु॒द्राय॑ ।\\
\\
3. च॒ । रु॒द्राय॑ । च॒ ।\\
च॒ रु॒द्राय॑ रु॒द्राय॑ च च रु॒द्राय॑ च च रु॒द्राय॑ च च रु॒द्राय॑ च ।\\
\\
4. रु॒द्राय॑ । च॒ । नमः॑ ।\\
रु॒द्राय॑ च च रु॒द्राय॑ रु॒द्राय॑ च॒ नमो॒ नम॑श्च रु॒द्राय॑ रु॒द्राय॑ च॒ नमः॑ ।\\
\\
5. च॒ । नमः॑ । श॒र्वाय॑ ।\\
च॒ नमो॒ नम॑श्च च॒ नमः॑ श॒र्वाय॑ श॒र्वाय॒ नम॑श्च च॒ नमः॑ श॒र्वाय॑ ।\\
\\
6. नमः॑ । श॒र्वाय॑ । च॒ ।\\
नमः॑ श॒र्वाय॑ श॒र्वाय॒ नमो॒ नमः॑ श॒र्वाय॑ च च श॒र्वाय॒ नमो॒ नमः॑\\
श॒र्वाय॑ च ।\\
\\
7. श॒र्वाय॑ । च॒ । प॒शु॒पत॑ये ।\\
श॒र्वाय॑ च च श॒र्वाय॑ श॒र्वाय॑ च पशु॒पत॑ये पशु॒पत॑ये च श॒र्वाय॑\\
श॒र्वाय॑ च पशु॒पत॑ये ।\\
\\
8. च॒ । प॒शु॒पत॑ये । च॒ ।\\
च॒ प॒शु॒पत॑ये पशु॒पत॑ये च च पशु॒पत॑ये च च पशु॒पत॑ये च च\\
पशु॒पत॑ये च ।\\
\\
9. प॒शु॒पत॑ये । च॒ । नमः॑ ।\\
प॒शु॒पत॑ये च च पशु॒पत॑ये पशु॒पत॑ये च॒ नमो॒ नम॑श्च पशु॒पत॑ये\\
पशु॒पत॑ये च॒ नमः॑ ।\\
\\
10. प॒शु॒पत॑ये ।\\
प॒शु॒पत॑य॒ इति॑ पशु - पत॑ये ।\\
\\
11. च॒ । नमः॑ । नील॑ग्रीवाय ।\\
च॒ नमो॒ नम॑श्च च॒ नमो॒ नील॑ग्रीवाय॒ नील॑ग्रीवाय॒ नम॑श्च च॒ नमो॒\\
नील॑ग्रीवाय ।\\
\\
12. नमः॑ । नील॑ग्रीवाय । च॒ ।\\
नमो॒ नील॑ग्रीवाय॒ नील॑ग्रीवाय॒ नमो॒ नमो॒ नील॑ग्रीवाय च च॒ नील॑ग्रीवाय॒\\
नमो॒ नमो॒ नील॑ग्रीवाय च ।\\
\\
13. नील॑ग्रीवाय । च॒ । शि॒ति॒कण्ठा॑य ।\\
नील॑ग्रीवाय च च॒ नील॑ग्रीवाय॒ नील॑ग्रीवाय च शिति॒कण्ठा॑य\\
शिति॒कण्ठा॑य च॒ नील॑ग्रीवाय॒ नील॑ग्रीवाय च शिति॒कण्ठा॑य ।\\
\\
14. नील॑ग्रीवाय ।\\
नील॑ग्रीवा॒येति॒ नील॑ - ग्री॒वा॒य॒ ।\\
\\
15. च॒ । शि॒ति॒कण्ठा॑य । च॒ ।\\
च॒ शि॒ति॒कण्ठा॑य शिति॒कण्ठा॑य च च शिति॒कण्ठा॑य च च शिति॒कण्ठा॑य च\\
च शिति॒कण्ठा॑य च ।\\
\\
16. शि॒ति॒कण्ठा॑य । च॒ । नमः॑ ।\\
शि॒ति॒कण्ठा॑य च च शिति॒कण्ठा॑य शिति॒कण्ठा॑य च॒ नमो॒ नम॑श्च\\
शिति॒कण्ठा॑य शिति॒कण्ठा॑य च॒ नमः॑ ।\\
\\
17. शि॒ति॒कण्ठा॑य ।\\
शि॒ति॒कण्ठा॒येति॑ शिति - कण्ठा॑य ।\\
\\
18. च॒ । नमः॑ । क॒प॒र्दिने᳚ ।\\
च॒ नमो॒ नम॑श्च च॒ नमः॑ कप॒र्दिने॑ कप॒र्दिने॒ नम॑श्च च॒ नमः॑ कप॒र्दिने᳚ ।\\
\\
19. नमः॑ । क॒प॒र्दिने᳚ । च॒ ।\\
नमः॑ कप॒र्दिने॑ कप॒र्दिने॒ नमो॒ नमः॑ कप॒र्दिने॑ च च कप॒र्दिने॒ नमो॒\\
नमः॑ कप॒र्दिने॑ च ।\\
\\
20. क॒प॒र्दिने᳚ । च॒ । व्यु॑प्तकेशाय ।\\
क॒प॒र्दिने॑ च च कप॒र्दिने॑ कप॒र्दिने॑ च॒ व्यु॑प्तकेशाय॒ व्यु॑प्तकेशाय च\\
कप॒र्दिने॑ कप॒र्दिने॑ च॒ व्यु॑प्तकेशाय ।\\
\\
21. च॒ । व्यु॑प्तकेशाय । च॒ ।\\
च॒ व्यु॑प्तकेशाय॒ व्यु॑प्तकेशाय च च॒ व्यु॑प्तकेशाय च च॒ व्यु॑प्तकेशाय\\
च च॒ व्यु॑प्तकेशाय च ।\\
\\
22. व्यु॑प्तकेशाय । च॒ । नमः॑ ।\\
व्यु॑प्तकेशाय च च॒ व्यु॑प्तकेशाय॒ व्यु॑प्तकेशाय च॒ नमो॒ नम॑श्च॒\\
व्यु॑प्तकेशाय॒ व्यु॑प्तकेशाय च॒ नमः॑ ।\\
\\
23. व्यु॑प्तकेशाय ।\\
व्यु॑प्तकेशा॒येति॒ व्यु॑प्त - के॒शा॒य॒ ।\\
\\
24. च॒ । नमः॑ । स॒ह॒स्रा॒क्षाय॑ ।\\
च॒ नमो॒ नम॑श्च च॒ नमः॑ सहस्रा॒क्षाय॑ सहस्रा॒क्षाय॒ नम॑श्च च॒ नमः॑\\
सहस्रा॒क्षाय॑ ।\\
\\
25. नमः॑ । स॒ह॒स्रा॒क्षाय॑ । च॒ ।\\
नमः॑ सहस्रा॒क्षाय॑ सहस्रा॒क्षाय॒ नमो॒ नमः॑ सहस्रा॒क्षाय॑ च च सहस्रा॒क्षाय॒\\
नमो॒ नमः॑ सहस्रा॒क्षाय॑ च ।\\
\\
26. स॒ह॒स्रा॒क्षाय॑ । च॒ । श॒तध॑न्वने ।\\
स॒ह॒स्रा॒क्षाय॑ च च सहस्रा॒क्षाय॑ सहस्रा॒क्षाय॑ च श॒तध॑न्वने श॒तध॑न्वने\\
च सहस्रा॒क्षाय॑ सहस्रा॒क्षाय॑ च श॒तध॑न्वने ।\\
\\
27. स॒ह॒स्रा॒क्षाय॑ ।\\
स॒ह॒स्रा॒क्षायेति॑ सहस्र - अ॒क्षाय॑ ।\\
\\
28. च॒ । श॒तध॑न्वने । च॒ ।\\
च॒ श॒तध॑न्वने श॒तध॑न्वने च च श॒तध॑न्वने च च श॒तध॑न्वने च च\\
श॒तध॑न्वने च ।\\
\\
29. श॒तध॑न्वने । च॒ । नमः॑ ।\\
श॒तध॑न्वने च च श॒तध॑न्वने श॒तध॑न्वने च॒ नमो॒ नम॑श्च श॒तध॑न्वने\\
श॒तध॑न्वने च॒ नमः॑ ।\\
\\
30. श॒तध॑न्वने ।\\
श॒तध॑न्वन॒ इति॑ श॒त - ध॒न्व॒ने॒ ।\\
\\
31. च॒ । नमः॑ । गि॒रि॒शाय॑ ।\\
च॒ नमो॒ नम॑श्च च॒ नमो॑ गिरि॒शाय॑ गिरि॒शाय॒ नम॑श्च च॒ नमो॑ गिरि॒शाय॑ ।\\
\\
32. नमः॑ । गि॒रि॒शाय॑ । च॒ ।\\
नमो॑ गिरि॒शाय॑ गिरि॒शाय॒ नमो॒ नमो॑ गिरि॒शाय॑ च च गिरि॒शाय॒ नमो॒\\
नमो॑ गिरि॒शाय॑ च ।\\
\\
33. गि॒रि॒शाय॑ । च॒ । शि॒पि॒वि॒ष्टाय॑ ।\\
गि॒रि॒शाय॑ च च गिरि॒शाय॑ गिरि॒शाय॑ च शिपिवि॒ष्टाय॑ शिपिवि॒ष्टाय॑ च\\
गिरि॒शाय॑ गिरि॒शाय॑ च शिपिवि॒ष्टाय॑ ।\\
\\
34. च॒ । शि॒पि॒वि॒ष्टाय॑ । च॒ ।\\
च॒ शि॒पि॒वि॒ष्टाय॑ शिपिवि॒ष्टाय॑ च च शिपिवि॒ष्टाय॑ च च शिपिवि॒ष्टाय॑ च\\
च शिपिवि॒ष्टाय॑ च ।\\
\\
35. शि॒पि॒वि॒ष्टाय॑ । च॒ । नमः॑ ।\\
शि॒पि॒वि॒ष्टाय॑ च च शिपिवि॒ष्टाय॑ शिपिवि॒ष्टाय॑ च॒ नमो॒ नम॑श्च शिपिवि॒ष्टाय॑\\
शिपिवि॒ष्टाय॑ च॒ नमः॑ ।\\
\\
36. शि॒पि॒वि॒ष्टाय॑ ।\\
शि॒पि॒वि॒ष्टायेति॑ शिपि - वि॒ष्टाय॑ ।\\
\\
37. च॒ । नमः॑ । मी॒ढुष्ट॑माय ।\\
च॒ नमो॒ नम॑श्च च॒ नमो॑ मी॒ढुष्ट॑माय मी॒ढुष्ट॑माय॒ नम॑श्च च॒ नमो॑\\
मी॒ढुष्ट॑माय ।\\
\\
38. नमः॑ । मी॒ढुष्ट॑माय । च॒ ।\\
नमो॑ मी॒ढुष्ट॑माय मी॒ढुष्ट॑माय॒ नमो॒ नमो॑ मी॒ढुष्ट॑माय च च मी॒ढुष्ट॑माय॒\\
नमो॒ नमो॑ मी॒ढुष्ट॑माय च ।\\
\\
39. मी॒ढुष्ट॑माय । च॒ । इषु॑मते ।\\
मी॒ढुष्ट॑माय च च मी॒ढुष्ट॑माय मी॒ढुष्ट॑माय॒ चेषु॑मत॒ इषु॑मते च मी॒ढुष्ट॑माय\\
मी॒ढुष्ट॑माय॒ चेषु॑मते ।\\
\\
40. मी॒ढुष्ट॑माय ।\\
मी॒ढुष्ट॑मा॒येति॑ मी॒ढुः - त॒मा॒य॒ ।\\
\\
41. च॒ । इषु॑मते । च॒ ।\\
चेषु॑मत॒ इषु॑मते च॒ चेषु॑मते च॒ चेषु॑मते च॒ चेषु॑मते च ।\\
\\
42. इषु॑मते । च॒ । नमः॑ ।\\
इषु॑मते च॒ चेषु॑मत॒ इषु॑मते च॒ नमो॒ नम॒ श्चेषु॑मत॒ इषु॑मते च॒ नमः॑ ।\\
\\
43. इषु॑मते ।\\
इषु॑मत॒ इतीषु॑ - म॒ते॒ ।\\
\\
44. च॒ । नमः॑ । ह्र॒स्वाय॑ ।\\
च॒ नमो॒ नम॑श्च च॒ नमो᳚ ह्र॒स्वाय॑ ह्र॒स्वाय॒ नम॑श्च च॒ नमो᳚ ह्र॒स्वाय॑ ।\\
\\
45. नमः॑ । ह्र॒स्वाय॑ । च॒ ।\\
नमो᳚ ह्र॒स्वाय॑ ह्र॒स्वाय॒ नमो॒ नमो᳚ ह्र॒स्वाय॑ च च ह्र॒स्वाय॒ नमो॒ नमो᳚\\
ह्र॒स्वाय॑ च ।\\
\\
46. ह्र॒स्वाय॑ । च॒ । वा॒म॒नाय॑ ।\\
ह्र॒स्वाय॑ च च ह्र॒स्वाय॑ ह्र॒स्वाय॑ च वाम॒नाय॑ वाम॒नाय॑ च ह्र॒स्वाय॑\\
ह्र॒स्वाय॑ च वाम॒नाय॑ ।\\
\\
47. च॒ । वा॒म॒नाय॑ । च॒ ।\\
च॒ वा॒म॒नाय॑ वाम॒नाय॑ च च वाम॒नाय॑ च च वाम॒नाय॑ च च वाम॒नाय॑ च ।\\
\\
48. वा॒म॒नाय॑ । च॒ । नमः॑ ।\\
वा॒म॒नाय॑ च च वाम॒नाय॑ वाम॒नाय॑ च॒ नमो॒ नम॑श्च वाम॒नाय॑ वाम॒नाय॑\\
च॒ नमः॑ ।\\
\\
49. च॒ । नमः॑ । बृ॒ह॒ते ।\\
च॒ नमो॒ नम॑श्च च॒ नमो॑ बृह॒ते बृ॑ह॒ते नम॑श्च च॒ नमो॑ बृह॒ते ।\\
\\
50. नमः॑ । बृ॒ह॒ते । च॒ ।\\
नमो॑ बृह॒ते बृ॑ह॒ते नमो॒ नमो॑ बृह॒ते च॑ च बृह॒ते नमो॒ नमो॑ बृह॒ते च॑ ।\\
\\
51. बृ॒ह॒ते । च॒ । वर्.षी॑यसे ।\\
बृ॒ह॒ते च॑ च बृह॒ते बृ॑ह॒ते च॒ वर्.षी॑यसे॒ वर्.षी॑यसे च बृह॒ते बृ॑ह॒ते\\
च॒ वर्.षी॑यसे ।\\
\\
52. च॒ । वर्.षी॑यसे । च॒ ।\\
च॒ वर्.षी॑यसे॒ वर्.षी॑यसे च च॒ वर्.षी॑यसे च च॒ वर्.षी॑यसे च च॒\\
वर्.षी॑यसे च ।\\
\\
53. वर्.षी॑यसे । च॒ । नमः॑ ।\\
वर्.षी॑यसे च च॒ वर्.षी॑यसे॒ वर्.षी॑यसे च॒ नमो॒ नम॑श्च॒ वर्.षी॑यसे॒\\
वर्.षी॑यसे च॒ नमः॑ ।\\
\\
54. च॒ । नमः॑ । वृ॒द्धाय॑ ।\\
च॒ नमो॒ नम॑श्च च॒ नमो॑ वृ॒द्धाय॑ वृ॒द्धाय॒ नम॑श्च च॒ नमो॑ वृ॒द्धाय॑ ।\\
\\
55. नमः॑ । वृ॒द्धाय॑ । च॒ ।\\
नमो॑ वृ॒द्धाय॑ वृ॒द्धाय॒ नमो॒ नमो॑ वृ॒द्धाय॑ च च वृ॒द्धाय॒ नमो॒ नमो॑\\
वृ॒द्धाय॑ च ।\\
\\
56. वृ॒द्धाय॑ । च॒ । स॒म्ँवृद्ध्व॑ने ।\\
वृ॒द्धाय॑ च च वृ॒द्धाय॑ वृ॒द्धाय॑ च स॒म्ँवृद्ध्व॑ने स॒म्ँवृद्ध्व॑ने च वृ॒द्धाय॑\\
वृ॒द्धाय॑ च स॒म्ँवृद्ध्व॑ने ।\\
\\
57. च॒ । स॒म्ँवृद्ध्व॑ने । च॒ ।\\
च॒ स॒म्ँवृद्ध्व॑ने स॒म्ँवृद्ध्व॑ने च च स॒म्ँवृद्ध्व॑ने च च स॒म्ँवृद्ध्व॑ने\\
च च स॒म्ँवृद्ध्व॑ने च ।\\
\\
58. स॒म्ँवृद्ध्व॑ने । च॒ । नमः॑ ।\\
स॒म्ँवृद्ध्व॑ने च च स॒म्ँवृद्ध्व॑ने स॒म्ँवृद्ध्व॑ने च॒ नमो॒ नम॑श्च\\
स॒म्ँवृद्ध्व॑ने स॒म्ँवृद्ध्व॑ने च॒ नमः॑ ।\\
\\
59. स॒म्ँवृद्ध्व॑ने ।\\
स॒म्ँवृद्ध्व॑न॒ इति॑ सम् - वृद्ध्व॑ने ।\\
\\
60. च॒ । नमः॑ । अग्रि॑याय ।\\
च॒ नमो॒ नम॑श्च च॒ नमो॒ अग्रि॑या॒या ग्रि॑याय॒ नम॑श्च च॒ नमो॒ अग्रि॑याय ।\\
\\
61. नमः॑ । अग्रि॑याय । च॒ ।\\
नमो॒ अग्रि॑या॒या ग्रि॑याय॒ नमो॒ नमो॒ अग्रि॑याय च॒ चाग्रि॑याय॒ नमो॒ नमो॒\\
अग्रि॑याय च ।\\
\\
62. अग्रि॑याय । च॒ । प्र॒थ॒माय॑ ।\\
अग्रि॑याय च॒ चाग्रि॑या॒या ग्रि॑याय च प्रथ॒माय॑ प्रथ॒माय॒ चाग्रि॑या॒या ग्रि॑याय\\
च प्रथ॒माय॑ ।\\
\\
63. च॒ । प्र॒थ॒माय॑ । च॒ ।\\
च॒ प्र॒थ॒माय॑ प्रथ॒माय॑ च च प्रथ॒माय॑ च च प्रथ॒माय॑ च च प्रथ॒माय॑ च ।\\
\\
64. प्र॒थ॒माय॑ । च॒ । नमः॑ ।\\
प्र॒थ॒माय॑ च च प्रथ॒माय॑ प्रथ॒माय॑ च॒ नमो॒ नम॑श्च प्रथ॒माय॑ प्रथ॒माय॑ च॒\\
नमः॑ ।\\
\\
65. च॒ । नमः॑ । आ॒शवे᳚ ।\\
च॒ नमो॒ नम॑श्च च॒ नम॑ आ॒शव॑ आ॒शवे॒ नम॑श्च च॒ नम॑ आ॒शवे᳚ ।\\
\\
66. नमः॑ । आ॒शवे᳚ । च॒ ।\\
नम॑ आ॒शव॑ आ॒शवे॒ नमो॒ नम॑ आ॒शवे॑ च चा॒शवे॒ नमो॒ नम॑ आ॒शवे॑ च ।\\
\\
67. आ॒शवे᳚ । च॒ । अ॒जि॒राय॑ ।\\
आ॒शवे॑ च चा॒शव॑ आ॒शवे॑ चा जि॒राया॑ जि॒राय॑ चा॒शव॑ आ॒शवे॑\\
चा जि॒राय॑ ।\\
\\
68. च॒ । अ॒जि॒राय॑ । च॒ ।\\
चा॒ जि॒राया॑ जि॒राय॑ च चा जि॒राय॑ च चा जि॒राय॑ च चा जि॒राय॑ च ।\\
\\
69. अ॒जि॒राय॑ । च॒ । नमः॑ ।\\
अ॒जि॒राय॑ च चा जि॒राया॑ जि॒राय॑ च॒ नमो॒ नम॑श्चा जि॒राया॑ जि॒राय॑\\
च॒ नमः॑ ।\\
\\
70. च॒ । नमः॑ । शीघ्रि॑याय ।\\
च॒ नमो॒ नम॑श्च च॒ नमः॒ शीघ्रि॑याय॒ शीघ्रि॑याय॒ नम॑श्च च॒ नमः॒ शीघ्रि॑याय ।\\
\\
71. नमः॑ । शीघ्रि॑याय । च॒ ।\\
नमः॒ शीघ्रि॑याय॒ शीघ्रि॑याय॒ नमो॒ नमः॒ शीघ्रि॑याय च च॒ शीघ्रि॑याय॒\\
नमो॒ नमः॒ शीघ्रि॑याय च ।\\
\\
72. शीघ्रि॑याय । च॒ । शीभ्या॑य ।\\
शीघ्रि॑याय च च॒ शीघ्रि॑याय॒ शीघ्रि॑याय च॒ शीभ्या॑य॒ शीभ्या॑य च॒ शीघ्रि॑याय॒\\
शीघ्रि॑याय च॒ शीभ्या॑य ।\\
\\
73. च॒ । शीभ्या॑य । च॒ ।\\
च॒ शीभ्या॑य॒ शीभ्या॑य च च॒ शीभ्या॑य च च॒ शीभ्या॑य च च॒\\
शीभ्या॑य च ।\\
\\
74. शीभ्या॑य । च॒ । नमः॑ ।\\
शीभ्या॑य च च॒ शीभ्या॑य॒ शीभ्या॑य च॒ नमो॒ नम॑श्च॒ शीभ्या॑य॒\\
शीभ्या॑य च॒ नमः॑ ।\\
\\
75. च॒ । नमः॑ । ऊ॒र्म्या॑य ।\\
च॒ नमो॒ नम॑श्च च॒ नम॑ ऊ॒र्म्या॑ यो॒र्म्या॑य॒ नम॑श्च च॒ नम॑ ऊ॒र्म्या॑य ।\\
\\
76. नमः॑ । ऊ॒र्म्या॑य । च॒ ।\\
नम॑ ऊ॒र्म्या॑ यो॒र्म्या॑य॒ नमो॒ नम॑ ऊ॒र्म्या॑य च चो॒र्म्या॑य॒ नमो॒ नम॑\\
ऊ॒र्म्या॑य च ।\\
\\
77. ऊ॒र्म्या॑य । च॒ । अ॒व॒स्व॒न्या॑य ।\\
ऊ॒र्म्या॑य च चो॒र्म्या॑ यो॒र्म्या॑य चा वस्व॒न्या॑या वस्व॒न्या॑य चो॒र्म्या॑ यो॒र्म्या॑य\\
चा वस्व॒न्या॑य ।\\
\\
78. च॒ । अ॒व॒स्व॒न्या॑य । च॒ ।\\
चा॒ व॒स्व॒न्या॑या वस्व॒न्या॑य च चा वस्व॒न्या॑य च चा वस्व॒न्या॑य च\\
चा वस्व॒न्या॑य च ।\\
\\
79. अ॒व॒स्व॒न्या॑य । च॒ । नमः॑ ।\\
अ॒व॒स्व॒न्या॑य च चा वस्व॒न्या॑या वस्व॒न्या॑य च॒ नमो॒ नम॑श्चा वस्व॒न्या॑या\\
वस्व॒न्या॑य च॒ नमः॑ ।\\
\\
80. अ॒व॒स्व॒न्या॑य ।\\
अ॒व॒स्व॒न्या॑येत्य॑व - स्व॒न्या॑य ।\\
\\
81. च॒ । नमः॑ । स्रो॒त॒स्या॑य ।\\
च॒ नमो॒ नम॑श्च च॒ नमः॑ स्रोत॒स्या॑य स्रोत॒स्या॑य॒ नम॑श्च च॒ नमः॑\\
स्रोत॒स्या॑य ।\\
\\
82. नमः॑ । स्रो॒त॒स्या॑य । च॒ ।\\
नमः॑ स्रोत॒स्या॑य स्रोत॒स्या॑य॒ नमो॒ नमः॑ स्रोत॒स्या॑य च च स्रोत॒स्या॑य॒\\
नमो॒ नमः॑ स्रोत॒स्या॑य च ।\\
\\
83. स्रो॒त॒स्या॑य । च॒ । द्वीप्या॑य ।\\
स्रो॒त॒स्या॑य च च स्रोत॒स्या॑य स्रोत॒स्या॑य च॒ द्वीप्या॑य॒ द्वीप्या॑य च\\
स्रोत॒स्या॑य स्रोत॒स्या॑य च॒ द्वीप्या॑य ।\\
\\
84. च॒ । द्वीप्या॑य । च॒ ॥\\
च॒ द्वीप्या॑य॒ द्वीप्या॑य च च॒ द्वीप्या॑य च च॒ द्वीप्या॑य च च॒ द्वीप्या॑य च ।\\
\\
85. द्वीप्या॑य । च॒ ॥\\
द्वीप्या॑य च च॒ द्वीप्या॑य॒ द्वीप्या॑य च ।\\
\\
86. च॒ ॥\\
चेति॑ च ।\\
\subsection{\eng{Anuvaka 6}}
1. नमः॑ । ज्ये॒ष्ठाय॑ । च॒ ।\\
नमो᳚ ज्ये॒ष्ठाय॑ ज्ये॒ष्ठाय॒ नमो॒ नमो᳚ ज्ये॒ष्ठाय॑ च च ज्ये॒ष्ठाय॒ नमो॒ नमो᳚\\
ज्ये॒ष्ठाय॑ च ।\\
\\
2. ज्ये॒ष्ठाय॑ । च॒ । क॒नि॒ष्ठाय॑ ।\\
ज्ये॒ष्ठाय॑ च च ज्ये॒ष्ठाय॑ ज्ये॒ष्ठाय॑ च कनि॒ष्ठाय॑ कनि॒ष्ठाय॑ च ज्ये॒ष्ठाय॑\\
ज्ये॒ष्ठाय॑ च कनि॒ष्ठाय॑ ।\\
\\
3. च॒ । क॒नि॒ष्ठाय॑ । च॒ ।\\
च॒ क॒नि॒ष्ठाय॑ कनि॒ष्ठाय॑ च च कनि॒ष्ठाय॑ च च कनि॒ष्ठाय॑ च च\\
कनि॒ष्ठाय॑ च ।\\
\\
4. क॒नि॒ष्ठाय॑ । च॒ । नमः॑ ।\\
क॒नि॒ष्ठाय॑ च च कनि॒ष्ठाय॑ कनि॒ष्ठाय॑ च॒ नमो॒ नम॑श्च कनि॒ष्ठाय॑\\
कनि॒ष्ठाय॑ च॒ नमः॑ ।\\
\\
5. च॒ । नमः॑ । पू॒र्व॒जाय॑ ।\\
च॒ नमो॒ नम॑श्च च॒ नमः॑ पूर्व॒जाय॑ पूर्व॒जाय॒ नम॑श्च च॒ नमः॑ पूर्व॒जाय॑ ।\\
\\
6. नमः॑ । पू॒र्व॒जाय॑ । च॒ ।\\
नमः॑ पूर्व॒जाय॑ पूर्व॒जाय॒ नमो॒ नमः॑ पूर्व॒जाय॑ च च पूर्व॒जाय॒ नमो॒ नमः॑\\
पूर्व॒जाय॑ च ।\\
\\
7. पू॒र्व॒जाय॑ । च॒ । अ॒प॒र॒जाय॑ ।\\
पू॒र्व॒जाय॑ च च पूर्व॒जाय॑ पूर्व॒जाय॑ चा पर॒जाया॑ पर॒जाय॑ च पूर्व॒जाय॑\\
पूर्व॒जाय॑ चा पर॒जाय॑ ।\\
\\
8. पू॒र्व॒जाय॑ ।\\
पू॒र्व॒जायेति॑ पूर्व - जाय॑ ।\\
\\
9. च॒ । अ॒प॒र॒जाय॑ । च॒\\
चा॒ प॒र॒जाया॑ पर॒जाय॑ च चा पर॒जाय॑ च चा पर॒जाय॑ च चा पर॒जाय॑ च ।\\
\\
10. अ॒प॒र॒जाय॑ । च॒ । नमः॑ ।\\
अ॒प॒र॒जाय॑ च चा पर॒जाया॑ पर॒जाय॑ च॒ नमो॒ नम॑श्चा पर॒जाया॑\\
पर॒जाय॑ च॒ नमः॑ ।\\
\\
11. अ॒प॒र॒जाय॑ ।\\
अ॒प॒र॒जायेत्य॑पर - जाय॑ ।\\
\\
12. च॒ । नमः॑ । म॒द्ध्य॒माय॑ ।\\
च॒ नमो॒ नम॑श्च च॒ नमो॑ मद्ध्य॒माय॑ मद्ध्य॒माय॒ नम॑श्च च॒ नमो॑\\
मद्ध्य॒माय॑ ।\\
\\
13. नमः॑ । म॒द्ध्य॒माय॑ । च॒ ।\\
नमो॑ मद्ध्य॒माय॑ मद्ध्य॒माय॒ नमो॒ नमो॑ मद्ध्य॒माय॑ च च मद्ध्य॒माय॒\\
नमो॒ नमो॑ मद्ध्य॒माय॑ च ।\\
\\
14. म॒द्ध्य॒माय॑ । च॒ । अ॒प॒ग॒ल्भाय॑ ।\\
म॒द्ध्य॒माय॑ च च मद्ध्य॒माय॑ मद्ध्य॒माय॑ चा पग॒ल्भाया॑ पग॒ल्भाय॑\\
च मद्ध्य॒माय॑ मद्ध्य॒माय॑ चा पग॒ल्भाय॑ ।\\
\\
15. च॒ । अ॒प॒ग॒ल्भाय॑ । च॒ ।\\
चा॒ प॒ग॒ल्भाया॑ पग॒ल्भाय॑ च चा पग॒ल्भाय॑ च चा पग॒ल्भाय॑ च चा\\
पग॒ल्भाय॑ च ।\\
\\
16. अ॒प॒ग॒ल्भाय॑ । च॒ । नमः॑ ।\\
अ॒प॒ग॒ल्भाय॑ च चा पग॒ल्भाया॑ पग॒ल्भाय॑ च॒ नमो॒ नम॑श्चा पग॒ल्भाया॑\\
पग॒ल्भाय॑ च॒ नमः॑ ।\\
\\
17. अ॒प॒ग॒ल्भाय॑ ।\\
अ॒प॒ग॒ल्भायेत्य॑प - ग॒ल्भाय॑ ।\\
\\
18. च॒ । नमः॑ । ज॒घ॒न्या॑य ।\\
च॒ नमो॒ नम॑श्च च॒ नमो॑ जघ॒न्या॑य जघ॒न्या॑य॒ नम॑श्च च॒ नमो॑ जघ॒न्या॑य ।\\
\\
19. नमः॑ । ज॒घ॒न्या॑य । च॒ ।\\
नमो॑ जघ॒न्या॑य जघ॒न्या॑य॒ नमो॒ नमो॑ जघ॒न्या॑य च च जघ॒न्या॑य॒ नमो॒\\
नमो॑ जघ॒न्या॑य च ।\\
\\
20. ज॒घ॒न्या॑य । च॒ । बुद्ध्नि॑याय ।\\
ज॒घ॒न्या॑य च च जघ॒न्या॑य जघ॒न्या॑य च॒ बुद्ध्नि॑याय॒ बुद्ध्नि॑याय च\\
जघ॒न्या॑य जघ॒न्या॑य च॒ बुद्ध्नि॑याय ।\\
\\
21. च॒ । बुद्ध्नि॑याय । च॒ ।\\
च॒ बुद्ध्नि॑याय॒ बुद्ध्नि॑याय च च॒ बुद्ध्नि॑याय च च॒ बुद्ध्नि॑याय च च॒\\
बुद्ध्नि॑याय च ।\\
\\
22. बुद्ध्नि॑याय । च॒ । नमः॑ ।\\
बुद्ध्नि॑याय च च॒ बुद्ध्नि॑याय॒ बुद्ध्नि॑याय च॒ नमो॒ नम॑श्च॒ बुद्ध्नि॑याय॒\\
बुद्ध्नि॑याय च॒ नमः॑ ।\\
\\
23. च॒ । नमः॑ । सो॒भ्या॑य ।\\
च॒ नमो॒ नम॑श्च च॒ नमः॑ सो॒भ्या॑य सो॒भ्या॑य॒ नम॑श्च च॒ नमः॑ सो॒भ्या॑य ।\\
\\
24. नमः॑ । सो॒भ्या॑य । च॒ ।\\
नमः॑ सो॒भ्या॑य सो॒भ्या॑य॒ नमो॒ नमः॑ सो॒भ्या॑य च च सो॒भ्या॑य॒ नमो॒\\
नमः॑ सो॒भ्या॑य च ।\\
\\
25. सो॒भ्या॑य । च॒ । प्र॒ति॒स॒र्या॑य ।\\
सो॒भ्या॑य च च सो॒भ्या॑य सो॒भ्या॑य च प्रतिस॒र्या॑य प्रतिस॒र्या॑य च\\
सो॒भ्या॑य सो॒भ्या॑य च प्रतिस॒र्या॑य ।\\
\\
26. च॒ । प्र॒ति॒स॒र्या॑य । च॒ ।\\
च॒ प्र॒ति॒स॒र्या॑य प्रतिस॒र्या॑य च च प्रतिस॒र्या॑य च च प्रतिस॒र्या॑य\\
च च प्रतिस॒र्या॑य च ।\\
\\
27. प्र॒ति॒स॒र्या॑य । च॒ । नमः॑ ।\\
प्र॒ति॒स॒र्या॑य च च प्रतिस॒र्या॑य प्रतिस॒र्या॑य च॒ नमो॒ नम॑श्च प्रतिस॒र्या॑य\\
प्रतिस॒र्या॑य च॒ नमः॑ ।\\
\\
28. प्र॒ति॒स॒र्या॑य ।\\
प्र॒ति॒स॒र्या॑येति॑ प्रति - स॒र्या॑य ।\\
\\
29. च॒ । नमः॑ । याम्या॑य ।\\
च॒ नमो॒ नम॑श्च च॒ नमो॒ याम्या॑य॒ याम्या॑य॒ नम॑श्च च॒ नमो॒ याम्या॑य ।\\
\\
30. नमः॑ । याम्या॑य । च॒ ।\\
नमो॒ याम्या॑य॒ याम्या॑य॒ नमो॒ नमो॒ याम्या॑य च च॒ याम्या॑य॒ नमो॒\\
नमो॒ याम्या॑य च ।\\
\\
31. याम्या॑य । च॒ । क्षेम्या॑य ।\\
याम्या॑य च च॒ याम्या॑य॒ याम्या॑य च॒ क्षेम्या॑य॒ क्षेम्या॑य च॒ याम्या॑य॒\\
याम्या॑य च॒ क्षेम्या॑य ।\\
\\
32. च॒ । क्षेम्या॑य । च॒ ।\\
च॒ क्षेम्या॑य॒ क्षेम्या॑य च च॒ क्षेम्या॑य च च॒ क्षेम्या॑य च च॒ क्षेम्या॑य च ।\\
\\
33. क्षेम्या॑य । च॒ । नमः॑ ।\\
क्षेम्या॑य च च॒ क्षेम्या॑य॒ क्षेम्या॑य च॒ नमो॒ नम॑श्च॒ क्षेम्या॑य॒\\
क्षेम्या॑य च॒ नमः॑ ।\\
\\
34. च॒ । नमः॑ । उ॒र्व॒र्या॑य ।\\
च॒ नमो॒ नम॑श्च च॒ नम॑ उर्व॒र्या॑ योर्व॒र्या॑य॒ नम॑श्च च॒ नम॑ उर्व॒र्या॑य ।\\
\\
35. नमः॑ । उ॒र्व॒र्या॑य । च॒ ।\\
नम॑ उर्व॒र्या॑ योर्व॒र्या॑य॒ नमो॒ नम॑ उर्व॒र्या॑य च चोर्व॒र्या॑य॒ नमो॒ नम॑\\
उर्व॒र्या॑य च ।\\
\\
36. उ॒र्व॒र्या॑य । च॒ । खल्या॑य ।\\
उ॒व॒र्र्या॑य च चोर्व॒र्या॑ योर्व॒र्या॑य च॒ खल्या॑य॒ खल्या॑य चोर्व॒र्या॑\\
योर्व॒र्या॑य च॒ खल्या॑य ।\\
\\
37. च॒ । खल्या॑य । च॒ ।\\
च॒ खल्या॑य॒ खल्या॑य च च॒ खल्या॑य च च॒ खल्या॑य च च॒ खल्या॑य च ।\\
\\
38. खल्या॑य । च॒ । नमः॑ ।\\
खल्या॑य च च॒ खल्या॑य॒ खल्या॑य च॒ नमो॒ नम॑श्च॒ खल्या॑य॒\\
खल्या॑य च॒ नमः॑ ।\\
\\
39. च॒ । नमः॑ । श्लोक्या॑य ।\\
च॒ नमो॒ नम॑श्च च॒ नमः॒ श्लोक्या॑य॒ श्लोक्या॑य॒ नम॑श्च च॒ नमः॒\\
श्लोक्या॑य ।\\
\\
40. नमः॑ । श्लोक्या॑य । च॒ ।\\
नमः॒ श्लोक्या॑य॒ श्लोक्या॑य॒ नमो॒ नमः॒ श्लोक्या॑य च च॒ श्लोक्या॑य॒\\
नमो॒ नमः॒ श्लोक्या॑य च ।\\
\\
41. श्लोक्या॑य । च॒ । अ॒व॒सा॒न्या॑य ।\\
श्लोक्या॑य च च॒ श्लोक्या॑य॒ श्लोक्या॑य चा वसा॒न्या॑या वसा॒न्या॑य\\
च॒ श्लोक्या॑य॒ श्लोक्या॑य चा वसा॒न्या॑य ।\\
\\
42. च॒ । अ॒व॒सा॒न्या॑य । च॒ ।\\
चा॒ व॒सा॒न्या॑या वसा॒न्या॑य च चा वसा॒न्या॑य च चा वसा॒न्या॑य च चा\\
वसा॒न्या॑य च ।\\
\\
43. अ॒व॒सा॒न्या॑य । च॒ । नमः॑ ।\\
अ॒व॒सा॒न्या॑य च चा वसा॒न्या॑या वसा॒न्या॑य च॒ नमो॒ नम॑श्चा वसा॒न्या॑या\\
वसा॒न्या॑य च॒ नमः॑ ।\\
\\
44. अ॒व॒सा॒न्या॑य ।\\
अ॒व॒सा॒न्या॑येत्य॑व - सा॒न्या॑य ।\\
\\
45. च॒ । नमः॑ । वन्या॑य ।\\
च॒ नमो॒ नम॑श्च च॒ नमो॒ वन्या॑य॒ वन्या॑य॒ नम॑श्च च॒ नमो॒ वन्या॑य ।\\
\\
46. नमः॑ । वन्या॑य । च॒ ।\\
नमो॒ वन्या॑य॒ वन्या॑य॒ नमो॒ नमो॒ वन्या॑य च च॒ वन्या॑य॒ नमो॒ नमो॒\\
वन्या॑य च ।\\
\\
47. वन्या॑य । च॒ । कक्ष्या॑य ।\\
वन्या॑य च च॒ वन्या॑य॒ वन्या॑य च॒ कक्ष्या॑य॒ कक्ष्या॑य च॒ वन्या॑य॒ वन्या॑य\\
च॒ कक्ष्या॑य ।\\
\\
48. च॒ । कक्ष्या॑य । च॒ ।\\
च॒ कक्ष्या॑य॒ कक्ष्या॑य च च॒ कक्ष्या॑य च च॒ कक्ष्या॑य च च॒ कक्ष्या॑य च ।\\
\\
49. कक्ष्या॑य । च॒ । नमः॑ ।\\
कक्ष्या॑य च च॒ कक्ष्या॑य॒ कक्ष्या॑य च॒ नमो॒ नम॑श्च॒ कक्ष्या॑य॒ कक्ष्या॑य\\
च॒ नमः॑ ।\\
\\
50. च॒ । नमः॑ । श्र॒वाय॑ ।\\
च॒ नमो॒ नम॑श्च च॒ नमः॑ श्र॒वाय॑ श्र॒वाय॒ नम॑श्च च॒ नमः॑ श्र॒वाय॑ ।\\
\\
51. नमः॑ । श्र॒वाय॑ । च॒ ।\\
नमः॑ श्र॒वाय॑ श्र॒वाय॒ नमो॒ नमः॑ श्र॒वाय॑ च च श्र॒वाय॒ नमो॒ नमः॑\\
श्र॒वाय॑ च ।\\
\\
52. श्र॒वाय॑ । च॒ । प्र॒ति॒श्र॒वाय॑ ।\\
श्र॒वाय॑ च च श्र॒वाय॑ श्र॒वाय॑ च प्रतिश्र॒वाय॑ प्रतिश्र॒वाय॑ च श्र॒वाय॑\\
श्र॒वाय॑ च प्रतिश्र॒वाय॑ ।\\
\\
53. च॒ । प्र॒ति॒श्र॒वाय॑ । च॒ ।\\
च॒ प्र॒ति॒श्र॒वाय॑ प्रतिश्र॒वाय॑ च च प्रतिश्र॒वाय॑ च च प्रतिश्र॒वाय॑ च\\
च प्रतिश्र॒वाय॑ च ।\\
\\
54. प्र॒ति॒श्र॒वाय॑ । च॒ । नमः॑ ।\\
प्र॒ति॒श्र॒वाय॑ च च प्रतिश्र॒वाय॑ प्रतिश्र॒वाय॑ च॒ नमो॒ नम॑श्च प्रतिश्र॒वाय॑\\
प्रतिश्र॒वाय॑ च॒ नमः॑ ।\\
\\
55. प्र॒ति॒श्र॒वाय॑ ।\\
प्र॒ति॒श्र॒वायेति॑ प्रति - श्र॒वाय॑ ।\\
\\
56. च॒ । नमः॑ । आ॒शुषे॑णाय ।\\
च॒ नमो॒ नम॑श्च च॒ नम॑ आ॒शुषे॑णाया॒ शुषे॑णाय॒ नम॑श्च च॒ नम॑ आ॒शुषे॑णाय ।\\
\\
57. नमः॑ । आ॒शुषे॑णाय । च॒ ।\\
नम॑ आ॒शुषे॑णाया॒ शुषे॑णाय॒ नमो॒ नम॑ आ॒शुषे॑णाय च चा॒ शुषे॑णाय॒ नमो॒\\
नम॑ आ॒शुषे॑णाय च ।\\
\\
58. आ॒शुषे॑णाय । च॒ । आ॒शुर॑थाय ।\\
आ॒शुषे॑णाय च चा॒ शुषे॑णाया॒ शुषे॑णाय चा॒शुर॑थाया॒ शुर॑थाय चा॒शुषे॑णाया॒\\
शुषे॑णाय चा॒शुर॑थाय ।\\
\\
59. आ॒शुषे॑णाय ।\\
आ॒शुषे॑णा॒येत्या॒शु - से॒ना॒य॒ ।\\
\\
60. च॒ । आ॒शुर॑थाय । च॒ ।\\
चा॒शुर॑थाया॒ शुर॑थाय च चा॒शुर॑थाय च चा॒शुर॑थाय च चा॒शुर॑थाय च ।\\
\\
61. आ॒शुर॑थाय । च॒ । नमः॑ ।\\
आ॒शुर॑थाय च चा॒शुर॑थाया॒ शुर॑थाय च॒ नमो॒ नम॑श्चा॒शुर॑थाया॒\\
शुर॑थाय च॒ नमः॑ ।\\
\\
62. आ॒शुर॑थाय ।\\
आ॒शुर॑था॒येत्या॒शु - र॒था॒य॒ ।\\
\\
63. च॒ । नमः॑ । शूरा॑य ।\\
च॒ नमो॒ नम॑श्च च॒ नमः॒ शूरा॑य॒ शूरा॑य॒ नम॑श्च च॒ नमः॒ शूरा॑य ।\\
\\
64. नमः॑ । शूरा॑य । च॒ ।\\
नमः॒ शूरा॑य॒ शूरा॑य॒ नमो॒ नमः॒ शूरा॑य च च॒ शूरा॑य॒ नमो॒ नमः॒ शूरा॑य च ।\\
\\
65. शूरा॑य । च॒ । अ॒व॒भि॒न्द॒ते ।\\
शूरा॑य च च॒ शूरा॑य॒ शूरा॑य चावभिन्द॒ते॑ऽवभिन्द॒ते च॒ शूरा॑य॒ शूरा॑य\\
चावभिन्द॒ते ।\\
\\
66. च॒ । अ॒व॒भि॒न्द॒ते । च॒ ।\\
चा॒व॒भि॒न्द॒ते॑ऽवभिन्द॒ते च॑ चावभिन्द॒ते च॑ चावभिन्द॒ते च॑ चावभिन्द॒ते च॑ ।\\
\\
67. अ॒व॒भि॒न्द॒ते । च॒ । नमः॑ ।\\
अ॒व॒भि॒न्द॒ते च॑ चावभिन्द॒ते॑ऽवभिन्द॒ते च॒ नमो॒ नम॑श्चावभिन्द॒ते॑ऽवभिन्द॒ते च॒ नमः॑ ।\\
\\
68. अ॒व॒भि॒न्द॒ते ।\\
अ॒व॒भि॒न्द॒त इत्य॑व - भि॒न्द॒ते ।\\
\\
69. च॒ । नमः॑ । व॒र्मिणे᳚ ।\\
च॒ नमो॒ नम॑श्च च॒ नमो॑ व॒र्मिणे॑ व॒र्मिणे॒ नम॑श्च च॒ नमो॑ व॒र्मिणे᳚ ।\\
\\
70. नमः॑ । व॒र्मिणे᳚ । च॒ ।\\
नमो॑ व॒र्मिणे॑ व॒र्मिणे॒ नमो॒ नमो॑ व॒र्मिणे॑ च च व॒र्मिणे॒ नमो॒ नमो॑\\
व॒र्मिणे॑ च ।\\
\\
71. व॒र्मिणे᳚ । च॒ । व॒रू॒थिने᳚ ।\\
व॒र्मिणे॑ च च व॒र्मिणे॑ व॒र्मिणे॑ च वरू॒थिने॑ वरू॒थिने॑ च व॒र्मिणे॑\\
व॒र्मिणे॑ च वरू॒थिने᳚ ।\\
\\
72. च॒ । व॒रू॒थिने᳚ । च॒ ।\\
च॒ व॒रू॒थिने॑ वरू॒थिने॑ च च वरू॒थिने॑ च च वरू॒थिने॑ च च\\
वरू॒थिने॑ च ।\\
\\
73. व॒रू॒थिने᳚ । च॒ । नमः॑ ।\\
व॒रू॒थिने॑ च च वरू॒थिने॑ वरू॒थिने॑ च॒ नमो॒ नम॑श्च वरू॒थिने॑\\
वरू॒थिने॑ च॒ नमः॑ ।\\
\\
74. च॒ । नमः॑ । बि॒ल्मिने᳚ ।\\
च॒ नमो॒ नम॑श्च च॒ नमो॑ बि॒ल्मिने॑ बि॒ल्मिने॒ नम॑श्च च॒ नमो॑ बि॒ल्मिने᳚ ।\\
\\
75. नमः॑ । बि॒ल्मिने᳚ । च॒ ।\\
नमो॑ बि॒ल्मिने॑ बि॒ल्मिने॒ नमो॒ नमो॑ बि॒ल्मिने॑ च च बि॒ल्मिने॒ नमो॒ नमो॑\\
बि॒ल्मिने॑ च ।\\
\\
76. बि॒ल्मिने᳚ । च॒ । क॒व॒चिने᳚ ।\\
बि॒ल्मिने॑ च च बि॒ल्मिने॑ बि॒ल्मिने॑ च कव॒चिने॑ कव॒चिने॑ च बि॒ल्मिने॑\\
बि॒ल्मिने॑ च कव॒चिने᳚ ।\\
\\
77. च॒ । क॒व॒चिने᳚ । च॒ ।\\
च॒ क॒व॒चिने॑ कव॒चिने॑ च च कव॒चिने॑ च च कव॒चिने॑ च च\\
कव॒चिने॑ च ।\\
\\
78. क॒व॒चिने᳚ । च॒ । नमः॑ ।\\
क॒व॒चिने॑ च च कव॒चिने॑ कव॒चिने॑ च॒ नमो॒ नम॑श्च कव॒चिने॑\\
कव॒चिने॑ च॒ नमः॑ ।\\
\\
79. च॒ । नमः॑ । श्रु॒ताय॑ ।\\
च॒ नमो॒ नम॑श्च च॒ नमः॑ श्रु॒ताय॑ श्रु॒ताय॒ नम॑श्च च॒ नमः॑ श्रु॒ताय॑ ।\\
\\
80. नमः॑ । श्रु॒ताय॑ । च॒ ।\\
नमः॑ श्रु॒ताय॑ श्रु॒ताय॒ नमो॒ नमः॑ श्रु॒ताय॑ च च श्रु॒ताय॒ नमो॒ नमः॑\\
श्रु॒ताय॑ च ।\\
\\
81. श्रु॒ताय॑ । च॒ । श्रु॒त॒से॒नाय॑ ।\\
श्रु॒ताय॑ च च श्रु॒ताय॑ श्रु॒ताय॑ च श्रुतसे॒नाय॑ श्रुतसे॒नाय॑ च श्रु॒ताय॑\\
श्रु॒ताय॑ च श्रुतसे॒नाय॑ ।\\
\\
82. च॒ । श्रु॒त॒से॒नाय॑ । च॒ ॥\\
च॒ श्रु॒त॒से॒नाय॑ श्रुतसे॒नाय॑ च च श्रुतसे॒नाय॑ च च श्रुतसे॒नाय॑ च च\\
श्रुतसे॒नाय॑ च ।\\
\\
83. श्रु॒त॒से॒नाय॑ । च॒ ॥\\
श्रु॒त॒से॒नाय॑ च च श्रुतसे॒नाय॑ श्रुतसे॒नाय॑ च ।\\
\\
84. श्रु॒त॒से॒नाय॑ ।\\
श्रु॒त॒से॒नायेति॑ श्रुत - से॒नाय॑ ।\\
\\
85. च॒ ॥\\
चेति॑ च ।\\
\subsection{\eng{Anuvaka 7}}
1. नमः॑ । दु॒न्दु॒भ्या॑य । च॒ ।\\
नमो॑ दुन्दु॒भ्या॑य दुन्दु॒भ्या॑य॒ नमो॒ नमो॑ दुन्दु॒भ्या॑य च च दुन्दु॒भ्या॑य॒ नमो॒\\
नमो॑ दुन्दु॒भ्या॑य च ।\\
\\
2. दु॒न्दु॒भ्या॑य । च॒ । आ॒ह॒न॒न्या॑य ।\\
दु॒न्दु॒भ्या॑य च च दुन्दु॒भ्या॑य दुन्दु॒भ्या॑य चा हन॒न्या॑या हन॒न्या॑य च\\
दुन्दु॒भ्या॑य दुन्दु॒भ्या॑य चा हन॒न्या॑य ।\\
\\
3. च॒ । आ॒ह॒न॒न्या॑य । च॒ ।\\
चा॒ ह॒न॒न्या॑या हन॒न्या॑य च चा हन॒न्या॑य च चा हन॒न्या॑य च चा\\
हन॒न्या॑य च ।\\
\\
4. आ॒ह॒न॒न्या॑य । च॒ । नमः॑ ।\\
आ॒ह॒न॒न्या॑य च चा हन॒न्या॑या हन॒न्या॑य च॒ नमो॒ नम॑श्चा हन॒न्या॑या\\
हन॒न्या॑य च॒ नमः॑ ।\\
\\
5. आ॒ह॒न॒न्या॑य ।\\
आ॒ह॒न॒न्या॑येत्या᳚ - ह॒न॒न्या॑य ।\\
\\
6. च॒ । नमः॑ । धृ॒ष्णवे᳚ ।\\
च॒ नमो॒ नम॑श्च च॒ नमो॑ धृ॒ष्णवे॑ धृ॒ष्णवे॒ नम॑श्च च॒ नमो॑ धृ॒ष्णवे᳚ ।\\
\\
7. नमः॑ । धृ॒ष्णवे᳚ । च॒ ।\\
नमो॑ धृ॒ष्णवे॑ धृ॒ष्णवे॒ नमो॒ नमो॑ धृ॒ष्णवे॑ च च धृ॒ष्णवे॒ नमो॒ नमो॑\\
धृ॒ष्णवे॑ च ।\\
\\
8. धृ॒ष्णवे᳚ । च॒ । प्र॒मृ॒शाय॑ ।\\
धृ॒ष्णवे॑ च च धृ॒ष्णवे॑ धृ॒ष्णवे॑ च प्रमृ॒शाय॑ प्रमृ॒शाय॑ च धृ॒ष्णवे॑ धृ॒ष्णवे॑\\
च प्रमृ॒शाय॑ ।\\
\\
9. च॒ । प्र॒मृ॒शाय॑ । च॒ ।\\
च॒ प्र॒मृ॒शाय॑ प्रमृ॒शाय॑ च च प्रमृ॒शाय॑ च च प्रमृ॒शाय॑ च च प्रमृ॒शाय॑ च ।\\
\\
10. प्र॒मृ॒शाय॑ । च॒ । नमः॑ ।\\
प्र॒मृ॒शाय॑ च च प्रमृ॒शाय॑ प्रमृ॒शाय॑ च॒ नमो॒ नम॑श्च प्रमृ॒शाय॑ प्रमृ॒शाय॑\\
च॒ नमः॑ ।\\
\\
11. प्र॒मृ॒शाय॑ ।\\
प्र॒मृ॒शायेति॑ प्र - मृ॒शाय॑ ।\\
\\
12. च॒ । नमः॑ । दू॒ताय॑ ।\\
च॒ नमो॒ नम॑श्च च॒ नमो॑ दू॒ताय॑ दू॒ताय॒ नम॑श्च च॒ नमो॑ दू॒ताय॑ ।\\
\\
13. नमः॑ । दू॒ताय॑ । च॒ ।\\
नमो॑ दू॒ताय॑ दू॒ताय॒ नमो॒ नमो॑ दू॒ताय॑ च च दू॒ताय॒ नमो॒ नमो॑ दू॒ताय॑ च ।\\
\\
14. दू॒ताय॑ । च॒ । प्रहि॑ताय ।\\
दू॒ताय॑ च च दू॒ताय॑ दू॒ताय॑ च॒ प्रहि॑ताय॒ प्रहि॑ताय च दू॒ताय॑ दू॒ताय॑ च॒\\
प्रहि॑ताय ।\\
\\
15. च॒ । प्रहि॑ताय । च॒ ।\\
च॒ प्रहि॑ताय॒ प्रहि॑ताय च च॒ प्रहि॑ताय च च॒ प्रहि॑ताय च च॒ प्रहि॑ताय च ।\\
\\
16. प्रहि॑ताय । च॒ । नमः॑ ।\\
प्रहि॑ताय च च॒ प्रहि॑ताय॒ प्रहि॑ताय च॒ नमो॒ नम॑श्च॒ प्रहि॑ताय॒\\
प्रहि॑ताय च॒ नमः॑ ।\\
\\
17. प्रहि॑ताय ।\\
प्रहि॑ता॒येति॒ प्र - हि॒ता॒य॒ ।\\
\\
18. च॒ । नमः॑ । नि॒ष॒ङ्गिणे᳚ ।\\
च॒ नमो॒ नम॑श्च च॒ नमो॑ निष॒ङ्गिणे॑ निष॒ङ्गिणे॒ नम॑श्च च॒ नमो॑ निष॒ङ्गिणे᳚ ।\\
\\
19. नमः॑ । नि॒ष॒ङ्गिणे᳚ । च॒ ।\\
नमो॑ निष॒ङ्गिणे॑ निष॒ङ्गिणे॒ नमो॒ नमो॑ निष॒ङ्गिणे॑ च च निष॒ङ्गिणे॒ नमो॒ नमो॑\\
निष॒ङ्गिणे॑ च ।\\
\\
20. नि॒ष॒ङ्गिणे᳚ । च॒ । इ॒षु॒धि॒मते᳚ ।\\
नि॒ष॒ङ्गिणे॑ च च निष॒ङ्गिणे॑ निष॒ङ्गिणे॑ चेषुधि॒मत॑ इषुधि॒मते॑ च निष॒ङ्गिणे॑\\
निष॒ङ्गिणे॑ चेषुधि॒मते᳚ ।\\
\\
21. नि॒ष॒ङ्गिणे᳚ ।\\
नि॒ष॒ङ्गिण॒ इति॑ नि - स॒ङ्गिने᳚ ।\\
\\
22. च॒ । इ॒षु॒धि॒मते᳚ । च॒ ।\\
चे॒षु॒धि॒मत॑ इषुधि॒मते॑ च चेषुधि॒मते॑ च चेषुधि॒मते॑ च चेषुधि॒मते॑ च ।\\
\\
23. इ॒षु॒धि॒मते᳚ । च॒ । नमः॑ ।\\
इ॒षु॒धि॒मते॑ च चेषुधि॒मत॑ इषुधि॒मते॑ च॒ नमो॒ नम॑ श्चेषुधि॒मत॑ इषुधि॒मते॑\\
च॒ नमः॑ ।\\
\\
24. इ॒षु॒धि॒मते᳚ ।\\
इ॒षु॒धि॒मत॒ इती॑षुधि - मते᳚ ।\\
\\
25. च॒ । नमः॑ । ती॒क्ष्णेष॑वे ।\\
च॒ नमो॒ नम॑श्च च॒ नम॑ स्ती॒क्ष्णेष॑वे ती॒क्ष्णेष॑वे॒ नम॑श्च च॒ नम॑ स्ती॒क्ष्णेष॑वे ।\\
\\
26. नमः॑ । ती॒क्ष्णेष॑वे । च॒ ।\\
नम॑ स्ती॒क्ष्णेष॑वे ती॒क्ष्णेष॑वे॒ नमो॒ नम॑ स्ती॒क्ष्णेष॑वे च च ती॒क्ष्णेष॑वे॒ नमो॒\\
नम॑ स्ती॒क्ष्णेष॑वे च ।\\
\\
27. ती॒क्ष्णेष॑वे । च॒ । आ॒यु॒धिने᳚ ।\\
ती॒क्ष्णेष॑वे च च ती॒क्ष्णेष॑वे ती॒क्ष्णेष॑वे चा यु॒धिन॑ आयु॒धिने॑ च ती॒क्ष्णेष॑वे\\
ती॒क्ष्णेष॑वे चा यु॒धिने᳚ ।\\
\\
28. ती॒क्ष्णेष॑वे ।\\
ती॒क्ष्णेष॑व॒ इति॑ ती॒क्ष्ण - इ॒ष॒वे॒ ।\\
\\
29. च॒ । आ॒यु॒धिने᳚ । च॒ ।\\
चा॒ यु॒धिन॑ आयु॒धिने॑ च चा यु॒धिने॑ च चा यु॒धिने॑ च चा यु॒धिने॑ च ।\\
\\
30. आ॒यु॒धिने᳚ । च॒ । नमः॑ ।\\
आ॒यु॒धिने॑ च चा यु॒धिन॑ आयु॒धिने॑ च॒ नमो॒ नम॑श्चा यु॒धिन॑\\
आयु॒धिने॑ च॒ नमः॑ ।\\
\\
31. च॒ । नमः॑ । स्वा॒यु॒धाय॑ ।\\
च॒ नमो॒ नम॑श्च च॒ नमः॑ स्वायु॒धाय॑ स्वायु॒धाय॒ नम॑श्च च॒ नमः॑ स्वायु॒धाय॑ ।\\
\\
32. नमः॑ । स्वा॒यु॒धाय॑ । च॒ ।\\
नमः॑ स्वायु॒धाय॑ स्वायु॒धाय॒ नमो॒ नमः॑ स्वायु॒धाय॑ च च स्वायु॒धाय॒ नमो॒\\
नमः॑ स्वायु॒धाय॑ च ।\\
\\
33. स्वा॒यु॒धाय॑ । च॒ । सु॒धन्व॑ने ।\\
स्वा॒यु॒धाय॑ च च स्वायु॒धाय॑ स्वायु॒धाय॑ च सु॒धन्व॑ने सु॒धन्व॑ने च\\
स्वायु॒धाय॑ स्वायु॒धाय॑ च सु॒धन्व॑ने ।\\
\\
34. स्वा॒यु॒धाय॑ ।\\
स्वा॒यु॒धायेति॑ सु - आ॒यु॒धाय॑ ।\\
\\
35. च॒ । सु॒धन्व॑ने । च॒ ।\\
च॒ सु॒धन्व॑ने सु॒धन्व॑ने च च सु॒धन्व॑ने च च सु॒धन्व॑ने च च\\
सु॒धन्व॑ने च ।\\
\\
36. सु॒धन्व॑ने । च॒ । नमः॑ ।\\
सु॒धन्व॑ने च च सु॒धन्व॑ने सु॒धन्व॑ने च॒ नमो॒ नम॑श्च सु॒धन्व॑ने\\
सु॒धन्व॑ने च॒ नमः॑ ।\\
\\
37. सु॒धन्व॑ने ।\\
सु॒धन्व॑न॒ इति॑ सु - धन्व॑ने ।\\
\\
38. च॒ । नमः॑ । स्रुत्या॑य ।\\
च॒ नमो॒ नम॑श्च च॒ नमः॒ स्रुत्या॑य॒ स्रुत्या॑य॒ नम॑श्च च॒ नमः॒ स्रुत्या॑य ।\\
\\
39. नमः॑ । स्रुत्या॑य । च॒ ।\\
नमः॒ स्रुत्या॑य॒ स्रुत्या॑य॒ नमो॒ नमः॒ स्रुत्या॑य च च॒ स्रुत्या॑य॒ नमो॒ नमः॒\\
स्रुत्या॑य च ।\\
\\
40. स्रुत्या॑य । च॒ । पथ्या॑य ।\\
स्रुत्या॑य च च॒ स्रुत्या॑य॒ स्रुत्या॑य च॒ पथ्या॑य॒ पथ्या॑य च॒ स्रुत्या॑य॒\\
स्रुत्या॑य च॒ पथ्या॑य ।\\
\\
41. च॒ । पथ्या॑य । च॒ ।\\
च॒ पथ्या॑य॒ पथ्या॑य च च॒ पथ्या॑य च च॒ पथ्या॑य च च॒ पथ्या॑य च ।\\
\\
42. पथ्या॑य । च॒ । नमः॑ ।\\
पथ्या॑य च च॒ पथ्या॑य॒ पथ्या॑य च॒ नमो॒ नम॑श्च॒ पथ्या॑य॒ पथ्या॑य च॒ नमः॑ ।\\
\\
43. च॒ । नमः॑ । का॒ट्या॑य ।\\
च॒ नमो॒ नम॑श्च च॒ नमः॑ का॒ट्या॑य का॒ट्या॑य॒ नम॑श्च च॒ नमः॑ का॒ट्या॑य ।\\
\\
44. नमः॑ । का॒ट्या॑य । च॒ ।\\
नमः॑ का॒ट्या॑य का॒ट्या॑य॒ नमो॒ नमः॑ का॒ट्या॑य च च का॒ट्या॑य॒ नमो॒ नमः॑\\
का॒ट्या॑य च ।\\
\\
45. का॒ट्या॑य । च॒ । नी॒प्या॑य ।\\
का॒ट्या॑य च च का॒ट्या॑य का॒ट्या॑य च नी॒प्या॑य नी॒प्या॑य च का॒ट्या॑य\\
का॒ट्या॑य च नी॒प्या॑य ।\\
\\
46. च॒ । नी॒प्या॑य । च॒ ।\\
च॒ नी॒प्या॑य नी॒प्या॑य च च नी॒प्या॑य च च नी॒प्या॑य च च नी॒प्या॑य च ।\\
\\
47. नी॒प्या॑य । च॒ । नमः॑ ।\\
नी॒प्या॑य च च नी॒प्या॑य नी॒प्या॑य च॒ नमो॒ नम॑श्च नी॒प्या॑य\\
नी॒प्या॑य च॒ नमः॑ ।\\
\\
48. च॒ । नमः॑ । सूद्या॑य ।\\
च॒ नमो॒ नम॑श्च च॒ नमः॒ सूद्या॑य॒ सूद्या॑य॒ नम॑श्च च॒ नमः॒ सूद्या॑य ।\\
\\
49. नमः॑ । सूद्या॑य । च॒ ।\\
नमः॒ सूद्या॑य॒ सूद्या॑य॒ नमो॒ नमः॒ सूद्या॑य च च॒ सूद्या॑य॒ नमो॒ नमः॒\\
सूद्या॑य च ।\\
\\
50. सूद्या॑य । च॒ । स॒र॒स्या॑य ।\\
सूद्या॑य च च॒ सूद्या॑य॒ सूद्या॑य च सर॒स्या॑य सर॒स्या॑य च॒ सूद्या॑य॒ सूद्या॑य\\
च सर॒स्या॑य ।\\
\\
51. च॒ । स॒र॒स्या॑य । च॒ ।\\
च॒ स॒र॒स्या॑य सर॒स्या॑य च च सर॒स्या॑य च च सर॒स्या॑य च च\\
सर॒स्या॑य च ।\\
\\
52. स॒र॒स्या॑य । च॒ । नमः॑ ।\\
स॒र॒स्या॑य च च सर॒स्या॑य सर॒स्या॑य च॒ नमो॒ नम॑श्च सर॒स्या॑य सर॒स्या॑य\\
च॒ नमः॑ ।\\
\\
53. च॒ । नमः॑ । ना॒द्याय॑ ।\\
च॒ नमो॒ नम॑श्च च॒ नमो॑ ना॒द्याय॑ ना॒द्याय॒ नम॑श्च च॒ नमो॑ ना॒द्याय॑ ।\\
\\
54. नमः॑ । ना॒द्याय॑ । च॒ ।\\
नमो॑ ना॒द्याय॑ ना॒द्याय॒ नमो॒ नमो॑ ना॒द्याय॑ च च ना॒द्याय॒ नमो॒ नमो॑\\
ना॒द्याय॑ च ।\\
\\
55. ना॒द्याय॑ । च॒ । वै॒श॒न्ताय॑ ।\\
ना॒द्याय॑ च च ना॒द्याय॑ ना॒द्याय॑ च वैश॒न्ताय॑ वैश॒न्ताय॑ च ना॒द्याय॑\\
ना॒द्याय॑ च वैश॒न्ताय॑ ।\\
\\
56. च॒ । वै॒श॒न्ताय॑ । च॒ ।\\
च॒ वै॒श॒न्ताय॑ वैश॒न्ताय॑ च च वैश॒न्ताय॑ च च वैश॒न्ताय॑ च च\\
वैश॒न्ताय॑ च ।\\
\\
57. वै॒श॒न्ताय॑ । च॒ । नमः॑ ।\\
वै॒श॒न्ताय॑ च च वैश॒न्ताय॑ वैश॒न्ताय॑ च॒ नमो॒ नम॑श्च वैश॒न्ताय॑\\
वैश॒न्ताय॑ च॒ नमः॑ ।\\
\\
58. च॒ । नमः॑ । कूप्या॑य ।\\
च॒ नमो॒ नम॑श्च च॒ नमः॒ कूप्या॑य॒ कूप्या॑य॒ नम॑श्च च॒ नमः॒ कूप्या॑य ।\\
\\
59. नमः॑ । कूप्या॑य । च॒ ।\\
नमः॒ कूप्या॑य॒ कूप्या॑य॒ नमो॒ नमः॒ कूप्या॑य च च॒ कूप्या॑य॒ नमो॒ नमः॒\\
कूप्या॑य च ।\\
\\
60. कूप्या॑य । च॒ । अ॒व॒ट्या॑य ।\\
कूप्या॑य च च॒ कूप्या॑य॒ कूप्या॑य चा व॒ट्या॑या व॒ट्या॑य च॒ कूप्या॑य॒ कूप्या॑य\\
चा व॒ट्या॑य ।\\
\\
61. च॒ । अ॒व॒ट्या॑य । च॒ ।\\
चा॒ व॒ट्या॑या व॒ट्या॑य च चा व॒ट्या॑य च चा व॒ट्या॑य च चा व॒ट्या॑य च ।\\
\\
62. अ॒व॒ट्या॑य । च॒ । नमः॑ ।\\
अ॒व॒ट्या॑य च चा व॒ट्या॑या व॒ट्या॑य च॒ नमो॒ नम॑श्चा व॒ट्या॑या व॒ट्या॑य\\
च॒ नमः॑ ।\\
\\
63. च॒ । नमः॑ । वर्ष्या॑य ।\\
च॒ नमो॒ नम॑श्च च॒ नमो॒ वर्ष्या॑य॒ वर्ष्या॑य॒ नम॑श्च च॒ नमो॒ वर्ष्या॑य ।\\
\\
64. नमः॑ । वर्ष्या॑य । च॒ ।\\
नमो॒ वर्ष्या॑य॒ वर्ष्या॑य॒ नमो॒ नमो॒ वर्ष्या॑य च च॒ वर्ष्या॑य॒ नमो॒\\
नमो॒ वर्ष्या॑य च ।\\
\\
65. वर्ष्या॑य । च॒ । अ॒व॒र्ष्याय॑ ।\\
वर्ष्या॑य च च॒ वर्ष्या॑य॒ वर्ष्या॑य चा व॒र्ष्याया॑ व॒र्ष्याय॑ च॒ वर्ष्या॑य॒\\
वर्ष्या॑य चा व॒र्ष्याय॑ ।\\
\\
66. च॒ । अ॒व॒र्ष्याय॑ । च॒ ।\\
चा॒ व॒र्ष्याया॑ व॒र्ष्याय॑ च चा व॒र्ष्याय॑ च चा व॒र्ष्याय॑ च चा\\
व॒र्ष्याय॑ च ।\\
\\
67. अ॒व॒र्ष्याय॑ । च॒ । नमः॑ ।\\
अ॒व॒र्ष्याय॑ च चा व॒र्ष्याया॑ व॒र्ष्याय॑ च॒ नमो॒ नम॑श्चा व॒र्ष्याया॑ व॒र्ष्याय॑\\
च॒ नमः॑ ।\\
\\
68. च॒ । नमः॑ । मे॒घ्या॑य ।\\
च॒ नमो॒ नम॑श्च च॒ नमो॑ मे॒घ्या॑य मे॒घ्या॑य॒ नम॑श्च च॒ नमो॑ मे॒घ्या॑य ।\\
\\
69. नमः॑ । मे॒घ्या॑य । च॒ ।\\
नमो॑ मे॒घ्या॑य मे॒घ्या॑य॒ नमो॒ नमो॑ मे॒घ्या॑य च च मे॒घ्या॑य॒ नमो॒ नमो॑\\
मे॒घ्या॑य च ।\\
\\
70. मे॒घ्या॑य । च॒ । वि॒द्यु॒त्या॑य ।\\
मे॒घ्या॑य च च मे॒घ्या॑य मे॒घ्या॑य च विद्यु॒त्या॑य विद्यु॒त्या॑य च मे॒घ्या॑य\\
मे॒घ्या॑य च विद्यु॒त्या॑य ।\\
\\
71. च॒ । वि॒द्यु॒त्या॑य । च॒ ।\\
च॒ वि॒द्यु॒त्या॑य विद्यु॒त्या॑य च च विद्यु॒त्या॑य च च विद्यु॒त्या॑य च च\\
विद्यु॒त्या॑य च ।\\
\\
72. वि॒द्यु॒त्या॑य । च॒ । नमः॑ ।\\
वि॒द्यु॒त्या॑य च च विद्यु॒त्या॑य विद्यु॒त्या॑य च॒ नमो॒ नम॑श्च विद्यु॒त्या॑य\\
विद्यु॒त्या॑य च॒ नमः॑ ।\\
\\
73. वि॒द्यु॒त्या॑य ।\\
वि॒द्यु॒त्या॑येति॑ वि - द्यु॒त्या॑य ।\\
\\
74. च॒ । नमः॑ । ई॒द्ध्रिया॑य ।\\
च॒ नमो॒ नम॑श्च च॒ नम॑ ई॒द्ध्रिया॑ ये॒द्ध्रिया॑य॒ नम॑श्च च॒ नम॑ ई॒द्ध्रिया॑य ।\\
\\
75. नमः॑ । ई॒द्ध्रिया॑य । च॒ ।\\
नम॑ ई॒द्ध्रिया॑ ये॒द्ध्रिया॑य॒ नमो॒ नम॑ ई॒द्ध्रिया॑य च चे॒द्ध्रिया॑य॒ नमो॒ नम॑\\
ई॒द्ध्रिया॑य च ।\\
\\
76. ई॒द्ध्रिया॑य । च॒ । आ॒त॒प्या॑य ।\\
ई॒द्ध्रिया॑य च चे॒द्ध्रिया॑ ये॒द्ध्रिया॑य चा त॒प्या॑या त॒प्या॑य चे॒द्ध्रिया॑\\
ये॒द्ध्रिया॑य चा त॒प्या॑य ।\\
\\
77. च॒ । आ॒त॒प्या॑य । च॒ ।\\
चा॒ त॒प्या॑या त॒प्या॑य च चा त॒प्या॑य च चा त॒प्या॑य च चा त॒प्या॑य च ।\\
\\
78. आ॒त॒प्या॑य । च॒ । नमः॑ ।\\
आ॒त॒प्या॑य च चा त॒प्या॑या त॒प्या॑य च॒ नमो॒ नम॑श्चा त॒प्या॑या\\
त॒प्या॑य च॒ नमः॑ ।\\
\\
79. आ॒त॒प्या॑य ।\\
आ॒त॒प्या॑येत्या᳚ - त॒प्या॑य ।\\
\\
80. च॒ । नमः॑ । वात्या॑य ।\\
च॒ नमो॒ नम॑श्च च॒ नमो॒ वात्या॑य॒ वात्या॑य॒ नम॑श्च च॒ नमो॒ वात्या॑य ।\\
\\
81. नमः॑ । वात्या॑य । च॒ ।\\
नमो॒ वात्या॑य॒ वात्या॑य॒ नमो॒ नमो॒ वात्या॑य च च॒ वात्या॑य॒ नमो॒ नमो॒\\
वात्या॑य च ।\\
\\
82. वात्या॑य । च॒ । रेष्मि॑याय ।\\
वात्या॑य च च॒ वात्या॑य॒ वात्या॑य च॒ रेष्मि॑याय॒ रेष्मि॑याय च॒ वात्या॑य॒ वात्या॑य\\
च॒ रेष्मि॑याय ।\\
\\
83. च॒ । रेष्मि॑याय । च॒ ।\\
च॒ रेष्मि॑याय॒ रेष्मि॑याय च च॒ रेष्मि॑याय च च॒ रेष्मि॑याय च च॒\\
रेष्मि॑याय च ।\\
\\
84. रेष्मि॑याय । च॒ । नमः॑ ।\\
रेष्मि॑याय च च॒ रेष्मि॑याय॒ रेष्मि॑याय च॒ नमो॒ नम॑श्च॒ रेष्मि॑याय॒\\
रेष्मि॑याय च॒ नमः॑ ।\\
\\
85. च॒ । नमः॑ । वा॒स्त॒व्या॑य ।\\
च॒ नमो॒ नम॑श्च च॒ नमो॑ वास्त॒व्या॑य वास्त॒व्या॑य॒ नम॑श्च च॒ नमो॑\\
वास्त॒व्या॑य ।\\
\\
86. नमः॑ । वा॒स्त॒व्या॑य । च॒ ।\\
नमो॑ वास्त॒व्या॑य वास्त॒व्या॑य॒ नमो॒ नमो॑ वास्त॒व्या॑य च च वास्त॒व्या॑य॒\\
नमो॒ नमो॑ वास्त॒व्या॑य च ।\\
\\
87. वा॒स्त॒व्या॑य । च॒ । वा॒स्तु॒पाय॑ ।\\
वा॒स्त॒व्या॑य च च वास्त॒व्या॑य वास्त॒व्या॑य च वास्तु॒पाय॑ वास्तु॒पाय॑ च\\
वास्त॒व्या॑य वास्त॒व्या॑य च वास्तु॒पाय॑ ।\\
\\
88. च॒ । वा॒स्तु॒पाय॑ । च॒ ॥\\
च॒ वा॒स्तु॒पाय॑ वास्तु॒पाय॑ च च वास्तु॒पाय॑ च च वास्तु॒पाय॑ च\\
च वास्तु॒पाय॑ च ।\\
\\
89. वा॒स्तु॒पाय॑ । च॒ ॥\\
वा॒स्तु॒पाय॑ च च वास्तु॒पाय॑ वास्तु॒पाय॑ च ।\\
\\
90. वा॒स्तु॒पाय॑ ।\\
वा॒स्तु॒पायेति॑ वास्तु - पाय॑ ।\\
\\
91. च॒ ॥\\
चेति॑ च ।\\
\subsection{\eng{Anuvaka 8}}
1. नमः॑ । सोमा॑य । च॒ ।\\
नमः॒ सोमा॑य॒ सोमा॑य॒ नमो॒ नमः॒ सोमा॑य च च॒ सोमा॑य॒ नमो॒ नमः॒\\
सोमा॑य च ।\\
\\
2. सोमा॑य । च॒ । रु॒द्राय॑ ।\\
सोमा॑य च च॒ सोमा॑य॒ सोमा॑य च रु॒द्राय॑ रु॒द्राय॑ च॒ सोमा॑य॒ सोमा॑य\\
च रु॒द्राय॑ ।\\
\\
3. च॒ । रु॒द्राय॑ । च॒ ।\\
च॒ रु॒द्राय॑ रु॒द्राय॑ च च रु॒द्राय॑ च च रु॒द्राय॑ च च रु॒द्राय॑ च ।\\
\\
4. रु॒द्राय॑ । च॒ । नमः॑ ।\\
रु॒द्राय॑ च च रु॒द्राय॑ रु॒द्राय॑ च॒ नमो॒ नम॑श्च रु॒द्राय॑ रु॒द्राय॑ च॒ नमः॑ ।\\
\\
5. च॒ । नमः॑ । ता॒म्राय॑ ।\\
च॒ नमो॒ नम॑श्च च॒ नम॑ स्ता॒म्राय॑ ता॒म्राय॒ नम॑श्च च॒ नम॑ स्ता॒म्राय॑ ।\\
\\
6. नमः॑ । ता॒म्राय॑ । च॒ ।\\
नम॑ स्ता॒म्राय॑ ता॒म्राय॒ नमो॒ नम॑ स्ता॒म्राय॑ च च ता॒म्राय॒ नमो॒ नम॑ स्ता॒म्राय॑ च ।\\
\\
7. ता॒म्राय॑ । च॒ । अ॒रु॒णाय॑ ।\\
ता॒म्राय॑ च च ता॒म्राय॑ ता॒म्राय॑ चा रु॒णाया॑ रु॒णाय॑ च ता॒म्राय॑ ता॒म्राय॑\\
चा रु॒णाय॑ ।\\
\\
8. च॒ । अ॒रु॒णाय॑ । च॒ ।\\
चा॒ रु॒णाया॑ रु॒णाय॑ च चा रु॒णाय॑ च चा रु॒णाय॑ च चा रु॒णाय॑ च ।\\
\\
9. अ॒रु॒णाय॑ । च॒ । नमः॑ ।\\
अ॒रु॒णाय॑ च चा रु॒णाया॑ रु॒णाय॑ च॒ नमो॒ नम॑श्चा रु॒णाया॑ रु॒णाय॑ च॒ नमः॑ ।\\
\\
10. च॒ । नमः॑ । श॒ङ्गाय॑ ।\\
च॒ नमो॒ नम॑श्च च॒ नमः॑ श॒ङ्गाय॑ श॒ङ्गाय॒ नम॑श्च च॒ नमः॑ श॒ङ्गाय॑ ।\\
\\
11. नमः॑ । श॒ङ्गाय॑ । च॒ ।\\
नमः॑ श॒ङ्गाय॑ श॒ङ्गाय॒ नमो॒ नमः॑ श॒ङ्गाय॑ च च श॒ङ्गाय॒ नमो॒ नमः॑\\
श॒ङ्गाय॑ च ।\\
\\
12. श॒ङ्गाय॑ । च॒ । प॒शु॒पत॑ये ।\\
श॒ङ्गाय॑ च च श॒ङ्गाय॑ श॒ङ्गाय॑ च पशु॒पत॑ये पशु॒पत॑ये च श॒ङ्गाय॑ श॒ङ्गाय॑ च\\
पशु॒पत॑ये ।\\
\\
13. च॒ । प॒शु॒पत॑ये । च॒ ।\\
च॒ प॒शु॒पत॑ये पशु॒पत॑ये च च पशु॒पत॑ये च च पशु॒पत॑ये च च\\
पशु॒पत॑ये च ।\\
\\
14. प॒शु॒पत॑ये । च॒ । नमः॑ ।\\
प॒शु॒पत॑ये च च पशु॒पत॑ये पशु॒पत॑ये च॒ नमो॒ नम॑श्च पशु॒पत॑ये पशु॒पत॑ये\\
च॒ नमः॑ ।\\
\\
15. प॒शु॒पत॑ये ।\\
प॒शु॒पत॑य॒ इति॑ पशु - पत॑ये ।\\
\\
16. च॒ । नमः॑ । उ॒ग्राय॑ ।\\
च॒ नमो॒ नम॑श्च च॒ नम॑ उ॒ग्रा यो॒ग्राय॒ नम॑श्च च॒ नम॑ उ॒ग्राय॑ ।\\
\\
17. नमः॑ । उ॒ग्राय॑ । च॒ ।\\
नम॑ उ॒ग्रा यो॒ग्राय॒ नमो॒ नम॑ उ॒ग्राय॑ च चो॒ग्राय॒ नमो॒ नम॑ उ॒ग्राय॑ च ।\\
\\
18. उ॒ग्राय॑ । च॒ । भी॒माय॑ ।\\
उ॒ग्राय॑ च चो॒ग्रा यो॒ग्राय॑ च भी॒माय॑ भी॒माय॑ चो॒ग्रा यो॒ग्राय॑ च भी॒माय॑ ।\\
\\
19. च॒ । भी॒माय॑ । च॒ ।\\
च॒ भी॒माय॑ भी॒माय॑ च च भी॒माय॑ च च भी॒माय॑ च च भी॒माय॑ च ।\\
\\
20. भी॒माय॑ । च॒ । नमः॑ ।\\
भी॒माय॑ च च भी॒माय॑ भी॒माय॑ च॒ नमो॒ नम॑श्च भी॒माय॑ भी॒माय॑ च॒ नमः॑ ।\\
\\
21. च॒ । नमः॑ । अ॒ग्रे॒व॒धाय॑ ।\\
च॒ नमो॒ नम॑श्च च॒ नमो॑ अग्रेव॒धाया᳚ ग्रेव॒धाय॒ नम॑श्च च॒ नमो॑ अग्रेव॒धाय॑ ।\\
\\
22. नमः॑ । अ॒ग्रे॒व॒धाय॑ । च॒ ।\\
नमो॑ अग्रेव॒धाया᳚ ग्रेव॒धाय॒ नमो॒ नमो॑ अग्रेव॒धाय॑ च चा ग्रेव॒धाय॒ नमो॒\\
नमो॑ अग्रेव॒धाय॑ च ।\\
\\
23. अ॒ग्रे॒व॒धाय॑ । च॒ । दू॒रे॒व॒धाय॑ ।\\
अ॒ग्रे॒व॒धाय॑ च चा ग्रेव॒धाया᳚ ग्रेव॒धाय॑ च दूरेव॒धाय॑ दूरेव॒धाय॑ चा ग्रेव॒धाया᳚\\
ग्रेव॒धाय॑ च दूरेव॒धाय॑ ।\\
\\
24. अ॒ग्रे॒व॒धाय॑ ।\\
अ॒ग्रे॒व॒धायेत्य॑ग्रे - व॒धाय॑ ।\\
\\
25. च॒ । दू॒रे॒व॒धाय॑ । च॒ ।\\
च॒ दू॒रे॒व॒धाय॑ दूरेव॒धाय॑ च च दूरेव॒धाय॑ च च दूरेव॒धाय॑ च च\\
दूरेव॒धाय॑ च ।\\
\\
26. दू॒रे॒व॒धाय॑ । च॒ । नमः॑ ।\\
दू॒रे॒व॒धाय॑ च च दूरेव॒धाय॑ दूरेव॒धाय॑ च॒ नमो॒ नम॑श्च दूरेव॒धाय॑\\
दूरेव॒धाय॑ च॒ नमः॑ ।\\
\\
27. दू॒रे॒व॒धाय॑ ।\\
दू॒रे॒व॒धायेति॑ दूरे - व॒धाय॑ ।\\
\\
28. च॒ । नमः॑ । ह॒न्त्रे ।\\
च॒ नमो॒ नम॑श्च च॒ नमो॑ ह॒न्त्रे ह॒न्त्रे नम॑श्च च॒ नमो॑ ह॒न्त्रे ।\\
\\
29. नमः॑ । ह॒न्त्रे । च॒ ।\\
नमो॑ ह॒न्त्रे ह॒न्त्रे नमो॒ नमो॑ ह॒न्त्रे च॑ च ह॒न्त्रे नमो॒ नमो॑ ह॒न्त्रे च॑ ।\\
\\
30. ह॒न्त्रे । च॒ । हनी॑यसे ।\\
ह॒न्त्रे च॑ च ह॒न्त्रे ह॒न्त्रे च॒ हनी॑यसे॒ हनी॑यसे च ह॒न्त्रे ह॒न्त्रे च॒ हनी॑यसे ।\\
\\
31. च॒ । हनी॑यसे । च॒ ।\\
च॒ हनी॑यसे॒ हनी॑यसे च च॒ हनी॑यसे च च॒ हनी॑यसे च च॒ हनी॑यसे च ।\\
\\
32. हनी॑यसे । च॒ । नमः॑ ।\\
हनी॑यसे च च॒ हनी॑यसे॒ हनी॑यसे च॒ नमो॒ नम॑श्च॒ हनी॑यसे॒ हनी॑यसे\\
च॒ नमः॑ ।\\
\\
33. च॒ । नमः॑ । वृ॒क्षेभ्यः॑ ।\\
च॒ नमो॒ नम॑श्च च॒ नमो॑ वृ॒क्षेभ्यो॑ वृ॒क्षेभ्यो॒ नम॑श्च च॒ नमो॑ वृ॒क्षेभ्यः॑ ।\\
\\
34. नमः॑ । वृ॒क्षेभ्यः॑ । हरि॑केशेभ्यः ।\\
नमो॑ वृ॒क्षेभ्यो॑ वृ॒क्षेभ्यो॒ नमो॒ नमो॑ वृ॒क्षेभ्यो॒ हरि॑केशेभ्यो॒ हरि॑केशेभ्यो\\
वृ॒क्षेभ्यो॒ नमो॒ नमो॑ वृ॒क्षेभ्यो॒ हरि॑केशेभ्यः ।\\
\\
35. वृ॒क्षेभ्यः॑ । हरि॑केशेभ्यः । नमः॑ ।\\
वृ॒क्षेभ्यो॒ हरि॑केशेभ्यो॒ हरि॑केशेभ्यो वृ॒क्षेभ्यो॑ वृ॒क्षेभ्यो॒ हरि॑केशेभ्यो॒\\
नमो॒ नमो॒ हरि॑केशेभ्यो वृ॒क्षेभ्यो॑ वृ॒क्षेभ्यो॒ हरि॑केशेभ्यो॒ नमः॑ ।\\
\\
36. हरि॑केशेभ्यः । नमः॑ । ता॒राय॑ ।\\
हरि॑केशेभ्यो॒ नमो॒ नमो॒ हरि॑केशेभ्यो॒ हरि॑केशेभ्यो॒ नम॑ स्ता॒राय॑ ता॒राय॒ नमो॒\\
हरि॑केशेभ्यो॒ हरि॑केशेभ्यो॒ नम॑ स्ता॒राय॑ ।\\
\\
37. हरि॑केशेभ्यः ।\\
हरि॑केशेभ्य॒ इति॒ हरि॑ - के॒शे॒भ्यः॒ ।\\
\\
38. नमः॑ । ता॒राय॑ । नमः॑ ।\\
नम॑ स्ता॒राय॑ ता॒राय॒ नमो॒ नम॑ स्ता॒राय॒ नमो॒ नम॑ स्ता॒राय॒ नमो॒ नम॑\\
स्ता॒राय॒ नमः॑ ।\\
\\
39. ता॒राय॑ । नमः॑ । शं॒भवे᳚ ।\\
ता॒राय॒ नमो॒ नम॑ स्ता॒राय॑ ता॒राय॒ नमः॑ शं॒भवे॑ शं॒भवे॒ नम॑ स्ता॒राय॑ ता॒राय॒\\
नमः॑ शं॒भवे᳚ ।\\
\\
40. नमः॑ । शं॒भवे᳚ । च॒ ।\\
नमः॑ शं॒भवे॑ शं॒भवे॒ नमो॒ नमः॑ शं॒भवे॑ च च शं॒भवे॒ नमो॒ नमः॑\\
शं॒भवे॑ च ।\\
\\
41. शं॒भवे᳚ । च॒ । म॒यो॒भवे᳚ ।\\
शं॒भवे॑ च च शं॒भवे॑ शं॒भवे॑ च मयो॒भवे॑ मयो॒भवे॑ च शं॒भवे॑ शं॒भवे॑\\
च मयो॒भवे᳚ ।\\
\\
42. शं॒भवे᳚ ।\\
शं॒भव॒ इति॑ शं - भवे᳚ ।\\
\\
43. च॒ । म॒यो॒भवे᳚ । च॒ ।\\
च॒ म॒यो॒भवे॑ मयो॒भवे॑ च च मयो॒भवे॑ च च मयो॒भवे॑ च च मयो॒भवे॑ च ।\\
\\
44. म॒यो॒भवे᳚ । च॒ । नमः॑ ।\\
म॒यो॒भवे॑ च च मयो॒भवे॑ मयो॒भवे॑ च॒ नमो॒ नम॑श्च मयो॒भवे॑ मयो॒भवे॑\\
च॒ नमः॑ ।\\
\\
45. म॒यो॒भवे᳚ ।\\
म॒यो॒भव॒ इति॑ मयः - भवे᳚ ।\\
\\
46. च॒ । नमः॑ । श॒ङ्क॒राय॑ ।\\
च॒ नमो॒ नम॑श्च च॒ नमः॑ शङ्क॒राय॑ शङ्क॒राय॒ नम॑श्च च॒ नमः॑ शङ्क॒राय॑ ।\\
\\
47. नमः॑ । श॒ङ्क॒राय॑ । च॒ ।\\
नमः॑ शङ्क॒राय॑ शङ्क॒राय॒ नमो॒ नमः॑ शङ्क॒राय॑ च च शङ्क॒राय॒ नमो॒ नमः॑\\
शङ्क॒राय॑ च ।\\
\\
48. श॒ङ्क॒राय॑ । च॒ । म॒य॒स्क॒राय॑ ।\\
श॒ङ्क॒राय॑ च च शङ्क॒राय॑ शङ्क॒राय॑ च मयस्क॒राय॑ मयस्क॒राय॑ च शङ्क॒राय॑\\
शङ्क॒राय॑ च मयस्क॒राय॑ ।\\
\\
49. श॒ङ्क॒राय॑ ।\\
श॒ङ्क॒रायेति॑ शं - क॒राय॑ ।\\
\\
50. च॒ । म॒य॒स्क॒राय॑ । च॒ ।\\
च॒ म॒य॒स्क॒राय॑ मयस्क॒राय॑ च च मयस्क॒राय॑ च च मयस्क॒राय॑ च च\\
मयस्क॒राय॑ च ।\\
\\
51. म॒य॒स्क॒राय॑ । च॒ । नमः॑ ।\\
म॒य॒स्क॒राय॑ च च मयस्क॒राय॑ मयस्क॒राय॑ च॒ नमो॒ नम॑श्च मयस्क॒राय॑\\
मयस्क॒राय॑ च॒ नमः॑ ।\\
\\
52. म॒य॒स्क॒राय॑ ।\\
म॒य॒स्क॒रायेति॑ मयः - क॒राय॑ ।\\
\\
53. च॒ । नमः॑ । शि॒वाय॑ ।\\
च॒ नमो॒ नम॑श्च च॒ नमः॑ शि॒वाय॑ शि॒वाय॒ नम॑श्च च॒ नमः॑ शि॒वाय॑ ।\\
\\
54. नमः॑ । शि॒वाय॑ । च॒ ।\\
नमः॑ शि॒वाय॑ शि॒वाय॒ नमो॒ नमः॑ शि॒वाय॑ च च शि॒वाय॒ नमो॒ नमः॑\\
शि॒वाय॑ च ।\\
\\
55. शि॒वाय॑ । च॒ । शि॒वत॑राय ।\\
शि॒वाय॑ च च शि॒वाय॑ शि॒वाय॑ च शि॒वत॑राय शि॒वत॑राय च शि॒वाय॑\\
शि॒वाय॑ च शि॒वत॑राय ।\\
\\
56. च॒ । शि॒वत॑राय । च॒ ।\\
च॒ शि॒वत॑राय शि॒वत॑राय च च शि॒वत॑राय च च शि॒वत॑राय च च\\
शि॒वत॑राय च ।\\
\\
57. शि॒वत॑राय । च॒ । नमः॑ ।\\
शि॒वत॑राय च च शि॒वत॑राय शि॒वत॑राय च॒ नमो॒ नम॑श्च शि॒वत॑राय\\
शि॒वत॑राय च॒ नमः॑ ।\\
\\
58. शि॒वत॑राय ।\\
शि॒वत॑रा॒येति॑ शि॒व - त॒रा॒य॒ ।\\
\\
59. च॒ । नमः॑ । तीर्थ्या॑य ।\\
च॒ नमो॒ नम॑श्च च॒ नम॒ स्तीर्थ्या॑य॒ तीर्थ्या॑य॒ नम॑श्च च॒ नम॒ स्तीर्थ्या॑य ।\\
\\
60. नमः॑ । तीर्थ्या॑य । च॒ ।\\
नम॒ स्तीर्थ्या॑य॒ तीर्थ्या॑य॒ नमो॒ नम॒ स्तीर्थ्या॑य च च॒ तीर्थ्या॑य॒ नमो॒\\
नम॒ स्तीर्थ्या॑य च ।\\
\\
61. तीर्थ्या॑य । च॒ । कूल्या॑य ।\\
तीर्थ्या॑य च च॒ तीर्थ्या॑य॒ तीर्थ्या॑य च॒ कूल्या॑य॒ कूल्या॑य च॒ तीर्थ्या॑य॒\\
तीर्थ्या॑य च॒ कूल्या॑य ।\\
\\
62. च॒ । कूल्या॑य । च॒ ।\\
च॒ कूल्या॑य॒ कूल्या॑य च च॒ कूल्या॑य च च॒ कूल्या॑य च च॒ कूल्या॑य च ।\\
\\
63. कूल्या॑य । च॒ । नमः॑ ।\\
कूल्या॑य च च॒ कूल्या॑य॒ कूल्या॑य च॒ नमो॒ नम॑श्च॒ कूल्या॑य॒ कूल्या॑य\\
च॒ नमः॑ ।\\
\\
64. च॒ । नमः॑ । पा॒र्या॑य ।\\
च॒ नमो॒ नम॑श्च च॒ नमः॑ पा॒र्या॑य पा॒र्या॑य॒ नम॑श्च च॒ नमः॑ पा॒र्या॑य ।\\
\\
65. नमः॑ । पा॒र्या॑य । च॒ ।\\
नमः॑ पा॒र्या॑य पा॒र्या॑य॒ नमो॒ नमः॑ पा॒र्या॑य च च पा॒र्या॑य॒ नमो॒ नमः॑\\
पा॒र्या॑य च ।\\
\\
66. पा॒र्या॑य । च॒ । अ॒वा॒र्या॑य ।\\
पा॒र्या॑य च च पा॒र्या॑य पा॒र्या॑य चा वा॒र्या॑या वा॒र्या॑य च पा॒र्या॑य पा॒र्या॑य चा\\
वा॒र्या॑य ।\\
\\
67. च॒ । अ॒वा॒र्या॑य । च॒ ।\\
चा॒ वा॒र्या॑या वा॒र्या॑य च चा वा॒र्या॑य च चा वा॒र्या॑य च चा वा॒र्या॑य च ।\\
\\
68. अ॒वा॒र्या॑य । च॒ । नमः॑ ।\\
अ॒वा॒र्या॑य च चा वा॒र्या॑या वा॒र्या॑य च॒ नमो॒ नम॑श्चा वा॒र्या॑या\\
वा॒र्या॑य च॒ नमः॑ ।\\
\\
69. च॒ । नमः॑ । प्र॒तर॑णाय ।\\
च॒ नमो॒ नम॑श्च च॒ नमः॑ प्र॒तर॑णाय प्र॒तर॑णाय॒ नम॑श्च च॒ नमः॑ प्र॒तर॑णाय ।\\
\\
70. नमः॑ । प्र॒तर॑णाय । च॒ ।\\
नमः॑ प्र॒तर॑णाय प्र॒तर॑णाय॒ नमो॒ नमः॑ प्र॒तर॑णाय च च प्र॒तर॑णाय॒ नमो॒ नमः॑\\
प्र॒तर॑णाय च ।\\
\\
71. प्र॒तर॑णाय । च॒ । उ॒त्तर॑णाय ।\\
प्र॒तर॑णाय च च प्र॒तर॑णाय प्र॒तर॑णाय चो॒त्तर॑णा यो॒त्तर॑णाय च प्र॒तर॑णाय\\
प्र॒तर॑णाय चो॒त्तर॑णाय ।\\
\\
72. प्र॒तर॑णाय ।\\
प्र॒तर॑णा॒येति॑ प्र - तर॑णाय ।\\
\\
73. च॒ । उ॒त्तर॑णाय । च॒ ।\\
चो॒त्तर॑णा यो॒त्तर॑णाय च चो॒त्तर॑णाय च चो॒त्तर॑णाय च चो॒त्तर॑णाय च ।\\
\\
74. उ॒त्तर॑णाय । च॒ । नमः॑ ।\\
उ॒त्तर॑णाय च चो॒त्तर॑णा यो॒त्तर॑णाय च॒ नमो॒ नम॑ श्चो॒त्तर॑णा\\
यो॒त्तर॑णाय च॒ नमः॑ ।\\
\\
75. उ॒त्तर॑णाय ।\\
उ॒त्तर॑णा॒येत्यु॑त् - तर॑णाय ।\\
\\
76. च॒ । नमः॑ । आ॒ता॒र्या॑य ।\\
च॒ नमो॒ नम॑श्च च॒ नम॑ आता॒र्या॑या ता॒र्या॑य॒ नम॑श्च च॒ नम॑ आता॒र्या॑य ।\\
\\
77. नमः॑ । आ॒ता॒र्या॑य । च॒ ।\\
नम॑ आता॒र्या॑या ता॒र्या॑य॒ नमो॒ नम॑ आता॒र्या॑य च चा ता॒र्या॑य॒ नमो॒ नम॑\\
आता॒र्या॑य च ।\\
\\
78. आ॒ता॒र्या॑य । च॒ । आ॒ला॒द्या॑य ।\\
आ॒ता॒र्या॑य च चा ता॒र्या॑या ता॒र्या॑य चा ला॒द्या॑या ला॒द्या॑य चा ता॒र्या॑या ता॒र्या॑य\\
चा ला॒द्या॑य ।\\
\\
79. आ॒ता॒र्या॑य ।\\
आ॒ता॒र्या॑येत्या᳚ - ता॒र्या॑य ।\\
\\
80. च॒ । आ॒ला॒द्या॑य । च॒ ।\\
चा॒ ला॒द्या॑या ला॒द्या॑य च चा ला॒द्या॑य च चा ला॒द्या॑य च चा ला॒द्या॑य च ।\\
\\
81. आ॒ला॒द्या॑य । च॒ । नमः॑ ।\\
आ॒ला॒द्या॑य च चा ला॒द्या॑या ला॒द्या॑य च॒ नमो॒ नम॑श्चा ला॒द्या॑या ला॒द्या॑य\\
च॒ नमः॑ ।\\
\\
82. आ॒ला॒द्या॑य ।\\
आ॒ला॒द्या॑येत्या᳚ - ला॒द्या॑य ।\\
\\
83. च॒ । नमः॑ । शष्प्या॑य ।\\
च॒ नमो॒ नम॑श्च च॒ नमः॒ शष्प्या॑य॒ शष्प्या॑य॒ नम॑श्च च॒ नमः॒ शष्प्या॑य ।\\
\\
84. नमः॑ । शष्प्या॑य । च॒ ।\\
नमः॒ शष्प्या॑य॒ शष्प्या॑य॒ नमो॒ नमः॒ शष्प्या॑य च च॒ शष्प्या॑य॒ नमो॒ नमः॒\\
शष्प्या॑य च ।\\
\\
85. शष्प्या॑य । च॒ । फेन्या॑य ।\\
शष्प्या॑य च च॒ शष्प्या॑य॒ शष्प्या॑य च॒ फेन्या॑य॒ फेन्या॑य च॒ शष्प्या॑य॒\\
शष्प्या॑य च॒ फेन्या॑य ।\\
\\
86. च॒ । फेन्या॑य । च॒ ।\\
च॒ फेन्या॑य॒ फेन्या॑य च च॒ फेन्या॑य च च॒ फेन्या॑य च च॒ फेन्या॑य च ।\\
\\
87. फेन्या॑य । च॒ । नमः॑ ।\\
फेन्या॑य च च॒ फेन्या॑य॒ फेन्या॑य च॒ नमो॒ नम॑श्च॒ फेन्या॑य॒ फेन्या॑य च॒ नमः॑ ।\\
\\
88. च॒ । नमः॑ । सि॒क॒त्या॑य ।\\
च॒ नमो॒ नम॑श्च च॒ नमः॑ सिक॒त्या॑य सिक॒त्या॑य॒ नम॑श्च च॒ नमः॑\\
सिक॒त्या॑य ।\\
\\
89. नमः॑ । सि॒क॒त्या॑य । च॒ ।\\
नमः॑ सिक॒त्या॑य सिक॒त्या॑य॒ नमो॒ नमः॑ सिक॒त्या॑य च च सिक॒त्या॑य॒\\
नमो॒ नमः॑ सिक॒त्या॑य च ।\\
\\
90. सि॒क॒त्या॑य । च॒ । प्र॒वा॒ह्या॑य ।\\
सि॒क॒त्या॑य च च सिक॒त्या॑य सिक॒त्या॑य च प्रवा॒ह्या॑य प्रवा॒ह्या॑य च\\
सिक॒त्या॑य सिक॒त्या॑य च प्रवा॒ह्या॑य ।\\
\\
91. च॒ । प्र॒वा॒ह्या॑य । च॒ ॥\\
च॒ प्र॒वा॒ह्या॑य प्रवा॒ह्या॑य च च प्रवा॒ह्या॑य च च प्रवा॒ह्या॑य च च\\
प्रवा॒ह्या॑य च ।\\
\\
92. प्र॒वा॒ह्या॑य । च॒ ॥\\
प्र॒वा॒ह्या॑य च च प्रवा॒ह्या॑य प्रवा॒ह्या॑य च ।\\
\\
93. प्र॒वा॒ह्या॑य ।\\
प्र॒वा॒ह्या॑येति॑ प्र - वा॒ह्या॑य ।\\
\\
94. च॒ ॥\\
चेति॑ च ।\\
\subsection{\eng{Anuvaka 9}}
1. नमः॑ । इ॒रि॒ण्या॑य । च॒ ।\\
नम॑ इरि॒ण्या॑ येरि॒ण्या॑य॒ नमो॒ नम॑ इरि॒ण्या॑य च चे रि॒ण्या॑य॒ नमो॒ नम॑\\
इरि॒ण्या॑य च ।\\
\\
2. इ॒रि॒ण्या॑य । च॒ । प्र॒प॒थ्या॑य ।\\
इ॒रि॒ण्या॑य च चे रि॒ण्या॑ येरि॒ण्या॑य च प्रप॒थ्या॑य प्रप॒थ्या॑य चे रि॒ण्या॑\\
येरि॒ण्या॑य च प्रप॒थ्या॑य ।\\
\\
3. च॒ । प्र॒प॒थ्या॑य । च॒ ।\\
च॒ प्र॒प॒थ्या॑य प्रप॒थ्या॑य च च प्रप॒थ्या॑य च च प्रप॒थ्या॑य च च\\
प्रप॒थ्या॑य च ।\\
\\
4. प्र॒प॒थ्या॑य । च॒ । नमः॑ ।\\
प्र॒प॒थ्या॑य च च प्रप॒थ्या॑य प्रप॒थ्या॑य च॒ नमो॒ नम॑श्च प्रप॒थ्या॑य\\
प्रप॒थ्या॑य च॒ नमः॑ ।\\
\\
5. प्र॒प॒थ्या॑य ।\\
प्र॒प॒थ्या॑येति॑ प्र - प॒थ्या॑य ।\\
\\
6. च॒ । नमः॑ । कि॒ꣳ॒शि॒लाय॑ ।\\
च॒ नमो॒ नम॑श्च च॒ नमः॑ किꣳशि॒लाय॑ किꣳशि॒लाय॒ नम॑श्च च॒ नमः॑\\
किꣳशि॒लाय॑ ।\\
\\
7. नमः॑ । कि॒ꣳ॒शि॒लाय॑ । च॒ ।\\
नमः॑ किꣳशि॒लाय॑ किꣳशि॒लाय॒ नमो॒ नमः॑ किꣳशि॒लाय॑ च च\\
किꣳशि॒लाय॒ नमाे॒ नमः॑ किꣳशि॒लाय॑ च ।\\
\\
8. कि॒ꣳ॒शि॒लाय॑ । च॒ । क्षय॑णाय ।\\
कि॒ꣳ॒शि॒लाय॑ च च किꣳशि॒लाय॑ किꣳशि॒लाय॑ च॒ क्षय॑णाय॒ क्षय॑णाय\\
च किꣳशि॒लाय॑ किꣳशि॒लाय॑ च॒ क्षय॑णाय ।\\
\\
9. च॒ । क्षय॑णाय । च॒ ।\\
च॒ क्षय॑णाय॒ क्षय॑णाय च च॒ क्षय॑णाय च च॒ क्षय॑णाय च च॒ क्षय॑णाय च ।\\
\\
10. क्षय॑णाय । च॒ । नमः॑ ।\\
क्षय॑णाय च च॒ क्षय॑णाय॒ क्षय॑णाय च॒ नमो॒ नम॑श्च॒ क्षय॑णाय॒ क्षय॑णाय\\
च॒ नमः॑ ।\\
\\
11. च॒ । नमः॑ । क॒प॒र्दिने᳚ ।\\
च॒ नमो॒ नम॑श्च च॒ नमः॑ कप॒र्दिने॑ कप॒र्दिने॒ नम॑श्च च॒ नमः॑ कप॒र्दिने᳚ ।\\
\\
12. नमः॑ । क॒प॒र्दिने᳚ । च॒ ।\\
नमः॑ कप॒र्दिने॑ कप॒र्दिने॒ नमो॒ नमः॑ कप॒र्दिने॑ च च कप॒र्दिने॒ नमो॒ नमः॑\\
कप॒र्दिने॑ च ।\\
\\
13. क॒प॒र्दिने᳚ । च॒ । पु॒ल॒स्तये᳚ ।\\
क॒प॒र्दिने॑ च च कप॒र्दिने॑ कप॒र्दिने॑ च पुल॒स्तये॑ पुल॒स्तये॑ च कप॒र्दिने॑\\
कप॒र्दिने॑ च पुल॒स्तये᳚ ।\\
\\
14. च॒ । पु॒ल॒स्तये᳚ । च॒ ।\\
च॒ पु॒ल॒स्तये॑ पुल॒स्तये॑ च च पुल॒स्तये॑ च च पुल॒स्तये॑ च च\\
पुल॒स्तये॑ च ।\\
\\
15. पु॒ल॒स्तये᳚ । च॒ । नमः॑ ।\\
पु॒ल॒स्तये॑ च च पुल॒स्तये॑ पुल॒स्तये॑ च॒ नमो॒ नम॑श्च पुल॒स्तये॑\\
पुल॒स्तये॑ च॒ नमः॑ ।\\
\\
16. च॒ । नमः॑ । गोष्ठ्या॑य ।\\
च॒ नमो॒ नम॑श्च च॒ नमो॒ गोष्ठ्या॑य॒ गोष्ठ्या॑य॒ नम॑श्च च॒ नमो॒ गोष्ठ्या॑य ।\\
\\
17. नमः॑ । गोष्ठ्या॑य । च॒ ।\\
नमो॒ गोष्ठ्या॑य॒ गोष्ठ्या॑य॒ नमो॒ नमो॒ गोष्ठ्या॑य च च॒ गोष्ठ्या॑य॒ नमो॒ नमो॒\\
गोष्ठ्या॑य च ।\\
\\
18. गोष्ठ्या॑य । च॒ । गृह्या॑य ।\\
गोष्ठ्या॑य च च॒ गोष्ठ्या॑य॒ गोष्ठ्या॑य च॒ गृह्या॑य॒ गृह्या॑य च॒ गोष्ठ्या॑य॒\\
गोष्ठ्या॑य च॒ गृह्या॑य ।\\
\\
19. गोष्ठ्या॑य ।\\
गोष्ठ्या॒येति॒ गो - स्थ्या॒य॒ ।\\
\\
20. च॒ । गृह्या॑य । च॒ ।\\
च॒ गृह्या॑य॒ गृह्या॑य च च॒ गृह्या॑य च च॒ गृह्या॑य च च॒ गृह्या॑य च ।\\
\\
21. गृह्या॑य । च॒ । नमः॑ ।\\
गृह्या॑य च च॒ गृह्या॑य॒ गृह्या॑य च॒ नमो॒ नम॑श्च॒ गृह्या॑य॒ गृह्या॑य च॒ नमः॑ ।\\
\\
22. च॒ । नमः॑ । तल्प्या॑य ।\\
च॒ नमो॒ नम॑श्च च॒ नम॒ स्तल्प्या॑य॒ तल्प्या॑य॒ नम॑श्च च॒ नम॒ स्तल्प्या॑य ।\\
\\
23. नमः॑ । तल्प्या॑य । च॒ ।\\
नम॒ स्तल्प्या॑य॒ तल्प्या॑य॒ नमो॒ नम॒ स्तल्प्या॑य च च॒ तल्प्या॑य॒ नमो॒\\
नम॒ स्तल्प्या॑य च ।\\
\\
24. तल्प्या॑य । च॒ । गेह्या॑य ।\\
तल्प्या॑य च च॒ तल्प्या॑य॒ तल्प्या॑य च॒ गेह्या॑य॒ गेह्या॑य च॒ तल्प्या॑य॒ तल्प्या॑य\\
च॒ गेह्या॑य ।\\
\\
25. च॒ । गेह्या॑य । च॒ ।\\
च॒ गेह्या॑य॒ गेह्या॑य च च॒ गेह्या॑य च च॒ गेह्या॑य च च॒ गेह्या॑य च ।\\
\\
26. गेह्या॑य । च॒ । नमः॑ ।\\
गेह्या॑य च च॒ गेह्या॑य॒ गेह्या॑य च॒ नमो॒ नम॑श्च॒ गेह्या॑य॒ गेह्या॑य च॒ नमः॑ ।\\
\\
27. च॒ । नमः॑ । का॒ट्या॑य ।\\
च॒ नमो॒ नम॑श्च च॒ नमः॑ का॒ट्या॑य का॒ट्या॑य॒ नम॑श्च च॒ नमः॑ का॒ट्या॑य ।\\
\\
28. नमः॑ । का॒ट्या॑य । च॒ ।\\
नमः॑ का॒ट्या॑य का॒ट्या॑य॒ नमो॒ नमः॑ का॒ट्या॑य च च का॒ट्या॑य॒ नमो॒ नमः॑\\
का॒ट्या॑य च ।\\
\\
29. का॒ट्या॑य । च॒ । ग॒ह्व॒रे॒ष्ठाय॑ ।\\
का॒ट्या॑य च च का॒ट्या॑य का॒ट्या॑य च गह्वरे॒ष्ठाय॑ गह्वरे॒ष्ठाय॑ च का॒ट्या॑य\\
का॒ट्या॑य च गह्वरे॒ष्ठाय॑ ।\\
\\
30. च॒ । ग॒ह्व॒रे॒ष्ठाय॑ । च॒ ।\\
च॒ ग॒ह्व॒रे॒ष्ठाय॑ गह्वरे॒ष्ठाय॑ च च गह्वरे॒ष्ठाय॑ च च गह्वरे॒ष्ठाय॑ च च\\
गह्वरे॒ष्ठाय॑ च ।\\
\\
31. ग॒ह्व॒रे॒ष्ठाय॑ । च॒ । नमः॑ ।\\
ग॒ह्व॒रे॒ष्ठाय॑ च च गह्वरे॒ष्ठाय॑ गह्वरे॒ष्ठाय॑ च॒ नमो॒ नम॑श्च गह्वरे॒ष्ठाय॑ गह्वरे॒ष्ठाय॑\\
च॒ नमः॑ ।\\
\\
32. ग॒ह्व॒रे॒ष्ठाय॑ ।\\
ग॒ह्व॒रे॒ष्ठायेति॑ गह्वरे - स्थाय॑ ।\\
\\
33. च॒ । नमः॑ । ह्र॒द॒य्या॑य ।\\
च॒ नमो॒ नम॑श्च च॒ नमो᳚ ह्रद॒य्या॑य ह्रद॒य्या॑य॒ नम॑श्च च॒ नमो᳚ ह्रद॒य्या॑य ।\\
\\
34. नमः॑ । ह्र॒द॒य्या॑य । च॒ ।\\
नमो᳚ ह्रद॒य्या॑य ह्रद॒य्या॑य॒ नमो॒ नमो᳚ ह्रद॒य्या॑य च च ह्रद॒य्या॑य॒ नमो॒ नमो᳚\\
ह्रद॒य्या॑य च ।\\
\\
35. ह्र॒द॒य्या॑य । च॒ । नि॒वे॒ष्प्या॑य ।\\
ह्र॒द॒य्या॑य च च ह्रद॒य्या॑य ह्रद॒य्या॑य च निवे॒ष्प्या॑य निवे॒ष्प्या॑य च ह्रद॒य्या॑य\\
ह्रद॒य्या॑य च निवे॒ष्प्या॑य ।\\
\\
36. च॒ । नि॒वे॒ष्प्या॑य । च॒ ।\\
च॒ नि॒वे॒ष्प्या॑य निवे॒ष्प्या॑य च च निवे॒ष्प्या॑य च च निवे॒ष्प्या॑य च च\\
निवे॒ष्प्या॑य च ।\\
\\
37. नि॒वे॒ष्प्या॑य । च॒ । नमः॑ ।\\
नि॒वे॒ष्प्या॑य च च निवे॒ष्प्या॑य निवे॒ष्प्या॑य च॒ नमो॒ नम॑श्च निवे॒ष्प्या॑य\\
निवे॒ष्प्या॑य च॒ नमः॑ ।\\
\\
38. नि॒वे॒ष्प्या॑य ।\\
नि॒वे॒ष्प्या॑येति॑ नि - वे॒ष्प्या॑य ।\\
\\
39. च॒ । नमः॑ । पा॒ꣳ॒स॒व्या॑य ।\\
च॒ नमो॒ नम॑श्च च॒ नमः॑ पाꣳस॒व्या॑य पाꣳस॒व्या॑य॒ नम॑श्च च॒ नमः॑\\
पाꣳस॒व्या॑य ।\\
\\
40. नमः॑ । पा॒ꣳ॒स॒व्या॑य । च॒ ।\\
नमः॑ पाꣳस॒व्या॑य पाꣳस॒व्या॑य॒ नमो॒ नमः॑ पाꣳस॒व्या॑य च च पाꣳस॒व्या॑य॒\\
नमो॒ नमः॑ पाꣳस॒व्या॑य च ।\\
\\
41. पा॒ꣳ॒स॒व्या॑य । च॒ । र॒ज॒स्या॑य ।\\
पा॒ꣳ॒स॒व्या॑य च च पाꣳस॒व्या॑य पाꣳस॒व्या॑य च रज॒स्या॑य रज॒स्या॑य च\\
पाꣳस॒व्या॑य पाꣳस॒व्या॑य च रज॒स्या॑य ।\\
\\
42. च॒ । र॒ज॒स्या॑य । च॒ ।\\
च॒ र॒ज॒स्या॑य रज॒स्या॑य च च रज॒स्या॑य च च रज॒स्या॑य च च\\
रज॒स्या॑य च ।\\
\\
43. र॒ज॒स्या॑य । च॒ । नमः॑ ।\\
र॒ज॒स्या॑य च च रज॒स्या॑य रज॒स्या॑य च॒ नमो॒ नम॑श्च रज॒स्या॑य रज॒स्या॑य\\
च॒ नमः॑ ।\\
\\
44. च॒ । नमः॑ । शुष्क्या॑य ।\\
च॒ नमो॒ नम॑श्च च॒ नमः॒ शुष्क्या॑य॒ शुष्क्या॑य॒ नम॑श्च च॒ नमः॒ शुष्क्या॑य ।\\
\\
45. नमः॑ । शुष्क्या॑य । च॒ ।\\
नमः॒ शुष्क्या॑य॒ शुष्क्या॑य॒ नमो॒ नमः॒ शुष्क्या॑य च च॒ शुष्क्या॑य॒ नमो॒ नमः॒\\
शुष्क्या॑य च ।\\
\\
46. शुष्क्या॑य । च॒ । ह॒रि॒त्या॑य ।\\
शुष्क्या॑य च च॒ शुष्क्या॑य॒ शुष्क्या॑य च हरि॒त्या॑य हरि॒त्या॑य च॒ शुष्क्या॑य॒\\
शुष्क्या॑य च हरि॒त्या॑य ।\\
\\
47. च॒ । ह॒रि॒त्या॑य । च॒ ।\\
च॒ ह॒रि॒त्या॑य हरि॒त्या॑य च च हरि॒त्या॑य च च हरि॒त्या॑य च च हरि॒त्या॑य च ।\\
\\
48. ह॒रि॒त्या॑य । च॒ । नमः॑ ।\\
ह॒रि॒त्या॑य च च हरि॒त्या॑य हरि॒त्या॑य च॒ नमो॒ नम॑श्च हरि॒त्या॑य हरि॒त्या॑य च॒\\
नमः॑ ।\\
\\
49. च॒ । नमः॑ । लोप्या॑य ।\\
च॒ नमो॒ नम॑श्च च॒ नमो॒ लोप्या॑य॒ लोप्या॑य॒ नम॑श्च च॒ नमो॒ लोप्या॑य ।\\
\\
50. नमः॑ । लोप्या॑य । च॒ ।\\
नमो॒ लोप्या॑य॒ लोप्या॑य॒ नमो॒ नमो॒ लोप्या॑य च च॒ लोप्या॑य॒ नमो॒ नमो॒\\
लोप्या॑य च ।\\
\\
51. लोप्या॑य । च॒ । उ॒ल॒प्या॑य ।\\
लोप्या॑य च च॒ लोप्या॑य॒ लोप्या॑य चो ल॒प्या॑यो ल॒प्या॑य च॒ लोप्या॑य॒\\
लोप्या॑य चो ल॒प्या॑य ।\\
\\
52. च॒ । उ॒ल॒प्या॑य । च॒ ।\\
चो॒ल॒प्या॑ योल॒प्या॑य च चो ल॒प्या॑य च चो ल॒प्या॑य च चो ल॒प्या॑य च ।\\
\\
53. उ॒ल॒प्या॑य । च॒ । नमः॑ ।\\
उ॒ल॒प्या॑य च चो ल॒प्या॑ योल॒प्या॑य च॒ नमो॒ नम॑श्चो ल॒प्या॑ योल॒प्या॑य\\
च॒ नमः॑ ।\\
\\
54. च॒ । नमः॑ । ऊ॒र्व्या॑य ।\\
च॒ नमो॒ नम॑श्च च॒ नम॑ ऊ॒र्व्या॑ यो॒र्व्या॑य॒ नम॑श्च च॒ नम॑ ऊ॒र्व्या॑य ।\\
\\
55. नमः॑ । ऊ॒र्व्या॑य । च॒ ।\\
नम॑ ऊ॒र्व्या॑ यो॒र्व्या॑य॒ नमो॒ नम॑ ऊ॒र्व्या॑य च चो॒र्व्या॑य॒ नमो॒ नम॑ ऊ॒र्व्या॑य च ।\\
\\
56. ऊ॒र्व्या॑य । च॒ । सू॒र्म्या॑य ।\\
ऊ॒र्व्या॑य च चो॒र्व्या॑ यो॒र्व्या॑य च सू॒र्म्या॑य सू॒र्म्या॑य चो॒र्व्या॑ यो॒र्व्या॑य च\\
सू॒र्म्या॑य ।\\
\\
57. च॒ । सू॒र्म्या॑य । च॒ ।\\
च॒ सू॒र्म्या॑य सू॒र्म्या॑य च च सू॒र्म्या॑य च च सू॒र्म्या॑य च च सू॒र्म्या॑य च ।\\
\\
58. सू॒र्म्या॑य । च॒ । नमः॑ ।\\
सू॒र्म्या॑य च च सू॒र्म्या॑य सू॒र्म्या॑य च॒ नमो॒ नम॑श्च सू॒र्म्या॑य सू॒र्म्या॑य\\
च॒ नमः॑ ।\\
\\
59. च॒ । नमः॑ । प॒र्ण्या॑य ।\\
च॒ नमो॒ नम॑श्च च॒ नमः॑ प॒र्ण्या॑य प॒र्ण्या॑य॒ नम॑श्च च॒ नमः॑ प॒र्ण्या॑य ।\\
\\
60. नमः॑ । प॒र्ण्या॑य । च॒ ।\\
नमः॑ प॒र्ण्या॑य प॒र्ण्या॑य॒ नमो॒ नमः॑ प॒र्ण्या॑य च च प॒र्ण्या॑य॒ नमो॒ नमः॑\\
प॒र्ण्या॑य च ।\\
\\
61. प॒र्ण्या॑य । च॒ । प॒र्ण॒श॒द्या॑य ।\\
प॒र्ण्या॑य च च प॒र्ण्या॑य प॒र्ण्या॑य च पर्णश॒द्या॑य पर्णश॒द्या॑य च प॒र्ण्या॑य\\
प॒र्ण्या॑य च पर्णश॒द्या॑य ।\\
\\
62. च॒ । प॒र्ण॒श॒द्या॑य । च॒ ।\\
च॒ प॒र्ण॒श॒द्या॑य पर्णश॒द्या॑य च च पर्णश॒द्या॑य च च पर्णश॒द्या॑य च च\\
पर्णश॒द्या॑य च ।\\
\\
63. प॒र्ण॒श॒द्या॑य । च॒ । नमः॑ ।\\
प॒र्ण॒श॒द्या॑य च च पर्णश॒द्या॑य पर्णश॒द्या॑य च॒ नमो॒ नम॑श्च पर्णश॒द्या॑य\\
पर्णश॒द्या॑य च॒ नमः॑ ।\\
\\
64. प॒र्ण॒श॒द्या॑य ।\\
प॒र्ण॒श॒द्या॑येति॑ पर्ण - श॒द्या॑य ।\\
\\
65. च॒ । नमः॑ । अ॒प॒गु॒रमा॑णाय ।\\
च॒ नमो॒ नम॑श्च च॒ नमो॑ऽपगु॒रमा॑णाया पगु॒रमा॑णाय॒ नम॑श्च च॒ नमो॑ऽपगु॒रमा॑णाय ।\\
\\
66. नमः॑ । अ॒प॒गु॒रमा॑णाय । च॒ ।\\
नमो॑ऽपगु॒रमा॑णाया पगु॒रमा॑णाय॒ नमो॒ नमो॑ऽपगु॒रमा॑णाय च चा\\
पगु॒रमा॑णाय॒ नमो॒ नमो॑ऽपगु॒रमा॑णाय च ।\\
\\
67. अ॒प॒गु॒रमा॑णाय । च॒ । अ॒भि॒घ्न॒ते ।\\
अ॒प॒गु॒रमा॑णाय च चा पगु॒रमा॑णाया पगु॒रमा॑णाय चा भिघ्न॒ते॑ऽभिघ्न॒ते\\
चा॑ पगु॒रमा॑णाया पगु॒रमा॑णाय चा भिघ्न॒ते ।\\
\\
68. अ॒प॒गु॒रमा॑णाय ।\\
अ॒प॒गु॒रमा॑णा॒येत्य॑प - गु॒रमा॑णाय ।\\
\\
69. च॒ । अ॒भि॒घ्न॒ते । च॒ ।\\
चा॒ भि॒घ्न॒ते॑ऽभिघ्न॒ते च॑ चा भिघ्न॒ते च॑ चा भिघ्न॒ते च॑ चा भिघ्न॒ते च॑ ।\\
\\
70. अ॒भि॒घ्न॒ते । च॒ । नमः॑ ।\\
अ॒भि॒घ्न॒ते च॑ चा भिघ्न॒ते॑ऽभिघ्न॒ते च॒ नमो॒ नम॑श्चा भिघ्न॒ते॑ऽभिघ्न॒ते च॒ नमः॑ ।\\
\\
71. अ॒भि॒घ्न॒ते ।\\
अ॒भि॒घ्न॒त इत्य॑भि - घ्न॒ते ।\\
\\
72. च॒ । नमः॑ । आ॒क्खि॒द॒ते ।\\
च॒ नमो॒ नम॑श्च च॒ नम॑ आक्खिद॒त आ᳚क्खिद॒ते नम॑श्च च॒ नम॑\\
आक्खिद॒ते ।\\
\\
73. नमः॑ । आ॒क्खि॒द॒ते । च॒ ।\\
नम॑ आक्खिद॒त आ᳚क्खिद॒ते नमो॒ नम॑ आक्खिद॒ते च॑ चाक्खिद॒ते नमो॒\\
नम॑ आक्खिद॒ते च॑ ।\\
\\
74. आ॒क्खि॒द॒ते । च॒ । प्र॒क्खि॒द॒ते ।\\
आ॒क्खि॒द॒ते च॑ चाक्खिद॒त आ᳚क्खिद॒ते च॑ प्रक्खिद॒ते प्र॑क्खिद॒ते\\
चा᳚क्खिद॒त आ᳚क्खिद॒ते च॑ प्रक्खिद॒ते ।\\
\\
75. आ॒क्खि॒द॒ते ।\\
आ॒क्खि॒द॒त इत्या᳚ - खि॒द॒ते ।\\
\\
76. च॒ । प्र॒क्खि॒द॒ते । च॒ ।\\
च॒ प्र॒क्खि॒द॒ते प्र॑क्खिद॒ते च॑ च प्रक्खिद॒ते च॑ च प्रक्खिद॒ते च॑ च\\
प्रक्खिद॒ते च॑ ।\\
\\
77. प्र॒क्खि॒द॒ते । च॒ । नमः॑ ।\\
प्र॒क्खि॒द॒ते च॑ च प्रक्खिद॒ते प्र॑क्खिद॒ते च॒ नमो॒ नम॑श्च प्रक्खिद॒ते\\
प्र॑क्खिद॒ते च॒ नमः॑ ।\\
\\
78. प्र॒क्खि॒द॒ते ।\\
प्र॒क्खि॒द॒त इति॑ प्र - खि॒द॒ते ।\\
\\
79. च॒ । नमः॑ । वः॒ ।\\
च॒ नमो॒ नम॑श्च च॒ नमो॑ वो वो॒ नम॑श्च च॒ नमो॑ वः ।\\
\\
80. नमः॑ । वः॒ । कि॒रि॒केभ्यः॑ ।\\
नमो॑ वो वो॒ नमो॒ नमो॑ वः किरि॒केभ्यः॑ किरि॒केभ्यो॑ वो॒ नमो॒ नमो॑ वः\\
किरि॒केभ्यः॑ ।\\
\\
81. वः॒ । कि॒रि॒केभ्यः॑ । दे॒वाना᳚म् ।\\
वः॒ कि॒रि॒केभ्यः॑ किरि॒केभ्यो॑ वो वः किरि॒केभ्यो॑ दे॒वानां᳚ दे॒वानां᳚\\
किरि॒केभ्यो॑ वो वः किरि॒केभ्यो॑ दे॒वाना᳚म् ।\\
\\
82. कि॒रि॒केभ्यः॑ । दे॒वाना᳚म् । हृद॑येभ्यः ।\\
कि॒रि॒केभ्यो॑ दे॒वानां᳚ दे॒वानां᳚ किरि॒केभ्यः॑ किरि॒केभ्यो॑ दे॒वाना॒ꣳ॒ हृद॑येभ्यो॒\\
हृद॑येभ्यो दे॒वानां᳚ किरि॒केभ्यः॑ किरि॒केभ्यो॑ दे॒वाना॒ꣳ॒ हृद॑येभ्यः ।\\
\\
83. दे॒वाना᳚म् । हृद॑येभ्यः । नमः॑ ।\\
दे॒वाना॒ꣳ॒ हृद॑येभ्यो॒ हृद॑येभ्यो दे॒वानां᳚ दे॒वाना॒ꣳ॒ हृद॑येभ्यो॒ नमो॒ नमो॒\\
हृद॑येभ्यो दे॒वानां᳚ दे॒वाना॒ꣳ॒ हृद॑येभ्यो॒ नमः॑ ।\\
\\
84. हृद॑येभ्यः । नमः॑ । वि॒क्षी॒ण॒केभ्यः॑ ।\\
हृद॑येभ्यो॒ नमो॒ नमो॒ हृद॑येभ्यो॒ हृद॑येभ्यो॒ नमो॑ विक्षीण॒केभ्यो॑ विक्षीण॒केभ्यो॒\\
नमो॒ हृद॑येभ्यो॒ हृद॑येभ्यो॒ नमो॑ विक्षीण॒केभ्यः॑ ।\\
\\
85. नमः॑ । वि॒क्षी॒ण॒केभ्यः॑ । नमः॑ ।\\
नमो॑ विक्षीण॒केभ्यो॑ विक्षीण॒केभ्यो॒ नमो॒ नमो॑ विक्षीण॒केभ्यो॒ नमो॒ नमो॑\\
विक्षीण॒केभ्यो॒ नमो॒ नमो॑ विक्षीण॒केभ्यो॒ नमः॑ ।\\
\\
86. वि॒क्षी॒ण॒केभ्यः॑ । नमः॑ । वि॒चि॒न्व॒त्केभ्यः॑ ।\\
वि॒क्षी॒ण॒केभ्यो॒ नमो॒ नमो॑ विक्षीण॒केभ्यो॑ विक्षीण॒केभ्यो॒ नमो॑\\
विचिन्व॒त्केभ्यो॑ विचिन्व॒त्केभ्यो॒ नमो॑ विक्षीण॒केभ्यो॑ विक्षीण॒केभ्यो॒\\
नमो॑ विचिन्व॒त्केभ्यः॑ ।\\
\\
87. वि॒क्षी॒ण॒केभ्यः॑ ।\\
वि॒क्षी॒ण॒केभ्य॒ इति॑ वि - क्षी॒ण॒केभ्यः॑ ।\\
\subsection{\eng{Anuvaka 10}}
1. द्रापे᳚ । अन्ध॑सः । प॒ते॒ ।\\
द्रापे॒ अन्ध॑सो॒ अन्ध॑सो॒ द्रापे॒ द्रापे॒ अन्ध॑स स्पते प॒तेऽन्ध॑सो॒ द्रापे॒ द्रापे॒\\
अन्ध॑स स्पते ।\\
\\
2. अन्ध॑सः । प॒ते॒ । दरि॑द्रत् ।\\
अन्ध॑स स्पते प॒तेऽन्ध॑सो॒ अन्ध॑स स्पते॒ दरि॑द्र॒द् दरि॑द्रत् प॒तेऽन्ध॑सो॒\\
अन्ध॑स स्पते॒ दरि॑द्रत् ।\\
\\
3. प॒ते॒ । दरि॑द्रत् । नील॑लोहित ॥\\
प॒ते॒ दरि॑द्र॒द् दरि॑द्रत् पते पते॒ दरि॑द्र॒न् नील॑लोहित॒ नील॑लोहित॒ दरि॑द्रत्\\
पते पते॒ दरि॑द्र॒न् नील॑लोहित ।\\
\\
4. दरि॑द्रत् । नील॑लोहित ॥\\
दरि॑द्र॒न् नील॑लोहित॒ नील॑लोहित॒ दरि॑द्र॒द् दरि॑द्र॒न् नील॑लोहित ।\\
\\
5. नील॑लोहित ॥\\
नील॑लोहि॒तेति॒ नील॑ - लो॒हि॒त॒ ।\\
\\
6. ए॒षाम् । पुरु॑षाणाम् । ए॒षाम् ।\\
ए॒षां पुरु॑षाणां॒ पुरु॑षाणा मे॒षा मे॒षां पुरु॑षाणा मे॒षा मे॒षां पुरु॑षाणा मे॒षा\\
मे॒षां पुरु॑षाणा मे॒षाम् ।\\
\\
7. पुरु॑षाणाम् । ए॒षाम् । प॒शू॒नाम् ।\\
पुरु॑षाणा मे॒षा मे॒षां पुरु॑षाणां॒ पुरु॑षाणा मे॒षां प॑शू॒नां प॑शू॒ना मे॒षां\\
पुरु॑षाणां॒ पुरु॑षाणा मे॒षां प॑शू॒नाम् ।\\
\\
8. ए॒षाम् । प॒शू॒नाम् । मा ।\\
ए॒षां प॑शू॒नां प॑शू॒ना मे॒षा मे॒षां प॑शू॒नां मा मा प॑शू॒ना मे॒षा मे॒षां\\
प॑शू॒नां मा ।\\
\\
9. प॒शू॒नाम् । मा । भेः ।\\
प॒शू॒नां मा मा प॑शू॒नां प॑शू॒नां मा भेर् भेर् मा प॑शू॒नां प॑शू॒नां मा भेः ।\\
\\
10. मा । भेः । मा ।\\
मा भेर् भेर् मा मा भेर् मा मा भेर् मा मा भेर् मा ।\\
\\
11. भेः । मा । अ॒रः॒ ।\\
भेर् मा मा भेर् भेर् माऽरो॑ अरो॒ मा भेर् भेर् माऽरः॑ ।\\
\\
12. मा । अ॒रः॒ । मो ।\\
माऽरो॑ अरो॒ मा माऽरो॒ मो मो अ॑रो॒ मा माऽरो॒ मो ।\\
\\
13. अ॒रः॒ । मो । ए॒षा॒म् ।\\
अ॒रो॒ मो मो अ॑रो अरो॒ मो ए॑षा मेषां॒ मो अ॑रो अरो॒ मो ए॑षाम् ।\\
\\
14. मो । ए॒षा॒म् । किम् ।\\
मो ए॑षा मेषां॒ मो मो ए॑षां॒ किङ् किमे॑षां॒ मो मो ए॑षां॒ किम् ।\\
\\
15. मो ।\\
मो इति॒ मो ।\\
\\
16. ए॒षा॒म् । किम् । च॒न ।\\
ए॒षां॒ किङ् किमे॑षा मेषां॒ किञ्च॒न च॒न किमे॑षा मेषां॒ किञ्च॒न ।\\
\\
17. किम् । च॒न । आ॒म॒म॒त् ॥\\
किञ्च॒न च॒न किङ् किञ्च॒ना म॑मदा ममच् च॒न किङ्\\
किञ्च॒ना म॑मत् ।\\
\\
18. च॒न । आ॒म॒म॒त् ॥\\
च॒ना म॑मदा ममच् च॒न च॒ना म॑मत् ।\\
\\
19. आ॒म॒म॒त् ॥\\
आ॒म॒म॒दित्या॑ ममत् ।\\
\\
20. या । ते॒ । रु॒द्र॒ ।\\
या ते॑ ते॒ या या ते॑ रुद्र रुद्र ते॒ या या ते॑ रुद्र ।\\
\\
21. ते॒ । रु॒द्र॒ । शि॒वा ।\\
ते॒ रु॒द्र॒ रु॒द्र॒ ते॒ ते॒ रु॒द्र॒ शि॒वा शि॒वा रु॑द्र ते ते रुद्र शि॒वा ।\\
\\
22. रु॒द्र॒ । शि॒वा । त॒नूः ।\\
रु॒द्र॒ शि॒वा शि॒वा रु॑द्र रुद्र शि॒वा त॒नू स्त॒नूः शि॒वा रु॑द्र रुद्र शि॒वा त॒नूः ।\\
\\
23. शि॒वा । त॒नूः । शि॒वा ।\\
शि॒वा त॒नू स्त॒नूः शि॒वा शि॒वा त॒नूः शि॒वा शि॒वा त॒नूः शि॒वा\\
शि॒वा त॒नूः शि॒वा ।\\
\\
24. त॒नूः । शि॒वा । वि॒श्वाह॑भेषजी ॥\\
त॒नूः शि॒वा शि॒वा त॒नू स्त॒नूः शि॒वा वि॒श्वाह॑भेषजी वि॒श्वाह॑भेषजी शि॒वा\\
त॒नू स्त॒नूः शि॒वा वि॒श्वाह॑भेषजी ।\\
\\
25. शि॒वा । वि॒श्वाह॑भेषजी ॥\\
शि॒वा वि॒श्वाह॑भेषजी वि॒श्वाह॑भेषजी शि॒वा शि॒वा वि॒श्वाह॑भेषजी ।\\
\\
26. वि॒श्वाह॑भेषजी ॥\\
वि॒श्वाह॑भेष॒जीति॑ वि॒श्वाह॑ - भे॒ष॒जी॒ ।\\
\\
27. शि॒वा । रु॒द्रस्य॑ । भे॒ष॒जी ।\\
शि॒वा रु॒द्रस्य॑ रु॒द्रस्य॑ शि॒वा शि॒वा रु॒द्रस्य॑ भेष॒जी भे॑ष॒जी रु॒द्रस्य॑ शि॒वा\\
शि॒वा रु॒द्रस्य॑ भेष॒जी ।\\
\\
28. रु॒द्रस्य॑ । भे॒ष॒जी । तया᳚ ।\\
रु॒द्रस्य॑ भेष॒जी भे॑ष॒जी रु॒द्रस्य॑ रु॒द्रस्य॑ भेष॒जी तया॒ तया॑ भेष॒जी रु॒द्रस्य॑\\
रु॒द्रस्य॑ भेष॒जी तया᳚ ।\\
\\
29. भे॒ष॒जी । तया᳚ । नः॒ ।\\
भे॒ष॒जी तया॒ तया॑ भेष॒जी भे॑ष॒जी तया॑ नो न॒ स्तया॑ भेष॒जी भे॑ष॒जी\\
तया॑ नः ।\\
\\
30. तया᳚ । नः॒ । मृ॒ड॒ ।\\
तया॑ नो न॒ स्तया॒ तया॑ नो मृड मृड न॒ स्तया॒ तया॑ नो मृड ।\\
\\
31. नः॒ । मृ॒ड॒ । जी॒वसे᳚ ॥\\
नो॒ मृ॒ड॒ मृ॒ड॒ नो॒ नो॒ मृ॒ड॒ जी॒वसे॑ जी॒वसे॑ मृड नो नो मृड जी॒वसे᳚ ।\\
\\
32. मृ॒ड॒ । जी॒वसे᳚ ॥\\
मृ॒ड॒ जी॒वसे॑ जी॒वसे॑ मृड मृड जी॒वसे᳚ ।\\
\\
33. जी॒वसे᳚ ॥\\
जी॒वस॒ इति॑ जी॒वसे᳚ ।\\
\\
34. इ॒माम् । रु॒द्राय॑ । त॒वसे᳚ ।\\
इ॒माꣳ रु॒द्राय॑ रु॒द्रा ये॒मा मि॒माꣳ रु॒द्राय॑ त॒वसे॑ त॒वसे॑ रु॒द्रा ये॒मा मि॒माꣳ\\
रु॒द्राय॑ त॒वसे᳚ ।\\
\\
35. रु॒द्राय॑ । त॒वसे᳚ । क॒प॒र्दिने᳚ ।\\
रु॒द्राय॑ त॒वसे॑ त॒वसे॑ रु॒द्राय॑ रु॒द्राय॑ त॒वसे॑ कप॒र्दिने॑ कप॒र्दिने॑ त॒वसे॑ रु॒द्राय॑\\
रु॒द्राय॑ त॒वसे॑ कप॒र्दिने᳚ ।\\
\\
36. त॒वसे᳚ । क॒प॒र्दिने᳚ । क्ष॒यद्वी॑राय ।\\
त॒वसे॑ कप॒र्दिने॑ कप॒र्दिने॑ त॒वसे॑ त॒वसे॑ कप॒र्दिने᳚ क्ष॒यद्वी॑राय क्ष॒यद्वी॑राय\\
कप॒र्दिने॑ त॒वसे॑ त॒वसे॑ कप॒र्दिने᳚ क्ष॒यद्वी॑राय ।\\
\\
37. क॒प॒र्दिने᳚ । क्ष॒यद्वी॑राय । प्र ।\\
क॒प॒र्दिने᳚ क्ष॒यद्वी॑राय क्ष॒यद्वी॑राय कप॒र्दिने॑ कप॒र्दिने᳚ क्ष॒यद्वी॑राय॒ प्र प्र\\
क्ष॒यद्वी॑राय कप॒र्दिने॑ कप॒र्दिने᳚ क्ष॒यद्वी॑राय॒ प्र ।\\
\\
38. क्ष॒यद्वी॑राय । प्र । भ॒रा॒म॒हे॒ ।\\
क्ष॒यद्वी॑राय॒ प्र प्र क्ष॒यद्वी॑राय क्ष॒यद्वी॑राय॒ प्र भ॑रामहे भरामहे॒ प्र क्ष॒यद्वी॑राय\\
क्ष॒यद्वी॑राय॒ प्र भ॑रामहे ।\\
\\
39. क्ष॒यद्वी॑राय ।\\
क्ष॒यद्वी॑रा॒येति॑ क्ष॒यत् - वी॒रा॒य॒ ।\\
\\
40. प्र । भ॒रा॒म॒हे॒ । म॒तिम् ॥\\
प्र भ॑रामहे भरामहे॒ प्र प्र भ॑रामहे म॒तिं म॒तिं भ॑रामहे॒ प्र प्र\\
भ॑रामहे म॒तिम् ।\\
\\
41. भ॒रा॒म॒हे॒ । म॒तिम् ॥\\
भ॒रा॒म॒हे॒ म॒तिं म॒तिं भ॑रामहे भरामहे म॒तिम् ।\\
\\
42. म॒तिम् ॥\\
म॒तिमिति॑ म॒तिम् ।\\
\\
43. यथा᳚ । नः॒ । शम् ।\\
यथा॑ नो नो॒ यथा॒ यथा॑ नः॒ शꣳ शन्नो॒ यथा॒ यथा॑ नः॒ शम् ।\\
\\
44. नः॒ । शम् । अस॑त् ।\\
नः॒ शꣳ शन्नो॑ नः॒ शमस॒ दस॒च्छन्नो॑ नः॒ शमस॑त् ।\\
\\
45. शम् । अस॑त् । द्वि॒पदे᳚ ।\\
शमस ॒दस॒च्छꣳ शमस॑द् द्वि॒पदे᳚ द्वि॒पदे॒ अस॒च्छꣳ शमस॑द् द्वि॒पदे᳚ ।\\
\\
46. अस॑त् । द्वि॒पदे᳚ । चतु॑ष्पदे ।\\
अस॑द् द्वि॒पदे᳚ द्वि॒पदे॒ अस॒ दस॑द् द्वि॒पदे॒ चतु॑ष्पदे॒ चतु॑ष्पदे द्वि॒पदे॒\\
अस॒ दस॑द् द्वि॒पदे॒ चतु॑ष्पदे ।\\
\\
47. द्वि॒पदे᳚ । चतु॑ष्पदे । विश्व᳚म् ।\\
द्वि॒पदे॒ चतु॑ष्पदे॒ चतु॑ष्पदे द्वि॒पदे᳚ द्वि॒पदे॒ चतु॑ष्पदे॒ विश्वं॒ँविश्व॒ञ्चतु॑ष्पदे\\
द्वि॒पदे᳚ द्वि॒पदे॒ चतु॑ष्पदे॒ विश्व᳚म् ।\\
\\
48. द्वि॒पदे᳚ ।\\
द्वि॒पद॒ इति॑ द्वि - पदे᳚ ।\\
\\
49. चतु॑ष्पदे । विश्व᳚म् । पु॒ष्टम् ।\\
चतु॑ष्पदे॒ विश्वं॒ँविश्व॒ञ्चतु॑ष्पदे॒ चतु॑ष्पदे॒ विश्वं॑ पु॒ष्टं पु॒ष्टंँविश्व॒ञ्चतु॑ष्पदे॒\\
चतु॑ष्पदे॒ विश्वं॑ पु॒ष्टम् ।\\
\\
50. चतु॑ष्पदे ।\\
चतु॑ष्पद॒ इति॒ चतुः॑ - प॒दे॒ ।\\
\\
51. विश्व᳚म् । पु॒ष्टम् । ग्रामे᳚ ।\\
विश्वं॑ पु॒ष्टं पु॒ष्टंँविश्वं॒ँविश्वं॑ पु॒ष्टं ग्रामे॒ ग्रामे॑ पु॒ष्टंँविश्वं॒ँविश्वं॑ पु॒ष्टं ग्रामे᳚ ।\\
\\
52. पु॒ष्टम् । ग्रामे᳚ । अ॒स्मिन्न् ।\\
पु॒ष्टं ग्रामे॒ ग्रामे॑ पु॒ष्टं पु॒ष्टं ग्रामे॑ अ॒स्मिन् न॒स्मिन् ग्रामे॑ पु॒ष्टं पु॒ष्टं ग्रामे॑\\
अ॒स्मिन्न् ।\\
\\
53. ग्रामे᳚ । अ॒स्मिन्न् । अना॑तुरम् ॥\\
ग्रामे॑ अ॒स्मिन् न॒स्मिन् ग्रामे॒ ग्रामे॑ अ॒स्मिन् नना॑तुर॒ मना॑तुर म॒स्मिन् ग्रामे॒\\
ग्रामे॑ अ॒स्मिन् नना॑तुरम् ।\\
\\
54. अ॒स्मिन्न् । अना॑तुरम् ॥\\
अ॒स्मिन् नना॑तुर॒ मना॑तुर म॒स्मिन् न॒स्मिन् नना॑तुरम् ।\\
\\
55. अना॑तुरम् ॥\\
अना॑तुर॒ मित्यना᳚ - तु॒र॒म् ।\\
\\
56. मृ॒ड । नः॒ । रु॒द्र॒ ।\\
मृ॒डा नो॑ नो मृ॒ड मृ॒डा नो॑ रुद्र रुद्र नो मृ॒ड मृ॒डा नो॑ रुद्र ।\\
\\
57. नः॒ । रु॒द्र॒ । उ॒त ।\\
नो॒ रु॒द्र॒ रु॒द्र॒ नो॒ नो॒ रु॒द्रो॒तोत रु॑द्र नो नो रुद्रो॒त ।\\
\\
58. रु॒द्र॒ । उ॒त । नः॒ ।\\
रु॒द्रो॒तोत रु॑द्र रुद्रो॒त नो॑ न उ॒त रु॑द्र रुद्रो॒त नः॑ ।\\
\\
59. उ॒त । नः॒ । मयः॑ ।\\
उ॒त नो॑ न उ॒तोत नो॒ मयो॒ मयो॑ न उ॒तोत नो॒ मयः॑ ।\\
\\
60. नः॒ । मयः॑ । कृ॒धि॒ ।\\
नो॒ मयो॒ मयो॑ नो नो॒ मय॑ स्कृधि कृधि॒ मयो॑ नो नो॒ मय॑ स्कृधि ।\\
\\
61. मयः॑ । कृ॒धि॒ । क्ष॒यद्वी॑राय ।\\
मय॑ स्कृधि कृधि॒ मयो॒ मय॑ स्कृधि क्ष॒यद्वी॑राय क्ष॒यद्वी॑राय कृधि॒ मयो॒\\
मय॑ स्कृधि क्ष॒यद्वी॑राय ।\\
\\
62. कृ॒धि॒ । क्ष॒यद्वी॑राय । नम॑सा ।\\
कृ॒धि॒ क्ष॒यद्वी॑राय क्ष॒यद्वी॑राय कृधि कृधि क्ष॒यद्वी॑राय॒ नम॑सा॒ नम॑सा\\
क्ष॒यद्वी॑राय कृधि कृधि क्ष॒यद्वी॑राय॒ नम॑सा ।\\
\\
63. क्ष॒यद्वी॑राय । नम॑सा । वि॒धे॒म॒ ।\\
क्ष॒यद्वी॑राय॒ नम॑सा॒ नम॑सा क्ष॒यद्वी॑राय क्ष॒यद्वी॑राय॒ नम॑सा विधेम विधेम॒\\
नम॑सा क्ष॒यद्वी॑राय क्ष॒यद्वी॑राय॒ नम॑सा विधेम ।\\
\\
64. क्ष॒यद्वी॑राय ।\\
क्ष॒यद्वी॑रा॒येति॑ क्ष॒यत् - वी॒रा॒य॒ ।\\
\\
65. नम॑सा । वि॒धे॒म॒ । ते॒ ॥\\
नम॑सा विधेम विधेम॒ नम॑सा॒ नम॑सा विधेम ते ते विधेम॒ नम॑सा॒ नम॑सा\\
विधेम ते ।\\
\\
66. वि॒धे॒म॒ । ते॒ ॥\\
वि॒धे॒म॒ ते॒ ते॒ वि॒धे॒म॒ वि॒धे॒म॒ ते॒ ।\\
\\
67. ते॒ ॥\\
त॒ इति॑ ते ।\\
\\
68. यत् । शम् । च॒ ।\\
यच्छꣳ शंँयद् यच्छञ्च॑ च॒ शंँयद् यच्छञ्च॑ ।\\
\\
69. शम् । च॒ । योः ।\\
शञ्च॑ च॒ शꣳ शञ्च॒ योर् योश्च॒ शꣳ शञ्च॒ योः ।\\
\\
70. च॒ । योः । च॒ ।\\
च॒ योर् योश्च॑ च॒ योश्च॑ च॒ योश्च॑ च॒ योश्च॑ ।\\
\\
71. योः । च॒ । मनुः॑ ।\\
योश्च॑ च॒ योर् योश्च॒ मनु॒र् मनु॑श्च॒ योर् योश्च॒ मनुः॑ ।\\
\\
72. च॒ । मनुः॑ । आ॒य॒जे ।\\
च॒ मनु॒र् मनु॑श्च च॒ मनु॑ राय॒ज आ॑य॒जे मनु॑श्च च॒ मनु॑ राय॒जे ।\\
\\
73. मनुः॑ । आ॒य॒जे । पि॒ता ।\\
मनु॑ राय॒ज आ॑य॒जे मनु॒र् मनु॑ राय॒जे पि॒ता पि॒ताऽऽय॒जे मनु॒र् मनु॑ राय॒जे पि॒ता ।\\
\\
74. आ॒य॒जे । पि॒ता । तत् ।\\
आ॒य॒जे पि॒ता पि॒ताऽऽय॒ज आ॑य॒जे पि॒ता तत् तत् पि॒ताऽऽय॒ज आ॑य॒जे\\
पि॒ता तत् ।\\
\\
75. आ॒य॒जे ।\\
आ॒य॒ज इत्या᳚ - य॒जे ।\\
\\
76. पि॒ता । तत् । अ॒श्या॒म॒ ।\\
पि॒ता तत् तत् पि॒ता पि॒ता तद॑श्यामा श्याम॒ तत् पि॒ता पि॒ता तद॑श्याम ।\\
\\
77. तत् । अ॒श्या॒म॒ । तव॑ ।\\
तद॑श्यामा श्याम॒ तत् तद॑श्याम॒ तव॒ तवा᳚श्याम॒ तत् तद॑श्याम॒ तव॑ ।\\
\\
78. अ॒श्या॒म॒ । तव॑ । रु॒द्र॒ ।\\
अ॒श्या॒म॒ तव॒ तवा᳚श्यामा श्याम॒ तव॑ रुद्र रुद्र॒ तवा᳚श्यामा श्याम॒ तव॑ रुद्र ।\\
\\
79. तव॑ । रु॒द्र॒ । प्रणी॑तौ ॥\\
तव॑ रुद्र रुद्र॒ तव॒ तव॑ रुद्र॒ प्रणी॑तौ॒ प्रणी॑तौ रुद्र॒ तव॒ तव॑ रुद्र॒ प्रणी॑तौ ।\\
\\
80. रु॒द्र॒ । प्रणी॑तौ ॥\\
रु॒द्र॒ प्रणी॑तौ॒ प्रणी॑तौ रुद्र रुद्र॒ प्रणी॑तौ ।\\
\\
81. प्रणी॑तौ ॥\\
प्रणी॑ता॒विति॒ प्र - नी॒तौ॒ ।\\
\\
82. मा । नः॒ । म॒हान्त᳚म् ।\\
मा नो॑ नो॒ मा मा नो॑ म॒हान्तं॑ म॒हान्तं॑ नो॒ मा मा नो॑ म॒हान्त᳚म् ।\\
\\
83. नः॒ । म॒हान्त᳚म् । उ॒त ।\\
नो॒ म॒हान्तं॑ म॒हान्तं॑ नो नो म॒हान्त॑ मु॒तोत म॒हान्तं॑ नो नो म॒हान्त॑ मु॒त ।\\
\\
84. म॒हान्त᳚म् । उ॒त । मा ।\\
म॒हान्त॑ मु॒तोत म॒हान्तं॑ म॒हान्त॑ मु॒त मा मोत म॒हान्तं॑ म॒हान्त॑ मु॒त मा ।\\
\\
85. उ॒त । मा । नः॒ ।\\
उ॒त मा मोतोत मा नो॑ नो॒ मोतोत मा नः॑ ।\\
\\
86. मा । नः॒ । अ॒र्भ॒कम् ।\\
मा नो॑ नो॒ मा मा नो॑ अर्भ॒क म॑र्भ॒कं नो॒ मा मा नो॑ अर्भ॒कम् ।\\
\\
87. नः॒ । अ॒र्भ॒कम् । मा ।\\
नो॒ अ॒र्भ॒क म॑र्भ॒कं नो॑ नो अर्भ॒कं मा माऽर्भ॒कं नो॑ नो अर्भ॒कं मा ।\\
\\
88. अ॒र्भ॒कम् । मा । नः॒ ।\\
अ॒र्भ॒कं मा माऽर्भ॒क म॑र्भ॒कं मा नो॑ नो॒ माऽर्भ॒क म॑र्भ॒कं मा नः॑ ।\\
\\
89. मा । नः॒ । उक्ष॑न्तम् ।\\
मा नो॑ नो॒ मा मा न॒ उक्ष॑न्त॒ मुक्ष॑न्तं नो॒ मा मा न॒ उक्ष॑न्तम् ।\\
\\
90. नः॒ । उक्ष॑न्तम् । उ॒त ।\\
न॒ उक्ष॑न्त॒ मुक्ष॑न्तं नो न॒ उक्ष॑न्त मु॒तो तोक्ष॑न्तं नो न॒ उक्ष॑न्त मु॒त ।\\
\\
91. उक्ष॑न्तम् । उ॒त । मा ।\\
उक्ष॑न्त मु॒तो तोक्ष॑न्त॒ मुक्ष॑न्त मु॒त मा मो तोक्ष॑न्त॒ मुक्ष॑न्त मु॒त मा ।\\
\\
92. उ॒त । मा । नः॒ ।\\
उ॒त मा मोतोत मा नो॑ नो॒ मो तोत मा नः॑ ।\\
\\
93. मा । नः॒ । उ॒क्षि॒तम् ॥\\
मा नो॑ नो॒ मा मा न॑ उक्षि॒त मु॑क्षि॒तं नो॒ मा मा न॑ उक्षि॒तम् ।\\
\\
94. नः॒ । उ॒क्षि॒तम् ॥\\
न॒ उ॒क्षि॒त मु॑क्षि॒तं नो॑ न उक्षि॒तम् ।\\
\\
95. उ॒क्षि॒तम् ॥\\
उ॒क्षि॒तमित्यु॑ क्षि॒तम् ।\\
\\
96. मा । नः॒ । व॒धीः॒ ।\\
मा नो॑ नो॒ मा मा नो॑ वधीर् वधीर् नो॒ मा मा नो॑ वधीः ।\\
\\
97. नः॒ । व॒धीः॒ । पि॒तर᳚म् ।\\
नो॒ व॒धी॒र् व॒धी॒र् नो॒ नो॒ व॒धीः॒ पि॒तरं॑ पि॒तरं॑ँवधीर् नो नो वधीः पि॒तर᳚म् ।\\
\\
98. व॒धीः॒ । पि॒तर᳚म् । मा ।\\
व॒धीः॒ पि॒तरं॑ पि॒तरं॑ँवधीर् वधीः पि॒तरं॒ मा मा पि॒तरं॑ँवधीर्\\
वधीः पि॒तरं॒ मा ।\\
\\
99. पि॒तर᳚म् । मा । उ॒त ।\\
पि॒तरं॒ मा मा पि॒तरं॑ पि॒तरं॒ मोतोत मा पि॒तरं॑ पि॒तरं॒ मोत ।\\
\\
100. मा । उ॒त । मा॒तर᳚म् ।\\
मोतोत मा मोत मा॒तरं॑ मा॒तर॑ मु॒तमा मोत मा॒तर᳚म् ।\\
\\
101. उ॒त । मा॒तर᳚म् । प्रि॒याः ।\\
उ॒त मा॒तरं॑ मा॒तर॑ मु॒तोत मा॒तरं॑ प्रि॒याः प्रि॒या मा॒तर॑ मु॒तोत मा॒तरं॑ प्रि॒याः ।\\
\\
102. मा॒तर᳚म् । प्रि॒याः । मा ।\\
मा॒तरं॑ प्रि॒याः प्रि॒या मा॒तरं॑ मा॒तरं॑ प्रि॒या मा मा प्रि॒या मा॒तरं॑\\
मा॒तरं॑ प्रि॒या मा ।\\
\\
103. प्रि॒याः । मा । नः॒ ।\\
प्रि॒या मा मा प्रि॒याः प्रि॒या मा नो॑ नो॒ मा प्रि॒याः प्रि॒या मा नः॑ ।\\
\\
104. मा । नः॒ । त॒नुवः॑ ।\\
मा नो॑ नो॒ मा मा न॑ स्त॒नुव॑ स्त॒नुवो॑ नो॒ मा मा न॑ स्त॒नुवः॑ ।\\
\\
105. नः॒ । त॒नुवः॑ । रु॒द्र॒ ।\\
न॒ स्त॒नुव॑ स्त॒नुवो॑ नो न स्त॒नुवो॑ रुद्र रुद्र त॒नुवो॑ नो न स्त॒नुवो॑ रुद्र ।\\
\\
106. त॒नुवः॑ । रु॒द्र॒ । री॒रि॒षः॒ ॥\\
त॒नुवो॑ रुद्र रुद्र त॒नुव॑ स्त॒नुवो॑ रुद्र रीरिषो रीरिषो रुद्र त॒नुव॑ स्त॒नुवो॑\\
रुद्र रीरिषः ।\\
\\
107. रु॒द्र॒ । री॒रि॒षः॒ ॥\\
रु॒द्र॒ री॒रि॒षो॒ री॒रि॒षो॒ रु॒द्र॒ रु॒द्र॒ री॒रि॒षः॒ ।\\
\\
108. री॒रि॒षः॒ ॥\\
री॒रि॒ष॒ इति॑ रीरिषः ।\\
\\
109. मा । नः॒ । तो॒के ।\\
मा नो॑ नो॒ मा मा न॑ स्तो॒के तो॒के नो॒ मा मा न॑ स्तो॒के ।\\
\\
110. नः॒ । तो॒के । तन॑ये ।\\
न॒ स्तो॒के तो॒के नो॑ न स्तो॒के तन॑ये॒ तन॑ये तो॒के नो॑ न स्तो॒के तन॑ये ।\\
\\
111. तो॒के । तन॑ये । मा ।\\
तो॒के तन॑ये॒ तन॑ये तो॒के तो॒के तन॑ये॒ मा मा तन॑ये तो॒के तो॒के तन॑ये॒ मा ।\\
\\
112. तन॑ये । मा । नः॒ ।\\
तन॑ये॒ मा मा तन॑ये॒ तन॑ये॒ मा नो॑ नो॒ मा तन॑ये॒ तन॑ये॒ मा नः॑ ।\\
\\
113. मा । नः॒ । आयु॑षि ।\\
मा नो॑ नो॒ मा मा न॒ आयु॒ ष्यायु॑षि नो॒ मा मा न॒ आयु॑षि ।\\
\\
114. नः॒ । आयु॑षि । मा ।\\
न॒ आयु॒ ष्यायु॑षि नो न॒ आयु॑षि॒ मा माऽऽयु॑षि नो न॒ आयु॑षि॒ मा ।\\
\\
115. आयु॑षि । मा । नः॒ ।\\
आयु॑षि॒ मा माऽऽयु॒ ष्यायु॑षि॒ मा नो॑ नो॒ माऽऽयु॒ ष्यायु॑षि॒ मा नः॑ ।\\
\\
116. मा । नः॒ । गोषु॑ ।\\
मा नो॑ नो॒ मा मा नो॒ गोषु॒ गोषु॑ नो॒ मा मा नो॒ गोषु॑ ।\\
\\
117. नः॒ । गोषु॑ । मा ।\\
नो॒ गोषु॒ गोषु॑ नो नो॒ गोषु॒ मा मा गोषु॑ नो नो॒ गोषु॒ मा ।\\
\\
118. गोषु॑ । मा । नः॒ ।\\
गोषु॒ मा मा गोषु॒ गोषु॒ मा नो॑ नो॒ मा गोषु॒ गोषु॒ मा नः॑ ।\\
\\
119. मा । नः॒ । अश्वे॑षु ।\\
मा नो॑ नो॒ मा मा नो॒ अश्वे॒ ष्वश्वे॑षु नो॒ मा मा नो॒ अश्वे॑षु ।\\
\\
120. नः॒ । अश्वे॑षु । री॒रि॒षः॒ ॥\\
नो॒ अश्वे॒ ष्वश्वे॑षु नो नो॒ अश्वे॑षु रीरिषो रीरिषो॒ अश्वे॑षु नो नो॒ अश्वे॑षु\\
रीरिषः ।\\
\\
121. अश्वे॑षु । री॒रि॒षः॒ ॥\\
अश्वे॑षु रीरिषो रीरिषो॒ अश्वे॒ ष्वश्वे॑षु रीरिषः ।\\
\\
122. री॒रि॒षः॒ ॥\\
री॒रि॒ष॒ इति॑ रीरिषः ।\\
\\
123. वी॒रान् । मा । नः॒ ।\\
वी॒रान् मा मा वी॒रान्. वी॒रान् मा नो॑ नो॒ मा वी॒रान्. वी॒रान् मा नः॑ ।\\
\\
124. मा । नः॒ । रु॒द्र॒ ।\\
मा नो॑ नो॒ मा मा नो॑ रुद्र रुद्र नो॒ मा मा नो॑ रुद्र ।\\
\\
125. नः॒ । रु॒द्र॒ । भा॒मि॒तः ।\\
नो॒ रु॒द्र॒ रु॒द्र॒ नो॒ नो॒ रु॒द्र॒ भा॒मि॒तो भा॑मि॒तो रु॑द्र नो नो रुद्र भामि॒तः ।\\
\\
126. रु॒द्र॒ । भा॒मि॒तः । व॒धीः॒ ।\\
रु॒द्र॒ भा॒मि॒तो भा॑मि॒तो रु॑द्र रुद्र भामि॒तो व॑धीर् वधीर् भामि॒तो रु॑द्र रुद्र\\
भामि॒तो व॑धीः ।\\
\\
127. भा॒मि॒तः । व॒धीः॒ । ह॒विष्म॑न्तः ।\\
भा॒मि॒तो व॑धीर् वधीर् भामि॒तो भा॑मि॒तो व॑धीर्. ह॒विष्म॑न्तो ह॒विष्म॑न्तो\\
वधीर् भामि॒तो भा॑मि॒तो व॑धीर्. ह॒विष्म॑न्तः ।\\
\\
128. व॒धीः॒ । ह॒विष्म॑न्तः । नम॑सा ।\\
व॒धी॒र्. ह॒विष्म॑न्तो ह॒विष्म॑न्तो वधीर् वधीर्. ह॒विष्म॑न्तो॒ नम॑सा॒ नम॑सा\\
ह॒विष्म॑न्तो वधीर् वधीर्. ह॒विष्म॑न्तो॒ नम॑सा ।\\
\\
129. ह॒विष्म॑न्तः । नम॑सा । वि॒धे॒म॒ ।\\
ह॒विष्म॑न्तो॒ नम॑सा॒ नम॑सा ह॒विष्म॑न्तो ह॒विष्म॑न्तो॒ नम॑सा विधेम विधेम॒\\
नम॑सा ह॒विष्म॑न्तो ह॒विष्म॑न्तो॒ नम॑सा विधेम ।\\
\\
130. नम॑सा । वि॒धे॒म॒ । ते॒ ॥\\
नम॑सा विधेम विधेम॒ नम॑सा॒ नम॑सा विधेम ते ते विधेम॒ नम॑सा॒ नम॑सा\\
विधेम ते ।\\
\\
131. वि॒धे॒म॒ । ते॒ ॥\\
वि॒धे॒म॒ ते॒ ते॒ वि॒धे॒म॒ वि॒धे॒म॒ ते॒ ।\\
\\
132. ते॒ ॥\\
त॒ इति॑ ते ।\\
\\
133. आ॒रात् । ते॒ । गो॒घ्ने ।\\
आ॒रात्ते॑ त आ॒रा दा॒रात्ते॑ गो॒घ्ने गो॒घ्ने त॑ आ॒रा दा॒रात्ते॑ गो॒घ्ने ।\\
\\
134. ते॒ । गो॒घ्ने । उ॒त ।\\
ते॒ गो॒घ्ने गो॒घ्ने ते॑ ते गो॒घ्न उ॒तोत गो॒घ्ने ते॑ ते गो॒घ्न उ॒त ।\\
\\
135. गो॒घ्ने । उ॒त । पू॒रु॒ष॒घ्ने ।\\
गो॒घ्न उ॒तोत गो॒घ्ने गो॒घ्न उ॒त पू॑रुष॒घ्ने पू॑रुष॒घ्न उ॒त गो॒घ्ने गो॒घ्न उ॒त\\
पू॑रुष॒घ्ने ।\\
\\
136. गो॒घ्ने ।\\
गो॒घ्न इति॑ गो - घ्ने ।\\
\\
137. उ॒त । पू॒रु॒ष॒घ्ने । क्ष॒यद्वी॑राय ।\\
उ॒त पू॑रुष॒घ्ने पू॑रुष॒घ्न उ॒तोत पू॑रुष॒घ्ने क्ष॒यद्वी॑राय क्ष॒यद्वी॑राय पूरुष॒घ्न उ॒तोत\\
पू॑रुष॒घ्ने क्ष॒यद्वी॑राय ।\\
\\
138. पू॒रु॒ष॒घ्ने । क्ष॒यद्वी॑राय । सु॒म्नम् ।\\
पू॒रु॒ष॒घ्ने क्ष॒यद्वी॑राय क्ष॒यद्वी॑राय पूरुष॒घ्ने पू॑रुष॒घ्ने क्ष॒यद्वी॑राय सु॒म्नꣳ सु॒म्नं\\
क्ष॒यद्वी॑राय पूरुष॒घ्ने पू॑रुष॒घ्ने क्ष॒यद्वी॑राय सु॒म्नम् ।\\
\\
139. पू॒रु॒ष॒घ्ने ।\\
पू॒रु॒ष॒घ्न इति॑ पूरुष - घ्ने ।\\
\\
140. क्ष॒यद्वी॑राय । सु॒म्नम् । अ॒स्मे ।\\
क्ष॒यद्वी॑राय सु॒म्नꣳ सु॒म्नं क्ष॒यद्वी॑राय क्ष॒यद्वी॑राय सु॒म्न म॒स्मे अ॒स्मे सु॒म्नं\\
क्ष॒यद्वी॑राय क्ष॒यद्वी॑राय सु॒म्न म॒स्मे ।\\
\\
141. क्ष॒यद्वी॑राय ।\\
क्ष॒यद्वी॑रा॒येति॑ क्ष॒यत् - वी॒रा॒य॒ ।\\
\\
142. सु॒म्नम् । अ॒स्मे । ते॒ ।\\
सु॒म्न म॒स्मे अ॒स्मे सु॒म्नꣳ सु॒म्न म॒स्मे ते॑ ते अ॒स्मे सु॒म्नꣳ सु॒म्न\\
म॒स्मे ते᳚ ।\\
\\
143. अ॒स्मे । ते॒ । अ॒स्तु॒ ॥\\
अ॒स्मे ते॑ ते अ॒स्मे अ॒स्मे ते॑ अस्त्वस्तु ते अ॒स्मे अ॒स्मे ते॑ अस्तु ।\\
\\
144. अ॒स्मे ।\\
अ॒स्मे इत्य॒स्मे ।\\
\\
145. ते॒ । अ॒स्तु॒ ॥\\
ते॒ अ॒स्त्व॒स्तु॒ ते॒ ते॒ अ॒स्तु॒ ।\\
\\
146. अ॒स्तु॒ ॥\\
अ॒स्त्वित्य॑स्तु ।\\
\\
147. रक्ष॑ । च॒ । नः॒ ।\\
रक्षा॑ च च॒ रक्ष॒ रक्षा॑ च नो नश्च॒ रक्ष॒ रक्षा॑ च नः ।\\
\\
148. च॒ । नः॒ । अधि॑ ।\\
च॒ नो॒ न॒श्च॒ च॒ नो॒ अध्यधि॑ नश्च च नो॒ अधि॑ ।\\
\\
149. नः॒ । अधि॑ । च॒ ।\\
नो॒ अध्यधि॑ नो नो॒ अधि॑ च॒ चाधि॑ नो नो॒ अधि॑ च ।\\
\\
150. अधि॑ । च॒ । दे॒व॒ ।\\
अधि॑ च॒ चाध्यधि॑ च देव देव॒ चाध्यधि॑ च देव ।\\
\\
151. च॒ । दे॒व॒ । ब्रू॒हि॒ ।\\
च॒ दे॒व॒ दे॒व॒ च॒ च॒ दे॒व॒ ब्रू॒हि॒ ब्रू॒हि॒ दे॒व॒ च॒ च॒ दे॒व॒ ब्रू॒हि॒ ।\\
\\
152. दे॒व॒ । ब्रू॒हि॒ । अध॑ ।\\
दे॒व॒ ब्रू॒हि॒ ब्रू॒हि॒ दे॒व॒ दे॒व॒ ब्रू॒ह्यधाध॑ ब्रूहि देव देव ब्रू॒ह्यध॑ ।\\
\\
153. ब्रू॒हि॒ । अध॑ । च॒ ।\\
ब्रू॒ह्यधाध॑ ब्रूहि ब्रू॒ह्यधा॑ च॒ चाध॑ ब्रूहि ब्रू॒ह्यधा॑ च ।\\
\\
154. अध॑ । च॒ । नः॒ ।\\
अधा॑ च॒ चा धा धा॑ च नो न॒श्चा धा धा॑ च नः ।\\
\\
155. च॒ । नः॒ । शर्म॑ ।\\
च॒ नो॒ न॒श्च॒ च॒ नः॒ शर्म॒ शर्म॑ नश्च च नः॒ शर्म॑ ।\\
\\
156. नः॒ । शर्म॑ । य॒च्छ॒ ।\\
नः॒ शर्म॒ शर्म॑ नो नः॒ शर्म॑ यच्छ यच्छ॒ शर्म॑ नो नः॒ शर्म॑ यच्छ ।\\
\\
157. शर्म॑ । य॒च्छ॒ । द्वि॒बर्.हाः᳚ ॥\\
शर्म॑ यच्छ यच्छ॒ शर्म॒ शर्म॑ यच्छ द्वि॒बर्.हा᳚ द्वि॒बर्.हा॑ यच्छ॒ शर्म॒ शर्म॑\\
यच्छ द्वि॒बर्.हाः᳚ ।\\
\\
158. य॒च्छ॒ । द्वि॒बर्.हाः᳚ ॥\\
य॒च्छ॒ द्वि॒बर्.हा᳚ द्वि॒बर्.हा॑ यच्छ यच्छ द्वि॒बर्.हाः᳚ ।\\
\\
159. द्वि॒बर्.हाः᳚ ॥\\
द्वि॒बर्.हा॒ इति॑ द्वि - बर्.हाः᳚ ।\\
\\
160. स्तु॒हि । श्रु॒तम् । ग॒र्त॒सद᳚म् ।\\
स्तु॒हि श्रु॒तꣲ श्रु॒तꣲ स्तु॒हि स्तु॒हि श्रु॒तं ग॑र्त॒सदं॑ गर्त॒सदꣲ॑ श्रु॒तꣲ स्तु॒हि\\
स्तु॒हि श्रु॒तं ग॑र्त॒सद᳚म् ।\\
\\
161. श्रु॒तम् । ग॒र्त॒सद᳚म् । युवा॑नम् ।\\
श्रु॒तङ् ग॑र्त॒सद॑ङ् गर्त॒सदꣲ॑ श्रु॒तꣲ श्रु॒तङ् ग॑र्त॒सदं॒ँयुवा॑नं॒ँयुवा॑नङ्\\
गर्त॒सदꣲ॑ श्रु॒तꣲ श्रु॒तङ् ग॑र्त॒सदं॒ँयुवा॑नम् ।\\
\\
162. ग॒र्त॒सद᳚म् । युवा॑नम् । मृ॒गम् ।\\
ग॒र्त॒सदं॒ँयुवा॑नं॒ँयुवा॑नं गर्त॒सदं॑ गर्त॒सदं॒ँयुवा॑नं मृ॒गं मृ॒गंँयुवा॑नं गर्त॒सदं॑\\
गर्त॒सदं॒ँयुवा॑नं मृ॒गम् ।\\
\\
163. ग॒र्त॒सद᳚म् ।\\
ग॒र्त॒सद॒ मिति॑ गर्त - सद᳚म् ।\\
\\
164. युवा॑नम् । मृ॒गम् । न ।\\
युवा॑नं मृ॒गं मृ॒गंँयुवा॑नं॒ँयुवा॑नं मृ॒गं न न मृ॒गंँयुवा॑नं॒ँयुवा॑नं मृ॒गं न ।\\
\\
165. मृ॒गम् । न । भी॒मम् ।\\
मृ॒गं न न मृ॒गं मृ॒गं न भी॒मं भी॒मं न मृ॒गं मृ॒गं न भी॒मम् ।\\
\\
166. न । भी॒मम् । उ॒प॒ह॒त्नुम् ।\\
न भी॒मं भी॒मं न न भी॒म मु॑पह॒त्नु मु॑पह॒त्नुं भी॒मं न न भी॒म मु॑पह॒त्नुम् ।\\
\\
167. भी॒मम् । उ॒प॒ह॒त्नुम् । उ॒ग्रम् ॥\\
भी॒म मु॑पह॒त्नु मु॑पह॒त्नुं भी॒मं भी॒म मु॑पह॒त्नु मु॒ग्र मु॒ग्र मु॑पह॒त्नुं भी॒मं\\
भी॒म मु॑पह॒त्नु मु॒ग्रम् ।\\
\\
168. उ॒प॒ह॒त्नुम् । उ॒ग्रम् ॥\\
उ॒प॒ह॒त्नु मु॒ग्र मु॒ग्र मु॑पह॒त्नु मु॑पह॒त्नु मु॒ग्रम् ।\\
\\
169. उ॒ग्रम् ॥\\
उ॒ग्रमित्यु॒ग्रम् ।\\
\\
170. मृ॒ड । ज॒रि॒त्रे । रु॒द्र॒ ।\\
मृ॒डा ज॑रि॒त्रे ज॑रि॒त्रे मृ॒ड मृ॒डा ज॑रि॒त्रे रु॑द्र रुद्र जरि॒त्रे मृ॒ड मृ॒डा\\
ज॑रि॒त्रे रु॑द्र ।\\
\\
171. ज॒रि॒त्रे । रु॒द्र॒ । स्तवा॑नः ।\\
ज॒रि॒त्रे रु॑द्र रुद्र जरि॒त्रे ज॑रि॒त्रे रु॑द्र॒ स्तवा॑नः॒ स्तवा॑नो रुद्र जरि॒त्रे ज॑रि॒त्रे\\
रु॑द्र॒ स्तवा॑नः ।\\
\\
172. रु॒द्र॒ । स्तवा॑नः । अ॒न्यम् ।\\
रु॒द्र॒ स्तवा॑नः॒ स्तवा॑नो रुद्र रुद्र॒ स्तवा॑नो अ॒न्य म॒न्यꣲ स्तवा॑नो रुद्र रुद्र॒\\
स्तवा॑नो अ॒न्यम् ।\\
\\
173. स्तवा॑नः । अ॒न्यम् । ते॒ ।\\
स्तवा॑नो अ॒न्य म॒न्यꣲ स्तवा॑नः॒ स्तवा॑नो अ॒न्यन्ते॑ ते अ॒न्यꣲ स्तवा॑नः॒\\
स्तवा॑नो अ॒न्यन्ते᳚ ।\\
\\
174. अ॒न्यम् । ते॒ । अ॒स्मत् ।\\
अ॒न्यन्ते॑ ते अ॒न्य म॒न्यन्ते॑ अ॒स्म द॒स्मत् ते॑ अ॒न्य म॒न्यन्ते॑ अ॒स्मत् ।\\
\\
175. ते॒ । अ॒स्मत् । नि ।\\
ते॒ अ॒स्म द॒स्मत् ते॑ ते अ॒स्मन् नि न्य॑स्मत् ते॑ ते अ॒स्मन् नि ।\\
\\
176. अ॒स्मत् । नि । व॒प॒न्तु॒ ।\\
अ॒स्मन् नि न्य॑स्म द॒स्मन् नि व॑पन्तु वपन्तु॒ न्य॑स्म द॒स्मन् नि व॑पन्तु ।\\
\\
177. नि । व॒प॒न्तु॒ । सेनाः᳚ ॥\\
निव॑पन्तु वपन्तु॒ नि निव॑पन्तु॒ सेनाः॒ सेना॑ वपन्तु॒ नि निव॑पन्तु॒ सेनाः᳚ ।\\
\\
178. व॒प॒न्तु॒ । सेनाः᳚ ॥\\
व॒प॒न्तु॒ सेनाः॒ सेना॑ वपन्तु वपन्तु॒ सेनाः᳚ ।\\
\\
179. सेनाः᳚ ॥\\
सेना॒ इति॒ सेनाः᳚ ।\\
\\
180. परि॑ । नः॒ । रु॒द्रस्य॑ ।\\
परि॑ णो नः॒ परि॒ परि॑ णो रु॒द्रस्य॑ रु॒द्रस्य॑ नः॒ परि॒ परि॑ णो रु॒द्रस्य॑ ।\\
\\
181. नः॒ । रु॒द्रस्य॑ । हे॒तिः ।\\
नो॒ रु॒द्रस्य॑ रु॒द्रस्य॑ नो नो रु॒द्रस्य॑ हे॒तिर्. हे॒ती रु॒द्रस्य॑ नो नो रु॒द्रस्य॑\\
हे॒तिः ।\\
\\
182. रु॒द्रस्य॑ । हे॒तिः । वृ॒ण॒क्तु॒ ।\\
रु॒द्रस्य॑ हे॒तिर्. हे॒ती रु॒द्रस्य॑ रु॒द्रस्य॑ हे॒तिर् वृ॑णक्तु वृणक्तु हे॒ती रु॒द्रस्य॑\\
रु॒द्रस्य॑ हे॒तिर् वृ॑णक्तु ।\\
\\
183. हे॒तिः । वृ॒ण॒क्तु॒ । परि॑ ।\\
हे॒तिर् वृ॑णक्तु वृणक्तु हे॒तिर्. हे॒तिर् वृ॑णक्तु॒ परि॒ परि॑ वृणक्तु हे॒तिर्. हे॒तिर्\\
वृ॑णक्तु॒ परि॑ ।\\
\\
184. वृ॒ण॒क्तु॒ । परि॑ । त्वे॒षस्य॑ ।\\
वृ॒ण॒क्तु॒ परि॒ परि॑ वृणक्तु वृणक्तु॒ परि॑ त्वे॒षस्य॑ त्वे॒षस्य॒ परि॑ वृणक्तु वृणक्तु॒\\
परि॑ त्वे॒षस्य॑ ।\\
\\
185. परि॑ । त्वे॒षस्य॑ । दु॒र्म॒तिः ।\\
परि॑ त्वे॒षस्य॑ त्वे॒षस्य॒ परि॒ परि॑ त्वे॒षस्य॑ दुर्म॒तिर् दु॑र्म॒ति स्त्वे॒षस्य॒ परि॒ परि॑\\
त्वे॒षस्य॑ दुर्म॒तिः ।\\
\\
186. त्वे॒षस्य॑ । दु॒र्म॒तिः । अ॒घा॒योः ॥\\
त्वे॒षस्य॑ दुर्म॒तिर् दु॑र्म॒ति स्त्वे॒षस्य॑ त्वे॒षस्य॑ दुर्म॒ति र॑घा॒यो र॑घा॒योर् दु॑र्म॒ति\\
स्त्वे॒षस्य॑ त्वे॒षस्य॑ दुर्म॒ति र॑घा॒योः ।\\
\\
187. दु॒र्म॒तिः । अ॒घा॒योः ॥\\
दु॒र्म॒ति र॑घा॒यो र॑घा॒योर् दु॑र्म॒तिर् दु॑र्म॒ति र॑घा॒योः ।\\
\\
188. दु॒र्म॒तिः ।\\
दु॒र्म॒तिरिति॑ दुः - म॒तिः ।\\
\\
189. अ॒घा॒योः ॥\\
अ॒घा॒यो रित्य॑घ - योः ।\\
\\
190. अव॑ । स्थि॒रा । म॒घव॑द्भ्यः ।\\
अव॑ स्थि॒रा स्थि॒राऽवाव॑ स्थि॒रा म॒घव॑द्भ्यो म॒घव॑द्भ्यः स्थि॒राऽवाव॑\\
स्थि॒रा म॒घव॑द्भ्यः ।\\
\\
191. स्थि॒रा । म॒घव॑द्भ्यः । त॒नु॒ष्व॒ ।\\
स्थि॒रा म॒घव॑द्भ्यो म॒घव॑द्भ्यः स्थि॒रा स्थि॒रा म॒घव॑द्भ्य स्तनुष्व तनुष्व\\
म॒घव॑द्भ्यः स्थि॒रा स्थि॒रा म॒घव॑द्भ्य स्तनुष्व ।\\
\\
192. म॒घव॑द्भ्यः । त॒नु॒ष्व॒ । मीढ्वः॑ ।\\
म॒घव॑द्भ्य स्तनुष्व तनुष्व म॒घव॑द्भ्यो म॒घव॑द्भ्य स्तनुष्व॒ मीढ्वो॒ मीढ्व॑\\
स्तनुष्व म॒घव॑द्भ्यो म॒घव॑द्भ्य स्तनुष्व॒ मीढ्वः॑ ।\\
\\
193. म॒घव॑द्भ्यः ।\\
म॒घव॑द्भ्य॒ इति॑ म॒घव॑त् - भ्यः॒ ।\\
\\
194. त॒नु॒ष्व॒ । मीढ्वः॑ । तो॒काय॑ ।\\
त॒नु॒ष्व॒ मीढ्वो॒ मीढ्व॑ स्तनुष्व तनुष्व॒ मीढ्व॑ स्तो॒काय॑ तो॒काय॒ मीढ्व॑\\
स्तनुष्व तनुष्व॒ मीढ्व॑ स्तो॒काय॑ ।\\
\\
195. मीढ्वः॑ । तो॒काय॑ । तन॑याय ।\\
मीढ्व॑ स्तो॒काय॑ तो॒काय॒ मीढ्वो॒ मीढ्व॑ स्तो॒काय॒ तन॑याय॒ तन॑याय तो॒काय॒\\
मीढ्वो॒ मीढ्व॑ स्तो॒काय॒ तन॑याय ।\\
\\
196. तो॒काय॑ । तन॑याय । मृ॒ड॒य॒ ॥\\
तो॒काय॒ तन॑याय॒ तन॑याय तो॒काय॑ तो॒काय॒ तन॑याय मृडय मृडय॒ तन॑याय\\
तो॒काय॑ तो॒काय॒ तन॑याय मृडय ।\\
\\
197. तन॑याय । मृ॒ड॒य॒ ॥\\
तन॑याय मृडय मृडय॒ तन॑याय॒ तन॑याय मृडय ।\\
\\
198. मृ॒ड॒य॒ ॥\\
मृ॒ड॒येति॑ मृडय ।\\
\\
199. मीढु॑ष्टम । शिव॑तम । शि॒वः ।\\
मीढु॑ष्टम॒ शिव॑तम॒ शिव॑तम॒ मीढु॑ष्टम॒ मीढु॑ष्टम॒ शिव॑तम शि॒वः शि॒वः\\
शिव॑तम॒ मीढु॑ष्टम॒ मीढु॑ष्टम॒ शिव॑तम शि॒वः ।\\
\\
200. मीढु॑ष्टम ।\\
मीढु॑ष्ट॒मेति॒ मीढुः॑ - त॒म॒ ।\\
\\
201. शिव॑तम । शि॒वः । नः॒ ।\\
शिव॑तम शि॒वः शि॒वः शिव॑तम॒ शिव॑तम शि॒वो नो॑ नः शि॒वः शिव॑तम॒\\
शिव॑तम शि॒वो नः॑ ।\\
\\
202. शिव॑तम ।\\
शिव॑त॒मेति॒ शिव॑ - त॒म॒ ।\\
\\
203. शि॒वः । नः॒ । सु॒मनाः᳚ ।\\
शि॒वो नो॑ नः शि॒वः शि॒वो नः॑ सु॒मनाः᳚ सु॒मना॑ नः शि॒वः शि॒वो नः॑\\
सु॒मनाः᳚ ।\\
\\
204. नः॒ । सु॒मनाः᳚ । भ॒व॒ ॥\\
नः॒ सु॒मनाः᳚ सु॒मना॑ नो नः सु॒मना॑ भव भव सु॒मना॑ नो नः सु॒मना॑ भव ।\\
\\
205. सु॒मनाः᳚ । भ॒व॒ ॥\\
सु॒मना॑ भव भव सु॒मनाः᳚ सु॒मना॑ भव ।\\
\\
206. सु॒मनाः᳚ ॥\\
सु॒मना॒ इति॑ सु - मनाः᳚ ।\\
\\
207. भ॒व॒ ।\\
भ॒वेति॑ भव ।\\
\\
208. प॒र॒मे । वृ॒क्षे । आयु॑धम् ।\\
प॒र॒मे वृ॒क्षे वृ॒क्षे प॑र॒मे प॑र॒मे वृ॒क्ष आयु॑ध॒ मायु॑धंँवृ॒क्षे प॑र॒मे प॑र॒मे\\
वृ॒क्ष आयु॑धम् ।\\
\\
209. वृ॒क्षे । आयु॑धम् । नि॒धाय॑ ।\\
वृ॒क्ष आयु॑ध॒ मायु॑धंँवृ॒क्षे वृ॒क्ष आयु॑धन् नि॒धाय॑ नि॒धाया यु॑धंँवृ॒क्षे वृ॒क्ष\\
आयु॑धन् नि॒धाय॑ ।\\
\\
210. आयु॑धम् । नि॒धाय॑ । कृत्ति᳚म् ।\\
आयु॑धन् नि॒धाय॑ नि॒धाया यु॑ध॒ मायु॑धन् नि॒धाय॒ कृत्तिं॒ कृत्ति॑न् नि॒धाया यु॑ध॒\\
मायु॑धन् नि॒धाय॒ कृत्ति᳚म् ।\\
\\
211. नि॒धाय॑ । कृत्ति᳚म् । वसा॑नः ।\\
नि॒धाय॒ कृत्तिं॒ कृत्ति॑न् नि॒धाय॑ नि॒धाय॒ कृत्तिं॒ँवसा॑नो॒ वसा॑नः॒ कृत्ति॑न्\\
नि॒धाय॑ नि॒धाय॒ कृत्तिं॒ँवसा॑नः ।\\
\\
212. नि॒धाय॑ ।\\
नि॒धायेति॑ नि - धाय॑ ।\\
\\
213. कृत्ति᳚म् । वसा॑नः । आ ।\\
कृत्तिं॒ँवसा॑नो॒ वसा॑नः॒ कृत्तिं॒ कृत्तिं॒ँवसा॑न॒ आ वसा॑नः॒ कृत्तिं॒ कृत्तिं॒ँवसा॑न॒ आ ।\\
\\
214. वसा॑नः । आ । च॒र॒ ।\\
वसा॑न॒ आ वसा॑नो॒ वसा॑न॒ आच॑र च॒रा वसा॑नो॒ वसा॑न॒ आच॑र ।\\
\\
215. आ । च॒र॒ । पिना॑कम् ।\\
आच॑र च॒रा च॑र॒ पिना॑कं॒ पिना॑कञ्च॒रा च॑र॒ पिना॑कम् ।\\
\\
216. च॒र॒ । पिना॑कम् । बिभ्र॑त् ।\\
च॒र॒ पिना॑कं॒ पिना॑कञ्चर चर॒ पिना॑कं॒ बिभ्र॒द् बिभ्र॒त् पिना॑कञ्चर\\
चर॒ पिना॑कं॒ बिभ्र॑त् ।\\
\\
217. पिना॑कम् । बिभ्र॑त् । आ ।\\
पिना॑कं॒ बिभ्र॒द् बिभ्र॒त् पिना॑कं॒ पिना॑कं॒ बिभ्र॒दा बिभ्र॒त् पिना॑कं॒ पिना॑कं॒\\
बिभ्र॒दा ।\\
\\
218. बिभ्र॑त् । आ । ग॒हि॒ ॥\\
बिभ्र॒दा बिभ्र॒द् बिभ्र॒दा ग॑हि ग॒ह्या बिभ्र॒द् बिभ्र॒दा ग॑हि ।\\
\\
219. आ । ग॒हि॒ ॥\\
आग॑हि ग॒ह्या ग॑हि ।\\
\\
220. ग॒हि॒  ॥\\
ग॒हीति॑ गहि ।\\
\\
221. विकि॑रिद । विलो॑हित । नमः॑ ।\\
विकि॑रिद॒ विलो॑हित॒ विलो॑हित॒ विकि॑रिद॒ विकि॑रिद॒ विलो॑हित॒ नमो॒ नमो॒\\
विलो॑हित॒ विकि॑रिद॒ विकि॑रिद॒ विलो॑हित॒ नमः॑ ।\\
\\
222. विकि॑रिद ।\\
विकि॑रि॒देति॒ वि - कि॒रि॒द॒ ।\\
\\
223. विलो॑हित । नमः॑ । ते॒ ।\\
विलो॑हित॒ नमो॒ नमो॒ विलो॑हित॒ विलो॑हित॒ नम॑स्ते ते॒ नमो॒ विलो॑हित॒\\
विलो॑हित॒ नम॑स्ते ।\\
\\
224. विलो॑हित ।\\
विलो॑हि॒तेति॒ वि - लो॒हि॒त॒ ।\\
\\
225. नमः॑ । ते॒ । अ॒स्तु॒ ।\\
नम॑स्ते ते॒ नमो॒ नम॑स्ते अस्त्वस्तु ते॒ नमो॒ नम॑स्ते अस्तु ।\\
\\
226. ते॒ । अ॒स्तु॒ । भ॒ग॒वः॒ ॥\\
ते॒ अ॒स्त्व॒स्तु॒ ते॒ ते॒ अ॒स्तु॒ भ॒ग॒वो॒ भ॒ग॒वो॒ अ॒स्तु॒ ते॒ ते॒ अ॒स्तु॒ भ॒ग॒वः॒ ।\\
\\
227. अ॒स्तु॒ । भ॒ग॒वः॒ ॥\\
अ॒स्तु॒ भ॒ग॒वो॒ भ॒ग॒वो॒ अ॒स्त्व॒स्तु॒ भ॒ग॒वः॒ ।\\
\\
228. भ॒ग॒वः॒ ॥\\
भ॒ग॒व॒ इति॑ भग - वः॒ ।\\
\\
229. याः । ते॒ । स॒हस्र᳚म् ।\\
यास्ते॑ ते॒ या यास्ते॑ स॒हस्रꣳ॑ स॒हस्रं॑ ते॒ या यास्ते॑ स॒हस्र᳚म् ।\\
\\
230. ते॒ । स॒हस्र᳚म् । हे॒तयः॑ ।\\
ते॒ स॒हस्रꣳ॑ स॒हस्रं॑ ते ते स॒हस्रꣳ॑ हे॒तयाे॑ हे॒तयः॑ स॒हस्रं॑ ते ते स॒हस्रꣳ॑\\
हे॒तयः॑ ।\\
\\
231. स॒हस्र᳚म् । हे॒तयः॑ । अ॒न्यम् ।\\
स॒हस्रꣳ॑ हे॒तयो॑ हे॒तयः॑ स॒हस्रꣳ॑ स॒हस्रꣳ॑ हे॒तयो॒ऽन्य म॒न्यꣳ हे॒तयः॑\\
स॒हस्रꣳ॑ स॒हस्रꣳ॑ हे॒तयो॒ऽन्यम् ।\\
\\
232. हे॒तयः॑ । अ॒न्यम् । अ॒स्मत् ।\\
हे॒तयो॒ऽन्य म॒न्यꣳ हे॒तयो॑ हे॒तयो॒ऽन्य म॒स्म द॒स्म द॒न्यꣳ हे॒तयो॑ हे॒तयो॒ऽन्य म॒स्मत् ।\\
\\
233. अ॒न्यम् । अ॒स्मत् । नि ।\\
अ॒न्य म॒स्म द॒स्म द॒न्य म॒न्य म॒स्मन् नि न्य॑स्म द॒न्य म॒न्य म॒स्मन् नि ।\\
\\
234. अ॒स्मत् । नि । व॒प॒न्तु॒ ।\\
अ॒स्मन् नि न्य॑स्म द॒स्मन् नि व॑पन्तु वपन्तु॒ न्य॑स्म द॒स्मन् नि व॑पन्तु ।\\
\\
235. नि । व॒प॒न्तु॒ । ताः ॥\\
नि व॑पन्तु वपन्तु॒ नि नि व॑पन्तु॒ ता स्ता व॑पन्तु॒ नि नि व॑पन्तु॒ ताः ।\\
\\
236. व॒प॒न्तु॒ । ताः ॥\\
व॒प॒न्तु॒ तास्ता व॑पन्तु वपन्तु॒ ताः ।\\
\\
237. ताः ॥\\
ता इति॒ ताः ।\\
\\
238. स॒हस्रा॑णि । स॒ह॒स्र॒धा । बा॒हु॒वोः ।\\
स॒हस्रा॑णि सहस्र॒धा स॑हस्र॒धा स॒हस्रा॑णि स॒हस्रा॑णि सहस्र॒धा बा॑हु॒वोर्\\
बा॑हु॒वोः स॑हस्र॒धा स॒हस्रा॑णि स॒हस्रा॑णि सहस्र॒धा बा॑हु॒वोः ।\\
\\
239. स॒ह॒स्र॒धा । बा॒हु॒वोः । तव॑ ।\\
स॒ह॒स्र॒धा बा॑हु॒वोर् बा॑हु॒वोः स॑हस्र॒धा स॑हस्र॒धा बा॑हु॒वो स्तव॒ तव॑ बाहु॒वोः\\
स॑हस्र॒धा स॑हस्र॒धा बा॑हु॒वो स्तव॑ ।\\
\\
240. स॒ह॒स्र॒धा ।\\
स॒ह॒स्र॒धेति॑ सहस्र - धा ।\\
\\
241. बा॒हु॒वोः । तव॑ । हे॒तयः॑ ॥\\
बा॒हु॒वो स्तव॒ तव॑ बाहु॒वोर् बा॑हु॒वो स्तव॑ हे॒तयो॑ हे॒तय॒ स्तव॑ बाहु॒वोर्\\
बा॑हु॒वो स्तव॑ हे॒तयः॑ ।\\
\\
242. तव॑ । हे॒तयः॑ ॥\\
तव॑ हे॒तयो॑ हे॒तय॒ स्तव॒ तव॑ हे॒तयः॑ ।\\
\\
243. हे॒तयः॑ ॥\\
हे॒तय॒ इति॑ हे॒तयः॑ ।\\
\\
244. तासा᳚म् । ईशा॑नः । भ॒ग॒वः॒ ।\\
तासा॒ मीशा॑न॒ ईशा॑न॒ स्तासां॒ तासा॒ मीशा॑नो भगवो भगव॒ ईशा॑न॒ स्तासां॒\\
तासा॒ मीशा॑नो भगवः ।\\
\\
245. ईशा॑नः । भ॒ग॒वः॒ । प॒रा॒चीना᳚ ।\\
ईशा॑नो भगवो भगव॒ ईशा॑न॒ ईशा॑नो भगवः परा॒चीना॑ परा॒चीना॑ भगव॒\\
ईशा॑न॒ ईशा॑नो भगवः परा॒चीना᳚ ।\\
\\
246. भ॒ग॒वः॒ । प॒रा॒चीना᳚ । मुखा᳚ ।\\
भ॒ग॒वः॒ प॒रा॒चीना॑ परा॒चीना॑ भगवो भगवः परा॒चीना॒ मुखा॒ मुखा॑ परा॒चीना॑\\
भगवो भगवः परा॒चीना॒ मुखा᳚ ।\\
\\
247. भ॒ग॒वः॒ ।\\
भ॒ग॒व॒ इति॑ भग - वः॒ ।\\
\\
248. प॒रा॒चीना᳚ । मुखा᳚ । कृ॒धि॒ ॥\\
प॒रा॒चीना॒ मुखा॒ मुखा॑ परा॒चीना॑ परा॒चीना॒ मुखा॑ कृधि कृधि॒ मुखा॑\\
परा॒चीना॑ परा॒चीना॒ मुखा॑ कृधि ।\\
\\
249. मुखा᳚ । कृ॒धि॒ ॥\\
मुखा॑ कृधि कृधि॒ मुखा॒ मुखा॑ कृधि ।\\
\\
250. कृ॒धि॒ ॥\\
कृ॒धीति॑ कृधि ।\\
\subsection{\eng{Anuvaka 11}}
1. स॒हस्रा॑णि । स॒ह॒स्र॒शः । ये ।\\
स॒हस्रा॑णि सहस्र॒शः स॑हस्र॒शः स॒हस्रा॑णि स॒हस्रा॑णि सहस्र॒शो ये ये\\
स॑हस्र॒शः स॒हस्रा॑णि स॒हस्रा॑णि सहस्र॒शो ये ।\\
\\
2. स॒ह॒स्र॒शः । ये । रु॒द्राः ।\\
स॒ह॒स्र॒शो ये ये स॑हस्र॒शः स॑हस्र॒शो ये रु॒द्रा रु॒द्रा ये स॑हस्र॒शः स॑हस्र॒शो\\
ये रु॒द्राः ।\\
\\
3. स॒ह॒स्र॒शः ।\\
स॒ह॒स्र॒श इति॑ सहस्र - शः ।\\
\\
4. ये । रु॒द्राः । अधि॑ ।\\
ये रु॒द्रा रु॒द्रा ये ये रु॒द्रा अध्यधि॑ रु॒द्रा ये ये रु॒द्रा अधि॑ ।\\
\\
5. रु॒द्राः । अधि॑ । भूम्या᳚म् ॥\\
रु॒द्रा अध्यधि॑ रु॒द्रा रु॒द्रा अधि॒ भूम्यां॒ भूम्या॒ मधि॑ रु॒द्रा रु॒द्रा\\
अधि॒ भूम्या᳚म् ।\\
\\
6. अधि॑ । भूम्या᳚म् ॥\\
अधि॒ भूम्यां॒ भूम्या॒ मध्यधि॒ भूम्या᳚म् ।\\
\\
7. भूम्या᳚म् ॥\\
भूम्या॒मिति॒ भूम्या᳚म् ।\\
\\
8. तेषा᳚म् । स॒ह॒स्र॒यो॒ज॒ने । अव॑ ।\\
तेषाꣳ॑ सहस्रयोज॒ने स॑हस्रयोज॒ने तेषां॒ तेषाꣳ॑ सहस्रयोज॒नेऽवाव॑ सहस्रयोज॒ने तेषां॒ तेषाꣳ॑ सहस्रयोज॒नेऽव॑ ।\\
\\
9. स॒ह॒स्र॒यो॒ज॒ने । अव॑ । धन्वा॑नि ।\\
स॒ह॒स्र॒यो॒ज॒नेऽवाव॑ सहस्रयोज॒ने स॑हस्रयोज॒नेऽव॒ धन्वा॑नि॒ धन्वा॒ न्यव॑\\
सहस्रयोज॒ने स॑हस्रयोज॒नेऽव॒ धन्वा॑नि ।\\
\\
10. स॒ह॒स्र॒यो॒ज॒ने ।\\
स॒ह॒स्र॒यो॒ज॒न इति॑ सहस्र - यो॒ज॒ने ।\\
\\
11. अव॑ । धन्वा॑नि । त॒न्म॒सि॒ ॥\\
अव॒ धन्वा॑नि॒ धन्वा॒ न्यवाव॒ धन्वा॑नि तन्मसि तन्मसि॒ धन्वा॒ न्यवाव॒ धन्वा॑नि\\
तन्मसि ।\\
\\
12. धन्वा॑नि । त॒न्म॒सि॒ ॥\\
धन्वा॑नि तन्मसि तन्मसि॒ धन्वा॑नि॒ धन्वा॑नि तन्मसि ।\\
\\
13. त॒न्म॒सि॒ ॥\\
त॒न्म॒सीति॑ तन्मसि ।\\
\\
14. अ॒स्मिन्न् । म॒ह॒ति । अ॒र्ण॒वे ।\\
अ॒स्मिन् म॑ह॒ति म॑ह॒ त्य॑स्मिन् न॒स्मिन् म॑ह॒ त्य॑र्ण॒वे᳚ऽर्ण॒वे म॑ह॒ त्य॑स्मिन्\\
न॒स्मिन् म॑ह॒ त्य॑र्ण॒वे ।\\
\\
15. म॒ह॒ति । अ॒र्ण॒वे । अ॒न्तरि॑क्षे ।\\
म॒ह॒ त्य॑र्ण॒वे᳚ऽर्ण॒वे म॑ह॒ति म॑ह॒ त्य॑र्ण॒वे᳚ऽन्तरि॑क्षे॒ऽन्तरि॑क्षेऽर्ण॒वे म॑ह॒ति\\
म॑ह॒ त्य॑र्ण॒वे᳚ऽन्तरि॑क्षे ।\\
\\
16. अ॒र्ण॒वे । अ॒न्तरि॑क्षे । भ॒वाः ।\\
अ॒र्ण॒वे᳚ऽन्तरि॑क्षे॒ऽन्तरि॑क्षेऽर्ण॒वे᳚ऽर्ण॒वे᳚ऽन्तरि॑क्षे भ॒वा भ॒वा अ॒न्तरि॑क्षेऽर्ण॒वे᳚ऽर्ण॒वे᳚ऽन्तरि॑क्षे भ॒वाः ।\\
\\
17. अ॒न्तरि॑क्षे । भ॒वाः । अधि॑ ॥\\
अ॒न्तरि॑क्षे भ॒वा भ॒वा अ॒न्तरि॑क्षे॒ऽन्तरि॑क्षे भ॒वा अध्यधि॑ भ॒वा अ॒न्तरि॑क्षे॒ऽन्तरि॑क्षे भ॒वा अधि॑ ।\\
\\
18. भ॒वाः । अधि॑ ॥\\
भ॒वा अध्यधि॑ भ॒वा भ॒वा अधि॑ ।\\
\\
19. अधि॑ ॥\\
अधीत्यधि॑ ।\\
\\
20. नील॑ग्रीवाः । शि॒ति॒कण्ठाः᳚ । श॒र्वाः ।\\
नील॑ग्रीवाः शिति॒कण्ठाः᳚ शिति॒कण्ठा॒ नील॑ग्रीवा॒ नील॑ग्रीवाः शिति॒कण्ठाः᳚\\
श॒र्वाः श॒र्वाः शि॑ति॒कण्ठा॒ नील॑ग्रीवा॒ नील॑ग्रीवाः शिति॒कण्ठाः᳚ श॒र्वाः ।\\
\\
21. नील॑ग्रीवाः ।\\
नील॑ग्रीवा॒ इति॒ नील॑ - ग्री॒वाः॒ ।\\
\\
22. शि॒ति॒कण्ठाः᳚ । श॒र्वाः । अ॒धः ।\\
शि॒ति॒कण्ठाः᳚ श॒र्वाः श॒र्वाः शि॑ति॒कण्ठाः᳚ शिति॒कण्ठाः᳚ श॒र्वा अ॒धो॑ऽधः\\
श॒र्वाः शि॑ति॒कण्ठाः᳚ शिति॒कण्ठाः᳚ श॒र्वा अ॒धः ।\\
\\
23. शि॒ति॒कण्ठाः᳚ ।\\
शि॒ति॒कण्ठा॒ इति॑ शिति - कण्ठाः᳚ ।\\
\\
24. श॒र्वाः । अ॒धः । क्ष॒मा॒च॒राः ॥\\
श॒र्वा अ॒धो॑ऽधः श॒र्वाः श॒र्वा अ॒धः क्ष॑माच॒राः क्ष॑माच॒रा अ॒धः श॒र्वाः श॒र्वा\\
अ॒धः क्ष॑माच॒राः ।\\
\\
25. अ॒धः । क्ष॒मा॒च॒राः ॥\\
अ॒धः क्ष॑माच॒राः क्ष॑माच॒रा अ॒धो॑ऽधः क्ष॑माच॒राः ।\\
\\
26. क्ष॒मा॒च॒राः ॥\\
क्ष॒मा॒च॒रा इति॑ क्षमाच॒राः ।\\
\\
27. नील॑ग्रीवाः । शि॒ति॒कण्ठाः᳚ । दिव᳚म् ।\\
नील॑ग्रीवाः शिति॒कण्ठाः᳚ शिति॒कण्ठा॒ नील॑ग्रीवा॒ नील॑ग्रीवाः शिति॒कण्ठा॒\\
दिवं॒ दिवꣳ॑ शिति॒कण्ठा॒ नील॑ग्रीवा॒ नील॑ग्रीवाः शिति॒कण्ठा॒ दिव᳚म् ।\\
\\
28. नील॑ग्रीवाः ।\\
नील॑ग्रीवा॒ इति॒ नील॑ - ग्री॒वाः॒ ।\\
\\
29. शि॒ति॒कण्ठाः᳚ । दिव᳚म् । रु॒द्राः ।\\
शि॒ति॒कण्ठा॒ दिवं॒ दिवꣳ॑ शिति॒कण्ठाः᳚ शिति॒कण्ठा॒ दिवꣳ॑ रु॒द्रा रु॒द्रा\\
दिवꣳ॑ शिति॒कण्ठाः᳚ शिति॒कण्ठा॒ दिवꣳ॑ रु॒द्राः ।\\
\\
30. शि॒ति॒कण्ठाः᳚ ।\\
शि॒ति॒कण्ठा॒ इति॑ शिति - कण्ठाः᳚ ।\\
\\
31. दिव᳚म् । रु॒द्राः । उप॑श्रिताः ॥\\
दिवꣳ॑ रु॒द्रा रु॒द्रा दिवं॒ दिवꣳ॑ रु॒द्रा उप॑श्रिता॒ उप॑श्रिता रु॒द्रा दिवं॒ दिवꣳ॑\\
रु॒द्रा उप॑श्रिताः ।\\
\\
32. रु॒द्राः । उप॑श्रिताः ॥\\
रु॒द्रा उप॑श्रिता॒ उप॑श्रिता रु॒द्रा रु॒द्रा उप॑श्रिताः ।\\
\\
33. उप॑श्रिताः ॥\\
उप॑श्रिता॒ इत्युप॑ - श्रि॒ताः॒ ।\\
\\
34. ये । वृ॒क्षेषु॑ । स॒स्पिञ्ज॑राः ।\\
ये वृ॒क्षेषु॑ वृ॒क्षेषु॒ ये ये वृ॒क्षेषु॑ स॒स्पिञ्ज॑राः स॒स्पिञ्ज॑रा वृ॒क्षेषु॒ ये ये\\
वृ॒क्षेषु॑ स॒स्पिञ्ज॑राः ।\\
\\
35. वृ॒क्षेषु॑ । स॒स्पिञ्ज॑राः । नील॑ग्रीवाः ।\\
वृ॒क्षेषु॑ स॒स्पिञ्ज॑राः स॒स्पिञ्ज॑रा वृ॒क्षेषु॑ वृ॒क्षेषु॑ स॒स्पिञ्ज॑रा॒ नील॑ग्रीवा॒\\
नील॑ग्रीवाः स॒स्पिञ्ज॑रा वृ॒क्षेषु॑ वृ॒क्षेषु॑ स॒स्पिञ्ज॑रा॒ नील॑ग्रीवाः ।\\
\\
36. स॒स्पिञ्ज॑राः । नील॑ग्रीवाः । विलो॑हिताः ॥\\
स॒स्पिञ्ज॑रा॒ नील॑ग्रीवा॒ नील॑ग्रीवाः स॒स्पिञ्ज॑राः स॒स्पिञ्ज॑रा॒ नील॑ग्रीवा॒\\
विलो॑हिता॒ विलो॑हिता॒ नील॑ग्रीवाः स॒स्पिञ्ज॑राः स॒स्पिञ्ज॑रा॒ नील॑ग्रीवा॒\\
विलो॑हिताः ।\\
\\
37. नील॑ग्रीवाः । विलो॑हिताः ॥\\
नील॑ग्रीवा॒ विलो॑हिता॒ विलो॑हिता॒ नील॑ग्रीवा॒ नील॑ग्रीवा॒ विलो॑हिताः ।\\
\\
38. नील॑ग्रीवाः ।\\
नील॑ग्रीवा॒ इति॒ नील॑ - ग्री॒वाः॒ ।\\
\\
39. विलो॑हिताः ॥\\
विलो॑हिता॒ इति॒ वि - लो॒हि॒ताः॒ ।\\
\\
40. ये । भू॒ताना᳚म् । अधि॑पतयः ।\\
ये भू॒तानां᳚ भू॒तानां॒ँये ये भू॒ताना॒ मधि॑पत॒योऽधि॑पतयो भू॒तानां॒ँये ये\\
भू॒ताना॒ मधि॑पतयः ।\\
\\
41. भू॒ताना᳚म् । अधि॑पतयः । वि॒शि॒खासः॑ ।\\
भू॒ताना॒ मधि॑पत॒योऽधि॑पतयो भू॒तानां᳚ भू॒ताना॒ मधि॑पतयो विशि॒खासो॑\\
विशि॒खासोऽधि॑पतयो भू॒तानां᳚ भू॒ताना॒ मधि॑पतयो विशि॒खासः॑ ।\\
\\
42. अधि॑पतयः । वि॒शि॒खासः॑ । क॒प॒र्दिनः॑ ॥\\
अधि॑पतयो विशि॒खासो॑ विशि॒खासोऽधि॑पत॒योऽधि॑पतयो विशि॒खासः॑\\
कप॒र्दिनः॑ कप॒र्दिनो॑ विशि॒खासोऽधि॑पत॒योऽधि॑पतयो विशि॒खासः॑\\
कप॒र्दिनः॑ ।\\
\\
43. अधि॑पतयः ।\\
अधि॑पतय॒ इत्यधि॑ - प॒त॒यः॒ ।\\
\\
44. वि॒शि॒खासः॑ । क॒प॒र्दिनः॑ ॥\\
वि॒शि॒खासः॑ कप॒र्दिनः॑ कप॒र्दिनो॑ विशि॒खासो॑ विशि॒खासः॑ कप॒र्दिनः॑ ।\\
\\
45. वि॒शि॒खासः॑ ।\\
वि॒शि॒खास॒ इति॑ वि - शि॒खासः॑ ।\\
\\
46. क॒प॒र्दिनः॑ ॥\\
क॒प॒र्दिन॒ इति॑ कप॒र्दिनः॑ ।\\
\\
47. ये । अन्ने॑षु । वि॒विद्ध्य॑न्ति ।\\
ये अन्ने॒ ष्वन्ने॑षु॒ ये ये अन्ने॑षु वि॒विद्ध्य॑न्ति वि॒विद्ध्य॒ न्त्यन्ने॑षु॒ ये ये अन्ने॑षु\\
वि॒विद्ध्य॑न्ति ।\\
\\
48. अन्ने॑षु । वि॒विद्ध्य॑न्ति । पात्रे॑षु ।\\
अन्ने॑षु वि॒विद्ध्य॑न्ति वि॒विद्ध्य॒ न्त्यन्ने॒ ष्वन्ने॑षु वि॒विद्ध्य॑न्ति॒ पात्रे॑षु॒ पात्रे॑षु\\
वि॒विद्ध्य॒ न्त्यन्ने॒ ष्वन्ने॑षु वि॒विद्ध्य॑न्ति॒ पात्रे॑षु ।\\
\\
49. वि॒विद्ध्य॑न्ति । पात्रे॑षु । पिब॑तः ।\\
वि॒विद्ध्य॑न्ति॒ पात्रे॑षु॒ पात्रे॑षु वि॒विद्ध्य॑न्ति वि॒विद्ध्य॑न्ति॒ पात्रे॑षु॒ पिब॑तः॒\\
पिब॑तः॒ पात्रे॑षु वि॒विद्ध्य॑न्ति वि॒विद्ध्य॑न्ति॒ पात्रे॑षु॒ पिब॑तः ।\\
\\
50. वि॒विद्ध्य॑न्ति ।\\
वि॒विद्ध्य॒न्तीति॑ वि - विद्ध्य॑न्ति ।\\
\\
51. पात्रे॑षु । पिब॑तः । जनान्॑ ॥\\
पात्रे॑षु॒ पिब॑तः॒ पिब॑तः॒ पात्रे॑षु॒ पात्रे॑षु॒ पिब॑तो॒ जना॒न् जना॒न् पिब॑तः॒ पात्रे॑षु॒\\
पात्रे॑षु॒ पिब॑तो॒ जनान्॑ ।\\
\\
52. पिब॑तः । जनान्॑ ॥\\
पिब॑तो॒ जना॒न् जना॒न् पिब॑तः॒ पिब॑तो॒ जनान्॑ ।\\
\\
53. जनान्॑ ॥\\
जना॒निति॒ जनान्॑ ।\\
\\
54. ये । प॒थाम् । प॒थि॒रक्ष॑यः ।\\
ये प॒थां प॒थांँये ये प॒थां प॑थि॒रक्ष॑यः पथि॒रक्ष॑यः प॒थांँये ये प॒थां\\
प॑थि॒रक्ष॑यः ।\\
\\
55. प॒थाम् । प॒थि॒रक्ष॑यः । ऐ॒ल॒बृ॒दाः ।\\
प॒थां प॑थि॒रक्ष॑यः पथि॒रक्ष॑यः प॒थां प॒थां प॑थि॒रक्ष॑य ऐलबृ॒दा ऐ॑लबृ॒दाः\\
प॑थि॒रक्ष॑यः प॒थां प॒थां प॑थि॒रक्ष॑य ऐलबृ॒दाः ।\\
\\
56. प॒थि॒रक्ष॑यः । ऐ॒ल॒बृ॒दाः । य॒व्युधः॑ ॥\\
प॒थि॒रक्ष॑य ऐलबृ॒दा ऐ॑लबृ॒दाः प॑थि॒रक्ष॑यः पथि॒रक्ष॑य ऐलबृ॒दा य॒व्युधो॑\\
य॒व्युध॑ ऐलबृ॒दाः प॑थि॒रक्ष॑यः पथि॒रक्ष॑य ऐलबृ॒दा य॒व्युधः॑ ।\\
\\
57. प॒थि॒रक्ष॑यः ।\\
प॒थि॒रक्ष॑य॒ इति॑ पथि - रक्ष॑यः ।\\
\\
58. ऐ॒ल॒बृ॒दाः । य॒व्युधः॑ ॥\\
ऐ॒ल॒बृ॒दा य॒व्युधो॑ य॒व्युध॑ ऐलबृ॒दा ऐ॑लबृ॒दा य॒व्युधः॑ ।\\
\\
59. य॒व्युधः॑ ॥\\
य॒व्युध॒ इति॑ य॒व्युधः॑ ।\\
\\
60. ये । ती॒र्थानि॑ । प्र॒चर॑न्ति ।\\
ये ती॒र्थानि॑ ती॒र्थानि॒ ये ये ती॒र्थानि॑ प्र॒चर॑न्ति प्र॒चर॑न्ति ती॒र्थानि॒ ये ये\\
ती॒र्थानि॑ प्र॒चर॑न्ति ।\\
\\
61. ती॒र्थानि॑ । प्र॒चर॑न्ति । सृ॒काव॑न्तः ।\\
ती॒र्थानि॑ प्र॒चर॑न्ति प्र॒चर॑न्ति ती॒र्थानि॑ ती॒र्थानि॑ प्र॒चर॑न्ति सृ॒काव॑न्तः\\
सृ॒काव॑न्तः प्र॒चर॑न्ति ती॒र्थानि॑ ती॒र्थानि॑ प्र॒चर॑न्ति सृ॒काव॑न्तः ।\\
\\
62. प्र॒चर॑न्ति । सृ॒काव॑न्तः । नि॒ष॒ङ्गिणः॑ ॥\\
प्र॒चर॑न्ति सृ॒काव॑न्तः सृ॒काव॑न्तः प्र॒चर॑न्ति प्र॒चर॑न्ति सृ॒काव॑न्तो निष॒ङ्गिणो॑\\
निष॒ङ्गिणः॑ सृ॒काव॑न्तः प्र॒चर॑न्ति प्र॒चर॑न्ति सृ॒काव॑न्तो निष॒ङ्गिणः॑ ।\\
\\
63. प्र॒चर॑न्ति ।\\
प्र॒चर॒न्तीति॑ प्र - चर॑न्ति ।\\
\\
64. सृ॒काव॑न्तः । नि॒ष॒ङ्गिणः॑ ॥\\
सृ॒काव॑न्तो निष॒ङ्गिणो॑ निष॒ङ्गिणः॑ सृ॒काव॑न्तः सृ॒काव॑न्तो निष॒ङ्गिणः॑ ।\\
\\
65. सृ॒काव॑न्तः ।\\
सृ॒काव॑न्त॒ इति॑ सृ॒का - व॒न्तः॒ ।\\
\\
66. नि॒ष॒ङ्गिणः॑ ॥\\
नि॒ष॒ङ्गिण॒ इति॑ नि - स॒ङ्गिनः॑ ।\\
\\
67. ये । ए॒ताव॑न्तः । च॒ ।\\
य ए॒ताव॑न्त ए॒ताव॑न्तो॒ ये य ए॒ताव॑न्तश्च चै॒ताव॑न्तो॒ ये य ए॒ताव॑न्तश्च ।\\
\\
68. ए॒ताव॑न्तः । च॒ । भूयाꣳ॑सः ।\\
ए॒ताव॑न्तश्च चै॒ताव॑न्त ए॒ताव॑न्तश्च॒ भूयाꣳ॑सो॒ भूयाꣳ॑स श्चै॒ताव॑न्त ए॒ताव॑न्तश्च॒\\
भूयाꣳ॑सः ।\\
\\
69. च॒ । भूयाꣳ॑सः । च॒ ।\\
च॒ भूयाꣳ॑सो॒ भूयाꣳ॑सश्च च॒ भूयाꣳ॑सश्च च॒ भूयाꣳ॑सश्च च॒ भूयाꣳ॑सश्च ।\\
\\
70. भूयाꣳ॑सः । च॒ । दिशः॑ ।\\
भूयाꣳ॑सश्च च॒ भूयाꣳ॑सो॒ भूयाꣳ॑सश्च॒ दिशो॒ दिश॑श्च॒ भूयाꣳ॑सो॒\\
भूयाꣳ॑सश्च॒ दिशः॑ ।\\
\\
71. च॒ । दिशः॑ । रु॒द्राः ।\\
च॒ दिशो॒ दिश॑श्च च॒ दिशो॑ रु॒द्रा रु॒द्रा दिश॑श्च च॒ दिशो॑ रु॒द्राः ।\\
\\
72. दिशः॑ । रु॒द्राः । वि॒त॒स्थि॒रे ॥\\
दिशो॑ रु॒द्रा रु॒द्रा दिशो॒ दिशो॑ रु॒द्रा वि॑तस्थि॒रे वि॑तस्थि॒रे रु॒द्रा दिशो॒ दिशो॑\\
रु॒द्रा वि॑तस्थि॒रे ।\\
\\
73. रु॒द्राः । वि॒त॒स्थि॒रे ॥\\
रु॒द्रा वि॑तस्थि॒रे वि॑तस्थि॒रे रु॒द्रा रु॒द्रा वि॑तस्थि॒रे ।\\
\\
74. वि॒त॒स्थि॒रे ॥\\
वि॒त॒स्थि॒र इति॑ वि - त॒स्थि॒रे ।\\
\\
75. तेषा᳚म् । स॒ह॒स्र॒यो॒ज॒ने । अव॑ ।\\
तेषाꣳ॑ सहस्रयोज॒ने स॑हस्रयोज॒ने तेषां॒ तेषाꣳ॑ सहस्रयोज॒नेऽवाव॑ सहस्रयोज॒ने तेषां॒ तेषाꣳ॑ सहस्रयोज॒नेऽव॑ ।\\
\\
76. स॒ह॒स्र॒यो॒ज॒ने । अव॑ । धन्वा॑नि ।\\
स॒ह॒स्र॒यो॒ज॒नेऽवाव॑ सहस्रयोज॒ने स॑हस्रयोज॒नेऽव॒ धन्वा॑नि॒ धन्वा॒ न्यव॑\\
सहस्रयोज॒ने स॑हस्रयोज॒नेऽव॒ धन्वा॑नि ।\\
\\
77. स॒ह॒स्र॒यो॒ज॒ने ।\\
स॒ह॒स्र॒यो॒ज॒न इति॑ सहस्र - यो॒ज॒ने ।\\
\\
78. अव॑ । धन्वा॑नि । त॒न्म॒सि॒ ॥\\
अव॒ धन्वा॑नि॒ धन्वा॒ न्यवाव॒ धन्वा॑नि तन्मसि तन्मसि॒ धन्वा॒ न्यवाव॒ धन्वा॑नि\\
तन्मसि ।\\
\\
79. धन्वा॑नि । त॒न्म॒सि॒ ॥\\
धन्वा॑नि तन्मसि तन्मसि॒ धन्वा॑नि॒ धन्वा॑नि तन्मसि ।\\
\\
80. त॒न्म॒सि॒ ॥\\
त॒न्म॒सीति॑ तन्मसि ।\\
\\
81. नमः॑ । रु॒द्रेभ्यः॑ । ये ।\\
नमो॑ रु॒द्रेभ्यो॑ रु॒द्रेभ्यो॒ नमो॒ नमो॑ रु॒द्रेभ्यो॒ ये ये रु॒द्रेभ्यो॒ नमो॒ नमो॑\\
रु॒द्रेभ्यो॒ ये ।\\
\\
82. रु॒द्रेभ्यः॑ । ये । पृ॒थि॒व्याम् ।\\
रु॒द्रेभ्यो॒ ये ये रु॒द्रेभ्यो॑ रु॒द्रेभ्यो॒ ये पृ॑थि॒व्यां पृ॑थि॒व्यांँये रु॒द्रेभ्यो॑ रु॒द्रेभ्यो॒\\
ये पृ॑थि॒व्याम् ।\\
\\
83. ये । पृ॒थि॒व्याम् । ये ।\\
ये पृ॑थि॒व्यां पृ॑थि॒व्यांँये ये पृ॑थि॒व्यांँये ये पृ॑थि॒व्यांँये ये पृ॑थि॒व्यांँये ।\\
\\
84. पृ॒थि॒व्याम् । ये । अ॒न्तरि॑क्षे ।\\
पृ॒थि॒व्यांँये ये पृ॑थि॒व्यां पृ॑थि॒व्यांँये᳚ऽन्तरि॑क्षे॒ऽन्तरि॑क्षे॒ ये पृ॑थि॒व्यां पृ॑थि॒व्यांँये᳚ऽन्तरि॑क्षे ।\\
\\
85. ये । अ॒न्तरि॑क्षे । ये ।\\
ये᳚ऽन्तरि॑क्षे॒ऽन्तरि॑क्षे॒ ये ये᳚ऽन्तरि॑क्षे॒ ये ये᳚ऽन्तरि॑क्षे॒ ये ये᳚ऽन्तरि॑क्षे॒ ये ।\\
\\
86. अ॒न्तरि॑क्षे । ये । दि॒वि ।\\
अ॒न्तरि॑क्षे॒ ये ये᳚ऽन्तरि॑क्षे॒ऽन्तरि॑क्षे॒ ये दि॒वि दि॒वि ये᳚ऽन्तरि॑क्षे॒ऽन्तरि॑क्षे॒ ये\\
दि॒वि ।\\
\\
87. ये । दि॒वि । येषा᳚म् ।\\
ये दि॒वि दि॒वि ये ये दि॒वि येषां॒ँयेषां᳚ दि॒वि ये ये दि॒वि येषा᳚म् ।\\
\\
88. दि॒वि । येषा᳚म् । अन्न᳚म् ।\\
दि॒वि येषां॒ँयेषां᳚ दि॒वि दि॒वि येषा॒ मन्न॒ मन्नं॒ँयेषां᳚ दि॒वि दि॒वि येषा॒ मन्न᳚म् ।\\
\\
89. येषा᳚म् । अन्न᳚म् । वातः॑ ।\\
येषा॒ मन्न॒ मन्नं॒ँयेषां॒ँयेषा॒ मन्नं॒ँवातो॒ वातोऽन्नं॒ँयेषां॒ँयेषा॒ मन्नं॒ँवातः॑ ।\\
\\
90. अन्न᳚म् । वातः॑ । व॒र्॒.षम् ।\\
अन्नं॒ँवातो॒ वातोऽन्न॒ मन्नं॒ँवातो॑ व॒र्.॒षंँव॒र्.॒षंँवातोऽन्न॒ मन्नं॒ँवातो॑\\
व॒र्.॒षम् ।\\
\\
91. वातः॑ । व॒र्॒.षम् । इष॑वः ।\\
वातो॑ व॒र्.॒षंँव॒र्.॒षंँवातो॒ वातो॑ व॒र्.॒ष मिष॑व॒ इष॑वो व॒र्.॒षंँवातो॒ वातो॑\\
व॒र्.॒ष मिष॑वः ।\\
\\
92. व॒र्॒.षम् । इष॑वः । तेभ्यः॑ ।\\
व॒र्.॒ष मिष॑व॒ इष॑वो व॒र्.॒षंँव॒र्.॒ष मिष॑व॒ स्तेभ्य॒ स्तेभ्य॒ इष॑वो व॒र्.॒षंँव॒र्.॒ष मिष॑व॒ स्तेभ्यः॑ ।\\
\\
93. इष॑वः । तेभ्यः॑ । दश॑ ।\\
इष॑व॒ स्तेभ्य॒ स्तेभ्य॒ इष॑व॒ इष॑व॒ स्तेभ्यो॒ दश॒ दश॒ तेभ्य॒ इष॑व॒ इष॑व॒\\
स्तेभ्यो॒ दश॑ ।\\
\\
94. तेभ्यः॑ । दश॑ । प्राचीः᳚ ।\\
तेभ्यो॒ दश॒ दश॒ तेभ्य॒ स्तेभ्यो॒ दश॒ प्राचीः॒ प्राची॒र् दश॒ तेभ्य॒ स्तेभ्यो॒\\
दश॒ प्राचीः᳚ ।\\
\\
95. दश॑ । प्राचीः᳚ । दश॑ ।\\
दश॒ प्राचीः॒ प्राची॒र् दश॒ दश॒ प्राची॒र् दश॒ दश॒ प्राची॒र् दश॒ दश॒ प्राची॒र्\\
दश॑ ।\\
\\
96. प्राचीः᳚ । दश॑ । द॒क्षि॒णा ।\\
प्राची॒र् दश॒ दश॒ प्राचीः॒ प्राची॒र् दश॑ दक्षि॒णा द॑क्षि॒णा दश॒ प्राचीः॒ प्राची॒र्\\
दश॑ दक्षि॒णा ।\\
\\
97. दश॑ । द॒क्षि॒णा । दश॑ ।\\
दश॑ दक्षि॒णा द॑क्षि॒णा दश॒ दश॑ दक्षि॒णा दश॒ दश॑ दक्षि॒णा दश॒ दश॑\\
दक्षि॒णा दश॑ ।\\
\\
98. द॒क्षि॒णा । दश॑ । प्र॒तीचीः᳚ ।\\
द॒क्षि॒णा दश॒ दश॑ दक्षि॒णा द॑क्षि॒णा दश॑ प्र॒तीचीः᳚ प्र॒तीची॒र् दश॑ दक्षि॒णा\\
द॑क्षि॒णा दश॑ प्र॒तीचीः᳚ ।\\
\\
99. दश॑ । प्र॒तीचीः᳚ । दश॑ ।\\
दश॑ प्र॒तीचीः᳚ प्र॒तीची॒र् दश॒ दश॑ प्र॒तीची॒र् दश॒ दश॑ प्र॒तीची॒र् दश॒ दश॑\\
प्र॒तीची॒र् दश॑ ।\\
\\
100. प्र॒तीचीः᳚ । दश॑ । उदी॑चीः ।\\
प्र॒तीची॒र् दश॒ दश॑ प्र॒तीचीः᳚ प्र॒तीची॒र् दशोदी॑ची॒ रुदी॑ची॒र् दश॑ प्र॒तीचीः᳚\\
प्र॒तीची॒र् दशोदी॑चीः ।\\
\\
101. दश॑ । उदी॑चीः । दश॑ ।\\
दशोदी॑ची॒ रुदी॑ची॒र् दश॒ दशोदी॑ची॒र् दश॒ दशोदी॑ची॒र् दश॒ दशोदी॑ची॒र्\\
दश॑ ।\\
\\
102. उदी॑चीः । दश॑ । ऊ॒र्द्ध्वाः ।\\
उदी॑ची॒र् दश॒ दशोदी॑ची॒ रुदी॑ची॒र् दशो॒र्द्ध्वा ऊ॒र्द्ध्वा दशोदी॑ची॒ रुदी॑ची॒र्\\
दशो॒र्द्ध्वाः ।\\
\\
103. दश॑ । ऊ॒र्द्ध्वाः । तेभ्यः॑ ।\\
दशो॒र्द्ध्वा ऊ॒र्द्ध्वा दश॒ दशो॒र्द्ध्वा स्तेभ्य॒ स्तेभ्य॑ ऊ॒र्द्ध्वा दश॒ दशो॒र्द्ध्वा\\
स्तेभ्यः॑ ।\\
\\
104. ऊ॒र्द्ध्वाः । तेभ्यः॑ । नमः॑ ।\\
ऊ॒र्द्ध्वा स्तेभ्य॒ स्तेभ्य॑ ऊ॒र्द्ध्वा ऊ॒र्द्ध्वा स्तेभ्यो॒ नमो॒ नम॒ स्तेभ्य॑ ऊ॒र्द्ध्वा\\
ऊ॒र्द्ध्वा स्तेभ्यो॒ नमः॑ ।\\
\\
105. तेभ्यः॑ । नमः॑ । ते ।\\
तेभ्यो॒ नमो॒ नम॒ स्तेभ्य॒ स्तेभ्यो॒ नम॒ स्ते ते नम॒ स्तेभ्य॒ स्तेभ्यो॒ नम॒स्ते ।\\
\\
106. नमः॑ । ते । नः॒ ।\\
नम॒स्ते ते नमो॒ नम॒स्ते नो॑ न॒स्ते नमो॒ नम॒स्ते नः॑ ।\\
\\
107. ते । नः॒ । मृ॒ड॒य॒न्तु॒ ।\\
ते नो॑ न॒स्ते ते नो॑ मृडयन्तु मृडयन्तु न॒स्ते ते नो॑ मृडयन्तु ।\\
\\
108. नः॒ । मृ॒ड॒य॒न्तु॒ । ते ।\\
नो॒ मृ॒ड॒य॒न्तु॒ मृ॒ड॒य॒न्तु॒ नो॒ नो॒ मृ॒ड॒य॒न्तु॒ ते ते मृ॑डयन्तु नो नो मृडयन्तु॒ ते ।\\
\\
109. मृ॒ड॒य॒न्तु॒ । ते । यम् ।\\
मृ॒ड॒य॒न्तु॒ ते ते मृ॑डयन्तु मृडयन्तु॒ ते यंँयं ते मृ॑डयन्तु मृडयन्तु॒ ते यम् ।\\
\\
110. ते । यम् । द्वि॒ष्मः ।\\
ते यंँयन्ते तेयं द्वि॒ष्मो द्वि॒ष्मो यं ते ते यं द्वि॒ष्मः ।\\
\\
111. यम् । द्वि॒ष्मः । यः ।\\
यं द्वि॒ष्मो द्वि॒ष्मो यंँयं द्वि॒ष्मो यो यो द्वि॒ष्मो यंँयं द्वि॒ष्मो यः ।\\
\\
112. द्वि॒ष्मः । यः । च॒ ।\\
द्वि॒ष्मो यो यो द्वि॒ष्मो द्वि॒ष्मो यश्च॑ च॒ यो द्वि॒ष्मो द्वि॒ष्मो यश्च॑ ।\\
\\
113. यः । च॒ । नः॒ ।\\
यश्च॑ च॒ यो यश्च॑ नो नश्च॒ यो यश्च॑ नः ।\\
\\
114. च॒ । नः॒ । द्वेष्टि॑ ।\\
च॒ नो॒ न॒श्च॒ च॒ नो॒ द्वेष्टि॒ द्वेष्टि॑ नश्च च नो॒ द्वेष्टि॑ ।\\
\\
115. नः॒ । द्वेष्टि॑ । तम् ।\\
नो॒ द्वेष्टि॒ द्वेष्टि॑ नो नो॒ द्वेष्टि॒ तं तं द्वेष्टि॑ नो नो॒ द्वेष्टि॒ तम् ।\\
\\
116. द्वेष्टि॑ । तम् । वः॒ ।\\
द्वेष्टि॒ तं तं द्वेष्टि॒ द्वेष्टि॒ तंँवो॑ व॒ स्तं द्वेष्टि॒ द्वेष्टि॒ तंँवः॑ ।\\
\\
117. तम् । वः॒ । जंभे᳚ ।\\
तंँवो॑ व॒ स्तं तंँवो॒ जम्भे॒ जम्भे॑ व॒ स्तं तंँवो॒ जम्भे᳚ ।\\
\\
118. वः॒ । जंभे᳚ । द॒धा॒मि॒ ॥\\
वो॒ जम्भे॒ जम्भे॑ वो वो॒ जम्भे॑ दधामि दधामि॒ जम्भे॑ वो वो॒\\
जम्भे॑ दधामि ।\\
\\
119. जंभे᳚ । द॒धा॒मि॒ ॥\\
जम्भे॑ दधामि दधामि॒ जम्भे॒ जम्भे॑ दधामि ।\\
\\
120. द॒धा॒मि॒ ॥\\
द॒धा॒मीति॑ दधामि ।\\
\subsection{\eng{Triyambakam}}
1. त्र्यं॑बकम् । य॒जा॒म॒हे॒ । सु॒ग॒न्धिम् ।\\
त्र्यं॑बकंँयजामहे यजामहे॒ त्र्यं॑बकं॒ त्र्यं॑बकंँयजामहे सुग॒न्धिꣳ सु॑ग॒न्धिंँय॑जामहे॒ त्र्यं॑बकं॒ त्र्यं॑बकंँयजामहे सुग॒न्धिम् ।\\
\\
2. त्र्यं॑बकम् ।\\
त्र्यं॑बक॒मिति॒ त्रि - अं॒ब॒क॒म्॒ ।\\
\\
3. य॒जा॒म॒हे॒ । सु॒ग॒न्धिम् । पु॒ष्टि॒वर्द्ध॑नम् ॥\\
य॒जा॒म॒हे॒ सु॒ग॒न्धिꣳ सु॑ग॒न्धिंँय॑जामहे यजामहे सुग॒न्धिं पु॑ष्टि॒वर्द्ध॑नं\\
पुष्टि॒वर्द्ध॑नꣳ सुग॒न्धिंँय॑जामहे यजामहे सुग॒न्धिं पु॑ष्टि॒वर्द्ध॑नम् ।\\
\\
4. सु॒ग॒न्धिम् । पु॒ष्टि॒वर्द्ध॑नम् ॥\\
सु॒ग॒न्धिम् पु॑ष्टि॒वर्द्ध॑नं पुष्टि॒वर्द्ध॑नꣳ सुग॒न्धिꣳ सु॑ग॒न्धिं पु॑ष्टि॒वर्द्ध॑नम् ।\\
\\
5. सु॒ग॒न्धिम् ।\\
सु॒ग॒न्धि मिति॑ सु - ग॒न्धिम् ।\\
\\
6. पु॒ष्टि॒वर्द्ध॑नम् ॥\\
पु॒ष्टि॒वर्द्ध॑न॒ मिति॑ पुष्टि - वर्द्ध॑नम् ।\\
\\
7. उ॒र्वा॒रु॒कम् । इ॒व॒ । बन्ध॑नात् ।\\
उ॒र्वा॒रु॒क मि॑वे वोर्वारु॒क मु॑र्वारु॒क मि॑व॒ बन्ध॑ना॒द् बन्ध॑ना दिवोर्वारु॒क\\
मु॑र्वारु॒क मि॑व॒ बन्ध॑नात् ।\\
\\
8. इ॒व॒ । बन्ध॑नात् । मृ॒त्योः ।\\
इ॒व॒ बन्ध॑ना॒द् बन्ध॑ना दिवेव॒ बन्ध॑नान् मृ॒त्योर् मृ॒त्योर् बन्ध॑ना\\
दिवेव॒ बन्ध॑नान् मृ॒त्योः ।\\
\\
9. बन्ध॑नात् । मृ॒त्योः । मु॒क्षी॒य॒ ।\\
बन्ध॑नान् मृ॒त्योर् मृ॒त्योर् बन्ध॑ना॒द् बन्ध॑नान् मृ॒त्योर् मु॑क्षीय मुक्षीय मृ॒त्योर्\\
बन्ध॑ना॒द् बन्ध॑नान् मृ॒त्योर् मु॑क्षीय ।\\
\\
10. मृ॒त्योः । मु॒क्षी॒य॒ । मा ।\\
मृ॒त्योर् मु॑क्षीय मुक्षीय मृ॒त्योर् मृ॒त्योर् मु॑क्षीय॒ मा मा मु॑क्षीय मृ॒त्योर्\\
मृ॒त्योर् मु॑क्षीय॒ मा ।\\
\\
11. मु॒क्षी॒य॒ । मा । अ॒मृता᳚त् ॥\\
मु॒क्षी॒य॒ मा मा मु॑क्षीय मुक्षीय॒ माऽमृता॑ द॒मृता॒न् मा मु॑क्षीय\\
मुक्षीय॒ माऽमृता᳚त् ।\\
\\
12. मा । अ॒मृता᳚त् ॥\\
माऽमृता॑ द॒मृता॒न् मा माऽमृता᳚त् ।\\
\\
13. अ॒मृता᳚त् ॥\\
अ॒मृता॒दित् त्य॒मृता᳚त् ।\\
\\
14. यः । रु॒द्रः । अ॒ग्नौ ।\\
यो रु॒द्रो रु॒द्रो यो यो रु॒द्रो अ॒ग्ना व॒ग्नौ रु॒द्रो यो यो रु॒द्रो अ॒ग्नौ ।\\
\\
15. रु॒द्रः । अ॒ग्नौ । यः ।\\
रु॒द्रो अ॒ग्ना व॒ग्नौ रु॒द्रो रु॒द्रो अ॒ग्नौ यो यो᳚ऽग्नौ रु॒द्रो रु॒द्रो अ॒ग्नौ यः ।\\
\\
16. अ॒ग्नौ । यः । अ॒फ्सु ।\\
अ॒ग्नौ यो यो᳚ऽग्ना व॒ग्नौ यो अ॒फ्स्व॑फ्सु यो᳚ऽग्ना व॒ग्नौ यो अ॒फ्सु ।\\
\\
17. यः । अ॒फ्सु । यः ।\\
यो अ॒फ्स्व॑फ्सु यो यो अ॒फ्सु यो यो अ॒फ्सु यो यो अ॒फ्सु यः ।\\
\\
18. अ॒फ्सु । यः । ओष॑धीषु ।\\
अ॒फ्सु यो यो᳚ऽ(1॒)फ्स्व॑फ्सु य ओष॑धी॒ ष्वोष॑धीषु॒ यो᳚ऽ(1॒)फ्स्व॑फ्सु\\
य ओष॑धीषु ।\\
\\
19. अ॒फ्सु ।\\
अ॒फ्स्वित्य॑प् - सु ।\\
\\
20. यः । ओष॑धीषु । यः ।\\
य ओष॑धी॒ ष्वोष॑धीषु॒ यो य ओष॑धीषु॒ यो य ओष॑धीषु॒ यो\\
य ओष॑धीषु॒ यः ।\\
\\
21. ओष॑धीषु । यः । रु॒द्रः ।\\
ओष॑धीषु॒ यो य ओष॑धी॒ ष्वोष॑धीषु॒ यो रुद्रो॒ रुद्रो॒ य ओष॑धी॒ ष्वोष॑धीषु॒\\
यो रु॒द्रः ।\\
\\
22. यः । रु॒द्रः । विश्वा᳚ ।\\
यो रु॒द्रो रु॒द्रो यो यो रु॒द्रो विश्वा॒ विश्वा॑ रु॒द्रो यो यो रु॒द्रो विश्वा᳚ ।\\
\\
23. रु॒द्रः । विश्वा᳚ । भुव॑ना ।\\
रु॒द्रो विश्वा॒ विश्वा॑ रु॒द्रो रु॒द्रो विश्वा॒ भुव॑ना॒ भुव॑ना॒ विश्वा॑ रु॒द्रो रु॒द्रो\\
विश्वा॒ भुव॑ना ।\\
\\
24. विश्वा᳚ । भुव॑ना । आ॒वि॒वेश॑ ।\\
विश्वा॒ भुव॑ना॒ भुव॑ना॒ विश्वा॒ विश्वा॒ भुव॑नाऽऽवि॒वेशा॑ वि॒वेश॒ भुव॑ना॒\\
विश्वा॒ विश्वा॒ भुव॑नाऽऽवि॒वेश॑ ।\\
\\
25. भुव॑ना । आ॒वि॒वेश॑ । तस्मै᳚ ।\\
भुव॑नाऽऽवि॒वेशा॑ वि॒वेश॒ भुव॑ना॒ भुव॑नाऽऽवि॒वेश॒ तस्मै॒ तस्मा॑ आवि॒वेश॒\\
भुव॑ना॒ भुव॑नाऽऽवि॒वेश॒ तस्मै᳚ ।\\
\\
26. आ॒वि॒वेश॑ । तस्मै᳚ । रु॒द्राय॑ ।\\
आ॒वि॒वेश॒ तस्मै॒ तस्मा॑ आवि॒वेशा॑ वि॒वेश॒ तस्मै॑ रु॒द्राय॑ रु॒द्राय॒ तस्मा॑\\
आवि॒वेशा॑ वि॒वेश॒ तस्मै॑ रु॒द्राय॑ ।\\
\\
27. आ॒वि॒वेश॑ ।\\
आ॒वि॒वेशेत्या᳚ - वि॒वेश॑ ।\\
\\
28. तस्मै᳚ । रु॒द्राय॑ । नमः॑ ।\\
तस्मै॑ रु॒द्राय॑ रु॒द्राय॑ तस्मै॒ तस्मै॑ रु॒द्राय॒ नमो॒ नमो॑ रु॒द्राय॒ तस्मै॒ तस्मै॑\\
रु॒द्राय॒ नमः॑ ।\\
\\
29. रु॒द्राय॑ । नमः॑ । अ॒स्तु॒ ॥\\
रु॒द्राय॒ नमो॒ नमो॑ रु॒द्राय॑ रु॒द्राय॒ नमो॑ अस्त्वस्तु॒ नमो॑ रु॒द्राय॑ रु॒द्राय॒ नमाे॑\\
अस्तु ।\\
\\
30. नमः॑ । अ॒स्तु॒ ॥\\
नमो॑ अस्त्वस्तु॒ नमो॒ नमो॑ अस्तु ।\\
\\
31. अ॒स्तु॒ ॥\\
अ॒स्त्वित्य॑स्तु ।\\
\\
1. तमुम् । स्तु॒हि॒ । यः ।\\
तमु॑ष्टुहि स्तुह्यु॒ तं तमु॑ष्टुहि॒ यो य स्तु॑ह्यु॒ तं तमु॑ष्टुहि॒ यः ।\\
\\
2. ऊम् ।\\
ऊ॒ङ् इत्यू᳚ङ् ।\\
\\
3. स्तु॒हि॒ । यः । स्वि॒षुः ।\\
स्तु॒हि॒ यो यः स्तु॑हि स्तुहि॒ यः स्वि॒षुः स्वि॒षुर् यः स्तु॑हि स्तुहि॒\\
यः स्वि॒षुः ।\\
\\
4. यः । स्वि॒षुः । सु॒धन्वा᳚ ।\\
यः स्वि॒षुः स्वि॒षुर् यो यः स्वि॒षुः सु॒धन्वा᳚ सु॒धन्वा᳚ स्वि॒षुर् यो यः\\
स्वि॒षुः सु॒धन्वा᳚ ।\\
\\
5. स्वि॒षुः । सु॒धन्वा᳚ । यः ।\\
स्वि॒षुः सु॒धन्वा᳚ सु॒धन्वा᳚ स्वि॒षुः स्वि॒षुः सु॒धन्वा॒ यो यः सु॒धन्वा᳚ स्वि॒षुः\\
स्वि॒षुः सु॒धन्वा॒ यः ।\\
\\
6. स्वि॒षुः ।\\
स्वि॒षुरिति॑ सु॒ - इ॒षुः ।\\
\\
7. सु॒धन्वा᳚ । यः । विश्व॑स्य ।\\
सु॒धन्वा॒ यो यः सु॒धन्वा᳚ सु॒धन्वा॒ यो विश्व॑स्य॒ विश्व॑स्य॒ यः सु॒धन्वा᳚\\
सु॒धन्वा॒ यो विश्व॑स्य ।\\
\\
8. सु॒धन्वा᳚ ।\\
सु॒धन्वेति॑ सु॒ - धन्वा᳚ ।\\
\\
9. यः । विश्व॑स्य । क्षय॑ति ।\\
यो विश्व॑स्य॒ विश्व॑स्य॒ यो यो विश्व॑स्य॒ क्षय॑ति॒ क्षय॑ति॒ विश्व॑स्य॒ यो\\
यो विश्व॑स्य॒ क्षय॑ति ।\\
\\
10. विश्व॑स्य । क्षय॑ति । भे॒ष॒जस्य॑ ॥\\
विश्व॑स्य॒ क्षय॑ति॒ क्षय॑ति॒ विश्व॑स्य॒ विश्व॑स्य॒ क्षय॑ति भेष॒जस्य॑ भेष॒जस्य॒\\
क्षय॑ति॒ विश्व॑स्य॒ विश्व॑स्य॒ क्षय॑ति भेष॒जस्य॑ ।\\
\\
11. क्षय॑ति । भे॒ष॒जस्य॑ ॥\\
क्षय॑ति भेष॒जस्य॑ भेष॒जस्य॒ क्षय॑ति॒ क्षय॑ति भेष॒जस्य॑ ।\\
\\
12. भे॒ष॒जस्य॑ ॥\\
भे॒ष॒जस्येति॑ भे॒ष॒जस्य॑ ॥\\
\\
13. यक्ष्व॑ । म॒हे । सौ॒म॒न॒साय॑ ।\\
यक्ष्वा᳚ म॒हे म॒हे यक्ष्व॒ यक्ष्वा᳚ म॒हे सौ᳚मन॒साय॑ सौमन॒साय॑ म॒हे\\
यक्ष्व॒ यक्ष्वा᳚ म॒हे सौ᳚मन॒साय॑ ।\\
\\
14. म॒हे । सौ॒म॒न॒साय॑ । रु॒द्रम् ।\\
म॒हे सौ᳚मन॒साय॑ सौमन॒साय॑ म॒हे म॒हे सौ᳚मन॒साय॑ रु॒द्रं रु॒द्रं सौ᳚मन॒साय॑\\
म॒हे म॒हे सौ᳚मन॒साय॑ रु॒द्रम् ।\\
\\
15. सौ॒म॒न॒साय॑ । रु॒द्रम् । नमो॑भिः ।\\
सौ॒म॒न॒साय॑ रु॒द्रं रु॒द्रं सौ᳚मन॒साय॑ सौमन॒साय॑ रु॒द्रं नमो᳚भि॒र् नमो᳚भी\\
रु॒द्रं सौ᳚मन॒साय॑ सौमन॒साय॑ रु॒द्रं नमो᳚भिः ।\\
\\
16. रु॒द्रम् । नमो॑भिः । दे॒वम् ।\\
रु॒द्रं नमो᳚भि॒र् नमो᳚भी रु॒द्रं रु॒द्रं नमो᳚भिर् दे॒वं दे॒वं नमो᳚भी रु॒द्रं रु॒द्रं\\
नमो᳚भिर् दे॒वम् ।\\
\\
17. नमो॑भिः । दे॒वम् । असु॑रम् ।\\
नमो᳚भिर् दे॒वं दे॒वं नमो᳚भि॒र् नमो᳚भिर् दे॒व मसु॑र॒ मसु॑रं दे॒वं नमो᳚भि॒र्\\
नमो᳚भिर् दे॒व मसु॑रम् ।\\
\\
18. नमो॑भिः ।\\
नमो᳚भि॒रिति॒ नमः॑ - भिः ।\\
\\
19. दे॒वम् । असु॑रम् । दु॒व॒स्य॒ ॥\\
दे॒व मसु॑र॒ मसु॑रं दे॒वं दे॒व मसु॑रं दुवस्य दुव॒स्या सु॑रं दे॒वं दे॒व\\
मसु॑रं दुवस्य ।\\
\\
20. असु॑रम् । दु॒व॒स्य॒ ॥\\
असु॑रं दुवस्य दुव॒स्या सु॑र॒ मसु॑रं दुवस्य ।\\
\\
21. दु॒व॒स्य॒ ॥\\
दु॒वस्येति॑ दुवस्य ।\\
\\
22. अ॒यम् । मे॒ । हस्तः॑ ।\\
अ॒यं मे᳚ मे॒ऽय म॒यं मे॒ हस्तो॒ हस्तो᳚ मे॒ऽय म॒यं मे॒ हस्तः॑ ।\\
\\
23. मे॒ । हस्तः॑ । भग॑वान् ।\\
मे॒ हस्तो॒ हस्तो᳚ मे मे॒ हस्तो॒ भग॑वा॒न् भग॑वा॒न् हस्तो᳚ मे मे॒ हस्तो॒\\
भग॑वान् ।\\
\\
24. हस्तः॑ । भग॑वान् । अ॒यम् ।\\
हस्तो॒ भग॑वा॒न् भग॑वा॒न् हस्तो॒ हस्तो॒ भग॑वा न॒य म॒यं भग॑वा॒न् हस्तो॒\\
हस्तो॒ भग॑वा न॒यम् ।\\
\\
25. भग॑वान् । अ॒यम् । मे॒ ।\\
भग॑वा न॒य म॒यं भग॑वा॒न् भग॑वा न॒यं मे᳚ मे॒ऽयं भग॑वा॒न् भग॑वा न॒यं मे᳚ ।\\
\\
26. भग॑वान् ।\\
भग॑वा॒निति॒ भग॑ - वान् ।\\
\\
27. अ॒यम् । मे॒ । भग॑वत्तरः ॥\\
अ॒यं मे᳚ मे॒ऽय म॒यं मे॒ भग॑वत्तरो॒ भग॑वत्तरो मे॒ऽय म॒यं मे॒ भग॑वत्तरः ।\\
\\
28. मे॒ । भग॑वत्तरः ॥\\
मे॒ भग॑वत्तरो॒ भग॑वत्तरो मे मे॒ भग॑वत्तरः ।\\
\\
29. भग॑वत्तरः ॥\\
भग॑वत्तर॒ इति॒ भग॑वत् - तरः ।\\
\\
30. अ॒यम् । मे॒ । वि॒श्वभे॑षजः ।\\
अ॒यं मे᳚ मे॒ऽय म॒यं मे᳚ वि॒श्वभे᳚षजो वि॒श्वभे᳚षजो मे॒ऽय म॒यं मे᳚\\
वि॒श्वभे᳚षजः ।\\
\\
31. मे॒ । वि॒श्वभे॑षजः । अ॒यम् ।\\
मे॒ वि॒श्वभे᳚षजो वि॒श्वभे᳚षजो मे मे वि॒श्वभे᳚षजो॒ऽय म॒यं वि॒श्वभे᳚षजो मे मे\\
वि॒श्वभे᳚षजो॒ऽयम् ।\\
\\
32. वि॒श्वभे॑षजः । अ॒यम् । शि॒वाभि॑मर्शनः ॥\\
वि॒श्वभे᳚षजो॒ऽय म॒यं वि॒श्वभे᳚षजो वि॒श्वभे᳚षजो॒ऽयं शि॒वाभि॑मर्शनः\\
शि॒वाभि॑मर्शनो॒ऽयं वि॒श्वभे᳚षजो वि॒श्वभे᳚षजो॒ऽयं शि॒वाभि॑मर्शनः ।\\
\\
33. वि॒श्वभे॑षजः ।\\
वि॒श्वभे᳚षज॒ इति॑ वि॒श्व - भे᳚षजः ।\\
\\
34. अ॒यम् । शि॒वाभि॑मर्शनः ॥\\
अ॒यं शि॒वाभि॑मर्शनः शि॒वाभि॑मर्शनो॒ऽय म॒यं शि॒वाभि॑मर्शनः ।\\
\\
35. शि॒वाभि॑मर्शनः ॥\\
शि॒वाभि॑मर्शन॒ इति॑ शि॒व - अभि॑मर्शनः ।\\



\end{document}
