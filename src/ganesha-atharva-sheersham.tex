\section{\eng{Ganesha Atharva Sheersham}}
\subsection{\eng{Anuvaka 0}}
ॐ भ॒द्रङ् कर्णे॑भिश् शृणु॒ याम॑ देवाः ।\\
भ॒द्रं प॑श्ये मा॒क्ष भि॒र्य ज॑त्राः ।\\
स्थि॒रै रङ्गै᳚स् तुष्ठु॒वाग्ं स॑स्त॒ नूभिः॑ ।\\
व्यशे॑म दे॒वहि॑ तं॒यँ दायुः॑ ।\\
स्व॒स्ति न॒ इन्द्रो॑ वृ॒द्धश्र॑वाः ।\\
स्व॒स्ति नः॑ पू॒षा वि॒श्व वे॑दाः ।\\
स्व॒स्तिन॒स् तार्क्ष्यो॒ ,अरि॑ष्टनेमिः ।\\
स्व॒स्ति नो॒ बृह॒स्पति॑र्दधातु ॥\\
ॐ शान्ति॒-श्शान्ति॒-श्शान्तिः॑ ॥\\
\subsection{\eng{Anuvaka 1}}
ओन् नम॑स्ते ग॒णप॑तये ।\\
त्वमे॒व प्र॒त्यक्ष॒न् तत्त्व॑मसि ।\\
त्वमे॒व के॒वल॒ङ् कर्ता॑ ऽसि ।\\
त्वमे॒व के॒वलं॒ हर्ता॑-ऽसि ।\\
त्वमे॒व के॒वल॒न् धर्ता॑-ऽसि ।\\
त्वमेव सर्वङ् खल्विदं॑ ब्रह्मा॒सि ।\\
त्वं साक्षा दात्मा॑-ऽसि नि॒त्यम् ॥ 1 ॥\\
\subsection{\eng{Anuvaka 2}}
ऋ॑तं व॒च्मि । स॑त्यं-व॒च्मि ॥ 2 ॥\\
\\
\subsection{\eng{Anuvaka 3}}
अ॒व त्वं॒ माम् । अव॑ व॒क्तारम्᳚ ।\\
अव॑ श्रो॒तारम्᳚ । अव॑ दा॒तारम्᳚ । अव॑ धा॒तारम्᳚ ।\\
अवानू चा नम॑व शि॒ष्यम् । अव॑ प॒श्चात् ता᳚त् ।\\
अव॑ पु॒रस्ता᳚त् । अवोत्त॒रात्ता᳚त् ।\\
अव॑ द॒क्षिणात्ता᳚त् । अव॑ चो॒र्ध्वात्ता᳚त् ।\\
अवा ध॒रात्ता᳚त् । सर्वतो मां पाहि पाहि॑ सम॒न्तात् ॥  3 ॥\\
\subsection{\eng{Anuvaka 4}}
त्वं वान् मय॑स् त्व-ञ्चिन्म॒यः ।\\
त्व मानन्द मय॑स् त्वं ब्रह्म॒मयः ।\\
त्वं सच्चिदा नन्दाद् वि॑तीयो॒-ऽसि ।\\
त्वं प्र॒त्यक्षं॒ ब्रह्मा॑सि ।\\
त्वञ् ज्ञान मयो विज्ञा न॑म यो॒-ऽसि ॥   ॥\\
\subsection{\eng{Anuvaka 5}}
सर्वञ् जगदिदन् त्व॑त् तो जा॒यते ।\\
सर्वञ् जगदिदन् त्व॑त् तस् ति॒ष्ठति ।\\
सर्वञ् जगदिदन् त्वयि लय॑ मेष्य॒ति ।\\
सर्वञ् जगदिदन् त्वयि॑ प्रत् ये॒ति ।\\
त्वं भूमि रापो-ऽनलो-ऽनि॑लो न॒भः ।\\
त्वञ् चत्वारि वा᳚क् पदा॒नि ॥ 5 ॥\\
\subsection{\eng{Anuvaka 6}}
त्वङ् गु॒ण त्र॑याती॒तः । त्वं अवस्था त्र॑याती॒तः ।\\
त्वन् दे॒ह त्र॑या ती॒तः । त्वङ् का॒ल त्र॑याती॒तः ।\\
त्वं मूला धारस् थितो॑-ऽसि नि॒त्यम् । त्वं शक्ति त्र॑यात्म॒कः ।\\
त्वां-योँगिनो ध्याय॑न्ति नि॒त्यम् ।\\
त्वं ब्रह्मा त्वं विँष्णुस् त्वं रुद्रस् त्व मिन्द्रस् त्व मग्निस् \\
त्वं वाँयुस् त्वं सूर्यस् त्वञ् चन्द्रमास् त्वं ब्रह्म॒ भूर्भुव॒स् स्वरोम् ॥  6 ॥\\
\subsection{\eng{Anuvaka 7}}
ग॒णादिं पूर्व॑ मुच्चा॒र्य॒ व॒र्णादिन् तद न॒न्तरम् ।\\
अनुस् वारः प॑र त॒रः । अर्धे᳚न् दुल॒ सितम् ।\\
तारे॑ण ऋ॒द्धम् । एतत् तव मनु॑स्व रू॒पम् ।\\
गकारः पू᳚र्व रू॒पम् । अकारो मध्य॑म रू॒पम् ।\\
अनुस् वारश् चा᳚न् त्य रू॒पम् । बिन् दुरुत्त॑र रू॒पम् ।\\
नादः॑ सन्धा॒नम् । सग्ं हि॑ता स॒न्धिः ।\\
सैषा गाणे॑श वि॒द्या । गण॑क ऋ॒षिः ।\\
निचृद् गाय॑त्रीच् छ॒न्दः । श्री महा गणपति॑र् देवता ।\\
ओ-ङ्ग-ङ्ग॒णप॑तये नमः ॥ 7 ॥\\
\subsection{\eng{Anuvaka 8}}
एक द॒न्ताय॑ वि॒द्महे॑ वक्र तु॒ण्डाय॑ धीमहि ।\\
तन्नो॑ दन्ती प्रचो॒दया᳚त् ॥ 8 ॥\\
\subsection{\eng{Anuvaka 9}}
एक द॒न्तञ् च॑तुर् ह॒स्तं॒ पा॒श म॑ङ्कुश॒ धारि॑णम् ।\\
रद॑ञ्च॒ वर॑दं ह॒स्तै॒र् बि॒भ्राणं॑ मूष॒ कध्व॑जम् ।\\
रक्तं॑ ल॒म्बोद॑रं शू॒र्प॒ कर्णकं॑ रक्त॒ वास॑सम् ।\\
रक्त॑ ग॒न्धानु॑ लिप्ता॒ङ्गं॒ र॒क्त पु॑ष्पैस् सु॒पूजि॑तम् ।\\
भक्ता॑नु॒ कम्पि॑नन् दे॒व॒ञ्ज॒ गत् का॑रण॒ मच्यु॑तम् ।\\
आवि॑र् भू॒तञ्च॑ सृ॒ष्ट्या॒दौ॒ प्र॒कृतेः᳚ पुरु॒ षात्प॑रम् ।\\
एव॑न् ध्या॒यति॑ यो नि॒त्यं॒ स॒ योगी॑ योगि॒नां वँ॑रः ॥ 9 ॥\\
\subsection{\eng{Anuvaka 10}}
नमो व्रातपतये नमो गणपतये नमः प्रमथ पतये \\
नमस्ते-ऽस्तु लम्बोद रायैक दन्ताय विघ्न विनाशिने \\
शिवसुताय श्री वरद मूर्तये॒ नमो नमः ॥ 10 ॥\\
\subsection{\eng{Anuvaka 11}}
एतद थर्व शीर्षं योऽधी॒ते स ब्रह्म भूया॑य क॒ल्पते ।\\
स सर्व विघ्नै᳚र्न बा॒ध्यते । स सर्व त्रसुख॑ मेध॒ते ।\\
स पञ्च महापापा᳚त् प्रमु॒च्यते । \\
सा॒य म॑धीया॒नो॒ दिवसकृतं पाप॑न् नाश॒यति ।\\
प्रा॒त र॑धीया॒नो॒ रात्रिकृतं पाप॑न् नाश॒यति ।\\
सायं प्रातः प्र॑युञ्जा॒नो॒ पापो-ऽपा॑पो भ॒वति ।\\
सर्व त्राधी यानो ऽपवि॑घ्नो भवति । धर्मार्थ काम मोक्ष॑ञ्च वि॒न्दति ।\\
इदम थर्व शीर्ष मशिष्याय॑ न दे॒यम् । \\
यो यदि मो॑हाद् दा॒स्यति स पापी॑यान् भ॒वति ।\\
सहस्रा वर्त नाद्यं यङ्काम॑ मधी॒ते तन्त मने॑न सा॒धयेत् ॥  11 ॥\\
\subsection{\eng{Anuvaka 12}}
अनेन गणपति म॑भि षि॒ञ्चति स वाग्मी भ॒वति ।\\
चतुर् थ्या मन॑श् नन् ज॒पति स विद्या॑ वान् भ॒वति ।\\
इत्य थर्व॑ण वा॒क्यम् । \\
ब्रह्माद्या॒ वर॑णं वि॒द्यान् नबि भेति कदा॑च ने॒ति ॥  12 ॥\\
\subsection{\eng{Anuvaka 13}}
यो दूर्वाङ् कु॑रैर्य॒ जति स वैश्रव णोप॑मो भ॒वति ।\\
यो ला॑जैर् य॒जति स यशो॑वान् भ॒वति । स मेधा॑वान् भ॒वति ।\\
यो मोदक सहस्रे॑ण य॒जति स वाञ्छित फलम॑ वाप्नो॒ति ।\\
यस् साज्य समि॑द् भिर्य॒जति स सर्वं लभते स स॑र्वं ल॒भते ॥  13 ॥\\
\subsection{\eng{Anuvaka 14}}
अष्टौ ब्राह्मणान् सम्यग् ग्रा॑ह यि॒त्वा सूर्य वर्च॑स्वी भ॒वति ।\\
सूर्यग्रहे म॑हा न॒द्यां प्रतिमा सन्निधौ वा ज॒प्त्वा सिद्ध म॑न्त्रो भ॒वति ।\\
महा विघ्ना᳚त् प्रमु॒च्यते । महा दोषा᳚त् प्रमु॒च्यते ।\\
महा पापा᳚त् प्रमु॒च्यते । महा प्रत्य वाया᳚त् प्रमु॒च्यते ।\\
स सर्व॑ विद् भवति स सर्व॑ विद् भ॒वति ।\\
य ए॑वं वे॒द । इत्यु॑प॒निष॑त् ॥ 14 ॥\\
\subsection{\eng{Anuvaka 15}}
ॐ भ॒द्रङ् कर्णे॑भिश् शृणु॒ याम॑ देवाः ।\\
भ॒द्रं प॑श्ये मा॒क्ष भि॒र्य ज॑त्राः ।\\
स्थि॒रै रङ्गै᳚स् तुष्ठु॒वाग्ं स॑स्त॒ नूभिः॑ ।\\
व्यशे॑म दे॒वहि॑ तं॒यँ दायुः॑ ।\\
स्व॒स्ति न॒ इन्द्रो॑ वृ॒द्धश्र॑वाः ।\\
स्व॒स्ति नः॑ पू॒षा वि॒श्व वे॑दाः ।\\
स्व॒स्तिन॒स् तार्क्ष्यो॒ ,अरि॑ष्टनेमिः ।\\
स्व॒स्ति नो॒ बृह॒स्पति॑र्दधातु ॥\\
ॐ शान्ति॒-श्शान्ति॒-श्शान्तिः॑ ॥\\
