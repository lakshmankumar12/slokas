\section{\eng{Boo Sooktam}}
ओं भूमि॑र् भू॒म्ना द्यौर् व॑रि॒णाऽन् तरि॑क्षं महि॒त्वा ।\\
उ॒पस् थे॑ते देव्यदि ते॒ऽग् निम॑न् ना॒द-म॒न् नाद् या॒या द॑धे ॥   ( 1 )\\
आऽयङ्गौः पृश् नि॑रक् रमी॒ दस॑नन् मा॒तरं॒ पुन॑ ।\\
पि॒तरं॑ च प्र॒यन्त् सुवः॑ ॥ ( 2 )\\
त्रि॒ग्ं॒ शद् धाम॒ विरा॑जति॒ वाक् प॑त॒ङ्गाय॑ शिश् रिये ।\\
प्रत् य॑स्य वह॒द् युभिः॑ ॥ ( 3 )\\
अ॒स्यप् प्रा॒णाद॑ पान॒त्य॑न् तश् च॑रति रोच॒ना ।\\
व्य॑ख् यन् महि॒षः सुव॑ ॥ ( 4 )\\
यत् त्वा᳚ क्रु॒द्धः प॑रो॒ वप॑ म॒न्युना॒ यद व॑र्त्या ।\\
सु॒ कल्प॑ मग्ने॒ तत्तव॒ पुन॒स् त्वोद् दी॑पया मसि ॥ ( 5 )\\
यत्ते॑ म॒न्यु प॑रोप् तस्य पृथि॒वी मनु॑ दध्व॒से ।\\
आ॒दि॒त्या विश्वे॒ तद् दे॒वा वस॑ वश्च स॒मा भ॑रन् ॥ ( 6 )\\
मनो॒ ज्योति॑र् जुष॒तामाज्य॒म् विच्छि॑न्नं॒ य॑ज्ञ॒ग्ं स॑मि॒मं द॑धातु ।\\
ब्रुह॒स् पति॑स् तनु॒ता मि॒मंनो॒ विश्वे॑ दे॒वा, इ॒हमा॑ दयन्ताम् ।\\
स॒प्त ते॑, अग्ने स॒मिध॑स्॒ स॒प्त जि॒ह्वास्॒ स॒प्तर्ष॑यस्॒ स॒प्त धाम॑ प्रि॒याणि॑ ।\\
स॒प्त होत्रा᳚स् सप्त॒ धात्वा॑ यजन्ति स॒प्त योनी॒ रा पृ॑णस्वा घृ॒तेन॑ ।\\
पुन॑ रू॒र्जा निव॑र्त् तस्व॒ पुन॑ रग्र इ॒षा यु॑षा । पुन॑र्नः पाहि वि॒श्वतः॑ ।\\
"स॒ह र॒य्या निव॑र्त् त॒स्वाग्ने॒ पिन् व॑स्व॒ धार॑या । \\
{\small\eng{vish vaps niya}} वि॒श्व फ्स् नि॑या वि॒श्व त॒स्परि॑ ।"\\
लेक॒स् सले॑कस् सु॒लेक॒स् तेन॑ आदि॒त्या \\
आज्य॑ ञ्जुषा॒णा वि॑यन्तु॒ केत॒स् सकेत॒स् सु॒केत॒स् तेन॑ आदि॒त्या\\
आज्य॑ ञ्जुषा॒णा वि॑यन्तु॒ विव॑स् वा॒ꣳ॒ अदि॑ति॒र् देव॑ जूति॒स् तेन॑ आदि॒त्या\\
आज्य॑ ञ्जुषा॒णा वि॑यन्तु ।\\
मे॒दिनी॑ दे॒वी व॒सुन्ध॑रा स्या॒द् वसु॑धा दे॒वी वा॒सवी᳚ ।\\
ब्र॒ह्म॒व॒र्च॒ सः पि॑तृ॒णाम् श्रो᳚त्रं॒ चक्षु॒र्मन॑: ॥\\
दे॒वी हिर॑ण्य गर्भिणी दे॒वी प्र॒सूव॑री᳚ ।\\
दस॑ने स॒त्याय॑ने सीद ।\\
स॒मु॒द्रव॑ती सावि॒त्री ह॒नो दे॒वी म॒ह्यगी᳚ ।\\
म॒हा धर॑णी म॒होव् यथि॑ष् ठाश्, शृ॒ङ्गे शृ॑ङ्गे य॒ज्ञे य॑ज्ञे विभी॒ षणी᳚ ॥\\
इन्द्र॑ पत्॒नी व्या॒जनी॑ सु॒रसि॑त इ॒ह ।\\
वा॒यु॒परी॑ जल॒ शय॑नी {\small\eng{shvayan}} श्व॒यन् धा॒रा स॒त्यन् धो॒परि॑ मेदिनी ।\\
श्वो॒परि॑ धत्तं॒ गाय ।\\
वि॒ष्णु॒प॒त्॒नीं म॑हीं दे॒वीं॒ मा॒ध॒वीं मा॑धव॒ प्रियाम् ।\\
लक्ष्मीं प्रि॒यस॑खीं दे॒वीं॒ न॒मा॒म् यच्यु॑त व॒ल्लभाम् ॥\\
ध॒नु॒र् ध॒रायै॑ वि॒द्महे॑ सर्व सि॒द्ध्यै च॑ धीमहि ।\\
तन्नो॑ धरा प्रचो॒दया᳚त् ।\\
म॒हीं दे॒वीं विष्णु॑पत्॒नी-मजू॒र्याम् । प्र॒तीची॑ मेनाग्ं ह॒विषा॑ यजामः ।\\
त्रे॒धा विष्णु॑रुरु गा॒यो विच॑क्रमे । म॒हीं दिवं॑ पृथि॒वी म॒न्तरि॑क्षम् ।\\
तच् छ्रो॒णै ति॒श्रव॑-इ॒च्छ मा॑ना । पुण्य॒ग्ग्॒ श्लोकं॒ यज॑मानाय कृण्व॒ती ॥\\
ओं शान्ति॒: शान्ति॒: शान्ति॑: ॥\\
