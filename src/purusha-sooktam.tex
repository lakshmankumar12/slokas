\section{\eng{Purusha Sooktam}}
\subsection{\eng{Beginning}}
ओं तच्छं॒ योरावृ॑णीमहे । गा॒तुं य॒ज्ञाय॑ ।\\
गा॒तुं य॒ज्ञप॑तये । दैवी᳚: स्व॒स्तिर॑स्तु नः ।\\
स्व॒स्तिर्मानु॑षेभ्यः । ऊ॒र्ध्वं जि॑गातु भेष॒जम् ।\\
शन्नो॑ अस्तु द्वि॒पदे᳚ । शं चतु॑ष्पदे ॥\\
ओं शान्ति॒: शान्ति॒: शान्ति॑: ॥ \\
\subsection{\eng{Moolam}}
ओं स॒हस्र॑शीर्षा॒ पुरु॑षः । स॒ह॒स्रा॒क्षः स॒हस्र॑पात् ।\\
स भूमिं॑ वि॒श्वतो॑ वृ॒त्वा । अत्य॑तिष्ठद्दशाङ्गु॒लम् ।\\
पुरु॑ष ए॒वेदग्ं सर्वम्᳚ । यद्भू॒तं यच्च॒ भव्यम्᳚ ।\\
उ॒तामृ॑त॒त्वस्येशा॑नः । य॒दन्ने॑नाति॒रोह॑ति ।\\
ए॒तावा॑नस्य महि॒मा ।\\
अतो॒ ज्यायाग्॑श्च॒ पूरु॑षः ॥ 1 ॥\\
\\
पादो᳚ऽस्य॒ विश्वा॑ भू॒तानि॑ । त्रि॒पाद॑स्या॒मृतं॑ दि॒वि ।\\
त्रि॒पादू॒र्ध्व उदै॒त्पुरु॑षः ।\\
पादो᳚ऽस्ये॒हाऽऽभ॑वा॒त्पुन॑: ।\\
ततो॒ विष्व॒ङ्व्य॑क्रामत् ।\\
सा॒श॒ना॒न॒श॒ने अ॒भि । तस्मा᳚द्वि॒राड॑जायत ।\\
वि॒राजो॒ अधि॒ पूरु॑षः । स जा॒तो अत्य॑रिच्यत ।\\
प॒श्चाद्भूमि॒मथो॑ पु॒रः ॥ 2 ॥\\
\\
यत्पुरु॑षेण ह॒विषा᳚ । दे॒वा य॒ज्ञमत॑न्वत ।\\
व॒स॒न्तो अ॑स्यासी॒दाज्यम्᳚ । ग्री॒ष्म इ॒ध्मश्श॒रद्ध॒विः ।\\
स॒प्तास्या॑सन्परि॒धय॑: । त्रिः स॒प्त स॒मिध॑: कृ॒ताः ।\\
दे॒वा यद्य॒ज्ञं त॑न्वा॒नाः ।\\
अब॑ध्न॒न्पुरु॑षं प॒शुम् ।\\
तं य॒ज्ञं ब॒र्हिषि॒ प्रौक्षन्॑ ।\\
पुरु॑षं जा॒तम॑ग्र॒तः ॥ 3 ॥\\
\\
तेन॑ दे॒वा अय॑जन्त । सा॒ध्या ऋष॑यश्च॒ ये ।\\
तस्मा᳚द्य॒ज्ञात्स॑र्व॒हुत॑: । सम्भृ॑तं पृषदा॒ज्यम् ।\\
प॒शूग्‍स्ताग्‍श्च॑क्रे वाय॒व्यान्॑ । आ॒र॒ण्यान्ग्रा॒म्याश्च॒ ये ।\\
तस्मा᳚द्य॒ज्ञात्स॑र्व॒हुत॑: । ऋच॒: सामा॑नि जज्ञिरे ।\\
छन्दाग्ं॑सि जज्ञिरे॒ तस्मा᳚त् । यजु॒स्तस्मा॑दजायत ॥ 4 ॥\\
\\
तस्मा॒दश्वा॑ अजायन्त । ये के चो॑भ॒याद॑तः ।\\
गावो॑ ह जज्ञिरे॒ तस्मा᳚त् । तस्मा᳚ज्जा॒ता अ॑जा॒वय॑: ।\\
यत्पुरु॑षं॒ व्य॑दधुः । क॒ति॒धा व्य॑कल्पयन् ।\\
मुखं॒ किम॑स्य॒ कौ बा॒हू । कावू॒रू पादा॑वुच्येते ।\\
ब्रा॒ह्म॒णो᳚ऽस्य॒ मुख॑मासीत् । बा॒हू रा॑ज॒न्य॑: कृ॒तः ॥ 5 ॥\\
\\
ऊ॒रू तद॑स्य॒ यद्वैश्य॑: । प॒द्भ्याग्ं शू॒द्रो अ॑जायत ।\\
च॒न्द्रमा॒ मन॑सो जा॒तः । चक्षो॒: सूर्यो॑ अजायत ।\\
मुखा॒दिन्द्र॑श्चा॒ग्निश्च॑ । प्रा॒णाद्वा॒युर॑जायत ।\\
नाभ्या॑ आसीद॒न्तरि॑क्षम् । शी॒र्ष्णो द्यौः सम॑वर्तत ।\\
प॒द्भ्यां भूमि॒र्दिश॒: श्रोत्रा᳚त् ।\\
तथा॑ लो॒काग्ं अ॑कल्पयन् ॥ 6 ॥\\
\\
वेदा॒हमे॒तं पुरु॑षं म॒हान्तम्᳚ ।\\
आ॒दि॒त्यव॑र्णं॒ तम॑स॒स्तु पा॒रे ।\\
सर्वा॑णि रू॒पाणि॑ वि॒चित्य॒ धीर॑: ।\\
नामा॑नि कृ॒त्वाऽभि॒वद॒न्॒ यदास्ते᳚ ।\\
धा॒ता पु॒रस्ता॒द्यमु॑दाज॒हार॑ ।\\
श॒क्रः प्रवि॒द्वान्प्र॒दिश॒श्चत॑स्रः ।\\
तमे॒वं वि॒द्वान॒मृत॑ इ॒ह भ॑वति ।\\
नान्यः पन्था॒ अय॑नाय विद्यते ।\\
य॒ज्ञेन॑ य॒ज्ञम॑यजन्त दे॒वाः ।\\
तानि॒ धर्मा॑णि प्रथ॒मान्या॑सन् ।\\
ते ह॒ नाकं॑ महि॒मान॑: सचन्ते ।\\
यत्र॒ पूर्वे॑ सा॒ध्याः सन्ति॑ दे॒वाः ॥ 7 ॥\\
\subsection{\eng{Uttra Narayanam}}
अ॒द्भ्यः सम्भू॑तः पृथि॒व्यै रसा᳚च्च ।\\
वि॒श्वक॑र्मण॒: सम॑वर्त॒ताधि॑ ।\\
तस्य॒ त्वष्टा॑ वि॒दध॑द्रू॒पमे॑ति ।\\
तत्पुरु॑षस्य॒ विश्व॒माजा॑न॒मग्रे᳚ ।\\
वेदा॒हमे॒तं पुरु॑षं म॒हान्तम्᳚ ।\\
आ॒दि॒त्यव॑र्णं॒ तम॑स॒: पर॑स्तात् ।\\
तमे॒वं वि॒द्वान॒मृत॑ इ॒ह भ॑वति ।\\
नान्यः पन्था॑ विद्य॒तेय॑ऽनाय ।\\
प्र॒जाप॑तिश्चरति॒ गर्भे॑ अ॒न्तः ।\\
अ॒जाय॑मानो बहु॒धा विजा॑यते ॥ 8 ॥\\
\\
तस्य॒ धीरा॒: परि॑जानन्ति॒ योनिम्᳚ ।\\
मरी॑चीनां प॒दमि॑च्छन्ति वे॒धस॑: ।\\
यो दे॒वेभ्य॒ आत॑पति ।\\
यो दे॒वानां᳚ पु॒रोहि॑तः ।\\
पूर्वो॒ यो दे॒वेभ्यो॑ जा॒तः ।\\
नमो॑ रु॒चाय॒ ब्राह्म॑ये ।\\
रुचं॑ ब्रा॒ह्मं ज॒नय॑न्तः ।\\
दे॒वा अग्रे॒ तद॑ब्रुवन् ।\\
यस्त्वै॒वं ब्रा᳚ह्म॒णो वि॒द्यात् ।\\
तस्य॑ दे॒वा अस॒न् वशे᳚ ॥ 9 ॥\\
\\
ह्रीश्च॑ ते ल॒क्ष्मीश्च॒ पत् न्यौ᳚ ।\\
अ॒हो॒रा॒त्रे पा॒र्श्वे । नक्ष॑त्राणि रू॒पम् ।\\
अ॒श्विनौ॒ व्यात्तम्᳚ । इ॒ष्टं म॑निषाण ।\\
अ॒मुं म॑निषाण । सर्वं॑ मनिषाण ॥ 10 ॥\\
\\
ओं शान्ति॒: शान्ति॒: शान्ति॑: ॥\\
